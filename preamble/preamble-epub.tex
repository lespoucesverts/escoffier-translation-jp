\newcommand{\medlarge}{\fontsize{11}{13}\selectfont}
\newcommand{\medsmall}{\fontsize{9.23}{9.5}\selectfont}
\newcommand{\twelveq}{\jsc@setfontsize\twelveq{9.230769}{9.75}\selectfont}
\newcommand{\thirteenq}{\jsc@setfontsize\fourteenq{10}{11}\selectfont}
\newcommand{\fourteenq}{\jsc@setfontsize\fourteenq{10.7692}{13}\selectfont}
\newcommand{\fifteenq}{\jsc@setfontsize\fifteenq{11.53846}{14}\selectfont}

\newcommand{\frchap}[1]{#1}

\newcommand{\frsec}[1]{#1}
\newcommand{\frsecb}[1]{#1}
\newcommand{\frsub}[1]{#1}
%\
\makeatother
%%%%%%%%%レシピと本文%%%%%%%%%%%%



%%% 本文
\newenvironment{Main}{}{}
\newenvironment{recette}{}{}


%%% 脚注番号のページ毎のリセットと脚注位置の調整
%\renewcommand{\footnotesize}{\small}

\makeatletter

\usepackage[bottom,perpage,stable]{footmisc}%
%\setlength{\skip\footins}{4mm plus 4mm}
%\usepackage{footnpag}
\renewcommand\@makefntext[1]{%
  \advance\leftskip 0\zw
  \parindent 1\zw
  \noindent
  \llap{\@thefnmark\hskip0.5\zw}#1}


\let\footnotes@ve=\footnote
\def\footnote{\inhibitglue\footnotes@ve}
\let\footnotemarks@ve=\footnotemark
%\def\footnotemark{\inhibitglue\footnotemarks@ve}
\renewcommand{\footnotemark}{\footnotemarks@ve}%変更
% %\def\thefootnote{\ifnum\c@footnote>\z@\leavevmode\lower.5ex\hbox{(}\@arabic\c@footnote\hbox{)}\fi}
\renewcommand{\thefootnote}{\ifnum\c@footnote>\z@\leavevmode\hbox{}\@arabic\c@footnote\hbox{)}\fi}
%\makeatletter
% \@addtoreset{footnote}{page}
% \makeatother
%\usepackage{dblfnote}
%\usepackage[bottom,perpage]{footmisc}

\makeatother

%subsubsectionに連番をつける
%\usepackage{remreset}

\renewcommand{\thechapter}{}
\renewcommand{\thesection}{\hskip-1\zw}
\renewcommand{\thesubsection}{}
\renewcommand{\thesubsubsection}{}
\renewcommand{\theparagraph}{}


% \makeatletter
% \@removefromreset{subsubsection}{subsection}
% \def\thesubsubsection{\arabic{subsubsection}.}
% \newcounter{rnumber}
% \renewcommand{\thernumber}{\refstepcounter{rnumber} }
% \makeatother
\renewcommand{\prepartname}{\if@english Part~\else {}\fi}
\renewcommand{\postpartname}{\if@english\else {}\fi}
\renewcommand{\prechaptername}{\if@english Chapter~\else {}\fi}
\renewcommand{\postchaptername}{\if@english\else {}\fi}
\renewcommand{\presectionname}{}%  第
\renewcommand{\postsectionname}{}% 節

%リスト環境
\def\tightlist{\itemsep1pt\parskip0pt\parsep0pt}%pandoc対策


\let\cite=\autocite

% % % % 
\date{}



%%%%インデックス準備
%\usepackage{makeidx}
\usepackage{index}
%\usepackage[useindex]{splitidx}
\newindex{src}{idx1}{ind1}{ソース名から料理を探す}
\makeindex


%%%% pandoc が三点リーダーを勝手に変える対策
\renewcommand{\ldots}{\noindent…}
%%%%%下線
\usepackage{umoline}
\setlength{\UnderlineDepth}{2pt}
\let\ul\Underline

\newcommand{\maeaki}{}%使用しないので無効化
\newcommand{\atoaki}{\vspace{1.25mm}}
%%分数の表記Obsolete
\usepackage{xfrac}
\let\frac\sfrac
\newcommand{\undemi}{\hspace{.25\zw}$\sfrac{1}{2}$}
\newcommand{\untiers}{\hspace{.25\zw}$\sfrac{1}{3}$}
\newcommand{\deuxtiers}{\hspace{.25\zw}$\sfrac{2}{3}$}
\newcommand{\unquart}{\hspace{.25\zw}$\sfrac{1}{4}$}
\newcommand{\troisquarts}{\hspace{.25\zw}$\sfrac{3}{4}$}
\newcommand{\quatrequatrieme}{\hspace{.25\zw}$\sfrac{4}{4$}}
\newcommand{\uncinquieme}{\hspace{.25\zw}$\sfrac{1}{5}$}
\newcommand{\deuxcinquiemes}{\hspace{.25\zw}$\sfrac{2}{5}$}
\newcommand{\troiscinquiemes}{\hspace{.25\zw}$\sfrac{3}{5}$}
\newcommand{\quatrecinquiemes}{\hspace{.25\zw}$\sfrac{4}{5}$}
\newcommand{\unsixieme}{\hspace{.25\zw}$\sfrac{1}{6}$}
\newcommand{\cinqsixiemes}{\hspace{.25\zw}$\sfrac{5}{6}$}
\newcommand{\quatrequart}{\hspace{.25\zw}$\sfrac{4}{4}$}

\makeatletter
\def\ps@headings{%
  \let\@oddfoot\@empty
  \let\@evenfoot\@empty
  \def\@evenhead{%
    \if@mparswitch \hss \fi
    \underline{\hbox to \fullwidth{\ltjsetparameter{autoxspacing={true}}
%      \textbf{\thepage}\hfil\leftmark}}%
       \normalfont\thepage\hfill\scshape\small\leftmark\normalfont}}%
    \if@mparswitch\else \hss \fi}%
  \def\@oddhead{\underline{\hbox to \fullwidth{\ltjsetparameter{autoxspacing={true}}
        {\if@twoside\scshape\small\rightmark\else\scshape\small\leftmark\fi}\hfil\thepage\normalfont}}\hss}%
  \let\@mkboth\markboth
  \def\chaptermark##1{\markboth{%
    \ifnum \c@secnumdepth >\m@ne
      \if@mainmatter
        \if@omit@number\else
          \@chapapp\thechapter\@chappos\hskip1\zw
        \fi
      \fi
    \fi
    ##1}{}}%
  \def\sectionmark##1{\markright{%
%    \ifnum \c@secnumdepth >\z@ \thesection \hskip1\zw\fi
    \ifnum \c@secnumdepth >\z@ \thesection \hskip-1\zw\fi
    ##1}}}%
\makeatother

\makeatletter
%%%%%%%% Lua GC
\patchcmd\@outputpage{\stepcounter{page}}{%
  \directlua{%
	if jit then
      local k = collectgarbage("count")
      if k>900000 then 
        collectgarbage("collect")
        texio.write_nl("term and log", "GC: ", math.floor(k), math.floor(collectgarbage("count")))
      end
	end
  }%
  \stepcounter{page}%
}{}{}
\makeatother
%\usepackage{vgrid}% here only to help visualize the problem


