\documentclass[twoside,14Q,a4paper,openany]{ltjsbook}

\usepackage{amsmath}
  \let\equation\gather
  \let\endequation\endgather
\usepackage{amssymb}
\usepackage[no-math]{fontspec}
\usepackage{geometry}
\usepackage{luaotfload}
\usepackage{graphicx}

\usepackage{setspace}


%%%%%%%%% hyperref %%%%%%%%%%%%%
\usepackage{refcount}
\usepackage[unicode=true,hyperfootnotes=false,pageanchor]{hyperref}
\hypersetup{hyperindex=false,%
             breaklinks=true,%
             bookmarks=true,%
             pdfauthor={五島 学},%
             pdftitle={エスコフィエ『料理の手引き』全注解},%
             %colorlinks=false%true,%
             colorlinks=true,%
             citecolor=blue,%
             urlcolor=cyan,%
             linkcolor=magenta,%
             bookmarksdepth=subsubsection,%
             pdfborder={0 0 0},%
             hyperfootnotes=false,%
             plainpages=false,
             }
\urlstyle{same}




%% 欧文フォント設定
% Libertine/Biolinum
\setmainfont[Ligatures=Historic,Scale=1.0]{Linux Libertine O}
\setsansfont[Ligatures=TeX, Scale=MatchLowercase]{Linux Biolinum O} 
%\usepackage{libertine}
\usepackage{unicode-math}
\setmathfont[Scale=1.2]{libertinusmath-regular.otf}
%\unimathsetup{math-style=ISO,bold-style=ISO}
%\setmathfont{xits-math.otf}
%\setmathfont{xits-math.otf}[range={cal,bfcal},StylisticSet=1]




%% Garamond
%\usepackage{ebgaramond-maths}
%\setmainfont[Ligatures=Historic,Scale=1.1]{EB Garamond}%fontspecによるフォント設定

%\usepackage{qpalatin}%palatino

%\setmainfont[Ligatures=Historic,Scale=MatchLowercase]{Tex Gyre Schola}
%\setmainfont[Ligatures=Historic,Scale=MatchLowercase]{Tex Gyre Pagella}
%\setsansfont[Scale=MatchLowercase]{TeX Gyre Heros}  % \sffamily のフォント
%\setsansfont[Scale=MatchLowercase]{TeX Gyre Adventor}  % \sffamily のフォント

%\setmonofont[Scale=MatchLowercase]{Inconsolata}       % \ttfamily のフォント

%\usepackage[cmintegrals,cmbraces]{newtxmath}%数式フォント

\usepackage{luatexja}
\usepackage{luatexja-fontspec}
%\ltjdefcharrange{8}{"2000-"2013, "2015-"2025, "2027-"203A, "203C-"206F}
%\ltjsetparameter{jacharrange={-2, +8}}
\usepackage{luatexja-ruby}

%%%%和文フォント設定
%\usepackage[sourcehan,bold,jis2004,expert,deluxe]{luatexja-preset}%Adobe源ノ明朝、ゴチ
%\usepackage[hiragino-pron,bold,jis2004,expert,deluxe]{luatexja-preset}
%\usepackage[yu-osx,expert,jis2004,bold]{luatexja-preset}
%\usepackage[moga-mogo-ex,bold]{luatexja-preset}

\newopentypefeature{PKana}{On}{pkna} % "PKana" and "On" can be arbitrary string
%%%%明朝にIPAexMincho、ゴチ(太字)にMoboGoBを使う設定。和文カナプロプーショナル使用可能だが読みづらくなる。
\setmainjfont[%
%   YokoFeatures={JFM=prop,PKana=On},%
%    CharacterWidth=AlternateProportional,%
%    CharacterWidth=Proportional,%Mogo, IPAExMinchoには不可
%       Kerning=On,%
         BoldFont={ MoboGoB },%
         ItalicFont={ MoboGoB },%
         BoldItalicFont={ MoboGoExB }%
%         ]{ MogaHMin }
          ]{ IPAExMincho }
%         ]{ IPAmjMincho }
    \setsansjfont[%
%        YokoFeatures={JFM=prop,PKana=On},%
%         CharacterWidth=AlternateProportional,%
%         CharacterWidth=Proportional,%Mobo, IPAExGOthicには不可
%         Kerning=On,
         BoldFont={ MoboGoB },%
         ItalicFont={ MoboGoB },%
         BoldItalicFont={ MoboGoExB }%
         % ]{ MoboGo}
          ]{ IPAExGothic }
% %  %%%% 和文仮名プロプーショナルここまで
% %\ltjsetparameter{jacharrange={-2}}%キリル文字%引数に-3を付けるとギリシア文字も可能になるが、%三点リーダーも欧文化されてしまうので注意%


\renewcommand{\bfdefault}{bx}%和文ボールドを有効にする
\renewcommand{\headfont}{\gtfamily\sffamily\bfseries}%和文ボールドを有効にする
%\addfontfeature{Fractions=On}


\defaultfontfeatures[\rmfamily]{Scale=1.2}%効いていない様子
\defaultjfontfeatures{Scale=0.92487}%和文フォントのサイズ調整。デフォルトは 0.962212 倍%ltjsclassesでは不要?
%\defaultjfontfeatures{Scale=0.962212}
%\usepackage{libertineotf}%linux libertine font %ギリシア語含む
%\usepackage[T1]{fontenc}
%\usepackage[full]{textcomp}
%\usepackage[osfI,scaled=1.0]{garamondx}
%\usepackage{tgheros,tgcursor}
%\usepackage[garamondx]{newtxmath}

\usepackage{layout}

%% レイアウト調整(A4Paper,14Q,twoside,ltjsbook.cls) 
%%
\setlength{\hoffset}{0\zw}
\setlength{\oddsidemargin}{0\zw}%タブレット前提の中央配置
\setlength{\evensidemargin}{\oddsidemargin}
% \setlength{\oddsidemargin}{1\zw}%製本時に右ページのみをオフセット
%\setlength{\evensidemargin}{0pt}%
\setlength{\fullwidth}{45\zw}
\setlength{\textwidth}{45\zw}%%ltjsclassesのみ有効
%\setlength{\fullwidth}{159mm}
%\setlength{\textwidth}{159mm}
\setlength{\marginparsep}{0pt}
\setlength{\marginparwidth}{0pt}
\setlength{\footskip}{0pt}
\setlength{\voffset}{-17mm}
\setlength{\textheight}{265mm}
\setlength{\parskip}{0pt}
%\setlength{\parindent}{0pt}
%%%ベースライン調整
%\ltjsetparameter{yjabaselineshift=0pt,yalbaselineshift=-.75pt}


%\usepackage{fancyhdr}






%文字サイズ、見出しなどの再定義
\makeatletter
%\renewcommand{\large}{\jsc@setfontsize\large\@xipt{14}}
%\renewcommand{\Large}{\jsc@setfontsize\Large{13}{15}}

\newcommand{\medlarge}{\fontsize{11}{13}\selectfont}
\newcommand{\medsmall}{\fontsize{9.23}{9.5}\selectfont}
\newcommand{\twelveq}{\jsc@setfontsize\twelveq{9.230769}{9.75}\selectfont}
\newcommand{\fourteenq}{\jsc@setfontsize\fourteenq{10.7692}{13}\selectfont}
\newcommand{\fifteenq}{\jsc@setfontsize\fifteenq{11.53846}{14}\selectfont}


\renewcommand{\chapter}{%
  \if@openleft\cleardoublepage\else
  \if@openright\cleardoublepage\else\clearpage\fi\fi
  \plainifnotempty % 元: \thispagestyle{plain}
  \global\@topnum\z@
  \if@english \@afterindentfalse \else \@afterindenttrue \fi
  \secdef
    {\@omit@numberfalse\@chapter}%
    {\@omit@numbertrue\@schapter}}
\def\@chapter[#1]#2{%
  \ifnum \c@secnumdepth >\m@ne
    \if@mainmatter
      \refstepcounter{chapter}%
      \typeout{\@chapapp\thechapter\@chappos}%
      \addcontentsline{toc}{chapter}%
        {\protect\numberline
        % {\if@english\thechapter\else\@chapapp\thechapter\@chappos\fi}%
        {\@chapapp\thechapter\@chappos}%
        #1}%
    \else\addcontentsline{toc}{chapter}{#1}\fi
  \else
    \addcontentsline{toc}{chapter}{#1}%
  \fi
  \chaptermark{#1}%
  \addtocontents{lof}{\protect\addvspace{10\jsc@mpt}}%
  \addtocontents{lot}{\protect\addvspace{10\jsc@mpt}}%
  \if@twocolumn
    \@topnewpage[\@makechapterhead{#2}]%
  \else
    \@makechapterhead{#2}%
    \@afterheading
  \fi}
\def\@makechapterhead#1{%
  \vspace*{0\Cvs}% 欧文は50pt
  {\parindent \z@ \centering \normalfont
    \ifnum \c@secnumdepth >\m@ne
      \if@mainmatter
        \huge\headfont \@chapapp\thechapter\@chappos%変更
        \par\nobreak
        \vskip \Cvs % 欧文は20pt
      \fi
    \fi
    \interlinepenalty\@M
    \huge \headfont #1\par\nobreak
    \vskip 1\Cvs}} % 欧文は40pt%変更

\renewcommand{\section}{%
    \if@slide\clearpage\fi
    \@startsection{section}{1}{\z@}%
    {\Cvs \@plus.5\Cdp \@minus.2\Cdp}% 前アキ
    % {.5\Cvs \@plus.3\Cdp}% 後アキ
    {.5\Cvs}
    {\normalfont\Large\headfont\bfseries\centering}}%変更

\renewcommand{\subsection}{\@startsection{subsection}{2}{\z@}%
    {\Cvs \@plus.5\Cdp \@minus.2\Cdp}% 前アキ
    % {.5\Cvs \@plus.3\Cdp}% 後アキ
    {.5\Cvs}
    {\normalfont\large\headfont\bfseries\centering}} %変更


\renewcommand{\subsubsection}{\@startsection{subsubsection}{3}{\z@}%
    {.25\Cvs \@plus.5\Cdp \@minus.5\Cdp}%変更
    {\if@slide .5\Cvs \@plus.3\Cdp \else \z@ \fi}%
    {\normalfont\medlarge\headfont\leftskip -1\zw}}

\renewcommand{\paragraph}{\@startsection{paragraph}{4}{\z@}%
    {0.5\Cvs \@plus.5\Cdp \@minus.2\Cdp}%
    % {\if@slide .5\Cvs \@plus.3\Cdp \else -1\zw\fi}% 改行せず 1\zw のアキ
    {1sp}%後アキ
    {\normalfont\normalsize\headfont}}
\renewcommand{\subparagraph}{\@startsection{subparagraph}{5}{\z@}%
    {\z@}{\if@slide .5\Cvs \@plus.3\Cdp \else -.5\zw\fi}%
    {\normalfont\normalsize\headfont\hskip-.5\zw\noindent}}  



\newcommand{\frsec}[1]{\vspace*{-1\zw}\begin{center}\normalfont\headfont\Large\setstretch{0.8}\scshape#1\normalfont\normalsize\end{center}\vspace{0.5\zw}\setstretch{1.0}}

\newcommand{\frsecb}[1]{\vspace*{-1\zw}\begin{center}\normalfont\headfont\large\setstretch{0.8}\scshape#1\normalfont\normalsize\end{center}\vspace{0.5\zw}\setstretch{1.0}}

%\newcounter{frsub}[subsubsection]
%\newcommand{\frsub}{\@startsection{frsub}{6}{\z@}%
%  {1sp}{1sp}%
%  {\normalfont\normalsize\bfseries\baselineskip-.8ex\leftskip-1\zw}}
%\let\frsubmark\@gobble
%\newcommand*{\l@frsub}{%
%          \@tempdima\jsc@tocl@width \advance\@tempdima 16.183\zw
%          \@dottedtocline{7}{\@tempdima}{6.5\zw}}
%\renewcommand{\thefrsub}{6}
%\let\frsub\paragraph

\newenvironment{frsubenv}{\begin{spacing}{0.2}\setlength{\leftskip}{-1\zw}\bfseries}{\end{spacing}\normalfont\normalsize\setlength{\leftskip}{0pt}}
\newcommand{\frsub}[1]{\begin{frsubenv}#1\end{frsubenv}\par\vspace{1.1\zw}}

%\newcommand{\frsub}[1]{\vskip -.8ex\hskip -1\zw\textbf{#1}\leftskip0pt}
%\newcommand{\frsub}{\@startsection{frsub}{6}{\z@}%
%   {-1\zw}% 改行せず 1\zw のアキ
%   {-1\zw}%後アキ     
%   {\normalfont\normalsize\bfseries\leftskip -1\zw\baselineskip -.5ex}}%normalsizeから変更
%\newcommand*{\l@frsub}{%
%          \@tempdima\jsc@tocl@width \advance\@tempdima 16.183\zw
%          \@dottedtocline{5}{\@tempdima}{6.5\zw}}


%%%%%%%%%レシピと本文%%%%%%%%%%%%
\usepackage{multicol}
\setlength{\columnsep}{3\zw}

%%% 本文
\newenvironment{Main}{}{}
%%% レシピ
% \setlength{\columnwidth}{24\zw}
%本文ヨリ小%\small
%\newenvironment{recette}{\setlength{\parindent}{0pt}\begin{small}\begin{spaceing}{0.8}\begin{multicols}{2}}{\end{multicols}\end{spacing}\end{small}}
%本文やや小%\medsmall
%\newenvironment{recette}{\setlength{\parindent}{0pt}\begin{medsmall}\begin{spacing}{0.75}\begin{multicols}{2}}{\end{multicols}\end{spacing}\end{medsmall}}
%本文ナミ(無指定)
\newenvironment{recette}{\setlength{\parindent}{0pt}\begin{spacing}{0.8}\begin{multicols}{2}}{\end{multicols}\end{spacing}}


\makeatother


%%% 脚注番号のページ毎のリセットと脚注位置の調整
\makeatletter

\usepackage[bottom,perpage,stable]{footmisc}%
%\setlength{\skip\footins}{4mm plus 2mm}
%\usepackage{footnpag}
\renewcommand\@makefntext[1]{%
  \advance\leftskip 0\zw
  \parindent 1\zw
  \noindent
  \llap{\@thefnmark\hskip0.5\zw}#1}


\let\footnotes@ve=\footnote
\def\footnote{\inhibitglue\footnotes@ve}
\let\footnotemarks@ve=\footnotemark
%\def\footnotemark{\inhibitglue\footnotemarks@ve}
\renewcommand{\footnotemark}{\footnotemarks@ve}%変更
% %\def\thefootnote{\ifnum\c@footnote>\z@\leavevmode\lower.5ex\hbox{(}\@arabic\c@footnote\hbox{)}\fi}
\renewcommand{\thefootnote}{\ifnum\c@footnote>\z@\leavevmode\hbox{}\@arabic\c@footnote\hbox{)}\fi}
%\makeatletter
% \@addtoreset{footnote}{page}
% \makeatother
%\usepackage{dblfnote}
%\usepackage[bottom,perpage]{footmisc}

\makeatother

%subsubsectionに連番をつける
%\usepackage{remreset}

\renewcommand{\thechapter}{}
\renewcommand{\thesection}{\hskip-1\zw}
\renewcommand{\thesubsection}{}
\renewcommand{\thesubsubsection}{}
\renewcommand{\theparagraph}{}


%\makeatletter
%\@removefromreset{subsubsection}{subsection}
%\def\thesubsubsection{\arabic{subsubsection}.}
%\newcounter{rnumber}
%\renewcommand{\thernumber}{\refstepcounter{rnumber} }

\renewcommand{\prepartname}{\if@english Part~\else {}\fi}
\renewcommand{\postpartname}{\if@english\else {}\fi}
\renewcommand{\prechaptername}{\if@english Chapter~\else {}\fi}
\renewcommand{\postchaptername}{\if@english\else {}\fi}
\renewcommand{\presectionname}{}%  第
\renewcommand{\postsectionname}{}% 節

%リスト環境
\def\tightlist{\itemsep1pt\parskip0pt\parsep0pt}%pandoc対策

\makeatletter
  \parsep   = 0pt
  \labelsep = .5\zw
  \def\@listi{%
     \leftmargin = 0pt \rightmargin = 0pt
     \labelwidth\leftmargin \advance\labelwidth-\labelsep
     \topsep     = 0pt%\baselineskip
     %\topsep -0.1\baselineskip \@plus 0\baselineskip \@minus 0.1 \baselineskip
     \partopsep  = 0pt \itemsep       = 0pt
     \itemindent = -.5\zw \listparindent = 0\zw}
  \let\@listI\@listi
  \@listi
  \def\@listii{%
     \leftmargin = 1.8\zw \rightmargin = 0pt
     \labelwidth\leftmargin \advance\labelwidth-\labelsep
     \topsep     = 0pt \partopsep     = 0pt \itemsep   = 0pt
     \itemindent = 0pt \listparindent = 1\zw}
  \let\@listiii\@listii
  \let\@listiv\@listii
  \let\@listv\@listii
  \let\@listvi\@listii
\makeatother








%% %%%%%%行取りマクロ
% \makeatletter
% \ifx\Cht\undefined
%  \newdimen\Cht\newdimen\Cdp
%  \setbox0\hbox{\char\jis"2121}\Cht=\ht0\Cdp=\dp0\fi
% \catcode`@=11
% \long\def\linespace#1#2{\par\noindent
%   \dimen@=\baselineskip
%   \multiply\dimen@ #1\advance\dimen@-\baselineskip
%   \advance\dimen@-\Cht\advance\dimen@\Cdp
%   \setbox0\vbox{\noindent #2}%
%   \advance\dimen@\ht0\advance\dimen@-\dp0%
%   \vtop to\z@{\hbox{\vrule width\z@ height\Cht depth\z@
%    \raise-.5\dimen@\hbox{\box0}}\vss}%
%   \dimen@=\baselineskip
%   \multiply\dimen@ #1\advance\dimen@-2\baselineskip
%   \par\nobreak\vskip\dimen@
%   \hbox{\vrule width\z@ height\Cht depth\z@}\vskip\z@}
% \catcode`@=12
% \setlength{\parskip}{0pt}
% \setlength{\topskip}{\Cht}
% \setlength{\textheight}{43\baselineskip}
% \addtolength{\textheight}{1\zh}
% \makeatother
 
%%%%%%%%%%%%失敗%%%%%%%%%%%%
%\let\formule\subsubsection
%\renewcommand{\subsubsection}[1]{\linespace{1}{\formule#1}}
%%%%%%%%%%%%失敗%%%%%%%%%%%%






% PDF/X-1a
% \usepackage[x-1a]{pdfx}
% \Keywords{pdfTeX\sep PDF/X-1a\sep PDF/A-b}
% \Title{Sample LaTeX input file}
% \Author{LaTeX project team}
% \Org{TeX Users Group}
% \pdfcompresslevel=0
%\usepackage[cmyk]{xcolor}

%biblatex
%\usepackage[notes,strict,backend=biber,autolang=other,%
%                   bibencoding=inputenc,autocite=footnote]{biblatex-chicago}
%\addbibresource{hist-agri.bib}
\let\cite=\autocite

% % % % 
\date{}



\makeatletter
\renewenvironment{theindex}{% 索引を3段組で出力する環境
  \if@twocolumn
      \onecolumn\@restonecolfalse
    \else
      \clearpage\@restonecoltrue
    \fi
    \columnseprule.4pt \columnsep 2\zw
    \ifx\multicols\@undefined
      \twocolumn[\@makeschapterhead{\indexname}%
      \addcontentsline{toc}{chapter}{\indexname}]%変更点
    \else
      \ifdim\textwidth<\fullwidth
        \setlength{\evensidemargin}{\oddsidemargin}
        \setlength{\textwidth}{\fullwidth}
        \setlength{\linewidth}{\fullwidth}
        \begin{multicols}{3}[\chapter*{\indexname}
	\addcontentsline{toc}{chapter}{\indexname}]%変更点%
      \else
        \begin{multicols}{3}[\chapter*{\indexname}
	\addcontentsline{toc}{chapter}{\indexname}]%変更点%
      \fi
    \fi
    \@mkboth{\indexname}{\indexname}%
    \plainifnotempty % \thispagestyle{plain}
    \parindent\z@
    \parskip\z@ \@plus .3\p@\relax
    \let\item\@idxitem
    \raggedright
    \footnotesize\narrowbaselines
  }{
    \ifx\multicols\@undefined
      \if@restonecol\onecolumn\fi
    \else
      \end{multicols}
      \fi
      \clearpage
    }
\makeatother


%%%% 本文中の参照ページ番号表示 %%%%%%%

\makeatletter

%\AtBeginDocument{%
%  \DeclareRobustCommand\ref{\@ifstar\@refstar\@refstar}%
%  \DeclareRobustCommand\pageref{\@ifstar\@pagerefstar\@pagerefstar}}
\let\orig@Hy@EveryPageAnchor\Hy@EveryPageAnchor
\def\Hy@EveryPageAnchor{%
    \begingroup
    \hypersetup{pdfview=Fit}%
    \orig@Hy@EveryPageAnchor
    \endgroup
  }
  \usepackage{etoolbox}
\if@mainmatter{\let\myhyperlink\hyperlink%
\renewcommand{\hyperlink}[2]{\myhyperlink{#1}{#2} [p.\getpagerefnumber{#1}{}] }}
  \AtBeginEnvironment{recette}{%
\let\myhyperlink\hyperlink%
\renewcommand{\hyperlink}[2]{\myhyperlink{#1}{#2} [p.\getpagerefnumber{#1}{}] }}
  \AtBeginEnvironment{Main}{%
\let\myhyperlink\hyperlink%
\renewcommand{\hyperlink}[2]{\myhyperlink{#1}{#2} [p.\getpagerefnumber{#1}{}] }}


% \def\@@wrindex#1|#2|#3\\{%
%  \if@filesw
%  \ifx\\#2\\%
%   \protected@write\@indexfile{}{% 
%     %\string\indexentry{#1}{\thepage}%%%改変部分。もとは{#1|hyperpage}{\thepage}%
%     \string\indexentry{#1|hyperpage}{\thepage}%%オリジナル%
%         }%
%         \else
%           \HyInd@@@wrindex{#1}#2\\%
%         \fi
%       \fi
%       \endgroup
%       \@esphack
%     }

\makeatother



%%%%% Obsolete Reference Page Numbers 

%\newcommand{\pref}[1]{[p.\pageref{#1}]}
%\newcommand{\pref}[1]{}




%%%%インデックス準備
\usepackage{makeidx}

\makeindex



%%%% pandoc が三点リーダーを勝手に変える対策
\renewcommand{\ldots}{\noindent…}
%%%%%下線
\usepackage{umoline}
\setlength{\UnderlineDepth}{2pt}
\let\ul\Underline

\newcommand{\maeaki}{}%使用しないので無効化

%%分数の表記Obsolete
\usepackage{xfrac}
\let\frac\sfrac
\newcommand{\undemi}{\hspace{.25\zw}$\sfrac{1}{2}$}
\newcommand{\untiers}{\hspace{.25\zw}$\sfrac{1}{3}$}
\newcommand{\deuxtiers}{\hspace{.25\zw}$\sfrac{2}{3}$}
\newcommand{\unquart}{\hspace{.25\zw}$\sfrac{1}{4}$}
\newcommand{\troisquarts}{\hspace{.25\zw}$\sfrac{3}{4}$}
\newcommand{\quatrequatrieme}{\hspace{.25\zw}$\sfrac{4}{4$}}
\newcommand{\uncinquieme}{\hspace{.25\zw}$\sfrac{1}{5}$}
\newcommand{\deuxcinquiemes}{\hspace{.25\zw}$\sfrac{2}{5}$}
\newcommand{\troiscinquiemes}{\hspace{.25\zw}$\sfrac{3}{5}$}
\newcommand{\quatrecinquiemes}{\hspace{.25\zw}$\sfrac{4}{5}$}
\newcommand{\unsixieme}{\hspace{.25\zw}$\sfrac{1}{6}$}
\newcommand{\cinqsixiemes}{\hspace{.25\zw}$\sfrac{5}{6}$}
\newcommand{\quatrequart}{\hspace{.25\zw}$\sfrac{4}{4}$}

\makeatletter
\def\ps@headings{%
  \let\@oddfoot\@empty
  \let\@evenfoot\@empty
  \def\@evenhead{%
    \if@mparswitch \hss \fi
    \underline{\hbox to \fullwidth{\ltjsetparameter{autoxspacing={true}}
%      \textbf{\thepage}\hfil\leftmark}}%
       \normalfont\thepage\hfill\scshape\small\leftmark\normalfont}}%
    \if@mparswitch\else \hss \fi}%
  \def\@oddhead{\underline{\hbox to \fullwidth{\ltjsetparameter{autoxspacing={true}}
        {\if@twoside\scshape\small\rightmark\else\scshape\small\leftmark\fi}\hfil\thepage\normalfont}}\hss}%
  \let\@mkboth\markboth
  \def\chaptermark##1{\markboth{%
    \ifnum \c@secnumdepth >\m@ne
      \if@mainmatter
        \if@omit@number\else
          \@chapapp\thechapter\@chappos\hskip1\zw
        \fi
      \fi
    \fi
    ##1}{}}%
  \def\sectionmark##1{\markright{%
%    \ifnum \c@secnumdepth >\z@ \thesection \hskip1\zw\fi
    \ifnum \c@secnumdepth >\z@ \thesection \hskip-1\zw\fi
    ##1}}}%
\makeatother

\makeatletter
%%%%%%%% Lua GC
\patchcmd\@outputpage{\stepcounter{page}}{%
  \directlua{%
	if jit then
      local k = collectgarbage("count")
      if k>900000 then 
        collectgarbage("collect")
        texio.write_nl("term and log", "GC: ", math.floor(k), math.floor(collectgarbage("count")))
      end
	end
  }%
  \stepcounter{page}%
}{}{}
\makeatother 