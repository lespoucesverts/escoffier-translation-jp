\hypertarget{l-asperge-verte-ux30b0ux30eaux30fcux30f3ux30a2ux30b9ux30d1ux30e9ux306eux6599ux7406}{%
\subsection{L' ASPERGE VERTE
 グリーンアスパラの料理}\label{l-asperge-verte-ux30b0ux30eaux30fcux30f3ux30a2ux30b9ux30d1ux30e9ux306eux6599ux7406}}

La cuisson inspirée des légumes « à la grecque
» 調理法は「ア・ラ・グレック」(ギリシャ風の野菜)から着想

\hypertarget{section}{%
\subsubsection{}\label{section}}

La sauce dérivée de la « sauce Tartare
» ソースは「タルタルソース」から応用

\hypertarget{sauce-tartare}{%
\subsubsection[タルタルソース]{\texorpdfstring{タルタルソース\footnote{タルタル(タタール)=フランス人から見て東方の蛮族、というイメージで語られがちだが、カレーム『19世紀フランス料理』にあるSauce
  Rémoulade à la Mogol
  (Mongoleの誤植と思われる)「モンゴル風ソース・レムラード」およびSauce
  à la
  Tartare「タルタル風ソース」のレシピを見るかぎり、誤解という可能性も感じられる。前者は固茹で卵の卵黄に塩、こしょう、ナツメグ、カイエンヌ、砂糖、油、エストラゴンヴィネガーを合わせてピュレ状にして布で漉し、サフランを煎じた汁で美しい黄色に染め、刻んだシブレットを加えて仕上げるというもの。後者はソース・アルマンドとマスタード同量に生の卵黄2個を加え、塩、こしょう、ナツメグで調味してエクス産の油レードル2杯分とレードル
  \(\frac{1}{2}\)
  杯のエストラゴンヴィネガーを少しずつ加えながら混ぜていく。みじん切りにして下茹でしたエシャロット少々とにんにく少々、エストラゴンとセルフイユのみじん切りを大さじ1杯加える、というもの(pp.137-138)。少なくともこれらのレシピにおいて、タルタルすなわち野蛮、というニュアンスを見出すことは出来ないだろう。なお、Steak
  tartareタルタルステーキのレシピは本書には掲載されておらず、1938年の『ラルース・ガストロノミック』初版が初出と思われる(p.1019)。}}{タルタルソース}}\label{sauce-tartare}}

\frsub{Sauce Tartare}

\index{たるたる@タルタル!そーすれいせい@---ソース(冷製)}
\index{そーす@ソース!たるたるれいせい@タルタル---(冷製)}
\index{sauce@sauce!tartare@--- Tartare (froide)}
\index{tartare@tartare!sauce froide@Sauce --- (froide)}
\index[src]{tartare@Tartare} \index[src]{たるたる@▽たるたる}

固茹で卵の黄身8個をすり潰して滑らかになるまでよく練る。塩、挽きたてのこしょう各1つまみ強で味付けする。油1
Lとヴィネガー大さじ2杯を加えながらソースを立てていく\footnote{明記されていないが、\protect\hyperlink{mayonnaise}{マヨネーズ}や\protect\hyperlink{sauce-gribiche}{ソース・グリビッシュ}と同様に作業すること。}。若どりの玉ねぎ\footnote{いわゆる「オニオンヌーヴォー」だが、日本でこの名称で流通しているものは黄色系の品種が多いのに対し、フランスでは白系品種(oignon
  blanc オニョンブロン)が多く、風味が異なることに注意。}の葉またはシブレット20
gをすり潰してマヨネーズ大さじ2杯でのばし、目の細かい網で裏漉ししたものを加えて仕上げる。

\ldots{}\ldots{}このソースは、冷製の家禽や肉料理、魚料理、甲殻類いずれにも合う。また、「ディアーブル(悪魔風)」仕立ての肉料理、鶏料理にも用いられる。

\hypertarget{le-poisson-kinme-ou-sawara-je-sais-pas-encore-ux9b5aux6599ux7406ux91d1ux76eeux9bdbux307eux305fux306fux9c06}{%
\subsection{◆LE POISSON ( kinme ou Sawara je sais pas encore )
魚料理(金目鯛または鰆)}\label{le-poisson-kinme-ou-sawara-je-sais-pas-encore-ux9b5aux6599ux7406ux91d1ux76eeux9bdbux307eux305fux306fux9c06}}

・En mise en valeur de « la sauce tomate » 「トマトソース」に光を当てて

\hypertarget{sauce-tomate}{%
\subsubsection{トマトソース}\label{sauce-tomate}}

\frsub{Sauce tomate}

\index{そーす@ソース!とまとそーす@トマト---}
\index{とまと@トマト!ソース@---ソース}
\index{sauce@sauce!tomate@--- tomate}
\index{tomate@tomate!sauce@Sauce ---} \index[src]{tomate@Tomate}
\index[src]{とまと@トマト}

(仕上がり5 L分)

\begin{itemize}
\item
  主素材\ldots{}\ldots{}トマトピュレ4 L、または生のトマト6 kg。
\item
  \protect\hyperlink{mirepoix}{ミルポワ}\ldots{}\ldots{}さいの目に切って下茹でしておいた塩漬け豚ばら肉140
  g 、1〜2 mm 角のさいの目に刻んだにんじん200 gと玉ねぎ150
  g、ローリエの葉 1枚、タイム1枝、バター100 g。
\item
  追加素材\ldots{}\ldots{}小麦粉150 g、白いフォン2 L、にんにく2片。
\item
  調味料\ldots{}\ldots{}塩20 g、砂糖30 g、こしょう1つまみ。
\item
  作業手順\ldots{}\ldots{}厚手の片手鍋で、塩漬け豚ばら肉を軽く色付くまで炒める。ミルポワの野菜を加え、野菜も色よく炒める。小麦粉を振りかける。ブロンド色になるまで炒めてから、トマトピュレまたは潰した生トマトと白いフォン、砕いたにんにく、塩、砂糖、こしょうを加える。
\end{itemize}

火にかけて混ぜながら沸騰させる。鍋に蓋をして弱火のオーブンに入れ1時間半〜2時間加熱する。

目の細かい漉し器または布で漉す。再度、火にかけて数分間沸騰させる。保存用の器に移し、ソースが空気に触れて表面に膜が張らないよう、バターのかけらを載せてやる。

\hypertarget{nota-sauce-tomate}{%
\subparagraph{【原注】}\label{nota-sauce-tomate}}

トマトピュレを使い、小麦粉は使わず、その他は上記のとおりに作ってもいい。漉し器か布で漉してから、充分な濃度になるまでしっかり煮詰めてやること。

\index{そーす@ソース!きほん@\textbf{基本---}|)}
\index{sauce@sauce!grandes@\textbf{Grandes ---s de Base}|)}

\hypertarget{le-homardux30aaux30deux30fcux30ebux6d77ux8001ux3092ux4f7fux3063ux305fux6599ux7406}{%
\subsection{◆LE
HOMARD オマール海老を使った料理}\label{le-homardux30aaux30deux30fcux30ebux6d77ux8001ux3092ux4f7fux3063ux305fux6599ux7406}}

・Inspiré de la préparation « à la Française
» 「ア・ラ・フランセーズ」の調理法から着想

\hypertarget{homard-a-la-francaise}{%
\subsubsection{オマール・アラフランセーズ}\label{homard-a-la-francaise}}

\frsub{Homard à la Française}

オマール海老は筒切り\footnote{tronçon (トロンソン)}にする。バター50
gを熱したソテー鍋にこれを入れ、塩、こしょう、カイエンヌ\footnote{赤唐辛子の一般名としても使われるが、本来、Piment
  de Cayenne
  という品種であり、辛さは日本で一般的なタカノツメと比べると辛さと示す度数であるスコビル値もタカノツメが4万〜5万であるのに対して、3万から5万とやや低め。なにより風味が決定的に異なるが、辛いためにごく少量しか用いることが出来ず、品種としての風味のよさを評価される機会はとても少ない。近年では辛みのあまり感じられず風味のよいPiment
  d'Espelette(ピモンデスプレットまたはピモンデスプレット)や、多少辛みはあるがおなじく風味のよいPiment
  de
  Piquillo(ピモンドピキヨまたはピモンドピキリョ)などが好んで使われている。前者は乾燥粉末として、後者は生あるいはピクルス等に用いられることが多い。伝統的にフランス料理では唐辛子の系統の辛み(カプサイシン)が強いのを嫌う傾向にあり、あくまでも料理の風味を引き締める目的で使われると考えるといい。}ごく少量で味付けする。

身が色付かない程度に表面を焼いたら\footnote{raidir (レディール)。}、白ワイン
2 dLとグラス1杯弱 \footnote{原文、un petit verre de
  (アンプティヴェールド)は本書において通常、10clすなわし100mlを指す。一般的なコニャック1杯は6〜9
  cl すなわち 60〜90 ml
  であるから、どちらの意味で理解しても差し支えないだろう。}のコニャックを注ぎ。中位の大きさのオマールの場合には、玉ねぎとにんじんを1〜2
mmくらいの細さの千切りにしてバターを加えて弱火で蒸し煮したものを大さじ2杯\footnote{本書における「大さじ1杯」une
  cuillerée (ユヌキュイユレ)が 15
  mlとはかぎらないことに注意。ニュアンスとしては「大きなスプーンで、場合によっては山盛り1杯」の意である。レードル1杯と読みかえてもいいと考えられるケースも多いので注意。}程度加える。粗みじん切りにしたパセリ1つまみと\protect\hyperlink{fumet-de-poisson}{魚のフォン}大さじ5〜6杯を加える。

鍋に蓋をして15分間程加熱する。

平皿またはやや深さのある皿に盛り付ける。煮汁に\protect\hyperlink{veloute-de-poisson}{魚のヴルテ}大さじ2杯を加えてとろみを付け、上等なバター100
gを入れてソースを仕上げる。ソースはオマールにかけて供する\footnote{ソース入れで別添にしない、ということ。}。

\hypertarget{le-veau-ux4ed4ux725bux306eux6599ux7406}{%
\subsection{◆LE VEAU
 仔牛の料理}\label{le-veau-ux4ed4ux725bux306eux6599ux7406}}

\hypertarget{les-sucs-duxe9rivuxe9s-du-veau-sautuxe9-uxe0-la-paysanne-ux4ed4ux725bux306eux30a2ux30e9ux30daux30a4ux30b6ux30f3ux30ccux304bux3089ux5fdcux7528ux3057ux305fux30b7ux30e5ux30c3ux30af}{%
\subsubsection{・Les sucs dérivés du « veau sauté à la paysanne
» 「仔牛のア・ラ・ペイザンヌ」から応用したシュック}\label{les-sucs-duxe9rivuxe9s-du-veau-sautuxe9-uxe0-la-paysanne-ux4ed4ux725bux306eux30a2ux30e9ux30daux30a4ux30b6ux30f3ux30ccux304bux3089ux5fdcux7528ux3057ux305fux30b7ux30e5ux30c3ux30af}}

\hypertarget{une-uxe9mulsion-veau-foin-inspiruxe9e-de-la-blanquette-de-veau-ux4ed4ux725bux306eux30d6ux30e9ux30f3ux30b1ux30c3ux30c8ux304bux3089ux7740ux60f3ux3057ux3066ux85c1ux3092ux4f7fux3063ux305fux4ed4ux725bux306eux30a8ux30dfux30e5ux30ebux30b8ux30e7ux30f3}{%
\subsubsection{・Une émulsion veau-foin inspirée de la « blanquette de
veau
» 「仔牛のブランケット」から着想して、藁を使った仔牛のエミュルジョン}\label{une-uxe9mulsion-veau-foin-inspiruxe9e-de-la-blanquette-de-veau-ux4ed4ux725bux306eux30d6ux30e9ux30f3ux30b1ux30c3ux30c8ux304bux3089ux7740ux60f3ux3057ux3066ux85c1ux3092ux4f7fux3063ux305fux4ed4ux725bux306eux30a8ux30dfux30e5ux30ebux30b8ux30e7ux30f3}}

\hypertarget{blanquette-de-veau-a-l-ancienne}{%
\subsubsection{仔牛のブランケット・クラシック}\label{blanquette-de-veau-a-l-ancienne}}

\frsub{Blanquette de veau à l'Ancienne}

ブランケットには、仔牛のばら肉、肩肉、肩ロースなどの部位を用いる。

肉を適当な大きさに切り分ける。肉が完全にかぶる位、たっぷりの\protect\hyperlink{fonds-blanc-ordinaire}{白いフォン}を注いて火にかける。ほんの少しだけ塩を加える。こまめに混ぜながら弱火で沸騰させること。浮いてくる泡などは丁寧に取り除くこと\footnote{écumer
  (エキュメ)。}。

にんじん小1本、クローブを刺した玉ねぎ1個、ブーケガルニ(ポワロー、パセリの枝、タイム、ローリエ)を加え、1時間半弱火で煮る。

\protect\hyperlink{roux-blanc}{白いルー}100 gと仔牛の煮汁1
\(\frac{3}{4}\)
Lで\protect\hyperlink{veloute}{ヴルテ}を作る。新鮮なマッシュルームの切りくず1つかみを加え、15分程煮ながら、浮いてくる不純物を丁寧に取り除く\footnote{dépouiller
  (デプイエ) ≒
  エキュメ。ソースなどの仕上げの際に時間をかけて丁寧に不純物を取り除くこと、および、さらに布で漉して陶製の容器などに入れてゆっくり混ぜながら冷まる(vanner
  ヴァネ)までの一連の作業もこの語で表現されることがしばしばだった。現代日本の調理場ではほとんど用いられない用語。}。

仔牛肉を取り出して水気をきり、必要ならきれいに形を整えてやる。別の鍋に移し、ブラン\footnote{肉や野菜を下茹でする際に、冷水に小麦粉と塩、ヴィネガーを加えて沸かし、クローヴを刺した玉ねぎとブーケガルニを加えてものを使用することがあり、これをブランと呼ぶ(blanc
  原義は白)。}で茹でた小玉ねぎ20個とマッシュルーム20個\footnote{このマッシュルームは通常、トゥルネと呼ばれる螺旋状の模様が入るように皮を剥いたもの。本文上部にあるマッシュルームの切りくずはこの際に出るものを利用している。}を加える。

提供直前に、卵黄5個と生クリーム 1 dLを加えてとろみを付ける\footnote{このとき、卵黄を生クリームでしっかり溶いておき、鍋全体をよく混ぜながらであれば、煮汁が沸騰した状態で加えても滑かなとろみが付けられる。20世紀中葉を代表する料理人のひとり、レモン・オリヴェ(Raymond
  Oliver,
  1909-1990)がテレビ放送の黎明期1954年〜1967年にかけて出演した「料理の魔法のテクニック」Arts
  et magie de la
  cuisineにおいて、鶏のフリカセの回でこの方法を実際に画面に示した。このとき、煮汁は完全な沸騰状態だった。}。レモン果汁少々、ナツメグ粉末少々を加え、ソースを仕上げる。ソースを布で漉す。これを鍋に入れた肉と野菜類にかけ、沸騰しない程度に温める。深皿に盛り、パセリのみじん切り少々を振りかける。
