\hypertarget{ux30dbux30efux30a4ux30c8ux7cfbux306eux6d3eux751fux30bdux30fcux30b9}{%
\section{ホワイト系の派生ソース}\label{ux30dbux30efux30a4ux30c8ux7cfbux306eux6d3eux751fux30bdux30fcux30b9}}

\hypertarget{petites-sauces-blanches-composuxe9es-et-de-ruxe9ductions}{%
\subsection{Petites Sauces Blanches Composées et de
Réductions}\label{petites-sauces-blanches-composuxe9es-et-de-ruxe9ductions}}

\maeaki
\begin{recette}
\hypertarget{ux30bdux30fcux30b9ux30a2ux30ebux30d3ux30e5ux30d5ux30a7ux30e91}{%
\subsubsection[ソース・アルビュフェラ]{\texorpdfstring{ソース・アルビュフェラ\footnote{ナポレオン軍の元帥、ルイ・ガブリエル・スーシェ
  Louis-Gabriel Suchet, duc d'Albufera
  (1770〜1826)のこと。スペイン戦役の際にそれ
  までの軍功を称えられ、ナポレオンが1812年にアルビュフェラ公爵位を新
  設して授けた。帝政期の英雄のひとりであり、アルビュフェラおよびスー
  シェの名を冠した料理がいくつかある。1814年に帝政が崩壊した後も軍務、
  政務に携わり、最終的にフランス貴族院議員の地位を得た。アルビュフェ
  ラ公爵位については、1815年7月24日の勅令においてに正式に抹消されて
  いる。このソースの特徴は赤ピーマン(パプリカ)を加熱してなめらかに
  すり潰し、バターに練り込んだものを使う点にあるが、どのような経緯で
  このソースに赤ピーマンを用いるようになったのかは不明。ただし、この
  ソースを合わせる「肥鶏 アルビュフェラ」は詰め物(ファルス)に米を
  用いるが、アルビュフェラは湖の周辺の湿地帯で米の生産がおこなわれて
  いるという点では一応の関連性が認められよう。なお、アルビュフェラは
  バレンシアの湖とそこに形成された潟であり、現在はバレンシア州のアル
  ブフェーラ自然公園となっている。}}{ソース・アルビュフェラ}}\label{ux30bdux30fcux30b9ux30a2ux30ebux30d3ux30e5ux30d5ux30a7ux30e91}}

\hypertarget{sauce-albufera}{%
\paragraph{Sauce Albuféra}\label{sauce-albufera}}

\index{そーす@ソース!あるひゆふえら@---・アルビュフェラ}
\index{あるひゆふえら@アルビュフェラ!そーす@ソース・---}
\index{sauce@sauce!albufera@--- Albuféra}
\index{albufera@Albuféra!sauce@Sauce ---}

\protect\hyperlink{sauce-supreme}{ソース・シュプレーム}1
Lあたりに、溶かしたブロンド色
の\protect\hyperlink{glace-de-viande}{グラスドヴィアンド}2
dlと、標準的な分量比率で作っ た\href{}{赤ピーマンバター}50 gを加える。

\maeaki

\hypertarget{ux30bdux30fcux30b9ux30a2ux30e1ux30eaux30b1ux30fcux30cc3}{%
\subsubsection[ソース・アメリケーヌ]{\texorpdfstring{ソース・アメリケーヌ\footnote{アメリケーヌという名称の由来は諸説あるが、19世紀フランスの料理人
  ピエール・フレス Pierre Fraysse がアメリカで働いた後にパリで1853年
  に開いたレストラン「シェ・ピーターズ」でこの料理名で提供したという
  のが定説。ただし、1853年以前にレストラン「ボヌフォワ」に「ラングドッ
  ク産オマール ソース・アメリケーヌ添え」というメニューあり、フレス
  はその料理に改変を加えたか、名前だけをシンプルに「アメリケーヌ」と
  した程度という説もある。かつては、オマールの主産地のひとつブルター
  ニュ地方を意味する古い形容詞 armoricain(e) アルモリカン、アルモリ
  ケーヌの音が変化した料理名だと主張されることもあったが、19世紀には
  南仏産が中心であったトマトを用いる点で矛盾が生じてしまう。いずれに
  しても、この料理名がフレスの店シェ・ピーターズを基点として広く知ら
  れるようになったことは事実と考えていい。}}{ソース・アメリケーヌ}}\label{ux30bdux30fcux30b9ux30a2ux30e1ux30eaux30b1ux30fcux30cc3}}

\hypertarget{sauce-americaine}{%
\paragraph{Sauce Américaine}\label{sauce-americaine}}

\index{そーす@ソース!あめりけーぬ@---・アメリケーヌ}
\index{あめりふう@アメリカ風!そーす@ソース・アメリケーヌ}
\index{sauce@sauce!americaine@--- Américaine}
\index{americain@américain!sauce americaine@Sauce Américaine}

このソースは\protect\hyperlink{homard-a-l-americaine}{オマール・アメリケーヌ}という料理
そのものと言っていい(「魚料理」の章、甲殻類、\protect\hyperlink{homard-a-l-americaine}{オマール・アメリケー
ヌ}参照)。

このソースは通常、オマールの身をガルニチュールとした魚料理に添えられる。
オマールの身をやや斜めになるよう厚さ1 cm程度の輪切りにし\footnote{escalopper
  エスカロペ。エスカロップに切る。ここで用いられるオマー
  ルは900g〜1kg程度の大きさのものを想定していることに注意。}、魚料理の
ガルニチュールとして供するわけだ。

\maeaki

\hypertarget{ux30a2ux30f3ux30c1ux30e7ux30d3ux30bdux30fcux30b9}{%
\subsubsection{アンチョビソース}\label{ux30a2ux30f3ux30c1ux30e7ux30d3ux30bdux30fcux30b9}}

\hypertarget{sauce-anchois}{%
\paragraph{Sauce Anchois}\label{sauce-anchois}}

\index{そーす@ソース!あんちょうい@アンチョビ---}
\index{あんちょひ@アンチョビ!そーす@---ソース}
\index{sauce@sauce!anchois@--- Anchois}
\index{anchois@anchois!sauce anchois@Sauce ---}

\href{}{ノルマンディー風ソース}8
dlを、バターを加える前のところまで作る。こ
れに\href{}{アンチョビバター}125 gを混ぜ込む。アンチョビのフィレ50
gを洗い、 よく水気を絞ってから小さなさいの目に切ったのを加えて仕上げる。

\ldots{}\ldots{}魚料理用。

\maeaki

\hypertarget{ux30bdux30fcux30b9ux30aaux30fcux30edux30fcux30eb4}{%
\subsubsection[ソース・オーロール]{\texorpdfstring{ソース・オーロール\footnote{夜明けの光、曙光のこと。オーロラの意味もあるため、日本では「オーロラソース」と呼ばれることもあるが、マヨネーズとトマトケチャップを同量で混ぜ合わせたものもそう呼ばれることが多いので注意。}}{ソース・オーロール}}\label{ux30bdux30fcux30b9ux30aaux30fcux30edux30fcux30eb4}}

\hypertarget{sauce-aurore}{%
\paragraph{Sauce Aurore}\label{sauce-aurore}}

\index{そーす@ソース!おーろーる@---・オーロール}
\index{おーろーる@オーロール!そーす@ソース・---}
\index{sauce@sauce!aurore@--- Aurore}
\index{aurore@aurore!sauce@Sauce ---}

\protect\hyperlink{veloute}{ヴルテ}に真っ赤なトマトピュレを加えたもの。分量は、ヴルテが\troisquarts{}に対し、トマトピュレ\unquart{}とする。仕上げに、ソース1
Lあたり100 gのバターを加える。

\ldots{}\ldots{}卵料理、仔牛、仔羊肉の料理、鶏料理用。

\maeaki

\hypertarget{ux9b5aux6599ux7406ux7528ux30bdux30fcux30b9ux30aaux30fcux30edux30fcux30eb}{%
\subsubsection{魚料理用ソース・オーロール}\label{ux9b5aux6599ux7406ux7528ux30bdux30fcux30b9ux30aaux30fcux30edux30fcux30eb}}

\hypertarget{sauce-aurore-maigre}{%
\paragraph{Sauce Aurore maigre}\label{sauce-aurore-maigre}}

\index{そーす@ソース!おーろーるさかなよう@魚料理用---・オーロール}
\index{おーろーる@オーロール!そーすさかな@魚料理用ソース・---}
\index{sauce@sauce!aurore maigre@--- Aurore maigre}
\index{aurore@aurore!sauce maigre@Sauce --- maigre}

\protect\hyperlink{veloute-de-poisson}{魚料理用ヴルテ}に、上記と同じ割合でトマトピュレ
を加える。ソース1 Lあたりバター125 gを加えて仕上げる。

\ldots{}\ldots{}魚料理用

\maeaki

\hypertarget{ux30d0ux30a4ux30a8ux30ebux30f3ux98a8ux30bdux30fcux30b9}{%
\subsubsection{バイエルン風ソース}\label{ux30d0ux30a4ux30a8ux30ebux30f3ux98a8ux30bdux30fcux30b9}}

\hypertarget{sauce-bavaroise}{%
\paragraph{Sauce Bavaroise}\label{sauce-bavaroise}}

\index{そーす@ソース!はいえるんふう@バイエルン風---}
\index{はいえるんふう@バイエルン風!そーす@---ソース}
\index{sauce@sauce!bavarois@--- Bavaroise}
\index{bavarois@bavarois!sauce bavaroise@Sauce Bavaroise}

ヴィネガー5
dlにタイムとローリエの葉少々とパセリの枝4本、大粒のこしょう7〜8個と、おろした\footnote{原文
  râpé \textless{} râpe
  ラープと呼ばれる器具を用いておろすが、日本のおろし金と目の大きさが違うので注意。多くの場合、マンドリーヌ
  mandrine と呼ばれる野菜用スライサーにこの機能が付属している。}レフォール\footnote{raifort
  西洋わさび、ホースラディッシュ。}大さじ2杯を加え、半量になるまで煮詰める。

この煮詰めた汁に卵黄6個を加え\footnote{卵黄を加える前に一度漉しておいたほうがいいだろう。}、\protect\hyperlink{sauce-hollandaise}{オランデーズソース}を作る要領で、バター400
gと大さじ1\undemi{}杯の水を少しずつ加えながら、ソースがしっかり乳化するまで混ぜていく。布で漉す。

\protect\hyperlink{beurre-d-ecrevisse}{エクルヴィスバター}100
gと泡立てた生クリーム大さじ2杯、さいの目に切ったエクルヴィスの尾の身を加えて仕上げる。

\ldots{}\ldots{}魚料理用のこのソースは、ムースのような仕上りにすること。

\end{recette}