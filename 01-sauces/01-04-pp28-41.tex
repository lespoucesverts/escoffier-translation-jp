\hypertarget{ux30dbux30efux30a4ux30c8ux7cfbux306eux6d3eux751fux30bdux30fcux30b9}{%
\section{ホワイト系の派生ソース}\label{ux30dbux30efux30a4ux30c8ux7cfbux306eux6d3eux751fux30bdux30fcux30b9}}

\hypertarget{petites-sauces-blanches-composuxe9es-et-de-ruxe9ductions}{%
\subsection{Petites Sauces Blanches, Composées et de
Réductions}\label{petites-sauces-blanches-composuxe9es-et-de-ruxe9ductions}}

\maeaki
\begin{recette}
\hypertarget{ux30bdux30fcux30b9ux30a2ux30ebux30d3ux30e5ux30d5ux30a7ux30e91}{%
\subsubsection[ソース・アルビュフェラ]{\texorpdfstring{ソース・アルビュフェラ\footnote{ナポレオン軍の元帥、ルイ・ガブリエル・スーシェ
  Louis-Gabriel Suchet, duc d'Albufera
  (1770〜1826)のこと。スペイン戦役の際にそれ
  までの軍功を称えられ、ナポレオンが1812年にアルビュフェラ公爵位を新
  設して授けた。帝政期の英雄のひとりであり、アルビュフェラおよびスー
  シェの名を冠した料理がいくつかある。1814年に帝政が崩壊した後も軍務、
  政務に携わり、最終的にフランス貴族院議員の地位を得た。アルビュフェ
  ラ公爵位については、1815年7月24日の勅令においてに正式に抹消されて
  いる。このソースの特徴は赤ピーマン(パプリカ)を加熱してなめらかに
  すり潰し、バターに練り込んだものを使う点にあるが、どのような経緯で
  このソースに赤ピーマンを用いるようになったのかは不明。ただし、この
  ソースを合わせる「肥鶏 アルビュフェラ」は詰め物(ファルス)に米を
  用いるが、アルビュフェラは湖の周辺の湿地帯で米の生産がおこなわれて
  いるという点では一応の関連性が認められよう。なお、アルビュフェラは
  バレンシアの湖とそこに形成された潟であり、現在はバレンシア州のアル
  ブフェーラ自然公園となっている。}}{ソース・アルビュフェラ}}\label{ux30bdux30fcux30b9ux30a2ux30ebux30d3ux30e5ux30d5ux30a7ux30e91}}

\hypertarget{sauce-albufera}{%
\paragraph{Sauce Albuféra}\label{sauce-albufera}}

\index{そーす@ソース!あるひゆふえら@---・アルビュフェラ}
\index{あるひゆふえら@アルビュフェラ!そーす@ソース・---}
\index{sauce@sauce!albufera@--- Albuféra}
\index{albufera@Albuféra!sauce@Sauce ---}

\protect\hyperlink{sauce-supreme}{ソース・シュプレーム}1
Lあたりに、溶かしたブロンド色
の\protect\hyperlink{glace-de-viande}{グラスドヴィアンド}2
dlと、標準的な分量比率で作っ た\href{}{赤ピーマンバター}50 gを加える。

\maeaki

\hypertarget{ux30bdux30fcux30b9ux30a2ux30e1ux30eaux30b1ux30fcux30cc3}{%
\subsubsection[ソース・アメリケーヌ]{\texorpdfstring{ソース・アメリケーヌ\footnote{アメリケーヌという名称の由来は諸説あるが、19世紀フランスの料理人
  ピエール・フレス Pierre Fraysse がアメリカで働いた後にパリで1853年
  に開いたレストラン「シェ・ピーターズ」でこの料理名で提供したという
  のが定説。ただし、1853年以前にレストラン「ボヌフォワ」に「ラングドッ
  ク産オマール ソース・アメリケーヌ添え」というメニューあり、フレス
  はその料理に改変を加えたか、名前だけをシンプルに「アメリケーヌ」と
  した程度という説もある。かつては、オマールの主産地のひとつブルター
  ニュ地方を意味する古い形容詞 armoricain(e) アルモリカン、アルモリ
  ケーヌの音が変化した料理名だと主張されることもあったが、19世紀には
  南仏産が中心であったトマトを用いる点で矛盾が生じてしまう。いずれに
  しても、この料理名がフレスの店シェ・ピーターズを基点として広く知ら
  れるようになったことは事実と考えていい。}}{ソース・アメリケーヌ}}\label{ux30bdux30fcux30b9ux30a2ux30e1ux30eaux30b1ux30fcux30cc3}}

\hypertarget{sauce-americaine}{%
\paragraph{Sauce Américaine}\label{sauce-americaine}}

\index{そーす@ソース!あめりけーぬ@---・アメリケーヌ}
\index{あめりふう@アメリカ風!そーす@ソース・アメリケーヌ}
\index{sauce@sauce!americaine@--- Américaine}
\index{americain@américain!sauce americaine@Sauce Américaine}

このソースは\protect\hyperlink{homard-a-l-americaine}{オマール・アメリケーヌ}という料理
そのものと言っていい(「魚料理」の章、甲殻類、\protect\hyperlink{homard-a-l-americaine}{オマール・アメリケー
ヌ}参照)。

このソースは通常、オマールの身をガルニチュールとした魚料理に添えられる。
オマールの身をやや斜めになるよう厚さ1 cm程度の輪切りにし\footnote{escalopper
  エスカロペ。エスカロップに切る。ここで用いられるオマー
  ルは900g〜1kg程度の大きさのものを想定していることに注意。}、魚料理の
ガルニチュールとして供するわけだ。

\maeaki

\hypertarget{ux30a2ux30f3ux30c1ux30e7ux30d3ux30bdux30fcux30b9}{%
\subsubsection{アンチョビソース}\label{ux30a2ux30f3ux30c1ux30e7ux30d3ux30bdux30fcux30b9}}

\hypertarget{sauce-anchois}{%
\paragraph{Sauce Anchois}\label{sauce-anchois}}

\index{そーす@ソース!あんちょうい@アンチョビ---}
\index{あんちょひ@アンチョビ!そーす@---ソース}
\index{sauce@sauce!anchois@--- Anchois}
\index{anchois@anchois!sauce anchois@Sauce ---}

\href{}{ノルマンディー風ソース}8
dlを、バターを加える前のところまで作る。こ
れに\href{}{アンチョビバター}125 gを混ぜ込む。アンチョビのフィレ50
gを洗い、 よく水気を絞ってから小さなさいの目に切ったのを加えて仕上げる。

\ldots{}\ldots{}魚料理用。

\maeaki

\hypertarget{ux30bdux30fcux30b9ux30aaux30fcux30edux30fcux30eb4}{%
\subsubsection[ソース・オーロール]{\texorpdfstring{ソース・オーロール\footnote{夜明けの光、曙光のこと。オーロラの意味もあるため、日本では「オー
  ロラソース」と呼ばれることもあるが、マヨネーズとトマトケチャップを
  同量で混ぜ合わせたものもそう呼ばれることが多いので注意。}}{ソース・オーロール}}\label{ux30bdux30fcux30b9ux30aaux30fcux30edux30fcux30eb4}}

\hypertarget{sauce-aurore}{%
\paragraph{Sauce Aurore}\label{sauce-aurore}}

\index{そーす@ソース!おーろーる@---・オーロール}
\index{おーろーる@オーロール!そーす@ソース・---}
\index{sauce@sauce!aurore@--- Aurore}
\index{aurore@aurore!sauce@Sauce ---}

\protect\hyperlink{veloute}{ヴルテ}に真っ赤なトマトピュレを加えたもの。分量は、ヴルテが\troisquarts{}に対し、トマトピュレ\unquart{}とする。仕上げに、ソース1
Lあたり100 gのバターを加える。

\ldots{}\ldots{}卵料理、仔牛、仔羊肉の料理、鶏料理用。

\maeaki

\hypertarget{ux9b5aux6599ux7406ux7528ux30bdux30fcux30b9ux30aaux30fcux30edux30fcux30eb}{%
\subsubsection{魚料理用ソース・オーロール}\label{ux9b5aux6599ux7406ux7528ux30bdux30fcux30b9ux30aaux30fcux30edux30fcux30eb}}

\hypertarget{sauce-aurore-maigre}{%
\paragraph{Sauce Aurore maigre}\label{sauce-aurore-maigre}}

\index{そーす@ソース!おーろーるさかなよう@魚料理用---・オーロール}
\index{おーろーる@オーロール!そーすさかな@魚料理用ソース・---}
\index{sauce@sauce!aurore maigre@--- Aurore maigre}
\index{aurore@aurore!sauce maigre@Sauce --- maigre}

\protect\hyperlink{veloute-de-poisson}{魚料理用ヴルテ}に、上記と同じ割合でトマトピュレ
を加える。ソース1 Lあたりバター125 gを加えて仕上げる。

\ldots{}\ldots{}魚料理用

\maeaki

\hypertarget{ux30d0ux30a4ux30a8ux30ebux30f3ux98a8ux30bdux30fcux30b9}{%
\subsubsection{バイエルン風ソース}\label{ux30d0ux30a4ux30a8ux30ebux30f3ux98a8ux30bdux30fcux30b9}}

\hypertarget{sauce-bavaroise}{%
\paragraph{Sauce Bavaroise}\label{sauce-bavaroise}}

\index{そーす@ソース!はいえるんふう@バイエルン風---}
\index{はいえるんふう@バイエルン風!そーす@---ソース}
\index{sauce@sauce!bavarois@--- Bavaroise}
\index{bavarois@bavarois!sauce bavaroise@Sauce Bavaroise}

ヴィネガー5
dlにタイムとローリエの葉少々とパセリの枝4本、大粒のこしょう7〜8個と、おろした\footnote{原文
  râpé \textless{} râpe
  ラープと呼ばれる器具を用いておろすが、日本のおろし金と目の大きさが違うので注意。多くの場合、マンドリーヌ
  mandrine と呼ばれる野菜用スライサーにこの機能が付属している。}レフォール\footnote{raifort
  西洋わさび、ホースラディッシュ。}大さじ2杯を加え、半量になるまで煮詰める。

この煮詰めた汁に卵黄6個を加え\footnote{卵黄を加える前に一度漉しておいたほうがいいだろう。}、\protect\hyperlink{sauce-hollandaise}{オランデーズソース}を作る要領で、バター400
gと大さじ1\undemi{}杯の水を少しずつ加えながら、ソースがしっかり乳化するまで混ぜていく。布で漉す。

\protect\hyperlink{beurre-d-ecrevisse}{エクルヴィスバター}100
gと泡立てた生クリーム大さじ2杯、さいの目に切ったエクルヴィスの尾の身を加えて仕上げる。

\ldots{}\ldots{}魚料理用のこのソースは、ムースのような仕上りにすること。

\maeaki

\hypertarget{ux30bdux30fcux30b9ux30d9ux30a2ux30ebux30cdux30fcux30ba8}{%
\subsubsection[ソース・ベアルネーズ]{\texorpdfstring{ソース・ベアルネーズ\footnote{ベアルヌというのは旧地方名で、フランス南西部、現在のピレネー・ア
  トランティック県のことを指すが、このソースはその地方とはまったく関
  係がない。19世紀パリ郊外のレストラン「パヴィヨン・アンリIV」の店名
  に掲げられているアンリ四世がベアルヌのポーで生誕したことにちなんで
  命名したソース名というのが定説。}}{ソース・ベアルネーズ}}\label{ux30bdux30fcux30b9ux30d9ux30a2ux30ebux30cdux30fcux30ba8}}

\hypertarget{sauce-bearnaise}{%
\paragraph{Sauce Béarnaise}\label{sauce-bearnaise}}

\index{そーす@ソース!へあるねーす@---・ベアルネーズ}
\index{へあるぬふう@ベアルヌ風!そーす@ソース・---}
\index{へあるねーす@ベアルネーズ!そーす@ソース・---}
\index{sauce@sauce!bearnaise@--- Béarnaise}
\index{bearnais@béarnais!sauce bearnaise@Sauce Béarnaise}

白ワイン2 dlとエストラゴンヴィネガー2 dlに、エシャロットのみじん切り大
さじ4杯、枝のままの粗く刻んだエストラゴン20 g、セルフイユ10 g、粗挽き
こしょう5 g、塩1つまみを加えて、\untiers{}量になるまで煮詰める。

煮詰まったら、数分間放置して温度を下げる。ここに卵黄6個を加え、弱火に
かけて、生のバター(あるいはあらかじめ溶かしておいてもいい)500 gを加
えて軽くホイップしながらなめらかになるよう混ぜる。

卵黄に徐々に火が通っていくことでソースにとろみが付くので、絶対に弱火で
作業をすること\footnote{卵黄をソースのとろみ付けに用いること自体は中世から行なわれていた。
  開放式の炉の上に鍋を鉤で吊っている場合は鍋を火から外す必要があった
  が、その後の閉鎖式かまどや、オーブンの機能も備えた fourneau フルノー
  (日本の調理現場ではストーブあるいはピアノと呼ばれることも多い)の
  場合、熱の弱い部分に鍋を置けばいいことになる。また、このソースのよ
  うにバターが中心となる場合は水よりも高温になりやすいので本文にある
  ように注意が必要だが、ブランケットのような水が中心のものに卵黄を加
  えてとろみを付ける場合、よく溶きほぐした卵黄を、鍋全体をしっかり混
  ぜながら加えれば比較的高温(微沸騰程度)でも問題なくきれいにとろみ
  が付く。}。

バターを混ぜ込んだら、布で漉して味を調える。カイエンヌごく少量を加えて
風味を引き締める。仕上げに、刻んだエストラゴン大さじ杯とセルフイユ大さ
じ\undemi{}杯を加える。

\ldots{}\ldots{}牛、羊肉のグリル用。

\hypertarget{ux539fux6ce8}{%
\subparagraph{【原注】}\label{ux539fux6ce8}}

このソースを熱々で提供しようとは考えないこと。このソースは要するにバター
で作ったマヨネーズなのだ。ほの温い程度で充分であり、もし熱くし過ぎてし
まうと、ソースが分離してしまう。

そうなってしまったら、冷水少々を加えて泡立て器でホイップして元のあるべ
き状態に戻してやること。

\maeaki

\hypertarget{ux30c8ux30deux30c8ux5165ux308aux30bdux30fcux30b9ux30d9ux30a2ux30ebux30cdux30fcux30ba-ux30bdux30fcux30b9ux30b7ux30e7ux30edux30f310}{%
\subsubsection[トマト入りソース・ベアルネーズ /
ソース・ショロン]{\texorpdfstring{トマト入りソース・ベアルネーズ /
ソース・ショロン\footnote{19世紀後半、パリで有名レストラン「ヴォワザン」の料理長を務めた
  アレクサンドル・ショロン Alexandre Choron (1837〜1924)。自ら考案
  し、命名したという。}}{トマト入りソース・ベアルネーズ / ソース・ショロン}}\label{ux30c8ux30deux30c8ux5165ux308aux30bdux30fcux30b9ux30d9ux30a2ux30ebux30cdux30fcux30ba-ux30bdux30fcux30b9ux30b7ux30e7ux30edux30f310}}

\hypertarget{sauce-bearnaise-tomatee}{%
\paragraph{Sauce Béarnaise tomatée, dite Sauce
Choron}\label{sauce-bearnaise-tomatee}}

\index{そーす@ソース!へあるねーすとまといり@トマト入り---・ベアルネーズ}
\index{へあるぬふう@ベアルヌ風!そーすとまといり@トマト入りソース・ベアルネーズ}
\index{そーす@ソース!しょろん@---・ショロン}
\index{しょろん@ショロン!そーす@ソース・---}
\index{sauce@sauce!bearnaise tomatee@--- Béarnaise tomatée}
\index{bearnais@b\'earnais!sauce bearnaise tomatee@Sauce Béarnaise tomatée}
\index{sauce@sauce!choron@--- Choron}
\index{choron@Choron!sauce@Sauce ---}

ソース・ベアルネーズを上記のとおりに作るが、最後にセルフイユとエストラ
ゴンのみじん切りは加えない。充分固めに作っておき、ソースの\unquart{}量
の、充分に煮詰めたトマトピュレを加える。ソースの濃度が丁度いい具合にな
るよう注意すること。

\ldots{}\ldots{}\href{}{トゥルヌド・ショロン}、および他のさまざまな料理に添える。

\maeaki

\hypertarget{ux30b0ux30e9ux30b9ux30c9ux30f4ux30a3ux30a2ux30f3ux30c9ux5165ux308aux30bdux30fcux30b9ux30d9ux30a2ux30ebux30cdux30fcux30ba-ux30bdux30fcux30b9ux30d5ux30a9ux30a4ux30e811-ux30bdux30fcux30b9ux30f4ux30a1ux30edux30ef12}{%
\subsubsection[グラスドヴィアンド入りソース・ベアルネーズ /
ソース・フォイヨ /
ソース・ヴァロワ]{\texorpdfstring{グラスドヴィアンド入りソース・ベアルネーズ
/ ソース・フォイヨ\footnote{19世紀〜20世紀初頭にパリにあったレストランおよびそのオーナーシェ
  フの名。このソースを使った「仔牛の背肉・フォイヨ」がスペシャリテだっ
  たという。} / ソース・ヴァロワ\footnote{ヴァロワ王家およびヴァロワ公爵であったルイ・フィリップ(7月王政
  期のフランス国王。在位1830〜1848)にちなんだ名称。前出のフォイヨは
  レストランを開く以前、ルイ・フィリップに仕えていた。}}{グラスドヴィアンド入りソース・ベアルネーズ / ソース・フォイヨ / ソース・ヴァロワ}}\label{ux30b0ux30e9ux30b9ux30c9ux30f4ux30a3ux30a2ux30f3ux30c9ux5165ux308aux30bdux30fcux30b9ux30d9ux30a2ux30ebux30cdux30fcux30ba-ux30bdux30fcux30b9ux30d5ux30a9ux30a4ux30e811-ux30bdux30fcux30b9ux30f4ux30a1ux30edux30ef12}}

\hypertarget{sauce-bearnaise-a-la-glace-de-viande}{%
\paragraph{Sauce Béarnaise à la glace de viande, dite Foyot, ou
Valois}\label{sauce-bearnaise-a-la-glace-de-viande}}

\index{へあるぬふう@ベアルヌ風!そーすぐらすとういあんといり@グラスドヴィアンド入りソース・ベアルネーズ}
\index{そーす@ソース!へあるねーすくらすどういあんといり@---・ベアルネーズ(グラス・ド・ヴィアンド入り)}
\index{そーす@ソース!ふおいよ@---・フォイヨ}
\index{ふおいよ@フォイヨ!そーす@ソース・---}
\index{そーす@ソース!うあろわ@---・ヴァロワ}
\index{うあろわ@ヴァロワ!そーす@ソース・---}
\index{sauce@sauce!bearnaise a la glace de viande@--- Béarnaise à la glace de viande}
\index{bearnais@b\'earnais!sauce bearnaise a la glace de viande@Sauce Béarnaise à la glace de viande}
\index{sauce@sauce!foyot@--- Foyot} \index{foyot@Foyot!sauce@Sauce ---}
\index{sauce@sauce!valois@--- Valois}
\index{valois@Valois!sauce@Sauce ---}

標準的な\protect\hyperlink{sauce-bearnaise}{ソース・ベアルネーズ}を上記の分量で、固めに作る。溶かした\protect\hyperlink{glace-de-viande}{グラスドヴィアンド}を少しずつ加えて仕上げる。

\ldots{}\ldots{}牛、羊肉のグリル用。

\maeaki

\hypertarget{ux30bdux30fcux30b9ux30d9ux30ebux30b7ux30fc13}{%
\subsubsection[ソース・ベルシー]{\texorpdfstring{ソース・ベルシー\footnote{パリ東部、セーヌ川左岸にある地名。かつては荷揚げ港があり、19世
  紀には小さなレストランが多く店を構えていたという。}}{ソース・ベルシー}}\label{ux30bdux30fcux30b9ux30d9ux30ebux30b7ux30fc13}}

\hypertarget{sauce-bercy}{%
\paragraph{Sauce Bercy}\label{sauce-bercy}}

\index{そーす@ソース!へるしー@---・ベルシー}
\index{へるしー@ベルシー!そーす@ソース・---}
\index{sauce@sauce!bercy@--- Bercy} \index{bercy@Bercy!sauce@Sauce ---}

細かくみじん切りにしたエシャロット大さじ2杯をバターでさっと色付かない
よう炒める。白ワイン2\undemi{}
dlと\protect\hyperlink{fumet-de-poisson}{魚のフュメ}か、
このソースを合わせる魚の煮汁2\undemi{} dlを注ぐ。

\deuxtiers{}量弱まで煮詰めたら、\protect\hyperlink{veloute-de-poisson}{ヴル
テ}\troisquarts{} Lを加える。ひと煮立ちさせてから、
鍋を火から外し、バター100 gとパセリのみじん切り大さじ1杯を加えて仕上げ
る。

\maeaki

\hypertarget{ux30bdux30fcux30b9ux30aaux30d6ux30fcux30eb16-ux30bdux30fcux30b9ux30d0ux30bfux30ebux30c914}{%
\subsubsection[ソース・オ・ブール /
ソース・バタルド]{\texorpdfstring{ソース・オ・ブール\footnote{本書には、日本でもかつて有名だった、エシャロットのみじん切りを
  加えたヴィネガーを煮詰めてバターを溶かし込んだ魚料理用ソース「ソー
  ス・ブールブラン」Sauce (au) Beurre blanc は収録されていない。この
  ソース・ブールブランはナント地方やアンジュー地方で淡水魚アローズや
  ブロシェに合わせる伝統的なソース。1890年頃にナント地方の女性料理人
  クレマンス・ルフーヴルが、ソース・ベアルネーズを作るつもりが誤って
  卵を加えるのを忘れてしまった結果として出来たものだとも言われている。}
/ ソース・バタルド\footnote{バタルドは「雑種の、中間の」の意。卵黄とバターだけでとろみを付
  ける\protect\hypertarget{sauce-hollandaise}{}{ソース・オランデーズ}と似てはいるが小麦粉
  も使うことからこの名が付いたと言われている。なお、パンのバタール
  bâtard も同じ語だが、細いバゲットと太いドゥーリーヴルの「中間」
  の太さとだからというのが通説。}}{ソース・オ・ブール / ソース・バタルド}}\label{ux30bdux30fcux30b9ux30aaux30d6ux30fcux30eb16-ux30bdux30fcux30b9ux30d0ux30bfux30ebux30c914}}

\hypertarget{sauce-au-beurre}{%
\paragraph{Sauce au Beurre, dite Sauce Bâtarde}\label{sauce-au-beurre}}

\index{そーす@ソース!ふーる@---・オ・ブール}
\index{はたー@バター!そーす@ソース・オ・ブール}
\index{そーす@ソース!はたると@---・バタルド}
\index{はたると@バタルド!そーす@ソース・---}
\index{sauce@sauce!beurre@--- au Beurre}
\index{beurre@beurre!sauce@Sauce au Beurre}
\index{sauce@sauce!batarde@--- Bâtarde}
\index{batard@bâtard!sauce@Sauce Bâtarde}

小麦粉45 gと溶かしバター45gをよく混ぜ合わせ粘土状にする。そこに、7 gの
塩を加えた熱湯7\undemi{} dlを一気に注ぎ、泡立て器で勢いよく混ぜ合わせ
る。とろみ付け用の卵黄5個を生クリーム大さじ1\undemi{}杯でゆるめたもの
と、レモン汁少々を加える。

布で漉し、鍋を火から外して、良質なバター300gを加えて仕上げる。

\ldots{}\ldots{}アスパラガスや、さまざまな魚のブイイ\footnote{茹でたもの、の意。}

\hypertarget{ux539fux6ce8-1}{%
\subparagraph{【原注】}\label{ux539fux6ce8-1}}

このソースはとろみを付けた後、湯煎にかけておき、提供直前にバターを加え
るようにするといい。

\maeaki

\hypertarget{ux30bdux30fcux30b9ux30dcux30ccux30d5ux30a9ux30ef-ux767dux30efux30a4ux30f3ux3067ux4f5cux308bux30dcux30ebux30c9ux30fcux98a8ux30bdux30fcux30b9}{%
\subsubsection{ソース・ボヌフォワ /
白ワインで作るボルドー風ソース}\label{ux30bdux30fcux30b9ux30dcux30ccux30d5ux30a9ux30ef-ux767dux30efux30a4ux30f3ux3067ux4f5cux308bux30dcux30ebux30c9ux30fcux98a8ux30bdux30fcux30b9}}

\hypertarget{sauce-bonnefoy}{%
\paragraph{Sauce Bonnefoy, ou Sauce Bordelaise au vin
blanc}\label{sauce-bonnefoy}}

\index{ほぬふおわ@ボヌフォワ!そーす@ソース・---}
\index{そーす@ソース!おぬふおわ@---・ボヌフォワ}
\index{そーす@ソース!ほるどーふうしろわいん@ボルドー風--- (白)}
\index{ほるどーふう@ボルドー風!そーす@---ソース(白)}
\index{sauce@sauce!bonnefoy@--- Bonnefoy}
\index{bonnefoy@Bonnefoy!sauce@Sauce ---}
\index{sauce@sauce!bordelaise vin blanc@--- Bordelaise au vin blanc}
\index{bordelais@bordelais!sauce vin blanc@Sauce Bordelaise au vin blanc}

ブラウン系の派生ソースの節で採り上げた、赤ワインを用いて作る\protect\hyperlink{sauce-bordelaise}{ボルドー
風ソース}とまったく同じ作り方だが、赤ワインではなく、
グラーヴかソテルヌの白ワインを用いる。また\protect\hyperlink{sauce-espagnole}{ソース・エスパニョ
ル}ではなく\protect\hyperlink{veloute}{標準的なヴルテ}を使うこと。

このソースは仕上げに、みじん切りにしたエストラゴンを加える。

\ldots{}\ldots{}魚のグリル、白身肉のグリル用。

\maeaki

\hypertarget{ux30d6ux30ebux30bfux30fcux30cbux30e5ux98a8ux30bdux30fcux30b9}{%
\subsubsection{ブルターニュ風ソース}\label{ux30d6ux30ebux30bfux30fcux30cbux30e5ux98a8ux30bdux30fcux30b9}}

\hypertarget{sauce-bretonne}{%
\paragraph{Sauce Bretonne}\label{sauce-bretonne}}

\index{そーす@ソース!ぶるたーにゅふうしろ@ブルターニュ風---(ホワイト系)}
\index{ぶるたーにゅふう@ブルターニュ風!そーすしろ@---ソース(ホワイト系)}
\index{sauce@sauce!bretonne blanche@--- Bretonne (blanche)}
\index{breton@breton!sauce blanche@Sauce Bretonne (blanche)}

長さ3〜5 cm位の、ごく細い千切り\footnote{julienne ジュリエンヌ。}にしたポワローの白い部分30
gとセロリの白い部分30 g、玉ねぎ30 g、マッシュルーム30
gをバターで完全に火が通るまで鍋に蓋をして弱火で蒸し煮する\footnote{étuver
  エチュヴェ。本来は油脂とごく少量の水分を加えて弱火で蒸し煮することだが、野菜については、バターだけを使う場合も多い。étouffer
  エトゥフェとほぼ同じ意味で用いられることも多い。}。

\protect\hyperlink{veloute-de-poisson}{魚のヴルテ}\troisquarts{}
Lを加え、しばらく弱火にかけて浮いてくる不純物を丁寧に取り除く\footnote{dépouiller
  デプイエ ≒ écumer エキュメ。}。生クリーム大さじ3杯とバター50gを加えて仕上げる。

\maeaki

\hypertarget{ux30bdux30fcux30b9ux30abux30ceux30c6ux30a3ux30a8ux30fcux30eb20}{%
\subsubsection[ソース・カノティエール]{\texorpdfstring{ソース・カノティエール\footnote{小舟の漕ぎ手、の意。}}{ソース・カノティエール}}\label{ux30bdux30fcux30b9ux30abux30ceux30c6ux30a3ux30a8ux30fcux30eb20}}

\hypertarget{sauce-canotiuxe8re}{%
\paragraph{Sauce Canotière}\label{sauce-canotiuxe8re}}

\index{そーす@ソース!かのてぃえーる@---・カノティエール}
\index{かのてぃえーる@カノティエール!そーす@ソース・---}
\index{sauce@sauce!canotiere@--- Canotière}
\index{canotiere@Canotière!sauce@Sauce ---}

淡水魚を煮るのに用いた、\href{}{白ワイン入りクールブイヨン}を\untiers{}量に
煮詰める。クールブイヨンにはしっかり香り付けしてあり塩はごく少量しか入っ
ていないこと。

1 Lあたり80 gのブールマニエを加えてとろみを付ける。軽く煮立たせたら、
鍋を火から外してバター150 gとカイエンヌごく少量を加えて仕上げる。

\ldots{}\ldots{}淡水魚のクールブイヨン煮用。

\hypertarget{ux539fux6ce8-2}{%
\subparagraph{【原注】}\label{ux539fux6ce8-2}}

バターでグラセした小玉ねぎと小ぶりのマッシュルームを加えると、「\protect\hyperlink{sauce-matelote-blanche}{白いソー
ス・マトロット}」の代用となる。

\maeaki

\hypertarget{ux30b1ux30a4ux30d1ux30fcux5165ux308aux30bdux30fcux30b9}{%
\subsubsection{ケイパー入りソース}\label{ux30b1ux30a4ux30d1ux30fcux5165ux308aux30bdux30fcux30b9}}

\hypertarget{sauce-aux-cuxe2pres}{%
\paragraph{Sauce aux Câpres}\label{sauce-aux-cuxe2pres}}

\index{そーす@ソース!けいぱー@ケイパー---}
\index{けいぱー@ケイパー!そーす@---ソース}
\index{sauce@sauce!capres@--- aux Câpres}
\index{capre@câpre!sauce capres@Sauce aux Câpres}

上記の\protect\hyperlink{sauce-au-beurre}{ソース・オ・ブール}に、ソース1
Lあたり大さじ4 杯のケイパーを提供直前に加える。

\ldots{}\ldots{}いろいろな種類の魚を煮た料理に用いる。

\maeaki

\hypertarget{ux30bdux30fcux30b9ux30abux30ebux30c7ux30a3ux30caux30eb21}{%
\subsubsection[ソース・カルディナル]{\texorpdfstring{ソース・カルディナル\footnote{カトリックの枢機卿(カルディナル)の衣が伝統的に赤いものである
  ことと、オマールが「海の枢機卿」と呼ばれることに由来。}}{ソース・カルディナル}}\label{ux30bdux30fcux30b9ux30abux30ebux30c7ux30a3ux30caux30eb21}}

\hypertarget{sauce-cardinal}{%
\paragraph{Sauce Cardinal}\label{sauce-cardinal}}

\index{そーす@ソース!かるでぃなる@---・カルディナル}
\index{かるでぃなる@カルディナル!そーす@ソース・---}
\index{sauce@sauce!cardinal@--- Cardinal}
\index{cardinal@cardinal!sauce@Sauce ---}

\protect\hyperlink{sauce-bechamel}{ベシャメルソース}\troisquarts{}
Lに、(1)\protect\hyperlink{fumet-de-poisson}{魚のフュ
メ}とトリュフエッセンスを同量ずつ合わせて
\troisquarts{}量まで煮詰めたものを1\undemi{} dl加える。(2)生クリーム
1\undemi{} dlを加える。

鍋を火から外し、真っ赤に作った\protect\hyperlink{beurre-de-homard}{オマールバター}を加え、カ
イエンヌごく少量で風味を引き締める。

\ldots{}\ldots{}魚料理用。

\maeaki

\hypertarget{ux30deux30c3ux30b7ux30e5ux30ebux30fcux30e0ux5165ux308aux30bdux30fcux30b9}{%
\subsubsection{マッシュルーム入りソース}\label{ux30deux30c3ux30b7ux30e5ux30ebux30fcux30e0ux5165ux308aux30bdux30fcux30b9}}

\hypertarget{sauce-aux-champignons}{%
\paragraph{Sauce aux Champignons}\label{sauce-aux-champignons}}

\index{そーす@ソース!まっしゅるーむしろ@マッシュルーム---(ホワイト系)}
\index{まっしゅるーむ@マッシュルーム!そーすしろ@---ソース(ホワイト系)}
\index{sauce@sauce!champignonsblanche@--- aux Champignons (blanches)}
\index{champignon@champignon!sauce blanche@Sauce aux Champignons (blanche)}

マッシュルームの煮汁3
dlを\untiers{}量まで煮詰める。\protect\hyperlink{sauce-allemande}{パリ風ソース(ソー
ス・アルマンド)}\troisquarts{} Lを加え、数分間沸騰
させる。あらかじめトゥルネして茹でておいた真っ白で小さなマッシュルーム
100 gを加えて仕上げる。

\ldots{}\ldots{}鶏料理用。魚料理に添えることもある。魚料理に合わせる場合は、パリ風
ソース(ソース・アルマンド)ではなく\protect\hyperlink{veloute-de-poisson}{魚料理用ヴル
テ}を用いること。

\maeaki

\hypertarget{ux30bdux30fcux30b9ux30b7ux30e3ux30f3ux30c6ux30a3ux30a422}{%
\subsubsection[ソース・シャンティイ]{\texorpdfstring{ソース・シャンティイ\footnote{生クリームをホイップしたクレーム・シャンティイが有名だが、パリ
  北方に位置する町の名。17世紀、コンデ公ルイ2世(大コンデとも呼ばれ
  る)の館があり、ヴァテルがメートルドテルとして仕えていた。その館で
  ルイ14世をはじめとして約千名もの賓客を招いて開かれた数日にわたる宴
  会の際に、食材の魚が少ししか届かないと誤解したヴァテルは責任をとる
  ために自殺したと言われている。なお、魚はその後すぐに大量に館に届け
  られたという。}}{ソース・シャンティイ}}\label{ux30bdux30fcux30b9ux30b7ux30e3ux30f3ux30c6ux30a3ux30a422}}

\hypertarget{sauce-chantilly-chaude}{%
\paragraph{Sauce Chantilly}\label{sauce-chantilly-chaude}}

\index{そーす@ソース!しやんていい@---・シャンティイ}
\index{しやんていい@シャンティイ!そーす@ソース・---}
\index{sauce@sauce!chantilly@--- Chantilly}
\index{Chantilly@Chantilly!sauce@Sauce ---}

まれに「ソース・シャンティイ」の名で呼ばれることもあるが、これは後述の
「\protect\hyperlink{sauce-mousseline}{ソース・ムスリーヌ}」に他ならない。
\end{recette}