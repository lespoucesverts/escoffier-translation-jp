\href{未、原文対照チェック}{} \href{未、日本語表現校正}{}
\href{未、その他修正}{} \href{未、原稿最終校正}{}

\hypertarget{petites-sauces-blanches}{%
\section{ホワイト系の派生ソース}\label{petites-sauces-blanches}}

\frsec{Petites Sauces Blanches, Composées et de Réductions}

\index{そーす@ソース!ほわいとはせい@ホワイト系の派生---}
\index{sauce@sauce!petites blanches composees@Petites ---s Blanches Composées}
\begin{recette}
\hypertarget{sauce-albufera}{%
\subsubsection[ソース・アルビュフェラ]{\texorpdfstring{ソース・アルビュフェラ\footnote{ナポレオン軍の元帥、ルイ・ガブリエル・スーシェ
  Louis-Gabriel Suchet, duc d'Albufera
  (1770〜1826)のこと。スペイン戦役の際にそれまでの軍功を称えられ、ナポレオンが1812年にアルビュフェラ公爵位を新設して授けた。帝政期の英雄のひとりであり、アルビュフェラおよびスーシェの名を冠した料理がいくつかある。1814年に帝政が崩壊した後も軍務、政務に携わり、最終的にフランス貴族院議員の地位を得た。アルビュフェラ公爵位については、1815年7月24日の勅令においてに正式に抹消されている。このソースの特徴は赤ピーマン(パプリカ)を加熱してなめらかにすり潰し、バターに練り込んだものを使う点にあるが、どのような経緯でこのソースに赤ピーマンを用いるようになったのかは不明。ただし、このソースを合わせる\protect\hyperlink{poularde-albufera}{「肥鶏 アルビュフェラ」}は詰め物(ファルス)に米を用いるが、アルビュフェラは湖の周辺の湿地帯で米の生産がおこなわれているという点では一応の関連性が認められよう。なお、アルビュフェラはバレンシアの湖とそこに形成された潟であり、現在はバレンシア州のアルブフェーラ自然公園となっている。}}{ソース・アルビュフェラ}}\label{sauce-albufera}}

\frsub{Sauce Albuféra}

\index{そーす@ソース!あるひゆふえら@---・アルビュフェラ}
\index{あるひゆふえら@アルビュフェラ!そーす@ソース・---}
\index{sauce@sauce!albufera@--- Albuféra}
\index{albufera@Albuféra!sauce@Sauce ---}

\protect\hyperlink{sauce-supreme}{ソース・シュプレーム}1
Lあたりに、溶かしたブロンド色の\protect\hyperlink{glace-de-viande}{グラスドヴィアンド}2
dLと、標準的な分量比率で作った\protect\hyperlink{beurre-de-pimentos}{赤ピーマンバター}50
gを加える。

\ldots{}\ldots{}鶏など家禽のポシェ\footnote{ポシェは通常、沸騰させない程度の温度で茹でること、だが、ここで想定しているのは丸鶏をポシェしたもの。つまりは「仕立て」であり、前の注で触れた\protect\hyperlink{poularde-albufera}{「肥鶏 アルビュフェラ」}がこれに相当する。「仕立て」としての鶏のポシェは通常、中抜きした部分に詰め物(ファルス)をして手羽を脚を畳むようにしてまとめて糸で縫い(brider
  ブリデ)、さらに豚背脂のシートで包んでちょうどいい大きさの鍋に入れて、あらかじめ用意しておいた\protect\hyperlink{fonds-blanc}{白いフォン})
  が鶏にかぶる程度まで注ぐ。鍋を火にかけていったん沸騰したら、火を弱めるかオーブンに入れて、蓋をしてポシェの温度すなわち微沸騰を保つようにして加熱する。詳しくは\protect\hyperlink{les-poches}{第7章肉料理「ポシェ」の項}参照。}またはブレゼ\footnote{ポシェと同様に丸鶏をブレゼという「仕立て」に調理したものを想定しているので注意。肉料理の「仕立て」としてのブレゼについては\protect\hyperlink{les-braisages-de-viandes-blanches}{第7章肉料理「ブレゼ」の白身肉のブレゼ}参照。}にソースとして添える。

\hypertarget{sauce-americaine}{%
\subsubsection[ソース・アメリケーヌ]{\texorpdfstring{ソース・アメリケーヌ\footnote{オマール・アメリケーヌという料理の由来は諸説あるが、19世紀フランスの料理人ピエール・フレス
  Pierre Fraysse
  がアメリカで働いた後にパリで1853年に開いたレストラン「シェ・ピーターズ」でこの料理名で提供したというのが定説。ただし、1853年以前にレストラン「ボヌフォワ」に「ラングドック産オマール ソース・アメリケーヌ添え」というメニューあり、フレスはその料理に改変を加えたか、名前だけをシンプルに「アメリケーヌ」とした程度という説もある。かつては、オマールの主産地のひとつブルターニュ地方を意味する古い形容詞
  armoricain(e)
  アルモリカン、アルモリケーヌの音が変化した料理名だと主張されることもあったが、
  19世紀には南仏産が中心であったトマトを用いる点で矛盾が生じてしまう。いずれにしても、この料理名がフレスの店シェ・ピーターズを基点として広く知られるようになったことは事実。1867年のグフェ『料理の本』にはソース・エスパニョルをベースに白ワインとトマトで作るオマール・アメリケーヌのレシピが掲載されているので、比較的短期間で広まった料理なのは確か。また、オマールではないが、その20年程前に遡ってカレーム『19世紀フランス料理』には、「海亀のポタージュ アメリカ風」Potage
  de tortue à
  l'américaineおよびその派生型「海亀のソース アメリカ風」のレシピが掲載されている。レシピに先立って、カレームは「アメリカでも海亀のポタージュは本書のイギリス風海亀のポタージュと同様に、つまりロンドン風に調理するという。ところが、ボストンとニューヨークで暮したことのある人々から、アメリカ人は海亀のポタージュにうなぎのフィレを加えることを伝え聞いた。当然ながらイギリス風の海亀のポタージュとは異なる味わいのものとなる(t.1,
  pp.289-290」と述べている。ここではソースのほうの概要を見ておこう。皮を剥いた小さめのうなぎを筒切りにする。これをラグー鍋に入れてシャンパーニュを注ぐ。洗ったアンチョビのフィレとにんんく、玉ねぎ、薄切りにしたマッシュルーム、タイム、バジル、ローリエの葉、ローズマリー、マジョラム、サリエット、メース少々、粗く砕いたこしょう、カイエンヌ少々を加える。弱火にかけて煮込み、煮詰めていく。これを布で絞り漉す。ここにコンソメとソース・エスパニョルを加え、再度火にかけていい具合になるまで煮詰める。シャンパーニュをグラス\(\frac{1}{2}\)杯加えて布で漉す。提供直前にバター少々と鶏のグラス、レモン果汁を加える(t.3,
  pp.81-82)、というもの。}}{ソース・アメリケーヌ}}\label{sauce-americaine}}

\frsub{Sauce Américaine}

\index{そーす@ソース!あめりけーぬ@---・アメリケーヌ}
\index{あめりかん@アメリカン/アメリケーヌ!そーす@ソース・アメリケーヌ}
\index{sauce@sauce!americaine@--- américaine}
\index{americain@américain(e)!sauce americaine@sauce ---e}

このソースは\protect\hyperlink{homard-americaine}{オマール・アメリケーヌ}という料理そのものと言っていい(「魚料理」の章、甲殻類、\protect\hyperlink{homard-americaine}{オマール・アメリケーヌ}参照)。

このソースは通常、オマール\footnote{homard
  ロブスターのこと。なお高級料理では800〜900
  g程度の大きなものが好んで使用される。。}の身をガルニチュールとした魚料理に添えられる。オマールの身をやや斜めになるよう厚さ1
cm程度の輪切りにし\footnote{escalopper
  (エスカロペ)。エスカロップ、すなわち厚さ1〜2
  cm程度の円形に切ることだが、オマールの場合はやや斜めに切るようにして面積を大きくすることが一般的。ここで使用するオマールは900
  g〜1 kg程度のものを想定していることに注意。}、魚料理のガルニチュールとして供するわけだ。

\hypertarget{sauce-anchois}{%
\subsubsection{アンチョビソース}\label{sauce-anchois}}

\frsub{Sauce Anchois}

\index{そーす@ソース!あんちよひ@アンチョビ---}
\index{あんちよひ@アンチョビ!そーす@---ソース}
\index{sauce@sauce!anchois@--- Anchois}
\index{anchois@anchois!sauce anchois@Sauce ---}

\href{}{ノルマンディー風ソース}8
dLを、バターを加える前の段階まで作る。\href{}{アンチョビバター}125
gを混ぜ込む。アンチョビのフィレ50
gを洗い、よく水気を絞ってから小さなさいの目に切ったのを加えて仕上げる。

\ldots{}\ldots{}魚料理用。

\hypertarget{sauce-aurore}{%
\subsubsection[ソース・オーロール]{\texorpdfstring{ソース・オーロール\footnote{夜明けの光、曙光のこと。オーロラの意味もあるため、日本では「オーロラソース」と呼ばれることもあるが、マヨネーズとトマトケチャップを同量で混ぜ合わせたものもそう呼ばれることが多いので注意。なお、
  Sauce à l'aurore
  というほぼ同じ名称のものが1806年刊ヴィエアール『帝国料理の本』に掲載されているが、これはヴルテにレモン果汁とこしょう、ナツメグを加えたものを用意し、別に茹で卵の黄身を用意する。茹で卵の黄身を漉し器に圧し付けるようにして麺状に引き出す。提供直前に、ソースにこの黄身の麺を加える。ここからは決して沸騰させないこと、というもの(p.59)。麺状にした卵黄を朝の光の筋に見立てたもので、鍋で加えるか、ソース入れにソースを入れた上に載せるなどの方法も考えられるが、いずれにしてもヴィアールの時代(19世紀初頭)はフランス式サーヴィスつまり大きな食卓に何種もの料理を一度に並べるという方式だったために、このソースの見た目の美しさをある程度じっくりと食べ手は楽しむことが出来ただろう。その後の文献ではオドもカレームもこの名称のソースには触れておらず、デュボワとベルナールの『古典料理』(1867年)において、Sauce
  à
  l'Auroreとして、ベシャメルソースに煮詰めた仔牛のブロンドとマッシュルームの茹で汁、トマトソースを添加して、スライスしたマッシュルームを加えるというレシピが掲載されている(p.57)。初期のロシア式サービスにおいては、客に料理を最初に見せてまわり、その後に切り分けて供するという方式であったために、おそらくヴィアールの「ソース・アローロール」では一瞬で失なわれてしまったであろう美しさのポイントが、このようにソース色合いそのものに代えたことで、最後の食べ手の分を取り分けるまで美しさを維持できるようになった、つまりは初期のロシア式サービスの欠点を補うものとなったと考えられよう。なお、19世紀はトマトが食材として急激に普及、流行した時代であったこともこのソースの変化と関係があると思われる。}}{ソース・オーロール}}\label{sauce-aurore}}

\frsub{Sauce Aurore}

\index{そーす@ソース!おーろーる@---・オーロール}
\index{おーろーる@オーロール!そーす@ソース・---}
\index{sauce@sauce!aurore@--- Aurore}
\index{aurore@aurore!sauce@Sauce ---}

\protect\hyperlink{veloute}{ヴルテ}に真っ赤なトマトピュレを加えたもの。分量は、ヴルテが
\(\frac{3}{4}\)に対し、トマトピュレ
\(\frac{1}{4}\)とする。仕上げに、ソース1 Lあたり100 gのバターを加える。

\ldots{}\ldots{}卵料理、仔牛、仔羊肉の料理、鶏料理用。

\hypertarget{sauce-aurore-maigre}{%
\subsubsection{魚料理用ソース・オーロール}\label{sauce-aurore-maigre}}

\frsub{Sauce Aurore maigre}

\index{そーす@ソース!おーろーるさかなよう@魚料理用---・オーロール}
\index{おーろーる@オーロール!そーすさかな@魚料理用ソース・---}
\index{sauce@sauce!aurore maigre@--- Aurore maigre}
\index{aurore@aurore!sauce maigre@Sauce --- maigre}

\protect\hyperlink{veloute-de-poisson}{魚料理用ヴルテ}に、上記と同じ割合でトマトピュレを加える。ソース1
Lあたりバター125 gを加えて仕上げる。

\ldots{}\ldots{}魚料理用

\hypertarget{sauce-bavaroise}{%
\subsubsection{バイエルン風ソース}\label{sauce-bavaroise}}

\frsub{Sauce Bavaroise}

\index{そーす@ソース!はいえるんふう@バイエルン風---}
\index{はいえるんふう@バイエルン風!そーす@---ソース}
\index{sauce@sauce!bavaroise@--- Bavaroise}
\index{bavarois@bavarois(e)!sauce bavaroise@Sauce ---(e)}

ヴィネガー5
dLにタイムとローリエの葉少々とパセリの枝4本、大粒のこしょう7〜8個と、おろした\footnote{原文
  râpé \textless{} râpe
  ラープと呼ばれる器具を用いておろすが、日本のおろし金と目の大きさが違うので注意。多くの場合、マンドリーヌ
  mandrine と呼ばれる野菜用スライサーにこの機能が付属している。}レフォール\footnote{raifort
  西洋わさび、ホースラディッシュ。}大さじ2杯を加え、半量になるまで煮詰める。

この煮詰めた汁に卵黄6個を加え\footnote{卵黄を加える前に一度漉しておいたほうがいいだろう。}、\protect\hyperlink{sauce-hollandaise}{オランデーズソース}を作る要領で、バター400
gと大さじ1
\(\frac{1}{2}\)杯の水を少しずつ加えながら、ソースがしっかり乳化するまで混ぜていく。布で漉す。

\protect\hyperlink{beurre-d-ecrevisse}{エクルヴィスバター}100
gと泡立てた生クリーム大さじ2杯、さいの目に切ったエクルヴィス\footnote{ざりがにのこと。通常はヨーロッパザリガニécrevisse
  à pattes
  rougesエクルヴィスアパットルージュを指す。高級食材としてとても好まれている。現在は代用としてécrevisse
  de
  Californieエクルヴィスドカリフォルニ(ウチダザリガニ)が用いられることもある。日本在来のニホンザリガニや、外来種だが多く生息しているアメリカザリガニは通常、フランス料理には用いられない。いずれもジストマ(寄生虫)のリスクがあるため、生食は厳禁。}の尾の身を加えて仕上げる。

\ldots{}\ldots{}魚料理用のこのソースは、ムースのような仕上がりにすること。

\hypertarget{sauce-bearnaise}{%
\subsubsection[ソース・ベアルネーズ]{\texorpdfstring{ソース・ベアルネーズ\footnote{ベアルヌは旧地方名で、フランス南西部、現在のピレネー・アトランティック県のことを指すが、このソースはその地方とまったく関係がない。
  19世紀パリ郊外のレストランPavillon Henri
  IV(日本語に訳すと「アンリ4世亭」となろうか)が店名に掲げているアンリ4世がベアルヌのポー生まれであることにちなんで命名したソース名というのが定説。}}{ソース・ベアルネーズ}}\label{sauce-bearnaise}}

\frsub{Sauce Béarnaise}

\index{そーす@ソース!へあるねーす@---・ベアルネーズ}
\index{へあるぬふう@ベアルヌ風!そーす@ソース・---}
\index{へあるねーす@ベアルネーズ!そーす@ソース・---}
\index{sauce@sauce!bearnaise@--- Béarnaise}
\index{bearnais@béarnais(e)!sauce bearnaise@Sauce ---e}

白ワイン2 dLとエストラゴンヴィネガー2
dLに、エシャロットのみじん切り大さじ4杯、枝のままの粗く刻んだエストラゴン20
g、セルフイユ10 g、粗挽きこしょう5
g、塩1つまみを加えて、\(\frac{1}{3}\)量になるまで煮詰める。

煮詰まったら、数分間放置して温度を下げる。ここに卵黄6個を加え、弱火にかけて、生のバター(あるいはあらかじめ溶かしておいてもいい)500
gを加えて軽くホイップしながらなめらかになるよう混ぜる。

卵黄に徐々に火が通っていくことでソースにとろみが付くので、絶対に弱火で作業をすること。

バターを混ぜ込んだら、布で漉して味を調える。カイエンヌごく少量を加えて風味を引き締める。仕上げに、刻んだエストラゴン大さじ杯とセルフイユ大さじ\(\frac{1}{2}\)杯を加える。

\ldots{}\ldots{}牛、羊肉のグリル用。

\hypertarget{nota-sauce-bearnaise}{%
\subparagraph{【原注】}\label{nota-sauce-bearnaise}}

このソースを熱々で提供しようとは考えないこと。このソースは要するにバターで作ったマヨネーズなのだ。ほの温い程度で充分であり、もし熱くし過ぎてしまうと、ソースが分離してしまう。

そうなってしまったら、冷水少々を加えて泡立て器でホイップして元のあるべき状態に戻してやること。

\hypertarget{sauce-bearnaise-tomatee}{%
\subsubsection[トマト入りソース・ベアルネーズ/ソース・ショロン]{\texorpdfstring{トマト入りソース・ベアルネーズ/ソース・ショロン\footnote{19世紀後半、パリで有名レストラン「ヴォワザン」の料理長を務めたアレクサンドル・ショロン
  Alexandre Choron (1837〜1924)。自ら考案し、命名したという。}}{トマト入りソース・ベアルネーズ/ソース・ショロン}}\label{sauce-bearnaise-tomatee}}

\frsub{Sauce Béarnaise tomatée, dite Sauce Choron}

\index{そーす@ソース!へあるねーすとまといり@トマト入り---・ベアルネーズ}
\index{へあるぬふう@ベアルヌ風!そーすとまといり@トマト入りソース・ベアルネーズ}
\index{そーす@ソース!しよろん@---・ショロン}
\index{しよろん@ショロン!そーす@ソース・---}
\index{sauce@sauce!bearnaise tomatee@--- Béarnaise tomatée}
\index{bearnais@b\'earnais!sauce bearnaise tomatee@Sauce Béarnaise tomatée}
\index{sauce@sauce!choron@--- Choron}
\index{choron@Choron!sauce@Sauce ---}

ソース・ベアルネーズを上記のとおりに作るが、最後にセルフイユとエストラゴンのみじん切りは加えない。充分固めに作っておき、ソースの
\(\frac{1}{4}\)量の、充分に煮詰めたトマトピュレを加える。ソースの濃度が丁度いい具合になるよう注意すること。

\ldots{}\ldots{}\protect\hyperlink{tournedos-choron}{トゥルヌド・ショロン}、および他のさまざまな料理に添える。

\hypertarget{sauce-bearnaise-a-la-glace-de-viande}{%
\subsubsection[グラスドヴィアンド入りソース・ベアルネーズ/ソース・フォイヨ/ソース・ヴァロワ]{\texorpdfstring{グラスドヴィアンド入りソース・ベアルネーズ/ソース・フォイヨ/ソース・ヴァロワ\footnote{ソース・フォイヨの名称は、19世紀〜20世紀初頭にパリにあったレストランおよびそのオーナーシェフの名によるもの。このソースを使った「仔牛の背肉・フォイヨ」がスペシャリテだったという。ソース・ヴァロワについては、ヴァロワ王家およびヴァロワ公爵であったルイ・フィリップ(7月王政期のフランス国王。在位1830〜1848)にちなんだ名称。前出のフォイヨはレストランを開く以前、ルイ・フィリップに仕えていた。}}{グラスドヴィアンド入りソース・ベアルネーズ/ソース・フォイヨ/ソース・ヴァロワ}}\label{sauce-bearnaise-a-la-glace-de-viande}}

\frsub{Sauce Béarnaise à la glace de viande, dite Foyot, ou Valois}

\index{へあるぬふう@ベアルヌ風!そーすくらすとういあんといり@グラスドヴィアンド入りソース・ベアルネーズ}
\index{そーす@ソース!へあるねーすくらすとういあんといり@---・ベアルネーズ(グラス・ド・ヴィアンド入り)}
\index{そーす@ソース!ふおいよ@---・フォイヨ}
\index{ふおいよ@フォイヨ!そーす@ソース・---}
\index{そーす@ソース!うあろわ@---・ヴァロワ}
\index{うあろわ@ヴァロワ!そーす@ソース・---}
\index{sauce@sauce!bearnaise a la glace de viande@--- Béarnaise à la glace de viande}
\index{bearnais@b\'earnais!sauce bearnaise a la glace de viande@Sauce Béarnaise à la glace de viande}
\index{sauce@sauce!foyot@--- Foyot} \index{foyot@Foyot!sauce@Sauce ---}
\index{sauce@sauce!valois@--- Valois}
\index{valois@Valois!sauce@Sauce ---}

標準的な\protect\hyperlink{sauce-bearnaise}{ソース・ベアルネーズ}を上記の分量で、固めに作る。溶かした\protect\hyperlink{glace-de-viande}{グラスドヴィアンド}1
dLを少しずつ加えて仕上げる。

\ldots{}\ldots{}牛、羊肉のグリル用。

\hypertarget{sauce-bercy}{%
\subsubsection[ソース・ベルシー]{\texorpdfstring{ソース・ベルシー\footnote{パリ東部、セーヌ川左岸にある地名。かつては荷揚げ港があり、19世紀には小さなレストランが多く店を構えていたという。}}{ソース・ベルシー}}\label{sauce-bercy}}

\frsub{Sauce Bercy}

\index{そーす@ソース!へるしー@---・ベルシー}
\index{へるしー@ベルシー!そーす@ソース・---}
\index{sauce@sauce!bercy@--- Bercy} \index{bercy@Bercy!sauce@Sauce ---}

細かくみじん切りにしたエシャロット大さじ2杯をバターでさっと色付かないよう炒める。白ワイン2
\(\frac{1}{2}\)
dLと\protect\hyperlink{fumet-de-poisson}{魚のフュメ}か、このソースを合わせる魚の茹で汁2
\(\frac{1}{2}\) dLを注ぐ。

\(\frac{2}{3}\)量弱まで煮詰めたら、\protect\hyperlink{veloute-de-poisson}{ヴルテ}\(\frac{3}{4}\)
Lを加える。ひと煮立ちさせてから、鍋を火から外し、バター100
gとパセリのみじん切り大さじ1杯を加えて仕上げる。

\hypertarget{sauce-au-beurre}{%
\subsubsection[ソース・オ・ブール
/ソース・バタルド]{\texorpdfstring{ソース・オ・ブール
/ソース・バタルド\footnote{バタルドは「雑種の、中間の」の意。卵黄とバターだけでとろみを付ける\protect\hyperlink{sauce-hollandaise}{ソース・オランデーズ}と似てはいるが小麦粉も使うことからこの名が付いたと言われている。なお、パンのバタール
  bâtard
  も同じ語だが、細いバゲットと太いドゥーリーヴルの「中間」の太さとだからというのが通説。}}{ソース・オ・ブール /ソース・バタルド}}\label{sauce-au-beurre}}

\frsub{Sauce au Beurre, dite Sauce Bâtarde}

\index{そーす@ソース!ふーる@---・オ・ブール}
\index{はたー@バター!そーす@ソース・オ・ブール}
\index{そーす@ソース!はたると@---・バタルド}
\index{はたると@バタルド!そーす@ソース・---}
\index{sauce@sauce!beurre@--- au Beurre}
\index{beurre@beurre!sauce@Sauce au Beurre}
\index{sauce@sauce!batarde@--- Bâtarde}
\index{batard@bâtard!sauce@Sauce Bâtarde}

小麦粉45 gと溶かしバター45 gをよく混ぜ合わせ粘土状にする。そこに、7
gの塩を加えた熱湯7 \(\frac{1}{2}\)
dLを一気に注ぎ、泡立て器で勢いよく混ぜ合わせる。とろみ付け用の卵黄5個を生クリーム大さじ1
\(\frac{1}{2}\)杯でゆるめたものと、レモン汁少々を加える。

布で漉し、鍋を火から外して、良質なバター300 gを加えて仕上げる。

\ldots{}\ldots{}アスパラガスや、さまざまな茹でた魚\footnote{\protect\hyperlink{sauce-hachee-maigre}{魚料理用ソース・アシェ}訳注参照。}

\hypertarget{nota-sauce-au-beurre}{%
\subparagraph{【原注】}\label{nota-sauce-au-beurre}}

このソースはとろみを付けた後、湯煎にかけておき、提供直前にバターを加えるようにするといい。\footnote{本書には、日本でもかつて有名だった、エシャロットのみじん切りを加えたヴィネガーを煮詰めてバターを溶かし込んだ魚料理用ソース「ソース・ブールブラン」Sauce
  (au) Beurre blanc
  は収録されていない。このソース・ブールブランはナント地方やアンジュー地方で淡水魚アローズやブロシェに合わせる伝統的なソース。1890年頃にナント地方の女性料理人クレマンス・ルフーヴルが、ソース・ベアルネーズを作るつもりが誤って卵を加えるのを忘れてしまった結果として出来たものだとも言われている。}

\hypertarget{sauce-bonnefoy}{%
\subsubsection[ソース・ボヌフォワ/白ワインで作るボルドー風ソース]{\texorpdfstring{ソース・ボヌフォワ/白ワインで作るボルドー風ソース\footnote{ソース・ボヌフォワの名称は、19世紀中頃にあったレストランの名による。このレストランで考案されたソースだという説もある。}}{ソース・ボヌフォワ/白ワインで作るボルドー風ソース}}\label{sauce-bonnefoy}}

\frsub{Sauce Bonnefoy, ou Sauce Bordelaise au vin blanc}

\index{ほぬふおわ@ボヌフォワ!そーす@ソース・---}
\index{そーす@ソース!ほぬふおわ@---・ボヌフォワ}
\index{そーす@ソース!ほるとーふうしろわいん@ボルドー風--- (白)}
\index{ほるとーふう@ボルドー風!そーすしろ@---ソース(白)}
\index{sauce@sauce!bonnefoy@--- Bonnefoy}
\index{bonnefoy@Bonnefoy!sauce@Sauce ---}
\index{sauce@sauce!bordelaise vin blanc@--- Bordelaise au vin blanc}
\index{bordelais@bordelais!sauce vin blanc@Sauce Bordelaise au vin blanc}

ブラウン系の派生ソースの節で採り上げた、赤ワインを用いて作る\protect\hyperlink{sauce-bordelaise}{ボルドー風ソース}とまったく同じ作り方だが、赤ワインではなく、グラーヴかソテルヌの白ワインを用いる。また\protect\hyperlink{sauce-espagnole}{ソース・エスパニョル}ではなく\protect\hyperlink{veloute}{標準的なヴルテ}を使うこと。

このソースは仕上げに、みじん切りにしたエストラゴンを加える。

\ldots{}\ldots{}魚のグリル、白身肉のグリル用。

\hypertarget{sauce-bretonne-blanche}{%
\subsubsection{ブルターニュ風ソース}\label{sauce-bretonne-blanche}}

\frsub{Sauce Bretonne}

\index{そーす@ソース!ふるたーにゆふうしろ@ブルターニュ風---(ホワイト系)}
\index{ふるたーにゆふう@ブルターニュ風!そーすしろ@---ソース(ホワイト系)}
\index{sauce@sauce!bretonne blanche@--- Bretonne (blanche)}
\index{breton@breton!sauce blanche@Sauce Bretonne (blanche)}

長さ3〜5 cm位の、ごく細い千切り\footnote{julienne (ジュリエーヌ)。}にしたポワローの白い部分30
gとセロリの白い部分30 g、玉ねぎ30 g、マッシュルーム30
gをバターで完全に火が通るまで鍋に蓋をして弱火で蒸し煮する\footnote{étuver
  エチュヴェ。本来は油脂とごく少量の水分を加えて弱火で蒸し煮することだが、野菜については、バターだけを使う場合も多い。ほぼ同様の加熱方法に
  étouffer (エトゥフェ)がある。後者の原義は「窒息させる」。}。

\protect\hyperlink{veloute-de-poisson}{魚のヴルテ}\(\frac{3}{4}\)
Lを加え、しばらく弱火にかけて浮いてくる不純物を丁寧に取り除く\footnote{dépouiller
  デプイエ ≒ écumer エキュメ。}。生クリーム大さじ3杯とバター50
gを加えて仕上げる。

\hypertarget{sauce-canotiere}{%
\subsubsection[ソース・カノティエール]{\texorpdfstring{ソース・カノティエール\footnote{小舟の漕ぎ手、の意。}}{ソース・カノティエール}}\label{sauce-canotiere}}

\frsub{Sauce Canotière}

\index{そーす@ソース!かのていえーる@---・カノティエール}
\index{かのていえーる@カノティエール!そーす@ソース・---}
\index{sauce@sauce!canotiere@--- Canotière}
\index{canotier@canotier(ère)!sauce@Sauce Canotière}

淡水魚を煮るのに用いた、\protect\hyperlink{court-bouillon-b}{白ワイン入りクールブイヨン}を
\(\frac{1}{3}\)量に煮詰める。クールブイヨンにはしっかり香り付けしてあり塩はごく少量しか入っていないこと。

1 Lあたり80
gのブールマニエを加えてとろみを付ける。軽く煮立たせたら、鍋を火から外してバター150
gとカイエンヌごく少量を加えて仕上げる。

\ldots{}\ldots{}淡水魚のクールブイヨン煮用。

\hypertarget{nota-sauce-canotiere}{%
\subparagraph{【原注】}\label{nota-sauce-canotiere}}

バターでグラセした小玉ねぎと小ぶりのマッシュルームを加えると、「\protect\hyperlink{sauce-matelote-blanche}{白いソース・マトロット}」の代用となる。

\hypertarget{sauce-aux-capres}{%
\subsubsection{ケイパー入りソース}\label{sauce-aux-capres}}

\frsub{Sauce aux Câpres}

\index{そーす@ソース!けいはー@ケイパー入り---}
\index{けいはー@ケイパー!そーす@---入りソース}
\index{sauce@sauce!capres@--- aux Câpres}
\index{capre@câpre!sauce capres@Sauce aux Câpres}

上記の\protect\hyperlink{sauce-au-beurre}{ソース・オ・ブール}に、ソース1
Lあたり大さじ4 杯のケイパーを提供直前に加える。

\ldots{}\ldots{}いろいろな種類の魚を煮た料理に用いる。

\hypertarget{sauce-cardinal}{%
\subsubsection[ソース・カルディナル]{\texorpdfstring{ソース・カルディナル\footnote{カトリックの\ruby{枢機卿}{すうききよう}(カルディナル)の衣が伝統的に赤いものであること、およびオマールが「海の枢機卿」と呼ばれることに由来。}}{ソース・カルディナル}}\label{sauce-cardinal}}

\frsub{Sauce Cardinal}

\index{そーす@ソース!かるていなる@---・カルディナル}
\index{かるていなる@カルディナル!そーす@ソース・---}
\index{sauce@sauce!cardinal@--- Cardinal}
\index{cardinal@cardinal!sauce@Sauce ---}

\protect\hyperlink{sauce-bechamel}{ベシャメルソース}\(\frac{3}{4}\)
Lに、(1)\protect\hyperlink{fumet-de-poisson}{魚のフュメ}とトリュフエッセンスを同量ずつ合わせて
\(\frac{3}{4}\)量まで煮詰めたものを1 \(\frac{1}{2}\)
dL加える。(2)生クリーム 1 \(\frac{1}{2}\) dLを加える。

鍋を火から外し、真っ赤に作った\protect\hyperlink{beurre-de-homard}{オマールバター}を加え、カイエンヌごく少量で風味を引き締める。

\ldots{}\ldots{}魚料理用。

\hypertarget{sauce-aux-champignons-blanche}{%
\subsubsection{マッシュルーム入りソース}\label{sauce-aux-champignons-blanche}}

\frsub{Sauce aux Champignons}

\index{そーす@ソース!まつしゆるーむしろ@マッシュルーム---(ホワイト系)}
\index{まつしゆるーむ@マッシュルーム!そーすしろ@---ソース(ホワイト系)}
\index{sauce@sauce!champignonsblanche@--- aux Champignons (blanches)}
\index{champignon@champignon!sauce blanche@Sauce aux Champignons (blanche)}

マッシュルームを茹でた汁3 dLを
\(\frac{1}{3}\)量まで煮詰める。\protect\hyperlink{sauce-allemande}{ソース・アルマンド}\(\frac{3}{4}\)
Lを加え、数分間沸騰させる。あらかじめ\ruby{螺旋}{らせん}状に刻みを入れて整形\footnote{tourner
  トゥルネ。原義は「回す」。包丁を動かさずに材料の方を回すようにして切る、刻み目を入れることがこの用語の由来。マッシュルームの場合はその際に大量の切りくず(具体的には重量で15〜20%)が発生するので、それをソースなどの風味付けに利用する。}してから茹でておいた真っ白で小さなマッシュルーム100
gを加えて仕上げる。

\ldots{}\ldots{}鶏料理用。魚料理に添えることもある。魚料理に合わせる場合は、ソース・アルマンドではなく\protect\hyperlink{veloute-de-poisson}{魚料理用ヴルテ}を用いること。

\hypertarget{sauce-chantilly}{%
\subsubsection[ソース・シャンティイ]{\texorpdfstring{ソース・シャンティイ\footnote{料理および製菓では生クリームをホイップしたクレーム・シャンティイが有名だが、元来はパリ北方に位置する町の名。ここのシャンティイ城で17世紀に、歴史上主要なメートルドテルのひとりヴァテルが自害した事件は有名で小説化、映画化もされた。}}{ソース・シャンティイ}}\label{sauce-chantilly}}

\frsub{Sauce Chantilly}

\index{そーす@ソース!しやんていい@---・シャンティイ}
\index{しやんていい@シャンティイ!そーす@ソース・---}
\index{sauce@sauce!chantilly@--- Chantilly}
\index{chantilly@Chantilly!sauce@Sauce ---}

まれにこれを「ソース・シャンティイ」と呼ばれることもあるが、これは後述の「\protect\hyperlink{sauce-mousseline}{ソース・ムスリーヌ}」に他ならない\footnote{むしろ、初版から掲載されている冷製ソースの\protect\hyperlink{sauce-chantilly-froide}{ソース・シャンティイ}と混同しないよう留意すべきだろう。}。

\hypertarget{sauce-chateaubriand}{%
\subsubsection[ソース・シャトーブリヤン]{\texorpdfstring{ソース・シャトーブリヤン\footnote{料理において通常、シャトーブリヤンは牛フィレの中心部分を3
  cm程度の厚さに切ったものを指す。この名称の由来には主に2説あり、ひとつはフランスロマン主義文学の父と言われる小説家フランソワ・ルネ・シャトーブリヤン
  François René Chateaubriand
  (1768〜1848)の名を冠したというもの。ちなみにフランスロマン主義文学の母と呼ばれているのはスタール夫人Anne
  Louise Germaine de
  Staël(1766〜1817)。料理におけるシャトーブリヤンという名の由来のもうひとつの説は、ブルターニュ地方で畜産物の集積地であったシャトーブリヤン
  Châteaubriant
  という地名に由来するというもの。なお、本書の初版および第四版では
  Chateaubriandの綴り、第二版はChâteaubriantであり、第三版は
  Châteaubrian\textbf{d}という奇妙な綴りとなっている。}}{ソース・シャトーブリヤン}}\label{sauce-chateaubriand}}

\frsub{Sauce Chateaubriand}

\index{そーす@ソース!しゃとーふりやん@---・シャトーブリヤン}
\index{しゃとーふりやん@シャトーブリヤン! そーす@ソース・---}
\index{sauce@sauce!chateaubriand@--- Chateaubriand}
\index{chateaubriand@Chateaubriand!sauce@Sauce ---}

(仕上がり5 dL分)

白ワイン4
dLに、みじん切りにしたエシャロット4個分とタイム少々、ローリエの葉少々、マッシュルームの切りくず40
gを加え、\(\frac{1}{3}\)量になるまで煮詰める。

\protect\hyperlink{jus-de-veau-brun}{仔牛のジュ}\footnote{本書では「仔牛の茶色いジュ」のレシピは掲載されているが、仔牛の「白い」ジュについての言及はない。ここでは通常の仔牛の茶色いジュを用いればいい。また、\protect\hyperlink{sauce-colbert}{ソース・コルベール}の項(第二版で加えられた)で、\protect\hyperlink{beurre-colbert}{ブール・コルベール}とこのソースを比較するにあたり、このソースを「軽く仕上げたグラスドヴィアンドにバターとパセリのみじん切りを加えたもの」と述べている(\protect\hyperlink{sauce-colbert}{ソース・コルベール}本文参照)。このため、なぜこのソース・シャトーブリヤンが「ブラウン系の派生ソース」の節ではなく「ホワイト系の派生ソース」に分類されているのか疑問が残るところ。}4
dLを加え、半量になるまで煮詰める。布で漉し、鍋を火から外して、メートルドテルバター250
gと細かく刻んだエストラゴン小さじ \(\frac{1}{2}\)杯を加えて仕上げる。

\ldots{}\ldots{}牛、羊の赤身肉のグリル用。

\hypertarget{sauce-chaud-froid-blanche-ordinaire}{%
\subsubsection{白いソース・ショフロワ(標準)}\label{sauce-chaud-froid-blanche-ordinaire}}

\frsub{Sauce Chaud-froid blanche ordinaire}

\index{そーす@ソース!しよふろわしろ@白い---・ショフロワ(標準)}
\index{しよふろわ@ショフロワ!そーすしろ@白いソース---(標準)}
\index{sauce@sauce!chaud-froid blanche ordinaire@--- Chaud-froid blanche ordinaire}
\index{chaud-froid@chaud-froid!sauce blanche ordinaire@Sauce --- blanche ordinaire}

(仕上がり1
L分)\ldots{}\ldots{}\protect\hyperlink{veloute}{標準的なヴルテ}\(\frac{3}{4}\)
L、\protect\hyperlink{gelee-de-volaille}{鶏でとった白いジュレ}6〜7
dL、生クリーム\footnote{フランスの生クリームについては\protect\hyperlink{sauce-supreme}{ソース・シュプレーム}訳注参照。}3
dL。

厚手のソテー鍋にヴルテを入れる。強火にかけ、ヘラで混ぜながらジュレと用意した生クリーム
\(\frac{1}{3}\)量を少しずつ加えていく。

所定の分量にするには、\(\frac{2}{3}\)量くらいまで煮詰めることになる。

味見をして、固さを確認する。これを布で漉す\footnote{粘度の高いソースなどを布で漉す方法については、\protect\hyperlink{veloute}{ヴルテ}訳注参照。}。生クリームの残りを少しずつ加え、ゆっくり混ぜながら、ショフロワに仕立てる食材を覆うのにいい固さになるまで冷ましてやる。

\hypertarget{sauce-chaud-froid-blonde}{%
\subsubsection{ブロンドのソース・ショフロワ}\label{sauce-chaud-froid-blonde}}

\frsub{Sauce Chaud-froid blonde}

\index{そーす@ソース!しよふろわふろんと@ブロンドの---・ショフロワ}
\index{しよふろわ@ショフロワ!そーすふろんと@ブロンドのソース---}
\index{sauce@sauce!chaud-froid blonde@--- Chaud-froid blonde}
\index{chaud-froid@chaud-froid!sauce blonde@Sauce --- blonde}

上記と同様に作るが、ヴルテではなく\protect\hyperlink{sauce-allemande}{ソース・アルマンド}を用いる。また、生クリームの量は半分に減らすこと。

\hypertarget{sauce-chaud-froid-aurore}{%
\subsubsection[ソース・ショフロワ・オーロール]{\texorpdfstring{ソース・ショフロワ・オーロール\footnote{夜明け、曙光の意。}}{ソース・ショフロワ・オーロール}}\label{sauce-chaud-froid-aurore}}

\frsub{Sauce Chaud-froid Aurore}

\index{そーす@ソース!しよふろわおーろーる@---・ショフロワ・オーロール}
\index{しよふろわ@ショフロワ!そーすおーろーる@ソース・---・オーロール}
\index{おーろーる@オーロール!そーすしよふろわおーろーる@ソース・ショフロワ・---}
\index{sauce@sauce!chaud-froid aurore@--- Chaud-froid Aurore}
\index{chaud-froid@chaud-froid!sauce aurore@Sauce --- Aurore}
\index{aurore@aurore!sauce chaud-froid aurore@Sauce Chaud-froid ---}

標準的な\protect\hyperlink{sauce-chaud-froid-blanche-ordinaire}{白いソース・ショフロワ}を上記のとおり作る。そこに、真っ赤なトマトピュレを布で漉したもの
1 \(\frac{1}{2}\) dLとパプリカ粉末0.25
gを少量のコンソメで煎じた\footnote{infuser
  アンフュゼ。煮出す、煎じる、の意。}ものを加える。

\ldots{}\ldots{}鶏のショフロワ用。

\hypertarget{nota-sauce-chaud-froid-aurore}{%
\subparagraph{【原注】}\label{nota-sauce-chaud-froid-aurore}}

あまり鮮かな色にしたくない場合は、パプリカを煎じた汁は数滴だけ加えるにとどめるといい。

\hypertarget{sauce-choud-froid-vert-pre}{%
\subsubsection[ソース・ショフロワ・ヴェールプレ]{\texorpdfstring{ソース・ショフロワ・ヴェールプレ\footnote{緑の野原、草原、の意。}}{ソース・ショフロワ・ヴェールプレ}}\label{sauce-choud-froid-vert-pre}}

\frsub{Sauce Chaud-froid au Vert-pré}

\index{そーす@ソース!しよふろわうえーるふれ@---・ショフロワ・ヴェールプレ}
\index{しよふろわ@ショフロワ!そーすうえーるふれ@ソース・---・ヴェールプレ}
\index{うえーるふれ@ヴェールプレ!そーすしよふろわうえーるふれ@ソース・ショフロワ・---}
\index{sauce@sauce!chaud-froid vert-pre@--- Chaud-froid au Vert-pré}
\index{chaud-froid@chaud-froid!sauce vert-pre@Sauce --- au Vert-pré}
\index{vert-pre@vert-pré!sauce chaud-froid vert-pre@Sauce Chaud-froid au ---}

鍋に白ワイン2
dLを沸かし、セルフイユとエストラゴン、刻んだシブレット、刻んだパセリの葉を各1つまみずつ投入する。蓋をして火から外し、10分間煎じてから布で漉す。

最初に示したとおりの分量で\protect\hyperlink{sauce-chaud-froid-blanche-ordinaire}{標準的なソース・ショフロワ}を作り、煮詰めながら、上記の香草を煎じた液体を少しずつ混ぜ込む。この段階で1
Lになるまで煮詰めておくこと。

\protect\hyperlink{}{ほうれんそうから採った緑の色素}をソースに加え、\ul{ほんのり薄い緑色}にする。

この色素を加える際にはよく注意して、上で示したとおりの色合いになるよう少しずつ投入すること。

このソースは各種の鶏\footnote{日本語では鶏と一言で済ませるが、フランス語では
  poussin プサン(ひよこ、ひな鶏)、poulette
  プレット(若い雌鶏)、poulet プレ(若鶏)、poule
  プール(雌鶏)、poulet de grain プレドグラン(50〜70日の若鶏)、poulet
  reine
  プレレーヌ(若鶏と肥鶏の中間のサイズでソテーやローストにする)、poulet
  quatre quarts プレカトルカール(45日程で食用にする)、poularde
  プラルド(肥鶏、1.8 kg以上のものが多く、
  AOCを取得している産地もある)、chapon シャポン(去勢鶏、最大で 6kg
  程になるというが、肉質は雌鶏に近く、高級品とされている)、coq
  コック(雄鶏)などに細かく分類されている。}のショフロワ、とりわけ「\protect\hyperlink{}{ショフロワ・プランタニエ}」に用いる。

\hypertarget{sauce-chaud-froid-maigre}{%
\subsubsection{魚料理用ソース・ショフロワ}\label{sauce-chaud-froid-maigre}}

\frsub{Sauce Chaud-froid maigre}

\index{そーす@ソース!さかなりようりようしよふろわ@魚料理用---・ショフロワ}
\index{しよふろわ@ショフロワ!さかなりようりようそーす@魚料理用ソース---}
\index{sauce@sauce!chaud-froid maigre@--- Chaud-froid maigre}
\index{chaud-froid@chaud-froid!sauce maigre@Sauce --- maigre}

作り方の手順と分量は\protect\hyperlink{sauce-chaud-froid-blanche-ordinaire}{標準的なソース・ショフロワ}とまったく同じだが、以下の点を変更する。(1)通常の\protect\hyperlink{veloute}{ヴルテ}ではなく\protect\hyperlink{veloute-de-poisson}{魚料理用ヴルテ}を用いる。(2)\protect\hyperlink{}{鶏のジュレ}ではなく\protect\hyperlink{gelee-de-poisson-blanche}{白い魚のジュレ}を用いること。

\hypertarget{nota-sauce-chaud-froid-maigre}{%
\subparagraph{【原注】}\label{nota-sauce-chaud-froid-maigre}}

一般的に、このソースは魚のフィレやエスカロップ、甲殻類に\protect\hyperlink{mayonnaise-collee}{コーティング用マヨネーズ}の代わりとして用いることをお勧めする。コーティング用マヨネーズにはいろいろ不都合な点があり、そのうちの最大のものは、ゼラチンが溶けるにつれて油が浸み出してきてしまうことだ。こういう不都合はこの魚料理用ソース・ショフロワを使う場合には出てこない。このソースは風味も明確ですっきりしているからコーティング用マヨネーズよりも好ましいだろう。

\hypertarget{sacue-chivry}{%
\subsubsection[ソース・シヴリ]{\texorpdfstring{ソース・シヴリ\footnote{19世紀フランスの作家フレデリック・スリエ
  Frédéric Soulié (1800〜
  1847)の劇『ディアーヌ・ド・シヴリ』\emph{Diane de Chivry}
  (1838年)あるいは1897年に新聞「フィガロ」に掲載されたエルネスト・カペンデュの小説『ビビタパン』の登場人物名Chivryにちなんだか、あるいはまったく別の人物の名を冠したものかは不明。}}{ソース・シヴリ}}\label{sacue-chivry}}

\frsub{Sauce Chivry}

\index{そーす@ソース!しうり@---・シヴリ}
\index{しうり@シヴリ!そーす@ソース・---}
\index{sauce@sauce!chivry@--- Chivry}
\index{chivry@Chivry!sauce@Sauce ---}

白ワイン1 \(\frac{1}{2}\) dLに以下を各1つまみずつ投入する\footnote{明記されていないが、この時点で白ワインは沸かしておく。}\ldots{}\ldots{}セルフイユ、パセリ、エストラゴン、シブレット、時季が合えばサラダバーネット
\footnote{pumprenelle パンプルネル、和名ワレモコウ。}の若い葉。蓋をして鍋を火から外し、10分間煎じる\footnote{infuser
  アンフュゼ。}。布で絞るようにして漉す。

こうしてハーブ類を煎じた液体を、あらかじめ沸かしておいた\protect\hyperlink{veloute}{ヴルテ}\(\frac{3}{4}\)
Lに加える。火から外し、\protect\hyperlink{beurre-chivry}{ブール・シヴリ}100
gを加えて仕上げる(\protect\hyperlink{beurres-composes}{合わせバターの節}参照)。

\ldots{}\ldots{}ポシェ\footnote{pocher
  原則的には、沸騰しない程度の温度で加熱調理すること。この場合は、下処理した鶏一羽まるごとをぎりぎり入るくらいの大きさの鍋に入れて水あるいはクールブイヨンを用いてゆっくり火を通す調理を意味している(温度管理が難しい場合はオーブンを用いることもある)。}あるいは茹でた鶏の料理用。

\hypertarget{nota-sauce-chivry}{%
\subparagraph{【原注】}\label{nota-sauce-chivry}}

サラダバーネットは生育するにつれて苦味が強くなるの、必ず若いものを使うこと。

\hypertarget{sauce-choron}{%
\subsubsection{ソース・ショロン}\label{sauce-choron}}

\frsub{Sauce Choron}

\protect\hyperlink{sauce-bearnaise-tomatee}{トマト入りソース・ベアルネーズ}参照。

\hypertarget{sauce-creme}{%
\subsubsection{ソース・クレーム}\label{sauce-creme}}

\frsub{Sauce à la Crème}

\index{そーす@ソース!くれーむ@---・クレーム}
\index{くりーむ@クリーム!そーす@ソース・クレーム}
\index{sauce@sauce!creme@--- à la Crème}
\index{creme@crème!sauce@Sauce à la ---}

\protect\hyperlink{sauce-bechamel}{ベシャメルソース}1 Lに生クリーム2
dLを加えて、ヘラで混ぜながら強火で、全体量の\(\frac{3}{4}\)になるまで煮詰める。

布で漉す\footnote{粘度や濃度の高いソースを漉す方法については\protect\hyperlink{veloute}{ヴルテ}訳注参照。}。フレッシュなクレーム・ドゥーブル\footnote{乳酸醗酵させた濃度の高い生クリーム。詳しくは\protect\hyperlink{sauce-supreme}{ソース・シュプレーム}訳注参照。}2
\(\frac{1}{2}\) dLとレモン果汁半個分を少しずつ加えて仕上げる。

\ldots{}\ldots{}茹でた魚、野菜料理、鶏、卵料理用。

\hypertarget{sauce-aux-crevettes}{%
\subsubsection[ソース・クルヴェット]{\texorpdfstring{ソース・クルヴェット\footnote{小海老のこと。フランスでよく料理に用いられるのは生の状態で甲殻が灰色がかった小さめのcrevettes
  grises(クルヴェット・グリーズ)と、やや大きめでピンク色のcrevettes
  roses(クルヴェット・ローズ)。美味しい。ちなみに日本でよく食べられているブラックタイガーはフランス語にするとcrevette
  géante tigréeと言う。}}{ソース・クルヴェット}}\label{sauce-aux-crevettes}}

\frsub{Sauce aux Crevettes}

\index{そーす@ソース!くるうえつと@---・クルヴェット}
\index{くるうえつと@クルヴェット!そーす@ソース・---}
\index{sauce@sauce!crevette@--- aux Crevettes}
\index{crevette@crevette!sauce@Sauce aux Crevettes}

\protect\hyperlink{veloute-de-poisson}{魚料理用ヴルテ}または\protect\hyperlink{sauce-bechamel}{ベシャメルソース}1
Lに、生クリーム1 \(\frac{1}{2}\)
dLと\protect\hyperlink{fumet-de-poisson}{魚のフュメ}1 \(\frac{1}{2}\)
dLを加える。

火にかけて9
dLになるまで煮詰める。鍋を火から外し、\protect\hyperlink{}{ブール・ルージュ}25
g(ソース全体に淡いピンクの色合いを付けるのが目的)を足した\protect\hyperlink{}{クルヴェットバター}100
gを加える。殻を剥いたクルヴェットの尾の身大さじ3杯を加え、カイエンヌ1つまみで風味を引き締めて仕上げる。

\ldots{}\ldots{}魚料理およびある種の卵料理用。

\hypertarget{sauce-currie}{%
\subsubsection{カレーソース}\label{sauce-currie}}

\frsub{Sauce Currie}

\index{そーす@ソース!かれー@カレー---}
\index{かれー@カレー!そーす@---ソース}
\index{sauce@sauce!currie@---  Currie}
\index{currie@currie!sauce@Sauce ---}

以下の材料をバターで軽く色付くまで炒める\ldots{}\ldots{}玉ねぎ250
g、セロリ100 g、パセリの根\footnote{パセリには根パセリpersil
  tubéreux(ペルスィチュベルー)といって根が肥大する品種系統もある。平葉で、葉の香りはフランスで一般的なモスカールドタイプ(葉の縮れるタイプ)とやや異なる。イタリアンパセリのように用いることが可能。}30
g、これらはすべてやや厚めにスライスする。タイム1枝とローリエの葉少々、メース少々を加える。小麦粉50
gとカレー粉\footnote{カレーは植民地インドの料理としてイギリスに伝わり、18世紀にはC\&B
  社によって混合スパイスであるカレー粉が開発された。フランスはあまりインドやその他のカレーの食文化と接することもなかったために、こんにちでも「珍しい料理」の範疇にとどまっている。とはいえ、19世紀にインドからアンティル諸島のうちの英領地域に連れて来られたインド人たちがカレーを伝え、それが広まってフランス領アンティーユにおいてコロンボ
  colomboというカレーのバリエーションが成立した。コロンボはこんにちのフランスでも(インドのカレーとは別のものとして)比較的よく知られたものとなっている(少なくともcurry,
  currieという語よりは一般的認知度が高いと言えるだろう)。}小さじ1
杯弱を振り入れる。小麦粉が色付かない程度に炒めて火を通したら、\protect\hyperlink{}{白いコンソメ}
\(\frac{3}{4}\)
Lを注ぐ。沸騰したら、弱火にして約45分煮る。軽く押し絞るように布で漉す。ソースを温めて、浮いてきた油脂は取り除き
\footnote{dégraisser (デグレセ)。}、湯煎にかけておく。

\ldots{}\ldots{}魚料理、甲殻類、鶏、さまざまな卵料理に合わせる。

\hypertarget{nota-sauce-currie}{%
\subparagraph{【原注】}\label{nota-sauce-currie}}

ココナツミルクをソースに加えることもある。その場合、白いコンソメの
\(\frac{1}{4}\)量をココナツミルクに代えること。

\hypertarget{sauce-currie-indienne}{%
\subsubsection{インド風カレーソース}\label{sauce-currie-indienne}}

\frsub{Sauce Currie à l'Indienne}

\index{そーす@ソース!いんとふうかれー@インドカレー---}
\index{かれー@カレー!そーすいんとふう@インド---ソース}
\index{sauce@sauce!currie indienne@---  Currie à l'Indienne}
\index{currie@currie!sauce indienne@Sauce --- à l'Indienne}

みじん切り\footnote{原文ciseler
  シズレ。鋭利な刃物でみじん切りにすること、スライスすること。原義は「ハサミで切る」。なお、日本語でみじん切りに相当する用語にはhacherアシェもある(hache斧から派生した語)。後者は野菜の他、肉類を細かく刻む際にも用いられる。ミートチョッパーをフランス語ではhachoirアショワールと呼ぶ。}にした玉ねぎ1個と、パセリ、タイム、ローリエ、メース、シナモン各少々のブーケガルニを、バターとともに弱火にかけて色付かないよう蒸し煮する。

カレー粉3 gを振り入れ、ココナツミルク \(\frac{1}{2}\) Lを注ぐ。ヴルテ
\(\frac{1}{2}\)
Lを加える(ソースを肉料理に合わせるか、魚料理に合わせるかで、ヴルテも標準的なものを使うか、魚料理用を使うか決めること)。弱火で15分程煮る。布で漉し、生クリーム1
dLとレモン果汁少々を加えて仕上げる。

\hypertarget{nota-sauce-currie-indienne}{%
\subparagraph{【原注】}\label{nota-sauce-currie-indienne}}

ここで示した量のココナツミルクは、生のココヤシの実700 gをおろして、 4
\(\frac{1}{2}\)
dLの温めた牛乳で溶いて作る。それを布で強く絞って漉してから使うこと。

ココナツミルクがない場合には、同量のアーモンドミルクを用いてもいい。

インドの料理人によるこのソースの作り方はさまざまで、基本だけが同じというものだ。

だが、本来のレシピがあったところで、使い物にはならないだろう。インドのカレーは我が国の大多数にとっては我慢ならぬものだろうから。ここで記した作り方は、ヨーロッパ人の味覚を勘案したものなので、本来のものよりいい筈だ。

\hypertarget{sauce-diplomate}{%
\subsubsection[ソース・ディプロマット]{\texorpdfstring{ソース・ディプロマット\footnote{外交官風、の意。繊細で豪華な仕立ての料理に付けられる名称。}}{ソース・ディプロマット}}\label{sauce-diplomate}}

\frsub{Sauce Diplomate}

\index{そーす@ソース!ていふろまつと@---・ディプロマット}
\index{ていふろまつと@ディプロマット!そーす@ソース・---}
\index{sauce@sauce!diplomate@--- Diplomate}
\index{diplomat@diplomat(e)!sauce@Sauce ---e }

\ruby{既}{すで}に仕上げでおいた\protect\hyperlink{sauce-normande}{ノルマンディ風ソース}1
Lに、\protect\hyperlink{beurre-de-homard}{オマールバター}75 gを加える。

さいの目に切ったオマールの尾の身大さじ2杯と同様にさいの目に切ったトリュフ大さじ1杯を加えて仕上げる。

\ldots{}\ldots{}大きな魚一尾まるごとの\footnote{relevé
  (ルルヴェ)。17世紀〜19世紀前半ににスタイルとして完成したフランス式サービスでは、豪華な装飾を施した飾り台(socleソークル)に載せられ、皿の周囲を飾るようにガルニチュールが配され(borduresボルデュール)、主役である大きな塊肉や魚まるごと1尾の料理にはしばしば飾り串(hâteletアトレ)が刺してある、きわめて壮麗な大皿料理が置かれた。なお『料理の手引き』ではこうした仕立てについては時代に
  \ruby{似}{そぐ}わないものとして、ごく簡潔にしか説明されていないが、初版、第二版に付属している献立表、および第三版以降独立して出版された『メニューの本』にはルルヴェの語はしばしば見られる。}料理用。

\hypertarget{sauce-ecossaise}{%
\subsubsection{スコットランド風ソース}\label{sauce-ecossaise}}

\frsub{Sauce Ecossaise}

\index{そーす@ソース!すこつとらんとふう@スコットランド風---}
\index{すこつとらんとふう@スコットランド風!そーす@---ソース}
\index{sauce@sauce!ecossaise@--- Ecossaise}
\index{ecossais@écossais(e)!sauce@Sauce ---e}

上記の分量どおりに作った\protect\hyperlink{sauce-creme}{ソース・クレーム}9
dLに以下を加えて作る。1〜2
mmの細さに千切りにしたにんじん、セロリ、さやいんげんをバターを加えて鍋に蓋をして弱火で蒸し煮し\footnote{étuver
  エチュヴェ。}、\protect\hyperlink{}{白いコンソメ}に完全に浸したものを1
dL。

\noindent\ldots{}\ldots{}卵料理、鶏料理に添える。

\hypertarget{sauce-estragon-blanche}{%
\subsubsection[ソース・エストラゴン]{\texorpdfstring{ソース・エストラゴン\footnote{ヨモギ科のハーブ。詳しくは茶色い派生ソースの\protect\hyperlink{sauce-chasseur}{ソース・シャスール}訳注参照。}}{ソース・エストラゴン}}\label{sauce-estragon-blanche}}

\frsub{Sauce Estragon}

\index{そーす@ソース!えすとらこんしろ@---・エストラゴン(ホワイト系)}
\index{えすとらこん@エストラゴン!そーすしろ@ソース・--- (ホワイト系)}
\index{sauce@sauce!estragon blanche@--- Estragon (blanche)}
\index{estragon@estragon!sauce blanche@Sauce --- (blanche)}

エストラゴンの枝30 gを粗く刻み\footnote{concasser コンカセ。}、強火で下茹でする\footnote{blanchir
  ブランシール。}。水気をしっかりときり、エストラゴンをスプーンですり潰し、あらかじめ用意しておいた\protect\hyperlink{veloute}{ヴルテ}を大さじ4杯加える。これを布で漉す。こうして作ったエストラゴンのピュレを\protect\hyperlink{veloute-de-volaille}{鶏のヴルテ}または\protect\hyperlink{veloute-de-poisson}{魚料理用ヴルテ}1
Lに混ぜ込む。どちらのヴルテを使うから、合わせる料理によって決めること。味を調え、みじん切りにしたエストラゴン大さじ
\(\frac{1}{2}\)杯を加えて仕上げる。

\ldots{}\ldots{}卵料理、鶏肉料理、魚料理に合わせる。

\hypertarget{sauce-aux-fines-herbes-blanche}{%
\subsubsection{香草ソース}\label{sauce-aux-fines-herbes-blanche}}

\frsub{Sauce aux Fines Herbes}

\index{そーす@ソース!こうそうしろ@香草---(ホワイト系)}
\index{こうそう@香草!そーすしろ@---ソース(ホワイト系)}
\index{はーぶ@ハーブ ⇒ 香草!こうそうそーすしろ@香草ソース(ホワイト系)}
\index{sauce@sauce!fines herbes blanche@--- aux Fines Herbes (blanche)}
\index{fines herbes@fines herbes!sauce blanche@Sauce aux --- (blanche)}

(仕上がり5 dL分)

あらかじめ2種のうちどちらかの方法(\protect\hyperlink{sauce-vin-blanc}{白ワインソース}参照)で作っておいた\protect\hyperlink{sauce-vin-blanc}{白ワインソース}
\(\frac{1}{2}\)
Lに、\protect\hyperlink{beurre-d-echalote}{エシャロットバター}40
gと、パセリ、セルフイユ、エストラゴンのみじん切りを大さじ1
\(\frac{1}{2}\)杯加える。

\ldots{}\ldots{}魚料理用。

\hypertarget{sauce-foyot}{%
\subsubsection{ソース・フォイヨ}\label{sauce-foyot}}

\frsub{Sauce Foyot}

⇒
\protect\hyperlink{sauce-bearnaise-a-la-glace-de-viande}{グラスドヴィアンド入りソース・ベアルネーズ}参照。

\hypertarget{sauce-groseilles}{%
\subsubsection[ソース・グロゼイユ]{\texorpdfstring{ソース・グロゼイユ\footnote{日本語で「すぐりの実」のことだが、こんにちでは「黒すぐり」の方が一般的かも知れない。黒すぐりはフランス語では
  cassis カシスと呼ばれる。一般的なグロゼイユにはフサスグリと呼ばれる
  groseille rouge グロゼイユ・ルージュ(赤すぐり)とgroseille
  blancheグロゼイユ・ブランシュ(白すぐり)の2種があり、どちらもブドウのように房なりする。上記とは別に、このソースで用いられるgroseille
  à
  maquereauグロゼイヤマクロー(maquereauは鯖の意。日本では英語経由のグーズベリーまたはグースベリーの名称でも呼ばれることが多い。単に西洋すぐりとも呼ぶ)という比較的大粒で薄く縞模様の入る種類もある。これは通常は緑色だが、まれに紫色になる変種もあるという。いずれもフランスでは料理や菓子作りによく用いられる。}}{ソース・グロゼイユ}}\label{sauce-groseilles}}

\frsub{Sauce Groseilles}

\index{そーす@ソース!くろせいゆ@---・グロゼイユ}
\index{くろせいゆ@グロゼイユ!そーす@ソース・---}
\index{sauce@sauce!groseilles@--- Groseilles}
\index{groseille@groseille!sauce@Sauce ---s}

緑色の濃いグーズベリー500 gを銅の片手鍋で下茹でする。

5分間煮立てたら、水気をきって、粉砂糖大さじ3杯と白ワイン大さじ2〜3杯を加えて、完全に火をとおす。布で漉す。

こうして出来たピュレに、\protect\hyperlink{sauce-au-beurre}{ソース・オ・ブール}5
dLを加え、よく混ぜる。

\ldots{}\ldots{}このソースはグリルあるいはイギリス風\footnote{à
  l'anglaise
  (アラングレーズ)。通常は塩適量を加えた湯でボイルすることを指す。}に茹でた鯖によく合う。とはいえ、他の魚料理にも合わせてもいい。

\hypertarget{nota-sauce-groseiles}{%
\subparagraph{【原注】}\label{nota-sauce-groseiles}}

このソースは緑色の房なりのグロゼイユ\footnote{一般的なフサスグリであれば白系統の「未熟果」を用いるということと解釈される。}でも作ることが可能。

\hypertarget{sauce-hollandaise}{%
\subsubsection[オランデーズソース]{\texorpdfstring{オランデーズソース\footnote{ニューヨーク発祥の朝食メニューとして知られるエッグ・ベネディクト\emph{Egg
  Benedict}に必ず用いられることで有名なうえ、一般的には「バターで作るマヨネーズ」のイメージが強いかも知れない。実際のところは、ラ・ヴァレーヌ『フランス料理の本』(1651年)において「アスパラガスの白いソース添え」Asperges
  à la sauce
  blancheというレシピにおいて、このオランデーズソースの原型ともいうべきものが示されている。アスパラガスは固めに塩茹でする。「新鮮なバター、卵黄、塩、ナツメグ、ヴィネガー少々をよくかき混ぜる。ソースが滑らかになったら、アスパラガスに添えて供する(p.238)」。簡潔な記述だが、これがオランデーズソースの原型であることは間違いないだろう。おそらくはラ・ヴァレーヌ以前から存在していた可能性も否定できない。なお植物油を用いたマヨネーズが文献上で確認されるのが18世紀以降で、19世紀初頭から爆発的に流行し、広まったもの。また、マヨネーズについては、現代ヨーロッパにおいても卵黄ではなく全卵を用いて作るほうが多数を占めている点が異なることに注意。なお、オランデーズとは「オランダ風」の意だが、なぜこの名称となったのかについては不明な点が多い。また、2007年版の『ラルース・ガストロノミック』では、オランデーズソースを作る際には温度に注意することと、よくメッキされた銅鍋かステンレス製の鍋を用いる必要があり、アルミ製の鍋だと緑色に変色する可能性があることに注意を促している
  (p.455)。}}{オランデーズソース}}\label{sauce-hollandaise}}

\frsub{Sauce Hollandaise}

\index{そーす@ソース!おらんてーす@オランデーズ---}
\index{おらんてーす@オランデーズ!そーす@---ソース}
\index{おらんたふう@オランダ風!そーす@オランデーズソース}
\index{sauce@sauce!hollandaise@--- Hollandaise}
\index{hollandais@hollandais(e)!sauce@Sauce ---e}

大さじ4杯の水とヴィネガー大さじ2杯に、粗挽きこしょう1つまみと肌理の細かい塩1つまみを加えて、\(\frac{1}{3}\)量まで煮詰める。この鍋を熱源のそばか、湯煎にかける。

大さじ5杯の水と卵黄5個を加える。生のまま、あるいは溶かしたバター500 g
を加えながらしっかりホイップする。ホイップしている途中で、水を大さじ3〜
4杯、少量ずつ足してやる。水を足すのは、軽やかな仕上がりにするため。

レモンの搾り汁少々と必要なら塩を足して味を調え、布で漉す。

湯煎にかけておくが、ソースが分離しないように、温度は微温くしておく。

\ldots{}\ldots{}魚料理、野菜料理用。

\hypertarget{nota-sauce-hollandaise}{%
\subparagraph{【原注】}\label{nota-sauce-hollandaise}}

ヴィネガーを煮詰めて使うのは、いつも最高品質のものが使えるとはかぎらないからで、水は
\(\frac{1}{3}\)量まで減らしたほうがいい。ただし、煮詰める作業を完全に省いてしまわないこと。

\hypertarget{sauce-homard}{%
\subsubsection{ソース・オマール}\label{sauce-homard}}

\frsub{Sauce Homard}

\index{そーす@ソース!おまーる@---・オマール}
\index{おまーる@オマール!そーす@ソース・---}
\index{sauce@sauce!homard@--- Homard}
\index{homard@homard!sauce@Sauce ---}

\protect\hyperlink{veloute-de-poisson}{魚料理用ヴルテ}\(\frac{3}{4}\)
Lに、生クリーム1 \(\frac{1}{2}\)
dLと\protect\hyperlink{beurre-de-homard}{オマールバター}80
g、\protect\hyperlink{beurre-colorant-rouge}{赤いバター}40
gを加えて仕上げる。

\ldots{}\ldots{}魚料理用。

\hypertarget{nota-sauce-homard}{%
\subparagraph{【原注】}\label{nota-sauce-homard}}

このソースを魚1尾まるごとの料理に添える場合には、さいの目に切ったオマールの尾の身を大さじ3杯加える。

\hypertarget{sauce-hongroise}{%
\subsubsection[ハンガリー風ソース]{\texorpdfstring{ハンガリー風ソース\footnote{原書でも用いられている語paprikaパプリカはハンガリー語。唐辛子、ピーマンの仲間であり、16世紀以降17世紀にヨーロッパ全土に広まり、その土地ごとの風土に合わせて品種が多様化した。パプリカはとりわけ辛味成分をほとんど含んでいないのが特徴。ただし、ハンガリーの食文化において大きな役割を果すようになったのは19世紀以降になってからと言われている。}}{ハンガリー風ソース}}\label{sauce-hongroise}}

\frsub{Sauce Hongroise}

\index{そーす@ソース!はんかりーふう@ハンガリー風---}
\index{はんかりーふう@ハンガリー風!そーす@---ソース}
\index{sauce@sauce!hongroise@---  Hongroise}
\index{hongrois@hongrois(e)!sauce@Sauce ---e}

大きめの玉ねぎ1個のみじん切りをバターで色付かないよう強火で炒める。塩1
つまみとパプリカ粉末1 gで味付けする。

このソースを添える料理に合わせて\protect\hyperlink{veloute}{標準的なヴルテ}あるいは\protect\hyperlink{veloute-de-poisson}{魚料理用ヴルテ}
1 Lを加え、数分間軽く煮立てる。

布で漉し、バター100 gを加えて仕上げる。

このソースは淡いピンク色に仕上げるべきであり、その色を出しているのがパプリカ粉末だけによるものだということに注意。

\ldots{}\ldots{}仔羊や仔牛のノワゼット\footnote{noisette
  ロースの中心部分を円筒形に切り出して調理したもの。}にとりわけよく合う。卵料理、鶏料理、魚料理にも。

\hypertarget{sauce-aux-huitres}{%
\subsubsection{牡蠣入りソース}\label{sauce-aux-huitres}}

\frsub{Sauce aux Huîtres}

\index{そーす@ソース!かきいり@牡蠣入り---}
\index{かき@牡蠣!そーす@牡蠣入りソース}
\index{sauce@sauce!huitres@---  aux Huîtres}
\index{huitre@huître!sauce@Sauce aux Huîtres}

後述の\protect\hyperlink{sauce-normande}{ノルマンディ風ソース}に、ポシェ\footnote{pocher
  \textless{} poche
  ポシュ(ポケット)、からの派生語。ポーチドエッグを作る際に、ポケット状になるところからこの用語が定着した。沸騰しない程度の温度で加熱調理すること。}して周囲をきれいにした牡蠣の身を加えたもの。

\hypertarget{sauceindienne}{%
\subsubsection{インド風ソース}\label{sauceindienne}}

\frsub{Sauce Indienne}

\index{そーす@ソース!いんとふう@インド風---}
\index{いんとふう@インド風!そーす@---ソース}
\index{sauce@sauce!indienne@---  Indienne}
\index{indien@indien(ne)!sauce@Sauce ---ne}

⇒ \protect\hyperlink{sauce-currie-indienne}{インド風カレーソース}参照。

\hypertarget{sauce-ivoire}{%
\subsubsection[ソース・イヴォワール]{\texorpdfstring{ソース・イヴォワール\footnote{象牙、の意。}}{ソース・イヴォワール}}\label{sauce-ivoire}}

\frsub{Sauce Ivoire}

\index{そーす@ソース!いうおわーる@---・イヴォワール}
\index{いうおわーる@イヴォワール!そーす@ソース・---}
\index{sauce@sauce!ivoire@--- Ivoire}
\index{ivoire@ivoire!sauce@Sauce ---}

\protect\hyperlink{sauce-supreme}{ソース・シュプレーム}1
Lに、ブロンド色の\protect\hyperlink{glace-de-viande}{グラスドヴィアンド}大さじ3杯を加え、象牙のようなくすんだ色合いにする。

\ldots{}\ldots{}低めの温度でしっとり仕上がるよう茹でた\footnote{pocher
  (ポシェ)。}鶏に添える。

\hypertarget{sauce-joinville}{%
\subsubsection[ソース・ジョワンヴィル]{\texorpdfstring{ソース・ジョワンヴィル\footnote{19世紀、7月王政期の国王ルイ・フィリップの第3子、フランソワ・ドルレアン・ジョワンヴィル海軍大将(1818〜1900)のこと。エクルヴィスとクルヴェットを用いた料理に彼の名が冠されたものがいくつかある。}}{ソース・ジョワンヴィル}}\label{sauce-joinville}}

\frsub{Sauce Joinville}

\index{しよわんういる@ジョワンヴィル!そーす@ソース・---}
\index{そーす@ソース!しよわんういる@---・ジョワンヴィル}
\index{sauce@sauce!joinville@--- Joinville}
\index{joinville@Joinville!sauce@Sauce ---}

\protect\hyperlink{sauce-normande}{ノルマンディ風ソース}1
Lを、仕上げる直前の段階まで作る\footnote{すなわち、布で漉すところまで。}。\protect\hyperlink{beurre-d-ecrevisse}{エクルヴィスバター}60
gと\protect\hyperlink{}{クルヴェットバター}60 gを加えて仕上げる。

このソースを添える魚料理にガルニチュールが既にある場合は、これ以上は何も加えない。

ガルニチュールを伴なわない大きな茹でた魚\footnote{魚の場合は、クールブイヨンを用いてやや低めの温度で煮たもの。\protect\hyperlink{sauce-hachee-maigre}{魚料理用ソース・アシェ}訳注参照。}に添える場合には、細さ1〜
2 mmの千切りにした真黒なトリュフを大さじ2杯加えること。

\hypertarget{nota-sauce-joinville}{%
\subparagraph{【原注】}\label{nota-sauce-joinville}}

同様のソースはいろいろあるが、最後の仕上げにエクルヴィスバターとクルヴェットバターを組み合わせて加える点がソース・ジョワンビルが他のものと違うポイント。

\hypertarget{sauce-laguipiere}{%
\subsubsection[ソース・ラギピエール]{\texorpdfstring{ソース・ラギピエール\footnote{18世紀末〜19世紀初頭にかけて活躍したフランスを代表する料理人の名(?〜1812)。はじめコンデ公に仕え、革命時にコンデ公の亡命にも随行したが、後にフランスに帰国し、ナポレオン\ruby{麾下}{きか}に入った。ナポレオン自身は食に無頓着であったが、直接的にはミュラ元帥のもとで料理長として活躍した。タレーランに仕えていたアントナン・カレームは2年程の期間であったが、ラギピエールとともに宴席の仕事に携わり、生涯を通して師と仰ぐ程に尊敬してやまなかった。当然だが料理においてカレームはラギピエールから大きく影響を受け、そのことを後年、数冊の自著で明記している。ラギピエール自身はミュラ元帥に従ってロシア戦線に赴き、その撤退の途中、極寒の地で凍死した。カレームは1828年刊『パリ風の料理』の冒頭2ページを「ラギピエールの想い出に」と題し、とても力強い文体でその死を悼んだ。}}{ソース・ラギピエール}}\label{sauce-laguipiere}}

\frsub{Sauce Laguipière}

\index{らきひえーる@ラギピエール!そーす@ソース・---}
\index{そーす@ソース!らきひえーる@---・ラギピエール}
\index{sauce@sauce!laguipiere@--- Laguipière}
\index{laguipiere@Laguipière!sauce@Sauce ---}

上述のとおりに作った\protect\hyperlink{sauce-au-beurre}{ソース・オ・ブール}
1
Lに、レモン1個の搾り汁と\protect\hyperlink{glace-de-poisson}{魚のグラス}またはそれと同等に煮詰めた\protect\hyperlink{fumet-de-poisson}{魚のフュメ}大さじ4杯を加える。

このソースは茹でた魚に添える。

\hypertarget{nota-sauce-laguipiere}{%
\subparagraph{【原注】}\label{nota-sauce-laguipiere}}

カレームが考案したこのソースのレシピに、本書で加えた変更点はただ1箇所のみ、\protect\hyperlink{glace-de-volaille}{鶏のグラス}ではなく魚のグラスに代えたことだけだ。さらに言うと、このソースはカレームによって「ソース・オ・ブール ラギピエール風」と名付けられたものだ\footnote{カレームの未完の大著『19世紀フランス料理』第3巻に、このソースのレシピが掲載されている。少し長くなるが引用すると「ラグー用片手鍋に、
  \ul{魚料理用グランドソース}の章で示したソース・オ・ブールをレードル
  1杯入れる。ここに上等のコンソメ大さじ1杯か鶏のグラス少々を加える。塩1つまみ、ナツメグ少々、良質のヴィネガーまたはレモン果汁適量を加える。数秒間煮立たせ、上等なバターをたっぷり加えてから供する。(中略)ソースに火を通してからバターを加えるというこの方法によって、なめらかな口あたりで、油っぽくならない仕上がりになる。だからこそ私はこのソース・オ・ブールをグランドソースに分類しなかったのだし、バターを加える派生ソースにおいてこれは重要なことだからだ。それは魚料理用ソースについても同様のことだ(pp.117-118)」。このレシピにおいて、カレームの表現には矛盾がある。「魚用グランドソースの章で示した」とあるのに「グランドソースに分類しなかった」となっていることだ。実際、ソース・オ・ブールそれ自体はこの「ラギピエール風」の直前にある。さて、このソースが「ラギピエール風」であることの理由だが、同じ巻の「魚料理用ソース・エスパニョル」の説明の冒頭において、ラギピエールから聞いた話として、四旬節の期間(小斉=肉断ちをする慣習がカトリックに根強くあった)に、魚料理用のソースにコンソメや仔牛のブロンドのジュを混ぜている修道士料理人がいたの、と述べている。それなら美味しくて当然だろう、とカレームが問うと、ラギピエールは「そうやって作った料理は、通常の肉を食べていい時の料理とは違うものであり、かといって肉断ちの料理でもない、まさに中間のものだ。その判定は天のみぞ知るところだろう。結局のところ、修道士たちは元気に暮していたのだから、それは正しかったのだよ」と煙に巻いたという。カトリックの習慣としての小斉=肉断ちのための魚料理用ソースに、肉由来である鶏のグラスもしくはコンソメを加えるというところが、ラギピエール風と名付けた
  \ruby{所以}{ゆえん}であり、まさにこれこそがソース・ラギピエールの重要なポイントと考えられる。『料理の手引き』においてこのレシピを担当した執筆者はこのエピソードを読んでいなかったのだろうか?
  あるいは何らかの誤解ゆえに改変をしたのか、ラギピエール風の\ruby{所以}{ゆ
  えん}である鶏のグラス、コンソメを用いるべきところを、魚のグラスに代えてしまい、このソース名の由来を換骨奪胎してしまう結果となっている。本書の初版において、原注がその文体から、エスコフィエの手になるものか、あるいは聞き書きしたコメントであることはほぼ明らかなので、なぜエスコフィエがこの点を見逃したか、あるいは許容したのかは非常に興味深い。ところで、カレームが、バターを仕上げの際に加えるということ、いわゆるブールモンテmonter
  au
  beurreによってソースの口あたりをなめらかなものにし、色艶をよくするということをことさらに言及しているの点もまた、注目に値すべきだろう。}。

\hypertarget{sauce-livonienne}{%
\subsubsection[リヴォニア風ソース]{\texorpdfstring{リヴォニア風ソース\footnote{現在のラトビア東北部からエストニア南部にかけての古い地域名、いわゆるバルト三国の一地域と捉えていい。本書執筆時にはロシア帝国の一部となっていた。なお、料理名に冠される地名のうちの少からずのものに明確な由来のないのと同様に、このソースについても名称の由来は不明。}}{リヴォニア風ソース}}\label{sauce-livonienne}}

\frsub{Sauce Livonienne}

\index{そーす@ソース!りうおにあふう@リヴォニア風---}
\index{りうおにあふう@リヴォニア風!そーす@---ソース}
\index{sauce@sauce!livonienne@---  Livonienne}
\index{livonien@livonien(ne)!sauce@Sauce ---ne}

バターを加えて仕上げた\footnote{monter au beurre バターでモンテする。}\protect\hyperlink{veloute-de-poisson}{魚のフュメで作ったヴルテ}1
Lに、1〜2 mmの細さで長さ3〜4 cmの千切り\footnote{julienne ジュリエンヌ}にしたにんじん、セロリ、マッシュリューム、玉ねぎをあらかじめバターを加えて弱火で蒸し煮\footnote{étuver
  au beurre バターでエチュヴェする。}しておいたもの100
gを加える。最後に、1〜2
mmの細さのトリュフの千切りと粗く刻んだパセリを加える。\ldots{}\ldots{}その後、味を調えること。

\ldots{}\ldots{}このソースは、トラウト、サーモン、舌びらめ、チュルボタン\footnote{turbotin
  \textless{} turbo チュルボ。鰈の近縁種。}、バルビュ\footnote{barbue
  鰈の近縁種。}のような魚によく合う。

\hypertarget{sauce-maltaise}{%
\subsubsection[マルタ風ソース]{\texorpdfstring{マルタ風ソース\footnote{シチリアの南方に位置するマルタ島を中心とした国、マルタはオレンジをはじめとした柑橘類の産地であり、とりわけ19世紀にはマルタ産のブラッドオレンジが人気であった。一例としてバルザックの小説『二人の若妻の手記』において、つわりに苦しむ妻のために夫がマルセイユの街で「マルタ産、ポルトガル産、コルシカ産のオレンジを買い求めた」
  (p.312)と書かれている。}}{マルタ風ソース}}\label{sauce-maltaise}}

\frsub{Sauce Maltaise}

\index{そーす@ソース!まるたふう@マルタ風---}
\index{まるたふう@マルタ風!そーす@---ソース}
\index{maltais@maltais(e)!sauce@Sauce ---e}
\index{sauce@sauce!maltaise@--- Maltaise}

前述のとおりに、\protect\hyperlink{sauce-hollandaise}{ソース・オランデーズ}を作り、提供直前に、\textbf{ブラッドオレンジ}2個の搾り汁を加える。ブラッドオレンジを用いないとこのソースは成立しないので注意。オレンジの皮の表面をおろしたもの\footnote{zeste
  ゼスト。} 1つまみを加えて仕上げる。

\ldots{}\ldots{}アスパラガスに添える。

\hypertarget{sauce-mariniere}{%
\subsubsection[ソース・マリニエール]{\texorpdfstring{ソース・マリニエール\footnote{marinier/marinière
  \textless{} mare
  ラテン語「海」から派生した語。貝や魚を白ワインで煮た料理にも付けられる名称。}}{ソース・マリニエール}}\label{sauce-mariniere}}

\frsub{Sauce Marinière}

\index{まりにえーる@マリニエール!そーす@ソース・---}
\index{そーす@ソース!まりにえーる@---・マリニエール}
\index{sauce@sauce!mariniere@--- Marinière}
\index{marinier@marinier(ère)!sauce@Sauce Marinière}

\protect\hyperlink{sauce-bercy}{ソース・ベルシー}を本書で示したとおりの分量で用意する。これにムール貝の茹で汁を詰めたもの大さじ3〜4杯を加え、卵黄6個でとろみを付ける\footnote{卵黄でとろみ付けをする場合、あらかじめ生クリームあるいは茹で汁などで乳化させてからよく混ぜながら加えるのであれば、必ずしも弱火でなくても問題ない。ただし、沸騰状態だと滑かに仕上がらないリスクが残るので、ある程度は弱火にした方がいいだろう。}。

\ldots{}\ldots{}ムール貝の料理専用。

\hypertarget{sauce-matelote-blanche}{%
\subsubsection[白いソース・マトロット]{\texorpdfstring{白いソース・マトロット\footnote{水夫風、船員風、の意。}}{白いソース・マトロット}}\label{sauce-matelote-blanche}}

\frsub{Sauce Matelote blanche}

\index{そーす@ソース!まとろつとしろ@白い---・マトロット}
\index{まとろつと@マトロット!そーすしろ@白いソース・---}
\index{sauce@sauce!matelote blanche@--- Matelote blanche}
\index{matelote@matelote!sauce@Sauce --- blanche}

白ワインで作った魚のクールブイヨン3
dLにフレッシュなマッシュルームの切りくず\footnote{料理、ガルニチュールとして供するマッシュルームは、トゥルネといって螺旋(らせん)状に切り込みを入れて装飾するのが一般的。その下ごしらえの際に大量のマッシュルームの切りくず(おおむねに重量比で15〜20%)が出るのを利用する。}25
gを加えて \(\frac{1}{3}\)量まで煮詰める。

\protect\hyperlink{veloute-de-poisson}{魚料理用ヴルテ}8
dLを加える。数分間煮立たせる。布で漉し、バター150 gを加える。

カイエンヌ\footnote{cayenne
  唐辛子の1品種。日本で一般的なカエンペッパーよりは辛さがマイルドで風味も異なる。}ごく少量で風味を引き締める。

ガルニチュールとして、下茹でしてからバターで色艶よく炒めた\footnote{glacer
  au
  beurre(グラセオブール)。バターでグラセする、と表現する調理現場も多い。glace
  グラス(鏡)が語源であるため、本来は「光沢を出させる、照りをつける」の意だが、食材や料理によってその手法はさまざま。にんじんや小玉ねぎの場合にはあらかじめ下茹でしておく必要がある。}小玉ねぎ20個と、あらかじめ茹でておいた小さな白いマッシュルーム\footnote{これを用意している段階で、上述のトゥルネを行なう。常識的なこととして明記されていないことに注意。この作業の結果、ソースを作る際に魚の茹で汁(クールブイヨン)に加えるマッシュルームの切りくずが発生する。}20個を加える。

\hypertarget{sauce-mornay}{%
\subsubsection[ソース・モルネー]{\texorpdfstring{ソース・モルネー\footnote{19世紀中頃にパリのレストラン、デュランの料理長ジョゼフ・ヴォワロンが創案したと言われている。モルネーは人名だが、具体的に誰を指しているかについては諸説ある。}}{ソース・モルネー}}\label{sauce-mornay}}

\frsub{Sauce Mornay}

\index{もるねー@モルネー!そーす@---ソース}
\index{そーす@ソース!もるねー@---・モルネー}
\index{sauce@sauce!mornay@--- Mornay}
\index{mornay@Mornay!sauce@Sauce ---}

\protect\hyperlink{sauce-bechamel}{ベシャメルソース}1
Lに、このソースを合わせる魚の茹で汁 2
dLを加え、\(\frac{2}{3}\)量程に煮詰める\footnote{初版ではこの煮詰める作業はなく「固めに作ったベシャメルソース1
  L に対し、魚の茹で汁2 dLを加える」となっている。}。おろした\footnote{râper
  (ラペ) \textless{}
  râpe(ラープ)という器具を用いておろすこと。パルメザン(パルミジャーノ)は硬質チーズなので一般的な半筒形のチーズおろし器でいいが、グリュイエールは比較的軟質なので、より目の粗い器具(例えばマンドリーヌに付属している機能のうち、にんじんをおろす際に使う部分など)を用いるといい。}グリュイエールチーズ50
gとパルメザンチーズ50
gを加える。少しの間、火にかけたままにしてよく混ぜ、チーズを完全に溶かし込む。バター100
gを加えて仕上げる\footnote{monter au beurre
  モンテオブール。バターでモンテする、と表現することも多い。}。

\hypertarget{nota-sauce-mornay}{%
\subparagraph{【原注】}\label{nota-sauce-mornay}}

魚以外の料理に合わせる場合\footnote{例えば茹でた野菜などにかけて、サラマンダー(強力な上火だけのオーブンの一種)に入れて軽く焦げ目を付け、グラタンにするようなケースも多い。}も作り方はまったく同じだが、魚の茹で汁は加えない。

\hypertarget{sauce-mousseline}{%
\subsubsection[ソース・ムスリーヌ/ソース・シャンティイ]{\texorpdfstring{ソース・ムスリーヌ/ソース・シャンティイ\footnote{mousseline
  \textless{} mousse ムース。-ine
  は「小さい」を意味する接尾辞。その前にLの文字が入るのは、mousseの語源がメソポタミアの都市Mossoul
  (モスリン布の生産地だった)であることによる。シャンティイの名称で呼ばれることがあるのは、固く立てた生クリームすなわち
  crème
  Chantilly(クレーム・シャンティイ)を加えるところから。\protect\hyperlink{sauce-chantilly}{ソース・シャンティイ}訳注も参照。}}{ソース・ムスリーヌ/ソース・シャンティイ}}\label{sauce-mousseline}}

\frsub{Sauce Mousseline, dite Sauce Chantilly}

\index{むすりーぬ@ムスリーヌ!そーす@ソース・---}
\index{そーす@ソース!むすりーぬ@---・ムスリーヌ}
\index{sauce@sauce!mousseine@--- Mousseline}
\index{mousseline@mousseline!sauce@Sauce ---}

前述のとおりの分量と作り方で\protect\hyperlink{sauce-hollandaise}{オランデーズソース}を用意する(\protect\hyperlink{sauce-hollandaise}{オランデーズソース}参照)。

提供直前に、固く泡立てた生クリーム大さじ4杯\footnote{大さじ1杯 = 15
  ccという考えにとらわれないよう注意。この計量単位は日本で戦後普及したものに過ぎず、本書においては文字通りに「大きなスプーンで4杯」という大雑把な単位として考える必要がある。このソースの場合は「固く泡立てた生クリームを適量」と読み替えてもいいだろう。名称どおりに滑らかでふんわりとした口あたりに仕上げるのがポイント。}をソースに混ぜ込む。

\ldots{}\ldots{}このソースは、茹でた魚や、アスパラガス、カルドン\footnote{cardon
  アーティチョークの近縁種で、アーティチョークが開花前の蕾を食用とするのに対し、カルドンは軟白させた茎葉を食用とする。フランスではトゥーレーヌ地方産が有名。草丈1.5
  m位まで成長させた株を紐で束ねて軟白する。厳冬期は株元から刈り取って小屋などで保管するのが伝統的な手法。イタリア北部ピエモンテでは株を倒してその上に土を被せて軟白するというユニークな方法で栽培するcardo
  gobboカルドゴッボもよく知られている。}、セロリ\footnote{セロリには緑の濃い品種系統と、やや緑が薄く、中心部が自然に軟白されたようになる系統がある。野菜料理として用いられるのは主として後者の芯に近い、自然に軟白された部分。coeur
  de céleri
  クールドセルリと呼ぶ。前者については、もっぱら香味野菜としてフォンやポタージュ、煮込み料理などに用いられる。このタイプは風味に癖があるため、生食にはあまり適していない。}に添える。

\hypertarget{sauce-mousseuse}{%
\subsubsection[ソース・ムスーズ]{\texorpdfstring{ソース・ムスーズ\footnote{細かく泡立った、の意。なお、シャンパーニュのようなvin
  mousseux ヴァン・ムスー(発泡ワイン)のムスーは同じ語の男性形.}}{ソース・ムスーズ}}\label{sauce-mousseuse}}

\frsub{Sauce Mousseuse}

\index{むすー@ムスー(ズ)!そーす@ソース・ムスーズ}
\index{そーす@ソース!むすーす@---・ムスーズ}
\index{sauce@sauce!mousseuse@--- Mousseuse}
\index{mousseux@mousseux/mousseuse!sauce@Sauce Mousseuse}

沸騰した湯の中に、小さめのソテー鍋を入れて熱し、水気をよく拭き取る。このソテー鍋に、あらかじめ充分に柔らかくしておいたバター500
gを入れる。塩8 gを加え、泡立て器でしっかり混ぜながら、レモン
\(\frac{1}{4}\)個分の搾り汁と冷水4 dLを少しずつ加える。

最後に、固く泡立てた生クリーム大さじ4杯を混ぜ込む。

このレシピは、ソースに分類してはいるが、むしろ合わせバターというべきものだ。茹でた魚\footnote{クールブイヨンなどを用いてやや低温で煮た魚、の意。\protect\hyperlink{sauce-hachee-maigre}{魚料理用ソース・アシェ}訳注参照。}に合わせる。

茹でた魚から伝わる熱だけでバターは充分に溶けるので、見た目も風味も溶かしバターをソースにするよりずっといいものだ。

\hypertarget{sauce-moutarde}{%
\subsubsection[ソース・ムタルド]{\texorpdfstring{ソース・ムタルド\footnote{マスタードのこと。マスタードソースと呼んでもいいが、アメリカ風の印象を与えるかも知れない。}}{ソース・ムタルド}}\label{sauce-moutarde}}

\frsub{Sauce Moutarde}

\index{そーす@ソース!むたると@---・ムタルド}
\index{むたると@ムタルド(マスタード)!そーす@ソース・ムタルド}
\index{ますたーと@マスタード(ムタルド)!そーす@ソース・ムタルド}
\index{sauce@sauce!moutarde@--- Moutarde}
\index{moutarde@moutarde!sauce@Sauce ---}

普通、このソースは提供直前に作ること。

必要の分量の\protect\hyperlink{sauce-au-beurre}{ソース・オ・ブール}を用意する。鍋を火から外し、ソース2
\(\frac{1}{2}\) dLあたり大さじ1杯のマスタードを加える。

このソースを仕上げて、提供するまで時間を空けなくてはならない場合は、湯煎にかけておく。沸騰させないよう注意すること。

\hypertarget{sauce-nantua}{%
\subsubsection[ソース・ナンチュア]{\texorpdfstring{ソース・ナンチュア\footnote{ローヌ・アルプ地方にあるナンチュア湖でエクルヴィスが穫れることに由来したソース名。エクルヴィスについて詳しくは\protect\hyperlink{sauce-bavaroise}{バイエルン風ソース}訳注参照。}}{ソース・ナンチュア}}\label{sauce-nantua}}

\frsub{Sauce Nantua}

\index{そーす@ソース!なんちゆあ@---・ナンチュア}
\index{なんちゆあ@ナンチュア!そーす@ソース・---}
\index{えくるういす@エクルヴィス!そーす@ソース・ナンチュア}
\index{sauce@sauce!nantua@--- Nantua}
\index{nantua@Nantua!sauce@Sauce ---}
\index{ecrevisse@ecrevisse!sauce@Sauce Nantua}

\protect\hyperlink{sauce-bechamel}{ベシャメルソース}1 Lに生クリーム2
dLを加え、 \(\frac{2}{3}\)量まで煮詰める。

布で漉し、生クリームをさらに1 \(\frac{1}{2}\)
dL加えて、通常の濃度に戻す。

良質な\protect\hyperlink{beurre-d-ecrevisse}{エクルヴィスバター}125
gと、小さめのエクルヴィスの尾の身\footnote{しっかり下茹でして殻を剥いたものを用いること。}20を加えて仕上げる。

\hypertarget{sauce-new-burg-avec-le-homard-cru}{%
\subsubsection[活けオマールで作るソース・ニューバーグ]{\texorpdfstring{活けオマールで作るソース・ニューバーグ\footnote{ここでは英語由来のソース名のため英語風にカタカナ書きしたが、フランスでは「ニュブール」のように発音されることも多い。}}{活けオマールで作るソース・ニューバーグ}}\label{sauce-new-burg-avec-le-homard-cru}}

\frsub{Sauce New-burg avec le homard cru}

\index{そーす@ソース!にゆーはーくいけおまーる@活けオマールを使う---・ニューバーグ}
\index{にゆーはーく@ニューバーグ!そーすいけおまーる@活けオマールを使うソース・---}
\index{sauce@sauce!new-burg homard cru@--- New-burg avec le homard cru}
\index{new-burg@New-burg!sauce homard cru@Sauce --- avec le homard cru}

800〜900 gのオマールを切り分ける。

胴の中のクリーム状の部分をスプーンで取り出し、これをよくすり潰して30 g
のバターを合わせ、別に取り置いておく。

バター40
gと植物油大さじ4杯を鍋に入れて熱し、切り分けたオマールの身を色付くまで焼く。塩とカイエンヌで調味する。殻が真っ赤になったら、鍋の油を完全に捨て、コニャック大さじ2杯と、マルサラ酒もしくはマデイラの古酒2
dLを注いで火を付けてアルコール分を燃やす\footnote{flamber
  (フロンベ)フランベする。}。注いだ酒が
\(\frac{1}{3}\)量になるまで煮詰めたら、生クリーム2
dLと\protect\hyperlink{fumet-de-poisson}{魚のフュメ}2
dLを注ぐ。弱火で25分間煮る。

オマールの身をざるにあげて水気をきる。殻から身を取り出して、さいの目に切る。

取り置いておいたオマールのクリーム状の部分をソースに混ぜ込み、完全に火が通るように軽く煮立たせてやる。さいの目に切ったオマールの身を加えて混ぜる。味見をして、必要なら塩を加えて修正する。

\hypertarget{nota-sauce-new-burg-avec-le-homard-cru}{%
\subparagraph{【原注】}\label{nota-sauce-new-burg-avec-le-homard-cru}}

さいの目に切ったオマールの身をソースに混ぜ込むのは絶対必要というわけではない。薄くやや斜めにスライスして、このソースを合わせる魚料理に添えてもいい。

\hypertarget{sauce-new-burg-avec-le-homard-cuit}{%
\subsubsection[茹でたオマールで作るソース・ニューバーグ]{\texorpdfstring{茹でたオマールで作るソース・ニューバーグ\footnote{このソースの元となった料理「オマール・ニューバーグ」は、19世紀後半にニューヨークのレストラン、デルモニコーズで常連客のアイデアをもとにフランス出身の料理長シャルル・ラノフェール(チャールズ・レンフォーファー)が完成させたと言われており、そのレシピがラノフェールの著書\href{https://archive.org/details/epicureancomplet00ranhrich}{『ジ・エピキュリアン』}(英語)に掲載されている(p.411)。現在もデルモニコーズのスペシャリテ。なお、これらソースのレシピは第二版で追加されたが、もとの料理は初版から収録されている。}}{茹でたオマールで作るソース・ニューバーグ}}\label{sauce-new-burg-avec-le-homard-cuit}}

\frsub{Sauce New-burg avec le homard cuit}

\index{そーす@ソース!にゆーはーくゆてたおまーる@茹でたオマールを使う---・ニューバーグ}
\index{にゆーはーく@ニューバーグ!そーすゆてたおまーる@茹でたオマールを使うソース・---}
\index{sauce@sauce!new-burg homard cuit@--- New-burg avec le homard cuit}
\index{new-burg@New-burg!sauce homard cuit@Sauce --- avec le homard cuit}

オマールを\protect\hyperlink{court-bouillon-e}{標準的なクールブイヨン}で茹でる。尾の身を殻から外し、やや斜めに厚さ1
cm程度にスライスする\footnote{détailler en escalopes
  (デタイエオネスカロプ)= escalopper(エスカロペ)
  エスカロップ(厚さ1〜2 cm程度の薄切り)に切る。}。ソテー鍋の内側にたっぷりとバターを塗り、そこに切ったオマールを並べるように入れる。塩とカイエンヌでしっかりと味を付け、表皮が赤く発色するように両面を焼く。上等なマデイラ酒をひたひたの高さまで注ぎ、ほぼ完全になくなるまで煮詰める。

提供直前に、オマールのスライスの上に、生クリーム2
dLと卵黄3個を溶いたものを注ぎ、火から外して、ゆっくり混ぜながら\footnote{vanner
  ヴァネする。}しっかりととろみを付ける。

\hypertarget{nota-sauce-new-burg-b}{%
\subparagraph{【原注】}\label{nota-sauce-new-burg-b}}

\protect\hyperlink{sauce-americaine}{ソース・アメリケーヌ}と同様に、これら2種のソースも元来はオマールを供するための料理だった。ソースとオマールが、要するにひとつの料理を構成していたわけだ。

ところが、そのような料理は午餐(ランチ)でしか提供することが出来ない。多くの人々は胃が弱く、夕食では消化しきれないのだ\footnote{レシピにおいて指示されているオマールが大きなものであることに注意。}。

そうした問題解決のために、我々はこれを、舌びらめのフィレやムスリーヌに添えるオマールのソースとして使うことにしたのだ。オマールの身はガルニチュールとして添えるにとどめることにした。結果は好評であった。

カレー粉やパプリカ粉末を調味料として用いれば、このソースのとてもいいバリエーションが作れる。とりわけ舌びらめや脂身の少ない白身魚によく合う。\ldots{}\ldots{}その場合、魚に少量の\protect\hyperlink{riz-indienne}{インド風ライス}を添えるといい。

\hypertarget{sauce-noisette}{%
\subsubsection[ソース・ノワゼット]{\texorpdfstring{ソース・ノワゼット\footnote{ヘーゼルナッツ、はしばみの実。}}{ソース・ノワゼット}}\label{sauce-noisette}}

\frsub{Sauce Noisette}

\index{そーす@ソース!へーせるなっつ@---・ノワゼット}
\index{へーせるなつつ@ヘーゼルナッツ!そーす@ソース・ノワゼット}
\index{のわせつと@ノワゼット!へーぜるなっつそーす@ヘーゼルナッツソース}
\index{sauce@sauce!noisette@--- Noisette}
\index{noisette@noisette!sauce@Sauce ---}

\protect\hyperlink{sauce-hollandaise}{ソース・オランデーズ}を本書のレシピのとおりに作る。提供直前に仕上げとして、上等なバターで作った\protect\hyperlink{beurre-de-noisette}{ブール・ド・ノワゼット}75
gを加える。

\ldots{}\ldots{}ポシェ\footnote{pocher
  沸騰しない程度の温度で茹でること。魚の場合は\protect\hyperlink{court-bouillon-a}{クールブイヨン}を用いてやや低めの温度で火を通すこと。}したサーモン、トラウトにとてもよく合う。

\hypertarget{sauce-normande}{%
\subsubsection{ノルマンディ風ソース}\label{sauce-normande}}

\frsub{Sauce Normande}

\index{そーす@ソース!のるまんてふう@ノルマンディ風---}
\index{のるまんていふう@ノルマンディ風!そーす@---ソース}
\index{sauce@sauce!normande@--- Normande}
\index{normande@normande!sauce@Sauce ---}

\protect\hyperlink{veloute-de-poisson}{魚料理用ヴルテ}\(\frac{3}{4}\)
Lに\footnote{原書にはリットルの表記がないが、本書における標準的な仕上がり量が
  1 Lであることと、文脈から訳者が補った。}、マッシュルームの茹で汁1
dLとムール貝の茹で汁1 dL、舌びらめのフュメ\footnote{舌びらめの料理に合わせるソースであるために、舌びらめのアラなどが必然的に出るのを無駄にせず使うということだが、現代のレストランの厨房などではかえって無理が生じることになる。このレシピ通りに作る場合は何らかのオペレーション上の工夫が必要。}
2
dLを加える。レモン果汁少々と、とろみ付け用に卵黄5個を生クリーム2dLで溶いたものを加える。強火で
\(\frac{2}{3}\)量つまり約8 dLまで煮詰める。

布で漉し、クレーム・ドゥーブル\footnote{乳酸醗酵した濃い生クリーム。\protect\hyperlink{sauce-supreme}{ソース・シュプレーム}訳注参照。}1
dLとバター125 gを加える。

\ldots{}\ldots{}このソースは\protect\hyperlink{sole-normande}{舌びらめのノルマンディ風}専用。とはいえ、使い方によっては無限の可能性がある。

\hypertarget{nota-sauce-normande}{%
\subparagraph{【原注】}\label{nota-sauce-normande}}

基本的に本書では、どんなレシピにおいても、牡蠣の茹で汁は使わないことにしている。牡蠣の茹で汁は塩味がするだけで風味がない。だから、可能であればムール貝の茹で汁を大さじ何杯か加えるほうがずっといい\footnote{このレシピは初版からの異同が大きい。初版では「魚料理用ヴルテ1
  Lあたり卵黄6個でとろみを付け、牡蠣の茹で汁2
  dLと魚のエッセンス、生クリーム2
  dLを加えながら煮詰める。仕上げにバター100 gとクレーム・ドゥーブル1
  dLを加える」となっており、用途には触れられていない。第二版、第三版ではやや細かなレシピとなり用途も「舌びらめのノルマンディ風」と指定されて現行版に近いものになるが、牡蠣の茹で汁を使うことは初版と同じ。つまり、第四版で牡蠣の茹で汁からムール貝の茹で汁を使うことに変更し、この原注が付けられた。このソースにおける改変は、前出のソース・ラギピエールのケースとやや似ているところもある。牡蠣を用いることから、牡蠣の産地であるノルマンディ風という名称となったソースであるのに、そこから牡蠣を排除するという、いわば換骨奪胎がなされているからだ。}。

\hypertarget{sauce-orientale}{%
\subsubsection[オリエント風ソース]{\texorpdfstring{オリエント風ソース\footnote{フランス語の
  orient
  オリヨン(東方)は、具体的にいうと北アフリカの一部、アラビア半島、西アジアくらいまでを指すのが一般的。その意味では、カレー粉を加えたことで「オリエント風」と称するのは、当時のフランス人にとって、理解できなくもないだろうが実感は伴わなかった可能性がある。フランス人にとっての「オリエント」である北アフリカやトルコといった地域の食文化は19世紀にはかなりフランスに伝わっていたからだ。こういった文化的なイメージのずれは、エスコフィエ本人が料理長としてのキャリアの大半をイギリスで過ごしたこととも関係があると思われる。つまり、フランス人にとっての「オリエント」とインドという植民地を持つイギリス人の「オリエント」は同じ言葉であっても、想起される具体的な内容が違うということである。}}{オリエント風ソース}}\label{sauce-orientale}}

\frsub{Sauce Orientale}

\index{そーす@ソース!おりえんとふう@オリエント風---}
\index{おりえんとふう@オリエント風!そーす@---ソース}
\index{とうほうふう@東方風!おりえんたるそーす@オリエント風ソース}
\index{sauce@sauce!orientale@--- Orientale}
\index{oriental@oriental(e)!sauce@Sauce ---e}

\protect\hyperlink{sauce-americaine}{ソース・アメリケーヌ}
\(\frac{1}{2}\) Lを用意し、カレー粉で風味付けをして
\(\frac{2}{3}\)量まで煮詰める。鍋を火から外し、生クリーム 1
\(\frac{1}{2}\) dLを混ぜ込む。

\ldots{}\ldots{}このソースの用途は\protect\hyperlink{sauce-americaine}{ソース・アメリケーヌ}と同じ。

\hypertarget{sauce-paloise}{%
\subsubsection[ポー風ソース]{\texorpdfstring{ポー風ソース\footnote{ポーは15世紀以来、ベアルヌ地方の中心都市。}}{ポー風ソース}}\label{sauce-paloise}}

\frsub{Sauce paloise}

\index{そーす@ソース!ほーふう@ポー風---}
\index{ほーふう@ポー風!そーす@---ソース}
\index{sauce@sauce!paloise@--- Paloise}
\index{palois@palois!sauce@Sauce Paloise}
\index{pau@Pau!sauce paloise@Sauce Paloise}

\protect\hyperlink{sauce-bearnaise}{ソース・ベアルネーズ}を本書に書いてあるとおりの方法と分量で用意する(\protect\hyperlink{sauce-bearnaise}{ソース・ベアルネーズ}参照)が、以下の点を変える。

\begin{enumerate}
\def\labelenumi{\arabic{enumi}.}
\item
  香りの中心となるエストラゴンを同量のミント\footnote{フランス料理よりはむしろイギリス料理でよく使われるミントを用いたこのソースをポー風と呼ぶのは、かつてこの地がイギリス貴族たちに保養地として好まれたことにちなんでいるという説もある。}に変更し、白ワインとヴィネガーを煮詰める際に加える。
\item
  さらに、仕上げの際に、細かく刻んだエストラゴンも使わない。細かく刻んだミントを使う。
\end{enumerate}

\ldots{}\ldots{}このソースの用途はソース・ベアルネーズとまったく同じ。

\hypertarget{sauce-poulette}{%
\subsubsection{ソース・プレット}\label{sauce-poulette}}

\frsub{Sauce Poulette}

\index{そーす@ソース!ふれつと@---・プレット}
\index{ふれつと@プレット!そーす@ソース・---}
\index{sauce@sauce!poulette@--- Poulette}
\index{poulette@poulette!sauce@Sauce ---}

マッシュルームの茹で汁2 dLを
\(\frac{1}{3}\)量まで煮詰める。ここに\protect\hyperlink{sauce-allemande}{ソース・アルマンド}1
Lを加え、数分間沸騰させる。鍋を火から外し、レモン果汁少々とバター60
g、パセリのみじん切り大さじ1杯を加えて仕上げる。

\ldots{}\ldots{}このソースは野菜料理に合わせるが、羊の足の料理にもよく合う。

\hypertarget{sauce-ravigote}{%
\subsubsection[ソース・ラヴィゴット]{\texorpdfstring{ソース・ラヴィゴット\footnote{ravigote
  \textless{} ravigoter
  身体を丈夫にする、元気にさせる、の派生語。香草を主体として酸味を効かせたソース(および煮込み料理)は中世以来あったが、18世紀以降ravigoteという呼び名が一般的となり、19世紀以降はこの表現がしばしば使われるようになった。ソース・ラヴィゴットは冷製と温製の2種がある。なお、ソース・ラヴィゴットのレシピとして最初期のもののひとつ、1755年刊ムノン『宮廷の晩餐』第1巻に掲載されているソース・ラヴィゴットの作り方は、薄切りにしたにんにく、セルフイユ、サラダバーネット、エストラゴン、クレソンアレノワ(オルレアン芹)、シブレットを洗ってから圧し潰し、コップ1杯のコンソメ(=この当時のコンソメはグラスドヴィアンドに近いものであることに注意)に入れて沸騰させないよう1時間以上かけて煎じる。漉し器で押すようにして漉し、ブールマニエ、塩、こしょうで味付けをして火にかけ、レモンの搾り汁で仕上げる、というもの(p.135)。}}{ソース・ラヴィゴット}}\label{sauce-ravigote}}

\frsub{Sauce Ravigote}

\index{そーす@ソース!らういこつと@---・ラヴィゴット}
\index{らういこつと@ラヴィゴット!そーす@ソース・---}
\index{sauce@sauce!ravigote@--- Ravigote}
\index{ravigote@ravigote!sauce@Sauce ---}

白ワイン1 \(\frac{1}{2}\) dLとヴィネガー1 \(\frac{1}{2}\)
dLを半量になるまで煮詰める。\protect\hyperlink{veloute}{標準的なヴルテ}8
dLを加え、数分間煮立たせる。鍋を火から外し、\protect\hyperlink{beurre-d-echalote}{エシャロットバター}90〜100
gと、セルフイユ\footnote{cerfeuil チャービル。}とエストラゴン\footnote{estragon
  フレンチタラゴン。詳しくは\protect\hyperlink{sauce-chasseur}{ソース・シャスール}訳注参照。}、シブレット\footnote{ciboulette
  日本ではチャイブとも呼ばれる。アサツキと訳されることもあるが、風味がまったく異なるので代用は不可。春に紫色の小さくてきれいな花をたくさん咲かせるので、エディブルフラワーとしてもよく用いられる。}を細かく刻んだものを同量ずつ合わせたもの計大さじ1
\(\frac{1}{2}\)杯を加えて仕上げる。

\ldots{}\ldots{}茹でた鶏に合わせる。白い内臓\footnote{家畜の副生物すなわち正肉以外の部分のうち、内臓をabatsアバと呼ぶ。そのうちの、心臓、レバー、舌などはabats
  rougeアバルージュ(赤い内臓)、耳、尾、胃、腸、足、頭、仔牛および仔羊の胸腺肉(ris
  de veauリドヴォー、ris
  d'agneauリダニョー)や腸間膜(fraiseフレーズ)などはabats
  blancアバブロン(白い内臓、白い副生物)を呼ばれている。こうした副生物の料理は古くから好まれ、16世紀フランソワ・ラブレー『ガルガンチュアとパンタグリュエル』においてもしばしば登場する。とりわけ「ガルガンチュア」の冒頭では、出産間近なお妃が臓物料理を食べ過ぎるなどというエピソードが印象深い。なお、鶏の副生物(とさか、内臓、脚など)はabattisアバティと呼ばれるので混同しないよう注意。}料理にも合わせることがある。

\hypertarget{sauce-regence-pour-poissons}{%
\subsubsection[魚料理用ソース・レジャンス]{\texorpdfstring{魚料理用ソース・レジャンス\footnote{ソース・レジャンスという名称については「ブラウン系の派生ソース」の\protect\hyperlink{sauce-regence}{ソース・レジャンス}訳注参照。}}{魚料理用ソース・レジャンス}}\label{sauce-regence-pour-poissons}}

\frsub{Sauce Régence pour Poissons, et garnitures de Poissons}

\index{そーす@ソース!れしやんすさかなよう@魚料理用---・レジャンス}
\index{れしやんす@レジャンス!そーすさかなよう@魚料理用ソース・---}
\index{sauce@sauce!regence poissons@--- Régence pour Poissons}
\index{regence@Régence!sauce poissons@Sauce --- pour Poissons}

ライン産ワイン2 dLと\protect\hyperlink{fumet-de-poisson}{魚のフォン}2
dLに新鮮なマッシュルームの切りくず20
gと生トリュフの切りくず20gを加えて半量になるまで煮詰める。

煮詰まったら布で漉し、仕上げた状態の\protect\hyperlink{sauce-normande}{ノルマンディ風ソース}8
dLを加える。

トリュフエッセンス大さじ1杯を加えて仕上げる。

\hypertarget{sauce-regence-pour-garnitures-de-volaille}{%
\subsubsection[鶏料理のガルニチュール用ソース・レジャンス]{\texorpdfstring{鶏料理のガルニチュール用ソース・レジャンス\footnote{わかりやすい例としては、後述の\protect\hyperlink{garniture-regence}{ガルニチュール・レジャンス
  B}参照。}}{鶏料理のガルニチュール用ソース・レジャンス}}\label{sauce-regence-pour-garnitures-de-volaille}}

\frsub{Sauce Régence pour garnitures de Volaille}

\index{そーす@ソース!れしやんすとりりようりのかるにてゆーるよう@鶏料理のガルニチュール用---・レジャンス}
\index{れしやんす@レジャンス!そーすとりりようりののかるにてゅーるよう@鶏料理のガルニチュール用ソース・---}
\index{sauce@sauce!regence garnitures de volaille@--- Régence pour garnitures de Volaille}
\index{regence@Régence!sauce garnitures de volaille@Sauce --- pour garnitures de Volaille}

ライン産白ワイン2 dLとマッシュルームの茹で汁2 dLにトリュフの切りくず 40
gを加え、半量になるまで煮詰める。

\protect\hyperlink{sauce-allemande}{ソース・アルマンド}8
dLを加え、布で漉す。トリュフエッセンス大さじ1杯を加えて仕上げる。

\hypertarget{sauce-riche}{%
\subsubsection[ソース・リッシュ]{\texorpdfstring{ソース・リッシュ\footnote{リッチな、裕福な、の意。ソース・ディプロマットがそもそも豪華な料理に合わせるものであり、さらにトリュフを足すことでより一層「リッチ」なものにした、ということ。}}{ソース・リッシュ}}\label{sauce-riche}}

\frsub{Sauce Riche}

\index{そーす@ソース!りつしゆ@---・リッシュ}
\index{りつしゆ@リッシュ!そーす@ソース・---}
\index{sauce@sauce!riche@--- Riche} \index{riche@riche!sauce@Sauce ---}

\protect\hyperlink{sauce-diplomate}{ソース・ディプロマット}を本書で示したとおりの分量と作り方で用意する。

トリュフエッセンス1 dLと、さいの目に切った真黒なトリュフ80
gを加えて仕上げる。

\hypertarget{sauce-rubens}{%
\subsubsection[ソース・ルーベンス]{\texorpdfstring{ソース・ルーベンス\footnote{フランドル派の画家、Peter
  Paul Rubens
  ピーテル・パウル・ルーベンス(1577〜1640)のこと。フランスでは古くから
  Pierre Paul Rubens
  ピエール・ポール・リュベンスの表記が定着している、現代フランスでは原語のままの綴り、発音を尊重することが多い。}}{ソース・ルーベンス}}\label{sauce-rubens}}

\frsub{Sauce Rubens}

\index{そーす@ソース!るーへんす@---・ルーベンス}
\index{るーへんす@ルーベンス!そーす@ソース・---}
\index{rubens@Rubens!sauce@Sauce ---}
\index{sauce@sauce!rubens@--- Rubens}

1〜2 mm角の小さなさいの目\footnote{brunoise ブリュノワーズ}に切った\protect\hyperlink{}{標準的なミルポワ}100
gをバターで色付くまで炒める。白ワイン2
dLと\protect\hyperlink{fumet-de-poisson}{魚のフュメ}3
dLを注ぎ、25分間火にかけておく。

目の細かいシノワ\footnote{円錐形で取っ手の付いた漉し器。}で漉す。数分間静かに休ませてから、浮いてきた油脂を丁寧に取り除く\footnote{dégraisser
  デグレセ。}。 \(\frac{1}{2}\) dLになるまで煮詰め、マデイラ酒大さじ1
杯を加える。

ここに卵黄2個を加えてとろみを付け、普通のバター100
gと\protect\hyperlink{beurre-colorant-rouge}{色付け用の赤いバター}30
g、アンチョビエッセンス少々を加えて仕上げる。

\ldots{}\ldots{}茹でた、すなわちポシェ\footnote{魚の場合はクールブイヨンを用いて、沸騰しない程度の温度で加熱調理すること。}した魚にこのソースはとてもよく合う。

\hypertarget{sauce-saint-malo}{%
\subsubsection[サンマロ風ソース]{\texorpdfstring{サンマロ風ソース\footnote{ブルターニュ地方の港町。観光地として有名であり、バカンスシーズンには多くの人が訪れる。}}{サンマロ風ソース}}\label{sauce-saint-malo}}

\frsub{Sauce Saint-Malo}

\index{そーす@ソース!さんまろふう@サンマロ風---}
\index{さんまろふう@サンマロ風!そーす@---ソース}
\index{sauce@sauce!saint-malo@--- Saint-Malo}
\index{saint-malo@Saint-Malo!sauce@Sauce ---}

(仕上がり5 dL分)

本書で示したとおりに作った\protect\hyperlink{sauce-vin-blanc}{白ワインソース}
\(\frac{1}{2}\)
Lに細かく刻んで白ワインで茹でたエシャロット大さじ1杯、もしくは、可能なら、\protect\hyperlink{beurre-d-echalote}{エシャロットバター}50
gと、マスタード大さじ
\(\frac{1}{2}\)杯、アンチョビエッセンス少々を加える。

\ldots{}\ldots{}海水魚のグリルに合わせる。

\hypertarget{sauce-smitane}{%
\subsubsection[ソース・スミターヌ]{\texorpdfstring{ソース・スミターヌ\footnote{サワークリームを意味するロシア語
  \ltjsetparameter{jacharrange={-2}} Сметана
  \ltjsetparameter{jacharrange={+2}}スメタナが由来。ロシア料理とフランス料理との相互影響関係にいては、\protect\hyperlink{service-russe}{序
  p.II 訳注3}
  および\protect\hyperlink{sauce-moscovite}{モスクワ風ソース}訳注参照。}}{ソース・スミターヌ}}\label{sauce-smitane}}

\frsub{Sauce Smitane}

\index{すみたーぬ@スミターヌ!そーす@ソース・---}
\index{そーす@ソース!すみたーぬ@---・スミターヌ}
\index{さわーくりーむ@サワークリーム!そーすすみたーぬ@ソース・スミターヌ}
\index{sauce@sauce!smitane@--- Smitane}
\index{smitane@smitane!sauce@Sauce ---}
\index{creme aigre@crème aigre!sauce smitane@Sauce Smitane}

中位の大きさの玉ねぎ2個を細かくみじん切りにし、バターで色付くまで炒める。白ワイン2
dLを注ぎ、完全に煮詰める。サワークリーム \(\frac{1}{2}\) Lを加える。
5分間沸騰させたら、布で漉す。サワークリームの風味を生かすために、必要に応じてレモンの搾り汁少々を加える。

\ldots{}\ldots{}ジビエのソテーやカスロール仕立て\footnote{原文は gibiers
  sautés, ou cuits à la casserole
  となっており、ジビエのソテーまたはカスロール(片手鍋)で火を通したもの、というのが逐語訳だが、ここでは
  en casserole に解釈して訳した。雉、ペルドロー
  (山うずらの若鳥)、野生のうずらなどのen casserole
  が本書にも多数収録されているためである。カスロール仕立てen
  casseroleとは、油脂を熱したカスロールで肉を焼いた後に取り出し、フォンなどを加えてソースを作り、肉を鍋に戻し入れて鍋ごと供する仕立てのこと。なお、casserole
  のうちフランスに古くからあるタイプのものは比較的浅い鍋で、ソースパンとも呼ばれる。深いものはcasserole
  russe カスロールリュス(ロシア式片手鍋)と言う。}用。

\hypertarget{sauce-solferino}{%
\subsubsection{ソース・ソルフェリノ}\label{sauce-solferino}}

\frsub{Sauce Solférino}

\index{そーす@ソース!そるふえりの@---・ソルフェリノ}
\index{そるふえりの@ソルフェリノ!そーす@ソース・---}
\index{sauce@sauce!solferino@--- Solférino}
\index{solferino@Solférino!sauce@Sauce ---}

よく熟したトマト15個をしっかり搾って、その果汁を器に入れる。これを布で漉し、濃いシロップ状になるまで煮詰める\footnote{第2章ガルニチュール、\protect\hyperlink{essence-de-tomate}{トマトエッセンス}も参照。}。

溶かした\protect\hyperlink{glace-de-viande}{グラスドヴィアンド}大さじ3杯とカイエンヌ1つまみ、レモン\(\frac{1}{2}\)個分の搾り汁を加える。

火から外して、エストラゴン風味の\protect\hyperlink{beurre-maitre-d-hotel}{メートルドテルバター}100
gと\protect\hyperlink{beurre-d-echalote}{エシャロットバター}100
gを加える。

\ldots{}\ldots{}このソースはどんな肉のグリルにもよく合う。

\hypertarget{nota-sauce-solferino}{%
\subparagraph{【原注】}\label{nota-sauce-solferino}}

言い伝えによると、フランス軍がたびたび進軍して戦ったロンバルディア平野で、たくさんの料理が創作された。このソースもそのひとつであり、カプリアナ村においてフランスとサルデーニャの連合軍司令官の昼食に供されたという。その村の近くであの苛烈きわまるソルフェリノの戦い\footnote{1859年に起きたフランス=サルデーニャ連合軍とオーストリア帝国軍の戦闘。戦場視察したナポレオン三世はその光景のあまりの悲惨さにイタリア独立戦争への介入から手を引くことを決意したともいう。}が繰り広げられたのだ。

伝えられているレシピはおそらくは調理担当軍人によるものだろうが、充分に日常的に使えるものだった。このソースは、Sauce
Saint-Cloud ソース・サンクルー\footnote{サンクルーはパリ近郊の地名。普仏戦争時(1870〜1871)にパリ包囲戦の舞台となり、休戦協定の結ばれた2日後に大火に見舞われた。いずれにせよ戦争の悲惨さを蔭に持つソース名ということになるが、エスコフィエ自身が普仏戦争において従軍したために、その名称をこのソースに付けることは許し難かったのだろう。}と呼ばれることもあるが、それは誤りだ。作り方も材料もソース・サンクルーの名を付けるにはまったく値しない程の誤りだ。

\hypertarget{sauce-soubise}{%
\subsubsection[ソース・スビーズ/玉ねぎのクリ・スビーズ]{\texorpdfstring{ソース・スビーズ/玉ねぎのクリ・スビーズ\footnote{18世紀の代表的料理人のひとりFrançois
  Marinフランソワ・マラン(生没年不詳)が仕えたシャルル・ド・ロアン・スビーズ元帥のこと。マランは4巻からなる『コモス神の贈り物、あるいは食卓の悦楽』(1739
  年刊)を著した。}}{ソース・スビーズ/玉ねぎのクリ・スビーズ}}\label{sauce-soubise}}

\frsub{Sauce Soubise, ou Coulis d'oignons Soubise}

\index{そーす@ソース!すひーす@---・スビーズ}
\index{すひーす@スビーズ!そーす@ソース・---}
\index{くり@クリ!たまねぎのくりすひーず@玉ねぎのクリ・スビーズ}
\index{sauce@sauce!soubise@--- Soubise}
\index{soubise@Soubise!sauce@Sauce ---}
\index{coulis@coulis!oignons soubise@Coulis d'oignons Soubise}

このソースの作り方には以下の2つがある。

\begin{enumerate}
\def\labelenumi{\arabic{enumi}.}
\tightlist
\item
  玉ねぎ500 gを薄切りにする\footnote{émincer
    (エモンセ)日本ではエマンセと言うことが多い。}。これをしっかり下茹でしておく。
\end{enumerate}

玉ねぎはしっかりと水気をきって、バターを加えて鍋に蓋をして弱火で色付かないよう注意して蒸し煮する\footnote{étuver
  エチュヴェ。}。ここに濃厚に作った\protect\hyperlink{sauce-bechamel}{ベシャメルソース}
\(\frac{1}{2}\)
Lを加える。塩1つまみと白こしょう少々、粉砂糖1つまみ強を加える。

オーブンに入れてじっくり火入れする。布で漉し、鍋に移したソースを熱する。バター80
gと生クリーム1 dLを加えて仕上げる。

\begin{enumerate}
\def\labelenumi{\arabic{enumi}.}
\setcounter{enumi}{1}
\tightlist
\item
  上記と同様に薄切りにした玉ねぎを下茹でし、水気をきる。豚背脂の薄いシート\footnote{barde
    de lard
    (バルドドラール)豚背脂を薄くスライスしたもの。ベーコンと誤解されがちなので注意。エスコフィエ以前の時代のフランス料理ではきわめて多用されるとても重要なものなのでぜひとも覚えておきたい。自作する際には、豚背脂の塊を冷凍した後、適度な固さに戻してから機械などを使用してスライスすると作業が容易になる。}を敷き詰めた丁度いい大きさの深手の片鍋\footnote{casserole
    russe
    \protect\hyperlink{sauce-smitane}{ソース・スミターヌ}訳注参照。}に、下茹でして水気をきった玉ねぎをすぐに入れ、カロライナ米\footnote{長粒種。リゾットなどに適している。}120
  gと\protect\hyperlink{consomme-blanc}{白いコンソメ}7
  dL、塩、こしょう、砂糖は上記と同様に加え、さらにバター25gも加える。
\end{enumerate}

強火にかけて沸騰したら、オーブンに入れてゆっくり加熱する。

鉢に米と玉ねぎを移し入れてすり潰す。これを布で漉し、温める。上記と同様にバターを生クリームを加えて仕上げる。

\hypertarget{nota-sauce-soubise}{%
\subparagraph{【原注】}\label{nota-sauce-soubise}}

スビーズはソースというよりはむしろクリ\footnote{クリcoulisについては、\protect\hyperlink{sauce-salmis}{ソース・サルミ}訳注参照。}であって、真っ白な仕上がりにすべきだ。

ベシャメルを用いた作り方のほうが米を用いるよりもいいだろう。というのも、より滑らかな口あたりのクリになるからだ。その一方、米を使うとよりしっかりした仕上がりになる。

どちらの方法で作るかは、このスビーズを合わせる料理の種類によって決めるべきだ。

\hypertarget{sauce-soubise-tomatee}{%
\subsubsection{トマト入りソース・スビーズ}\label{sauce-soubise-tomatee}}

\frsub{Sauce Soubise tomatée}

\index{そーす@ソース!すひーすとまといり@トマト入り---・スゥビーズ}
\index{すひーす@スビーズ!そーすとまといり@トマト入りソース・---}
\index{sauce@sauce!soubise tomatee@--- Soubise tomatée}
\index{soubise@Soubise!sauce tomatee@Sauce --- tomatée}

上記のいずれかの方法で作ったソース・スビーズに
\(\frac{1}{3}\)量の、滑らかで真っ赤なトマトピュレを加える。

\hypertarget{sauce-souchet}{%
\subsubsection[ソース・スーシェ]{\texorpdfstring{ソース・スーシェ\footnote{ナポレオン軍の元帥を務めたルイ・スーシェ・アルビュフェラ公爵のこと。
  正しくはSuchetだが、料理名としては Souchet
  とも綴られる。\protect\hyperlink{sauce-albufera}{ソース・アルビュフェラ}訳注参照。}}{ソース・スーシェ}}\label{sauce-souchet}}

\frsub{Sauce Souchet}

\index{そーす@ソース!すーしえ@---・スーシェ}
\index{すーしえ@スーシェ!そーす@ソース・---}
\index{sauce@sauce!souchet@--- Souchet}
\index{souchet@Souchet!sauce@Sauce ---}

オランダおよびフランドル地方の\protect\hyperlink{}{ワーテルゾイ}から派生したソース。

いくらか変化したかたちでイギリス料理に取り入れられ、近代料理の原則に合うようにさらに手を加えたもの。

細さ1〜2 mm角、長さ3〜4 cmの千切り\footnote{julienne ジュリエンヌ。}にした、にんじん、根パセリ、セロリ計150
gを用意する。

これを鍋に入れてバターを加え、蓋をして蒸し煮する\footnote{étuver au
  beurre エチュヴェオブール}。\protect\hyperlink{fumet-de-poisson}{魚のフォン}\(\frac{3}{4}\)
Lと白ワイン2 dLを注ぐ。弱火で煮て、このクールブイヨン\footnote{court-bouillon
  原義は「量の少ないブイヨン」。実際、魚などを茹でる(ポシェする)際には、ぎりぎりの大きさの鍋を用いて茹で汁の量は出来るだけ少なく済むようにする。誤解しやすい用語なので注意。}を漉す。千切りにした野菜は別に取り置いておく。

このクールブイヨンで、切り分けた魚を煮る。

魚に火が通ったら、魚の身を取り出して、クールブイヨンはシノワ\footnote{円錐形に取っ手の付いた漉し器。}漉す。これを約
\(\frac{1}{4}\)量すなわち2 \(\frac{1}{2}\)
dLになるまで煮詰める。\protect\hyperlink{sauce-vin-blanc}{白ワインソース}を加えて適当なとろみが付くようにする。あるいは単純にブールマニエでとろみを付け、軽くバターを加えてもいい。

ソースの中に取り置いていた千切りの野菜を戻し入れる。魚の切り身を覆うようにソースをかけて供する。

\hypertarget{sauce-tyrolienne}{%
\subsubsection[チロル風ソース]{\texorpdfstring{チロル風ソース\footnote{そもそも\protect\hyperlink{sauce-choron}{ソース・ショロン}をバターではなく植物油を用いて作るものであるから、オーストリアのチロル地方とはまったく関係がない。1848年のイタリア、チロルでのオーストリアに対する反乱を記念した命名だという説もあるが、真偽は不明。ただし、本書の初版からほぼ異同のない内容で収録されているため、それなりに古くから存在しているソースと思われる。}}{チロル風ソース}}\label{sauce-tyrolienne}}

\frsub{Sauce Tyrolienne}

\index{そーす@ソース!ちろるふう@チロル風---}
\index{ちろるふう@チロル風!そーす@---ソース}
\index{sauce@sauce!tyrolienne@--- Tyrolienne}
\index{tyrolien@tyrolien(ne)!sauce@Sauce ---ne}

\protect\hyperlink{sauce-bearnaise}{ソース・ベアルネーズ}を作る場合とまったく同じ要領で、白ワインとヴィネガー、香草類を煮詰める(\protect\hyperlink{sauce-bearnaise}{ソース・ベアルネーズ})参照。布で漉してきつく絞る。

これに、よく煮詰めた真っ赤なトマトピュレ大さじ2杯と卵黄6個を加える。鍋をごく弱火にかけながら、\protect\hyperlink{mayonnaise}{マヨネーズ}を作る要領で植物油5
dLを加えてしっかりと乳化させる。最後に味を調え、カイエンヌ\footnote{赤唐辛子の一品種だが、日本のカエンペッパーより辛さもマイルドで風味が違うことに注意。}ごく少量で風味を引き締める。

\ldots{}\ldots{}このソースは牛肉、羊肉のグリルや魚のグリル焼きに合う。

\hypertarget{sauce-tyrolienne-a-l-ancienne}{%
\subsubsection[チロル風ソース・クラシック]{\texorpdfstring{チロル風ソース・クラシック\footnote{このレシピは第四版のみ。ここでの
  à l'ancienne
  は「昔ながらの」という意味ではないと解釈される。ベースとなっているソース・ポワヴラードが古くからあるソースであることからこの名称を付けたのだろう。}}{チロル風ソース・クラシック}}\label{sauce-tyrolienne-a-l-ancienne}}

\frsub{Sauce Tyrolienne à l'ancienne}

\index{そーす@ソース!ちろるふうくらしつく@チロル風---・クラシック}
\index{ちろるふう@チロル風!そーす@---ソース・クラシック}
\index{sauce@sauce!tyrolienne ancienne@--- Tyrolienne à l'ancienne}
\index{tyrolien@Tyrolien!sauce ancienne@Sauce Tyrolienne à l'ancienne}

大きめの玉ねぎ2個をごく薄くスライス\footnote{émincer エマンセ。}してバターで炒める。トマト3個を押し潰して皮を剥き、種を取り除いてから加える。\protect\hyperlink{sauce-poivrade}{ソース・ポワヴラード}5
dLを加える。7〜8分間煮て仕上げる。

\hypertarget{sauce-valois}{%
\subsubsection{ソース・ヴァロワ}\label{sauce-valois}}

\frsub{Sauce Valois}

\index{うあろわ@ヴァロワ!そーす@ソース・---}
\index{そーす@ソース!うあろわ@---・ヴァロワ}
\index{sauce@sauce!valois@--- Valois}
\index{valois@Valois!sauce@Sauce ---}

\protect\hyperlink{sauce-bearnaise-a-la-glace-de-viande}{グラスドヴィアンド入りソース・ベアルネーズ}のこと(\protect\hyperlink{sauce-bearnaise}{ソース・ベアルネーズ}参照)。

\hypertarget{nota-sauce-valois}{%
\subparagraph{【原注】}\label{nota-sauce-valois}}

「ソース・ヴァロワ」はグフェが1863年頃に創案したらしい。少なくともその頃に作られるようになったものであろう。近年では「ソース・フォイヨ」の名称のほうが一般的だが、いかにもあり得そうな異論反論を受けないためにもここでその起源を記しておくのがいいと思われた。

\hypertarget{sauce-venitienne}{%
\subsubsection[ヴェネツィア風ソース]{\texorpdfstring{ヴェネツィア風ソース\footnote{ヴェネツィア料理ではさまざまな香草を用いるものがあることから、その影響を受けた、あるいは類似したものにこの名称が付けられることが多い。なお、ヴェネツィアの近く、漁港で有名なキオッジャ近郊は農業がとても盛んで、地場品種の野菜もビーツやカボチャ、ラディッキオにもキオッジャの名が付く品種がある。}}{ヴェネツィア風ソース}}\label{sauce-venitienne}}

\frsub{Sauce Vénitienne}

\index{そーす@ソース!うえねついあふう@ヴェネツィア風---}
\index{うえねついあふう@ヴェネツィア風!そーす@---ソース}
\index{sauce@sauce!venitienne@--- Vénitienne}
\index{venitien@vénitien(ne)!sauce@Sauce ---ne}

エストラゴンヴィネガー4
dLに、エシャロットのみじん切り大さじ2杯とセルフイユ25
gを加え、\(\frac{1}{3}\)量まで煮詰める。煮詰めたら布で漉し、軽く絞ってやる。ここに\protect\hyperlink{sauce-vin-blanc}{白ワインソース}\(\frac{3}{4}\)
L
を加える。\protect\hyperlink{beurre-colorant-vert}{色付け用の緑のバター}125
gと、セルフイユとエストラゴンのみじん切り大さじ1杯を加えて仕上げる。

\ldots{}\ldots{}さまざまな魚料理に添える。

\hypertarget{sauce-veron}{%
\subsubsection[ソース・ヴェロン]{\texorpdfstring{ソース・ヴェロン\footnote{Luis
  Véron
  (1798〜1867)。医師であり、文学愛好家、美食家としても有名だった。文芸誌「ルヴュ・ド・パリ」を主宰した後、新聞「ル・コンスティチュショネル」の社主となり、ウージェーヌ・シューの新聞連載小説『彷徨えるユダヤ人』を掲載、大ヒットに導いた。、自宅は文壇サロンのようだったという。主著『パリのとあるブルジョワの回想録』(1853〜
  1955年刊)。}}{ソース・ヴェロン}}\label{sauce-veron}}

\frsub{Sauce Véron}

\index{そーす@ソース!うえろん@---・ヴェロン}
\index{うえろん@ヴェロン!そーす@ソース・---}
\index{veron@Véron!sauce@Sauce ---} \index{sauce@sauce!veron@--- Véron}

仕上げた状態の\protect\hyperlink{sauce-normande}{標準的なノルマンディ風ソース}\(\frac{3}{4}\)
Lに、\protect\hyperlink{sauce-tyrolienne}{チロル風ソース}\(\frac{1}{4}\)
Lを加える。よく混ぜ合わせ、溶かしたブロンド色の\protect\hyperlink{glace-de-viande}{グラスドヴィアンド}大さじ2杯とアンチョビエッセンス大さじ1杯を加えて仕上げる。

\ldots{}\ldots{}魚料理用。

\hypertarget{sauce-villageoise}{%
\subsubsection[村人風ソース]{\texorpdfstring{村人風ソース\footnote{文字通り「村人風」の意。このソースの他にもこの名称を冠した料理はあるが由来などは不明。}}{村人風ソース}}\label{sauce-villageoise}}

\frsub{Sauce Villageoise}

\index{そーす@ソース!むらひとふう@村人風---}
\index{むらひとふう@村人風!そーす@---ソース}
\index{sauce@sauce!villageoise@--- Villageoise}
\index{villageois@villageois!sauce@Sauce Villageoise}

\protect\hyperlink{veloute}{標準的なヴルテ}\(\frac{3}{4}\)
Lに、ブロンド色の\protect\hyperlink{jus-de-veau-brun}{仔牛のジュ}\footnote{本書には「仔牛の茶色いジュ」のレシピはあるが、ブロンド色のものについては記述がない。}1
dLとマッシュルームの茹で汁1 dLを加える。
\(\frac{2}{3}\)量くらいまで煮詰め、布で漉す。

\protect\hyperlink{sauce-soubise}{ベシャメルで作ったソース・スビーズ}\footnote{2つある作り方のうちの1の方。}2
dLと、とろみ付けの卵黄4個を加える。沸騰させないよう気をつけて温め、火から外してバター100
gを加えて仕上げる。

\ldots{}\ldots{}仔牛、仔羊などの白身肉に合わせる。

\hypertarget{sauce-villeroy}{%
\subsubsection[ソース・ヴィルロワ]{\texorpdfstring{ソース・ヴィルロワ\footnote{ルイ15世の養育係を務めたヴィルロワ元帥
  François de Villeroi の名を冠したものとされる。}}{ソース・ヴィルロワ}}\label{sauce-villeroy}}

\frsub{Sauce Villeroy}

\index{そーす@ソース!ういるろわ@---・ヴィルロワ}
\index{ういるろわ@ヴィルロワ!そーす@ソース・---}
\index{sauce@sauce!villeroy@--- Villeroy}
\index{villeroy@Villeroy!sauce@Sauce ---}

\protect\hyperlink{sauce-allemande}{ソース・アルマンド}1
Lに、トリュフエッセンス大さじ4
杯と\protect\hyperlink{essences-diverses}{ハムのエッセンス}大さじ4杯を加える。

ヘラで混ぜながら強火にかけ、主素材となるものをソースに漬けて取り出したとき際に、全体をソースが覆うようになるような濃さまで煮詰めていく。

\hypertarget{nota-sauce-villeroy}{%
\subparagraph{【原注】}\label{nota-sauce-villeroy}}

このソースの唯一の使い途は、素材をこのソースで包み込んでから、イギリス式パン粉衣を付けて揚げるものだ。この方法で調理したものは常に「ヴィルロワ風」の名称となる。このソースは、古典料理において「隠れたソース」と呼ばれていたもののうちの典型例と言える。

\hypertarget{sauce-villeroy-soubisee}{%
\subsubsection{スビーズ入りソース・ヴィルロワ}\label{sauce-villeroy-soubisee}}

\frsub{Sauce Villeroy Soubisée}

\index{そーす@ソース!ういるろわすひーすいり@スビーズ入り---・ヴィルロワ}
\index{ういるろわ@ヴィルロワ!そーすすひーすいり@スビーズ入りソース・---}
\index{sauce@sauce!villeroy soubisee@--- Villeroy Soubisée}
\index{villeroy@Villeroy!sauce soubisee@Sauce --- Soubisée}

\protect\hyperlink{sauce-allemande}{ソース・アルマンド}に
\(\frac{1}{3}\)量の\protect\hyperlink{sauce-soubise}{スビーズのピュレ}\footnote{ソース・スビーズは濃度があるのでピュレと呼んだと考えていいだろう。クリ
  coulis は「やや水分の多いピュレ」と同義だからだ。}を加え、上記と同様に煮詰めて作る。

このソースを付ける素材や仕立てに合わせて、ソース1 Lあたり80〜100
gのトリュフのみじん切りを加えることもある。

\hypertarget{sauce-villeroy-tomatee}{%
\subsubsection{トマト入りソース・ヴィルロワ}\label{sauce-villeroy-tomatee}}

\frsub{Sauce Villeroy tomatée}

\index{そーす@ソース!ういるろわとまといり@トマト入り---・ヴィルロワ}
\index{ういるろわ@ヴィルロワ!そーすとまといり@トマト入りソース・---}
\index{sauce@sauce!villeroy tomatee@--- Villeroy tomatée}
\index{villeroy@Villeroy!sauce tomatee@Sauce --- tomatée}

\protect\hyperlink{sauce-villeroy}{標準的なソース・ヴィルロワ}とまったく作り方は同じだが、\protect\hyperlink{sauce-allemande}{ソース・アルマンド}の
\(\frac{1}{3}\)量の上等で真っ赤なトマトピュレを加えて作る。

\hypertarget{sauce-vin-blanc}{%
\subsubsection{白ワインソース}\label{sauce-vin-blanc}}

\frsub{Sauce vin blanc}

\index{そーす@ソース!しろわいん@白ワイン---}
\index{しろわいん@白ワイン!そーす@---ソース}
\index{わいん@ワイン!しろわいん@白ワイン!そーす@---ソース}
\index{sauce@sauce!vin blanc@--- vin blanc}
\index{vin@vin!sauce vin blanc@Sauce vin blanc}

このソースには以下の3種類の作り方がある。

\begin{enumerate}
\def\labelenumi{\arabic{enumi}.}
\item
  \protect\hyperlink{veloute-de-poisson}{魚料理用ヴルテ}1
  Lに、ソースを合わせる魚でとった\protect\hyperlink{fumet-de-poisson}{フュメ}2
  dLと、卵黄4個を加える。\(\frac{2}{3}\) 量まで煮詰め、バター150
  gを加える。この「白ワインソース」は、仕上げにオーブンに入れて照りをつける魚料理に合わせる。
\item
  良質の\protect\hyperlink{fumet-de-poisson}{魚のフュメ}1
  dLを半分にまで煮詰める。卵黄
  5個を加え、\protect\hyperlink{sauce-hollandaise}{オランデーズソース}を作る際の要領で、バター500
  gを加えてよく乳化させる。
\item
  卵黄5個を片手鍋\footnote{casserole カスロール。}に入れて溶きほぐし、軽く温めてやる。バター500
  gを加えて乳化させていく途中で、上等な\protect\hyperlink{fumet-de-poisson}{魚のフュメ}1
  dLを少しずつ加えていく\footnote{いずれの作り方にも白ワインが出てこないのは、それぞれで使われている\protect\hyperlink{fumet-de-poisson}{魚のフュメ}において\ruby{既}{すで}に白ワインを用いているから。}。
\end{enumerate}
\end{recette}