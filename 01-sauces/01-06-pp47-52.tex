\hypertarget{ux51b7ux88fdux30bdux30fcux30b9}{%
\section{冷製ソース}\label{ux51b7ux88fdux30bdux30fcux30b9}}

\hypertarget{sauces-froides}{%
\subsection{Sauces Froides}\label{sauces-froides}}

\index{sauce@sauce!sauces froides@sauces froides}
\index{そーす@ソース!れいせいそーす@冷製ソース}
\begin{recette}
\hypertarget{ux30a2ux30a4ux30e8ux30ea2-ux30d7ux30edux30f4ux30a1ux30f3ux30b9ux30d0ux30bfux30fc}{%
\subsubsection[アイヨリ /
プロヴァンスバター]{\texorpdfstring{アイヨリ\footnote{ailloliとも綴るが、
  ail(にんにく)+
  oil(油)の合成語。19世前半紀には既にアカデミーフランセージの辞書に収録されており、広く知られていたようだ。ブイヤベースに添えるルイユとよく似ているが、ルイユがカイエンヌを加えるのに対して、こちらはにんにくと油、塩、レモン汁と少々の水だけで作る。用途も、茹でた塩鱈やじゃがいも、茹で卵、アーティチョーク、さやいんげん、などに合わせることが多い。}
/
プロヴァンスバター}{アイヨリ / プロヴァンスバター}}\label{ux30a2ux30a4ux30e8ux30ea2-ux30d7ux30edux30f4ux30a1ux30f3ux30b9ux30d0ux30bfux30fc}}

\hypertarget{sauce-aioli}{%
\paragraph{Sauce Aïoli, ou Beurre de Provence}\label{sauce-aioli}}

\index{そーす@ソース!れいせい@冷製---!あいより@アイヨリ}
\index{そーす@ソース!れいせい@冷製---!ふろふあんすはたー@プロヴァンスバター}
\index{あいより@アイヨリ}
\index{ふろふあんす@プロヴァンス!ふろふあんすはたー@プロヴァンスバター}
\index{はたー@バター!ふろふあんすはたー@プロヴァンスバター}
\index{sauce@sauce!sauce froide@sauce froide!aioli@--- Aïoli}
\index{sauce@sauce!sauce froide@sauce froide!beurre de provence@Beurre de Provence}
\index{aioli@Aïoli!sauce@Sauce ---}
\index{provence@Provence!Beurre de Provence (Aïoli)}
\index{beurre@beurre!beurre de provence@Beurre de Provence (Aïoli)}

にんにく4片(30 g)を鉢\footnote{この種の作業には、大理石製のものが伝統的によく用いられる。。}に入れて細かくすり潰す。ここに生の卵黄1個、塩1つまみを加える。混ぜながら、2\undemi{}
dlの油\footnote{原書ではとくに言及されていないが、プロヴァンス地方ではオリーブオイルを用いることが一般的。}を初めは1滴ずつ加えていき、ソースがまとまりはじめたら糸を垂らすようにして加える。この作業は鉢に入れたままで、棒をはげしく動かして行なう。

攪拌する作業の途中、レモン1個分の搾り汁と冷水大さじ\undemi{}杯を少しずつ加えて、ソースが固くなり過ぎないようにしてやること。

\hypertarget{ux539fux6ce8}{%
\subparagraph{【原注】}\label{ux539fux6ce8}}

このアイヨリソースが分離してしまいそうな時は、卵黄をさらに1個足して、
マヨネーズと場合と同様に修正すること。

\maeaki

\hypertarget{ux30a2ux30f3ux30c0ux30ebux30b7ux30a25ux98a8ux30bdux30fcux30b9}{%
\subsubsection[アンダルシア風ソース]{\texorpdfstring{アンダルシア\footnote{いうまでもなくスペインのアンダルシア地方のことだが、トマトやオリーブオイル、チョリソなどこの地方を「想起」させる食材が使われている料理などがこの名称になっている傾向がある。ところが、トマトにしろオリーブオイルにしろアンダルシア地方特有というわけではなく、アンダルシアが産地として有名なチョリソくらいしか、料理名の根拠となり得るものはない。逆に言えば、アンダルシア地方の食文化との関係は、そこに用いられている食材以外にはないものと考えてもいい。料理名に付けられた地方名がとりたてて根拠や由来のないものであることを示す一例。}風ソース}{アンダルシア風ソース}}\label{ux30a2ux30f3ux30c0ux30ebux30b7ux30a25ux98a8ux30bdux30fcux30b9}}

\hypertarget{sauce-andalouse}{%
\paragraph{Sauce Andalouse}\label{sauce-andalouse}}

\index{そーす@ソース!れいせい@冷製---!あんたるしあふう@アンダルシア風---}
\index{あんたるしあ@アンダルシア!そーす@---風ソース}
\index{そーす@ソース!あんたるしあふう@アンダルシア風---}
\index{sauce@sauce!sauce froide@sauce froide!Andalouse@--- Andalouse}
\index{sauce@sauce!andalouse@--- Andalouse}
\index{andalous@Andalous(e)!sauce@Sauce Andalouse}

ごく固く仕上げた\protect\hyperlink{mayonnaise}{ソース・マヨネーズ}\troisquarts{}
Lに、上等な赤いトマトピュレ2\undemi{}dlを加える。小さなさいの目に切ったポワヴロン\footnote{Poivron
  いわゆる日本で青果として輸入されているパプリカ(肉厚の辛くないピーマン)とほぼ同じものだが、香辛料として用いられる粉末のパプリカと混同を避けるため、あえてフランス語をそのままカタカナに訳した。}75
gを仕上げに加える。

\maeaki

\hypertarget{ux30bdux30fcux30b9ux30dcux30d8ux30dfux30a2ux306eux5a18}{%
\subsubsection{ソース・ボヘミアの娘}\label{ux30bdux30fcux30b9ux30dcux30d8ux30dfux30a2ux306eux5a18}}

\hypertarget{sauce-bohemienne}{%
\paragraph[Sauce Bohémienne]{\texorpdfstring{Sauce Bohémienne\footnote{アイルランド出身の作曲家マイケル・ウィリアム・バルフェMichael
  William Balfe (1808〜1870)のオペラ\emph{The Bohemien
  Girl}『ボヘミアの少女』のフランス語版タイトル\href{https://archive.org/details/labohmiennegrand00balf}{\emph{La
  Bohémienne}}にちなんだものと言われている。この作品はもとはロンドンで1843年初演、1862年に四幕形式のフランス語版がパリのオペラ=コミック劇場で上演され、大ヒットしたという。この名を冠した料理はいくつかあるが、いずれもチェコのボヘミア地方とは何の関連性も認められないため、オペラの人気作品にあやかった料理名と考えるのが妥当だろう。}}{Sauce Bohémienne}}\label{sauce-bohemienne}}

\index{そーす@ソース!れいせい@冷製---!ほへみあのむすめ@---ボヘミアの娘}
\index{ほへみあ@ボヘミア!そーす@ソース・---の娘}
\index{そーす@ソース!ほへみあ@---・ボヘミアの娘}
\index{sauce@sauce!sauce froide@sauce froide!bohemienne@--- Bohémienne}
\index{sauce@sauce!bohemienne@--- Bohémienne}
\index{bohemien@bohémien(ne)!sauce@Sauce Bohémienne}

陶製の容器に、濃厚でよく冷やした\protect\hyperlink{sauce-bechamel}{ベシャメルソース}1\undemi{}
dlと卵黄4個、塩10 g、こしょう少々、ヴィネガー数滴を入れる。

泡立て器で全体をよく混ぜ、標準的なマヨネーズを作るのとまったく▼おな要領で、油1
Lとエストラゴンヴィネガー大さじ2杯程を加える。

\ldots{}\ldots{}仕上げに、マスタード大さじ1杯を加える。
\end{recette}