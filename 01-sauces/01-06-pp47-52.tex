\hypertarget{ux51b7ux88fdux30bdux30fcux30b9}{%
\section{冷製ソース}\label{ux51b7ux88fdux30bdux30fcux30b9}}

\frsec{Sauces Froides}

\index{sauce@sauce!sauces froides@Sauces froides}
\index{そーす@ソース!れいせい@冷製---}
\begin{recette}
\hypertarget{sauce-aioli}{%
\subsubsection[アイヨリ/プロヴァンスバター]{\texorpdfstring{アイヨリ/プロヴァンスバター\footnote{aïoli(アイヨリ)はailloliとも綴るが、
  ail(にんにく)+
  oil(油)の合成語。19世紀前半には既にアカデミーフランセージの辞書に収録されており、広く知られていたようだ。茹でた塩鱈やじゃがいも、茹で卵、アーティチョーク、さやいんげんなどに合わせることが多い。Beurre
  de
  Provence(ブールドプロヴォンス)の名称を持つレシピとしてもっとも古いと思われるものは、1758年刊マラン『コモス神の贈り物』の「鳩のプロヴァンスバター添え」だろう(t.2,
  pp.290-230)。ただし、このレシピは卵黄と油の乳化ソースではない。また、オリーブオイルそのものを
  beurre de Provence
  プロヴァンスバターと呼ぶことも多かった。実際、オリーブオイルは品質にもよるが5℃以下でほぼ固形化する。}}{アイヨリ/プロヴァンスバター}}\label{sauce-aioli}}

\frsub{Sauce Aïoli, ou Beurre de Provence}

\index{そーす@ソース!あいより@アイヨリ}
\index{そーす@ソース!ふろふあんすはたー@プロヴァンスバター}
\index{あいより@アイヨリ}
\index{ふろふあんす@プロヴァンス!はたー@---バター}
\index{はたー@バター!ふろふあんす@プロヴァンス---}
\index{sauce@sauce!aioli@--- Aïoli}
\index{sauce@sauce!beurre de provence@Beurre de Provence}
\index{aioli@Aïoli!sauce@Sauce ---}
\index{provence@Provence!Beurre de --- (Aïoli)}
\index{beurre@beurre!beurre de provence@--- de Provence (Aïoli)}

にんにく4片(30 g)を鉢\footnote{この種の作業には、大理石製のものが伝統的によく用いられる。。}に入れて細かくすり潰す。ここに生の卵黄1個、塩1つまみを加える。混ぜながら、2
\(\frac{1}{2}\) dLの油\footnote{原書ではとくに言及されていないが、プロヴァンス地方ではオリーブオイルを用いることが一般的。}を初めは1滴ずつ加えていき、ソースがまとまりはじめたら糸を垂らすようにして加える。この作業は鉢に入れたままで、棒をはげしく動かして行なう。

攪拌する作業の途中、レモン1個分の搾り汁と冷水大さじ
\(\frac{1}{2}\)杯を少しずつ加えて、ソースが固くなり過ぎないようにしてやること。

\hypertarget{nota-sauce-aioli}{%
\subparagraph{【原注】}\label{nota-sauce-aioli}}

このアイヨリソースが分離してしまいそうな時は、卵黄をさらに1個足して、マヨネーズと場合と同様に修正すること。

\hypertarget{sauce-andalouse}{%
\subsubsection[アンダルシア風ソース]{\texorpdfstring{アンダルシア風ソース\footnote{いうまでもなくスペインのアンダルシア地方のことだが、トマトやオリーブオイル、チョリソなどこの地方を「想起」させる食材が使われている料理などがこの名称になっている傾向がある。ところが、トマトにしろオリーブオイルにしろアンダルシア地方特有というわけではなく、アンダルシアが産地として有名なチョリソくらいしか、料理名の根拠となり得るものはない。逆に言えば、アンダルシア地方の食文化との関係は、そこに用いられている食材以外にはないものと考えてもいい。料理名に付けられた地方名がとりたてて根拠や由来のないものであることを示す一例。}}{アンダルシア風ソース}}\label{sauce-andalouse}}

\frsub{Sauce Andalouse}

\index{あんたるしあ@アンダルシア!そーす@---風ソース}
\index{そーす@ソース!あんたるしあふう@アンダルシア風---}
\index{sauce@sauce!andalouse@--- Andalouse}
\index{andalous@Andalous(e)!sauce@Sauce Andalouse}

ごく固く仕上げた\protect\hyperlink{mayonnaise}{ソース・マヨネーズ}\troisquarts{}
Lに、上等な赤いトマトピュレ2 \(\frac{1}{2}\)
dLを加える。小さなさいの目に切ったポワヴロン\footnote{Poivron
  いわゆる日本で青果として輸入されているパプリカ(肉厚の辛くないピーマン)とほぼ同じものだが、香辛料として用いられる粉末のパプリカと混同を避けるため、あえてフランス語をそのままカタカナに訳した。}75
gを仕上げに加える。

\hypertarget{sauce-bohemienne}{%
\subsubsection[ソース・ボヘミアの娘]{\texorpdfstring{ソース・ボヘミアの娘\footnote{アイルランド出身の作曲家マイケル・ウィリアム・バルフェMichael
  William Balfe (1808〜1870)のオペラ\emph{The Bohemien
  Girl}『ボヘミアの少女』のフランス語版タイトル\href{https://archive.org/details/labohmiennegrand00balf}{\emph{La
  Bohémienne}}『ラボエミエーヌ』にちなんだものと言われている。この作品はロンドンで
  1843年初演、1862年に四幕形式のフランス語版がパリのオペラ=コミック劇場で上演され、大ヒットしたという。この名を冠した料理はいくつかあるが、いずれもチェコのボヘミア地方とは何の関連性も認められないため、オペラの人気作品にあやかった料理名と考えるのが妥当だろう。}}{ソース・ボヘミアの娘}}\label{sauce-bohemienne}}

\frsub{Sauce Bohémienne}

\index{ほへみあ@ボヘミア!そーす@ソース・---の娘}
\index{そーす@ソース!ほへみあ@---・ボヘミアの娘}
\index{sauce@sauce!bohemienne@--- Bohémienne}
\index{bohemien@bohémien(ne)!sauce@Sauce Bohémienne}

陶製の容器に、濃厚でよく冷やした\protect\hyperlink{sauce-bechamel}{ベシャメルソース}1
\(\frac{1}{2}\) dLと卵黄4個、塩10
g、こしょう少々、ヴィネガー数滴を入れる。

泡立て器で全体をよく混ぜ、標準的なマヨネーズを作るのとまったく同じ要領で、油1
Lとエストラゴンヴィネガー大さじ2杯程を加える。

\ldots{}\ldots{}仕上げに、マスタード大さじ1杯を加える。

\hypertarget{sauce-chantilly-froide}{%
\subsubsection[ソース・シャンティイ]{\texorpdfstring{ソース・シャンティイ\footnote{パリ近郊の地名。詳しくはホワイト系派生ソースの\protect\hyperlink{sauce-chantilly}{ソース・シャンティイ}訳注参照。}}{ソース・シャンティイ}}\label{sauce-chantilly-froide}}

\frsub{Sauce Chantilly}

\index{しやんていい@シャンティイ!そーす@ソース・---(冷製)}
\index{そーす@ソース!しやんていい@---・シャンティイ}
\index{sauce@sauce!chantilly@--- Chantilly (froide)}
\index{chantilly@Chantilly!sauce@Sauce --- (froide)}

酸味付けにレモンを用いて、固く仕上げた\protect\hyperlink{mayonnaise}{ソース・マヨネーズ}\troisquarts{}
Lを用意しておく。提供直前に、ごく固く泡立てた生クリーム大さじ4杯\footnote{大さじ1杯
  = 15
  ccという概念にとらわれないよう注意。原文は、大きなスプーンで泡立てた生クリームをざっくりと4回加えるイメージで書かれている。本書における通常のソースの仕上がり量が約1
  Lであることを考慮すると、最低でも100 mL以上は加えることになるだろう。}を加える。その後、味を\ruby{調}{ととの}える。

\ldots{}\ldots{}もっぱら、アスパラガスの冷製、温製に添える。

\hypertarget{nota-sauce-chantilly-froide}{%
\subparagraph{【原注】}\label{nota-sauce-chantilly-froide}}

生クリームを加えるのは、このソースを使うまさにその時にすること。前もって加えておくと、ソースが分離してしまう恐れがあるので注意。

\hypertarget{sauce-genoise-froids}{%
\subsubsection[ジェノヴァ風ソース]{\texorpdfstring{ジェノヴァ風ソース\footnote{あまり明確な由来はないが、ジェノヴァが地中海に面した港町であり、このソースが魚料理用であるという点で一応の説明はつくだろう。}}{ジェノヴァ風ソース}}\label{sauce-genoise-froids}}

\frsub{Sauce Génoise}

\index{しえのうあふう@ジェノヴァ風!そーす@ソース・---(冷製)}
\index{そーす@ソース!しえのうあふう@ジェノヴァ風---}
\index{sauce@sauce!genoise@--- Génoise (froide)}
\index{genois@Génois(e)!sauce@Sauce ---e (froide)}

殻と皮を剥いたばかりのピスタチオ40 gと、松の実25
g、松の実がない場合はスイートアーモンド20
gを鉢に入れてよくすり潰し、冷めた\protect\hyperlink{sauce-bechamel}{ベシャメルソース}小さじ1杯程度を加えて練ってペースト状にする。これを目の細かい網で裏漉しする。陶製の容器に卵黄6個、塩1つまみ、こしょう少々を入れる。泡立て器でよく混ぜる。油1
Lと中位の大きさのレモン2個の搾り汁を少しずつ加えてよく混ぜて乳化させていく\footnote{明記されていないが、ソースをしっかりと乳化させるためには\protect\hyperlink{mayonnaise}{マヨネーズ}と同様に作業すること。}。仕上げにハーブのピュレ大さじ3杯を加える。これは、パセリの葉とセルフイユ、エストラゴン、時季が合えばサラダバーネットを同量ずつ用意し、強火で2分間下茹でしてから湯をきり、冷水にさらしてから水気を強く絞り、裏漉しして作っておく。

\ldots{}\ldots{}冷製の魚料理全般に合わせられる。

\hypertarget{sauce-gribiche}{%
\subsubsection[ソース・グリビッシュ]{\texorpdfstring{ソース・グリビッシュ\footnote{由来不明の語。ノルマンディ方言で「子どもを怖がらせるおばさん」の意味で用いられるということが分かっているのみ。19世紀後半以降に創案もしくは一般化したソースと思われる。本書初版には当然のように既に収録されており、その後の大きな異同もない。ただ、本書初版以前に出版された料理書においてこのソースのレシピはまだ見つかっていない。ファーヴルは1905年刊『料理および食品衛生事典』第二版で「ある種のレムラードにレストランで付けられた名称」と定義し、掲載しているレシピは本書初版のものと大差ないが、「ウスターシャソース少々も加える」となっているところが目を引く。また、1913年初版のプルーストの長編小説『失なわれた時を求めて』の「スワン家の方へ」冒頭において「彼(=スワン)を招いていない夕食会のために、ソース・グリビッシュやパイナップルのサラダのレシピが必要になるや、ためらいもなく探しに行かせたりするのだった」(p.18)。もしこの語り手の記述が正確であるなら、19世紀末には広く知られたものであったと考えるべきだが、小説の場合は必ずしも歴史的事実と符号するわけではないので注意が必要。}}{ソース・グリビッシュ}}\label{sauce-gribiche}}

\frsub{Sauce Gribiche}

\index{くりひつしゆ@グリビッシュ!そーす@ソース・---(冷製)}
\index{そーす@ソース!くりひつしゆ@---・グリビッシュ}
\index{sauce@sauce!gribiche@--- Gribiche (froide)}
\index{gribiche@gribiche!sauce@Sauce --- (froide)}

茹であがったばかりの固茹で卵の黄身6個を陶製のボウルに入れ、マスタード小さじ1杯、塩1つまみ強、こしょう適量を加えてよく練り、滑らかなペースト状にする。植物油
\(\frac{1}{2}\) Lとヴィネガー大さじ1
\(\frac{1}{2}\)杯を加えながらよく混ぜて乳化させる。仕上げに、コルニションとケイパーのみじん切り計
100
gと、パセリとセルフイユ、エストラゴンのみじん切りのミックスを大さじ1杯、短かめの千切りにした固茹で卵の白身3個分を加える。

\ldots{}\ldots{}冷製の魚料理に添えるのが一般的。

\hypertarget{sauce-groseilles-au-raifort}{%
\subsubsection{レフォール風味のソース・グロゼイユ}\label{sauce-groseilles-au-raifort}}

\frsub{Sauce Groseilles au Raifort}

\index{くろせいゆ@グロゼイユ!そーすれふおーる@レフォール風味のソース・---(冷製)}
\index{れふおーる@レフォール!そーすくろせいゆ@---風味のソース・グロゼイユ(冷製)}
\index{そーす@ソース!れふおーるふうみくろせいゆ@レフォール風味の---・グロゼイユ(冷製)}
\index{sauce@sauce!groseille@--- Groseilles au Raifort (froide)}
\index{raifort@raifort!sauce@Sauce Groseilles au --- (froide)}
\index{groseille@groseille!sauce@Sauce --- au Raifort (froide)}

ポルト酒1 dLにナツメグ、シナモン、塩、こしょう各1つまみを加え、を
\(\frac{2}{3}\)量まで煮詰める。溶かした\protect\hyperlink{}{グロゼイユのジュレ}4
dLと細かくすりおろしたレフォール大さじ2杯を加える。

(さまざまな用途に使える)

\hypertarget{sauce-italienne-froide}{%
\subsubsection[イタリア風ソース]{\texorpdfstring{イタリア風ソース\footnote{このソースも温製のイタリア風ソースと同様に名称にとくに由来などはないと思われる。}}{イタリア風ソース}}\label{sauce-italienne-froide}}

\frsub{Sauce Italienne}

\index{いたりあふう@イタリア風!そーすれいせい@---ソース(冷製)}
\index{そーす@ソース!いたりあふうれいせい@イタリア風---(冷製)}
\index{sauce@sauce!italienne@--- Italienne (froide)}
\index{italien@italien(ne)!sauce froide@Sauce ---ne (froide)}

仔牛の脳半分を、香草を効かせたクールブイヨンで火を通し、目の細かい網で裏漉しする。同量の牛あるいは羊の脳でもいい。

裏漉ししたピュレを陶製の器に入れ、泡立て器で滑らかになるまで混ぜる。卵黄5個と塩10
g、こしょう1つまみ強、油1
Lとレモン果汁1個分でマヨネーズを作り、そこの脳のピュレを加える。パセリのみじん切り大さじ1杯強を加えて仕上げる。

\ldots{}\ldots{}どんな冷製の肉料理にも合う。

\hypertarget{mayonnaise}{%
\subsubsection[マヨネーズ]{\texorpdfstring{マヨネーズ\footnote{このソース名の語源には諸説あり、未だ定説と呼べるものはない。
  Mayonnaise
  という綴りそのものは1806年のヴィアール『帝国料理の本』が初出で、Saumon
  à la Mayonnaise, Filet de Sole en Mayonnaise, Poulet en Mayonnaise
  の3つのレシピが掲載されている。そのうちのひとつ、サーモンのマヨネーズは、筒切りにしたサーモンを茹でて冷まし、ジュレを混ぜたマヨネーズをかける、という内容であり、ソースについてはマヨネーズの項を参照となっているが、どういうわけかこの本にマヨネーズそのもののレシピはない。また、「鶏のマヨネーズ仕立て」におけるソースはどう見てもこんにち我々が理解しているマヨネーズとまったく違い、鶏のゼラチン質を冷し固める要素として利用したものだ。同じヴィアールの改訂版ともいうべき『王国料理の本』(1822
  年)にはマヨネーズのレシピが掲載されている。興味深いことに「このソースにはいろいろな作り方がある。生の卵黄を使うもの、ジュレを使うもの、仔牛のグラスを使うものや仔牛の脳を使うもの」として、もっとも一般的な方法として生の卵黄を使う方法が示されている。生の卵黄に攪拌しながら少しずつ油を加えていき、固くなってきたらヴィネガー少々を加えてコシをきる、という方法であり、こんにち我々の知るマヨネーズに非常に近いものとなっている。また、1814年刊ボヴィリエ『調理技法』のソース・マヨネーズは、焼き物の器に油大さじ3〜4杯とエストラゴンヴィネガー2杯を入れる。細かく刻んだエストラゴン、エシャロット、サラダバーネットをたっぷり加え、ジュレ大さじ2、
  3杯を加える。ソースがまとまって、ポマード状になったら、味を調える
  (p.66)、というもの。ここでも卵黄と植物油の乳化ソースとはなっていない。綴りについては、カレームはmagner(マニェ)捏ねる、という意味の動詞から派生したものだとして、magnonnaiseもしくはmagnionnaiseと綴るべきだと『パリ風料理の本』で力説している。グリモ・ド・ラ・レニェールは中世フランス語で卵黄を意味するmoyeuの派生語としてmoyeunnaiseという綴りを使っている。そのほかフランス大西洋岸の地名バイヨンヌの形容詞bayonnais(バヨネ)が語源だという説もある。綴りの起源についてある程度有力視されているのは、1756年にリシュリュー公爵が当時イギリスに占領されていたミノルカ島のマオン港
  Mahon を奪取したことにちなんで、mahonnaise
  と名づけられたというもの。ところで、植物油ではなくバターを用いるものとして、\protect\hyperlink{sauce-hollandaise}{オランデーズソース}の原型ともいうべきレシピが1651年のラ・ヴァレーヌ『フランス料理の本』に、Asperges
  à la Sauce blanche
  アスパラガスのホワイトソース添え(p.238)として掲載されていることや、卵黄をポタージュやラグーのとろみ付けに使うことが古くから行なわれていたことなどを総合すると、良質のオリーブオイルやひまわり油を利用しやすい環境にある南フランスの方がどちらかといえば、卵黄と植物油の乳化作用を利用したソースの発達、普及しやすい環境にあったとも想像されよう。なお、この『料理の手引き』では卵黄のみを用いたレシピとなっているが、全卵を用いる場合もある。日本の市販品でも卵黄のみを使うメーカーと全卵を使用しているメーカーが混在している。なお、マヨネーズの仕上がりは、卵黄のみか全卵を用いるかという問題もあるが、どのような植物油を使うかにも大きく左右されるので注意。}}{マヨネーズ}}\label{mayonnaise}}

\frsub{Sauce Mayonnaise}

\index{まよねーす@マヨネーズ}
\index{そーす@ソース!まよねーす@マヨネーズ}
\index{sauce@sauce!mayonnaise@--- Mayonnaise}
\index{mayonnaise@mayonnaise!sauce@Sauce ---}

冷製ソースのほとんどはマヨネーズの派生ソースだから、\protect\hyperlink{sauce-espagnole}{ソース・エスパニョル}や\protect\hyperlink{veloute}{ヴルテ}と同様に基本ソースと見なされる。マヨネーズの作り方はきわめてシンプルだが、以下に述べるポイントはしっかり頭に入れておく必要がある。

\hypertarget{proportions-mayonnaise}{%
\subparagraph{材料と分量}\label{proportions-mayonnaise}}

\ldots{}\ldots{}卵黄6個、「からざ」は取り除いておくこと。油1 L。塩 10
g、白こしょう1 g、ヴィネガー大さじ1
\(\frac{1}{2}\)杯または、より白い仕上がりを目指す場合にはヴィネガーと同等量のレモン果汁。

\begin{enumerate}
\def\labelenumi{\arabic{enumi}.}
\item
  塩、こしょう、ヴィネガーまたはレモン果汁ほんの少々を加えて、泡立て器で卵黄を溶く。
\item
  油を最初は1滴ずつ加えていき、滑らかにまとまり始めたら、糸を垂らすようにして油を加えていく。
\item
  何回かに分けてヴィネガーもしくはレモン果汁を少量ずつ加え、コシを切ってやること\footnote{原文
    rompre le corps de la sauce
    ソースの粘り気をヴィネガーなどを加えることで「ゆるめる」あるいは「のばす」こと。ここでは「コシをきる」と訳したが、日本の調理用語なので注意。この作業は、一見乳化したように見えてもまだ乳化が不完全であるため、何回かに分けて濃度を下げ、攪拌を続けることで乳化を促進させ安定したものにするのが目的。}。
\item
  最後に熱い湯を大さじ3杯加える。これは乳化をしっかりさせて、作り置きしておく必要がある場合でもソースが分離しないようにするため。
\end{enumerate}

\hypertarget{nota-mayonnaise}{%
\subparagraph{【原注】}\label{nota-mayonnaise}}

\noindent 1.
卵黄だけの段階で塩こしょうをするとソースが分離してしまうのではないかというのは思い込みに過ぎず、実際に調理現場で作業している者はそう考えていない。むしろ、塩を卵黄の水分に溶かし込んでおいた方が、卵黄がまとまりやすくなることは科学的に証明されている\footnote{当時の知見であることに注意。}。

\begin{enumerate}
\def\labelenumi{\arabic{enumi}.}
\setcounter{enumi}{1}
\item
  マヨネーズを作る際に、氷の上に容器を置いて作業するのも間違いだ。事実はまったく逆で、冷気が伝わることがもっとも分離させてしまいやすい原因だ。寒い季節には、油はやや微温めか、せめて厨房の室温くらいにするべきだ\footnote{オリーブオイルのように、飽和温度が高い種類の油ではよく見られる現象。ひまわり油でさえも寒さで濁るので、この指摘は正しい。}。
\item
  マヨネーズが分離してしまう原因としては\ldots{}\ldots{}

  \begin{enumerate}
  \def\labelenumii{\arabic{enumii}.}
  \tightlist
  \item
    最初に油を入れ過ぎてしまうこと。
  \item
    冷え過ぎた油を使うこと
  \item
    卵黄の量に対して油の量が多過ぎること。卵黄1個につき油を乳化させることが出来るのは、作り置きするのには1
    \(\frac{3}{4}\) dL、すぐに使う場合でも2 dLが限度\footnote{卵黄の乳化能力は含まれているレシチンの量で決まるので理論上はもっと大量の油を乳化することが可能。風味や仕上がりを考慮に入れて、この数字はあくまでも目安と考えたほうがいい。}。
  \end{enumerate}
\end{enumerate}

\hypertarget{mayonnaise-collee}{%
\subsubsection{コーティング用マヨネーズ}\label{mayonnaise-collee}}

\frsub{Sauce Mayonnaise collée}

\index{まよねーす@マヨネーズ!こーていんくよう@コーティング用---}
\index{そーす@ソース!こーていんくようまよねーす@コーティング用マヨネーズ}
\index{sauce@sauce!mayonnaise collee@--- Mayonnaise collée}
\index{mayonnaise@mayonnaise!sauce collée@Sauce --- collée}

コーティング用マヨネーズは、マヨネーズ7 dLに溶かしたジュレ3
dLを混ぜ込んだもの。野菜サラダをあえるのに使う他、\protect\hyperlink{}{「ロシア風」ショフロワ}の素材を覆うのにも使う。

\hypertarget{nota-mayonnaise-collee}{%
\subparagraph{【原注】}\label{nota-mayonnaise-collee}}

\protect\hyperlink{sauce-chaud-froid-maigre}{魚料理用ソース・ショフロワ}の項で述べたように、このコーティング用マヨネーズの代わりに魚料理用ソース・ショフロワを使う方がいい。その方がコーティング用マヨネーズを使う場合よりも風味も見た目もよくなる。というのも、コーティング用マヨネーズは、冷気によってゼラチンが固まるとともに収縮し、マヨネーズに圧力がかかるために、ソースで素材を覆った表面に油が浸み出してしまう\footnote{初版における原注は、「コーティング用マヨネーズで覆ったものは、数時間経つと、油の露で覆われたようになってしまうことがある。その原因は、冷気によってゼラチンが固まる際に収縮し、その結果マヨネーズに圧力がかかり、液体である油がソースを覆った表面に浸みだしてくることだ。これを避けるために、コーティング用マヨネーズはこんにちでは使われなくなっており、我々の場合だと、かなり以前から魚料理用ソース・ショフロワを用いている(p.163)」。第二版以降、多少の異同はあるが、ほぼ第四版の記述と同様。いずれにしても、ジュレ(親水性アミノ酸であるコラーゲンが主体)を加えたことで、親水基と疎水基を併せ持つ卵黄レシチンの乳化作用が崩れてマヨネーズが分離した結果だということには気付いていなかったと思われる。}。こういうふうに浸みが出ることを防ぐには、どんな場合でも、このコーティング用マヨネーズではなく魚料理用ソース・ショフロワを用いることをお勧めする\footnote{この『料理の手引き』ではジュレを加えたマヨネーズの使用に否定的だが、カレーム『19世紀フランス料理』ではSauce
  Magnonaiseとして、まず最初にジュレを加えるレシピが掲載されている。概略を示すと、氷の上に置いた陶製の容器に卵黄2個、塩、白こしょう少々、エストラゴンヴィネガー少々を入れる。木のさじで素早くかき混ぜる。まとまってきたら、エクス産の油大さじ1杯とヴィネガー少々を、少しずつ加えていく。容器の壁に叩きつけるようにしてソースを泡立てていく。この作業でマニョネーズの白さが決まるという。また、油をごく少量ずつ加えていくことを強調している。粘度が出て滑らかになったら、最後に油をグラス二杯(≒2
  dL)と\textbf{アスピック用ジュレ}をグラス
  \(\frac{1}{2}\)杯、エストラゴンヴィネガー適量を加えて仕上げる、というもの(t.3,
  p.132.
  強調は引用者による)。また、カレームは卵黄に含まれるレシチンによって乳化作用が起きることを経験的にさえも理解していなかったようであり、卵黄を用いないマニョネーズのレシピも掲載されている。なかでも特徴的なのは、「ジュレ入りの白いマニョネーズ」のレシピで、これは氷の上に鍋を置き、大きなレードル2杯の白いジュレと同量の油、レードル1杯のヴィネガー、塩、こしょうを入れて卵白用の泡立て器でよく混ぜ、途中何回かレモン果汁を少しずつ加えて白く仕上げるようにする、というもの(\emph{ibid}.,
  p.133)。とりわけ舞踏会や格式ある大規模な宴席で魚のフィレや鶏のアスピックを飾るのに適していると述べている。}。少なくとも、そうするのが一般的になりつつある。

\hypertarget{mayonnaise-fouette-a-la-russe}{%
\subsubsection{ロシア風ホイップマヨネーズ}\label{mayonnaise-fouette-a-la-russe}}

\frsub{Sauce Mayonnaise fouettée, à la Russe}

\index{まよねーす@マヨネーズ!ろしあふうほいつふ@ロシア風ホイップ---}
\index{そーす@ソース!ろしあふうほいつふまよねーす@ロシア風ホイップマヨネーズ}
\index{ろしあふう@ロシア風!ほいつふまよねーす@---ホイップマヨネーズ}
\index{sauce@sauce!mayonnaise fouettee russe@--- Mayonnaise fouettée à la Russe}
\index{mayonnaise@mayonnaise!sauce fouettée russe@Sauce --- fouettée à la Russe}
\index{russe@russe!sauce mayonnaise fouettee@Sauce Mayonnaise fouettée à la ---}

陶製かホーローの容器に、溶かしたジュレ4 dLとマヨネーズ3
dL、エストラゴンヴィネガー大さじ1杯、おろしてさらに細かく刻んだレフォール\footnote{ホースラディッシュ、西洋わさび。}大さじ
1杯を入れる。

全体を混ぜ、容器を氷の上に置いて泡立て器でホイップする。ムース状になり、軽く固まり始めるまで、つまりこのソースを使うのに充分な流動性がある状態のところで作業をやめる\footnote{分量比率を考えると、構造的には前項の注で言及したカレームのジュレを主体としたマニョネーズに近いものと思われる。}。\ldots{}\ldots{}主に、野菜のサラダを型に詰めて固めるのに用いる。

\hypertarget{mayonnaises-divierses}{%
\subsubsection{マヨネーズのバリエーション}\label{mayonnaises-divierses}}

\frsub{Sauce Mayonnaise diverses}

\index{まよねーす@マヨネーズ!はりえーしよん@---のバリエーション}
\index{そーす@ソース!まよねーすのはりえーしよん@マヨネーズのバリエーション}
\index{sauce@sauce!mayonnaises dieverses@---s Mayonnaises diverses}
\index{mayonnaise@mayonnaise!sauces diverses@Sauces --- diverses}

オードブルや冷製料理に合わせるのに、大型甲殻類\footnote{homard
  オマール、langouste ラングースト(≒伊勢エビ)など。}およびエクルヴィス\footnote{ざりがにのこと。詳しくは\protect\hyperlink{sauce-bavaroise}{バイエルン風ソース}訳注参照。}の卵やクリーム状の部分を用いたり、クルヴェット\footnote{小海老のこと。詳しくは\protect\hyperlink{sauce-aux-crevettes}{ソース・クルヴェット}訳注参照。}、キャビア、アンチョビなどを加えることでマヨネーズにバリエーションを付けることが出来る。

上記の材料のいずれかをすり潰してから少量のマヨネーズを加えてピュレ状にして布で漉す。これを適量のマヨネーズに混ぜ合わせればよい。

\hypertarget{sauce-mousquetaire}{%
\subsubsection[ソース・ムスクテール]{\texorpdfstring{ソース・ムスクテール\footnote{マスケット銃兵、近衛騎兵、の意。日本でも子どもむけに翻案されたもので有名な19世紀のアレクサンドル・デュマ(ペール)の小説
  \emph{Les Trois Mousquetaires} 『三銃士』の「銃士」がこれに相当する。}}{ソース・ムスクテール}}\label{sauce-mousquetaire}}

\frsub{Sauce Mousquetaire}

\index{むすくてーる@ムスクテール!そーす@ソース・---(冷製)}
\index{そーす@ソース!むすくてーる@---・ムスクテール(冷製)}
\index{sauce@sauce!mousquetaire@--- Mousquetaire (froide)}
\index{mousquetaire@mousquetaire!sauce@Sauce --- (froide)}

マヨネーズ1 Lに以下を加える。ごく細かいみじん切りにしたエシャロット80 g
を白ワイン1 \(\frac{1}{2}\)
dLに加えてほとんど煮詰めたもの。溶かした\protect\hyperlink{glace-de-viande}{グラスドヴィアンド}大さじ3杯、シブレット\footnote{チャイヴ。アサツキとも訳されることがあるが、日本のものとは風味が異なるので注意。}を細かく刻んだもの大さじ1杯強。カイエンヌごく少量かミルで挽いたこしょう少々で風味を引き締める。

\ldots{}\ldots{}羊、牛肉の冷製料理に添える。

\hypertarget{sauce-moutarde-a-la-creme}{%
\subsubsection{生クリーム入りソース・ムタルド}\label{sauce-moutarde-a-la-creme}}

\frsub{Sauce moutarde à la crème}

\index{そーす@ソース!むたるとなまくりーむいり@生クリーム入り---・ムタルド(冷製)}
\index{むたると@ムタルド(マスタード)!そーすなまくりーむいり@生クリーム入りソース・---(冷製)}
\index{ますたーと@マスタード(ムタルド)!そーすなまくりーむいり@生クリーム入りソース・ムタルド(冷製)}
\index{sauce@sauce!moutarde creme@--- moutarde à la crème (froide)}
\index{moutarde@moutarde!sauce creme@Sauce --- à la crème (froide)}

陶製の容器にマスタード大さじ3杯と塩1つまみ、こしょう少々とレモン果汁少々を入れて混ぜ合わせる。ここに少しずつ、マヨネーズを作る要領で、ごく新鮮なクレーム・エペス\footnote{乳酸醗酵させた、とても濃度のある生クリーム。}約2
dLを加える。

\ldots{}\ldots{}オードブル用。

\hypertarget{sauce-raifort-aux-noix}{%
\subsubsection{くるみ入りソース・レフォール}\label{sauce-raifort-aux-noix}}

\frsub{Sauce Raifort aux noix}

\index{そーす@ソース!くるみいりれふおーる@くるみ入り---・レフォール(冷製)}
\index{れふおーる@レフォール(ホーシュラディッシュ)!くるみいりそーす@くるみ入りソース・---(冷製)}
\index{くるみ@くるみ!くるみいりれふおーる@---入りソース・レフォール(冷製)}
\index{sauce@sauce!raifort noix@--- Raifort aux noix (froide)}
\index{raifort@raifort!sauce noix@Sauce --- aux noix (froide)}
\index{noix@noix!sauce raifort@Sauce Raifort aux --- (froide)}

陶製の器に、おろしたレフォール250 gと皮を剥いて刻んだくるみ250 g、塩5
g、砂糖15 g、クレーム・エペス3 dLを入れて混ぜ合わせる。

\ldots{}\ldots{}オンブルシュヴァリエ\footnote{サケ科の淡水魚。体長20〜30
  cmのものが多く、最大で70
  cmを越えるものもいるという。日本の岩魚に近い。フランスではアルプスのドイツおよびイタリアとの国境付近に生息するが、現代では養殖も多いという。}の冷製用。

\hypertarget{sauce-ravigote-froide}{%
\subsubsection[ソース・ラヴィゴット/ヴィネグレット]{\texorpdfstring{ソース・ラヴィゴット/ヴィネグレット\footnote{ラヴィゴットの意味などについてはホワイト系派生ソースの\protect\hyperlink{sauce-ravigote}{ソース・ラヴィゴット}参照。現代フランス語の
  vinaigrette(ヴィネグレット)はいわゆる「ドレッシング」を指す。語源的にはヴィネガーを意味するvinaigre(ヴィネーグル)に縮小辞
  -ette
  を付けたもの。ヴィネグレットという名称のレシピとしてもっとも古いのは14世紀に成立したとされる「タイユヴァン」のもので、いわゆる「ヴァチカン写本」に収録されており、\ul{Potaige Lyans}「とろみを付けた煮物」に分類されている。概要を示すと、menue-hasteムニュアット(豚の脾臓およびレバー半分と腎臓)をローストする。火を通しすぎないよう注意。それを切り分けて、鍋にラード、輪切りにした玉ねぎとともに入れて、炭火にかけ、よく混ぜながら火を通す。全体によく火が通ったら、牛のブイヨンとワインを注いで沸かす。マニゲット、サフランなどを鉢でよくすり潰したらヴィネガーでのばして加え、再度沸騰させる。全体にとろみがあって茶色に仕上げる、というもの(p.222)。これをほぼ書き写したと思われる14世紀末に書かれた『ル・メナジエ・ド・パリ』のレシピでは、肉の下処理としてよく洗ってから湯通しすること、とろみ付けの要素としてこんがり焼いたパンを香辛料とともにすり潰してワインとヴィネガーで溶く、という指示が追加されている。また、こんがり焼いたパンを使わずに茶色に仕上げられるわけがない云々という『ル・メナジエ・ド・パリ』の筆者自身の感想も記されている。15世紀に書かれたシカールの『料理について』でも豚のレバーを焼いてから煮込みヴィネガーを加えるもので、細部は違うが基本的に似たものであり、中世においては豚レバーを煮込んでヴィネガーで味付けしたもの、ということになる。これが変化したと思われるのは17世紀。1693年刊マシアロ『宮廷およびブルジョワ料理の本』にはBoeuf,
  Vinaigretteというレシピがあり、牛肉に背脂を刺して塩茹でして冷まし、ヴィネガーをひと垂らししてレモンのスライスを添えるというとても単純なもの。ところが、1694年のアカデミーフランセーズの辞書には既に「ヴィネガー、油、塩、こしょう、パセリ、シブール{[}葱{]}」で作る冷製ソース」という定義がなされている。こんにち我々がイメージするヴィネグレットの定義にほぼ近い。おおむね17世紀以降、とりわけ後半にヴィネガーと油、塩を合わせた冷製ソースというコンセンサスが形成されたと想像される。}}{ソース・ラヴィゴット/ヴィネグレット}}\label{sauce-ravigote-froide}}

\frsub{Sauce Ravigote, ou Vinaigrette}

\index{そーす@ソース!らういこつと@---・ラヴィゴット(冷製)}
\index{らういこつと@ラヴィゴット!そーす@ソース・---(冷製)}
\index{ういねくれつと@ヴィネグレット ⇒ソース・ラヴィゴット(冷製)}
\index{sauce@sauce!ravigotte froide@--- Ravigote, ou Vinaigrette (froide)}
\index{ravigote@ravigote!sauce@Sauce --- , ou vinaigrette (froide)}
\index{vinaigrette@vinaigrette ⇒ sauce ravigote (froide)}

\hypertarget{ux6750ux6599}{%
\subparagraph{材料}\label{ux6750ux6599}}

\ldots{}\ldots{}油5 dL、ヴィネガー2
dL、小さめのケイパー小さじ2杯、パセリ50
g、セルフイユとエストラゴン、シブレットを刻んだもの40
g、細かくみじん切りにした玉ねぎ70 g、塩4 g、こしょう1
g。以上をよく混ぜ合わせる。

\ldots{}\ldots{}仔牛の頭や足、羊の足などに合わせる。

\hypertarget{sauce-remoulade}{%
\subsubsection[ソース・レムラード]{\texorpdfstring{ソース・レムラード\footnote{ソース名としての初出はおそらくムノン『ブルジョワ屋敷勤めの女性料理人のための本』(1734)におけるSauce
  à la rémolade
  {[}sic.{]}だろう。レシピの概要は、エシャロット、パセリ、シブール、にんにく1片、アンチョビ、ケイパー、いずれもごく細かく刻んで鍋に入れ、塩、粗挽きこしょうを加え、マスタード少々と油、ヴィネガーでのばす、というもの。つまり、乳化ソースであるマヨネーズをベースにした本書のレムラードと、乳化させないという点が異なるのみで、基本的なところは共通していると見ていい。ヴィアール『帝国料理の本』第7版(1812年)にはRémouladeの綴りで、緑色のレムラード、レムラード、インド風レムラードと3種のレシピが掲載されている(この版にはまだマヨネーズのレシピは掲載されていない)。このうちのレムラードのレシピの概要は、グラス1杯のマスタードを器に入れ、エシャロットのみじん切り少々と香草少々を加える。油を大さじ6〜
  7杯、ヴィネガー大さじ3〜4杯、塩、粗挽きこしょうを加える。これらをよく混ぜ合わせ、生の卵黄2個を加えてさらによく混ぜる。ソースがよくまとまるように気をつけてしっかり綷。やや濃い仕上がりにする、というもの(p.53)。手順的にはやや異なるが、卵黄を用いて乳化させようとしていることがわかる。緑のレムラードも生の卵黄を用いるなど、香草をすり潰すことと、ほうれんそうの緑の色素を用いる以外はレムラードと同様。なお、インド風レムラードの場合は固茹で卵の卵黄10個をよくすり潰して大さじ8杯の油を加えてさらによく混ぜる。唐辛子とターメリックの粉末、塩、こしょう、ヴィネガーを加える。出来るだけ粘りが出るようにする。これを布で漉して供する(id.)。カレームに至るとさらにレシピは洗練されたものとなり、Sauce
  Rémoulade à la
  Ravigote(ソース・レムラード・アラ・ラヴィゴット)では、セルフイユとエストラゴン、サラダバーネット、シブレットを茹がいて水にさらした後に水気を搾り、固茹で卵の卵黄を加えてよくすり潰し、塩、こしょう、ナツメグで調味して、上等のマスタードを加える。ここにエクス産の油とエストラゴンヴィネガーを少しずつ加えていく。最後に布で漉す(t.1,
  p.135)というもの。いずれにしてもマヨネーズを基本ソースとして展開するという『料理の手引き』の発想、体系化にいたるまで100年近くを要したことになる。}}{ソース・レムラード}}\label{sauce-remoulade}}

\frsub{Sauce Rémoulade}

\index{れむらーと@レムラード!そーす@ソース・---(冷製)}
\index{そーす@ソース!れむらーと@---・レムラード(冷製)}
\index{sauce@sauce!remoulade@--- Rémoulade (froide)}
\index{remoulade@rémoulade!sauce@Sauce --- (froide)}

\protect\hyperlink{mayonnaise}{マヨネーズ}1
Lに以下のものを加える。マスタード大さじ 1
\(\frac{1}{2}\)杯。コルニション100とケイパー50gを細かく刻んで、圧して余分な水気を絞ったもの。パセリ、セルフイユ、エストラゴンのみじん切り大さじ1
杯。アンチョビエッセンス大さじ \(\frac{1}{2}\)杯。

\hypertarget{sauce-russe-froide}{%
\subsubsection{ロシア風ソース}\label{sauce-russe-froide}}

\frsub{Sauce Russe}

\index{ろしあふう@ロシア風!そーすれいせい@---ソース(冷製)}
\index{そーす@ソース!ろしあふうれいせい@ロシア風---(冷製)}
\index{sauce@sauce!russe@--- Russe (froide)}
\index{russe@russe!sauce froide@Sauce --- (froide)}

鉢に、オマール\footnote{homard ロブスター。}かラングースト\footnote{langouste
  ≒ 伊勢エビ。}の胴のクリーム状の部分100 gとキャビア100 g\footnote{チョウザメの卵の塩蔵品のことだが、「高級」とされる順に、beluga
  (ベルガ)、osciètre, ossetra(オシエートル、オセトラ)、
  sevruga(セヴルガ)の種類がある(ここで示した読みがなはフランス語風のもの)。}、マヨネーズ大さじ2〜3杯を加えてよくすり潰す。これを目の細かい漉し器で裏漉しする。こうして出来たピュレに、マヨネーズ
\(\frac{3}{4}\)
Lを加える。大さじ1杯強のマスタードと、同量のダービーソース\footnote{初版では原注として、風味付けにマスタードを加えることを示唆しているのみ。第二版では「マスタードとウスターシャソースを各大さじ1杯強」、第三版では「マスタードとエスコフィエソースを大さじ1杯強」と変遷している。なお、ダービーソースDerby
  Sauce
  の1946年の広告には、このブランド名でバーベキューソース、ステーキソース、ウスターシャソース、ホットソース、チャプスイソースのラインナップが記されている。現実問題として、もし加えるとするならリー\&ペリンのようなウスターシャソースということになろうか。}を加えて仕上げる。

\ldots{}\ldots{}魚および甲殻類の冷製料理に添える。

\hypertarget{sauce-tartare}{%
\subsubsection[タルタルソース]{\texorpdfstring{タルタルソース\footnote{タルタル(タタール)=フランス人から見て東方の蛮族、というイメージで語られがちだが、カレーム『19世紀フランス料理』にあるSauce
  Rémoulade à la Mogol
  (Mongoleの誤植と思われる)「モンゴル風ソース・レムラード」およびSauce
  à la
  Tartare「タルタル風ソース」のレシピを見るかぎり、誤解という可能性も感じられる。前者は固茹で卵の卵黄に塩、こしょう、ナツメグ、カイエンヌ、砂糖、油、エストラゴンヴィネガーを合わせてピュレ状にして布で漉し、サフランを煎じた汁で美しい黄色に染め、刻んだシブレットを加えて仕上げるというもの。後者はソース・アルマンドとマスタード同量に生の卵黄2個を加え、塩、こしょう、ナツメグで調味してエクス産の油レードル2杯分とレードル
  \(\frac{1}{2}\)
  杯のエストラゴンヴィネガーを少しずつ加えながら混ぜていく。みじん切りにして下茹でしたエシャロット少々とにんにく少々、エストラゴンとセルフイユのみじん切りを大さじ1杯加える、というもの(pp.137-138)。少なくともこれらのレシピにおいて、タルタルすなわち野蛮、というニュアンスを見出すことは出来ないだろう。なお、Steak
  tartareタルタルステーキのレシピは本書には掲載されておらず、1938年の『ラルース・ガストロノミック』初版が初出と思われる(p.1019)。}}{タルタルソース}}\label{sauce-tartare}}

\frsub{Sauce Tartare}

\index{たるたる@タルタル!そーすれいせい@---ソース(冷製)}
\index{そーす@ソース!たるたるれいせい@タルタル---(冷製)}
\index{sauce@sauce!tartare@--- Tartare (froide)}
\index{tartare@tartare!sauce froide@Sauce --- (froide)}

固茹で卵の黄身8個をすり潰して滑らかになるまでよく練る。塩、挽きたてのこしょう各1つまみ強で味付けする。油1
Lとヴィネガー大さじ2杯を加えながらソースを立てていく\footnote{明記されていないが、\protect\hyperlink{mayonnaise}{マヨネーズ}や\protect\hyperlink{sauce-gribiche}{ソース・グリビッシュ}と同様に作業すること。}。若どりの玉ねぎ\footnote{いわゆる「オニオンヌーヴォー」だが、日本でこの名称で流通しているものは黄色系の品種が多いのに対し、フランスでは白系品種(oignon
  blanc オニョンブロン)が多く、風味が異なることに注意。}の葉またはシブレット20
gをすり潰してマヨネーズ大さじ2杯でのばし、目の細かい網で裏漉ししたものを加えて仕上げる。

\ldots{}\ldots{}このソースは、冷製の家禽や肉料理、魚料理、甲殻類いずれにも合う。また、「ディアーブル(悪魔風)」仕立ての肉料理、鶏料理にも用いられる。

\hypertarget{sauce-verte}{%
\subsubsection[ソース・ヴェルト]{\texorpdfstring{ソース・ヴェルト\footnote{緑のソース、の意。この名称のソースは中世からある。このレシピではほうれんそうとクレソンが主体になっているが、時代とともにその材料には変遷がある。中世においては、麦の若葉をすり潰して用いるレシピが多かった。}}{ソース・ヴェルト}}\label{sauce-verte}}

\frsub{Sauce Verte}

\index{みとり@緑 ⇒ ヴェール/ヴェルト!ソース・ヴェルト(冷製)}
\index{うえーる@ヴェール/ヴェルト!そーす@ソース・ヴェルト(冷製)}
\index{そーす@ソース!うえると@---・ヴェルト(冷製)}
\index{sauce@sauce!verte@--- Verte}
\index{vert@vert(e)!sauce@Sauce ---e}

ほうれんそうの葉\footnote{日本では、ほうれんそうを葉のみではなく葉軸とともに利用するのが一般的だが、伝統的なフランス料理において葉軸は使われないのが普通。そもそも日本のほうれんそうは密植して葉が立つように仕立てて比較的若どりするのに対して、ヨーロッパ品種のほうれんそうは株間を充分にとってロゼッタ状に葉が広がるように栽培するのが伝統的な手法。この場合、葉は肉厚に仕上がるが、葉軸は太くて固いため可食部と見なされなかった。昔のフランスの八百屋の店先では軸を切り捨てる作業風景がよく見られたという。現代では機械収穫に適した立性の品種が増えており、専用の大型機械で株元近くから切り取り、自動的に軸をある程度除去して併走する巨大なコンテナに移すという収穫方法が普及しており、量産品のピュレなどに使用されている。}50
gとクレソンの葉50 g、パセリの葉とセルフイユ、エストラゴンを同量ずつ計50
gを、沸騰した湯に投入し、強火で5分間茹でる。水気をきり、手早く冷水にさらす。しっかりと圧し絞って水気をきり、鉢に入れてすり潰す。これをトーション\footnote{綿などの天然素材で出来た調理場及びホール業務に用いられる布。サイズは50〜55
  cm×70〜80 cmのものが多い。}でくるんできつく絞り、葉の濃い汁を1
dL搾りだす。

固く立てて風味付けをした\protect\hyperlink{mayonnaise}{マヨネーズ}9
dLにこの緑の汁を加える。

\ldots{}\ldots{}冷製の魚料理や甲殻類に合わせる。

\hypertarget{sauce-vincent}{%
\subsubsection[ソース・ヴァンサン]{\texorpdfstring{ソース・ヴァンサン\footnote{18世紀フランスを代表する料理人のひとり、Vincent
  La
  Chapelleヴァンサン・ラシャペル(1690または1703〜1745)の名を冠したソース。チェスターフィールド伯フィリップ・スタンホープに仕えていた頃に三巻からなる『近代料理』\emph{Modern
  Cook}英語版を1733年に上梓。そのフランス語版(全4巻)は1835年に\emph{Le
  Cuisinier
  moderne}のタイトルでアムステルダムで刊行。その後、全5巻からなる第二版を1742年に自費で出版した。}}{ソース・ヴァンサン}}\label{sauce-vincent}}

\frsub{Sauce Vencent}

\index{うあんさん@ヴァンサン!そーす@ソース・---(冷製)}
\index{そーす@ソース!うあんさん@---・ヴァンサン(冷製)}
\index{sauce@sauce!vincent@--- Vincent (froide)}
\index{vincent@Vincent!sauce@Sauce --- (froide)}

\hypertarget{sauce-vincent-1}{%
\subparagraph{作り方(1)}\label{sauce-vincent-1}}

\ldots{}\ldots{}オゼイユ\footnote{タデ科の葉菜。日本語ではソレルとも。日本のスカンポに近いが、オゼイユは野菜として品種の選抜育成が長期にわたって行なわれたことに留意。}の葉とパセリの葉、セルフイユ、エストラゴン、シブレット、サラダバーネット\footnote{pimprenelle
  パンプルネル。}のごく若い葉をきっちり同量ずつ、計100 g、クレソンの葉60
gとほうれんそうの葉60 gを沸騰した湯で強火で2〜3分間茹がく。

湯をきって、冷水にさらす。しっかり水分を圧し絞って、鉢\footnote{伝統的には大理石製の鉢がこの種の作業には用いられた。}に入れる。茹であがったばかりの固茹で卵の黄身6個を加えて滑かになるまですり潰す。

これを布で漉し\footnote{このように濃度のあるものを布で漉す方法については\protect\hyperlink{veloute}{ヴルテ}訳注参照。}、陶製の容器に移す。塩1つまみ強とこしょう適量、生の卵黄5個を加える。油8
dLとヴィネガー適量を加えながら混ぜ、滑らかに乳化させる。

風味付けにダービーソース\footnote{\protect\hyperlink{sauce-russe-froide}{ロシア風ソース}訳注参照。}大さじ1杯を加えて仕上げる。

\hypertarget{sauce-vincent-2}{%
\subparagraph{作り方(2)}\label{sauce-vincent-2}}

\ldots{}\ldots{}作り方(1)の香草と葉菜のピュレを作るところまでは同じ。

これに\protect\hyperlink{mayonnaise}{マヨネーズ}を加えて、同様に仕上げる。

\ldots{}\ldots{}冷製の魚料理、甲殻類にとりわけ合う。

\hypertarget{nota-sauce-vincent}{%
\subparagraph{【原注】}\label{nota-sauce-vincent}}

このソースは18世紀の偉大な料理人のひとり、ヴァンサン・ラシャペルが考案したもの\footnote{\protect\hyperlink{mayonnaise}{ソース・マヨネーズ}の訳注において述べたように、卵黄と植物油をベースとした乳化ソースとしてのマヨネーズの起源は判然としないところが多いが、19世紀初頭のヴィアールやカレームの記述を読むかぎりにおいて、卵黄レシチンによる油と水分の乳化作用については経験レベルでさえはっきりとは認識されていなかった。このソースあるいはこれに相当するレシピがヴァンサン・ラシャペルの著書に掲載されていないこと、ヴァンラン・ラシャペルがレストランの店主ではなく貴族に仕えていた料理人、メートルドテルであったことを考慮すると、このソースの考案者が彼である可能性も、自身の名をソース名に冠した可能性もきわめて低い。もっとも、香草の扱いを得意としていたのは事実のようで、Sauce
  en
  Ravigote(ソース・オン・ラヴィゴット)だけでも5種のレシピが掲載されている。香草と葉菜を茹でてすり潰したピュレをこれらのソースで使用していることから、後世にこの名称が付いた、あるいはこのソースの最大のポイントがヴァンサン・ラシャペルを思わせる香草のピュレだと考えるのが妥当だろう。}。

\hypertarget{sauce-suedoise}{%
\subsubsection[スウェーデン風ソース]{\texorpdfstring{スウェーデン風ソース\footnote{基本的にソース名はアルファベット順に掲載されているのだが、このソースだけが後からとって付けたように末尾にある。実際、このレシピは第二版から掲載となっているが、ある程度組版が進んだ段階で急遽追加されたのだろうか。なお、1907年の英語版には掲載されていない。原注の最後「このソースはマスタードで風味付けしてもいい」は第四版で追加されたものだが、他は第二版からまったく異同がなく、掲載順も変化していないのはいささか不思議なところ。}}{スウェーデン風ソース}}\label{sauce-suedoise}}

\frsub{Sauce Suédoise}

\index{すうえーてんふう@スウェーデン風!そーすれいせい@---ソース(冷製)}
\index{そーす@ソース!すうえーてんふうれいせい@スウェーデン風---(冷製)}
\index{sauce@sauce!suedoise@--- Suédoise (froide)}
\index{suedois@suédois(e)!sauce froide@Sauce ---e (froide)}

酸味のある固いリンゴを薄切りにして鍋にしっかり蓋をして煮る。普通の果肉が甘いリンゴを使う場合にはレモン果汁数滴を加えること。リンゴを煮る際には、白ワインを大さじ数杯だけ加えればいい。リンゴを煮るというよりは蒸気の圧力で溶かすイメージ。

これを目の細かい網で裏漉しする。このリンゴのピュレを2 \(\frac{1}{2}\)
dLになるまで煮詰める。充分に冷ましてから、\protect\hyperlink{mayonnaise}{マヨネーズ}
\(\frac{3}{4}\)
Lを加える。風味付けにおろした(または細かく刻んだ)レフォール大さじ1
\(\frac{1}{2}\)杯を加えて仕上げる。

\ldots{}\ldots{}このソースはとりわけ豚肉の冷製に合う。がちょうのローストの冷製にもよく合う。

\hypertarget{nota-sauce-suedoise}{%
\subparagraph{【原注】}\label{nota-sauce-suedoise}}

リンゴの時季でない場合は、リンゴのピュレの代わりに房なりの緑のグロゼイユ\footnote{すぐり。ここではホワイトカラントの若どりのものを指している。}またはグーズベリー\footnote{groseilles
  à maquereau (グロゼイユザマクロー)。}のピュレ2 \(\frac{1}{2}\)
dLを固く立てたマヨネーズ1
Lに加える。このソースはマスタードで風味付けしてもいい。
\end{recette}
\hypertarget{sauces-froides-anglaises}{%
\subsection[イギリス風ソース(冷製)]{\texorpdfstring{イギリス風ソース(冷製)\footnote{この節に収録されているレシピは初版から第四版まで、表現の異同はあるが、項目に変化はない。興味深いことに、1907年刊の英語版\emph{A
  Guide to Modern Cookery}においても全て掲載されている。}}{イギリス風ソース(冷製)}}\label{sauces-froides-anglaises}}

\frsec{Sauces Froides Anglaises}

\index{いきりすふう@イギリス風!そーすれいせい@---ソース(冷製)}
\index{sauce@sauce!froides anglaises@---s froides anglaises}
\index{anglais@anglais(e)!sauces froides@Sauces froides ---es}
\begin{recette}
\hypertarget{cambridge-sauce}{%
\subsubsection[ケンブリッジソース]{\texorpdfstring{ケンブリッジソース\footnote{ケンブリッジはイングランド東部のケンブリッジシャーの州都。大学都市として有名。}}{ケンブリッジソース}}\label{cambridge-sauce}}

\frsub{Sauce Cambridge\hspace{1em}\normalfont(\textit{Cambridge-Sauce})}

\index{いきりすふう@イギリス風!けんふりつし@ケンブリッジソース(冷製)}
\index{けんふりつし@ケンブリッジ!そーす@---ソース}
\index{cambridge@Cambridge!sauce@Sauce --- (Cambridge-Sauce)(froide)}
\index{anglais@anglais(e)!sauces cambridgeSauce Cambridge (Cambridge-Sauce)(froide)}

固茹で卵の黄身6個と、よく洗ったアンチョビのフィレ4枚、小さめのケイパー大さじ1杯、セルフイユとエストラゴンとシブレットのみじん切りを同量ずつ計大さじ1杯を鉢に入れてよくすり潰す。マヨネーズを作る際の要領で、マスタード小さじ1杯、油1
\(\frac{1}{2}\) dL\footnote{マヨネーズを作る際の要領で、と表現しているのに対して油の量が少なく思われるが、初版は「油1
  dL」、第二版以降は「1 \(\frac{1}{2}\) dL」となっている。}とヴィネガー大さじ1杯を加える。カイエンヌごく少量で風味を引き締める。ヘラでソースを混ぜながら布で漉し
\footnote{濃度のあるソースを布で漉す方法については\protect\hyperlink{veloute}{ヴルテ}訳注参照。}、ボウルに入れる。泡立て器で軽く混ぜて滑らかにしてやり、パセリのみじん切り小さじ1杯を加えて仕上げる。

\hypertarget{cumberland-sauce}{%
\subsubsection[カンバーランドソース]{\texorpdfstring{カンバーランドソース\footnote{イングランド北部の旧カウンティ(行政区分、ほぼ「州」と考えていい)のひとつ。現在はウェストモーランド、ランカシャー、ヨークシャーの一部と統合され、カンブリアとなっている。}}{カンバーランドソース}}\label{cumberland-sauce}}

\frsub{Sauce Cumberland\hspace{1em}\normalfont(\textit{Cumberland-Sauce})}

\index{いきりすふう@イギリス風!かんはーらんと@カンバーランドソース(冷製)}
\index{かんはーらんと@カンバーランド!そーす@---ソース(冷製)}
\index{cumberland@Cumberland!sauce@Sauce --- (Cumberland-Sauce)(froide)}
\index{anglais@anglais(e)!sauce cumberland@Sauce Cumberland (Cumberland-Sauce)(froide)}

鍋に\protect\hyperlink{gelee-de-groseilles-a}{グロゼイユのジュレ}大さじ4杯を入れて溶かし、そこにポルト酒1
dL と細かいみじん切りにして下茹でして水気を絞ったエシャロット大さじ
\(\frac{1}{2}\)杯、オレンジの表皮と\footnote{zeste
  ゼスト。柑橘類の硬い外皮をrâpe(ラプ)と呼ばれる器具を用いておろした場合にもこの語を用いる。}とレモンの表皮を薄く剥いてごく細い千切りにしてしっかり下茹でしてよく水気をきって冷ましたもの各大さじ1杯、オレンジ1個の搾り汁、レモン
\(\frac{1}{2}\)個分の搾り汁、マスタード小さじ1杯、カイエンヌごく少量、粉末の生姜少々を加える。

全体をよく混ぜる。

\ldots{}\ldots{}大型ジビエの冷製に合わせる。

\hypertarget{gloucester-sauce}{%
\subsubsection[グロスターソース]{\texorpdfstring{グロスターソース\footnote{イングランド南部、グロースターシャーの州都。}}{グロスターソース}}\label{gloucester-sauce}}

\frsub{Sauce Gloucester\hspace{1em}\normalfont(\textit{Gloucester-Sauce})}

\index{いきりすふう@イギリス風!くろすたー@グロスターソース(冷製)}
\index{くろすたー@グロスター!そーす@---ソース(冷製)}
\index{gloucester@Gloucester!sauce@Sauce --- (Gloucester-Sauce)(froide)}
\index{anglais@anglais(e)!sauce gloucester@Sauce Gloucester (Gloucester-Sauce)(froide)}

ごく固く立てた\protect\hyperlink{mayonnaise}{マヨネーズ}1 Lに、レモン
\(\frac{1}{2}\)個分の搾り汁を加えたサワークリーム2
dLと、細かく刻んだフェンネル1つまみ、ダービーソース\footnote{初版と第二版は「ウスターシャソース数滴」、第三版は「エスコフィエソース数滴」となっている。ダービーソースについては\protect\hyperlink{sauce-russe-froide}{ロシア風ソース}訳注も参照のこと。}大さじ2杯を加える。

\ldots{}\ldots{}主として肉の冷製料理に合わせる。

\hypertarget{mint-sauce}{%
\subsubsection{ミントソース}\label{mint-sauce}}

\frsub{Sauce Menthe\hspace{1em}\normalfont(\textit{Mint-Sauce})}

\index{いきりすふう@イギリス風!みんと@ミントソース(冷製)}
\index{みんと@ミント!そーす@---ソース(冷製)}
\index{menthe@menthe!sauce@Sauce --- (Mint-Sauce)(froide)}
\index{anglais@anglais(e)!sauce menthe@Sauce Menthe (Mint-Sauce)(froide)}

ミントの葉50
gをごく細い千切りか、みじん切りにする。これをボウルに入れて、白いカソナード\footnote{通常cassonadeすなわち粗糖は褐色のものが多い。}かパウダーシュガー25
gとヴィネガー1 \(\frac{1}{2}\)
dL、塩1つまみ、水大さじ4杯を加える。全体によく混ぜること。

\ldots{}\ldots{}仔羊\footnote{本書で仔羊agneau(アニョー)と言う場合はほぼ例外なく乳呑仔羊、
  agneau de
  lait(アニョードレ)を意味する。現代は仔羊という語の意味する範囲が広くなり、牧草および飼料によりある程度まで肥育した羊の赤身肉も「仔羊」として扱うが、乳呑仔羊は白身肉なので注意。}の温製、冷製に添える。

\hypertarget{oxford-sauce}{%
\subsubsection[オックスフォードソース]{\texorpdfstring{オックスフォードソース\footnote{イングランド東部、オックスフォードシャーの州都。英語圏では最古の大学であるオックスフォード大学を中心とした学園都市として有名。}}{オックスフォードソース}}\label{oxford-sauce}}

\frsub{Sauce Oxford\hspace{1em}\normalfont(\textit{Oxford Sauce})}

\index{いきりすふう@イギリス風!おつくすふおーとそーす@オックスフォードソース(冷製)}
\index{おつくすふおーと@オックスフォード!そーす@---ソース(冷製)}
\index{oxford@Oxford!sauce@Sauce --- (Oxford-Sauce)(froide)}
\index{anglais@anglais(e)!sauce oxford@Sauce Oxford (Oxford-Sauce)(froide)}

上述の\protect\hyperlink{cumberland-sauce}{カンバーランドソース}と同様に作るが、以下の2点を変更する\footnote{オレンジとレモンの皮の扱いと量を変えただけで別のソースとして扱うことに疑問はあるが、これについては初版から一貫してまったく説明がない。何らかのエピソードがこれらのソース名にはあったと思われるが不明。}。

\begin{enumerate}
\def\labelenumi{\arabic{enumi}.}
\item
  オレンジとレモンの外皮は千切りにするのではなく、器具を用いておろすこと。
\item
  その量は半分にする。つまり、おろした外皮はそれぞれ大さじ
  \(\frac{1}{2}\)杯にすること。
\end{enumerate}

\ldots{}\ldots{}用途はカンバーランドソースと同じ。

\hypertarget{cold-horseradish-sauce}{%
\subsubsection{ホースラディッシュソース}\label{cold-horseradish-sauce}}

\frsub{Sauce Raifort\hspace{1em}\normalfont(\textit{Cold horseradish sauce})}

\index{いきりすふう@イギリス風!ほーすらていつしゆれいせい@ホースラディッシュソース(冷製)}
\index{ほーすらていつしゆ@ホースラディッシュ!そーす@---ソース(冷製)}
\index{raifort@raifort!sauce@Sauce --- (Cold horseradish sauce)}
\index{anglais@anglais(e)!sauces raifort froide@Sauce Raifort (Cold horseradish sauce)}

陶製の器に、マスタード大さじ1杯、細かくおろしたレフォール50
g、パウダーシュガー50 g、塩1つまみ、生クリーム5
dL、牛乳に浸してからよく圧したパンの身250
g、ヴィネガー大さじ2杯を入れて混ぜ合わせる。

\ldots{}\ldots{}このソースは茹でた牛肉やローストに合わせる。よく冷やしてから供すること。

\hypertarget{horseradish-sauce}{%
\subparagraph{【原注】}\label{horseradish-sauce}}

ソースにヴィネガーを加えるのは作業の最後とすること。
\end{recette}