\begin{main}

\hypertarget{ux51b7ux88fdux30bdux30fcux30b9}{%
\section{冷製ソース}\label{ux51b7ux88fdux30bdux30fcux30b9}}

\begin{frsecenv}

Sauces Froides

\end{frsecenv}

\end{main}

\begin{recette}

\index{sauce@sauce!sauces froides@\textbf{---s froides}|(}
\index{そーす@ソース!れいせい@\textbf{冷製---}|(}

\hypertarget{sauce-aioli}{%
\subsubsection{アイヨリ/プロヴァンスバター{[}\^{}2{]}}\label{sauce-aioli}}

\index{あんたるしあ@アンダルシア!そーす@---風ソース}
\index{そーす@ソース!あんたるしあふう@アンダルシア風---}
\index{sauce@sauce!andalouse@--- Andalouse}
\index{andalous@Andalous(e)!sauce@Sauce Andalouse}
\index[src]{andalouse@Andalouse}
\index[src]{あんたるしあふう@アンダルシア風}

ごく固く仕上げた\protect\hyperlink{mayonnaise}{ソース・マヨネーズ}
\(\frac{2}{3}\) Lに、上等な赤いトマトピュレ2 \(\frac{1}{2}\)
dLを加える。小さなさいの目に切ったポワヴロン\footnote{Poivron
  いわゆる日本で青果として輸入されているパプリカ(肉厚の辛くないピーマン)とほぼ同じものだが、香辛料として用いられる粉末のパプリカと混同を避けるため、あえてフランス語をそのままカタカナに訳した。}75
gを仕上げに加える。

\atoaki{}

\hypertarget{sauce-bohemienne}{%
\subsubsection{ソース・ボヘミアの娘{[}\^{}6{]}}\label{sauce-bohemienne}}

\index{しやんていい@シャンティイ!そーす@ソース・---(冷製)}
\index{そーす@ソース!しやんていい@---・シャンティイ}
\index{sauce@sauce!chantilly@--- Chantilly (froide)}
\index{chantilly@Chantilly!sauce@Sauce --- (froide)}
\index[src]{chantilly froide@Chantilly (froide)}
\index[src]{しやんていいれいせい@シャンティイ(冷製)}

酸味付けにレモンを用いて、固く仕上げた\protect\hyperlink{mayonnaise}{ソース・マヨネーズ}
\(\frac{2}{3}\)
Lを用意しておく。提供直前に、ごく固く泡立てた生クリーム大さじ4杯\footnote{大さじ1杯
  = 15
  ccという概念にとらわれないよう注意。原文は、大きなスプーンで泡立てた生クリームをざっくりと4回加えるイメージで書かれている。本書における通常のソースの仕上がり量が約1
  Lであることを考慮すると、最低でも100 mL以上は加えることになるだろう。}を加える。その後、味を\ruby{調}{ととの}える。

\ldots{}\ldots{}もっぱら、アスパラガスの冷製、温製に添える。

\hypertarget{nota-sauce-chantilly-froide}{%
\subparagraph{【原注】}\label{nota-sauce-chantilly-froide}}

生クリームを加えるのは、このソースを使うまさにその時にすること。前もって加えておくと、ソースが分離してしまう恐れがあるので注意。

\atoaki{}

\hypertarget{sauce-genoise-froids}{%
\subsubsection{ジェノヴァ風ソース{[}\^{}12{]}}\label{sauce-genoise-froids}}

\index{くりひつしゆ@グリビッシュ!そーす@ソース・---(冷製)}
\index{そーす@ソース!くりひつしゆ@---・グリビッシュ}
\index{sauce@sauce!gribiche@--- Gribiche (froide)}
\index{gribiche@gribiche!sauce@Sauce --- (froide)}
\index[src]{gribiche@Gribiche}
\index[src]{くりひつしゆ@グリビッシュ(冷製)}

茹であがったばかりの固茹で卵の黄身6個を陶製のボウルに入れ、マスタード小さじ1杯、塩1つまみ強、こしょう適量を加えてよく練り、滑らかなペースト状にする。植物油
\(\frac{1}{2}\) Lとヴィネガー大さじ1
\(\frac{1}{2}\)杯を加えながらよく混ぜて乳化させる。仕上げに、コルニションとケイパーのみじん切り計100
gと、パセリとセルフイユ、エストラゴンのみじん切りのミックスを大さじ1杯、短かめの千切りにした固茹で卵の白身3個分を加える。

\ldots{}\ldots{}冷製の魚料理に添えるのが一般的。

\atoaki{}

\hypertarget{sauce-groseilles-au-raifort}{%
\subsubsection{ソース・グロゼイユ・レフォール風味}\label{sauce-groseilles-au-raifort}}

\index{いたりあふう@イタリア風!そーすれいせい@---ソース(冷製)}
\index{そーす@ソース!いたりあふうれいせい@イタリア風---(冷製)}
\index{sauce@sauce!italienne@--- Italienne (froide)}
\index{italien@italien(ne)!sauce froide@Sauce ---ne (froide)}
\index[src]{italienne froide@Italienne (froide)}
\index[src]{いたりあふうれいせい@イタリア風(冷製)}

仔牛の脳半分を、香草を効かせたクールブイヨンで火を通し、目の細かい網で裏漉しする。同量の牛あるいは羊の脳でもいい。

裏漉ししたピュレを陶製の器に入れ、泡立て器で滑らかになるまで混ぜる。卵黄5個と塩10
g、こしょう1つまみ強、油1
Lとレモン果汁1個分でマヨネーズを作り、そこの脳のピュレを加える。パセリのみじん切り大さじ1杯強を加えて仕上げる。

\ldots{}\ldots{}どんな冷製の肉料理にも合う。

\atoaki{}

\hypertarget{mayonnaise}{%
\subsubsection{マヨネーズ{[}\^{}18{]}}\label{mayonnaise}}

\index{まよねーす@マヨネーズ!こーていんくよう@コーティング用---}
\index{そーす@ソース!こーていんくようまよねーす@コーティング用マヨネーズ}
\index{sauce@sauce!mayonnaise collee@--- Mayonnaise collée}
\index{mayonnaise@mayonnaise!sauce collée@Sauce --- collée}

コーティング用マヨネーズは、マヨネーズ7 dLに溶かしたジュレ3
dLを混ぜ込んだもの。野菜サラダをあえるのに使う他、\protect\hyperlink{}{「ロシア風」ショフロワ}の素材を覆うのにも使う。

\hypertarget{nota-mayonnaise-collee}{%
\subparagraph{【原注】}\label{nota-mayonnaise-collee}}

\protect\hyperlink{sauce-chaud-froid-maigre}{魚料理用ソース・ショフロワ}の項で述べたように、このコーティング用マヨネーズの代わりに魚料理用ソース・ショフロワを使う方がいい。その方がコーティング用マヨネーズを使う場合よりも風味も見た目もよくなる。というのも、コーティング用マヨネーズは、冷気によってゼラチンが固まるとともに収縮し、マヨネーズに圧力がかかるために、ソースで素材を覆った表面に油が浸み出してしまう\footnote{初版における原注は、「コーティング用マヨネーズで覆ったものは、数時間経つと、油の露で覆われたようになってしまうことがある。その原因は、冷気によってゼラチンが固まる際に収縮し、その結果マヨネーズに圧力がかかり、液体である油がソースを覆った表面に浸みだしてくることだ。これを避けるために、コーティング用マヨネーズはこんにちでは使われなくなっており、我々の場合だと、かなり以前から魚料理用ソース・ショフロワを用いている(p.163)」。第二版以降、多少の異同はあるが、ほぼ第四版の記述と同様。いずれにしても、ジュレ(親水性アミノ酸であるコラーゲンが主体)を加えたことで、親水基と疎水基を併せ持つ卵黄レシチンの乳化作用が崩れてマヨネーズが分離した結果だということには気付いていなかったと思われる。}。こういうふうに浸みが出ることを防ぐには、どんな場合でも、このコーティング用マヨネーズではなく魚料理用ソース・ショフロワを用いることをお勧めする\footnote{この『料理の手引き』ではジュレを加えたマヨネーズの使用に否定的だが、カレーム『19世紀フランス料理』ではSauce
  Magnonaiseとして、まず最初にジュレを加えるレシピが掲載されている。概略を示すと、氷の上に置いた陶製の容器に卵黄2個、塩、白こしょう少々、エストラゴンヴィネガー少々を入れる。木のさじで素早くかき混ぜる。まとまってきたら、エクス産の油大さじ1杯とヴィネガー少々を、少しずつ加えていく。容器の壁に叩きつけるようにしてソースを泡立てていく。この作業でマニョネーズの白さが決まるという。また、油をごく少量ずつ加えていくことを強調している。粘度が出て滑らかになったら、最後に油をグラス二杯(≒2
  dL)と\textbf{アスピック用ジュレ}をグラス
  \(\frac{1}{2}\)杯、エストラゴンヴィネガー適量を加えて仕上げる、というもの(t.3,
  p.132.
  強調は引用者による)。また、カレームは卵黄に含まれるレシチンによって乳化作用が起きることを経験的にさえも理解していなかったようであり、卵黄を用いないマニョネーズのレシピも掲載されている。なかでも特徴的なのは、「ジュレ入りの白いマニョネーズ」のレシピで、これは氷の上に鍋を置き、大きなレードル2杯の白いジュレと同量の油、レードル1杯のヴィネガー、塩、こしょうを入れて卵白用の泡立て器でよく混ぜ、途中何回かレモン果汁を少しずつ加えて白く仕上げるようにする、というもの(\emph{ibid}.,
  p.133)。とりわけ舞踏会や格式ある大規模な宴席で魚のフィレや鶏のアスピックを飾るのに適していると述べている。}。少なくとも、そうするのが一般的になりつつある。

\atoaki{}

\hypertarget{mayonnaise-fouette-russe}{%
\subsubsection{ロシア風ホイップマヨネーズ}\label{mayonnaise-fouette-russe}}

\index{まよねーす@マヨネーズ!はりえーしよん@---のバリエーション}
\index{そーす@ソース!まよねーすのはりえーしよん@マヨネーズのバリエーション}
\index{sauce@sauce!mayonnaises dieverses@---s Mayonnaises diverses}
\index{mayonnaise@mayonnaise!sauces diverses@Sauces --- diverses}
\index[src]{mayonnaises diverses@Mayonnaises diverses}
\index[src]{まよねーすのはりえーしよん@マヨネーズのバリエーション}

オードブルや冷製料理に合わせるのに、大型甲殻類\footnote{homard
  オマール、langouste ラングースト(≒伊勢エビ)など。}およびエクルヴィス\footnote{ざりがにのこと。詳しくは\protect\hyperlink{sauce-bavaroise}{バイエルン風ソース}訳注参照。}の卵やクリーム状の部分を用いたり、クルヴェット\footnote{小海老のこと。詳しくは\protect\hyperlink{sauce-aux-crevettes}{ソース・クルヴェット}訳注参照。}、キャビア、アンチョビなどを加えることでマヨネーズにバリエーションを付けることが出来る。

上記の材料のいずれかをすり潰してから少量のマヨネーズを加えてピュレ状にして布で漉す。これを適量のマヨネーズに混ぜ合わせればよい。

\atoaki{}

\hypertarget{ux30bdux30fcux30b9ux30e0ux30b9ux30afux30c6ux30fcux30eb26-sauce-mousquetaire}{%
\subsubsection{ソース・ムスクテール{[}\^{}26{]}
\{\#sauce-mousquetaire}\label{ux30bdux30fcux30b9ux30e0ux30b9ux30afux30c6ux30fcux30eb26-sauce-mousquetaire}}

\index{そーす@ソース!むたるとなまくりーむいり@---・ムタルド・生クリーム入り(冷製)}
\index{むたると@ムタルド(マスタード)!そーすなまくりーむいり@ソース・---・生クリーム入り(冷製)}
\index{ますたーと@マスタード(ムタルド)!そーすなまくりーむいり@ソース・ムタルド・生クリーム入り(冷製)}
\index{sauce@sauce!moutarde creme@--- moutarde à la crème (froide)}
\index{moutarde@moutarde!sauce creme@Sauce --- à la crème (froide)}
\index[src]{moutarde creme@Moutarde à la Crème (froide)}
\index[src]{むたるとなまくりーむ@ムタルド・生クリーム入り}

陶製の容器にマスタード大さじ3杯と塩1つまみ、こしょう少々とレモン果汁少々を入れて混ぜ合わせる。ここに少しずつ、マヨネーズを作る要領で、ごく新鮮なクレーム・エペス\footnote{乳酸醗酵させた、とても濃度のある生クリーム。}約2
dLを加える。

\ldots{}\ldots{}オードブル用。

\atoaki{}

\hypertarget{sauce-raifort-aux-noix}{%
\subsubsection{くるみ入りソース・レフォール}\label{sauce-raifort-aux-noix}}

\index{そーす@ソース!らういこつと@---・ラヴィゴット(冷製)}
\index{らういこつと@ラヴィゴット!そーす@ソース・---(冷製)}
\index{ういねくれつと@ヴィネグレット|see{らういこつと@ラヴィゴット!そーす@---(冷製)}}
\index{sauce@sauce!ravigotte froide@--- Ravigote, ou Vinaigrette (froide)}
\index{ravigote@ravigote!sauce@Sauce --- , ou vinaigrette (froide)}
\index{vinaigrette@vinaigrette|see{ravigote@ragigote!sauce@sauce --- , ou vinaigrette (froide)}}
\index[src]{ravigote@Ravigote} \index[src]{らういこつと@ラヴィゴット}
\index[src]{vinaigrette@Vinaigrette|see{ravigote@Ravigote}}
\index[src]{ういねくれつと@ヴィネグレット|see{らういこつと@ラヴィゴット}}

\hypertarget{ux6750ux6599}{%
\subparagraph{材料}\label{ux6750ux6599}}

\ldots{}\ldots{}油5 dL、ヴィネガー2
dL、小さめのケイパー小さじ2杯、パセリ50
g、セルフイユとエストラゴン、シブレットを刻んだもの40
g、細かくみじん切りにした玉ねぎ70 g、塩4 g、こしょう1
g。以上をよく混ぜ合わせる。

\ldots{}\ldots{}仔牛の頭や足、羊の足などに合わせる。

\atoaki{}

\hypertarget{sauce-remoulade}{%
\subsubsection{ソース・レムラード{[}\^{}34{]}}\label{sauce-remoulade}}

\index{ろしあふう@ロシア風!そーすれいせい@---ソース(冷製)}
\index{そーす@ソース!ろしあふうれいせい@ロシア風---(冷製)}
\index{sauce@sauce!russe@--- Russe (froide)}
\index{russe@russe!sauce froide@Sauce --- (froide)}
\index[src]{russe@Russe} \index[src]{ろしあふう@ロシア風}

鉢に、オマール\footnote{homard ロブスター。}かラングースト\footnote{langouste
  ≒ 伊勢エビ。}の胴のクリーム状の部分100 gとキャビア100 g\footnote{チョウザメの卵の塩蔵品のことだが、「高級」とされる順に、beluga
  (ベルガ)、osciètre, ossetra(オシエートル、オセトラ)、
  sevruga(セヴルガ)の種類がある(ここで示した読みがなはフランス語風のもの)。}、マヨネーズ大さじ2〜3杯を加えてよくすり潰す。これを目の細かい漉し器で裏漉しする。こうして出来たピュレに、マヨネーズ
\(\frac{3}{4}\)
Lを加える。大さじ1杯強のマスタードと、同量のダービーソース\footnote{初版では原注として、風味付けにマスタードを加えることを示唆しているのみ。第二版では「マスタードとウスターシャソースを各大さじ1杯強」、第三版では「マスタードとエスコフィエソースを大さじ1杯強」と変遷している。なお、ダービーソースDerby
  Sauce
  の1946年の広告には、このブランド名でバーベキューソース、ステーキソース、ウスターシャソース、ホットソース、チャプスイソースのラインナップが記されている。現実問題として、もし加えるとするならリー\&ペリンのようなウスターシャソースということになろうか。}を加えて仕上げる。

\ldots{}\ldots{}魚および甲殻類の冷製料理に添える。

\atoaki{}

\hypertarget{sauce-tartare}{%
\subsubsection{タルタルソース{[}\^{}54{]}}\label{sauce-tartare}}

\index{みとり@緑 ⇒ ヴェール/ヴェルト!ソース・ヴェルト(冷製)}
\index{うえーる@ヴェール/ヴェルト!そーす@ソース・ヴェルト(冷製)}
\index{そーす@ソース!うえると@---・ヴェルト(冷製)}
\index{sauce@sauce!verte@--- Verte}
\index{vert@vert(e)!sauce@Sauce ---e} \index[src]{verte@verte}
\index[src]{うえると@ヴェルト(冷製)}

ほうれんそうの葉\footnote{日本では、ほうれんそうを葉のみではなく葉軸とともに利用するのが一般的だが、伝統的なフランス料理において葉軸は使われないのが普通。そもそも日本のほうれんそうは密植して葉が立つように仕立てて比較的若どりするのに対して、ヨーロッパ品種のほうれんそうは株間を充分にとってロゼッタ状に葉が広がるように栽培するのが伝統的な手法。この場合、葉は肉厚に仕上がるが、葉軸は太くて固いため可食部と見なされなかった。昔のフランスの八百屋の店先では軸を切り捨てる作業風景がよく見られたという。現代では機械収穫に適した立性の品種が増えており、専用の大型機械で株元近くから切り取り、自動的に軸をある程度除去して併走する巨大なコンテナに移すという収穫方法が普及しており、量産品のピュレなどに使用されている。}50
gとクレソンの葉50 g、パセリの葉とセルフイユ、エストラゴンを同量ずつ計50
gを、沸騰した湯に投入し、強火で5分間茹でる。水気をきり、手早く冷水にさらす。しっかりと圧し絞って水気をきり、鉢に入れてすり潰す。これをトーション\footnote{綿などの天然素材で出来た調理場及びホール業務に用いられる布。サイズは50〜55
  cm×70〜80 cmのものが多い。}でくるんできつく絞り、葉の濃い汁を1
dL搾りだす。

固く立てて風味付けをした\protect\hyperlink{mayonnaise}{マヨネーズ}9
dLにこの緑の汁を加える。

\ldots{}\ldots{}冷製の魚料理や甲殻類に合わせる。

\atoaki{}

\hypertarget{sauce-vincent}{%
\subsubsection{ソース・ヴァンサン{[}\^{}44{]}}\label{sauce-vincent}}

\index{すうえーてんふう@スウェーデン風!そーすれいせい@---ソース(冷製)}
\index{そーす@ソース!すうえーてんふうれいせい@スウェーデン風---(冷製)}
\index{sauce@sauce!suedoise@--- Suédoise (froide)}
\index{suedois@suédois(e)!sauce froide@Sauce ---e (froide)}
\index[src]{suedoise@Suédoise}
\index[src]{すうえーてんふうれいせい@スウェーデン風(冷製)}

酸味のある固いリンゴを薄切りにして鍋にしっかり蓋をして煮る。普通の果肉が甘いリンゴを使う場合にはレモン果汁数滴を加えること。リンゴを煮る際には、白ワインを大さじ数杯だけ加えればいい。リンゴを煮るというよりは蒸気の圧力で溶かすイメージ。

これを目の細かい網で裏漉しする。このリンゴのピュレを2 \(\frac{1}{2}\)
dLになるまで煮詰める。充分に冷ましてから、\protect\hyperlink{mayonnaise}{マヨネーズ}
\(\frac{3}{4}\)
Lを加える。風味付けにおろした(または細かく刻んだ)レフォール大さじ1
\(\frac{1}{2}\)杯を加えて仕上げる。

\ldots{}\ldots{}このソースはとりわけ豚肉の冷製に合う。がちょうのローストの冷製にもよく合う。

\hypertarget{nota-sauce-suedoise}{%
\subparagraph{【原注】}\label{nota-sauce-suedoise}}

リンゴの時季でない場合は、リンゴのピュレの代わりに房なりの緑のグロゼイユ\footnote{すぐり。ここではホワイトカラントの若どりのものを指している。}またはグーズベリー\footnote{groseilles
  à maquereau (グロゼイユザマクロー)。}のピュレ2 \(\frac{1}{2}\)
dLを固く立てたマヨネーズ1
Lに加える。このソースはマスタードで風味付けしてもいい。

\end{recette}

\hypertarget{sauces-froides-anglaises}{%
\subsection[イギリス風ソース(冷製)]{\texorpdfstring{イギリス風ソース(冷製)\footnote{この節に収録されているレシピは初版から第四版まで、表現の異同はあるが、項目に変化はない。興味深いことに、1907年刊の英語版\emph{A
  Guide to Modern Cookery}においても全て掲載されている。}}{イギリス風ソース(冷製)}}\label{sauces-froides-anglaises}}

\frsec{Sauces Froides Anglaises}

\begin{recette}

\index{いきりすふう@イギリス風!そーすれいせい@---ソース(冷製)|(}
\index{sauce@sauce!froides anglaises@---s froides anglaises|(}
\index{anglais@anglais(e)!sauces froides@Sauces froides ---es|(}

\hypertarget{cambridge-sauce}{%
\subsubsection{ケンブリッジソース{[}\^{}58{]}}\label{cambridge-sauce}}

\index{いきりすふう@イギリス風!かんはーらんと@カンバーランドソース(冷製)}
\index{かんはーらんと@カンバーランド!そーす@---ソース(冷製)}
\index{cumberland@Cumberland!sauce@Sauce --- (Cumberland-Sauce)(froide)}
\index{anglais@anglais(e)!sauce cumberland@Sauce Cumberland (Cumberland-Sauce)(froide)}
\index[src]{cumberland@Cumberland}
\index[src]{かんはーらんと@カンバーランド}

鍋に\protect\hyperlink{gelee-de-groseilles-a}{グロゼイユのジュレ}大さじ4杯を入れて溶かし、そこにポルト酒1
dL と細かいみじん切りにして下茹でして水気を絞ったエシャロット大さじ
\(\frac{1}{2}\)杯、オレンジの表皮と\footnote{zeste
  ゼスト。柑橘類の硬い外皮をrâpe(ラプ)と呼ばれる器具を用いておろした場合にもこの語を用いる。}とレモンの表皮を薄く剥いてごく細い千切りにしてしっかり下茹でしてよく水気をきって冷ましたもの各大さじ1杯、オレンジ1個の搾り汁、レモン
\(\frac{1}{2}\)個分の搾り汁、マスタード小さじ1杯、カイエンヌごく少量、粉末の生姜少々を加える。

全体をよく混ぜる。

\ldots{}\ldots{}大型ジビエの冷製に合わせる。

\atoaki{}

\hypertarget{gloucester-sauce}{%
\subsubsection{グロスターソース{[}\^{}61{]}}\label{gloucester-sauce}}

\index{いきりすふう@イギリス風!みんと@ミントソース(冷製)}
\index{みんと@ミント!そーす@---ソース(冷製)}
\index{menthe@menthe!sauce@Sauce --- (Mint-Sauce)(froide)}
\index{anglais@anglais(e)!sauce menthe@Sauce Menthe (Mint-Sauce)(froide)}
\index[src]{menthe@Menthe (Mint Sauce)} \index[src]{みんと@ミント}

ミントの葉50
gをごく細い千切りか、みじん切りにする。これをボウルに入れて、白いカソナード\footnote{通常cassonadeすなわち粗糖は褐色のものが多い。}かパウダーシュガー25
gとヴィネガー1 \(\frac{1}{2}\)
dL、塩1つまみ、水大さじ4杯を加える。全体によく混ぜること。

\ldots{}\ldots{}仔羊\footnote{本書で仔羊agneau(アニョー)と言う場合はほぼ例外なく乳呑仔羊、
  agneau de
  lait(アニョードレ)を意味する。現代は仔羊という語の意味する範囲が広くなり、牧草および飼料によりある程度まで肥育した羊の赤身肉も「仔羊」として扱うが、乳呑仔羊は白身肉なので注意。}の温製、冷製に添える。

\atoaki{}

\hypertarget{oxford-sauce}{%
\subsubsection{オックスフォードソース{[}\^{}65{]}}\label{oxford-sauce}}

\index{いきりすふう@イギリス風!ほーすらていつしゆれいせい@ホースラディッシュソース(冷製)}
\index{ほーすらていつしゆ@ホースラディッシュ!そーす@---ソース(冷製)}
\index{raifort@raifort!sauce@Sauce --- (Cold horseradish sauce)}
\index{anglais@anglais(e)!sauces raifort froide@Sauce Raifort (Cold horseradish sauce)}
\index[src]{coldhorseradish@Cold Hordradich}
\index[src]{ほーすらていつしゆれいせい@ホースラディッシュ(冷製)}

陶製の器に、マスタード大さじ1杯、細かくおろしたレフォール50
g、パウダーシュガー50 g、塩1つまみ、生クリーム5
dL、牛乳に浸してからよく圧したパンの身250
g、ヴィネガー大さじ2杯を入れて混ぜ合わせる。

\ldots{}\ldots{}このソースは茹でた牛肉やローストに合わせる。よく冷やしてから供すること。

\hypertarget{horseradish-sauce}{%
\subparagraph{【原注】}\label{horseradish-sauce}}

ソースにヴィネガーを加えるのは作業の最後とすること。

\index{いきりすふう@イギリス風!そーすれいせい@---ソース(冷製)|)}
\index{sauce@sauce!froides anglaises@---s froides anglaises|)}
\index{anglais@anglais(e)!sauces froides@Sauces froides ---es|)}

\index{sauce@sauce!sauces froides@\textbf{---s froides}|)}
\index{そーす@ソース!れいせい@\textbf{冷製---}|)}

\end{recette}
