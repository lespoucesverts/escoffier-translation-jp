\href{未、原文対照チェック}{} \href{未、日本語表現校正}{}
\href{未、その他修正}{} \href{未、原稿最終校正}{}

\hypertarget{beurres-composes}{%
\section{合わせバター}\label{beurres-composes}}

\vspace{-1\zw}
\begin{center}
\textbf{グリル、ソースの補助材料、オードブル用}
\end{center}
\vspace{1\zw}

\frsec{Beurres Composés pour Adjuvants de Sauces et Hors-d'oeuvre}

\index{あわせはたー@合わせバター} \index{はたー@バター ⇒ 合わせバター}
\index{ふーるこんほせ@ブール・コンポゼ ⇒ 合わせバター}
\index{みつくすはたー@ミックスバター ⇒ 合わせバター}
\index{beurre@beurre!beurres composes@Beurres Composés}

\hypertarget{observation-sur-les-beurres-composes}{%
\subsection{概説}\label{observation-sur-les-beurres-composes}}

本書においてレシピを掲載している合わせバター\footnote{beurre composé
  ブール・コンポゼ。ミックスバターとも。少なくともバターは中世以来用長く用いられてきた食材だが、中世〜ルネサンスにおいては獣脂(もっぱらラード)のほうが多く用いられる傾向にあった。17
  世紀以降はたとえばラ・ヴァレーヌ『フランス料理の本』におけるアスパラガスの白いソース添え(\protect\hyperlink{sauce-hollandaise}{ソース・オランデーズ}訳注参照)のように、バターを料理に用いることが中世の料理書と比較すると圧倒的に増えたのは事実である。ムノンの1741年刊『ブルジョワ屋敷に勤める女性料理人のための本』のバターの項には「良質のバターを用いることは料理でとても重要なことであり、バターが匂いを放っているようではどんな素晴しい皿も台無しだ。料理担当の女中であればこのことをよく理解しておくことと、良質なバターの価格を手に入れるのに金を惜しんではならないことを肝に銘じておくこと。最良のバターは自然な黄色をしており、白いものは大抵の場合、さして美味しくない。バルボットという植物から採った黄色で着色されたバターもある。こういうバターの色は、自然なバターの黄色よりもくすんだもので、慣れれば簡単に見分けることが出来る(p.320)」
  \textgreater{}\textgreater{}\textgreater{}\textgreater{}\textgreater{}\textgreater{}\textgreater{}
  Stashed changes}のうちのほとんどは、甲殻類の合わせバターを除いて、料理に直接用いられることがとても少ない。だが、合わせバターはさまざまなシチュエーションで役に立つ。ポタージュでは野菜の合わせバターが、その他の合わせバターはソース作りにおいて有用だ。ソースの風味と性格を明確に伝える決め手になるからだ。

だから、読者である料理人諸君には、ここに書いてあることを真剣に読みとっていただきたい。

甲殻類のバターについては、経験上、湯煎にかけながら煮出して\footnote{infuser
  (アンフュゼ)。}から、氷水で冷やした陶製の容器に布で漉し入れるといい。そうすれば、冷たい状態で作るよりも赤みがきれいに出る。だが逆に、熱によって風味の繊細さが失なわれてしまい、雑味さえも出てしまう。

この問題点を解決するために、我々は二種類の違うバターを作るという方式を採ることにした。ひとつは甲殻類の胴のクリーム状の部分と切りくずあるいは身そのものを生のバターとともに鉢ですり潰して、目の細かい網で裏漉しするか、布で漉すというもの。このバターはソースに完璧ともいうべき風味を添えてくれる。とりわけベシャメルソースをベースとしたソースの場合はそうだ。

もうひとつは、甲殻類の殻だけを用いて、熱して作るものだ。これは「色付け」の役割しか持たない。この方式はまことに素晴しい結果を得られるので、ぜひとも実行していただきたい。

場合によっては、我々はバターを同様の上等な生クリームに代えることがある。生クリームのほうがバターよりも、素材の持つ風味や香気をよく吸収する。こうすればソースやポタージュの仕上げに加えるのに文句ないクリ\footnote{coulis
  (クリ)水分のやや多いピュレをイメージするといい。「クーリ」と呼ぶ日本の調理現場は多い。}を作ることが出来るわけだ。

色付け用のバターを使うと、ソースがきれいに色付き、個性的なソースとなる。どんな場合でも、カルミン色素\footnote{コチニール色素ともいう。ラックカイガラムシなどを原料として抽出した色素。ヨーロッパでは古代から中世にかけてケルメスカイガラムシから抽出され利用されてきた、非常に歴史の古い色素。とりわけルネサン期には高級毛織物の染料として需要が高まった。また絵の具にも使用された。その後、ウチワサボテンでエンジムシを大量に養殖していた中南米を支配下に置いたスペインが、これを新大陸産のカルミンとしてヨーロッパ各国に売ることで巨万の富を得たという。かつて食品工業において多用された。
  1838年の『ラルース・ガストロノミック』初版では、「コチニールから抽出される鮮かな赤色色素で毒性はない。多くの食品に着色料として用いられている」とある。現在は使用が減りつつあり、代替品としてビーツから抽出したビートレッドなどが増えてきている。また、この本文でカルミン色素の使用を「くすんだ、情けない色合いを与える」として否定的に扱っているのは、この色素がpHによって色調が変化し、なおかつ蛋白質を多く含む料理に加えると色素自体が紫色に変化する(結果としてソースやポタージュ全体が濁ったような色になる)ことがあるためだろう。}よりもずっといい。カルミン色素はソースやポタージュにくすんだ、なさけない色合いしか与えてはくれないのだ。

合わせバターは一般的に、使う際にその都度作る\footnote{原文 au moment
  (オモモン)その都度、の意。à la minute
  (アラミニュット)と呼ぶ調理現場もある。}ものだが、作り置きしておかなければならない場合は、白い紙で円筒形に包んで冷蔵保管すること。
\begin{recette}
\hypertarget{beurre-d-ail}{%
\subsubsection{にんにくバター}\label{beurre-d-ail}}

\frsub{Beurre d'Ail}

\index{あわせはたー@合わせバター!にんにくはたー@にんにくバター}
\index{にんにく@にんにく!はたー@---バター}
\index{beurre@beurre!--- d'ail@Beurre d'Ail}
\index{ail@ail!beurre@Beurre d'---}

皮を剥いたにんにく200 gを強火でしっかり茹でる\footnote{生のにんにくには胃腸を刺激する酵素が含まれているが、熱により不活性化するので、よく火を通す必要がある。}。よく湯をきってから、鉢に入れてすり潰し、バター250
gと合わせ、布で漉す。

\hypertarget{beurre-d-anchois}{%
\subsubsection{アンチョビバター}\label{beurre-d-anchois}}

\frsub{Beurre d'Anchois}

\index{はたー@バター!あわせはたー@合わせバター!あんちよひはたー@アンチョビバター}
\index{あわせはたー@合わせバター!あんちよひはたー@アンチョビバター}
\index{あんちよひ@アンチョビ!はたー@---バター}
\index{beurre@beurre!anchois@--- d'Anchois}
\index{anchois@anchois!beurre@Beurre d'---}

アンチョビのフィレ200
gをよく洗い、しっかり水気を絞る。これを鉢に入れて細かくすり潰す。バター250
gを加えて布で漉す。

\hypertarget{beurre-d-amande}{%
\subsubsection{アーモンドバター}\label{beurre-d-amande}}

\frsub{Beurre d'Amande}

\index{あわせはたー@合わせバター!あーもんとはたー@アーモンドバター}
\index{あーもんと@アーモンド!はたー@---バター}
\index{beurre@beurre!amande@--- d'Amande}
\index{amande@amande!beurre@Beurre d'---}

アーモンド\footnote{アーモンドには一般的なスイートアーモンド amandes
  doucesと、苦味のあるビターアーモンドamande
  amèresの二種がある。後者はごく微量の青酸化合物を含むのであまり多く使われることはないが、香りがいいためリキュールなどの香り付けにごく少量が用いられることがある。}150
gを湯むきしてよく洗い、すぐに水数滴を加えてすり潰してペースト状にする。これをバター250
gと混ぜ合わせ、布で漉す。

\hypertarget{ux30d6ux30fcux30ebux30c0ux30f4ux30eaux30fcux30cc8}{%
\subsubsection[ブール・ダヴリーヌ]{\texorpdfstring{ブール・ダヴリーヌ\footnote{Aveline(アヴリーヌ)はヘーゼルナッツの仲間でセイヨウハシバミの大粒な変種。イタリア、ピエモンテ産やシチリア産が有名。}}{ブール・ダヴリーヌ}}\label{ux30d6ux30fcux30ebux30c0ux30f4ux30eaux30fcux30cc8}}

\hypertarget{beurre-d-aveline}{%
\paragraph{Beurre d'Aveline}\label{beurre-d-aveline}}

\index{あわせはたー@合わせバター!ふーるたうりーぬ@ブール・ダヴリーヌ}
\index{あうりーぬ@アヴリーヌ!ふーる@ブール・---}
\index{へーせるなつつ@ヘーゼルナッツ!ふーるたうりーぬ@ブール・ダヴリーヌ}
\index{beurre@beurre!aveline@--- d'Aveline}
\index{aveline@aveline!beurre@Beurre d'---}

アヴリーヌ150
gを焙煎して丁寧に皮を剥く。油が浮いてこないよう水を数滴加えてペースト状にすり潰す。これとバター250
gを混ぜ合わせる。目の細かい網で裏漉しするか、布で漉す。

\hypertarget{beurre-bercy}{%
\subsubsection[ブール・ベルシー]{\texorpdfstring{ブール・ベルシー\footnote{\protect\hyperlink{sauce-bercy}{ソース・ベルシー}訳注参照。}}{ブール・ベルシー}}\label{beurre-bercy}}

\frsub{Beurre Bercy}

\index{あわせはたー@合わせバター!ふーるたへるしー@ブール・ベルシー}
\index{へるしー@ベルシー!ふーる@ブール・---}
\index{beurre@beurre!bercy@--- Bercy}
\index{bercy@Bercy!beurre@Beurre ---}

白ワイン2
dLに細かく刻んだエシャロット大さじ1杯を加えて半量になるまで煮詰める。生温い程度まで冷ましてから、ポマード状に柔らかくした\footnote{ポマードは昔よく用いられた整髪料だが、現代では珍しいものとなってしまった。ただ、この表現は定型句のひとつであるとともに、現代日本の調理現場で現在もこの表現を用いるところがあるため、あえてこの訳語を採用した。意味としては「指がすっと入る程度の柔らかさ」であり、シリコンゴムのヘラなどで混ぜやすいけれども、溶けて液体にはなっているわけではない状態のことを意味している。}バター
200 gを混ぜ込む。牛骨髄500 gをさいの目に切って\footnote{原文 couper en
  dés。フランス語のまま「デにする(切る)」と表現することもある。}、沸騰しない程度の湯で火を通し、よく湯ぎりをして加える。パセリのみじん切り大さじ1杯と塩8
g、挽きたてのこしょう1つまみ強とレモン\(\frac{1}{2}\)個分の果汁を加えて仕上げる。

\hypertarget{beurre-de-caviar}{%
\subsubsection{キャビアバター}\label{beurre-de-caviar}}

\frsub{Beurre de Caviar}

\index{あわせはたー@合わせバター!きやひあはたー@キャビアバター}
\index{きやひあ@キャビア!はたー@---バター}
\index{beurre@beurre!caviar@--- de Caviar}
\index{caviar@caviar!beurre@Beurre de ---}

圧縮キャビア\footnote{もとはロシアで雪の中の樽で保存するために圧縮したもの。キャビアのグレードはベルガ、オセトラ、セヴルガが混ざっているのが多いという。比較的安価に利用できる。}75
gを細かくすり潰す。パター250 gを加えて、布で漉す。

\hypertarget{beurre-chivry}{%
\subsubsection[ブール・シヴリ/ブール・ラヴィゴット]{\texorpdfstring{ブール・シヴリ/ブール・ラヴィゴット\footnote{それぞれの名称などについては、\protect\hyperlink{sacue-chivry}{ソース・シヴリ}および\protect\hyperlink{sauce-ravigote}{ソース・ラヴィゴット}訳注参照。}}{ブール・シヴリ/ブール・ラヴィゴット}}\label{beurre-chivry}}

\frsub{Beurre Chivry, ou Beurre Ravigote}

\index{あわせはたー@合わせバター!しうり@ブール・シヴリ}
\index{あわせはたー@合わせバター!らういこつと@ブール・ラヴィゴット}
\index{しうり@シヴリ!ふーる@ブール・---}
\index{らういこつと@ラヴィゴット!ふーる@ブール・---}
\index{beurre@beurre!chivry@--- Chivrya}
\index{beurre@beurre!ravigote@--- Ravigote}
\index{chivry@Chivry!beurre@Beurre ---}
\index{ravigote@ravitote!beurre@Beurre ---}

パセリの葉とセルフイユ、エストラゴン、シヴレット、若摘みのサラダバーネット100
gを数分間下茹でし、水にさらしてから圧して余分な水気を絞る。エシャロットのみじん切り25
gも下茹でする。これらを鉢に入れてすり潰す。

バター125 gを加え、布で漉す。

\hypertarget{beurre-colbert}{%
\subsubsection[ブール・コルベール]{\texorpdfstring{ブール・コルベール\footnote{\protect\hyperlink{sauce-colbert}{ソース・コルベール}本文および訳注参照。}}{ブール・コルベール}}\label{beurre-colbert}}

\frsub{Beurre Colbert}

\index{あわせはたー@合わせバター!ふーるこるへーる@ブール・コルベール}
\index{こるへーる@コルベール!ふーる@ブール・---}
\index{beurre@beurre!colbert@--- Colbert}
\index{colbert@Colbert!beurre@Beurre ---}

\protect\hyperlink{beurre-maitre-d-hotel}{メートルドテルバター}200
gに、溶かした\protect\hyperlink{glace-de-viande}{グラスドヴィアンド}大さじ2杯と細かく刻んだエストラゴン小さじ2杯を加える。

\hypertarget{beurre-colorant-rouge}{%
\subsubsection{色付け用の赤いバター}\label{beurre-colorant-rouge}}

\frsub{Beurre Colorant rouge}

\index{あか@赤!いろつけようのはたー@色付け用の--いバター}
\index{あわせはたー@合わせバター!いろつけようのあかいはたー@色付け用の赤いバター}
\index{ちやくしよくそざい@着色素材!いろつけようのあかいはたー@色付け用の赤いバター}
\index{beurre@beurre!colorant rouge@--- Colorant rouge}
\index{colorant@colorant!beurre rouge@Beurre --- rouge}
\index{rouge@rouge!beurre colorant@Beurre colorant ---}

出来るだけ沢山の甲殻類の殻などの残りをまとめて用意する。殻の内側、外側に張り付いている膜などをきれいに取り除く。よく乾燥させてから、鉢\footnote{伝統的には大理石製の鉢が用いられることが多かった。}
に入れて細かく粉砕して、同じ重さのバターを加える。これを湯煎にかけてよく混ぜながら溶かす。氷水を入れた陶製の器に、布で漉し入れる。固まったバターをトーション\footnote{\protect\hyperlink{sauce-verte}{ソース・ヴェルト}訳注参照。}で包み、余計な水を絞り出す。

\hypertarget{nota-beurre-colorant-rouge}{%
\subparagraph{【原注】}\label{nota-beurre-colorant-rouge}}

この色付け用のバターを作るのに用いる甲殻類の殻がどうしてもない場合は、\protect\hyperlink{beurre-de-paprika}{パプリカバター}を用いてもいいだろう。だがいずれにせよ、どんなソースであっても、仕上がりの色合いを決めるには、出来るだけ、他の植物由来の赤色着色料の使用は避けることを勧める\footnote{この原注は第三版から。原文le
  rouge colorant
  végétal直訳すると「植物由来の赤色着色料」だが。ここではおそらくカルミン色素(コチニール色素)のことと思われる(\protect\hyperlink{observation-sur-les-beurres-composes}{本節「概説」参照})。他に赤系着色料として、ベニバナ色素、紅麹などもあるが、いずれも中国や日本において発達しことを考慮すると、両大戦間である1920年頃に「避けるべき」というほど普及していたのは、実際には昆虫由来であるコチニール色素と思われる。なお、ベニバナ色素も化学的にはカルミン酸色素。また、甲殻類の殻を茹でると赤くなるが、この色素はアスタキサンチンといい、1938年に物質として「発見」された。もちろんエスコフィエをはじめとする料理人は経験上、甲殻類の殻を適度に加熱することで、タンパク質と結びついていたアスタキサンチンがタンパク質の熱変性によって遊離して取り出せることを経験的によく知っており、それを利用してこの赤いバターを考案したと考えられる。ちなみにサーモン、鮭の身の赤色もおなじアスタキサンチンによるもので、近縁種の鱒と同様に本来は白身。}。

\hypertarget{beurre-colorant-vert}{%
\subsubsection{色付け用の緑のバター}\label{beurre-colorant-vert}}

\frsub{Beurre Colorant vert}

\index{みとり@緑!いろつけようのはたー@色付け用の--のバター}
\index{あわせはたー@合わせバター!いろつけようのみとりのはたー@色付け用の緑のバター}
\index{ちやくしよくそざい@着色素材!いろつけようのみとりのはたー@色付け用の緑のバター}
\index{beurre@beurre!colorant vert@--- Colorant vert}
\index{colorant@colorant!beurre vert@Beurre --- vert}
\index{vert@vert(e)!beurre colorantl@Beurre colorant ---}

ほうれんそうの葉1
kgをよく洗い、しっかり振って水気をきる。これを鉢に入れてすり潰す。トーション\footnote{\protect\hyperlink{sauce-verte}{ソース・ヴェルト}訳注参照。}で包んで緑の汁を絞り出す。これをソテー鍋に入れて湯煎にかけ、水分を蒸発させてペースト状にする\footnote{原文
  coaguler
  凝固させる、の意。ここでは説明的に意訳した。なお、ほうれんそうに限らず、植物の緑色は葉緑素(クロロフィル)によるものであり、葉緑素はマグネシウム(苦土)を核として窒素が周囲に結びついた構造を持つ化学物質。ほうれんそうの緑が濃いのは土壌からのマグネシウム吸収能力が高いため。食品に含まれるマグネシウムはカルシウムの吸収を促す作用があるとされている。ただし、フランスの伝統的なほうれんそうの栽培方法は、夏の終わりから初秋にかけた種を蒔き、11月頃から大きくなった葉を順次かき取って収穫するというもの。露地栽培でも1株で3
  回程度は春になるまでに収穫できるとされた。なお、フランス語で食材および調理したほうれんそうは常に
  épinard\textbf{s}
  と複数形で表される。古くは14世紀の『ル・メナジエ・ド・パリ』にespinarsという綴りで言及があり「ポレ(ブレット)の一種で、葉が長くて茎が細く、緑は普通のポレよりも濃い。espinochesとも言う。四旬節の初め頃に食べる」(t.2,
  p.141)とある。、中世・ルネサンス期のフランスでは、ほうれんそうの普及はり進まなかったようで、16世紀末にオリヴィエ・ド・セールが『農業経営論』(1600年)において「比較的新しい野菜」として栽培方法も含めて紹介している。また、épinardの語源をオリヴィエ・ド・セールは種子が尖っている(épineux)からだと書いているが、実際には、この野菜が西アジア起源のものであり、ペルシャ語のaspanāḫからフランス語に入って現在のépinardという語形に至ったと考えられている。いっぽう、日本のほうれんそうはごく一部の地域を除いては、戦後高度成長期に普及した葉菜のひとつであり、じつのところ歴史は浅い。しばしば言われる東洋系、西洋系の違いにしても、普及当初からその交配品種が使われるようになっていたために、あまり意味はない。日本で青果として流通しているほうれんそうのほとんどは、密植、立性にして比較的若どり(農協などの出荷団体によって違うが、概ね草丈25cm程度で5株から10株で200
  gの規格が平均的)のため、用いている品種がほぼ西洋系のものを交配親としている場合でも、立性に栽培するために、葉の厚みなどはあまり問題とされていない。上述のように、フランスではかつて、葉以外を可食部として見なさず、軸を切り捨てるのが普通だったことと比べると、食文化の違いの大きさがよくわかる一例だろう。}。

これを、ぴんと張ったナフキンの上に移し、さらに水気をきる。

パレットナイフを使って緑の色素を集め、鉢に入れてその倍の重さのバターを加えて練り込む。

\ldots{}\ldots{}布で漉し、冷蔵保存する。

\hypertarget{nota-beurre-colorant-vert}{%
\subparagraph{【原注】}\label{nota-beurre-colorant-vert}}

人工的な色素よりもこの緑の色素を用いたほうが利点が大きい。

\hypertarget{beurre-de-crevettes}{%
\subsubsection{クルヴェットバター}\label{beurre-de-crevettes}}

\frsub{Beurre de crevettes}

\index{あわせはたー@合わせバター!くるうえつとはたー@クルヴェットバター}
\index{くるうえつと@クルヴェット!はたー@---バター}
\index{beurre@beurre!crevette@--- de Crevettes}
\index{crevette@crevette!beurre@Beurre de ---s}

クルヴェット・グリーズ\footnote{フランスで好んで食される小海老の一種。\protect\hyperlink{sauce-aux-crevettes}{ソース・クルヴェット}訳注参照。}150
gを鉢に入れて細かくすり潰す。バター150 gを加えて、布で漉す。

\hypertarget{beurre-d-echalote}{%
\subsubsection{エシャロットバター}\label{beurre-d-echalote}}

\frsub{Beurre d'Echalote}

\index{あわせはたー@合わせバター!えしやろつとはたー@エシャロットバター}
\index{えしやろつと@エシャロット!はたー@---バター}
\index{beurre@beurre!echalote@--- d'Echalote}
\index{echalote@echalote!beurre@Beurre d' ---}

エシャロット125
gを鉢に入れてすり潰し、さっと茹でて湯をきり、トーションに包んで圧すようにして水気を取り除く。バター125gを加えて、布で漉す。

\hypertarget{beurre-d-ecrevisse}{%
\subsubsection{エクルヴィスバター}\label{beurre-d-ecrevisse}}

\frsub{Beurre d'Ecrevisse}

\index{あわせはたー@合わせバター!えくるういすはたー@エクルヴィスバター}
\index{えくるういす@エクルヴィス!はたー@---バター}
\index{beurre@beurre!ecrevisse@Beurre d'Ecrevisse}
\index{ecrevisse@écrevisse!beurre@Beurre d'---}

\protect\hyperlink{}{ビスク}を作る要領で、\protect\hyperlink{mirepoix}{ミルポワ}とともに茹でたエクルヴィス\footnote{ヨーロッパざりがに。詳しくは\protect\hyperlink{sauce-bavaroise}{バイエルン風ソース}訳注参照。}の胴や殻、尾などを鉢に入れて細かくすり潰す。これと同じ重さのバターを加え、布で漉す。

\hypertarget{beurre-pour-les-escargots}{%
\subsubsection{エスカルゴ用バター}\label{beurre-pour-les-escargots}}

\frsub{Beurre pour les Escargots}

\index{あわせはたー@合わせバター!えすかるこようはたー@エスカルゴ用バター}
\index{えすかるこ@エスカルゴ!はたー@---用バター}
\index{beurre@beurre!escargots@--- pour les escargots}
\index{escargot@escargot!beurre@Beurre pour les ---s}

(エスカルゴ50個分)

バター350 gに、細かいみじん切りにしたエシャロット35
gと、にんにく1片をすり潰してペースト状にしたもの、パセリのみじん切り25
g(大さじ1杯)、塩 12 g、こしょう2
gを加える。捏ねるようにしてよく混ぜ合わせ、冷蔵する。

\hypertarget{beurre-d-estragon}{%
\subsubsection{エストラゴンバター}\label{beurre-d-estragon}}

\frsub{Beurre d'Estragon}

\index{あわせはたー@合わせバター!えすとらこんはたー@エストラゴンバター}
\index{えすとらこん@エストラゴン!はたー@---バター}
\index{beurre@beurre!estragon@--- d'Estragon}
\index{estragon@estragon!beurre@Beurre d'---}

新鮮なエスゴラゴンの葉125
gを2分間茹がいてから湯きりして冷水にさらす。圧して余分な水気を絞る。これを鉢に入れてすり潰す。バター125
gを加えて、布で漉す。

\hypertarget{beurre-de-hereng}{%
\subsubsection{にしんバター}\label{beurre-de-hereng}}

\frsub{Beurre de Hareng}

\index{あわせはたー@合わせバター!にしんはたー@にしんバター}
\index{にしん@にしん!はたー@---バター}
\index{beurre@beurre!hareng@--- de Hareng}
\index{hareng@hareng!beurre@Beurre de ---}

にしんの燻製のフィレ\footnote{原文 hareng
  saur(アロンソール)。タイセイヨウニシンの内臓を抜いて10日程塩漬けにし、塩抜き後に24〜48時間乾燥させてから15時間以上、
  32℃程度で冷燻にしたもの。強い匂いが特徴。日本のにしんとは種が異なること、スモークサーモンと同様に冷燻であることに注意。}3枚の皮を剥いて、さいの目に切り、鉢に入れて細かくすり潰す。バター250
gを加え、布で漉す。

\hypertarget{beurre-de-homard}{%
\subsubsection{オマールバター}\label{beurre-de-homard}}

\frsub{Beurre de Homard}

\index{あわせはたー@合わせバター!おまーるはたー@オマールバター}
\index{おまーる@オマール!はたー@---バター}
\index{beurre@beurre!homard@--- de Homard}
\index{homard@homard!beurre@Beurre de ---}

使える範囲の量のオマールの胴のクリーム状の部分と卵やコライユ\footnote{オマールの胴の背側にある朱色の内子。}を鉢に入れてすり潰す。それと同じ重さのバターを加え、布で漉す。

\hypertarget{beurre-de-laitance}{%
\subsubsection{白子バター}\label{beurre-de-laitance}}

\frsub{Beurre de Laitance}

\index{あわせはたー@合わせバター!しらこはたー@白子バター}
\index{しらこ@白子!はたー@---バター}
\index{beurre@beurre!laitance@--- de Laitance}
\index{laitance@laitance!beurre@Beurre de ---}

沸騰しない程度の温度で茹で、よく冷ました白子\footnote{日本ではスケトウダラの白子が一般的だが、フランスの伝統的高級料理では鯉の白子がもっとも一般的。他に鯖やにしんの白子も用いられる。}125
gを鉢に入れてすり潰す。バター250
gとマスタード小さじ1杯を加えて、布で漉す。

\hypertarget{beurre-maitre-d-hotel}{%
\subsubsection[メートルドテルバター]{\texorpdfstring{メートルドテルバター\footnote{メートルドテル
  maître d'hôtel
  とは直訳すれば「館{[}やかた{]}の主」あるいは「館の指導者」の意だが、時代および王家あるいは貴族やブルジョワの館、近現代のレストランにおいてそれぞれ異なった意味で用いられる職名。(1)王家においては
  grand
  maîtreグランメートルを補佐する仕事として食卓関連の仕事を取り仕切る職のこと。王と親しくすることが出来るために、有力貴族がこの職に就くことを希望することが多かったという。(2)大貴族や大ブルジョワの館において、食材の手配やワインの管理、料理人の選抜などの一切を取り仕切り、とりわけ宴席においてはメニュー作りが重要な仕事のひとつとして課される職。歴史上もっとも有名なメートルドテルのひとり、ヴァテルはこの職に相当する。コンデ公に仕え、シャンティイ城でルイ14世らを招いて数日にわたって開催された千人規模大宴席の一切を取り仕切り、最後に手配した魚が届かないと誤解して自害した。彼が息を引きとってからすぐ後に魚は大量に届けられた、という(\protect\hyperlink{sauce-chantilly}{ソース・シャンティイ}訳注も参照)。(3)近代から20世紀中葉にかけて、とりわけ料理人がオーナーではないレストランの場合はメニューの決定、ソムリエおよび給仕人の指揮、客の応対などを担当し、その店で最高のサービス技術を誇る者のつく職名とされた。なお、現代ではほとんど「給仕長」程度の意味しか持たなくなってしまった職名といえる。上記を総合すると、この
  beurre à la maître d'hôtel
  という名称は「当家(当店)特製のバター」あるいは「当家(当店)自慢のバター」程度の意味ということになる。実際のところ、この名称の由来などは不明だが、たとえばフランソワ・マランFrançois
  Marin(生没年不詳)の著書、『コモス神の贈り物』のタイトルページに記された著者の肩書は「スビーズ元帥のメートルドテル、フランソワ・マラン」となっているように、本来はもっとも料理に精通した者の就く役職であった。このため、maître
  d'hôtel-cuisinier
  という語も18、19世紀には用いられていた。つまり、直接的に包丁を握り鍋を振ることはなくても、献立を組み、料理のレシピを考えるのもまたメートルドテルの重要な仕事であった。それを踏まえてカレームは1822年に、それ以前の主要な宴席の献立を詳細に分析した『フランスのメートルドテル』を出版した。つまり、カレームもまた、食卓外交の裏側でメートルドテル=キュイジニエの役割を果たしていたということになる。カレームをたんなるパティシエや料理人という現代的な狭い職の枠にはめて捉えることの出来ない時代だったとも言えよう。それは、エスコフィエについても言えることであり、初版および第二版の末尾には献立例が掲載され、第三版以降は『メニューの本』として独立させたが、総料理長であるということは即ちかつて貴族の館に仕えたメートルドテルの仕事を勤めるに他ならない、ということを示唆しているし、その点は現代の一流ホテルにおいてもあまり変化していないと思われる。この合わせバターの名称も含めた原型のひとつとして、上述のマラン『コモス神の贈り物』第2巻には、「いんげん豆のメートルドテル風」というレシピがある。これは水から茹でたいんげん豆を湯をきってから鍋に入れ、バター、パセリ、エシャロットの細かいみじん切り、塩、こしょうで味付けし、最後にレモン果汁かヴィネガー少々で仕上げるというもの(p.380)。カレームの未完の名著『19世紀フランス料理』第3巻では、「鯖用のメートルドテルバター」として、イジニー産バター8オンス(約250
  g弱)と大きめのレモン1個分の搾り汁、細かく刻んだパセリ大さじ2杯、塩2つまみ強、細かく挽いたこしょう1つまみ弱を木杓子を使ってよく混ぜ合わせる。食欲がわくような調味を心掛けるべし、とある(pp.128-129)。また、同じくヴィアールの『王国料理の本』(内容は1806年初版の『帝国料理の本』の改訂版であり、毎年のように改版され続けているために歴史的に貴重な史料)1846年版では、冷製メートルドテルとして、鍋に
  \(\frac{1}{4}\)ポンドのバターとパセリ少々、エシャロットのみじん切り少々、塩、粗挽きこしょう、レモン果汁を入れ、木杓子でよく練る。これを肉料理あるいは魚料理の下にでも、中にでも、上にでも流すといい、とある(p.48)。このように、
  19世紀前半にはメートルドテルバターの性格がほぼ定着していたと言えよう。}}{メートルドテルバター}}\label{beurre-maitre-d-hotel}}

\frsub{Beurre à la Maître-d'hôtel}

\index{あわせはたー@合わせバター!めーとるとてるはたー@メートルドテルバター}
\index{めーとるとてる@メートルドテル!はたー@---バター}
\index{beurre@beurre!maitre hotel@--- à la Maître d'hôtel}
\index{maitre hotel@maître d'hôtel!beurre@Beurre à la ---}

バター250
gをポマード状に柔らかくする。パセリのみじん切り大さじ1杯強と塩8
g、こしょう1 g、レモン
\(\frac{1}{4}\)個分の果汁を加えてよく混ぜ合わせる。

\hypertarget{nota-beurre-maitre-d-hotel}{%
\subparagraph{【原注】}\label{nota-beurre-maitre-d-hotel}}

このメートルドテルバターに大さじ1杯のマスタードを加えるのもバリエーションとしてお勧め。とりわけ牛、羊肉や魚のグリル焼きによく合う。

\hypertarget{beurre-manie}{%
\subsubsection{ブールマニエ}\label{beurre-manie}}

\frsub{Beurre Manié}

\index{あわせはたー@合わせバター!ふーるまにえ@ブールマニエ}
\index{ふーるまにえ@ブールマニエ} \index{beurre@beurre!manie@--- Manié}
\index{beurre manie@beurre manié}

これはマトロットの煮汁などに、手早くとろみ付けをするのに用いる。小麦粉75
gにバター100 gの割合が原則\footnote{このバターと小麦粉の割合は絶対というわけではなく、本書でもしばしば異なる割合で作ったブールマニエを用いる指示が見られる。}。

ブールマニエでとろみを付けたソースは、その後は出来るだけ沸騰させないこと。さもないと、生の小麦粉の不快な味が強まる危険性があるからだ。

\hypertarget{beurre-marchand-de-vin}{%
\subsubsection[ブール・マルシャンドヴァン]{\texorpdfstring{ブール・マルシャンドヴァン\footnote{「ワイン商人風」の意。煮詰めた赤ワインをバターを混ぜ込むところからの名称だろう。}}{ブール・マルシャンドヴァン}}\label{beurre-marchand-de-vin}}

\frsub{Beurre Marchand de vin}

\index{あわせはたー@合わせバター!ふーるまるしやんとうあん@ブール・マルシャンドヴァン}
\index{わいんしようにん@ワイン商人 ⇒ マルシャンドヴァン!ふーる@ブール・マルシャンドヴァン}
\index{まるしやんとうあん@マルシャンドヴァン!ふーる@ブール・マルシャンドヴァン}
\index{beurre@beurre!marchand vin@--- Marchand de vin}
\index{marchand vin@marcand de vin!beurre@Beurre ---}

赤ワイン2 dLに細かいみじん切りにしたエシャロット25
gを加えて半量になるまで煮詰める。塩1つまみ、挽きたて\footnote{原文
  poivre de moulin
  (ポワーヴルドムラン)、直訳すると「ミルで挽いたこしょう」だが、その場合は即座に使用するのが一般的なので、あえて「挽きたてのこしょう」と訳している。}(または粗く砕いた\footnote{原文
  mignonette
  (ミニョネット)。ミルを用いずに、包丁の側面などで圧し砕いたこしょうを指す。})こしょう1つまみ、溶かした\protect\hyperlink{glace-de-viande}{グラスドヴィアンド}大さじ1杯、ポマード状に柔らかくしたバター150
g、レモン\(\frac{1}{4}\)個分の果汁とパセリのみじん切り大さじ1杯を加える。全体をよく混ぜ合わせる。

\ldots{}\ldots{}グリル焼きにした牛リブロース\footnote{原文 entrecôte
  grillé(アントルコット グリエ)。}用。

\hypertarget{beurre-a-la-meuniere}{%
\subsubsection{ムニエル用バター}\label{beurre-a-la-meuniere}}

\frsub{Beurre à la Meunière}

\index{あわせはたー@合わせバター!むにえるようはたー@ムニエル用バター}
\index{むにえる@ムニエル!はたー@---用バター}
\index{beurre@beurre!meuniere@--- à la Meunière}
\index{meuniere@meunière (à la)!beurre@Beurre à la ---}

焦がしバターに、提供直前にレモン果汁数滴を加えたもの。

\ldots{}\ldots{}魚の「ムニエル\footnote{小麦粉をまぶして、バターで焼く手法および仕立て。原文にある
  à la meunière
  を直訳すると「粉挽き女風」の意。かつては主に水車を動力として石臼などを用い小麦を挽き、その後「ふるい」にかけていた。粉挽き職人は小麦粉の粉塵をかぶって真っ白になっていることが多かったところから付いた料理名。}」用。

\hypertarget{beurre-de-montpellier}{%
\subsubsection[モンペリエバター]{\texorpdfstring{モンペリエバター\footnote{南フランスの都市。モンプリエのようにも発音される。どちらも正しい。複数の発音が正しいとされる例として有名なもののひとつ。}}{モンペリエバター}}\label{beurre-de-montpellier}}

\frsub{Beurre de Montpellier}

\index{あわせはたー@合わせバター!もんへりえはたー@モンペリエバター}
\index{もんへりえ@モンペリエ!はたー@---バター}
\index{beurre@beurre!monpellier@--- de Montpellier}
\index{montepellier@Montpellier!beurre@Beurre de ---}

銅の鍋に湯を沸かし、クレソンの葉とパセリの葉、セルフイユ、シブレット、エスゴラゴンを同量ずつ計90〜100
gと、ほうれんそうの葉25
gを投入する。これとは別の鍋で同時に、エシャロットの細かいみじん切り40
gを下茹でする。ハーブは湯をきって冷水にさらす。しっかり圧し絞って余計な水気を取り除く。エシャロットも同様にする。これらを鉢に入れてすり潰す。

中くらいのサイズのコルニション3個と、水気を絞ったケイパー大さじ1杯、小さなにんにく1片、アンチョビのフィレ4枚を加える。全体が滑らかなペースト状になったら、バター750
gと固茹で卵の黄身3個、生の卵黄2個を加える。混ぜながら、最後に植物油2
dLを少しずつ加える。目の細かい漉し器か布で漉し、泡立て器で混ぜて滑らかにする。塩味を\ruby{調}{ととの}え、カイエンヌごく少量で風味を引き締める。

\ldots{}\ldots{}魚の冷製料理に添える。ビュッフェの場合には魚に覆いかけて供する。

\hypertarget{beurre-de-montpellier-pour-croutonnage-de-plats}{%
\subsubsection{装飾用モンペリエバター}\label{beurre-de-montpellier-pour-croutonnage-de-plats}}

\frsub{Beurre de Montpellier pour Croûtonnage de plats}

\index{あわせはたー@合わせバター!そうしよくようもんへりえはたー@装飾用モンペリエバター}
\index{もんへりえ@モンペリエ!そうしよくようはたー@装飾用---バター}
\index{beurre@beurre!beurre monpellier cretonnage@--- de Montpellier pour croûtonnage de plats}
\index{montepellier@Montpellier!beurre cretonnage@Beurre de --- pour Croûtonnage de plats}

モンペリエバターを装飾のためだけに作る場合には、植物油と茹で卵の黄身、生の卵黄は用いずに作る。平皿に流し入れて均等な厚みにしてやると細部の装飾作業が容易になる。

\hypertarget{beurre-de-moutarde}{%
\subsubsection{マスタードバター}\label{beurre-de-moutarde}}

\frsub{Beurre de Moutarde}

\index{あわせはたー@合わせバター!ますたーとはたー@マスタードバター}
\index{ますたーと@マスタード!はたー@---バター}
\index{beurre@beurre!moutarde@--- de Moutarde}
\index{moutarde@moutarde!beurre@Beurre de ---}

フランス産マスタード大さじ1
\(\frac{1}{2}\)杯をポマード状に柔らかくしたバター250
gに混ぜ込み、冷蔵する。

\hypertarget{beurre-noir-pour-les-grands-services}{%
\subsubsection[格式ある宴席用焦がしバター]{\texorpdfstring{格式ある宴席用焦がしバター\footnote{直訳すると、「大規模で格式の高い宴席において供する黒バター」。かつてのフランス式サービスによる宴席ではルルヴェと呼ばれる非常に壮麗な装飾を施した肉料理、魚料理がポタージュの後に供された。このバターはそういったケースを想定している。実際、カレームが焦がしバターについてこの「黒バター」beurre
  noir
  という表現を好んで用いていたことからも、『料理の手引き』の時代においてはやや大時代的な、過去の華やかな宴席のためのもの、というイメージだったと考えられる。}}{格式ある宴席用焦がしバター}}\label{beurre-noir-pour-les-grands-services}}

\frsub{Beurre noir pour les grands services}

\index{あわせはたー@合わせバター!かくしきあるえんせきようこかしはたー@格式ある宴席用焦がしバター}
\index{くろ@黒!はたー@---バター} \index{はたー@バター!こかし@焦がし---}
\index{beurre@beurre!noir grands services@--- noir pour les grands services}
\index{noir@noir(e)!beurre@Beurre --- pour les grands services}

(仕上がり10人分\footnote{原文 proportion pour un service
  (プロポルスィオンプーランセルヴィス)、直訳すると「1サーヴィスの分量」。17、18世紀から20世紀初頭にかけて宴席での人数の単位に
  service
  (セルヴィス)という語があてられた。原則として8〜12人分とされたが、ごく大雑把に10人前と捉えていい。舞踏会も含め、大規模で華やかな宴席が頻繁に行なわれていた時代においては、ある程度大まかに料理の単位を決めておくことで、食材の手配から仕込み、調理などを効率化していた。このため『料理の手引き』のレシピのほとんどは仕上がり量が1
  serviceになるよう記されている。})

バター125
gをフライパンに入れて火にかけて溶かし、茶色くなるまで加熱する。布で漉して湯煎にかける。\ruby{微温}{ぬる}くなったら、粗く砕いたこしょう\footnote{mignonette
  (ミニョネット)。こしょうの粒を肉叩きや包丁の側面などで押し潰して砕いたもの。}を加えて煮詰めたヴィネガー小さじ1杯を加える。提供直前に、丁度いい温度になるまで温めなおす。揚げたパセリの葉とケイパー大さじ1杯を料理にのせてから、この焦がしバターをかけてやる。

\hypertarget{beurre-de-noisette}{%
\subsubsection[ブール・ド・ノワゼット]{\texorpdfstring{ブール・ド・ノワゼット\footnote{焦がしバターのことを一般にbeurre
  noisette(ブールノワゼット)と呼ぶので、混同しないよう注意。}}{ブール・ド・ノワゼット}}\label{beurre-de-noisette}}

\frsub{Beurre de noisette}

\index{あわせはたー@合わせバター!ふーるとのわせつと@ブール・ド・ノワゼット}
\index{のわせつと@ノワゼット!ふーるとのわせつと@ブール・ド・---}
\index{beurre@beurre!noisette@--- de noisette}
\index{noisette@noisette!beurre@Beurre de ---}

⇒ \protect\hyperlink{beurre-d-aveline}{ブール・ダヴリーヌ}参照。

\hypertarget{beurre-de-paprika}{%
\subsubsection{パプリカバター}\label{beurre-de-paprika}}

\frsub{Beurre de Paprika}

\index{あわせはたー@合わせバター!はふりかはたー@パプリカバター}
\index{はふりか@パプリカ!はたー@---バター}
\index{beurre@beurre!Paprika@--- de Paprika}
\index{paprika@paprika!beurre@Beurre de ---}

玉ねぎのみじん切り大さじ1杯とパプリカ4
gをバターでいい色合いになるまで炒め、ポマード状に柔らかくしておいたバター250
gに混ぜる。布で漉す。

\hypertarget{beurre-de-pimentos}{%
\subsubsection{赤ピーマンバター}\label{beurre-de-pimentos}}

\frsub{Beurre de Pimentos}

\index{あわせはたー@合わせバター!あかひーまんはたー@赤ピーマンパプリカバター}
\index{あかひーまん@赤ピーマン!はたー@---バター}
\index{ほわうろん@ポワヴロン ⇒ 赤ピーマン}
\index{beurre@beurre!pimentos@--- de Pimentos}
\index{pimentos@pimentos!beurre@Beurre de ---}
\index{poivron@poivron ⇒ pimento!beurre pimentos!Beurre de Pimentos}

ブレゼ\footnote{野菜のブレゼの方法については\protect\hyperlink{}{第13章野菜料理}参照。}した赤いポワヴロン\footnote{原文
  poivron。日本の青果では「パプリカ」と呼ばれる肉厚で苦みの少ない品種。「カリフォルニア・ワンダー」が代表的品種。未熟なものは緑色だが完熟すると真っ赤になる。また、熟すと黄色、紫などになる品種もある。}100
gをバター250 gと合わせて細かくすり潰し、布で漉す。

\hypertarget{beurre-de-pistache}{%
\subsubsection{ピスタチオバター}\label{beurre-de-pistache}}

\frsub{Beurre de Pistache}

\index{あわせはたー@合わせバター!ひすたちおはたー@ピスタチオバター}
\index{ひすたちお@ピスタチオ!はたー@---バター}
\index{beurre@beurre!pistache@--- de Pistache}
\index{pistache@pistache!beurre@Beurre de ---}

殻から剥いて湯剥きしたばかりのピスタチオ150
gを、水数滴を加えながら細かくすり潰す。パター250 gを加え、布で漉す。

\hypertarget{beurre-a-la-polonaise}{%
\subsubsection{ポーランド風バター}\label{beurre-a-la-polonaise}}

\frsub{Beurre à la Polonaise}

\index{あわせはたー@合わせバター!ほーらんとふうはたー@ポーランド風バター}
\index{ポーランド@ポーランド!はたー@---風バター}
\index{beurre@beurre!polonaise@--- à la Polonaise}
\index{polonais@polonais(e)!beurre@Beurre à la ---e}

バター250 gをヘーゼルナッツ色\footnote{原文 cuire à la noisette
  (キュイーラノワゼット)すなわち「茶色く」なるまで火を通すということ。現代では、焦がしバターのことを
  beurre noisette
  (ブールノワゼット)と呼ぶことが多いが、本書においては\protect\hyperlink{beurre-de-noisette}{ヘーゼルナッツバター}という項目を立てているために、混同を避ける意味で、このような表現になっていると思われる。}になるまで火を通す。丁度いい色合いになったら、上等なパンの身60
gを投入する。

\hypertarget{beurre-de-raifort}{%
\subsubsection[レフォールバター]{\texorpdfstring{レフォールバター\footnote{ホースラディッシュ、西洋わさび。}}{レフォールバター}}\label{beurre-de-raifort}}

\frsub{Beurre de Raifort}

\index{あわせはたー@合わせバター!れふおーる@レフォールバター}
\index{れふおーる@レフォール!はたー@---バター}
\index{beurre@beurre!raifort@--- de Raifort}
\index{raifort@raifort!beurre@Beurre de ---}

器具を用いておろしたレフォール50 gを鉢に入れてすり潰す。バター250
gを加え、布で漉す。

\hypertarget{beurre-ravigote}{%
\subsubsection{ブール・ラヴィゴット/ブール・ヴェール}\label{beurre-ravigote}}

\frsub{Beurre Ravigote ou Beurre vert}

\index{あわせはたー@合わせバター!らういこつと@ブール・ラヴィゴット}
\index{あわせはたー@合わせバター!うえーる@ブール・ヴェール}
\index{みどり@緑 ⇒ ヴェール/ヴェルト!ブール・ヴェール}
\index{らういこつと@ラヴィゴット!ふーる@ブール・---}
\index{beurre@beurre!ravigote@--- Ravigote}
\index{beurre@beurre!vert@--- vert}
\index{ravigote@ravigote!beurre@Beurre ---}
\index{vert@vert(e)!beurre@Beurre ---}

⇒ \protect\hyperlink{beurre-chivry}{ブール・シヴリ}参照。

\hypertarget{beurre-de-saumon-fume}{%
\subsubsection{スモークサーモンバター}\label{beurre-de-saumon-fume}}

\frsub{Beurre de Saumon fumé}

\index{あわせはたー@合わせバター!すもーくさーもん@スモークサーモンバター}
\index{すもーくさーもん@スモークサーモン!はたー@---バター}
\index{beurre@beurre!saumon fumet@--- de Saumon fumé}
\index{saumon fume@saumon fumé!beurre@Beurre de ---}

スモークサーモン100 gとバター250 gを鉢に入れてよくすり潰し、布で漉す。

\hypertarget{beurre-de-truffe}{%
\subsubsection{トリュフバター}\label{beurre-de-truffe}}

\frsub{Beurre de Truffe}

\index{あわせはたー@合わせバター!とりゆふ@トリュフバター}
\index{とりゆふ@トリュフ!はたー@---バター}
\index{beurre@beurre!truffe@--- de Truffe}
\index{truffe@truffe!beurre@Beurre de ---}

真黒な黒トリュフ100
gと\protect\hyperlink{sauce-bechamel}{ベシャメルソース}小さじ1杯を鉢に入れてすり潰す。良質なバター200
gを加え、布で漉す。

\hypertarget{beurres-printaniers}{%
\subsubsection[ブール・プランタニエ各種]{\texorpdfstring{ブール・プランタニエ\footnote{printanier
  (プランタニエ)春の、を意味する語で、とりわけ春先の「はしり」の野菜を用いる場合によくこの表現があてられる。}各種}{ブール・プランタニエ各種}}\label{beurres-printaniers}}

\frsub{Beurres Printaniers}

\index{あわせはたー@合わせバター!ふーるとふらんたにえ@ブール・プランタニエ}
\index{ふらんたにえ@プランタニエ!ふーる@ブール・---}
\index{beurre@beurre!printaniers@--- Printaniers}
\index{printanier@printanier(ère)!beurre@Beurres Printaniers}

野菜の合わせバターはポタージュやソースの仕上げによく用いられる。

野菜はまず、それぞれの種類に応じた方法で火を通すこと。例えば、にんじんやナヴェ\footnote{原文
  navet 蕪のことだが、日本の蕪とは調理特性および風味が異なるので注意。}の場合はバターを加えて弱火で蒸し煮してからコンソメで煮る。緑の野菜、例えばプチポワ\footnote{原文
  petits pois
  いわゆるグリンピースのことだが、20世紀以降、だんだん若どりのものが好まれる傾向が強まっており、日本で一般的なグリンピースと比較するとかなり小さめの段階で収穫されるものが多く、火入れに必要な時間もごく短かい傾向にある。直径7〜8mm程度の若どりのものはグリンピース特有の青臭さが少なく、フレッシュであれば生食でも美味しい。フランスあるいはイタリア産の冷凍品が多く出回っているが加熱必須。}、さやいんげん\footnote{原文
  haricots verts (アリコヴェール)。}、アスパラガスの穂先などの場合は、しっかり下茹でして火を通す。

その後、野菜と同じ重さのバターとともに鉢に入れてすり潰し、布で漉す。

\hypertarget{coulis-divers}{%
\subsubsection[クリ各種]{\texorpdfstring{クリ\footnote{coulis
  (クリ)の基本概念としては、ピュレよりは水分の多いもの、と解していいのだが、ここではやや特殊な用法となっていることに注意。また、日本の調理現場では「クーリ」と呼ぶ傾向が根強く残っている。しかし、フランス語として見たとき、この語それ自体のアクセント(フランス語のアクセントは長音)は最後の
  i の音にある。ou (発音記号/ u /
  )は日本語にない音で、多くの日本人の耳には強く感じられるために、このような習慣が付いたのだと思われる。少なくとも日本語的な発音で「クーリ」と覚えても、フランス語としては通じない可能性が高いので注意。}各種}{クリ各種}}\label{coulis-divers}}

\frsub{Coulis divers}

\index{くーり@クーリ} \index{くり@クリ!}
\index{coulis divers@Coulis divers}

\begin{itemize}
\item
  エクルヴィス\footnote{ヨーロッパザリガニ。詳しくは\protect\hyperlink{sauce-bavaroise}{バイエルン風ソース}訳注参照。}の殻、または
\item
  クルヴェット\footnote{小海老。詳しくは\protect\hyperlink{sauce-aux-crevettes}{ソース・クルヴェット}訳注参照。}の胴や殻、または
\item
  オマールやラングスト\footnote{原文 langouste ≒ 伊勢エビ。}の胴にあるクリーム状の部分や卵、コライユ\footnote{朱色がかった内子のこと。}
\end{itemize}

を鉢に入れてすり潰す。

すり潰した材料100 gあたり大さじ4杯の新鮮な生クリームを加え、布で漉す。

これらのクリは提供直前に用いること。使い方はこの「合わせバター」の節冒頭に記しておいたので参照されたい。

\hypertarget{huile-de-crustaces}{%
\subsubsection{甲殻類のオイル}\label{huile-de-crustaces}}

\frsub{Huile de Crustacés}

\index{あふら@油 ⇒ オイル!こうかくるい@甲殻類の---}
\index{おいる@オイル!こうかくるい@甲殻類の---}
\index{こうかくるい@甲殻類!オイル@甲殻類のオイル}
\index{huile@huile!crustaces@--- de Crustacés}
\index{crustace@crustacé!huile@Huile de ---s}

このレシピは、オマールやラングストに添えるマヨネーズの仕上げに加えるため考案されたもので、言ってみればマヨネーズの新しいバリエーションだ。

使えるだけの甲殻類の殻などをすり潰し、バターではなくオイルを同量、つまり甲殻類と同じ重さだけ加える。つまり、オイルの重さが1
dLあたり95
gくらい、あるいは、すり潰した甲殻類の殻が大さじ6杯に対してオイル1
dLということになる。

甲殻類の殻をすり潰してペースト状になってきたら、すりこ木でよく混ぜながら油を少量ずつ加えていく。

まず目の細かい漉し器で漉し、さらに布で漉し、冷やしておく。このオイルは絶対に熱を与えてはいけない。
\end{recette}