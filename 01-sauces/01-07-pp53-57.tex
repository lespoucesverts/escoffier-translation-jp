\hypertarget{ux30d6ux30fcux30ebux30b3ux30f3ux30ddux30bc}{%
\section{ブール・コンポゼ}\label{ux30d6ux30fcux30ebux30b3ux30f3ux30ddux30bc}}

\vspace{0\zw}

\hypertarget{ux30b0ux30eaux30ebux30bdux30fcux30b9ux306eux88dcux52a9ux6750ux6599ux30aaux30fcux30c9ux30d6ux30ebux7528}{%
\subsection{グリル、ソースの補助材料、オードブル用}\label{ux30b0ux30eaux30ebux30bdux30fcux30b9ux306eux88dcux52a9ux6750ux6599ux30aaux30fcux30c9ux30d6ux30ebux7528}}

\vspace*{-1.5\zw}

\hypertarget{beurres-composuxe9s-pour-adjuvants-de-sauces-et-hors-doeuvre}{%
\subsection{Beurres Composés pour Adjuvants de Sauces et
Hors-d'oeuvre}\label{beurres-composuxe9s-pour-adjuvants-de-sauces-et-hors-doeuvre}}

\index{ふーるこんほせ@ブール・コンポゼ}
\index{はたー@バター ⇒ ブール・コンポゼ}
\index{あわせはたー@合わせバター ⇒ ブール・コンポゼ}
\index{みつくすはたー@ミックスバター ⇒ ブール・コンポゼ}
\index{beurre@Beurres Composés}

\hypertarget{ux6982ux8aac}{%
\subsection{概説}\label{ux6982ux8aac}}

本書においてレシピを掲載しているブール・コンポゼ\footnote{合わせバター、ミックスバター。}のうちのほとんどは、
甲殻類のブール・コンポゼを除いて、料理に直接用いられることがとても少な
い。とはいえ、ブール・コンポゼはさまざまなシチュエーションで役に立つ。
ポタージュでは野菜のブール・コンポゼが、その他のブール・コンポゼはソー
ス作りにおいて有用だ。ソースの風味と性格を明確に伝える決め手になるから
だ。

だから、読者である料理人諸君には、この概説に書いてあることを真剣に読みとっていただきたい。
\href{原文における内容矛盾。この後のパラグラフは甲殻類のバターについての注意点ばかりが目立つ}{}

甲殻類のバターについては、経験上、湯煎にかけながら煮出して\footnote{infuser
  アンフュゼ。}から、氷
水で冷やした陶製の容器に布で漉し入れるのがいい。そうすれば、冷たい状態
で作るよりも赤みがきれいに出る。逆に、熱によって風味の繊細さは失なわれ
てしまい、雑味さえも出てしまう。

この問題点を解決するために、我々は二種類の違うバターを作るという方式を
採ることにした。ひとつは甲殻類の胴のクリーム状の部分と切りくずあるいは
身そのものを生のバターとともに鉢ですり潰して、目の細かい網で裏漉しする
か、布で漉すというもの。このバターはソースに完璧もというべき風味を添え
てくれる。とりわけベシャメルソースをベースとしたソースの場合はそうだ。

もうひとつは、甲殻類の殻だけを用いて、熱して作るものだ。これは「色付け」
の役割しか持たない。この方式はまことに素晴しい結果を得られるので、ぜひ
とも実行していただきたい。

場合によっては、我々はバターを同様の上等な生クリームに代えることがある。
生クリームのほうがバターよりも、素材の持つ風味や香気をよく吸収する。こ
うすればソースやポタージュの仕上げに加えるのに文句ないクリ\footnote{Coulis
  水分のやや多いピュレをイメージするといい。}を作ることが 出来るわけだ。

色付け用のバターを使うと、ソースがきれいに色付き、個性的なソースとなる。
どんな場合でも、カルミン色素\footnote{コチニール色素ともいう。ラックカイガラムシなどを原料として抽出し
  た色素。かつて食品工業において多用されたが、アレルゲンとなることが
  わかり、使用は減っている。現在は代替品としてビーツから抽出したビー
  トレッドなどがの使用が増えている。}よりもずっといい。カルミン色素はソース
やポタージュにくすんだ、なさけない色合いしか与えてはくれないのだ。

ブール・コンポゼは一般的に、使う際にその都度作る\footnote{原文 au moment
  (オモモン)その都度、の意。à la minute アラミニュッ
  ト、と呼ぶ調理現場もある。}ものだが、作り置き
しておかなければならない場合は、白い紙で円筒形に包んで冷蔵保管すること。
\begin{recette}
\hypertarget{ux306bux3093ux306bux304fux30d0ux30bfux30fc}{%
\subsubsection{にんにくバター}\label{ux306bux3093ux306bux304fux30d0ux30bfux30fc}}

\hypertarget{beurre-d-ail}{%
\paragraph{Beurre d'Ail}\label{beurre-d-ail}}

\index{はたー@バター!ふーるこんほせ@ブール・コンポゼ!にんにくはたー@にんにくバター}
\index{ふーるこんほせ@ブール・コンポゼ!にんにくはたー@にんにくバター}
\index{にんにく@にんにく!はたー@---バター}
\index{beurre@beurre!beurres composes@Beurres Composés!beurre d'ail@Beurre d'Ail}
\index{beurres composes@Beurres Composés!beurre d'ail@Beurre d'Ail}
\index{ail@ail!beurre@Beurre d'---}

皮を剥いたにんにく200 gを強火でしっかり茹でる\footnote{生のにんにくには胃腸を刺激する酵素が含まれているが、熱により不活
  性化するので、よく火を通す必要がある。}。よく湯をきってから、鉢
に入れてすり潰し、バター250 gと合わせ、布で漉す。

\maeaki

\hypertarget{ux30a2ux30f3ux30c1ux30e7ux30d3ux30d0ux30bfux30fc}{%
\subsubsection{アンチョビバター}\label{ux30a2ux30f3ux30c1ux30e7ux30d3ux30d0ux30bfux30fc}}

\hypertarget{beurre-d-anchois}{%
\paragraph{Beurre d'Anchois}\label{beurre-d-anchois}}

\index{はたー@バター!ふーるこんほせ@ブール・コンポゼ!あんちよひはたー@アンチョビバター}
\index{ふーるこんほせ@ブール・コンポゼ!あんちよひはたー@アンチョビバター}
\index{あんちよひ@アンチョビ!はたー@---バター}
\index{beurre@beurre!beurres composes@Beurres Composés!beurre d'anchois@Beurre d'Anchois}
\index{beurres composes@Beurres Composés!beurre d'anchois@Beurre d'Anchois}
\index{anchois@anchois!beurre@Beurre d'---}

アンチョビのフィレ200 gをよく洗い、しっかり水気を絞る。これを鉢に入れ
て細かくすり潰す。バター250 gを加えて布で漉す。
\end{recette}