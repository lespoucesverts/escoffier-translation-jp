\href{未、原文対照チェック}{} \href{未、日本語表現校正}{}
\href{未、その他修正}{} \href{未、原稿最終校正}{}

\hypertarget{marinades-et-saumures}{%
\section[マリナードとソミュール]{\texorpdfstring{マリナードとソミュール\footnote{マリナードはマリネ液とも言う。marinade
  \textless{} mariner
  (マリネ)語源はラテン語のmare(海)。古フランス語では「海で泳ぐ、海に潜る」の意で使われていたが、16世紀には既に、料理用語として用いられていたようだ。ラブレー『ガルガンチュアとパンタグリュエル』第四の書(1548年)において、lancerons
  marinez (マリネしたブロシェの幼魚)という表現が見られる。なおブロシェ
  brochet
  はノーザンパイク、和名キタカワカマス。川カマス属の淡水、汽水魚。この場面はパンタグリュエルに「小斉」のご馳走として捧げられた料理のリストの一部であり、「塩漬けのメルルーサ、卵料理各種、モリュ(塩漬けにした鱈)、アドック(塩漬け後に燻製にした鱈)」などとともに列挙されており、いずれも塩辛いから、食後の消化をよくするために飲むワインの量が倍になった(p.681)とある。したがって、lancerons
  marinezのマリネとは「海水あるいは塩水に漬けた」の意に解釈されよう。一方、ソミュールについては、11世紀末頃に、「保存のため漬け込む塩水」の意味で
  salmuire
  という語形が使用され、16世紀には「塩水およびその他の液体からなるもの」としてsaumureという現在とおなじ語形が記録されている。マリナードとソミュールが明確に分化したのはおそらく17世紀頃。1651年刊ラ・ヴァレーヌ『フランス料理の本』に見られるマリナードの語には曖昧さが残っているが、例えば
  \emph{Poulets
  marinez}(鶏のマリネ)というレシピは「鶏を開いて叩き、しっかり味付けしたヴィネガーに漬ける。小麦粉をまぶすか、卵と小麦粉で作った衣を付けてラードで揚げる。マリナードに戻し入れて軽く弱火で煮てから供する(p.36)」。また、\emph{Longe
  de
  mouton}(仔羊の腰肉のロースト)は、「棒状に切った豚背脂をラルデ針を使って刺し込み、串を刺してローストする。玉ねぎ、塩、こしょう、ごく少量のオレンジまたはレモンの外皮(ゼスト)とブイヨンとヴィネガーでマリナードを作る。肉に火が通ったら、ソース(マリナード)とともに弱火で煮込む。とろみ付けには小麦粉をラードで茶色くなるまで炒めたもの、すなわち後代のルーの原型といえるものを少々加える(p.80)」とあり、別の項目では「(串を刺した肉の下の受け皿にある)マリナードを小まめにかけながらローストする(p.106)」とある。全体的な印象としてはラ・ヴァレーヌのマリナードとは中世のドディーヌにヴィネガーを効かせたもののようにも受け取れるが、最初に見たように、「漬け込む」ものとしてもヴィネガーを用いている点に注目すべきだろう。これは18、19世紀に引継がれ、1756年マラン『コモス神の贈り物』第1巻において、\emph{Cervelle
  de veau en
  marinade}(仔牛の脳のマリナード仕立て)では、血抜きした仔牛の脳を豚背脂のシートで包みブイヨン少々で茹で、「冷ましてからヴィネガーもしくはレモン果汁に漬け込む。その後、水気をきって溶き卵に浸し、パン粉をつけて揚げる。小麦粉を溶いた揚げ衣に浸して揚げてもいい(p.206)」とある。19世紀初頭のヴィアールも同様で、『帝国料理の本』初版(1806年)において、\emph{Pieds
  d'agneau en marinade} (仔羊の足のマリナード仕立て)などいくつかの
  marinadeを冠するレシピが掲載されている。肝心のマリナードについての記述は欠落しているが、この版においてはよく見られる現象。なお、仔羊の足のマリナード仕立ては、マリナードがない場合は「塩、こしょう、ビネガーに、茹でた仔羊の足を漬けてから、揚げ衣を付けて揚げる(p.214)」となっている。1814年ボヴィリエ『調理技法』では「加熱マリナード」のレシピが掲載されている。これは、卵くらいの大きさのバターを鍋に入れ、輪切りにしたにんじん1、2本、同様にした玉ねぎ、ローリエの葉1枚、にんにく1片、タイム、バジル、枝ごとのパセリ、シブール(≒葱)2〜3本を加えて強火で炒める。野菜が色付きはじめたら、約250
  mLの白ワインヴィネガーと約0.5
  Lの水を注ぎ、塩、こしょうする。そのまま沸かしてから漉し、必要に応じて使う(pp.60-61)、というもの。もっとも、仔牛の脳のマリナード仕立てなどマランのレシピと大差ない揚げものも同書に掲載されている。また、1834年版のオドにおいても鶏のマリナードはラ・ヴァレーヌのものと同工異曲のものに留まっている。1837年版でロースト用マリナードの項が追加され、豚背脂とにんにく1片を細かく刻み、パセリ1つまり、塩、こしょう、ヴィネガー大さじ1杯、油大さじ4杯を合わせてよく混ぜる
  (p.419)。1853年版ではマリネしたうなぎのグリル焼き、というレシピが掲載される。これは、皮を剥いてぶつ切りにし、バターでソテーしたうなぎを深皿に並べ、塩、こしょうハーブ、マッシュルーム、細かく刻んだエシャロットとシブールを被せ、油大さじ1杯をかける。2〜3時間マリネしたら、パン粉をまぶしてグリル焼きする(p.310)というもの。また、
  mariner(マリネ)という動詞は、オドの1834年版でもに、ノロ鹿の腿肉のローストにおいて、「オリーブオイルと塩で5〜6時間マリネする」
  (p.155)という記述が見られる。1867年刊グフェ『料理の本』では、ヴィネガーをベースとしたソースとしてのマリナード(p.404)と仕立てとしてのマリナードがある。後者の例としては
  \emph{Tête de veau en marinade}
  (仔牛の頭 マリナード仕立て)が好例だろう。仔牛の頭肉半分を3
  cm角に切り、下茹でしてから水にさらし、牛脂と小麦粉、香草類を加えた湯で茹でる。これを、塩、こしょう、油、ヴィネガーに1時間漬け込む。水気をきって揚げ衣を付けて油で揚げる(p.156)。ここでは肉を漬け込む液体としてmarinadeの語が用いられている。このように、marinadeという名詞とmariner「漬け込む」という動詞の用法に若干の不統一が見られるため、『料理の手引き』におけるマリナードすなわちマリネ液、という概念は比較的新しいものと思われる。}}{マリナードとソミュール}}\label{marinades-et-saumures}}

\frsec{Marinades et Saumures}

\index{marinade saumures@marinade et saumures} \index{marinade@marinade}
\index{まりなーととそみゆーる@マリナードとソミュール}
\index{まりなーと@マリナード}

\vspace{1\zw}

マリナードとソミュールにはいろいろな種類があるが、最終的な目的は同じで、

\begin{enumerate}
\def\labelenumi{\arabic{enumi}.}
\item
  素材に料理で使う香辛料やハーブの香りを浸み込ませる
\item
  ある種の肉を柔らかくさせる
\item
  場合によっては保存のために用いる。とりわけ温度と湿度で素材が駄目になってしまうような場合。さらに、目指す料理の仕上がりに合わせて素材の状態を調節する
\end{enumerate}
\begin{recette}
\hypertarget{marinade-instantanee}{%
\subsubsection{即席マリナード}\label{marinade-instantanee}}

\frsub{Marinade instantanée}

\index{marinade@marinade!marinade instantanee@marinade instantanée}
\index{まりなーと@マリナード!そくせき@即席---}

このマリナードはすぐに素材を使う場合、例えば赤身肉のグリル焼きや、ガランティーヌ、テリーヌ、パテのような冷製料理の補助材料\footnote{具体的には\protect\hyperlink{farces}{ファルス}のこと。}にする肉に用いる。

\begin{enumerate}
\def\labelenumi{\arabic{enumi}.}
\item
  グリル焼きにする肉の場合\ldots{}\ldots{}ごく薄くスライスしたエシャロットとパセリの枝、タイムの枝、ローリエの葉を肉の上に散らす。量は適宜加減すること。レモン果汁
  \(\frac{1}{2}\)個分に対して油大さじ1杯の割合で、上からかけてやる。
\item
  仔牛、ジビエのフィレ肉、ハム、豚背脂などを細かく切ったもの\footnote{原文
    lardon
    (ラルドン)、通常は拍子木状に切ったものを言うが、ここではファルスとして後で細かく挽くことになるので、形状はあまり問題にならない。}の場合\ldots{}\ldots{}塩こしょうしてから、白ワイン3、コニャック3、油1の割合のマリナードを上からかけてやる。
\end{enumerate}

ここで用いた風味付けの材料は、後でファルスにする際に加えることになる。

いずれの場合でも、マリナードに浸した肉を小まめに裏返してやり、マリナードがよく浸み込むようにしてやること。

\hypertarget{marinade-crue-pour-viandes-de-boucherie-ou-venaison}{%
\subsubsection{牛、羊肉および大型ジビエ用の非加熱マリナード}\label{marinade-crue-pour-viandes-de-boucherie-ou-venaison}}

\frsub{Marinade crue pour viandes de boucherie ou venaison}

\index{marinade@marinade!marinade crue viande boucherie venaison@marinade crue pour viande de boucherie ou venaison}
\index{まりなーと@マリナード!うしひつしおおかたしひえようひかねつ@牛、羊肉および大型ジビエ用非加熱---}

(仕上がり2 L分)

\begin{itemize}
\item
  香味素材\ldots{}\ldots{}にんじん100 g、玉ねぎ100 g、エシャロット40
  g、セロリ30 g、にんにく2片、パセリの枝3本、タイム1枝、ローリエの葉
  \(\frac{1}{2}\)枚、大粒のこしょう6個、クローブ2本。
\item
  使用する液体\ldots{}\ldots{}白ワイン1 \(\frac{1}{4}\) L、ヴィネガー5
  dL、油2 \(\frac{1}{2}\) dL。
\item
  作業手順\ldots{}\ldots{}マリネする素材に塩とこしょうを振る。にんじん、玉ねぎ、エシャロットを薄切り\footnote{émincer
    (エマンセ)薄切りにする、スライスする。}にし、半量を容器の底に敷く。容器の大きさは素材とマリナードがぴったり入る程度のものを用いること。素材を入れて、残りの香味野菜で蓋をするようにして、白ワインとヴィネガー、油を注ぎ入れる。
\end{itemize}

冷蔵し、マリネ液に漬かった素材を小まめに裏返してやること。

\hypertarget{marinade-cuite-pour-viandes-de-boucherie-ou-venaison}{%
\subsubsection{牛、羊肉および大型ジビエ用の加熱マリナード}\label{marinade-cuite-pour-viandes-de-boucherie-ou-venaison}}

\frsub{Marinade cuite pour viandes de boucherie ou venaison}

\index{marinade@marinade!marinade cuite viande boucherie venaison@marinade cuite pour viande de boucherie ou venaison}
\index{まりなーと@マリナード!うしひつしおおかたしひえようかねつ@牛、羊肉および大型ジビエ用加熱---}

(仕上がり2 L分)

\begin{itemize}
\item
  香味素材\ldots{}\ldots{}非加熱マリナードと同じ材料で同じ分量
\item
  使用する液体\ldots{}\ldots{}白ワイン1 \(\frac{1}{2}\) L、ヴィネガー3
  dL、油2 \(\frac{1}{2}\) dL。
\item
  作業手順\ldots{}\ldots{}鍋に油を熱し、ごく薄くスライスしたにんじん、玉ねぎ、エシャロットおよびその他の香味素材を軽く色付くまで炒める。

  白ワインとヴィネガーを注ぎ、弱火で約30分間火を通す。

  必ず、マリナードが完全に冷めてからマリネする素材にかけること。
\end{itemize}

\hypertarget{marinade-crue-ou-cuite-pour-grosse-venaison}{%
\subsubsection[とりわけ大型のジビエ用、非加熱および加熱マリナード]{\texorpdfstring{とりわけ大型のジビエ\footnote{具体的には赤鹿
  cerf(セール)
  や猪、トナカイの成獣など。ニホンジカやエゾジカはcerfに分類されるので、これを参考にするといいだろう。}用、非加熱および加熱マリナード}{とりわけ大型のジビエ用、非加熱および加熱マリナード}}\label{marinade-crue-ou-cuite-pour-grosse-venaison}}

\frsub{Marinade crue ou cuite pour grosse venaison}

\index{marinade@marinade!marinade crue cuite grosse venaison@marinade crue ou cuite pour grosse venaison}
\index{まりなーと@マリナード!とりわけおおかたのしひえようひかねつおよひかねつ@とりわけ大型のジビエ用非加熱および加熱---}

(仕上がり2 L分)

\begin{itemize}
\item
  香味素材\ldots{}\ldots{}牛、羊肉および大型ジビエ用のマリナードと同じだが、ローズマリー12
  gを追加する。
\item
  使用する液体\ldots{}\ldots{}ヴィネガー16 dL、油4 dL。
\item
  作業手順\ldots{}\ldots{}非加熱、加熱ともに作業手順は上記のレシピのとおり。
\end{itemize}

\hypertarget{marinade-cuite-pour-le-mouton-en-chevreuil}{%
\subsubsection[羊のシュヴルイユ仕立て用の加熱マリナード]{\texorpdfstring{羊のシュヴルイユ仕立て用の加熱マリナード\footnote{\protect\hyperlink{sauce-chevreuil}{ソース・シュヴルイユ}参照。}}{羊のシュヴルイユ仕立て用の加熱マリナード}}\label{marinade-cuite-pour-le-mouton-en-chevreuil}}

\frsub{Marinade cuite pour le mouton en chevreuil}

\index{marinade@marinade!marinade cuite mouton en chevreuil@marinade cuite pour le mouton en chevreuil}
\index{まりなーと@マリナード!ひつしのしゆうるいゆしたてようのかねつまりなーと@羊のシュヴルイユ仕立て用加熱---}

(仕上がり2 L分)

\begin{itemize}
\item
  香味素材\ldots{}\ldots{}上記のとおりの分量の素材に、ジュニパーベリー\footnote{セイヨウネズの実。ジンの香り付けに用いられている。}10粒とバジル1つまみ、ローズマリー1つまみを足す。
\item
  使用する液体\ldots{}\ldots{}牛、羊および大型ジビエ用の加熱マリナードと同じ。
\item
  作業手順\ldots{}\ldots{}鍋に油を熱し、薄切りにしたにんじん、玉ねぎ、エシャロットおよびその他の香味素材を軽く色付くまで炒める。

  白ワインとヴィネガーを注ぎ、弱火で約30分間火を通す。
\end{itemize}

\hypertarget{marinade-cuite-pour-le-mouton-en-chamois}{%
\subsubsection[羊のシャモワ仕立て用の加熱マリナード]{\texorpdfstring{羊のシャモワ仕立て\footnote{オートザルプ県の山岳地帯およびピレネー山脈に生息する野生の山羊。ピレネー山脈のものは
  Isard
  (イザール)と呼ばれる。若い獣の肉は大型ジビエのなかでもとりわけ美味とされる。成獣の肉は固く、しっかりマリネする必要があると言われている。しばしばノロ鹿と比較される。ここでは、羊肉を白ワインベースのマリナードに漬け込む仕立て、すなわちシュヴルイユ仕立てとの対比として、赤ワインでより強い風味のマリナードに漬け込むことで、シャモワ仕立てとしている。なお、本書にシャモワ仕立てのレシピは掲載されていない。シュヴルイユ仕立てと同様と考えていい。}用の加熱マリナード}{羊のシャモワ仕立て用の加熱マリナード}}\label{marinade-cuite-pour-le-mouton-en-chamois}}

\frsub{Marinade cuite pour le mouton en chamois}

\index{marinade@marinade!marinade cuite mouton en chevreuil@marinade cuite pour le mouton en chevreuil}
\index{まりなーと@マリナード!ひつしのしやもわしたてようのかねつまりなーと@羊のシャモワ仕立て用加熱---}

(仕上がり2 L分)

\begin{itemize}
\item
  香味素材\ldots{}\ldots{}非加熱マリナードと同じ分量の素材に、ジュニパーベリー\footnote{セイヨウネズの実。ジンの香り付けに用いられている。}15粒とバジル15
  g、ローズマリー15 gを足す。
\item
  使用する液体\ldots{}\ldots{}良質な赤ワイン1 \(\frac{1}{2}\)
  L、ヴィネガー3 dL、油2 \(\frac{1}{2}\) dL。
\item
  作業手順\ldots{}\ldots{}上記と同じ。

  このマリナードに上等な赤ワインを使える場合には、素材の量を次のように調整すること。赤ワイン12
  dL、ワインヴィネガー6 dL、油は上記の分量とする。

  ワインの酸味の強さによっては、ヴィネガーの量をワインと同量にすることさえ可能。
\end{itemize}

\hypertarget{observation-sur-les-marinades}{%
\subparagraph{マリナードについての注意事項}\label{observation-sur-les-marinades}}

\ldots{}\ldots{} 1.
加熱マリナードを使用するのは、素材へのマリナードの浸透作用を促進するのが目的。素材をマリナードに漬け込む時間は、加熱、非加熱ともに、素材の種類と大きさ、気温、環境の変化を勘案して決めること。

\begin{enumerate}
\def\labelenumi{\arabic{enumi}.}
\setcounter{enumi}{1}
\tightlist
\item
  一般的な牛、羊肉と肉質の柔らかい大型ジビエに使うマリナードに純粋な酢酸を用いるのは絶対にやめておくこと。酢酸の腐食作用によって肉の風味が失なわれてしまうからだ\footnote{この注記は第二版から。内容が当時の知見にもとづいたものであることに注意。ただし、19世紀には木酢液を原料として工業用の氷酢酸が既に製造されていた。また、タンパク質はpHの変化によって分解されるので、マリナードにヴィネガーを加えるのは理にかなっている。なお、肉を柔らかくする効果のあるタンパク質分解酵素(プロテアーゼ)の代表的なひとつであるパパインの発見は1940年代になってからのこと。パイナップルに含まれているブロメラインの効果は経験的に知られていた可能性もあるが、この酵素が60℃で不活性化することが広く知られるようになったのは、少なくとも日本では比較的近年のことに過ぎない。}。猪、赤鹿\footnote{cerf
    (セール)、ニホンジカやエゾジカもフランス語で表現するとこれに含まれるので、これらの料理について
    chevreuil (しゅう゛るいゆ)ノロ鹿の名をつけるは、厳密には誤り。}、トナカイなどの固い肉についても、純粋な酢酸だけを使うのは不可。
\end{enumerate}

\hypertarget{conservation-des-marinades}{%
\subsubsection{マリナードの保存方法}\label{conservation-des-marinades}}

\frsub{Conservation des marinades}

\index{marinade@marinade!conservation marinades@conservation des marinades}
\index{まりなーと@マリナード!ほそんほうほう@---の保存方法}

マリナードを長期間保存しておく必要がある場合には、とりわけ夏場は、本書で示した分量に対して2〜3
gのホウ酸を加えるといい。

さらに、夏のあいだは2日に一度、冬季は4〜5日に一度、マリナードを沸騰させ、冷めたら毎回そのマリナードに使っているのと同じワインを
2 dLとヴィネガー1 dLを足してやること。
\end{recette}
\hypertarget{saumures}{%
\subsection[ソミュール]{\texorpdfstring{ソミュール\footnote{この見出しは第四版のみ。初版〜第三版にかけては、マリナードとソミュールのレシピの間に区切りをつけるものは何も挿入されていない。}}{ソミュール}}\label{saumures}}

\frsecb{Saumures}

\index{saumure@saumure} \index{そみゆーる@ソミュール}
\begin{recette}
\hypertarget{saumure-au-sel}{%
\subsubsection{塩漬け用ソミュール}\label{saumure-au-sel}}

\frsub{Saumure au sel}

\index{saumure@saumure!sel@--- au sel}
\index{そみゆーる@ソミュール!しおつけよう@塩漬け用---}

このソミュールは、グレーソルト\footnote{フランス語は sel gris
  (セルグリ)または gros gris (グログリ)。灰色がかった粗塩。}1
kgに対して硝石\footnote{原文 salpêtre
  (サルペートル)硝酸カリウム。殺菌作用と、肉類を赤く発色させる効果を持つ。現代の日本では亜硝酸カリウム、亜硝酸ナトリウムが使われることが多い。いずれも日本では劇物指定されているが、シャルキュトリ(豚肉加工品の製造)においては不可欠とも言われるな薬品であり、とりわけボツリヌス菌対策の効果が大きい。そのため劇物ではあるが、食品添加物として認められており、使用限界量が厳密に定められている(食品添加物は国あるいは地域によって扱いが異なるので注意)。硝酸塩あるいは亜硝酸塩による肉の赤い発色を「着色料によるもの」と誤認する消費者は少なくない。これはかつて「魚肉ソーセージ」がコチニール色素でピンク色に染められていたことから連想される誤認と思われる。また、食品添加物イコール毒という安直な考えから忌避する消費者も少なくないのは事実だろう。こうしたことから、現代日本のレストランでは、製造後すぐに提供可能であるために、これら硝酸塩、亜硝酸塩の類を用いないところもある。}40
gの割合で作る。この硝石入りの塩の総量は、塩漬けにする肉の数と大きさで決まる。素材が完全に覆えて、重しが出来る分量とすること。

\begin{itemize}
\tightlist
\item
  作業手順\ldots{}\ldots{}肉を塩漬けにする前にまず、太い針を充分深く刺して穴を何箇所も空ける。次に硝石の粉末を肉の表面にすり付ける。塩1
  kgあたりタイム1枝、ローリエの葉
  \(\frac{1}{2}\)枚を加えて肉と塩を容器に詰める。
\end{itemize}

\hypertarget{saumure-liquide-pour-langues}{%
\subsubsection[舌肉用の液体ソミュール]{\texorpdfstring{舌肉用の液体ソミュール\footnote{このソミュールに舌肉を漬け込むと、硝石の作用で舌肉が赤く発色する。それを拍子木状などに切って鶏やフィレ肉の表面に、同様に切ったトリュフや豚背脂などとともに刺して装飾することが19世紀〜20世紀初頭までよく行なわれた。現代ではほとんど行なわれなくなった装飾方法。この場合はあくまでも料理の装飾を目的としたものであり、牛や豚の舌肉を保存食として利用する場合には塩漬けや燻製などの方法も用いられる。}}{舌肉用の液体ソミュール}}\label{saumure-liquide-pour-langues}}

\frsub{Saumure liquide pour langues}

\index{saumure@saumure!liquide langues@--- liquide pour langues}
\index{そみゆーる@ソミュール!したにくようのえきたい@舌肉用の液体---}

\begin{itemize}
\item
  材料\ldots{}\ldots{}水5 L、グレーソルト2.25 kg、硝石150
  g、茶色いカソナード\footnote{砂糖きびを原料とした粗糖。通常は茶褐色のものが多く「赤糖」とも呼ばれるが、精白したものもある。精製が不完全であるため独特の風味があり、料理および製菓でしばしば用いられる。}
  300 g、こしょう12 g、ジュニパーベリー12粒、タイム1枝、ローリエの葉1
  枚。
\item
  作業手順\ldots{}\ldots{}充分な大きさの鍋に材料を全て入れ、強火で沸騰させる。その後、完全に冷めてから、針で穴を複数空けて硝石をしっかりすり込んだ舌肉を入れた容器に注ぎ込む。平均的な重さの舌肉を漬け込む期間は冬季で8日間、夏季は6日間。
\end{itemize}

\hypertarget{grande-saumure}{%
\subsubsection[グランドソミュール]{\texorpdfstring{グランドソミュール\footnote{この項は第二版で追加された。通常はシャルキュティエすなわちシャルキュトリ専門の職人が行なう規模のものであり、料理人の仕事の範疇をやや越えるとも考えられる。}}{グランドソミュール}}\label{grande-saumure}}

\frsub{Grande saumure}

\index{saumure@saumure!grande@grande ---}
\index{そみゆーる@ソミュール!くらんと@グランド---}

(仕上がり50 L分)

\begin{itemize}
\tightlist
\item
  水\ldots{}\ldots{}50 L
\item
  塩\ldots{}\ldots{}25 kg
\item
  硝石\ldots{}\ldots{}2.7 kg
\item
  カソナード\ldots{}\ldots{}1.6 kg
\item
  作業手順\ldots{}\ldots{}メッキされた銅の鍋に材料を全て入れ、強火にかける。沸騰したら、皮を剥いたじゃがいも1個を投入する。じゃがいもが浮いてくるようであれば、じゃがいもが沈みはじめる寸前まで水を足す。逆に、じゃがいもが完全に底まで沈んでしまうようなら、じゃがいもが水面に見えてくるまで煮詰める必要がある。
\end{itemize}

ソミュールがちょうどいい具合になったら、鍋を火から外して、このソミュールで漬け込み槽に注ぎ込む。漬け込み槽の素材は、スレート製、岩製、セメント製、あるいはレンガ製でしっかりエナメル引きしたものを用いること。

漬け込み槽の底に、木製の網を敷き、その上に漬け込む肉を置くといい。肉が槽の底面に直接当たっていると、肉の下側にソミュール液が浸透しない可能性がある。

漬け込む肉は、たとえ小さなものであっても、専用の携行可能な注入器具を使ってソミュールを内部に注入してから、漬け込み槽に入れてやること。この準備作業を怠ると、肉全体が均等に塩漬けにならない可能性がある。肉の中心部がちょうどいい塩加減になる頃には外側は塩が強すぎるということになってしまうのだ。牛のランプ、イチボなどの塊肉で、4〜5
kgの大きさの場合は、ソミュール液を注入してやる方法を使えば8日間で漬かる。

牛舌肉をこの方法で漬ける場合は、出来るだけ新鮮なものを用いる必要がある。軟骨部分をきれいに取り除いてやり、肉叩きか麺棒で丁寧に叩いてやる。ブリデ針\footnote{主として鶏などの手羽や腿をまとめて整形し、その形状を保つよう糸で縫う際に用いる縫い針。}を使って、表面全体に刺し穴をつけてやる。それからソミュールに漬け込むが、何らかの重しをして浮き上がらないようにしてやること。

\hypertarget{observation-grande-saumure}{%
\subparagraph{【原注】}\label{observation-grande-saumure}}

ソミュールはマリナードほどは腐敗しにくいとはいえ、天候が悪い時季などはとりわけ、よく様子を見て、時々は沸騰させてやるのがいい。沸騰させれば多少は濃縮されてしまうから、本文記載の方法でじゃがいもを用いて、毎回少量の水を加える必要がある。
\end{recette}