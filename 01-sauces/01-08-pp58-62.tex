\hypertarget{ux30deux30eaux30caux30fcux30c9ux3068ux30bdux30dfux30e5ux30fcux30eb1}{%
\section[マリナードとソミュール]{\texorpdfstring{マリナードとソミュール\footnote{マリナードはマリネ液とも言う。marinade
  \textless{} mariner (マリネ)語源
  はラテン語のmare(海)だが、中世フランス語ではもっぱら「海で泳ぐ、
  海に潜る」の意で使われていたが、16世紀には既に、料理用語として用い
  られていた。ラブレー『ガルガンチュアとパンタグリュエル』第四の書
  (1548年)において、lancerons marinez (マリネしたブロシェの幼魚)
  という表現が見られる。なおブロシェ brochet はノーザンパイク、和名
  キタカワカマス。川カマス属の淡水、汽水魚。この場面はパンタグリュエ
  ルに「小斉」の御馳走として捧げられた料理のリストの一部であり、「塩
  漬けのメルルーサ、卵料理各種、モリュ(塩漬けにした鱈)、アドック
  (塩漬け後に燻製にした鱈)」などとともに列挙されているものであり、
  いずれも塩辛いために、それらを調理したものを食べて消化をよくするた
  めに飲むワインの量が倍になった(p.681」とある。したがって、 lancerons
  marinezのマリネとは「海水あるいは塩水に漬けた」の意に解
  釈されよう。一方、ソミュールについては、11世紀末頃に、「保存のため
  漬け込む塩水」の意味で salmuire という語形が使用され、16世紀には
  「塩水およびその他の液体からなるもの」としてsaumureという現在とお
  なじ語形が記録されている。マリナードとソミュールが明確に分化したの
  はおそらく17世紀頃で、1651年刊ラ・ヴァレーヌ『フランス料理の本』に
  見られるマリナードの語には曖昧さを免れないものもあるが、例えば
  \emph{Poulets marinez}
  (鶏のマリネ)というレシピは「鶏を開いて叩き、しっ
  かり味付けしたヴィネガーに付ける。提供直前に小麦粉をまぶすか、卵と
  小麦粉で作った衣を付け、ラードで揚げる。揚がったらマリナードに戻し
  入れて軽く弱火で煮てから供する(p.36)」あるいは \emph{Longe de
  mouton} 
  (仔羊の腰肉のロースト)のレシピは、「よく熟成させてから棒状に切っ
  た豚背脂をラルデ針を使って刺し込み、串を刺してローストする。玉ねぎ、
  塩、こしょう、ごく少量のオレンジまたはレモンの外皮{[}ゼスト{]}とブイ
  ヨンとヴィネガーでマリナードを作る。肉に火が通ったら、ソース{[}マリ
  ナード{]}とともに弱火で煮込む。とろみ付けには前項同様にした小麦粉
  {[}小麦粉をラードで茶色くなるまで炒めたもの、すなわち『料理の手引き』
  の時代のルーの原型{]}を少々加える(p.80)」とあり、別の項目では「(串
  を刺した肉の下の受け皿にある)マリナードを小まめにかけながら{[}アロ
  ゼしながら{]}ローストする(p.106)」という表現がある。レシピ数からい
  うとラ・ヴァレーヌにおいてマリナードとは中世のドディーヌにヴィネガー
  を効かせたもののようにも受け取れるが、最初に見たように、「漬け込む」
  ものとしてヴィネガーを用いている点に注目すべきだろう。この流れは18、
  19世紀に引継がれる。1756年マラン『コモス神の贈り物』第1巻において、
  \emph{Cervelle de veau en
  marinade}(仔牛の脳のマリナード仕立て)などの
  レシピがあり、血抜きした仔牛の脳を豚背脂のシートで包みブイヨン少々
  で茹で、「冷ましてからヴィネガーもしくはレモン果汁に漬け込む。その
  後、水気をきって溶き卵に浸し、パン粉をつけて揚げる。小麦粉を溶いた
  揚げ衣に浸して揚げてもいい(p.206)」とある。19世紀のヴィアールでも
  同様の料理は見られる。『帝国料理の本』初版(1806年)において、
  \emph{Pieds d'agneau en marinade} 仔羊の足のマリナード仕立てなどいくつ
  かのmarinadeを冠するレシピが掲載されている。肝心のマリナードについ
  ての記述は欠落しているが、この版においてはよく見られる現象。なお、
  仔羊の足のマリナード仕立ては、マリナードがない場合には「塩、こしょ
  う、ビネガーに茹でた仔羊の足を漬けてから、揚げ衣を付けて揚げる
  (p.214)」となっている。1814年ボヴィリエ『調理技法』では「加熱マリ
  ナード」のレシピが掲載されている。これは、卵くらいの大きさのバター
  を鍋に入れ、輪切りにしたにんじん1、2本、同様にした玉ねぎ、ローリエ
  の葉1枚、にんにく1片、タイム、バジル、枝ごとのパセリ、シブール{[}≒
  葱{]}2〜3本を加えて強火で炒める。野菜が色付きはじめたら、約250mlの
  白ワインヴィネガーと約0.5 Lの水を注ぎ、塩、こしょうする。そのまま
  沸かして、漉し器で漉して、必要に応じて使う(pp.60-61)、というもの。
  もっとも、仔牛の脳のマリナード仕立てなどマランのレシピと大差ない揚
  げものが目に付く。また、1834年版のオドにおいても鶏のマリナードはラ・
  ヴァレーヌのものと同工異曲に留まっている。1837年版ではロースト用マ
  リナードの項が追加され、豚背脂とにんにく1片を細かく刻み、パセリ1つ
  まり、塩、こしょう、ヴィネガー大さじ1杯、油大さじ4杯を合わせてよく
  混ぜる(p.419)。1853年版ではマリネしたうなぎのグリル焼き、というレ
  シピが掲載される。これは、皮を剥いてぶつ切りにし、バターでソテーし
  たうなぎを深皿に並べ、塩、こしょうハーブ、マッシュルーム、細かく刻
  んだエシャロットとシブールを被せ、油大さじ1杯をかける。2〜3時間マ
  リネしたら、パン粉をまぶしてグリル焼きする(p.310)というもの。いっ
  ぽう、mariner(マリネ)という動詞については、オドの1834年版で既に、
  ノロ鹿の腿肉のローストにおいて、「オリーブオイルと塩で5〜6時間マリ
  ネする」(p.155)という記述が見られる。1867年刊グフェ『料理の本』に
  おいては、ヴィネガーをベースとしたソースとしてのマリナード(p.404)
  と仕立てとしてのマリナードがあるが、後者もこんにちの概念に近く、例
  えば \emph{Tête de veau en marinade}
  (仔牛の頭 マリナード仕立て)は、 仔牛の頭肉半分を3
  cm角に切り、下茹でしてから水にさらし、牛脂と小麦
  粉、香草類を加えたブランで茹でる。これを、塩、こしょう、油、ヴィネ
  ガーに1時間漬け込む。水気をきって揚げ衣を付けて油で揚げる、という
  もの(p.156)。ここでは肉を漬け込む液体としてmarinadeの語が用いられ
  ている。このように、marinadeという名詞とmariner「漬け込む」という
  動詞の用法にややずれが見られるため、『料理の手引き』におけるマリナー
  ドすなわちマリネ液、という概念は19世紀後半になってからのものと思わ
  れる。}}{マリナードとソミュール}}\label{ux30deux30eaux30caux30fcux30c9ux3068ux30bdux30dfux30e5ux30fcux30eb1}}

\hypertarget{marinades-et-saumucres}{%
\subsection{Marinades et Saumures}\label{marinades-et-saumucres}}

\index{marinade saumures@marinade et saumures} \index{marinade@marinade}
\index{saumure@saumure}

マリナードとソミュールにはいろいろな種類があるが、最終的な目的は同じで、

\begin{enumerate}
\def\labelenumi{\arabic{enumi}.}
\item
  素材に料理で使う香辛料やハーブの香りを浸み込ませる
\item
  ある種の肉を柔らかくさせる
\item
  場合によっては保存のために用いる。とりわけ温度と湿度で素材が駄目になってしまうような場合。さらに、目指す料理の仕上がりに合わせて素材の状態を調節する
\end{enumerate}
\begin{recette}
\hypertarget{ux5373ux5e2dux30deux30eaux30caux30fcux30c9}{%
\subsubsection{即席マリナード}\label{ux5373ux5e2dux30deux30eaux30caux30fcux30c9}}

\hypertarget{marinade-instantanuxe9e}{%
\subsubsection{Marinade instantanée}\label{marinade-instantanuxe9e}}

このマリナードはすぐに素材を使う場合、例えば赤身肉のグリル焼きや、ガランティーヌ、テリーヌ、パテのような冷製料理の材料にする肉に用いる。
\end{recette}