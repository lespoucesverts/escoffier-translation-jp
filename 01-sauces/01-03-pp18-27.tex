%%
% Copyright (c) 2018, Pascal Wagler;  
% Copyright (c) 2014--2018, John MacFarlane
% 
% All rights reserved.
% 
% Redistribution and use in source and binary forms, with or without 
% modification, are permitted provided that the following conditions 
% are met:
% 
% - Redistributions of source code must retain the above copyright 
% notice, this list of conditions and the following disclaimer.
% 
% - Redistributions in binary form must reproduce the above copyright 
% notice, this list of conditions and the following disclaimer in the 
% documentation and/or other materials provided with the distribution.
% 
% - Neither the name of John MacFarlane nor the names of other 
% contributors may be used to endorse or promote products derived 
% from this software without specific prior written permission.
% 
% THIS SOFTWARE IS PROVIDED BY THE COPYRIGHT HOLDERS AND CONTRIBUTORS 
% "AS IS" AND ANY EXPRESS OR IMPLIED WARRANTIES, INCLUDING, BUT NOT 
% LIMITED TO, THE IMPLIED WARRANTIES OF MERCHANTABILITY AND FITNESS 
% FOR A PARTICULAR PURPOSE ARE DISCLAIMED. IN NO EVENT SHALL THE 
% COPYRIGHT OWNER OR CONTRIBUTORS BE LIABLE FOR ANY DIRECT, INDIRECT, 
% INCIDENTAL, SPECIAL, EXEMPLARY, OR CONSEQUENTIAL DAMAGES (INCLUDING,
% BUT NOT LIMITED TO, PROCUREMENT OF SUBSTITUTE GOODS OR SERVICES; 
% LOSS OF USE, DATA, OR PROFITS; OR BUSINESS INTERRUPTION) HOWEVER 
% CAUSED AND ON ANY THEORY OF LIABILITY, WHETHER IN CONTRACT, STRICT 
% LIABILITY, OR TORT (INCLUDING NEGLIGENCE OR OTHERWISE) ARISING IN 
% ANY WAY OUT OF THE USE OF THIS SOFTWARE, EVEN IF ADVISED OF THE 
% POSSIBILITY OF SUCH DAMAGE.
%%

%%
% For usage information and examples visit the GitHub page of this template:
% https://github.com/Wandmalfarbe/pandoc-latex-template
%%

\PassOptionsToPackage{unicode=true}{hyperref} % options for packages loaded elsewhere
\PassOptionsToPackage{hyphens}{url}
\PassOptionsToPackage{svgnames*,table}{xcolor}
%
\documentclass[14Q,a4paperpaper,,tablecaptionabove]{scrartcl}
\usepackage{lmodern}
\usepackage{amssymb,amsmath}
\usepackage{ifxetex,ifluatex}
\usepackage{fixltx2e} % provides \textsubscript
\ifnum 0\ifxetex 1\fi\ifluatex 1\fi=0 % if pdftex
  \usepackage[T1]{fontenc}
  \usepackage[utf8]{inputenc}
  \usepackage{textcomp} % provides euro and other symbols
\else % if luatex or xelatex
  \usepackage{unicode-math}
  \defaultfontfeatures{Ligatures=TeX,Scale=MatchLowercase}
\fi
% use upquote if available, for straight quotes in verbatim environments
\IfFileExists{upquote.sty}{\usepackage{upquote}}{}
% use microtype if available
\IfFileExists{microtype.sty}{%
\usepackage[]{microtype}
\UseMicrotypeSet[protrusion]{basicmath} % disable protrusion for tt fonts
}{}
\IfFileExists{parskip.sty}{%
\usepackage{parskip}
}{% else
\setlength{\parindent}{0pt}
\setlength{\parskip}{6pt plus 2pt minus 1pt}
}
\usepackage{hyperref}
\hypersetup{
            pdftitle={エスコフィエ『料理の手引き』全注解},
            pdfauthor={五 島 学},
            pdfborder={0 0 0},
            breaklinks=true}
\urlstyle{same}  % don't use monospace font for urls
\usepackage[margin=2.5cm,includehead=true,includefoot=true,centering]{geometry}
\setlength{\emergencystretch}{3em}  % prevent overfull lines
\providecommand{\tightlist}{%
  \setlength{\itemsep}{0pt}\setlength{\parskip}{0pt}}
\setcounter{secnumdepth}{0}
% Redefines (sub)paragraphs to behave more like sections
\ifx\paragraph\undefined\else
\let\oldparagraph\paragraph
\renewcommand{\paragraph}[1]{\oldparagraph{#1}\mbox{}}
\fi
\ifx\subparagraph\undefined\else
\let\oldsubparagraph\subparagraph
\renewcommand{\subparagraph}[1]{\oldsubparagraph{#1}\mbox{}}
\fi

% Make use of float-package and set default placement for figures to H
\usepackage{float}
\floatplacement{figure}{H}

\usepackage{amsmath}
  \let\equation\gather
  \let\endequation\endgather
\usepackage{amssymb}
\usepackage[no-math]{fontspec}
\usepackage{geometry}
\usepackage{luaotfload}
\usepackage{graphicx}

\usepackage{setspace}


%%%%%%%%% hyperref %%%%%%%%%%%%%
\usepackage{refcount}
\usepackage[unicode=true,hyperfootnotes=false,pageanchor]{hyperref}
\hypersetup{hyperindex=false,%
             breaklinks=true,%
             bookmarks=true,%
             pdfauthor={五島 学},%
             pdftitle={エスコフィエ『料理の手引き』全注解},%
             colorlinks=false%true,%
             %colorlinks=true,%
             citecolor=blue,%
             urlcolor=cyan,%
             linkcolor=magenta,%
             bookmarksdepth=subsubsection,%
             pdfborder={0 0 0},%
             hyperfootnotes=false,%
             plainpages=false,
             }
\urlstyle{same}




%% 欧文フォント設定
% Libertine/Biolinum
\setmainfont[Ligatures=Historic,Scale=1.0]{Linux Libertine O}
\setsansfont[Ligatures=TeX, Scale=MatchLowercase]{Linux Biolinum O} 
%\usepackage{libertine}
\usepackage{unicode-math}
\setmathfont[Scale=1.2]{libertinusmath-regular.otf}
%\unimathsetup{math-style=ISO,bold-style=ISO}
%\setmathfont{xits-math.otf}
%\setmathfont{xits-math.otf}[range={cal,bfcal},StylisticSet=1]




%% Garamond
%\usepackage{ebgaramond-maths}
%\setmainfont[Ligatures=Historic,Scale=1.1]{EB Garamond}%fontspecによるフォント設定

%\usepackage{qpalatin}%palatino

%\setmainfont[Ligatures=Historic,Scale=MatchLowercase]{Tex Gyre Schola}
%\setmainfont[Ligatures=Historic,Scale=MatchLowercase]{Tex Gyre Pagella}
%\setsansfont[Scale=MatchLowercase]{TeX Gyre Heros}  % \sffamily のフォント
%\setsansfont[Scale=MatchLowercase]{TeX Gyre Adventor}  % \sffamily のフォント

%\setmonofont[Scale=MatchLowercase]{Inconsolata}       % \ttfamily のフォント

%\usepackage[cmintegrals,cmbraces]{newtxmath}%数式フォント

\usepackage{luatexja}
\usepackage{luatexja-fontspec}
%\ltjdefcharrange{8}{"2000-"2013, "2015-"2025, "2027-"203A, "203C-"206F}
%\ltjsetparameter{jacharrange={-2, +8}}
\usepackage{luatexja-ruby}

%%%%和文フォント設定
%\usepackage[sourcehan,bold,jis2004,expert,deluxe]{luatexja-preset}%Adobe源ノ明朝、ゴチ
%\usepackage[hiragino-pron,bold,jis2004,expert,deluxe]{luatexja-preset}
%\usepackage[yu-osx,expert,jis2004,bold]{luatexja-preset}
%\usepackage[moga-mogo-ex,bold]{luatexja-preset}

\newopentypefeature{PKana}{On}{pkna} % "PKana" and "On" can be arbitrary string
%%%%明朝にIPAexMincho、ゴチ(太字)にMoboGoBを使う設定。和文カナプロプーショナル使用可能だが読みづらくなる。
\setmainjfont[%
     %YokoFeatures={JFM=prop,PKana=On},%
     %CharacterWidth=AlternateProportional,%
%    CharacterWidth=Proportional,%Mogo, IPAExMinchoには不可
     %Kerning=On,%
     BoldFont={ MoboGoB },%
     ItalicFont={ MoboGoB },%
     BoldItalicFont={ MoboGoExB }%
     % ]{ MogaHMin }
     ]{ IPAExMincho }
     % ]{ IPAmjMincho }
\setsansjfont[%
     %YokoFeatures={JFM=prop,PKana=On},%
     %CharacterWidth=AlternateProportional,%
     % CharacterWidth=Proportional,%Mobo, IPAExGOthicには不可
     %Kerning=On,
     BoldFont={ MoboGoB },%
     ItalicFont={ MoboGoB },%
     BoldItalicFont={ MoboGoExB }%
     % ]{ MoboGo}
     ]{ IPAExGothic }
% %  %%%% 和文仮名プロプーショナルここまで
% %\ltjsetparameter{jacharrange={-2}}%キリル文字%引数に-3を付けるとギリシア文字も可能になるが、%三点リーダーも欧文化されてしまうので注意%


\renewcommand{\bfdefault}{bx}%和文ボールドを有効にする
\renewcommand{\headfont}{\gtfamily\sffamily\bfseries}%和文ボールドを有効にする
%\addfontfeature{Fractions=On}


\defaultfontfeatures[\rmfamily]{Scale=1.2}%効いていない様子
\defaultjfontfeatures{Scale=0.92487}%和文フォントのサイズ調整。デフォルトは 0.962212 倍%ltjsclassesでは不要?
%\defaultjfontfeatures{Scale=0.962212}
%\usepackage{libertineotf}%linux libertine font %ギリシア語含む
%\usepackage[T1]{fontenc}
%\usepackage[full]{textcomp}
%\usepackage[osfI,scaled=1.0]{garamondx}
%\usepackage{tgheros,tgcursor}
%\usepackage[garamondx]{newtxmath}

\usepackage{layout}

%% レイアウト調整(A4Paper,14Q,twoside,ltjsbook.cls) 
%%
\setlength{\hoffset}{0\zw}
\setlength{\oddsidemargin}{0\zw}%タブレット前提の中央配置
\setlength{\evensidemargin}{\oddsidemargin}
% \setlength{\oddsidemargin}{1\zw}%製本時に右ページのみをオフセット
%\setlength{\evensidemargin}{0pt}%
\setlength{\fullwidth}{45\zw}
\setlength{\textwidth}{45\zw}%%ltjsclassesのみ有効
%\setlength{\fullwidth}{159mm}
%\setlength{\textwidth}{159mm}
\setlength{\marginparsep}{0pt}
\setlength{\marginparwidth}{0pt}
\setlength{\footskip}{0pt}
\setlength{\voffset}{-17mm}
\setlength{\textheight}{265mm}
\setlength{\parskip}{0pt}
\setlength{\parindent}{0pt}

%%%% b5j%%%%%%
% \setlength{\hoffset}{-10mm}
% \setlength{\oddsidemargin}{0mm}
% \setlength{\evensidemargin}{0mm}
% %\setlength{\textwidth}{\fullwidth}%%ltjsclassesのみ有効
% \setlength{\fullwidth}{145mm}
% \setlength{\textwidth}{145mm}
% \setlength{\marginparsep}{0pt}
% \setlength{\marginparwidth}{0pt}
% \setlength{\footskip}{0pt}
% \setlength{\voffset}{-10mm}
% \setlength{\textheight}{225mm}
% \setlength{\parskip}{0pt}
%%%ベースライン調整
%\ltjsetparameter{yjabaselineshift=0pt,yalbaselineshift=-.75pt}


%\usepackage{fancyhdr}






%文字サイズ、見出しなどの再定義
\makeatletter
%\renewcommand{\large}{\jsc@setfontsize\large\@xipt{14}}
%\renewcommand{\Large}{\jsc@setfontsize\Large{13}{15}}

\newcommand{\medlarge}{\fontsize{11}{13}\selectfont}
\newcommand{\medsmall}{\fontsize{9.23}{9.5}\selectfont}
\newcommand{\twelveq}{\jsc@setfontsize\twelveq{9.230769}{9.75}\selectfont}
\newcommand{\thirteenq}{\jsc@setfontsize\fourteenq{10}{11}\selectfont}
\newcommand{\fourteenq}{\jsc@setfontsize\fourteenq{10.7692}{13}\selectfont}
\newcommand{\fifteenq}{\jsc@setfontsize\fifteenq{11.53846}{14}\selectfont}

\renewcommand{\chapter}{%
  \if@openleft\cleardoublepage\else
  \if@openright\cleardoublepage\else\clearpage\fi\fi
  \plainifnotempty % 元: \thispagestyle{plain}
  \global\@topnum\z@
  \if@english \@afterindentfalse \else \@afterindenttrue \fi
  \secdef
    {\@omit@numberfalse\@chapter}%
    {\@omit@numbertrue\@schapter}}
\def\@chapter[#1]#2{%
  \ifnum \c@secnumdepth >\m@ne
    \if@mainmatter
      \refstepcounter{chapter}%
      \typeout{\@chapapp\thechapter\@chappos}%
      \addcontentsline{toc}{chapter}%
        {\protect\numberline
        % {\if@english\thechapter\else\@chapapp\thechapter\@chappos\fi}%
        {\@chapapp\thechapter\@chappos}%
        #1}%
    \else\addcontentsline{toc}{chapter}{#1}\fi
  \else
    \addcontentsline{toc}{chapter}{#1}%
  \fi
  \chaptermark{#1}%
  \addtocontents{lof}{\protect\addvspace{10\jsc@mpt}}%
  \addtocontents{lot}{\protect\addvspace{10\jsc@mpt}}%
  \if@twocolumn
    \@topnewpage[\@makechapterhead{#2}]%
  \else
    \@makechapterhead{#2}%
    \@afterheading
  \fi}
\def\@makechapterhead#1{%
  \vspace*{0\Cvs}% 欧文は50pt
  {\parindent \z@ \centering \normalfont
    \ifnum \c@secnumdepth >\m@ne
      \if@mainmatter
        \huge\headfont \@chapapp\thechapter\@chappos%変更
        \par\nobreak
        \vskip \Cvs % 欧文は20pt
      \fi
    \fi
    \interlinepenalty\@M
    \huge \headfont #1\par\nobreak
    \vskip 1\Cvs}} % 欧文は40pt%変更

\renewcommand{\section}{%
    \if@slide\clearpage\fi
    \@startsection{section}{1}{\z@}%
    {\Cvs \@plus.5\Cdp \@minus.2\Cdp}% 前アキ
    % {.5\Cvs \@plus.3\Cdp}% 後アキ
    {.5\Cvs}
    {\normalfont\Large\headfont\bfseries\centering}}%変更

\renewcommand{\subsection}{\@startsection{subsection}{2}{\z@}%
    {\Cvs \@plus.5\Cdp \@minus.2\Cdp}% 前アキ
    % {.5\Cvs \@plus.3\Cdp}% 後アキ
    {.5\Cvs}
  %  {\normalfont\large\headfont\bfseries\centering}} %変更
    {\normalfont\large\headfont\centering}} %変更

\renewcommand{\subsubsection}{\@startsection{subsubsection}{3}{\z@}%
  % {0\Cvs \@plus.8\Cdp \@minus.6\Cdp}%変更
    {1sp \@plus.5\Cdp \@minus.5\Cdp}%変更
    {\if@slide .5\Cvs \@plus.3\Cdp \else \z@ \fi}%
    % {\normalfont\medlarge\headfont\leftskip -1\zw}}
    {\normalfont\medlarge\headfont\leftskip -1\zw}}

\renewcommand{\paragraph}{\@startsection{paragraph}{4}{\z@}%
    {0.5\Cvs \@plus.5\Cdp \@minus.2\Cdp}%
    % {\if@slide .5\Cvs \@plus.3\Cdp \else -1\zw\fi}% 改行せず 1\zw のアキ
    {1sp}%後アキ
    {\normalfont\normalsize\headfont}}
\renewcommand{\subparagraph}{\@startsection{subparagraph}{5}{\z@}%
    {\z@}{\if@slide .5\Cvs \@plus.3\Cdp \else -.5\zw\fi}%
    {\normalfont\normalsize\headfont\hskip-.5\zw\noindent}}  

\newcommand{\frchap}[1]{\vspace*{-2ex}%
 \begin{center}\normalfont\headfont\LARGE\setstretch{0.8}
 \scshape#1\normalfont\normalsize
\end{center}\vspace{0.5\zw}\setstretch{1.0}}

\newcommand{\frsec}[1]{\vspace*{-2ex}%
 \begin{center}\normalfont\headfont\large\setstretch{0.8}
 \scshape#1\normalfont\normalsize
\end{center}\vspace{0.5\zw}\setstretch{1.0}}
  
\newcommand{\frsecb}[1]{\vspace*{-2ex}%
\begin{center}\normalfont\headfont\medlarge\setstretch{0.8}%
  \hspace{1em}\scshape#1\normalfont\normalsize%
\end{center}\vspace{0.5\zw}\setstretch{1.0}}

%\newcounter{frsub}[subsubsection]
%\newcommand{\frsub}{\@startsection{frsub}{6}{\z@}%
%  {1sp}{1sp}%
%  {\normalfont\normalsize\bfseries\baselineskip-.8ex\leftskip-1\zw}}
%\let\frsubmark\@gobble
%\newcommand*{\l@frsub}{%
%          \@tempdima\jsc@tocl@width \advance\@tempdima 16.183\zw
%          \@dottedtocline{7}{\@tempdima}{6.5\zw}}
%\renewcommand{\thefrsub}{6}
%\let\frsub\paragraph

\newenvironment{frsubenv}{\begin{spacing}{0.2}\setlength{\leftskip}{-1\zw}\bfseries}{\end{spacing}\normalfont\normalsize\setlength{\leftskip}{0pt}}
\newcommand{\frsub}[1]{\begin{frsubenv}#1\end{frsubenv}\par\vspace{1.1\zw}}

%\newcommand{\frsub}[1]{\vskip -.8ex\hskip -1\zw\textbf{#1}\leftskip0pt}
%\newcommand{\frsub}{\@startsection{frsub}{6}{\z@}%
%   {-1\zw}% 改行せず 1\zw のアキ
%   {-1\zw}%後アキ     
%   {\normalfont\normalsize\bfseries\leftskip -1\zw\baselineskip -.5ex}}%normalsizeから変更
%\newcommand*{\l@frsub}{%
%          \@tempdima\jsc@tocl@width \advance\@tempdima 16.183\zw
%          \@dottedtocline{5}{\@tempdima}{6.5\zw}}


%%%%%%%%%レシピと本文%%%%%%%%%%%%
\usepackage{multicol}
\setlength{\columnsep}{3\zw}

%%% 本文
\newenvironment{Main}{}{}
%%% レシピ
% \setlength{\columnwidth}{24\zw}
%本文ヨリ小%\small
%\newenvironment{recette}{\setlength{\parindent}{0pt}\begin{small}\begin{spaceing}{0.8}\begin{multicols}{2}}{\end{multicols}\end{spacing}\end{small}}
%本文やや小%\medsmall
%\newenvironment{recette}{\setlength{\parindent}{0pt}\begin{medsmall}\begin{spacing}{0.75}\begin{multicols}{2}}{\end{multicols}\end{spacing}\end{medsmall}}
%本文ナミ(無指定)
\newenvironment{recette}{\setlength{\parindent}{0pt}\begin{spacing}{0.8}\begin{multicols}{2}\setlength\topskip{.8\baselineskip}}{\end{multicols}\end{spacing}}


\makeatother


%%% 脚注番号のページ毎のリセットと脚注位置の調整
%\renewcommand{\footnotesize}{\small}

\makeatletter

\usepackage[bottom,perpage,stable]{footmisc}%
%\setlength{\skip\footins}{4mm plus 4mm}
%\usepackage{footnpag}
\renewcommand\@makefntext[1]{%
  \advance\leftskip 0\zw
  \parindent 1\zw
  \noindent
  \llap{\@thefnmark\hskip0.5\zw}#1}


\let\footnotes@ve=\footnote
\def\footnote{\inhibitglue\footnotes@ve}
\let\footnotemarks@ve=\footnotemark
%\def\footnotemark{\inhibitglue\footnotemarks@ve}
\renewcommand{\footnotemark}{\footnotemarks@ve}%変更
% %\def\thefootnote{\ifnum\c@footnote>\z@\leavevmode\lower.5ex\hbox{(}\@arabic\c@footnote\hbox{)}\fi}
\renewcommand{\thefootnote}{\ifnum\c@footnote>\z@\leavevmode\hbox{}\@arabic\c@footnote\hbox{)}\fi}
%\makeatletter
% \@addtoreset{footnote}{page}
% \makeatother
%\usepackage{dblfnote}
%\usepackage[bottom,perpage]{footmisc}

\makeatother

%subsubsectionに連番をつける
%\usepackage{remreset}

\renewcommand{\thechapter}{}
\renewcommand{\thesection}{\hskip-1\zw}
\renewcommand{\thesubsection}{}
\renewcommand{\thesubsubsection}{}
\renewcommand{\theparagraph}{}


% \makeatletter
% \@removefromreset{subsubsection}{subsection}
% \def\thesubsubsection{\arabic{subsubsection}.}
% \newcounter{rnumber}
% \renewcommand{\thernumber}{\refstepcounter{rnumber} }
% \makeatother
\renewcommand{\prepartname}{\if@english Part~\else {}\fi}
\renewcommand{\postpartname}{\if@english\else {}\fi}
\renewcommand{\prechaptername}{\if@english Chapter~\else {}\fi}
\renewcommand{\postchaptername}{\if@english\else {}\fi}
\renewcommand{\presectionname}{}%  第
\renewcommand{\postsectionname}{}% 節

%リスト環境
\def\tightlist{\itemsep1pt\parskip0pt\parsep0pt}%pandoc対策

\makeatletter
  \parsep   = 0pt
  \labelsep = .5\zw
  \def\@listi{%
     \leftmargin = 0pt \rightmargin = 0pt
     \labelwidth\leftmargin \advance\labelwidth-\labelsep
     \topsep     = 0pt%\baselineskip
     %\topsep -0.1\baselineskip \@plus 0\baselineskip \@minus 0.1 \baselineskip
     \partopsep  = 0pt \itemsep       = 0pt
     \itemindent = -.5\zw \listparindent = 0\zw}
  \let\@listI\@listi
  \@listi
  \def\@listii{%
     \leftmargin = 1.8\zw \rightmargin = 0pt
     \labelwidth\leftmargin \advance\labelwidth-\labelsep
     \topsep     = 0pt \partopsep     = 0pt \itemsep   = 0pt
     \itemindent = 0pt \listparindent = 1\zw}
  \let\@listiii\@listii
  \let\@listiv\@listii
  \let\@listv\@listii
  \let\@listvi\@listii
\makeatother








%% %%%%%%行取りマクロ
% \makeatletter
% \ifx\Cht\undefined
%  \newdimen\Cht\newdimen\Cdp
%  \setbox0\hbox{\char\jis"2121}\Cht=\ht0\Cdp=\dp0\fi
% \catcode`@=11
% \long\def\linespace#1#2{\par\noindent
%   \dimen@=\baselineskip
%   \multiply\dimen@ #1\advance\dimen@-\baselineskip
%   \advance\dimen@-\Cht\advance\dimen@\Cdp
%   \setbox0\vbox{\noindent #2}%
%   \advance\dimen@\ht0\advance\dimen@-\dp0%
%   \vtop to\z@{\hbox{\vrule width\z@ height\Cht depth\z@
%    \raise-.5\dimen@\hbox{\box0}}\vss}%
%   \dimen@=\baselineskip
%   \multiply\dimen@ #1\advance\dimen@-2\baselineskip
%   \par\nobreak\vskip\dimen@
%   \hbox{\vrule width\z@ height\Cht depth\z@}\vskip\z@}
% \catcode`@=12
% \setlength{\parskip}{0pt}
% \setlength{\topskip}{\Cht}
% \setlength{\textheight}{43\baselineskip}
% \addtolength{\textheight}{1\zh}
% \makeatother
 
%%%%%%%%%%%%失敗%%%%%%%%%%%%
%\let\formule\subsubsection
%\renewcommand{\subsubsection}[1]{\linespace{1}{\formule#1}}
%%%%%%%%%%%%失敗%%%%%%%%%%%%






% % PDF/X-1a
%  \usepackage[x-1a]{pdfx}
%  \Keywords{pdfTeX\sep PDF/X-1a\sep PDF/A-b}
%  \Title{「エスコフィエ『料理の手引き』全注解」}
%  \Author{五 島 学}
%  \Org{TeX Users Group}
%  \pdfcompresslevel=0
% \usepackage[cmyk]{xcolor}

%biblatex
%\usepackage[notes,strict,backend=biber,autolang=other,%
%                   bibencoding=inputenc,autocite=footnote]{biblatex-chicago}
%\addbibresource{hist-agri.bib}
\let\cite=\autocite

% % % % 
\date{}



%%%%インデックス準備
%\usepackage{makeidx}
\usepackage{index}
%\usepackage[useindex]{splitidx}
\newindex{src}{idx1}{ind1}{ソース名から料理を探す}
\makeindex
 
 \makeatletter
\renewenvironment{theindex}{% 索引を3段組で出力する環境
    \if@twocolumn
      \onecolumn\@restonecolfalse
    \else
      \clearpage\@restonecoltrue
    \fi
    \columnseprule.4pt \columnsep 2\zw
    \ifx\multicols\@undefined
      \twocolumn[\@makeschapterhead{\indexname}%
      \addcontentsline{toc}{chapter}{\indexname}]%
    \else
      \ifdim\textwidth=\fullwidth
        \setlength{\evensidemargin}{\oddsidemargin}
        \setlength{\textwidth}{\fullwidth}
        \setlength{\linewidth}{\fullwidth}
        \begin{multicols}{3}[\chapter*{\indexname}%
        \addcontentsline{toc}{chapter}{\indexname}]%
      \else
        \begin{multicols}{2}[\chapter*{\indexname}%
        \addcontentsline{toc}{chapter}{\indexname}]%
      \fi
    \fi
    \@mkboth{\indexname}{}%
    \plainifnotempty % \thispagestyle{plain}
    \parindent\z@
    \parskip\z@ \@plus .3\jsc@mpt\relax
    \let\item\@idxitem
    \raggedright
    \footnotesize\narrowbaselines
  }{
    \ifx\multicols\@undefined
      \if@restonecol\onecolumn\fi
    \else
      \end{multicols}
    \fi
    \clearpage
  }
 \makeatother


%%%% 本文中の参照ページ番号表示 %%%%%%%

\makeatletter

%\AtBeginDocument{%
%  \DeclareRobustCommand\ref{\@ifstar\@refstar\@refstar}%
%  \DeclareRobustCommand\pageref{\@ifstar\@pagerefstar\@pagerefstar}}
\let\orig@Hy@EveryPageAnchor\Hy@EveryPageAnchor
\def\Hy@EveryPageAnchor{%
    \begingroup
    \hypersetup{pdfview=Fit}%
    \orig@Hy@EveryPageAnchor
    \endgroup
  }
  \usepackage{etoolbox}
\if@mainmatter{\let\myhyperlink\hyperlink%
\renewcommand{\hyperlink}[2]{\myhyperlink{#1}{#2} [p.\getpagerefnumber{#1}{}] }}
  \AtBeginEnvironment{recette}{%
\let\myhyperlink\hyperlink%
\renewcommand{\hyperlink}[2]{\myhyperlink{#1}{#2} [p.\getpagerefnumber{#1}{}] }}
  \AtBeginEnvironment{Main}{%
\let\myhyperlink\hyperlink%
\renewcommand{\hyperlink}[2]{\myhyperlink{#1}{#2} [p.\getpagerefnumber{#1}{}] }}
% \if@mainmatter{\let\myhyperlink\hyperlink%
% \renewcommand{\hyperlink}[2]{\myhyperlink{#1}{#2} {\ltjsetparameter{yjabaselineshift=0pt,yalbaselineshift=-.75pt}\footnotesize [p.\getpagerefnumber{#1}{}]}}}
%   \AtBeginEnvironment{recette}{%
% \let\myhyperlink\hyperlink%
% \renewcommand{\hyperlink}[2]{\myhyperlink{#1}{#2} {\ltjsetparameter{yjabaselineshift=0pt,yalbaselineshift=-.75pt}\footnotesize [p.\getpagerefnumber{#1}{}]}}}
%   \AtBeginEnvironment{Main}{%
% \let\myhyperlink\hyperlink%
% \renewcommand{\hyperlink}[2]{\myhyperlink{#1}{#2} {\ltjsetparameter{yjabaselineshift=0pt,yalbaselineshift=-.75pt}\footnotesize [p.\getpagerefnumber{#1}{}]}}}


% \def\@@wrindex#1|#2|#3\\{%
%  \if@filesw
%  \ifx\\#2\\%
%   \protected@write\@indexfile{}{% 
%     %\string\indexentry{#1}{\thepage}%%%改変部分。もとは{#1|hyperpage}{\thepage}%
%     \string\indexentry{#1|hyperpage}{\thepage}%%オリジナル%
%         }%
%         \else
%           \HyInd@@@wrindex{#1}#2\\%
%         \fi
%       \fi
%       \endgroup
%       \@esphack
%     }

\makeatother



%%%%% Obsolete Reference Page Numbers 

%\newcommand{\pref}[1]{[p.\pageref{#1}]}
%\newcommand{\pref}[1]{}







%%%% pandoc が三点リーダーを勝手に変える対策
\renewcommand{\ldots}{\noindent…}
%%%%%下線
\usepackage{umoline}
\setlength{\UnderlineDepth}{2pt}
\let\ul\Underline

\newcommand{\maeaki}{}%使用しないので無効化
\newcommand{\atoaki}{\vspace{1.25mm}}
%%分数の表記Obsolete
\usepackage{xfrac}
\let\frac\sfrac
\newcommand{\undemi}{\hspace{.25\zw}$\sfrac{1}{2}$}
\newcommand{\untiers}{\hspace{.25\zw}$\sfrac{1}{3}$}
\newcommand{\deuxtiers}{\hspace{.25\zw}$\sfrac{2}{3}$}
\newcommand{\unquart}{\hspace{.25\zw}$\sfrac{1}{4}$}
\newcommand{\troisquarts}{\hspace{.25\zw}$\sfrac{3}{4}$}
\newcommand{\quatrequatrieme}{\hspace{.25\zw}$\sfrac{4}{4$}}
\newcommand{\uncinquieme}{\hspace{.25\zw}$\sfrac{1}{5}$}
\newcommand{\deuxcinquiemes}{\hspace{.25\zw}$\sfrac{2}{5}$}
\newcommand{\troiscinquiemes}{\hspace{.25\zw}$\sfrac{3}{5}$}
\newcommand{\quatrecinquiemes}{\hspace{.25\zw}$\sfrac{4}{5}$}
\newcommand{\unsixieme}{\hspace{.25\zw}$\sfrac{1}{6}$}
\newcommand{\cinqsixiemes}{\hspace{.25\zw}$\sfrac{5}{6}$}
\newcommand{\quatrequart}{\hspace{.25\zw}$\sfrac{4}{4}$}

\makeatletter
\def\ps@headings{%
  \let\@oddfoot\@empty
  \let\@evenfoot\@empty
  \def\@evenhead{%
    \if@mparswitch \hss \fi
    \underline{\hbox to \fullwidth{\ltjsetparameter{autoxspacing={true}}
%      \textbf{\thepage}\hfil\leftmark}}%
       \normalfont\thepage\hfill\scshape\small\leftmark\normalfont}}%
    \if@mparswitch\else \hss \fi}%
  \def\@oddhead{\underline{\hbox to \fullwidth{\ltjsetparameter{autoxspacing={true}}
        {\if@twoside\scshape\small\rightmark\else\scshape\small\leftmark\fi}\hfil\thepage\normalfont}}\hss}%
  \let\@mkboth\markboth
  \def\chaptermark##1{\markboth{%
    \ifnum \c@secnumdepth >\m@ne
      \if@mainmatter
        \if@omit@number\else
          \@chapapp\thechapter\@chappos\hskip1\zw
        \fi
      \fi
    \fi
    ##1}{}}%
  \def\sectionmark##1{\markright{%
%    \ifnum \c@secnumdepth >\z@ \thesection \hskip1\zw\fi
    \ifnum \c@secnumdepth >\z@ \thesection \hskip-1\zw\fi
    ##1}}}%
\makeatother

\makeatletter
%%%%%%%% Lua GC
\patchcmd\@outputpage{\stepcounter{page}}{%
  \directlua{%
	if jit then
      local k = collectgarbage("count")
      if k>900000 then 
        collectgarbage("collect")
        texio.write_nl("term and log", "GC: ", math.floor(k), math.floor(collectgarbage("count")))
      end
	end
  }%
  \stepcounter{page}%
}{}{}
\makeatother
% \usepackage{vgrid}% here only to help visualize the problem

\makeatletter

\def\srcGlaceDeViande#1#2#3#4{%
  \index[src]{glace de viande@Glace de viande!{#1}@{#2}}%
  \index[src]{くらすとういあんと@グラスドヴィアンド!{#3}@{#4}}}

%%%%%%基本ソース

\def\srcEspagnole#1#2#3#4{%
  \index[src]{espagnole@Espagnole!{#1}@{#2}}
  \index[src]{えすはによる@エスパニョル!{#3}@{#4}}}
 
\def\srcEspagnoleMaigre#1#2#3#4{%
  \index[src]{espagnole maigre@Espagnole maigre!{#1}@{#2}}%
  \index[src]{えすはによるさかな@エスパニョル(魚料理用)!{#3}@{#4}}}

\def\srcDemiGlace#1#2#3#4{%
  \index[src]{demi-glace@Demi-glace!{#1}@{#2}}%
  \index[src]{とうみくらす@ドゥミグラス!{#3}@{#4}}}

\def\srcJusDeVeauLie#1#2#3#4{%
  \index[src]{jus veau lie@Jus de veau lié!{#1}@{#2}}%
  \index[src]{とろみをつけたこうしのしゆ@とろみを付けた仔牛のジュ!{#3}@{#4}}}

\def\srcVeoute#1#2#3#4{%
  \index[src]{veloute@Velouté!{#1}@{#2}}%
  \index[src]{うるて@ヴルテ!{#3}@{#4}}}

\def\srcVeouteDeVolaille#1#2#3#4{%
  \index[src]{veloute de volaille@Velouté de volaille!{#1}@{#2}}%
  \index[src]{とりのうるて@鶏のヴルテ!{#3}@{#4}}}

\def\srcVeouteDePoisson#1#2#3#4{%
  \index[src]{veloute de poisson@Velouté de poisson!{#1}@{#2}}%
  \index[src]{さかなのうるて@魚のヴルテ!{#3}@{#4}}}

\def\srcAllemande#1#2#3#4{%
  \index[src]{allemande@Allemande!{#1}@{#2}}%
  \index[src]{あるまんと@アルマンド!{#3}@{#4}}}

\def\srcSupreme#1#2#3#4{%
  \index[src]{supreme@Suprême!{#1}@{#2}}%
  \index[src]{しゆふれーむ@シュプレーム!{#3}@{#4}}}

\def\srcBechamel#1#2#3#4{%
  \index[src]{bechamel@Bechamel!{#1}@{#2}}%
  \index[src]{へしやめる@ベシャメル!{#3}@{#4}}}

\def\srcTomate#1#2#3#4{%
  \index[src]{tomate@Tomate!{#1}@{#2}}%
  \index[src]{とまと@トマト!{#3}@{#4}}}

%%%%%%%%ブラウン系の派生ソース

\def\srcBigarade#1#2#3#4{%
  \index[src]{bigarade@Bigarade!{#1}@{#2}}%
  \index[src]{ひからーと@ビガラード!{#3}@{#4}}}

\def\srcBordelaise#1#2#3#4{%
  \index[src]{bordelaise@Bordelaise!{#1}@{#2}}%
  \index[src]{ほるとーふう@ボルドー風!{#1}@{#2}}}

\def\srcBourguignonne#1#2#3#4{%
  \index[src]{bourguignonne@Bourguignonne!{#1}@{#2}}%
  \index[src]{ふるこーにゆふう@ブルゴーニュ風!{#3}@{#4}}}

\def\srcBretonne#1#2#3#4{%
  \index[src]{bretonne@Bretonne!{#1}@{#2}}%
  \index[src]{ふるたーにゆふう@ブルターニュ風!{#3}@{#4}}}

\def\srcCerises#1#2#3#4{%
  \index[src]{cerises@Cerises (aux)!{#1}@{#2}}%
  \index[src]{すりーす@スリーズ!{#3}@{#4}}}

\def\srcChampignons#1#2#3#4{%
  \index[src]{champignons@Champignons (aux)!{#1}@{#2}}%
  \index[src]{しやんひによん@シャンピニョン!{#3}@{#4}}}

\def\srcChampignons#1#2#3#4{%
  \index[src]{charcutiere@Charcutière!{#1}@{#2}}%
  \index[src]{しやるきゆていえーる@シャルキュティエール!{#3}@{#4}}}

\def\srcChasseur#1#2#3#4{%
  \index[src]{chasseur@Chasseur!{#1}@{#2}}%
  \index[src]{しやすーる@シャスール!{#3}@{#4}}}

\def\srcChasseurEscoffier#1#2#3#4{%
  \index[src]{しやすーるえすこふいえ@シャスール(エスコフィエ)!{#1}@{#2}}%
  \index[src]{chasseur escoffier@Chasseur (Escoffier)!{#3}@{#4}}}

\def\srcChaudFroidBrune#1#2#3#4{%
  \index[src]{chaud-froid brune@Chaud-froid brune!{#1}@{#2}}%
  \index[src]{しよふろわしやいろ@ショフロワ(茶色)!{#3}@{#4}}}

\def\srcChaudFroidBruneCanard#1#2#3#4{%
  \index[src]{chaud-froid brune canards@Chaud-froid brune pour Canards!{#1}@{#2}}%
  \index[src]{しよふろわちやいろかも@ショフロワ(茶色、鴨用)!{#3}@{#4}}}

\def\srcChaudFroidBruneGibier#1#2#3#4{%
  \index[src]{chaud-froid brune gibier@Chaud-froid brune pour Gibier!{#1}@{#2}}%
  \index[src]{しよふろわちやいろしひえ@ショフロワ(茶色、ジビエ用)!{#3}@{#4}}}

\def\srcChaudFroidTomatee#1#2#3#4{%
  \index[src]{chaud-froid tometee@Chaud-froid tomatée!{#1}@{#2}}%
  \index[src]{しよふろわとまて@トマト入りショフロワ!{#3}@{#4}}}

\def\srcChevreuil#1#2#3#4{%
  \index[src]{chevreuil@Chevreuil!{#1}@{#2}}%
  \index[src]{しゆうるいゆ@シュヴルイユ!{#3}@{#4}}}

\def\srcColbert#1#2#3#4{%
  \index[src]{colbert@Colbert!{#1}@{#2}}%
  \index[src]{こるへーる@コルベール!{#3}@{#4}}}

\def\srcDiable#1#2#3#4{%
  \index[src]{diable@Diable!{#1}@{#2}}%
  \index[src]{ていあーふる@ディアーブル!{#3}@{#4}}}

\def\srcDiableEscoffier#1#2#3#4{%
  \index[src]{diable escoffier@Diable Escoffier!{#1}@{#2}}%
  \index[src]{ていあーふるえすこふいえ@ディアーブル・エスコフィエ!{#3}@{#4}}}

\def\srcDiane#1#2#3#4{%
  \index[src]{diane@Diane!{#1}@{#2}}%
  \index[src]{ていあーぬ@ディアーヌ!{#3}@{#4}}}

\def\srcDuxelles#1#2#3#4{%
  \index[src]{duxelles@Duxelles!{#1}@{#2}}%
  \index[src]{てゆくせる@デュクセル!{#3}@{#4}}}

\def\srcEstragon#1#2#3#4{%
  \index[src]{estragon@Estragon!{#1}@{#2}}%
  \index[src]{えすとらこん@エストラゴン!{#3}@{#4}}}

\def\srcFinanciere#1#2#3#4{%
  \index[src]{financierel@Financière!{#1}@{#2}}%
  \index[src]{ふいなんしえーる@フィナンシエール!{#3}@{#4}}}

\def\srcFinesHerbes#1#2#3#4{%
  \index[src]{fines herbes@Fines herbes (aux)!{#1}@{#2}}%
  \index[src]{こうそう@香草!{#3}@{#4}}}

\def\srcGenevoise#1#2#3#4{%
  \index[src]{genevoise@Genevoise!{#1}@{#2}}%
  \index[src]{しゆねーうふう@ジュネーヴ風!{#3}@{#4}}}

\def\srcGodard#1#2#3#4{%
  \index[src]{godard@Godard!{#1}@{#2}}%
  \index[src]{こたーる@ゴダール!{#3}@{#4}}}

\def\srcGrandVeneur#1#2#3#4{%
  \index[src]{grand-veneur@Grand-Veneur!{#1}@{#2}}%
  \index[src]{くらんうぬーる@グランヴヌール!{#3}@{#4}}}

\def\srcGrandVeneurEscoffier#1#2#3#4{%
  \index[src]{grand-veneur escoffier@Grand-Veneur Escoffier!{#1}@{#2}}%
  \index[src]{くらんうぬーるえすこふぃえ@グランヴヌール(エスコフィエ)!{#3}@{#4}}}

\def\srcGratin#1#2#3#4{%
  \index[src]{gratin@Gratin!{#1}@{#2}}%
  \index[src]{くらたん@グラタン!{#3}@{#4}}}

\def\srcHachee#1#2#3#4{%
  \index[src]{hachee@Hachée!{#1}@{#2}}%
  \index[src]{あしえ@アシェ!{#3}@{#4}}}

\def\srcHacheeMaigre#1#2#3#4{%
  \index[src]{hachee maigrel@Hachée maigre!{#1}@{#2}}%
  \index[src]{あしえさかな@アシェ(魚料理用)!{#3}@{#4}}}

\def\srcHussarde#1#2#3#4{%
  \index[src]{hussarde@Hussarde!{#1}@{#2}}%
  \index[src]{ゆさると@ユサルド!{#3}@{#4}}}

\def\srcItalienne#1#2#3#4{%
  \index[src]{italienne@Italienne!{#1}@{#2}}%
  \index[src]{いたりあふう@イタリア風!{#3}@{#4}}}

\def\srcJusLieEstragon#1#2#3#4{%
  \index[src]{jus lie a l'estragon@Jus lié à l'Estragon!{#1}@{#2}}%
  \index[src]{とろみをつけたしゆえすとらこん@とろみを付けたジュ・エストラゴン風味!{#3}@{#4}}}

\def\srcJusLieTomate#1#2#3#4{%
  \index[src]{jus lie tomate@Jus lié tomaté!{#1}@{#2}}%
  \index[src]{とろみをつけたしゆとまと@とろみを付けたジュ・トマト風味!{#3}@{#4}}}

\def\srcLyonnaise#1#2#3#4{%
  \index[src]{lyonnaise@Lyonnaise!{#1}@{#2}}%
  \index[src]{りよんふう@リヨン風!{#3}@{#4}}}

\def\srcMadere#1#2#3#4{%
  \index[src]{madere@Madère!{#1}@{#2}}%
  \index[src]{まてーる@マデール!{#3}@{#4}}}

\def\srcMatelote#1#2#3#4{%
  \index[src]{matelote@Matelote!{#1}@{#2}}%
  \index[src]{まとろつと@マトロット!{#3}@{#4}}}

\def\srcMoelle#1#2#3#4{%
  \index[src]{moelle@Moelle!{#1}@{#2}}%
  \index[src]{もわる@モワル!{#3}@{#4}}}

\def\srcMoscovite#1#2#3#4{%
  \index[src]{moscovite@Moscovite!{#1}@{#2}}%
  \index[src]{もすくわふう@モスクワ風!{#3}@{#4}}}

\def\srcPerigueux#1#2#3#4{%
  \index[src]{perigueux@Périgueux!{#1}@{#2}}%
  \index[src]{へりくー@ペリグー!{#3}@{#4}}}

\def\srcPerigourdine#1#2#3#4{%
  \index[src]{perigourdine@Périgourdine!{#1}@{#2}}%
  \index[src]{へりくるていーぬ@ペリグルディーヌ!{#3}@{#4}}}

\def\srcPiquante#1#2#3#4{%
  \index[src]{piquante@Piquante!{#1}@{#2}}%
  \index[src]{ひかんと@ピカント!{#3}@{#4}}}

\def\srcPoivradeOrdinaire#1#2#3#4{%
  \index[src]{poivrade ordinaire@Poivrade ordinaire!{#1}@{#2}}%
  \index[src]{ほわふらーとひようしゆん@ポワヴラード(標準)!{#3}@{#4}}}

\def\srcPoivradeGibier#1#2#3#4{%
  \index[src]{poivrade gibier@Poivrade pour Gibier!{#1}@{#2}}%
  \index[src]{ほわふらーとしひえ@ポワヴラード(ジビエ用)!{#3}@{#4}}}

\def\srcPorto#1#2#3#4{%
  \index[src]{porto@Porto!{#1}@{#2}}%
  \index[src]{ほると@ポルト!{#3}@{#4}}}

\def\srcPortugaise#1#2#3#4{%
  \index[src]{portugaise@Portugaise!{#1}@{#2}}
  \index[src]{ほるとかるふう@ポルトガル風!{#3}@{#4}}}

\def\srcProvencale#1#2#3#4{%
  \index[src]{provencale@Provençale!{#1}@{#2}}
  \index[src]{ふろうあんすふう@プロヴァンス風!{#3}@{#4}}}

\def\srcRegence#1#2#3#4{%
  \index[src]{regence@Régence!{#1}@{#2}}
  \index[src]{れしやんす@レジャンス!{#3}@{#4}}}

\def\srcRobert#1#2#3#4{%
  \index[src]{robert@Robert!{#1}@{#2}}
  \index[src]{ろへーる@ロベール!{#3}@{#4}}}

\def\srcRobertEscoffier#1#2#3#4{%
  \index[src]{robert escoffier@Robert Escoffier!{#1}@{#2}}
  \index[src]{ろへーるえすこふいえ@ロベール・エスコフィエ!{#3}@{#4}}}

\def\srcRomaine#1#2#3#4{%
  \index[src]{romaine@Romaine!{#1}@{#2}}
  \index[src]{ろーまふう@ローマ風!{#3}@{#4}}}

\def\srcRouennaise#1#2#3#4{%
  \index[src]{rouennaise@Rouennaise!{#1}@{#2}}
  \index[src]{るーあんふう@ルーアン風!{#3}@{#4}}}

\def\srcSalmis#1#2#3#4{%
  \index[src]{salmis@Salmis!{#1}@{#2}}
  \index[src]{さるみ@サルミ!{#3}@{#4}}}

\def\srcTortue#1#2#3#4{%
  \index[src]{tortue@Tortue!{#1}@{#2}}
  \index[src]{とるちゆ@トルチュ!{#3}@{#4}}}

\def\srcVenaison#1#2#3#4{%
  \index[src]{venaison@Venaison!{#1}@{#2}}
  \index[src]{うねそん@ヴネゾン!{#3}@{#4}}}

\def\srcVinRouge#1#2#3#4{%
  \index[src]{vin rouge@Vin rouge (au)!{#1}@{#2}}
  \index[src]{あかわいん@赤ワイン!{#3}@{#4}}}


\def\srcZingaraA#1#2#3#4{%
  \index[src]{zingara a@Zingara A!{#1}@{#2}}
  \index[src]{さんからa@ザンガラ A!{#3}@{#4}}}

\def\srcZingaraB#1#2#3#4{%
  \index[src]{zingara b@Zingara B!{#1}@{#2}}
  \index[src]{さんからb@ザンガラ B!{#3}@{#4}}}


%%%%%%% ホワイト系の派生ソース

\def\srcAlbufera#1#2#3#4{%
  \index[src]{albufera@Albuféra!{#1}@{#2}}
  \index[src]{あるひゆふえら@アルビュフェラ!{#3}@{#4}}}

\def\srcAmericaine#1#2#3#4{%
  \index[src]{americaine@Américaien!{#1}@{#2}}
  \index[src]{あめりけーぬ@アメリケーヌ!{#3}@{#4}}}

\def\srcAnchois#1#2#3#4{%
  \index[src]{anchois@Anchois!{#1}@{#2}}%
  \index[src]{あんちよひ@アンチョビ!{#3}@{#4}}}

\def\srcAurore#1#2#3#4{%
  \index[src]{aurore@Aurore!{#1}@{#2}}%
  \index[src]{おーろーる@オーロール!{#3}@{#4}}}

\def\srcAuroreMaigre#1#2#3#4{%
  \index[src]{aurore maigre@Aurore maigre!{#1}@{#2}}%
  \index[src]{おーろーるさかな@オーロール(魚料理用)!{#3}@{#4}}}

\def\srcBavaroise#1#2#3#4{%
  \index[src]{bavaroise@Bavaroise!{#1}@{#2}}%
  \index[src]{はいえるんふう@バイエルン風!{#3}@{#4}}}

\def\srcBearnaise#1#2#3#4{%
  \index[src]{bearnaise@Béarnaise!{#1}@{#2}}%
  \index[src]{へあるねーす@ベアルネーズ!{#3}@{#4}}}

\def\srcBearnaiseTomatee#1#2#3#4{%
  \index[src]{bearnaise tomatee@Béarnaise tomatée!{#1}@{#2}}%
  \index[src]{へあるねーすとまと@ベアルネース(トマト入り)!{#3}@{#4}}}

\def\srcChoron#1#2#3#4{%
  \index[src]{choron@Choron!{#1}@{#2}}%
  \index[src]{しよろん@ショロン!{#3}@{#4}}}

\def\srcBearnaiseGlaceDeViande#1#2#3#4{%
  \index[src]{bearnaise glace de viande@Bearnaise à la glace de viande!{#1}@{#2}}%
  \index[src]{へあるねーすくらすとういあんと@ベアルネーズ(グラスドヴィアンド入り)!{#3}@{#4}}}

\def\srcFoyot#1#2#3#4{%
  \index[src]{foyot@Foyot!{#1}@{#2}}%
  \index[src]{ふおいよ@フォイヨ!{#3}@{#4}}}

\def\srcValois#1#2#3#4{%
  \index[src]{valois@Valois!{#1}@{#2}}%
  \index[src]{うあろわ@ヴァロワ!{#3}@{#4}}}

\def\srcBercy#1#2#3#4{%
  \index[src]{bercy@Bercy!{#1}@{#2}}%
  \index[src]{へるしー@ベルシー!{#3}@{#4}}}

\def\srcBeurre#1#2#3#4{%
  \index[src]{beurre@Beurre (au)!{#1}@{#2}}%
  \index[src]{ふーる@オ・ブール!{#3}@{#4}}}

\def\srcBatarde#1#2#3#4{%
  \index[src]{batarde@Batarde!{#1}@{#2}}%
  \index[src]{はたると@バタルド!{#3}@{#4}}}

\def\srcBonnefoy#1#2#3#4{%
  \index[src]{bonnefoy@Bonnefoy!{#1}@{#2}}%
  \index[src]{ほぬふおわ@ボヌフォワ!{#3}@{#4}}}

\def\srcBordelaiseVinBlanc#1#2#3#4{%
  \index[src]{bordelaise vin blanc@Bordelaise au vin blanc!{#1}@{#2}}%
  \index[src]{ほるとーふうしろわいん@ボルドー風(白ワイン)!{#3}@{#4}}}

\def\srcBretonneBlanche#1#2#3#4{%
  \index[src]{bretonne blanche@Bretonne (blanche)!{#1}@{#2}}%
  \index[src]{ふるたーにゆふうしろ@ブルターニュ風(ホワイト系)!{#3}@{#4}}}

\def\srcCanotiere#1#2#3#4{%
  \index[src]{canotiere@Canotière!{#1}@{#2}}%
  \index[src]{かのていえーる@カノティエール!{#3}@{#4}}}

\def\srcCapres#1#2#3#4{%
  \index[src]{capres@Câpres (aux)!{#1}@{#2}}%
  \index[src]{けいはー@ケイパー!{#3}@{#4}}}

\def\srcCardinal#1#2#3#4{%
  \index[src]{cardinl@Cardinal!{#1}@{#2}}%
  \index[src]{かるていなる@カルディナル!{#3}@{#4}}}

\def\src#1#2#3#4{%
  \index[src]{champignons blanche@Champignons (aux)(blanche)!{#1}@{#2}}%
  \index[src]{まつしゆるーむしろ@マッシュルーム(ホワイト系)!{#3}@{#4}}}

\def\srcChantilly#1#2#3#4{%
  \index[src]{chantilly@Chantilly!{#1}@{#2}}%
  \index[src]{しやんていい@シャンティイ!{#3}@{#4}}}

\def\srcChateaubriand#1#2#3#4{%
  \index[src]{chateaubriand@Chateaubriand!{#1}@{#2}}
  \index[src]{しやとーふりやん@シャトーブリヤン!{#3}@{#4}}}

\def\srcChaudFroidBlancheOrdinaire#1#2#3#4{%
  \index[src]{choud-froid blanche ordinaire@Chaud-froid blanche ordinaire!{#1}@{#2}}
  \index[src]{しよふろわしろひようしゆん@ショフロワ(白)(標準)!{#3}@{#4}}}

\def\srcChaudFroidBlonde#1#2#3#4{%
  \index[src]{choud-froid blonde@Chaud-froid blonde!{#1}@{#2}}%
  \index[src]{しよふろわふろんと@ショフロワ(ブロンド)!{#3}@{#4}}}

\def\srcChaudFroidAurore#1#2#3#4{%
  \index[src]{chaud-froid aurore@Chaud-froid Aurore!{#1}@{#2}}%
  \index[src]{しよふろわおーろーる@ショフロワ・オーロール!{#3}@{#4}}}

\def\srcChaudFroidVertPre#1#2#3#4{%
  \index{chaud-froid vert-pre@Chaud-froid Vert-pré!{#1}@{#2}}%
  \index[src]{しよふろわうえーるふれ@ショフロワ・ヴェールプレ!{#3}@{#4}}}

\def\srcChaudFroidMaigre#1#2#3#4{%
  \index[src]{chaud-froid maigre@Chaud-froid maigre!{#1}@{#2}}%
  \index[src]{しよふろわさかな@ショフロワ(魚料理用)!{#3}@{#4}}}

\def\srcChivry#1#2#3#4{%
  \index[src]{chivry@Chivry!{#1}@{#2}}%
  \index[src]{しうり@シヴリ!{#3}@{#4}}}

\def\srcCreme#1#2#3#4{%
  \index[src]{creme@Crème (à la)!{#1}@{#2}}%
  \index[src]{くれーむ@クレーム!{#3}@{#4}}}

\def\srcCrevettes#1#2#3#4{%
  \index[src]{crevettes@Crevettes (aux)!{#1}@{#2}}%
  \index[src]{くるうえつと@クルヴェット!{#3}@{#4}}}

\def\srcCurrie#1#2#3#4{%
  \index[src]{currie@Currie!{#1}@{#2}}%
  \index[src]{かれー@カレー!{#3}@{#4}}}

\def\srcCurrieIndienne#1#2#3#4{%
  \index[src]{currie indienne@Currie à l'Indienne!{#1}@{#2}}%
  \index[src]{いんとふうかれー@インド風カレー!{#3}@{#4}}}

\def\srcDiplomate#1#2#3#4{%
  \index[src]{diplomate@Diplomate!{#1}@{#2}}%
  \index[src]{ていふろまつと@ディプロマット!{#3}@{#4}}}

\def\srcEcossaise#1#2#3#4{%
  \index[src]{ecossaise@Ecossaise!{#1}@{#2}}%
  \index[src]{すこつとらんとふう@スコットランド風!{#3}@{#4}}}

\def\srcEstragon#1#2#3#4{%
  \index[src]{estragon@Estragon!{#1}@{#2}}%
  \index[src]{えすとらこん@エストラゴン!{#3}@{#4}}}

\def\srcFinesHerbes#1#2#3#4{%
  \index[src]{fines herbes blanche@Fines herbes blanche (aux)!{#1}@{#2}}%
  \index[src]{こうそうしろ@香草(ホワイト系)!{#3}@{#4}}}

\def\srcGroseilles#1#2#3#4{%
  \index[src]{groseilles@Groseilles!{#1}@{#2}}%
  \index[src]{くろせいゆ@グロゼイユ!{#3}@{#4}}}

\def\srcHollandaise#1#2#3#4{%
  \index[src]{hollandaise@Hollandaise!{#1}@{#2}}%
  \index[src]{おらんてーす@オランデーズ!{#3}@{#4}}}

\def\srcHomard#1#2#3#4{%
  \index[src]{homard@Homard!{#1}@{#2}}%
  \index[src]{おまーる@オマール!{#3}@{#4}}}

\def\srcHongroise#1#2#3#4{%
  \index[src]{hongroise@Hongroise!{#1}@{#2}}%
  \index[src]{はんかりーふう@ハンガリー風!{#3}@{#4}}}

\def\srcHuitres#1#2#3#4{%
  \index[src]{huitres@Huîtres (aux)!{#1}@{#2}}%
  \index[src]{かきいり@牡蠣入り!{#3}@{#4}}}

\def\srcIndienne#1#2#3#4{%
  \index[src]{indienne@Indienne!{#1}@{#2}}%
  \index[src]{いんとふう@インド風!{#3}@{#4}}}

\def\srcIvoire#1#2#3#4{%
  \index[src]{ivoire@Ivoire!{#1}@{#2}}%
  \index[src]{いうおわーる@イヴォワール!{#3}@{#4}}}

\def\srcJoinville#1#2#3#4{%
  \index[src]{joinville@Joinville!{#1}@{#2}}%
  \index[src]{しよわんういる@ジョワンヴィル!{#3}@{#4}}}

\def\srcLaguipiere#1#2#3#4{%
  \index[src]{laguipiere@Laguipière!{#1}@{#2}}%
  \index[src]{らきひえーる@ラギピエール!{#3}@{#4}}}

\def\srcLivonienne#1#2#3#4{%
  \index[src]{livonienne@Livonienne!{#1}@{#2}}%
  \index[src]{りうおにあふう@リヴォニア風!{#3}@{#4}}}

\def\srcMaltaise#1#2#3#4{%
  \index[src]{maltaise@maltaise!{#1}@{#2}}%
  \index[src]{まるたふう@マルタ風!{#3}@{#4}}}

\def\srcMariniere#1#2#3#4{%
  \index[src]{mariniere@Marinière!{#1}@{#2}}%
  \index[src]{まりにえーる@マリニエール!{#3}@{#4}}}

\def\srcMateloteBlanche#1#2#3#4{%
  \index[src]{matelote blanche@Matelote blanche!{#1}@{#2}}%
  \index[src]{まとろつとしろ@マトロット(白)!{#3}@{#4}}}

\def\srcMornay#1#2#3#4{%
  \index[src]{mornay@Mornay!{#1}@{#2}}%
  \index[src]{もるねー@モルネー!{#3}@{#4}}}

\def\srcMousseline#1#2#3#4{%
  \index[src]{mousseline@Mousseline!{#1}@{#2}}%
  \index[src]{むすりーぬ@ムスリーヌ!{#3}@{#4}}}

\def\srcMousseuse#1#2#3#4{%
  \index[src]{mousseuse@Mousseuse!{#1}@{#2}}%
  \index[src]{むすーす@ムスーズ!{#3}@{#4}}}

\def\srcMoutarde#1#2#3#4{%
  \index[src]{moutarde@Moutarde!{#1}@{#2}}%
  \index[src]{むたると@ムタルド!{#3}@{#4}}}

\def\srcNantua#1#2#3#4{%
  \index[src]{nantua@Nantua!{#1}@{#2}}%
  \index[src]{なんちゆあ@ナンチュア!{#3}@{#4}}}

\def\srcNewBurgCru#1#2#3#4{%
  \index[src]{new-burg cru@New-burg avec le homard cru!{#1}@{#2}}%
  \index[src]{にゆーはーくいけ@ニューバーグ(活けオマール)!{#3}@{#4}}}

\def\srcNewBurgCuit#1#2#3#4{%
  \index[src]{new-burg cuit@New-burg avec le homard cuit!{#1}@{#2}}%
  \index[src]{にゆーはーくゆて@ニューバーグ(茹でオマール)!{#3}@{#4}}}

\def\srcNoisette#1#2#3#4{%
  \index[src]{noisette@Noisette!{#1}@{#2}}%
  \index[src]{のわせつと@ノワゼット!{#3}@{#4}}}

\def\srcNormande#1#2#3#4{%
  \index[src]{normande@Normande!{#1}@{#2}}%
  \index[src]{のるまんていふう@ノルマンディ風!{#3}@{#4}}}

\def\srcOrientale#1#2#3#4{%
  \index[src]{orientale@Orientale!{#1}@{#2}}%
  \index[src]{おりえんとふう@オリエント風!{#3}@{#4}}}

\def\srcPaloise#1#2#3#4{%
  \index[src]{paloise@Paloise!{#1}@{#2}}%
  \index[src]{ほーふう@ポー風!{#3}@{#4}}}  

\def\srcPoulette#1#2#3#4{%
  \index[src]{poulette@Poulette!{#1}@{#2}}%
  \index[src]{ふれつと@プレット!{#3}@{#4}}}

\def\srcRavigote#1#2#3#4{%
  \index[src]{ravigote@Ravigote!{#1}@{#2}}%
  \index[src]{らういこつと@ラヴィゴット!{#3}@{#4}}}

\def\srcRegencePoisson#1#2#3#4{%
  \index[src]{regence poisson@Régence pour Poissons!{#1}@{#2}}%
  \index[src]{れしやんすさかなよう@レジャンス(魚料理用)!{#3}@{#4}}}

\def\srcRegenceGarnituresVolaille#1#2#3#4{%
  \index[src]{regence garnitures volaille@Régence pour garnitures de Volaille!{#1}@{#2}}%
  \index[src]{れしやんすとりりようりのかるにちゆーるよう@レジャンス(鶏料理のガルニチュール用)!{#3}@{#4}}}

\def\srcRiche#1#2#3#4{%
  \index[src]{riche@Riche!{#1}@{#2}}%
  \index[src]{りつしゆ@リッシュ!{#3}@{#4}}}

\def\srcRubens#1#2#3#4{%
  \index[src]{rubens@Rubens!{#1}@{#2}}%
  \index[src]{るーへんす@ルーベンス!{#3}@{#4}}}

\def\srcSaintMalo#1#2#3#4{%
  \index[src]{saint-malo@Saint-Malo!{#1}@{#2}}%
  \index[src]{さんまろふう@サンマロ風!{#3}@{#4}}}

\def\srcSmitane#1#2#3#4{%
  \index[src]{smitane@Smitane!{#1}@{#2}}%
  \index[src]{すみたーぬ@スミターヌ!{#3}@{#4}}}

\def\srcSolferino#1#2#3#4{%
  \index[src]{solferino@Solférino!{#1}@{#2}}%
  \index[src]{そるふえりの@ソルフェリノ!{#3}@{#4}}}

\def\srcSoubise#1#2#3#4{%
  \index[src]{soubise@Soubise!{#1}@{#2}}
  \index[src]{すひーす@スビーズ!{#3}@{#4}}}

\def\srcSoubiseTomatee#1#2#3#4{%
  \index[src]{soubise tomatee@Soubise tomatée!{#1}@{#2}}%
  \index[src]{すひーすとまといり@スビース(トマト入り)!{#3}@{#4}}}

\def\srcSouchet#1#2#3#4{%
  \index[src]{souchet@Souchet!{#1}@{#2}}%
  \index[src]{すーしえ@スーシェ!{#3}@{#4}}}

\def\srcTyrolienne#1#2#3#4{%
  \index[src]{tyrolienne@Tyrolienne!{#1}@{#2}}%
  \index[src]{ちろるふう@チロル風!{#3}@{#4}}}

\def\srcTyrolienneAncienne#1#2#3#4{%
  \index[src]{tyroienne ancienne@!Tyrolienne à l'ancienne!{#1}@{#2}}%
  \index[src]{ちろるふうくらしつく@チロル風・クラシック!{#3}@{#4}}}

\def\srcValois#1#2#3#4{%
  \index[src]{valois@Valois!{#1}@{#2}}%
  \index[src]{うあろわ@ヴァロワ!{#3}@{#4}}}

\def\srcVenitienne#1#2#3#4{%
  \index[src]{venitienne@Vénitienne!{#1}@{#2}}%
  \index[src]{うえねついあふう@ヴェネツィア風!{#3}@{#4}}}

\def\srcVeron#1#2#3#4{%
  \index[src]{veron@Véron!{#1}@{#2}}%
  \index[src]{うえろん@ヴェロン!{#3}@{#4}}}

\def\srcVillageoise#1#2#3#4{%
  \index[src]{villageoise@Villageoise!{#1}@{#2}}%
  \index[src]{むらひとふう@村人風!{#3}@{#4}}}

\def\srcVilleroy#1#2#3#4{%
  \index[src]{villeroy@Villeroy!{#1}@{#2}}%
  \index[src]{ういるろわ@ヴィルロワ!{#3}@{#4}}}

\def\srcVilleroySoubisee#1#2#3#4{%
  \index[src]{villeroy soubisee@Villeroy soubisée!{#1}@{#2}}%
  \index[src]{ういるろわすひーすいり@ヴィルロワ(スビーズ入り)!{#3}@{#4}}}

\def\srcVilleroyTomatee#1#2#3#4{%
  \index[src]{villeroy tomatee@Villeroy tomatée!{#1}@{#2}}%
  \index[src]{ういるろわとまといり@ヴィルロワ(トマト入り)!{#3}@{#4}}}

\def\srcVinBlanc#1#2#3#4{%
  \index[src]{vin blanc@Vin blanc!{#1}@{#2}}%
  \index[src]{しろわいん@白ワイン!{#3}@{#4}}}

%%%%イギリス風ソース(温製)

\def\srcCranberriesSauce#1#2#3#4{%
  \index[src]{airelles@Airelles (aux)(Cramberries-Sauce)!{#1}@{#2}}%
  \index[src]{くらんへりー@クランベリー!{#3}@{#4}}}

\def\srcAlbert#1#2#3#4{%
  \index[src]{albert@Albert (Albert-Sauce)!{#1}@{#2}}%
  \index[src]{あるはーと@アルバート!{#3}@{#4}}}

\def\srcAromatic#1#2#3#4{%
  \index[src]{aromates@Aromates (aux)(Aromatic-Sauce)!{#1}@{#2}}%
  \index[src]{あろまていつく@アロマティック!{#3}@{#4}}}

\def\srcButter#1#2#3#4{%
  \index[src]{Beurre anglaise@Beurre (au)(Butter-Sauce)!{#1}@{#2}}%
  \index[src]{はたーそーすいきりす@バターソース(イギリス風)!{#3}@{#4}}}

\def\srcCaper#1#2#3#4{%
  \index[src]{capres anglaise@Câpres (aux)(Capers-Sauce)!{#1}@{#2}}%
  \index[src]{けいはーいきりす@ケイパー(イギリス風)!{#3}@{#4}}}

\def\srcCelery#1#2#3#4{%
  \index[src]{celeri anglaise@Céleri (au)(Celery-Sauce)!{#1}@{#2}}%
  \index[src]{せろりいきりす@セロリ(イギリス風)!{#3}@{#4}}}

\def\srcRoeBuck#1#2#3#4{%
  \index[src]{chevreuil anglaise@Chevreuil (Roe-buck Sauce)!{#1}@{#2}}%
  \index[src]{ろーはつく@ローバック(イギリス風)!{#3}@{#4}}}

\def\srcCream#1#2#3#4{%
  \index[src]{creme anglaise@Crème à l'anglaise (Cream-Sauce)!{#1}@{#2}}%
  \index[src]{くりーむ@クリーム(イギリス風)!{#3}@{#4}}}

\def\srcShrimps#1#2#3#4{%
  \index[src]{crevettes anglaise@Crevettes (aux)(Shrimps-Sauce)!{#1}@{#2}}%
  \index[src]{しゆりんふ@シュリンプ(イギリス風)!{#3}@{#4}}}

\def\srcDevilled#1#2#3#4{%
  \index[src]{diable anglaise@Diable à l'anglaise (Devilled-Sauce)!{#1}@{#2}}%
  \index[src]{てひる@デビル(イギリス風)!{#3}@{#4}}}

\def\srcEcossaise#1#2#3#4{%
  \index[src]{ecossaise@Ecossaise (Scotch eggs Sauce)!{#1}@{#2}}%
  \index[src]{すこつちえつく@スコッチエッグ!{#3}@{#4}}}

\def\srcFennel#1#2#3#4{%
  \index[src]{fenouil anglaise@Fenouil (Fennel Sauce)!{#1}@{#2}}%
  \index[src]{ふえんねるいきりす@フェンネル!{#3}@{#4}}}

\def\srcGooseberry#1#2#3#4{%
  \index[src]{groseilles anglaise@Groseilles (aux)(Gooseberry Sauce)!{#1}@{#2}}%
  \index[src]{くーすへりー@グーズベリー!{#3}@{#4}}}

\def\srcLobster#1#2#3#4{%
  \index[src]{homard anglaise@Homard à l'anglaise!{#1}@{#2}}%
  \index[src]{ろふすたー@ロブスター!{#3}@{#4}}}

\def\srcOyster#1#2#3#4{%
  \index[src]{huitres anglaise@Huîtres (aux)(Oyster Sauce)!{#1}@{#2}}%
  \index[src]{かきいきりす@牡蠣入り(イギリス風)!{#3}@{#4}}}

\def\srcBrownOyster#1#2#3#4{%
  \index[src]{huitres brune anglaise@Brune aux huîtres (Brown Oyster Sauce)!{#1}@{#2}}%
  \index[src]{かきいりふらうんいきりす@牡蠣入りブラウン(イギリス風)!{#3}@{#4}}}

\def\srcBrownGraby#1#2#3#4{%
  \index[src]{jus colore anglaise@Jus coloré (Brown Gravy)!{#1}@{#2}}%
  \index[src]{ふらうんくれいういー@ブラウングレイヴィー!{#3}@{#4}}}

\def\srcEggs#1#2#3#4{%
  \index[src]{oeufs anglaise@OEufs à l'anglaise (aux)(Egg Sauce)!{#1}@{#2}}%
  \index[src]{えつくそーす@エッグ(イギリス風)!{#3}@{#4}}}

\def\srcEggsAndButter#1#2#3#4{%
  \index[src]{oeufs beurre fondu anglaise@OEufs au Beurre fondu (aux)(Eggs and Butter Sauce)!{#1}@{#2}}%
  \index[src]{えつくあんとはたーそーす@エッグアンドバター(イギリス風)!{#3}@{#4}}}

\def\srcOnions#1#2#3#4{%
  \index[src]{oignons anglaise@Oignons (aux)(anglaise)!{#1}@{#2}}%
  \index[src]{おにおんそーす@オニオン(イギリス風)!{#3}@{#4}}}

\def\srcBread#1#2#3#4{%
  \index[src]{pain anglaise@Pain (au)(Bread Sauce)!{#1}@{#2}}%
  \index[src]{ふれつとそーす@ブレッド(イギリス風)!{#3}@{#4}}}

\def\srcFriedBread#1#2#3#4{%
  \index[src]{pain frit anglaise@Pain frit (au)(Fried Bread Sauce)!{#1}@{#2}}%
  \index[src]{ふらいとふれつとそーす@フライドブレッド!{#3}@{#4}}}

\def\srcPersley#1#2#3#4{%
  \index[src]{persil anglaise@Persil (Persley Sauce)!{#1}@{#2}}%
  \index[src]{はせりそーす@パセリ(イギリス風)!{#3}@{#4}}}

\def\srcPersilPoissons#1#2#3#4{%
  \index[src]{persil poissons anglaise@Persil pour Poissons (anglaise)!{#1}@{#2}}%
  \index[src]{さかなりようりようはせりそーす@魚料理用パセリ!{#3}@{#4}}}

\def\srcApple#1#2#3#4{%
  \index[src]{pomme anglaise@Pomme (aux)(Apple Sauce)!{#1}@{#2}}%
  \index[src]{あつふるそーす@アップル(イギリス風)!{#3}@{#4}}}

\def\srcPortWine#1#2#3#4{%
  \index[src]{porto@!Porto (au)(Port Wine Sauce)!{#1}@{#2}}%
  \index[src]{ほーとわいんそーす@ポートワイン(イギリス風)!{#3}@{#4}}}

\def\srcHorseRadhish#1#2#3#4{%
  \index[src]{raifort chaude@Raifort chaude (Horese radhish Sauce)!{#1}@{#2}}%
  \index[src]{ほーすらていしゆ@ホースラディッシュ(イギリス風)!{#3}@{#4}}}

\def\srcReform#1#2#3#4{%
  \index[src]{reforme anglais@Rérorme (Reform Sauce)!{#1}@{#2}}%
  \index[src]{りふおーむ@リフォーム(イギリス風)!{#3}@{#4}}}

\def\srcSageAndOnions#1#2#3#4{%
  \index[src]{sauge oignons anglaise@Sauge et Oignons (Sage and onions Sauce)!{#1}@{#2}}%
  \index[src]{せーしとたまねき@セージと玉ねぎ(イギリス風)!{#3}@{#4}}}

\def\srcYorkshire#1#2#3#4{%
  \index[src]{yorkshire@Yorkshire!{#1}@{#2}}%
  \index[src]{よーくしやー@ヨークシャー(イギリス風)!{#3}@{#4}}}


%%%%% 冷製ソース

\def\srcAioli#1#2#3#4{%
  \index[src]{aioli@aioli, ou beurre de provence!{#1}@{#2}}%
  \index[src]{あいより@アイヨリ/プロヴァンスバター!{#3}@{#4}}}

\def\srcAndalouse#1#2#3#4{%
  \index[src]{andalouse@Andalouse!{#1}@{#2}}%
  \index[src]{あんたるしあふう@アンダルシア風!{#3}@{#4}}}

\def\srcBohemienne#1#2#3#4{%
  \index[src]{bohemienne@Bohémienne!{#1}@{#2}}%
  \index[src]{ほへにあのむすめ@ボヘミアの娘!{#3}@{#4}}}

\def\srcChantillyFroide#1#2#3#4{%
  \index[src]{chantilly froide@Chantilly (froide)!{#1}@{#2}}%
  \index[src]{しやんていいれいせい@シャンティイ(冷製)!{#3}@{#4}}}

\def\srcGenoise#1#2#3#4{%
  \index[src]{genoise@Génoise (froide)!{#1}@{#2}}%
  \index[src]{しえのあうふう@ジェノヴァ風(冷製)!{#3}@{#4}}}

\def\srcGribiche#1#2#3#4{%
  \index[src]{gribiche@Gribiche!{#1}@{#2}}%
  \index[src]{くりひつしゆ@グリビッシュ(冷製)!{#3}@{#4}}}

\def\srcGroseillesRaifort#1#2#3#4{%
  \index[src]{groseilles raifort@Groseilles au Raifort (froide)!{#1}@{#2}}%
  \index[src]{くろせいゆれふおーる@グロゼイユ・レフォール風味!{#3}@{#4}}}

\def\srcItalienneFroide#1#2#3#4{%
  \index[src]{italienne froide@Italienne (froide)!{#1}@{#2}}%
  \index[src]{いたりあふうれいせい@イタリア風(冷製)!{#3}@{#4}}}

\def\srcMayonnaise#1#2#3#4{%
  \index[src]{mayonnaise@Mayonnaise!{#1}@{#2}}%
  \index[src]{まよねーす@マヨネーズ!{#3}@{#4}}}

\def\srcMayonnaiseFouette#1#2#3#4{%
  \index[src]{mayonnaise fouette@!Mayonnaise fouettée, à la russe{#1}@{#2}}%
  \index[src]{ろしあふうほいつふまよねーす@ロシア風ホイップマヨネーズ!{#3}@{#4}}}

\def\srcMayonnaisesDiverses#1#2#3#4{%
  \index[src]{mayonnaises diverses@Mayonnaises diverses!{#1}@{#2}}%
  \index[src]{まよねーすのはりえーしよん@マヨネーズのバリエーション!{#3}@{#4}}}

\def\srcMousquetqire#1#2#3#4{%
  \index[src]{mousquetaire@Mousquetaire (froide)!{#1}@{#2}}%
  \index[src]{むすくてーる@ムルクテール(冷製)!{#3}@{#4}}}

\def\srcMoutardeCreme#1#2#3#4{%
  \index[src]{moutarde creme@Moutarde à la Crème (froide)!{#1}@{#2}}%
  \index[src]{むたるとなまくりーむ@ムタルド・生クリーム入り!{#3}@{#4}}}

\def\srcRavigote#1#2#3#4{%
  \index[src]{ravigote@Ravigote!{#1}@{#2}}%
  \index[src]{らういこつと@ラヴィゴット!{#3}@{#4}}}

\def\srcVinaigrette#1#2#3#4{%
  \index[src]{vinaigrette@Vinaigrette!{#1}@{#2}}%
  \index[src]{ういねくれつと@ヴィネグレット!{#3}@{#4}}}

\def\srcRemoulade#1#2#3#4{%
  \index[src]{remoulade@Rémoulade!{#1}@{#2}}%
  \index[src]{れむらーと@レムラード!{#3}@{#4}}}

\def\srcRusse#1#2#3#4{%
  \index[src]{russe@Russe!{#1}@{#2}}%
  \index[src]{ろしあふう@ロシア風!{#3}@{#4}}}

\def\srcTartare#1#2#3#4{%
  \index[src]{tartare@Tartare!{#1}@{#2}}%
  \index[src]{たるたる@たるたる!{#3}@{#4}}}

\def\srcVerte#1#2#3#4{%
  \index[src]{verte@verte!{#1}@{#2}}%
  \index[src]{うえると@ヴェルト(冷製)!{#3}@{#4}}}

\def\srcVincent#1#2#3#4{%
  \index[src]{vincent@Vincent!{#1}@{#2}}%
  \index[src]{うあんさん@ヴァンサン(冷製)!{#3}@{#4}}}

\def\srcSuedoise#1#2#3#4{%
  \index[src]{suedoise@Suédoise!{#1}@{#2}}%
  \index[src]{すうえーてんふうれいせい@スウェーデン風(冷製)!{#3}@{#4}}}

\def\srcCambridge#1#2#3#4{%
  \index[src]{cambridge@Cambridge!{#1}@{#2}}%
  \index[src]{けんふりつし@ケンブリッジ!{#3}@{#4}}}

\def\srcCumberland#1#2#3#4{%
  \index[src]{cumberland@Cumberland!{#1}@{#2}}%
  \index[src]{かんはーらんと@カンバーランド!{#3}@{#4}}}

\def\srcGloucester#1#2#3#4{%
  \index[src]{gloucester@Gloucester!{#1}@{#2}}%
  \index[src]{くろすたー@グロスター!{#3}@{#4}}}

\def\srcMint#1#2#3#4{%
  \index[src]{menthe@Menthe (Mint Sauce)!{#1}@{#2}}%
  \index[src]{みんと@ミント!{#3}@{#4}}}

\def\srcOxford#1#2#3#4{%
  \index[src]{oxford@Oxford!{#1}@{#2}}%
  \index[src]{おつくすふおーと@オックスフォード!{#3}@{#4}}}

\def\srcColdHorsradish#1#2#3#4{%
  \index[src]{coldhorseradish@Cold Hordradich!{#1}@{#2}}%
  \index[src]{ほーすらていつしゆれいせい@ホースラディッシュ(冷製)!{#3}@{#4}}}




\def\srcBeurreEcrevisse#1#2#3#4{%
  \index[src]{beurre ecrevisse@Beurre d'Ecrevisse!{#1}@{#2}}%
  \index[src]{えくるういすはたー@エクルヴィスバター!{#3}@{#4}}}















\makeatother


%% Local Variables:
%% TeX-engine: luatex
%% End:

\title{エスコフィエ『料理の手引き』全注解}
\author{五 島 学}
\date{}





%%
%% added
%%

%
% No language specified? take American English.
%

\ifnum 0\ifxetex 1\fi\ifluatex 1\fi=0 % if pdftex
  \usepackage[shorthands=off,main=english]{babel}
\else
    % See issue https://github.com/reutenauer/polyglossia/issues/127
  \renewcommand*\familydefault{\sfdefault}
    % load polyglossia as late as possible as it *could* call bidi if RTL lang (e.g. Hebrew or Arabic)
  \usepackage{polyglossia}
  \setmainlanguage[]{english}
\fi


%
% colors
%
\usepackage[]{xcolor}

%
% listing colors
%
\definecolor{listing-background}{HTML}{F7F7F7}
\definecolor{listing-rule}{HTML}{B3B2B3}
\definecolor{listing-numbers}{HTML}{B3B2B3}
\definecolor{listing-text-color}{HTML}{000000}
\definecolor{listing-keyword}{HTML}{435489}
\definecolor{listing-identifier}{HTML}{435489}
\definecolor{listing-string}{HTML}{00999A}
\definecolor{listing-comment}{HTML}{8E8E8E}
\definecolor{listing-javadoc-comment}{HTML}{006CA9}

%\definecolor{listing-background}{rgb}{0.97,0.97,0.97}
%\definecolor{listing-rule}{HTML}{B3B2B3}
%\definecolor{listing-numbers}{HTML}{B3B2B3}
%\definecolor{listing-text-color}{HTML}{000000}
%\definecolor{listing-keyword}{HTML}{D8006B}
%\definecolor{listing-identifier}{HTML}{000000}
%\definecolor{listing-string}{HTML}{006CA9}
%\definecolor{listing-comment}{rgb}{0.25,0.5,0.35}
%\definecolor{listing-javadoc-comment}{HTML}{006CA9}

%
% for the background color of the title page
%

%
% TOC depth and 
% section numbering depth
%
\setcounter{tocdepth}{3}

%
% line spacing
%
\usepackage{setspace}
\setstretch{1.2}

%
% break urls
%
\PassOptionsToPackage{hyphens}{url}

%
% When using babel or polyglossia with biblatex, loading csquotes is recommended 
% to ensure that quoted texts are typeset according to the rules of your main language.
%
\usepackage{csquotes}

%
% captions
%
\definecolor{caption-color}{HTML}{777777}
\usepackage[font={stretch=1.2}, textfont={color=caption-color}, position=top, skip=4mm, labelfont=bf, singlelinecheck=false, justification=raggedright]{caption}
\setcapindent{0em}
\captionsetup[longtable]{position=above}

%
% blockquote
%
\definecolor{blockquote-border}{RGB}{221,221,221}
\definecolor{blockquote-text}{RGB}{119,119,119}
\usepackage{mdframed}
\newmdenv[rightline=false,bottomline=false,topline=false,linewidth=3pt,linecolor=blockquote-border,skipabove=\parskip]{customblockquote}
\renewenvironment{quote}{\begin{customblockquote}\list{}{\rightmargin=0em\leftmargin=0em}%
\item\relax\color{blockquote-text}\ignorespaces}{\unskip\unskip\endlist\end{customblockquote}}

%
% Source Sans Pro as the de­fault font fam­ily
% Source Code Pro for monospace text
%
% 'default' option sets the default 
% font family to Source Sans Pro, not \sfdefault.
%
\usepackage[default]{sourcesanspro}
\usepackage{sourcecodepro}

%
% heading color
%
\definecolor{heading-color}{RGB}{40,40,40}
\addtokomafont{section}{\color{heading-color}}
% When using the classes report, scrreprt, book, 
% scrbook or memoir, uncomment the following line.
%\addtokomafont{chapter}{\color{heading-color}}

%
% variables for title and author
%
\usepackage{titling}
\title{エスコフィエ『料理の手引き』全注解}
\author{五 島 学}

%
% tables
%

%
% remove paragraph indention
%
\setlength{\parindent}{0pt}
\setlength{\parskip}{6pt plus 2pt minus 1pt}
\setlength{\emergencystretch}{3em}  % prevent overfull lines

%
%
% Listings
%
%


%
% header and footer
%
\usepackage{fancyhdr}
\pagestyle{fancy}
\fancyhead{}
\fancyfoot{}
\lhead{エスコフィエ『料理の手引き』全注解}
\chead{}
\rhead{}
\lfoot{五 島 学}
\cfoot{}
\rfoot{\thepage}
\renewcommand{\headrulewidth}{0.4pt}
\renewcommand{\footrulewidth}{0.4pt}

%%
%% end added
%%

\begin{document}

%%
%% begin titlepage
%%


%%
%% end titlepage
%%


{
\setcounter{tocdepth}{2}
\tableofcontents
}
\begin{Main}

\hypertarget{petites-sauces-brunes-composees}{%
\section{ブラウン系の派生ソース}\label{petites-sauces-brunes-composees}}

\frsec{Petites Sauces Brunes Composées}

\index{そーす@ソース!ふらうんはせい@\textbf{ブラウン系の派生---}|(}
\index{sauce@sauce!petites brunes composees@\textbf{Petites ---s Brunes Composées}|(}

\end{Main}

\begin{recette}

\hypertarget{sauce-bigarade}{%
\subsubsection[ソース・ビガラード]{\texorpdfstring{ソース・ビガラード\footnote{ビガラードは本来、南フランスで栽培されるビターオレンジの一種。}}{ソース・ビガラード}}\label{sauce-bigarade}}

\frsub{Sauce Bigarade}

\index{そーす@ソース!ひからーと@---・ビガラード}
\index{ひからーと@ビガラード!そーす@ソース・---}
\index{sauce@sauce!bigarade@--- Bigarade}
\index{bigarade@bigarade!sauce@Sauce ---} \index[src]{bigarade@Bigarade}
\index[src]{ひからーと@ビガラード}

\hypertarget{sauce-bigarade-pour-caneton-braise}{%
\subparagraph{仔鴨のブレゼ用}\label{sauce-bigarade-pour-caneton-braise}}

\ldots{}\ldots{}仔鴨をブレゼ\footnote{料理の仕立てとしてのブレゼはたんに「蒸し煮」することではない。原則的な手順をごく簡単に述べておく。厚めに輪切りにしたにんじんと玉ねぎをバターまたはラードで炒め、ブーケガルニとともに鍋に入れる。表面を色よく焼き固めた肉を、脂身の少ない肉の場合には豚背脂のシートで包んで素材がぴったり入る大きさ鍋に入れ、\protect\hyperlink{fonds-brun}{茶色いフォン}を注ぎ、蓋をしてオーブンに入れ、微沸騰の状態を保つようにして煮込む。火が通ったら肉を取り出し、鍋に残った煮汁でソースを作る。詳細については\protect\hyperlink{releves-et-entrees}{第7章 肉料理}参照。}した際の煮汁を漉してから浮き脂を取り除き\footnote{dégraisser
  デグレセ。}、煮詰める。煮詰まったらさらに目の細かい布で漉し、ソース1
Lあたりオレンジ4個とレモン1個の搾り汁でのばす。

\hypertarget{sauce-bigarade-pour-caneton-poele}{%
\subparagraph{仔鴨のポワレ用}\label{sauce-bigarade-pour-caneton-poele}}

\ldots{}\ldots{}仔鴨をポワレ\footnote{ポワレについても簡単に述べておく。本書においてポワレは「フライパンで焼く」という意味で用いられることは決してない(フライパンで魚などを焼くことをポワレと呼ぶようになったのは20世紀後半のこと)。本書では「ローストの一種」と定義されており(この点がカレームとはまったく異なる)、3〜4mm角に切った香味野菜(マティニョン)を生のまま鍋の底に入れ、その上に味付けをした肉を置く。溶かしバターをかけてから、蓋をして中火のオーブンに入れて蒸し焼きにする。時折様子を見て溶かしバターをかけてやること。肉に火が通ったら鍋から取り出し、\protect\hyperlink{jus-de-veau-brun}{茶色い仔牛のフォン}を注いで弱火にかけて10分程煮込み、マティニョンとして用いた野菜から風味を引き出してソースにする。これがレシピにある「ポワレのフォン」となる。}のフォンから浮き脂を取り除き、でんぷんで軽くとろみ付けする。砂糖20
gに大さじ \(\frac{1}{2}\)
杯のヴィネガーを加えて火にかけカラメル状にしたものを加える。ブレゼ用と同様に、オレンジとレモンの搾り汁でのばす。

仔鴨のブレゼ用、ポワレ用いずれの場合も、細かい千切りにしてよく下茹でしておいたオレンジの皮大さじ2とレモンの皮\footnote{柑橘類の表皮を薄く剥いてごく細い千切りにしたり、器具を用いておろしたものをzeste(ゼスト)と呼ぶ。千切りにしたものは苦味を取り除くために下茹ですることが多い。}大さじ1を加えて仕上げる。

\atoaki{}

\hypertarget{sauce-bordelaise}{%
\subsubsection{ボルドー風ソース}\label{sauce-bordelaise}}

\frsub{Sauce Bordelaise}

\index{そーす@ソース!ほるとーふう@ボルドー風---}
\index{ほるとーふう@ボルドー風!そーす@---ソース}
\index{sauce@sauce!bordelaise@--- Bordelaise}
\index{bordelais@bordelais(e)!sauce@sauce ---e}
\index[src]{bordelaise@Bordelaise} \index[src]{ほるとーふう@ボルドー風}

赤ワイン3
dLにエシャロットのみじん切り大さじ2、粗く砕いたこしょう、タイム、ローリエの葉
\(\frac{1}{2}\)
枚を加えて火にかけ、\(\frac{1}{4}\)量になるまで煮詰める。ソース・エスパニョル1
dLを加えて火にかけ、浮いてくる夾雑物を丁寧に取り除きながら弱火で15分間煮る。目の細かい布で漉す。

溶かした\protect\hyperlink{glace-de-viande}{グラスドヴィアンド}大さじ1杯とレモン汁
\(\frac{1}{4}\)
個分、細かいさいの目か輪切りにしてポシェしておいた牛骨髄を加えて仕上げる。

\ldots{}\ldots{}牛、羊の赤身肉\footnote{原文 viande noire de boucherie
  (ヴィヨンドノワールドブシュリ)逐語訳すれば「肉屋の赤身肉」(noir(e)は黒の意だが肉の場合は赤身肉を指す)だが、一般的に
  viande de boucherie
  (ヴィヨンドドブシュリ)とだけ言えば、\ul{伝統的に肉屋で扱かわれてき\\た、白身肉を除く畜産精
  肉のことで、具体的には牛、羊、場合によっては馬も含まれる}(馬肉は食材としてあまり一般的ではないが、専門店がある)。副生物(内臓や足、耳、舌肉など)は含まれない。この場合の「白身肉」とは一般的に乳呑仔牛、乳呑仔羊のことであり、鶏(およびその他の家禽)や豚は別扱いになる。ここでわざわざviande
  noire
  赤身肉と表現しているのは、19世紀後半以降、上記のような区分がやや曖昧になったことによるものだろう。以下、本書の訳ではviande
  de boucherieの訳語として「牛、羊肉」をあてることにする。}のグリル用

\hypertarget{nota-sauce-bordelaise}{%
\subparagraph{【原注】}\label{nota-sauce-bordelaise}}

こんにちではボルドー風ソースをこのように赤ワインを用いて作るが、本来的には誤りである。元来は白ワインが用いられていた。これは\protect\hyperlink{sauce-bonnefoy}{ボルドー風ソース・ボヌフォワ}として後述。

\atoaki{}

\hypertarget{sauce-bourguignonne}{%
\subsubsection{ブルゴーニュ風ソース}\label{sauce-bourguignonne}}

\frsub{Sauce Bourguignonne}

\index{そーす@ソース!ふるこーにゆふう@ブルゴーニュ風---}
\index{ふるこーにゆふう@ブルゴーニュ風!そーす@---ソース}
\index{sauce@sauce!bourguignonne@--- Bourguignonne}
\index{bourguignon@bourguignon(ne)!sauce@Sauce Bourguignonne}
\index[src]{bourguignonne@Bourguignonne}
\index[src]{ふるこーにゆふう@ブルゴーニュ風}

上質の赤ワイン1 \(\frac{1}{2}\) L
に、エシャロット5個の薄切りとパセリの枝、タイム、ローリエの葉
\(\frac{1}{2}\) 枚、マッシュルームの切りくず\footnote{料理に使うマッシュルームは通常、トゥルネ(tourner
  包丁を持った側の手は動かさずに材料を回して切ることからついた用語)すなわち螺旋状に装飾して供するが、その際に少なくない量(具体的には重量で15〜20%)の切りくずが出るのでこれを使う。}25
gを加えて、半量になるまで煮詰める。布で漉し、ブールマニエ80 g(バター45
gと小麦粉 35 g)を加えてとろみを付ける。提供直前にバター150
gを溶かし込み、カイエンヌ\footnote{赤唐辛子の粉末だがカイエンヌは本来、品種名。日本のタカノツメと比べると辛さもややマイルドで、風味も異なる。}ごく少量で加えて風味よく仕上げる。

\ldots{}\ldots{}いろいろな卵料理や、家庭料理に好適なソース。

\atoaki{}

\hypertarget{sauce-bretonne}{%
\subsubsection{ブルターニュ風ソース}\label{sauce-bretonne}}

\frsub{Sauce Bretonne}

\index{そーす@ソース!ふるたーにゆふうふらうんけい@ブルターニュ風---(ブラウン系)}
\index{ふるたーにゆふう@ブルターニュ風!そーすふらうんけい@---ソース(ブラウン系)}
\index{sauce@sauce!bretonne brune@--- Bretonne (brune)}
\index{breton@breton(ne)!sauce brune@Sauce Bretonne (brune)}
\index[src]{bretonne@Bretonne}
\index[src]{ふるたーにゆふう@ブルターニュ風}

中位の玉ねぎ2個をみじん切りにして、バターでブロンド色になるまで炒める。白ワイン2
\(\frac{1}{2}\)
dLを注ぎ、半量になるまで煮詰める。ここにソース・エスパニョル3
\(\frac{1}{2}\)
dLおよびトマトソース同量を加える。7〜8分間煮立ててから、刻んだパセリを加えて仕上げる。

\hypertarget{nota-sauce-bretonne}{%
\subparagraph{【原注】}\label{nota-sauce-bretonne}}

このソースは\protect\hyperlink{haricots-blancs-bretonne}{白いんげん豆のブルターニュ風}以外にはほとんど使われない。

\atoaki{}

\hypertarget{sauce-aux-cerises}{%
\subsubsection[ソース・スリーズ]{\texorpdfstring{ソース・スリーズ\footnote{cerise
  (スリーズ)
  さくらんぼの意。このレシピではグロゼイユ(すぐり)のジュレを用いるが、古くはさくらんぼを用いていたことからこの名称となったと言われている。}}{ソース・スリーズ}}\label{sauce-aux-cerises}}

\frsub{Sauce aux Cerises}

\index{そーす@ソース!すりーす@---・スリーズ}
\index{くろせいゆ@グロゼイユ!そーす@ソース!すりーす@---・スリーズ}
\index{さくらんほ@サクランボ!そーす@ソース!すりーす@---・スリーズ}
\index{sauce@sauce!cerise@--- aux Cerises}
\index{cerise@cerise!sauce@sauce aux ---s}
\index[src]{cerises@Cerises (aux)} \index[src]{すりーす@スリーズ}

ポルト酒2 dLにイギリス風ミックススパイス\footnote{Mixed
  spiceのこと。Pudding
  spiceとも呼ばれる。シナモン、ナツメグ、オールスパイスの組み合わせが典型的。これにクローブ、生姜、コリアンダーシード、キャラウェイシードなどが加わっていることも多い。}1つまみと、すりおろしたオレンジの皮を大さじ
\(\frac{1}{2}\) 杯加えて \(\frac{2}{3}\)
量になるまで煮詰める。\protect\hyperlink{gelee-de-groseilles-a}{グロゼイユのジュレ}
2 \(\frac{1}{2}\) dLを加え、仕上げにオレンジ果汁を加える。

\ldots{}\ldots{}大型ジビエの料理用だが、鴨のポワレやブレゼにも用いられる。

\atoaki{}

\hypertarget{sauce-aux-champignons}{%
\subsubsection[ソース・シャンピニョン]{\texorpdfstring{ソース・シャンピニョン\footnote{champignons
  キノコ全般を意味する語だが、単独で用いられる場合は champignons de
  Paris(シャンピニョンドパリ)いわゆるマッシュルームを指す。}}{ソース・シャンピニョン}}\label{sauce-aux-champignons}}

\frsub{Sauce aux Champignons}

\index{そーす@ソース!まつしゆるーむ@マッシュルーム ⇒ ---・シャンピニョン}
\index{まっしゅるーむ@マッシュルーム ⇒ シャンピニョン}
\index{しやんひによん@シャンピニョン!そーすふらうんけいはせい@ソース・---(ブラウン系)}
\index{sauce@sauce!champignons brune@--- aux Champignons (brune)}
\index{champignon@champignon!sauce brune@Sauce aux Champignons (brune)}
\index[src]{champignons@Champignons (aux)}
\index[src]{しやんひによん@シャンピニョン}

マッシュルームの茹で汁2 \(\frac{1}{2}\) dL を半量になるまで煮詰める。
\protect\hyperlink{sauce-demi-glace}{ソース・ドゥミグラス} 8
dLを加えて数分間煮立てる。布で漉し、バター50
gを投入して味を調え、あらかじめ下茹でしておいた小さめのマッシュルームの笠100
gを加えて仕上げる。

\atoaki{}

\hypertarget{sauce-charcutiere}{%
\subsubsection[ソース・シャルキュティエール]{\texorpdfstring{ソース・シャルキュティエール\footnote{シャルキュトリ(豚肉加工業)風、の意。Charcutrieの語源はchar(肉)
  +cuite(調理された)+rie(業)。ハムやソーセージなどと定番の組合せであるマスタードを使う\protect\hyperlink{sauce-robert}{ソース・ロベール}と、おなじく定番のつけ合わせであるコルニション(小さいうちに収穫してヴィネガー漬けにしたきゅうり。専用品種がある)を使うことに由来。}}{ソース・シャルキュティエール}}\label{sauce-charcutiere}}

\frsub{Sauce Charcutière}

\index{そーす@ソース!しやるききゆとりふう@シャルキュトリ風 ⇒ ---・シャルキュティエール}
\index{しやるきゆとりふう@シャルキュトリ風!そーす@---ソース ⇒ ソース・シャルキュティエール}
\index{sauce@sauce!charcutiere@--- Charcutière}
\index{charcutier@charcutier(ère)!sauce@Sauce Charcutière}
\index[src]{charcutiere@Charcutière}
\index[src]{しやるきゆていえーる@シャルキュティエール}

提供直前に、\protect\hyperlink{sauce-robert}{ソース・ロベール} 1
Lに細さ2 mm程度、短かめの千切り\footnote{julenne
  (ジュリエーヌ)1〜2mm程度の細さの千切りにした野菜などのこと。調理現場によって「ジュリエンヌ」「ジュリアン」(なぜか男性名)と呼ぶところもある。}
にしたコルニション\footnote{シャルキュトリ(豚肉加工業)風、の意。Charcutrieの語源はchar(肉)
  +cuite(調理された)+rie(業)。ハムやソーセージなどと定番の組合せであるマスタードを使う\protect\hyperlink{sauce-robert}{ソース・ロベール}と、おなじく定番のつけ合わせであるコルニション(小さいうちに収穫してヴィネガー漬けにしたきゅうり。専用品種がある)を使うことに由来。}
100
gを加える(\protect\hyperlink{sauce-robert}{ソース・ロベール}参照)。

\atoaki{}

\hypertarget{sauce-chasseur}{%
\subsubsection[ソース・シャスール]{\texorpdfstring{ソース・シャスール\footnote{狩人風、の意。古くは猟獣肉をすり潰したものを使った料理を指したという説もある。マッシュルームとエシャロット、白ワインを使うのが特徴であり、このソースを使った料理にも「シャスール」の名が付けられる。}}{ソース・シャスール}}\label{sauce-chasseur}}

\frsub{Sauce Chasseur}

\index{そーす@ソース!しやすーる@---・シャスール}
\index{しやすーる@シャスール!そーす@ソース・---}
\index{かりうとふう@狩人風 ⇒ ソース・シャスール}
\index{sauce@sauce!chasseur@--- Chasseur}
\index{chasseur@chasseur!sauce@Sauce ---} \index[src]{chasseur@Chasseur}
\index[src]{しやすーる@シャスール}

生のマッシュルームを薄切りにしたもの150 gをバターで炒める。エシャロット
\footnote{échalote
  玉ねぎによく似ているが、小ぶりで水分が少なく、香味野菜としてよく用いられる。伝統的な品種は種子ではなく種球を植えて栽培する。なお、日本でしばしば「エシャレット」の名称で流通しているものはラッキョウの若どりであり、フランス料理で用いるエシャロットとはまったく異なる。}のみじん切り大さじ2
\(\frac{1}{2}\) 杯を加えてさらに軽く炒め、白ワイン3 dL
を注ぎ、半量になるまで煮詰める。\protect\hyperlink{sauce-tomate}{トマトソース}
3 dL と\protect\hyperlink{sauce-demi-glace}{ソース・ドゥミグラス} 2
dLを加える。数分間沸騰させたら、バター150 gと、セルフイユ\footnote{cerfeuil
  日本ではチャービルとも呼ばれるセリ科のハーブ。}とエストラゴン\footnote{estragon
  日本ではタラゴンとも呼ばれるヨモギ科のハーブ。フランス料理ではとても好まれる重要なハーブのひとつ。フレンチタラゴンとロシアンタラゴンの2種がある。料理に用いるのはフレンチタラゴンであり、この品種は種子ではなく株分けや挿し芽で殖やして栽培される。寒さには比較的強いが、日本の梅雨の湿度や夏の暑さには弱い。}をみじん切りにしたもの大さじ1
\(\frac{1}{2}\) 杯を加えて仕上げる。

\atoaki{}

\hypertarget{sauce-chasseur-procede-escoffier}{%
\subsubsection{ソース・シャスール(エスコフィエ流)}\label{sauce-chasseur-procede-escoffier}}

\frsub{Sauce Chasseur (Procédé Escoffier)}

\index{そーす@ソース!しやすーるえすこふいえ@---・シャスール(エスコフィエ流)}
\index{しやすーる@シャスール!そーすしゃすーるえすこふぃえ@ソース・---(エスコフィエ流)}
\index{かりうとふう@狩人風 ⇒ ソース・シャスール(エスコフィエ流)}
\index{sauce@sauce!chasseur escoffier@--- Chasseur (Procédé Escoffier)}
\index{chasseur@chasseur!sauce escoffier@Sauce --- (Procédé Escoffier)}
\index[src]{しやすーるえすこふいえ@シャスール(エスコフィエ)}
\index[src]{chasseur escoffier@Chasseur (Escoffier)}

生のマッシュルームを薄切りにしたもの150
gを、バターと植物油で軽く色付くまで炒める。みじん切りにしたエシャロット大さじ1杯を加え、なるべくすぐに余分な油をきる。白ワイン2
dL とコニャック約50 mL
を注ぎ、半量になるまで煮詰める。\protect\hyperlink{sauce-demi-glace}{ソース・ドゥミグラス}
4 dLと\protect\hyperlink{sauce-tomate}{トマトソース} 2
dL、\protect\hyperlink{glace-de-viande}{グラスドヴィアンド}大さじ
\(\frac{1}{2}\) 杯を加える。

5分間沸騰させたら、仕上げにパセリのみじん切り少々を加える。

\atoaki{}

\hypertarget{sauce-chaud-froid-brune}{%
\subsubsection[茶色いソース・ショフロワ]{\texorpdfstring{茶色いソース・ショフロワ\footnote{chaud-froid(ショフロワ)はchaudショ「熱い、温かい」とfroidフロワ「冷たい」の合成語で、火を通した肉や魚を冷まし、表面にこのソース・ショフロワを覆うように塗り付け、さらにジュレを覆いかけた料理。料理の発祥については諸説あり、なかでもルイ15世に仕えていた料理長ショフロワChaufroixが考案したという説を支持してなのか、英語ではこの料理をChaufroixと綴ることも多い。Chaud-froidの表記は19世紀後半には文献に見られる。なお、複数形はchauds-froidsと綴る。トリュフの薄切りやエストラゴンなどのハーブその他で表面に華麗な装飾を施すことが19世紀には盛んに行なわれていた。現代でも装飾に凝った仕立てにするケースは多い。}}{茶色いソース・ショフロワ}}\label{sauce-chaud-froid-brune}}

\frsub{Sauce Chaud-froid brune}

\index{そーす@ソース!しよふろわちやいろ@茶色い---・ショフロワ}
\index{しよふろわ@ショフロワ!そーすふらうんけい@茶色いソース・---}
\index{sauce@sauce!chaud-froid brune@--- Chaud-froid brune}
\index{chaud-froid@chaud-froid!sauce brune@Sauce --- brune}
\index[src]{chaud-froid brune@Chaud-froid brune}
\index[src]{しよふろわしやいろ@ショフロワ(茶色)}

(仕上がり1 L分)

\protect\hyperlink{sauce-demi-glace}{ソース・ドゥミグラス}
\(\frac{3}{4}\) Lとトリュフエッセンス1 dL、ジュレ6〜7 dLを用意する。

ソース・ドゥミグラスにトリュフエッセンスを加えて、強火で煮詰めるが、この時に鍋から離れないこと。煮詰めながらジュレを少量ずつ加えていく。最終的に\(\frac{2}{3}\)量程度まで煮詰める。

味見をして、ソースがショフロワに使うのに丁度いい濃さになっているか確認すること。

マデイラ酒またはポルト酒 \(\frac{1}{2}\)
dLを加える。布で漉し、ショフロワの主素材の表面に塗り付けるのに丁度いい固さになるまで、丁寧にゆっくり混ぜながら冷ます。

\atoaki{}

\hypertarget{sauce-chaud-froid-brune-pour-canards}{%
\subsubsection{茶色いソース・ショフロワ(鴨用)}\label{sauce-chaud-froid-brune-pour-canards}}

\frsub{Sauce Chaud-froid brune pour Canards}

\index{そーす@ソース!しよふろわちやいと@茶色い---・ショフロワ(鴨用)}
\index{しよふろわ@ショフロワ!ちやいろいそーすかもよう@茶色いソース・---(鴨用)}
\index{sauce@sauce!chaud-froid brune pour canards@--- Chaud-froid brune pour Canards}
\index{chaud-froid@chaud-froid!sauce brune pour Canards@Sauce --- brune pour Canards}
\index[src]{chaud-froid brune canards@Chaud-froid brune pour Canards}
\index[src]{しよふろわちやいろかも@ショフロワ(茶色、鴨用)}

作り方は上記、\protect\hyperlink{sauce-chaud-froid-brune}{茶色いソース・ショフロワ}と同様だが、トリュフエッセンスではなく、鴨のガラでとったフュメ1
\(\frac{1}{2}\)
dLを用いること。また、上記のレシピよりややしっかり煮詰めること。

ソースを布で漉したら、オレンジ3個分の搾り汁、とオレンジの皮をごく薄く剥いて細かい千切りにしたもの\footnote{zeste
  ゼスト。オレンジやレモンの皮の表面を器具を用いてすりおろすか、ナイフでごく薄く表皮を向き、細かい千切りにしたもの。ここでは後者を使う指定になっている。}大さじ2杯を加える。オレンジの皮の千切りはしっかりと下茹でしてよく水気をきっておくこと。

\atoaki{}

\hypertarget{sauce-chaud-froid-brune-pour-gibier}{%
\subsubsection{茶色いソース・ショフロワ(ジビエ用)}\label{sauce-chaud-froid-brune-pour-gibier}}

\frsub{Sauce Chaud-froid brune pour Gibier}

\index{そーす@ソース!しよふろわちやいろじびえよう@茶色い---・ショフロワ(ジビエ用)}
\index{しよふろわ@ショフロワ!そーすしよふろわちやいろじびえよう@茶色いソース・---(ジビエ用)}
\index{sauce@sauce!chaud-froid brune pour Gibier@--- Chaud-froid brune pour Gibier}
\index{chaud-froid@chaud-froid!sauce brune pour Gibier@Sauce --- brune pour Gibier}
\index[src]{chaud-froid brune gibier@Chaud-froid brune pour Gibier}
\index[src]{しよふろわちやいろしひえ@ショフロワ(茶色、ジビエ用)}

作り方は上記\protect\hyperlink{sauce-chaud-froid-brune}{標準的なソース・ショフロワ}と同じだが、トリュフエッセンスではなく、ショフロワとして供するジビエのガラでとったフュメ\footnote{\protect\hyperlink{fonds-de-gibier}{ジビエのフォン}参照。}2
dLを用いること。

\atoaki{}

\hypertarget{sauce-chaud-froid-tomatee}{%
\subsubsection{トマト入りソース・ショフロワ}\label{sauce-chaud-froid-tomatee}}

\frsub{Sauce Chaud-froid tomatée}

\index{そーす@ソース!しよふろわとまといり@トマト入り---・ショフロワ}
\index{しよふろわ@ショフロワ!そーすちやいろとまといり@トマト入りソース・---}
\index{sauce@sauce!chaud-froid tomatée@--- Chaud-froid tomatée}
\index{chaud-froid@chaud-froid!sauce tomatée@Sauce --- tomatée}
\index[src]{chaud-froid tometee@Chaud-froid tomatée}
\index[src]{しよふろわとまて@トマト入りショフロワ}

良質で、既によく煮詰めてあるトマトピュレ1 Lを、さらに煮詰めながら7〜8
dLのジュレを少しずつ加えていく。全体量が1 L以下になるまで煮詰めること。

布で漉し、使いやすい固さになるまで、ゆっくり混ぜながら冷ます。

\atoaki{}

\hypertarget{sauce-chevreuil}{%
\subsubsection{ソース・シュヴルイユ}\label{sauce-chevreuil}}

\frsub{Sauce Chevreuil}

\index{しゆうるいゆ@シュヴルイユ!そーす@ソース・---}
\index{そーす@ソース!しゆうるいゆ@---・シュヴルイユ}
\index{のろしか@ノロ鹿 ⇒ シュヴルイユ!そーす@ソース!しゆうるいゆ@ソース・シュヴルイユ}
\index{sauce@sauce!chevreuil@--- Chevreuil}
\index{chevreuil@chevreuil!sauce@Sauce ---}
\index[src]{chevreuil@Chevreuil} \index[src]{しゆうるいゆ@シュヴルイユ}

\protect\hyperlink{sauce-poivrade}{標準的なソース・ポワヴラード}と同様に作るが、

\begin{enumerate}
\def\labelenumi{\arabic{enumi}.}
\item
  マリネした牛・羊肉の料理に添える場合\footnote{chevreuil
    シュヴルイユはノロ鹿のことだが、このように事前にマリネした牛・羊肉を用いた料理にもこのソースを使い「シュヴルイユ(風)(仕立て)」と\ruby{謳}{うた}う。1806年刊ヴィアール『帝国料理の本』においてノロ鹿のフィレは香辛料を加えたワインヴィネガーで48時間マリネしてから調理すると書かれている。オド『女性料理人のための本』では、確認出来た1834年の第4版から1900年の第78版に至るまで、ノロ鹿の項において「一週間もヴィネガーたっぷりの漬け汁でマリネするのはやりすぎだが、強い味が好みなら1〜4日間」香辛料と赤ワインあるいはヴィネガーでマリネするといい、と説明されている。つまり、ノロ鹿とは必ずマリネしてから調理するものという一種のコンセンサスがあったために、マリネした牛・羊肉の料理にも「シュヴルイユ(風)」の名称が謳われるようになったと考えられる。}は、ハム入りの\protect\hyperlink{mirepoix}{ミルポワ}を加える。
\item
  ジビエ料理に添える場合は、そのジビエの端肉を加える。
\end{enumerate}

素材をヘラなどで強く押し付けるようにして漉す\footnote{シノワ(\protect\hyperlink{sauce-espagnole}{ソース・エスパニョル}訳注参照)などを用いる。}。良質の赤ワイン
1 \(\frac{1}{2}\)
dLをスプーン1杯ずつ加えながら煮て、浮き上がってくる不純物を丁寧に取り除いていく\footnote{dépouiller
  デプイエ ≒ écumer エキュメ。}。

最後に、カイエンヌごく少量と砂糖1つまみを加えて味を\ruby{調}{とと
の}え、布で漉す。

\atoaki{}

\hypertarget{sauce-colbert}{%
\subsubsection[ソース・コルベール]{\texorpdfstring{ソース・コルベール\footnote{17世紀の政治家、ジャン・バティスト・コルベール(1619〜1683)の名を冠したもの。}}{ソース・コルベール}}\label{sauce-colbert}}

\frsub{Sauce Colbert}

\index{そーす@ソース!こるへーる@---・コルベール}
\index{こるへーる@コルベール!そーす@ソース・---}
\index{sauce@sauce!colbert@--- Colbert}
\index{colbert@Colbert!sauce@Sauce ---} \index[src]{colbert@Colbert}
\index[src]{こるへーる@コルベール}

\protect\hyperlink{beurre-maitre-d-hotel}{メートルドテルバター}に\protect\hyperlink{glace-de-viande}{グラスドヴィアンド}を加えたもののことだが、正しくは「\protect\hyperlink{beurre-colbert}{ブール・コルベール}」と呼ぶべきものだ\footnote{具体的なレシピは\protect\hyperlink{beurre-colbert}{ブール・コルベール}参照のこと。}。

また、ブール・コルベールと\protect\hyperlink{sauce-chateaubriand}{ソース・シャトーブリアン}との違いを明確にさせようとして、メートルドテルバターにエストラゴンを加える者もいる。だが、必ずそうすべきということではない。実際、ブール・コルベールとソース・シャトーブリアンは明らかに違うものだからだ。ソース・シャトーブリアンは軽く仕上げたグラスドヴィアントにバターとパセリのみじん切りを加えたものである。一方、ブール・コルベールあるいはソース・コルベールと呼ばれているものはあくまでもバターが主であって、グラスドヴィアンドは補助的なものに過ぎない。

\atoaki{}

\hypertarget{sauce-diable}{%
\subsubsection[ソース・ディアーブル]{\texorpdfstring{ソース・ディアーブル\footnote{diable
  (ディアーブル)悪魔の意。}}{ソース・ディアーブル}}\label{sauce-diable}}

\frsub{Sauce Diable}

\index{そーす@ソース!ていあーふる@---・ディアーブル}
\index{ていあーふる@ディアーブル!そーす@ソース・---}
\index{あくま@悪魔 ⇒ ディアーブル!そーす@ソース!そーすていあーふる@ソース・ディアーブル}
\index{sauce@sauce!diable@--- Diable}
\index{diable@diable!sauce@Sauce ---} \index[src]{diable@Diable}
\index[src]{ていあーふる@ディアーブル}

このソースはごく少量ずつ作るのが一般的だが、ここではそれを守らずに、仕上り2
\(\frac{1}{2}\) dLとして説明する

白ワイン3
dLにエシャロット3個分のみじん切りを加え、\(\frac{1}{3}\)量以下になるまで煮詰める。

\protect\hyperlink{sauce-demi-glace}{ソース・ドゥミグラス} 2
dLを加えて数分間煮立たせ、仕上げにカイエンヌの粉末をたっぷり効かせる\footnote{「たっぷり」という表現に惑わされないよう注意。}。

\ldots{}\ldots{}鶏と鳩のグリルに合わせる。

\hypertarget{nota-sauce-diable}{%
\subparagraph{【原注】}\label{nota-sauce-diable}}

白ワインではなくヴィネガーを煮詰め、仕上げにハーブを加えて作る調理現場もあるが、著者としては本書で示しているの作り方がいいと思う。

\atoaki{}

\hypertarget{sauce-diable-escoffier}{%
\subsubsection{ソース・ディアーブル・エスコフィエ}\label{sauce-diable-escoffier}}

\frsub{Sauce Diable Escoffier}

\index{そーす@ソース!ていあーふるえすこふいえ@---・ディアーブル・エスコフィエ}
\index{ていあーふる@ディアーブル!そーす@ソース!えすこふいえ@ソース・---・エスコフィエ}
\index{あくま@悪魔 ⇒ ディアーブル!そーす@ソース!エスコフイエ@ソース・ディアーブル・エスコフィエ}
\index{sauce@sauce!diable escoffier@--- Diable Escoffier}
\index{diable@diable!sauce escoffier@Sauce --- Escoffier}
\index[src]{diable escoffier@Diable Escoffier}
\index[src]{ていあーふるえすこふいえ@ディアーブル・エスコフィエ}

このソースは完成品が市販\footnote{現在は市販されていないと思われる。フランスにおいては未確認だが、
  1980年代までアメリカ合衆国ではナビスコがソース・ロベール・エスコフィエとともに瓶詰めを生産、販売していた。初版ではこれら2つの製品への言及がなく、第二版で追加されたことから、1903年〜1907年の間に製品化された可能性もある。また、第二版(1907年)と同年の英訳版、第三版(1912年)にはソース・スリーズ・エスコフィエの記述が見られるが、これは第四版で削除されており、生産中止になったと思われる。エスコフィエ・ブランドの既製品ソースはさらに他にもあったようだが詳細は不明。なお、エスコフィエは1922年頃、ジュリユス・マジがブイヨンキューブ(日本では「マギーブイヨン」の商品名)を開発する際にも協力した。}されている。同量の柔くしたバターを混ぜ合わせるだけでいい。

\atoaki{}

\hypertarget{sauce-diane}{%
\subsubsection[ソース・ディアーヌ]{\texorpdfstring{ソース・ディアーヌ\footnote{ローマ神話の女神ディアーナのこと。ギリシア神話のアルテミスに相当し、狩猟、貞潔の女神。また月の女神ルーナ(セレーネー)と同一視された。ここでは大型ジビエ料理用のソースであるから、狩猟の女神という意味合いが強い。}}{ソース・ディアーヌ}}\label{sauce-diane}}

\frsub{Sauce Diane}

\index{そーす@ソース!ていあーぬ@---・ディアーヌ}
\index{ていあーぬ@ディアーヌ!そーす@ソース・---}
\index{sauce@sauce!diane@--- Diane} \index{diane@Diane!sauce@Sauce ---}
\index[src]{diane@Diane} \index[src]{ていあーぬ@ディアーヌ}

不純物を充分に取り除き、コクと風味ゆたかな\protect\hyperlink{sauce-poivrade}{ソース・ポワヴラード}
5 dLを用意する。提供直前に、泡立てた生クリーム4 dL (生クリーム2
dLを泡立てて倍量にする)と、小さな三日月の形にしたトリュフのスライスと固茹で卵の白身を加える。

\ldots{}\ldots{}大型ジビエの骨付き背肉および、その中心部を円筒形に切り出したもの
\footnote{noisette ノワゼット。}、フィレ料理用。

\atoaki{}

\hypertarget{sauce-duxelles}{%
\subsubsection[ソース・デュクセル]{\texorpdfstring{ソース・デュクセル\footnote{\protect\hyperlink{duxelles-seche}{デュクセル・セッシュ}(第2章ガルニチュール)を用いることからこの名称が用いられている。}}{ソース・デュクセル}}\label{sauce-duxelles}}

\frsub{Sauce Duxelles}

\index{そーす@ソース!てゆくせる@---・デュクセル}
\index{てゆくせる@デュクセル!そーす@ソース・---}
\index{sauce@sauce!duxelles@--- Duxelles}
\index{duxelles@duxelles!sauce@Sauce ---}
\index[src]{てゆくせる@デュクセル} \index[src]{duxelles@Duxelles}

白ワイン2 dLとマッシュルームの茹で汁2
dLにエシャロットのみじん切り大さじ2
杯を加えて、\(\frac{1}{3}\)量まで煮詰める。\protect\hyperlink{sauce-demi-glace}{ソース・ドゥミグラス}
\(\frac{1}{2}\) Lとトマトピュレ1 \(\frac{1}{2}\)
dL、\protect\hyperlink{duxelles-seche}{デュクセル・セッシュ}大さじ4杯を加える。5分間煮立たせ、パセリのみじん切り大さじ
\(\frac{1}{2}\) を加える。

\ldots{}\ldots{}グラタンの他、いろいろな料理に用いられる。

\hypertarget{nota-sauce-duxelles}{%
\subparagraph{【原注】}\label{nota-sauce-duxelles}}

ソース・デュクセルはイタリア風ソースと混同されることが多いが、ソース・デュクセルにはハムも、赤く漬けた舌肉も入れないので、まったく別のものだ。

\atoaki{}

\hypertarget{sauce-estragon}{%
\subsubsection[ソース・エストラゴン]{\texorpdfstring{ソース・エストラゴン\footnote{ヨモギ科のハーブ。\protect\hyperlink{sauce-chasseur}{ソース・シャスール}訳注参照。}}{ソース・エストラゴン}}\label{sauce-estragon}}

\frsub{Sauce Estragon}

\index{そーす@ソース!えすとらこんちゃいろ@---・エストラゴン(ブラウン系)}
\index{えすとらこん@エストラゴン!そーすふらうんけい@ソース・---(ブラウン系)}
\index{sauce@sauce!estragon brune@--- Estragon (brune)}
\index{estragon@estragon!sauce brune@Sauce --- (brune)}
\index[src]{estragon@Estragon} \index[src]{えすとらこん@エストラゴン}

(仕上がり2 \(\frac{1}{2}\) dL分)

白ワイン2 dLを沸かし、エストラゴンの枝20
gを投入する。蓋をして10分間、煎じる\footnote{infuser(アンフュゼ)煮出す、煎じる、の意。なおハーブティはこの派生語infusion(アンフュジオン)と呼ぶ。}。2
\(\frac{1}{2}\)
dLの\protect\hyperlink{sauce-demi-glace}{ソース・ドゥミグラス}または、\protect\hyperlink{jus-de-veau-lie}{とろみを付けた仔牛のジュ}を加え、約
\(\frac{2}{3}\)
量になるまで煮詰める。布で漉し、みじん切りにしたエストラゴン小さじ1杯を加えて仕上げる。

\ldots{}\ldots{}仔牛や仔羊の背肉の中心を円筒形に切り出した料理や家禽料理用。

\atoaki{}

\hypertarget{sauce-financiere}{%
\subsubsection[ソース・フィナンシエール]{\texorpdfstring{ソース・フィナンシエール\footnote{Financier徴税官(財務官)風の意。フランス革命以前の徴税官は、王に代わって徴税を行なう大貴族が就く役職であり、膨大な利権によりきわめて裕福であったという。このソースと組み合わせる\protect\hyperlink{garniture-financiere}{ガルニチュール・フィナンシエール}が、雄鶏のとさかと睾丸、仔羊の胸腺肉、トリュフなどの比較的入手困難あるいは高級とされる食材で構成されていることが名称の由来と思われる。ブリヤ=サヴァランは『美味礼讃』(味覚の生理学)において、徴税官たちは旬のはしりの食材を真っ先に食べられる、いわば特権階級だと述べている。なお、カレーム『19世紀フランス料理』においては、ソースとガルニチュールを分離せず、「ラグー・アラ・フィナンシエール」として採りあげられているが、全ての素材を別々に加熱調理してソースと合わせるものであり、いわゆる「煮込み」とは呼びがたいものとなっている。フランス料理の影響が比較的強かった北イタリアにこの原型に近いと思われるラグー「ピエモンテ風フィナンツィエラ」がある。鶏のとさか、肉垂、睾丸、鶏レバーおよび仔牛の胸腺肉などを煮込んだものだが、レシピを読む限りにおいては比較的庶民的あるいは農民的料理に変化したものと思われる
  (cf.~Anna Gosetti della Salda, \emph{Le Ricette Regionali Italiane},
  Milano, Solares, 1967,
  p.57.)。ちなみに焼き菓子のフィナンシエfinancierも同語源だが、何故その名称になったかは不明。}}{ソース・フィナンシエール}}\label{sauce-financiere}}

\frsub{Sauce Financière}

\index{そーす@ソース!ふいなんしえーる@---・フィナンシエール}
\index{ふいなんしえ@フィナンシエ/フィナンシエール!そーす@ソース・フィナンシエール}
\index{ちょうせいかんふう@徴税官風 ⇒ フィナンシエール!そーすふぃなんしえーる@ソース・フィナンシエール}
\index{sauce@sauce!financiere@--- Financière}
\index{financier@financier(ère)!sauce@Sauce Financière}
\index[src]{financierel@Financière}
\index[src]{ふいなんしえーる@フィナンシエール}

\protect\hyperlink{sauce-madere}{ソース・マデール} 1 \(\frac{1}{4}\) Lを
\(\frac{3}{4}\) 量以下になるまで煮詰め、火から外してトリュフエッセンス1
dLを加える。布で漉して仕上げる。

\ldots{}\ldots{}\protect\hyperlink{garniture-financiere}{ガルニチュール・フィナンシエール}用だが、その他の肉料理にも用いられる。

\atoaki{}

\hypertarget{sauce-aux-fines-herbes}{%
\subsubsection[香草ソース]{\texorpdfstring{香草ソース\footnote{料理名では、いわゆる「ハーブ」についてかつてfines
  herbesの表現が多く用いられた。だが、こんにちでは特定のハーブ名をソースや料理名に添えて言うことが多い。例えばCôtelette
  de veau au
  thymコトレットドヴォオタン(仔牛の骨付き背肉、タイム風味)、やFilet de
  bar poêlé, compote de tomate au basilicフィレドバールポワレ
  コンポットートドトマトバジリック(スズキのフィレとトマトのコンポート、バジル風味)など。また、栽培レベルで「香草、ハーブ」の総称としては
  herbes aromatiques
  (エルブザロマティック)、あるいはたんにaromatiques(アロマティック)が一般的。}}{香草ソース}}\label{sauce-aux-fines-herbes}}

\frsub{Sauce aux Fines Herbes}

\index{そーす@ソース!こうそうふらうんけい@香草---(ブラウン系)}
\index{こうそう@香草!そーすふらうんけい@---ソース(ブラウン系)}
\index{はーぶ@ハーブ ⇒ 香草!こうそうそーすふらうんけい@香草ソース(ブラウン系)}
\index{sauce@sauce!fines herbes@--- aux Fines Herbes}
\index{fines herbes@fines herbes!sauce@Sauce aux ---}
\index[src]{fines herbes@Fines herbes (aux)} \index[src]{こうそう@香草}

白ワイン3
dLを沸かし、パセリの葉、セルフイユ、エストラゴン、シブレットを各1つまみ強、投入する。約20分間煎じる。布で漉し、\protect\hyperlink{sauce-demi-glace}{ソース・ドゥミグラス}または\protect\hyperlink{jus-de-veau-lie}{とろみを付けた仔牛のジュ}
6
dLを加える。仕上げに、煎じるのに使ったのと同じ香草を細かく刻んだもの計、大さじ2
\(\frac{1}{2}\) 杯とレモンの搾り汁少々を加える。

\hypertarget{nota-sauce-aux-fines-herbes}{%
\subparagraph{【原注】}\label{nota-sauce-aux-fines-herbes}}

古典料理ではこの「香草ソース」と\protect\hyperlink{sauce-duxelles}{ソース・デュクセル}が混同されることもあったが、こんにちではまったく違うものとして扱われている。

\atoaki{}

\hypertarget{sauce-genevoise}{%
\subsubsection{ジュネーヴ風ソース}\label{sauce-genevoise}}

\frsub{Sauce Genevoise}

\index{そーす@ソース!しゆねーうふう@ジュネーヴ風---}
\index{しゆねーうふう@ジュネーヴ風!そーす@---ソース}
\index{sauce@sauce!genevoise@--- Genevoise}
\index{genevois@genevois(e)!sauce@Sauce Genevoise}
\index[src]{しゆねーうふう@ジュネーヴ風}
\index[src]{genevoise@Genevoise}

鍋にバターを熱し、細かく刻んだミルポワを色付かないよう強火でさっと炒める。ミルポワの材料は、にんじん100
g、玉ねぎ80 g、タイムとローリエ少々、パセリの枝20 g。そこにサーモンの頭1
kgと粗く砕いたこしょう1つまみを入れ、蓋をして弱火で15分程蒸し煮する。

鍋に残ったバターを捨て、赤ワイン1
Lを注ぐ。半量になるまで煮詰める。そこに\protect\hyperlink{sauce-espagnole-maigre}{魚料理用ソース・エスパニョル}
\(\frac{1}{2}\)
Lを加える。弱火で1時間煮込む。漉し器を使い、材料を押しつけながら漉す。しばらく休ませてから、表面に浮いた油脂を取り除く\footnote{dégraisser
  デグレセ。レードルなどを用いて浮いてきた余計な油脂を取り除く作業。}

さらに赤ワイン \(\frac{1}{2}\)
Lと、\protect\hyperlink{fumet-de-poisson}{魚のフュメ} \(\frac{1}{2}\)
Lを加える。ソースの表面に浮いてくる不純物を徹底的に取り除き\footnote{dépouiller
  デプイエ ≒ écumer エキュメ。}、丁度いい濃さになるまで煮詰める。

これを布で漉し、静かに混ぜながら、アンチョヴィのエッセンス大さじ1杯とバター150
gを加えて仕上げる。

\ldots{}\ldots{}サーモン、鱒料理用。

\hypertarget{nota-sauce-genevoise}{%
\subparagraph{【原注】}\label{nota-sauce-genevoise}}

このソースはもともとカレームが「ジェノヴァ風」\footnote{Sauce à la
  génoise au vin de Bordeaux
  ボルドー産ワインを用いたジェノヴァ風ソース(『19世紀フランス料理』第3巻、80頁)。本書のこのレシピと同様に魚料理用ソースだ。ボルドーの赤ワインにみじん切りにして下茹でしたマッシュルーム、トリュフ、エシャロットを加えてオールスパイスとこしょう少々を入れ、適度に煮詰める。ソース・エスパニョルと赤ワインを加え、湯煎にかけておく。提供直前にバター少量を加えて仕上げる、というもの。本書においてこのソースを「原型」とするのには疑問が残るところだろう。}と名付けたものだが、その後ルキュレ、グフェ\footnote{グフェ『料理の本』(1867年)の420ページにあるジュネーヴ風ソースは、薄切りにした玉ねぎ、エシャロット、粗挽きこしょう、にんにく、バターを鍋に入れて色付くまで炒め、そこにブルゴーニュ産赤ワインを注ぐ。弱火で玉ねぎに火が通るまで煮る。ソース・エスパニョルと仔牛のブロンドのジュを加えて煮詰め、布で漉す。提供直前にマデイラ酒の風味を加えて茹でたトリュフのみじん切りとアンチョビバターを加える、というもの。赤ワインと玉ねぎ、仕上げにアンチョビを加える点は共通しているが、グフェのが肉料理用であるのに対して、本書のこのソースは明らかに魚料理用であり、まったく同じソースと呼べるとは言い難い。}が立て続けに「ジュネーヴ風」の名称を用いた。だが、ジュネーヴは赤ワインの産地ではないから理屈としてはおかしい\footnote{料理名に冠された地名は、由来が明確にあるものがある一方で、まったく意味不明か、あるいはいい加減な思い付きで付けられたのではないかとさえ思われるものも少なくない。(à
  la) russe「ロシア風」や (à la)
  moscovite「モスクワ風」などはロシア料理起源か、あるいは18世紀末〜
  19世紀前半にかけてロシア帝国の宮廷や貴族がこぞってフランスから料理人を招聘し、帰国した彼らが創案した料理などはある程度しっかりとした由来がわかるものも多い。一方で、(à
  l')espagnole「スペイン風」(à l')italienne「イタリア風」(à la)
  romaine「ローマ風」(à la grecque) 「ギリシア風」(à
  l')allemande「ドイツ風」(à
  l')hollandaise「オランダ風」などは由来の不明なケースが非常に多い。\protect\hyperlink{sauce-espagnole}{ソース・エスパニョル}などはその典型例とも言うべきものだろう。この原注では由来に非常にこだわっているが、そもそもカレームのレシピは上述のように「ボルドー産ワインを用いたジェノバ風ソース」であるから、赤ワインの産地かどうかということは実はさしたる問題にはならない。重要なのは後半の、赤ワインを用いることがこのソースのポイントということ。}。

間違っているとはいえ、ジュネーヴ風という名称で定着してしまっているので、本書でもそのままにしている。だが、ジュネーヴ風であれジェノヴァ風であれ、カレーム、ルキュレ、デュボワ、グフェはいずれもこのソースに赤ワインを用いるよう指示している。つまり赤ワインを用いることがこのソースのポイント。

\atoaki{}

\hypertarget{sauce-godard}{%
\subsubsection[ソース・ゴダール]{\texorpdfstring{ソース・ゴダール\footnote{ガルニチュール・ゴダールの構成要素がガルニチュール・フィナンシエールとよく似ている点などから、おそらくは18世紀の徴税官(つまりフィナンシエ)であり作家としても活動したクロード・ゴダール・ドクール
  Claude Godard
  d'Aucour(1716〜1795)の名を冠したものと考えられる。なお、底本とした現行版(第四版)では最後がdではなくtとなっているが、初版から第三版にいたるまでdとなっており、現行版は明らかな誤植。}}{ソース・ゴダール}}\label{sauce-godard}}

\frsub{Sauce Godard}

\index{そーす@ソース!こたーる@---・ゴダール}
\index{こたーる@ゴダール!そーす@ソース・---}
\index{sauce@sauce!godard@--- Godard}
\index{godard@Godard!sauce@Sauce ---} \index[src]{godard@Godard}
\index[src]{こたーる@ゴダール}

シャンパーニュまたは辛口の白ワイン4
dLにハム入りの細かく刻んだ\protect\hyperlink{mirepoix}{ミルポワ}、\protect\hyperlink{sauce-demi-glace}{ソース・ドゥミグラス}
1 Lとマッシュルームのエッセンス2
dLを加える。弱火に10分かけ、シノワ\footnote{\protect\hyperlink{sauce-espagnole}{ソース・エスパニョル}訳注参照。}で漉す。

\(\frac{2}{3}\)量になるまで煮詰め、布で漉す。

\ldots{}\ldots{}\protect\hyperlink{garniture-godard}{ガルニチュール ゴタール}用。

\atoaki{}

\hypertarget{sauce-grand-veneur}{%
\subsubsection[ソース・グランヴヌール]{\texorpdfstring{ソース・グランヴヌール\footnote{王家や貴族に仕える狩猟長のことをgrand-veneur(グランヴヌール)と呼ぶ。}}{ソース・グランヴヌール}}\label{sauce-grand-veneur}}

\frsub{Sauce Grand-Veneur}

\index{そーす@ソース!くらんうぬーる@---・グランヴヌール}
\index{くらんうぬーる@グランヴヌール!そーす@ソース・---}
\index{sauce@sauce!grand-veneur@--- Grand-Veneur}
\index{grand-veneur@grand-veneur!sauce@Sauce ---}
\index[src]{くらんうぬーる@グランヴヌール}
\index[src]{grand-veneur@Grand-Veneur}

\protect\hyperlink{fonds-de-gibier}{大型ジビエのフュメ}で澄んだ色合いに作った\protect\hyperlink{sauce-poivrade}{ソース・ポワヴラード}に、ソース
1 Lあたり野うさぎの血1 dLをマリネ液1 dLで薄めたものを加える。

火をごく弱くして、血が沸騰しないよう気をつけながら数分間煮る。布で漉す。

\atoaki{}

\hypertarget{sauce-grand-veneur-procede-escoffier}{%
\subsubsection{ソース・グランヴヌール(エスコフィエ流)}\label{sauce-grand-veneur-procede-escoffier}}

\frsub{Sauce Grand-Veneur (Procédé Escoffier)}

\index{そーす@ソース!くらんうぬーるえすこふいえ@---・グランヴヌール(エスコフィエ流)}
\index{くらんうぬーる@グランヴヌール!そーすえすこふいえ@ソース・---(エスコフィエ流)}
\index{sauce@sauce!grand-veneur escoffier@--- Grand-Veneur (Procédé Escoffier)}
\index{grand-veneur@grand-veneur!sauce escoffier@Sauce --- (Procédé Escoffier)}
\index[src]{grand-veneur escoffier@Grand-Veneur Escoffier}
\index[src]{くらんうぬーるえすこふぃえ@グランヴヌール(エスコフィエ)}

軽く仕上げた\protect\hyperlink{sauce-poivrade}{ソース・ポワヴラード} 1
Lあたり\protect\hyperlink{gelee-de-groseilles-a}{グロゼイユのジュレ}大さじ2杯と生クリーム2
\(\frac{1}{2}\) dLを加える。

\ldots{}\ldots{}上記2つのソースは鹿、猪などの大きな塊肉の料理に用いる。

\atoaki{}

\hypertarget{sauce-gratin}{%
\subsubsection[ソース・グラタン]{\texorpdfstring{ソース・グラタン\footnote{魚のグラタン用ソースだが、グラタンの技術的ポイントについては\protect\hyperlink{gratins}{第
  7章「肉料理」のグタランの項目}参照。}}{ソース・グラタン}}\label{sauce-gratin}}

\frsub{Sauce Gratin}

\index{そーす@ソース!くらたん@---・グラタン}
\index{くらたん@グラタン!そーす@ソース・---}
\index{sauce@sauce!gratin@--- Gratin}
\index{gratin@gratin!sauce@Sauce ---} \index[src]{gratin@Gratin}
\index[src]{くらたん@グラタン}

白ワインと、このソースを合わせる魚のアラなどでとった\protect\hyperlink{fumet-de-poisson}{魚のフュメ}各3
dLにエシャロットのみじん切り大さじ1 \(\frac{1}{2}\)
杯を加え、半量以下になるまで煮詰める。

\protect\hyperlink{duxelles-seche}{デュクセル・セッシュ}大さじ3杯と、\protect\hyperlink{sauce-espagnole-maigre}{魚料理用ソース・エスパニョル}または\protect\hyperlink{sauce-demi-glace}{ソース・ドゥミグラス}
5 dLを加える。5〜6分間煮立たせる。提供直前に、パセリのみじん切り大さじ
\(\frac{1}{2}\) を加えて仕上げる。

\ldots{}\ldots{}舌びらめ、メルラン\footnote{タラの近縁種。}、バルビュ\footnote{鰈の近縁種。この場合のフィレはいわゆる「五枚おろし」にしたもの。}のフィレなどのグラタン用。

\atoaki{}

\hypertarget{sauce-hachee}{%
\subsubsection[ソース・アシェ]{\texorpdfstring{ソース・アシェ\footnote{細かく刻んだもの、の意。}}{ソース・アシェ}}\label{sauce-hachee}}

\frsub{Sauce Hachée}

\index{そーす@ソース!あしえ@---・アシェ}
\index{あしえ@アシェ!そーす@ソース・---}
\index{sauce@sauce!hachee@--- Hachée}
\index{hache@haché(e)!sauce@Sauce Hachée} \index[src]{hachee@Hachée}
\index[src]{あしえ@アシェ}

玉ねぎの細かいみじん切り100 gと、エシャロットの細かいみじん切り大さじ1
\(\frac{1}{2}\) 杯をバターで色付かないよう炒める。ヴィネガー3
dLを注ぎ、半量まで煮詰める。\protect\hyperlink{sauce-espagnole}{ソース・エスパニョル}
4 dLと\protect\hyperlink{sauce-tomate}{トマトソース} 1 \(\frac{1}{2}\)
dLを加える。5〜6分煮立たせる。

ハムの脂身のない部分を細かく刻んだもの大さじ1 \(\frac{1}{2}\)
杯と小ぶりのケイパー大さじ1 \(\frac{1}{2}\)
杯、{[}デュクセル・セッシュ{]}大さじ1 \(\frac{1}{2}\)
杯、パセリのみじん切り大さじ \(\frac{1}{2}\) 杯を加えて仕上げる

\ldots{}\ldots{}このソースは\protect\hyperlink{sauce-piquante}{ソース・ピカント}と等価のものと考えていい。用途も同じ。

\atoaki{}

\hypertarget{sauce-hachee-maigre}{%
\subsubsection{魚料理用ソース・アシェ}\label{sauce-hachee-maigre}}

\frsub{Sauce Hachée maigre}

\index{そーす@ソース!あしえ@魚料理用---・アシェ}
\index{あしえさかな@アシェ(魚料理用)!そーす@ソース・---}
\index{sauce@sauce!hachee maigre@--- Hachée maigre}
\index{hache@haché(e)!sauce maigre@Sauce Hachée maigre}
\index[src]{hachee maigrel@Hachée maigre}
\index[src]{あしえさかな@アシェ(魚料理用)}

上記と同様に、玉ねぎとエシャロットを色付かないようバターで炒め、ヴィネガーを注いで煮詰める。

魚の\protect\hyperlink{courts-bouillons-de-poisson}{クールブイヨン} 5
dLを注ぎ、\protect\hyperlink{roux-brun}{茶色いルー} 45
gまたはブールマニエ50 gでとろみを付ける。弱火で8〜 10分間煮込む。

提供直前に、細かく刻んだハーブミックス大さじ1杯と\protect\hyperlink{duxelles-seche}{デュクセル・セッシュ}大さじ1
\(\frac{1}{2}\) 杯、小粒のケイパー大さじ1 \(\frac{1}{2}\)
杯、アンチョヴィソース大さじ \(\frac{1}{2}\) 杯とバター60
g、または80〜100 gのアンチョヴィバターを加えて仕上げる。

\ldots{}\ldots{}エイのような、あまり高級ではない茹でた魚\footnote{原文
  poissons
  bouillis。このフランス語の表現だと加熱する際に沸騰させているニュアンスがあるが、本書の「魚料理」の章において、魚を塩を加えて茹でる、あるいはクールブイヨンで煮る際に、沸騰しない程度の温度で加熱(ポシェ
  pocher)すべきと強調されている。この表現は初版からのものであり、恐らくはこのソースの部分を実際に執筆した者と、魚料理の説明部分を執筆した者が異なることによるわかりにくさ、という可能性も排除出来ない。いずれにしても、このソースの場合は、合わせる魚をクールブイヨンで沸騰しない程度の温度で加熱(ポシェ)し、そのクールブイヨンの一部をソースに加えていることから、単に「茹でた魚」と言っても、本書における魚の加熱方法に則った調理をすべきと解されよう。}用。

\atoaki{}

\hypertarget{sauce-hussarde}{%
\subsubsection[ソース・ユサルド]{\texorpdfstring{ソース・ユサルド\footnote{もとはハンガリーで農家20戸につき1人の割合で招集された騎兵
  hussard
  を指す。この語は16世紀まで遡ることが出来るが、のちに「乱暴者」といったニュアンスでも使われるようになった。à
  la hussarde
  は「乱暴に、粗野に」の意味でも用いられるが、料理においてはレフォールを使ったものに名付けられることが多い。}}{ソース・ユサルド}}\label{sauce-hussarde}}

\frsub{Sauce Hussarde}

\index{そーす@ソース!ゆさると@---・ユサルド}
\index{ゆさると@ユサルド!そーす@ソース・---}
\index{sauce@sauce!hussarde@--- Hussarde}
\index{hussard@Hussard(e)!sauce@Sauce Hussarde}
\index[src]{hussarde@Hussarde} \index[src]{ゆさると@ユサルド}

玉ねぎ2個とエシャロット2個を細かくみじん切りにして、バターで色よく炒める。白ワイン4
dLを注ぎ、半量になるまで煮詰める。\protect\hyperlink{sauce-demi-glace}{ソース・ドゥミグラス}
4 dLとトマトピュレ大さじ2杯、\protect\hyperlink{fonds-blanc}{白いフォン}
2 dL、生ハムの脂身のないところ80
g、潰したにんにく1片、ブーケガルニを加える。弱火で25〜30分煮込む。

ハムを取り出して、ソースをスプーンで押すようにして布で漉す。

火にかけて温め、小さなさいの目\footnote{brunoise ブリュノワーズ。}に刻んだハムと、おろしたレフォール
\footnote{raifort (レフォール)いわゆる西洋わさび、ホースラディッシュ。}少々、パセリのみじん切りをたっぷり1つまみ加えて仕上げる。

\ldots{}\ldots{}牛、羊肉のグリルまたは串を刺してローストしてアントレ\footnote{通常、ローストは料理区分としてアントレに含められることはないが、牛フィレは牛の部位のなかでも比較的小さいものとして、まるごと1本のローストであっても原則的にはアントレに分類される。このソースを用いる「牛フィレ ユサルド」は牛フィレの塊に串を刺してローストし、ポム・デュシェスとマッシュルームを合わせる。}として供する際に用いる。

\atoaki{}

\hypertarget{sauce-italienne}{%
\subsubsection[イタリア風ソース]{\texorpdfstring{イタリア風ソース\footnote{この「イタリア風」には根拠も由来も見出すことが出来ない。地名、国名を料理名に冠した代表例のひとつ。}}{イタリア風ソース}}\label{sauce-italienne}}

\frsub{Sauce Italienne}

\index{そーす@ソース!いたりあふう@イタリア風---}
\index{いたりあふう@イタリア風!そーす@---ソース}
\index{sauce@sauce!Italienne@--- Italienne}
\index{italien@italien(ne)!sauce@Sauce Italienne}
\index[src]{italienne@Italienne} \index[src]{いたりあふう@イタリア風}

トマトの風味の効いた\protect\hyperlink{sauce-demi-glace}{ソース・ドゥミグラス}
\(\frac{3}{4}\)
Lに、\protect\hyperlink{duxelles-seche}{デュクセル・セッシュ}大さじ4杯と、加熱ハムの脂身のないところを小さなさいの目に切ったもの125
gを加える。5〜6分間煮る。提供直前に、パセリとセルフイユ、エスゴラゴンのみじん切り大さじ1杯を加えて仕上げる。

\ldots{}\ldots{}いろいろな肉料理に合わせる。

\hypertarget{nota-sauce-italienne}{%
\subparagraph{【原注】}\label{nota-sauce-italienne}}

このソースを魚料理に合わせる場合、ハムは使わずに\protect\hyperlink{fumet-de-poisson}{魚のフュメ}を煮詰めて加える。

\atoaki{}

\hypertarget{jus-lie-estragon}{%
\subsubsection{とろみを付けたジュ・エストラゴン風味}\label{jus-lie-estragon}}

\frsub{Jus lié à l'Estragon}

\index{そーす@ソース!とろみをつけたしゆえすとらこん@とろみを付けたジュエストラゴン風味}
\index{しゆ@ジュ!とろみをつけたえすとらごん@とろみを付けた---・エストラゴン風味}
\index{えすとらこん@エスゴラゴン!とろみをつけたしゆ@とろみを付けたジュ・---風味}
\index{sauce@sauce!jus lie estragon@Jus lié à l'Estragon}
\index{estragon@estragon!jus lie estragon@Jus lié à l'Estragon}
\index{jus@jus!lie estragon@--- lié à l'Estragon}
\index[src]{jus lie a l'estragon@Jus lié à l'Estragon}
\index[src]{とろみをつけたしゆえすとらこん@とろみを付けたジュ・エストラゴン風味}

\protect\hyperlink{jus-de-veau-brun}{仔牛のフォン}または\protect\hyperlink{fonds-de-volaille}{鶏のフォン}に、エストラゴン50
gを加えて香りを煮出し\footnote{imfuser アンフュゼ。}たもの。

布で漉してから、アロールート\footnote{コーンスターチで代用する。}または、でんぷん30
gでとろみを付ける。

\ldots{}\ldots{}白身肉のノワゼットや家禽のフィレなどに添える。

\atoaki{}

\hypertarget{jus-lie-tomate}{%
\subsubsection{とろみを付けたジュ・トマト風味}\label{jus-lie-tomate}}

\frsub{Jus lié tomaté}

\index{そーす@ソース!とろみをつけたしゆとまと@とろみを付けたジュ・トマト風味}
\index{しゆ@ジュ!とろみをつけたとまとふうみ@とろみを付けた---・トマト風味}
\index{とまと@トマト!とろみをつけたしゆ@とろみを付けたジュ・---風味}
\index{sauce@sauce!jus lie tomatee@Jus lié tomaté}
\index{tomate@tomate!jus lie tomate@Jus lié tomaté}
\index{jus@jus!lie tomate@--- lié tomaté}
\index[src]{jus lie tomate@Jus lié tomaté}
\index[src]{とろみをつけたしゆとまと@とろみを付けたジュ・トマト風味}

\protect\hyperlink{jus-de-veau-brun}{仔牛のフォン} 1
Lあたり\protect\hyperlink{essence-de-tomate}{トマトエッセンス} 3
dLを加え、 \(\frac{4}{5}\)量まで煮詰める。

\ldots{}\ldots{}牛、羊肉料理用。

\atoaki{}

\hypertarget{sauce-lyonnaise}{%
\subsubsection{リヨン風ソース}\label{sauce-lyonnaise}}

\frsub{Sauce Lyonnaise}

\index{そーす@ソース!りよんふう@リヨン風---}
\index{りよんふう@リヨン風!そーす@---ソース}
\index{sauce@sauce!lyonnaise@--- Lyonnaise}
\index{liyonnais@lyonnais(e)!sauce lyonnaise@Sauce ---e}
\index[src]{lyonnaise@Lyonnaise} \index[src]{りよんふう@リヨン風}

中位の大きさの玉ねぎ3個をみじん切りにし、バターでじっくり、ごく弱火でブロンド色になるまで炒める。白ワイン2
dLとヴィネガー2 dLを注ぐ。
\(\frac{1}{3}\)量まで煮詰め、\protect\hyperlink{sauce-demi-glace}{ソース・ドゥミグラス}
\(\frac{3}{4}\)
Lを加える。5〜6分かけて表面に浮いてくる不純物を丁寧に取り除き\footnote{dépouiller
  デプイエ。現代ではエキュメと呼ぶ現場が多い。}、布で漉す。

\hypertarget{nota-sauce-lyonnaise}{%
\subparagraph{【原注】}\label{nota-sauce-lyonnaise}}

このソースを合わせる料理によっては、ソースを布で漉さずに玉ねぎを残してもいい。

\atoaki{}

\hypertarget{sauce-madere}{%
\subsubsection{ソース・マデール}\label{sauce-madere}}

\frsub{Sauce Madère}

\index{そーす@ソース!まてーる@---・マデール}
\index{まてらしゆ@マデイラ酒 ⇒ マデール!そーす@ソース・マデール}
\index{sauce@sauce!madere@--- Madère}
\index{madere@madère!sauce@Sauce ---} \index[src]{madere@Madère}
\index[src]{まてーる@マデール}

\protect\hyperlink{sauce-demi-glace}{ソース・ドゥミグラス}を煮詰め\footnote{ソース・ドゥミグラスは既に煮詰めて仕上がった状態のものなので、9
  割程度にまでしか煮詰めないことに注意。}、火から外して、ソース1
Lあたりマデイラ酒1 dLの割合で加え、普通の濃度にする。

\atoaki{}

\hypertarget{sauce-matelote}{%
\subsubsection[ソース・マトロット]{\texorpdfstring{ソース・マトロット\footnote{水夫風、船員風、の意。トゥーレーヌ地方の郷土料理Matelote
  d'anguille(マトロットダンギーユ)うなぎの赤ワイン煮込み、が有名。とはいえ本書にも数種のレシピが収録されているように、赤ワイン煮込みにとどまらず、マトロットの名称を持つ料理は他にも複数存在する。}}{ソース・マトロット}}\label{sauce-matelote}}

\frsub{Sauce Matelote}

\index{そーす@ソース!まとろつと@---・マトロット}
\index{まとろつと@マトロット!そーす@ソース・---}
\index{sauce@sauce!matelote@--- Matelote}
\index{matelote@matelote!sauce@Sauce ---} \index[src]{matelote@Matelote}
\index[src]{まとろつと@マトロット}

魚をポシェするのに使った\protect\hyperlink{court-bouillon-c}{赤ワイン入りの魚用クールブイヨン}
3 dLにマッシュルームの切りくず25
gを加え、\(\frac{1}{3}\)量になるまで煮詰める。

煮詰めたら\protect\hyperlink{sauce-espagnole-maigre}{魚料理用ソース・エスパニョル}
8 dL を加えてひと煮立ちさせる。布で漉し、バター150
gとごく少量のカイエンヌの粉末を加えて仕上げる。

\atoaki{}

\hypertarget{sauce-moelle}{%
\subsubsection[ソース・モワル]{\texorpdfstring{ソース・モワル\footnote{moelle
  骨髄のこと。}}{ソース・モワル}}\label{sauce-moelle}}

\frsub{Sauce Moelle}

\index{そーす@ソース!もわる@---・モワル}
\index{こつずい@骨髄 ⇒ モワル!そーすもわる@ソース・モワル}
\index{sauce@sauce!moelle@--- Moelle}
\index{moelle@moelle!sauce@Sauce ---} \index[src]{moelle@Moelle}
\index[src]{もわる@モワル}

ソースの作り方は\protect\hyperlink{sauce-bordelaise}{ボルドー風ソース}とまったく同じだが、バターを加えるのは何らかの野菜料理に添える場合のみであり、その場合のバターの量は通常どおりとするこ。

どんな場合にせよ、仕上げに、小さなさいの目に切ってポシェしておいた骨髄をソース1
Lあたり150〜180 gおよび刻んで下茹でしたパセリの葉小さじ1杯を加える。

\atoaki{}

\hypertarget{sauce-moscovite}{%
\subsubsection[モスクワ風ソース]{\texorpdfstring{モスクワ風ソース\footnote{moscovite(モスコヴィット)すなわちモスクワ風の名称を持つ料理や菓子は多い。
  18世紀後半から19世紀前半にかけて、ロシアの宮廷や貴族らの間でフランスの食文化が流行し、多くのフランス人料理人が招聘され、彼らはロシア料理のレシピをフランスに持ち帰った。クーリビヤックなどが代表的な例だろう。また、19世紀後半になると、とりわけフランス料理においてもロシア料理からの影響が多く見られるようになる。キャビアとウォトカを食前に愉しむのが流行したのもその時代からである。フランスとロシアの食文化は相互に影響関係にあったと言えよう。}}{モスクワ風ソース}}\label{sauce-moscovite}}

\frsub{Sauce Moscovite}

\index{そーす@ソース!もすくわふう@モスクワ風---}
\index{もすくわふう@モスクワ風!そーす@---ソース}
\index{sauce@sauce!moscovite@--- Moscovite}
\index{moscovite@moscovite!sauce@Sauce ---}
\index[src]{moscovite@Moscovite} \index[src]{もすくわふう@モスクワ風}

\protect\hyperlink{fonds-de-gibier}{大型ジビエのフュメ}で作った\protect\hyperlink{sauce-poivrade}{ソース・ポワヴラード}を\(\frac{3}{4}\)
L用意する。提供直前にマラガ酒1 dL とジェニパーベリーを煎じた汁7
cL\footnote{1 cL(センチリットル) = 10 mL、つまりこの場合は70 mL。}、焼いた松の実かスライスして焼いたアーモンド40
g、大きさを揃えてぬるま湯でもどしておいたコリント産干しぶどう \footnote{小粒で黒いギリシア産干しぶどう。}40
gを加えて仕上げる。

\ldots{}\ldots{}大型ジビエ\footnote{venaison
  ヴネゾン。ジビエのうち大型のものを指す。実際はノロ鹿や猪を指すことがほとんど。}の塊肉の料理用。

\atoaki{}

\hypertarget{sauce-perigueux}{%
\subsubsection[ソース・ペリグー]{\texorpdfstring{ソース・ペリグー\footnote{トリュフの産地として有名なペリゴール地方の町の名。}}{ソース・ペリグー}}\label{sauce-perigueux}}

\frsub{Sauce Périgueux}

\index{そーす@ソース!へりくー@---・ペリグー}
\index{へりくー@ペリグー!そーす@ソース・---}
\index{sauce@sauce!perigueux@--- Périgueux}
\index{perigueux@Périgueux!sauce@Sauce ---}
\index[src]{perigueux@Périgueux} \index[src]{へりくー@ペリグー}

やや濃いめに煮詰めた\protect\hyperlink{sauce-demi-glace}{ソース・ドゥミグラス}
\(\frac{3}{4}\) Lに、トリュフエッセンス1 \(\frac{1}{2}\)
dLと細かく刻んだトリュフ100 gを加える。

\ldots{}\ldots{}いろいろな肉料理、\protect\hyperlink{}{タンバル}、\protect\hyperlink{}{温製パテ}に合わせる。

\atoaki{}

\hypertarget{sauce-perigourdine}{%
\subsubsection[ソース・ペリグルディーヌ]{\texorpdfstring{ソース・ペリグルディーヌ\footnote{périgourdin(e)
  (ペリグルダン/ペリグルディーヌ)ペリゴール地方風の意。}}{ソース・ペリグルディーヌ}}\label{sauce-perigourdine}}

\frsub{Sauce Périgourdine}

\index{そーす@ソース!へりくるていーぬ@---・ペリグルディーヌ}
\index{へりこーるふう@ペリゴール風 ⇒ ペリグルダン/ペリグルディーヌ!そーす@ソース・ペリグルディーヌ}
\index{sauce@sauce!perigourdine@--- Périgourdine}
\index{perigourdin@périgourdin(e)!sauce@Sauce Périgourdine}
\index[src]{perigourdine@Périgourdine}
\index[src]{へりくるていーぬ@ペリグルディーヌ}

ソース・ペリグーのバリエーション。トリュフを細かく刻むのではなく、オリーブ形か小さな真珠のような形状にナイフで成形\footnote{tourner
  トゥルネ。包丁を持っている側の手は動かさずに材料を回すようにして形を整えること。}したものを加える。トリュフを厚めにスライスして加える場合もある。

\atoaki{}

\hypertarget{sauce-piquante}{%
\subsubsection[ソース・ピカント]{\texorpdfstring{ソース・ピカント\footnote{piquant(e)
  (ピカン、ピカント)
  一般的には唐辛子などが「辛い」の意だが、このソースでは唐辛子の類は使われておらず、むしろ酸味の効いたソースと言えよう。古くからのソース名。}}{ソース・ピカント}}\label{sauce-piquante}}

\frsub{Sauce Piquante}

\index{そーす@ソース!ひかんと@---・ピカント}
\index{ひかんと@ピカント!そーす@ソース・---}
\index{sauce@sauce!piquante@--- Piquante}
\index{piquant@piquant(e)!sauce@Sauce Piquante}
\index[src]{piquante@Piquante} \index[src]{ひかんと@ピカント}

白ワイン3 dLと良質のヴィネガー3 dLにエシャロットのみじん切り大さじ2
\(\frac{1}{2}\) 杯を合わせて半量に煮詰める。

\protect\hyperlink{sauce-espagnole}{ソース・エスパニョル} 6
dLを加え、浮いてくる不純物を取り除きながら10分間煮る。

火から外し、コルニション\footnote{専用品種のきゅうりを小さなうちに収穫して酢漬けにしたもの。同様のピクルス用きゅうりとしてガーキンスという品種系統があるがもっぱらアメリカのハンバーガーなどに用いられ、フランス料理では用いない。}、パセリ、セルフイユ、エストラゴンを細かく刻んだもの大さじ2杯を加えて仕上げる。

\ldots{}\ldots{}豚肉のグリル焼き、ブイイ\footnote{bouilli
  茹で肉。もとはブイヨンをとった後の茹で肉のことを指した。単純に「茹でた肉」としてもいいが、17世紀にはこの食べ方が流行した。このため、野菜などと共に、あるいは他の素材なしに茹でた肉をこう呼ぶ。}、ローストによく合わせるソース。牛肉のブイイや牛や羊の\protect\hyperlink{}{エマンセ}にも合わせることが出来る。

\atoaki{}

\hypertarget{sauce-poivrade}{%
\subsubsection[ソース・ポワヴラード
(標準)]{\texorpdfstring{ソース・ポワヴラード\footnote{このソースは遅くとも16世紀まで遡ることが出来る。1505年に出版された\href{http://gallica.bnf.fr/ark:/12148/bpt6k792720}{『フランス語版プラティナ』}がpoivradeというフランス語の初出。この本において「ジビエ用こしょうのソース、ポワヴラード」Saulce
  de poyvre ou poyvrade pour
  saulvagieとしてレシピが見られる。パンをよく焼いてヴィネガーに浸してすり潰す。水でもどした干しぶどうと獣の血を加えて混ぜ、玉ねぎと未熟ぶどう果汁、パンを浸した残りのヴィネガーを加えて漉し器か布で漉す。これを鍋に入れ、こしょう、生姜、シナモンを入れて炭火の上で30分程煮込む。獣の肉を獣脂を熱したフライパンで焼き、皿に盛る。上からポワヴラードをかけて供する、という内容(f.LXII)。またこの本には、魚料理用のポワヴラードも掲載されている。ただし、これが現代まで続くソース・ポワヴラードの原型と捉えるのは早計に過ぎる。ここで注目すべきは、最終的に肉あるいは魚のような主素材とソースが一体化したものは中世〜ルネサンス期にはポタージュと呼ばれていたのに対し、ここではソースを別のものと捉えている点である。ポワヴラードという語そのものは「こしょうを効かせたもの」という意味に過ぎず、1660年刊ピエール・ド・リュヌPierre
  de Lune『新フランス料理』におけるPoivrade de pigeonneaux
  若鳩のポワヴラードは、背開きにした若鳩を平たくのばし、塩、こしょう
  をして弱火でグリルする。薔薇の香りもしくはにんにく風味のヴィネガーを添えて供する、というもの(p.190)。ピエール・ド・リュヌのレシピにおいてソースに相当するものはヴィネガーであり、むしろ味付けでこしょうを効かせているということが料理名の根拠となっているに過ぎない。ちなみに、生食可能な小さなサイズのアーティチョークも古くからポワヴラードと呼ばれている。}
(標準)}{ソース・ポワヴラード (標準)}}\label{sauce-poivrade}}

\frsub{Sauce Poivrade ordinaire}

\index{そーす@ソース!ほわうらーと@---・ポワヴラード(標準)}
\index{ほわうらーと@ポワヴラード!そーす@ソース・---(標準)}
\index{sauce@sauce!poivrade ordinaire@--- Poivrade ordinaire}
\index{poivrade@poivrade!sauce ordinaire@Sauce --- ordinaire}
\index[src]{poivrade ordinaire@Poivrade ordinaire}
\index[src]{ほわふらーとひようしゆん@ポワヴラード(標準)}

細かいさいの目に切ったにんじん100 gと玉ねぎ80
g、刻んだパセリの茎、タイム少々、ローリエの葉少々からなる\protect\hyperlink{mirepoix}{ミルポワ}を油で色付くまで炒める。

ヴィネガー1 dLとマリナード2
dLを注ぎ、\(\frac{1}{3}\)量になるまで煮詰める。\protect\hyperlink{sauce-espagnole}{ソース・エスパニョル}
1 Lを注ぎ、約45分間煮込む。

ソースを漉す10分前に、大粒のこしょう8個を叩きつぶして加える。ソースにこしょうを入れてからの時間がこれ以上少しでも長いと、こしょうの風味が支配的になり過ぎることになるので注意。

漉し器で香味素材を軽く押すようにして漉す。\protect\hyperlink{marinades-et-saumures}{マリナード}\footnote{ヴィネガーやワイン、香味素材、塩などを合わせて肉を漬け込む液体。マリネ液と呼ぶこともある。}
2 dLでソースをのばす。火にかけて35分間、所定の量\footnote{明記されていないが、ここでは約1
  L。}になるまで煮詰めながら、表面に浮いてくる不純物を徹底的に取り除く。

さらに布で漉し、バター50 gを加えて仕上げる\footnote{現代では、バターでモンテするmonter
  au beurreという表現を用いる現場も多い。}。

\atoaki{}

\hypertarget{sauce-poivrade-pour-gibier}{%
\subsubsection{ソース・ポワヴラード(ジビエ用)}\label{sauce-poivrade-pour-gibier}}

\frsub{Sauce Poivrade pour Gibier}

\index{そーす@ソース!ほわうらーとしひえ@---・ポワヴラード(ジビエ用)}
\index{ほわうらーと@ポワヴラード!そーすしひえ@ソース・---(ジビエ用)}
\index{sauce@sauce!poivrade gibier@--- Poivrade pour Gibier}
\index{poivrade@poivrade!sauce  gibier@Sauce --- pour Gibier}
\index[src]{poivrade gibier@Poivrade pour Gibier}
\index[src]{ほわふらーとしひえ@ポワヴラード(ジビエ用)}

細かいさいの目に切ったにんじん125 gと玉ねぎ125
g、タイムの枝と鳥類ではないジビエ\footnote{gibier à poil
  逐語訳すると「毛の生えているジビエ」すなわち」鹿、猪、野うさぎなどを指す。}の端肉1
kgからなる\protect\hyperlink{mirepoix}{ミルポワ}を油で色よく炒める。

ミルポワが色付いてきたら、鍋の油を捨てる。ヴィネガー3 dLと白ワイン2 dL
を注ぎ、完全に煮詰める。

\protect\hyperlink{sauce-espagnole}{ソース・エスパニョル} 1
Lと\protect\hyperlink{fonds-de-gibier}{ジビエの茶色いフォン} 2
L、\protect\hyperlink{marinades-et-saumures}{マリナード} 1 Lを加える。

鍋に蓋をして弱火にかける。可能ならオーブンがいい。3時間半〜4時間加熱する。

ソースを漉す8分前に、大粒のこしょう12個を叩きつぶして加える。

漉し器で材料を押すようにして漉す。

これをジビエのフォン\(\frac{1}{4}\) Lとマリナード\(\frac{1}{4}\)
Lでのばし、再び火にかけて40分間、表面に浮いてくる不純物を丁寧に取り除きながら、1
Lになるまで煮詰める。

これを布で漉し、バター75 gを加えて仕上げる。

\hypertarget{nota-sauce-poivrade-pour-gibier}{%
\subparagraph{【原注】}\label{nota-sauce-poivrade-pour-gibier}}

一般的にはジビエ料理のソースにはバターを加えないことになっているが、本書では軽くバターを加えることを推奨する。そうすると、ソースの色の赤みは薄まるが、繊細で滑らかな口あたりに仕上がる。

\atoaki{}

\hypertarget{sauce-au-porto}{%
\subsubsection{ソース・ポルト}\label{sauce-au-porto}}

\frsub{Sauce au Porto}

\index{そーす@ソース!ほると@---・ポルト}
\index{ほるとしゆ@ポルト酒 ⇒ ポルト!そーす@ソース・---}
\index{sauce@sauce!porto@--- au Porto}
\index{porto@Porto!sauce@Sauce au ---} \index[src]{porto@Porto}
\index[src]{ほると@ポルト}

マデイラ酒ではなくポルト酒を用いて、\protect\hyperlink{sauce-madere}{ソース・マデール}と同様に作る。

\atoaki{}

\hypertarget{sauce-portugaise}{%
\subsubsection[ポルトガル風ソース]{\texorpdfstring{ポルトガル風\footnote{日本でもフランス語のままソース・ポルチュゲーズと呼ばれることは多い。フランス料理においてポルトガル風の名称を付けた料理はトマトをベースとしたものがほとんど。ただし、トマトを使うからといってポルトガル風の名が必ず付くというわけではない。}ソース}{ポルトガル風ソース}}\label{sauce-portugaise}}

\frsub{Sauce Portugaise}

\index{そーす@ソース!ほるとかるふう@ポルトガル風---}
\index{ほるとかるふう@ポルトガル風!そーす@---ソース}
\index{sauce@sauce!porugaise@--- Portugaise}
\index{portugais@portugais(e)!sauce@Sauce Portugaise}
\index[src]{portugaise@Portugaise}
\index[src]{ほるとかるふう@ポルトガル風}

(仕上がり1 L分)

大きめの玉ねぎ1個を細かくみじん切りにする。鍋に油を熱し、強火で玉ねぎを炒める。玉ねぎがブロンド色になったら、皮を剥いて種子を取り除き、粗みじん切りにしたトマト750
gと、つぶしたにんにく1片、塩、こしょうを加える。トマトの酸味が強い場合は砂糖少々も加える。鍋に蓋をして、弱火で煮る。\protect\hyperlink{essence-de-tomate}{トマトエッセンス}少々と、薄めに作ったトマトソースを適量\footnote{仕上がりの全体量が1
  Lなので、トマトソースを加える量は、グラスドヴィアンドを加える前の段階で0.9
  L程度になるよう調整する。}、温めて溶かした\protect\hyperlink{glace-de-viande}{グラスドヴィアンド}
1 dL、新鮮なパセリの葉のみじん切り大さじ1杯を加えて仕上げる。

\atoaki{}

\hypertarget{sauce-provencale}{%
\subsubsection{プロヴァンス風ソース}\label{sauce-provencale}}

\frsub{Sauce Provençale}

\index{そーす@ソース!ふろうあんすふう@プロヴァンス風---}
\index{ふろうあんすふう@プロヴァンス風!そーす@---ソース}
\index{sauce@sauce!provencale@--- Provençale}
\index{provencal@provençal(e)!sauce@Sauce Provençale}
\index[src]{provencale@Provençale}
\index[src]{ふろうあんすふう@プロヴァンス風}

大ぶりのトマト12個の皮を剥き、つぶして種子は取り除いて、粗く刻む\footnote{concasser
  コンカセ。}。ソテー鍋に2 \(\frac{1}{2}\)
dLの油を熱し、そこにトマトを入れる。塩、こしょう、粉砂糖1つまみで味を調える。しっかりつぶしたにんにく(小)1片と細かく刻んだパセリ小さじ1杯を加える。

蓋をして弱火で30分間程、煮溶かす。

\hypertarget{nota-sauce-provencale}{%
\subparagraph{【原注】}\label{nota-sauce-provencale}}

このソースについてはさまざまな解釈があるが、本書ではブルジョワ料理における本物の「プロヴァンス風ソース」のレシピ、つまりはトマトの「フォンデュ」\footnote{加熱によって溶かしたもの、の意。このレシピはあくまでも「ソース」であり、料理を作る際のアパレイユ≒パーツとしてのいわゆる\protect\hyperlink{portugaise}{トマトフォンデュ}については第2章ガルニチュール、温製ガルニチュール用のアパレイユなど、の項を参照。}、を収録した。

\atoaki{}

\hypertarget{sauce-regence}{%
\subsubsection[ソース・レジャンス]{\texorpdfstring{ソース・レジャンス\footnote{Régence(レジョンス)はこの場合固有名詞としての「摂政時代」を指す。すなわちオルレアン公フィリップがルイ15世の幼少期に摂政を務めていたた時代(1715〜1723年)のこと。オルレアン公は美食家として有名で、とりわけシャンパーニュを好んだという。この時期はフランス宮廷料理の絶頂期でもあった。}}{ソース・レジャンス}}\label{sauce-regence}}

\frsub{Sauce Régence}

\index{そーす@ソース!れしやんす@---・レジャンス}
\index{れしやんす@レジャンス!そーす@ソース・---}
\index{sauce@sauce!regence@--- Régence}
\index{regence@Régence!sauce@Sauce ---} \index[src]{regence@Régence}
\index[src]{れしやんす@レジャンス}

ライン産ワイン3
dLに、細かく刻んであらかじめ火を通しておいた\protect\hyperlink{mirepoix}{ミルポワ}
1 dLと生トリュフの切りくず25
gを加え、半量になるまで煮詰める。トリュフのシーズンでない時季はトリュフエッセンスを使う。\protect\hyperlink{sauce-demi-glace}{ソース・ドゥミグラス}
8 dLを加え、数分間弱火にかけて浮いてくる不純物を丁寧に取り除き\footnote{dépouiller
  デプイエ ≒ écumer エキュメ。}、布で漉す。

\ldots{}\ldots{}牛、羊の大きな塊肉の料理用。

\atoaki{}

\hypertarget{sauce-robert}{%
\subsubsection[ソース・ロベール]{\texorpdfstring{ソース・ロベール\footnote{この名称のソースは古くからある。文献で初めて出てくるのは16世紀フランソワ・ラブレーの小説『ガルガンチュアとパンタグリュエル』。その「第四の書」で料理人の名が大量に列挙される章がある。そのうちの多くは架空の人名だが、その中のロベールという料理人がこのソースを考案したと書いている。ただし、具体的にどのようなソースかまでは描写されておらず「うさぎのロースト、鴨、加工していない豚肉、卵のポシェ、塩漬けのメルラン{[}鱈の近縁種{]}、その他まことに多くの料理に欠かせないソース」と書いてあるのみ(第40章)。どんな料理にも合うと書かれてしまうとむしろ特徴を捉え難くなってしまう。いずれにせよ、遅くとも16世紀には「ソース」として成立していたと考えられる。また、17世紀のシャルル・ペロー著『物語集』の「眠れる森の美女」においても、このソース名が登場する一節がある。このように16世紀以降多くの文学作品をはじめとする文献にこのソース名は見られる。レシピとしては、1651年刊ラ・ヴァレーヌ『フランス料理の本』における「豚腰肉 ソース・ロベール添え」がもっとも古いもののひとつだろう。概略は、豚腰肉を、ヴェルジュ{[}未熟ぶどう果汁、中世料理においてよく用いられた{]}とヴィネガー、セージを振り掛けながらローストする。下に置いた脂受け皿に焼いた豚肉から流れ落ちた脂がたまるので、これを使って玉ねぎをこんがり炒める。炒めた玉ねぎの上に豚後ろ身を載せ、豚腰肉をローストする際にかけたのと同じソースをかける。このソースはソースロベールと呼ばれている(p.51)。また、干鱈のソース・ロベール添えの場合は、バターとヴェルジュ少々、マスタードで作るが、ケイパーやシブール{[}葱{]}を加えてもいい(p.202)とあり、同じ名称のソースとは見做しがたい。18世紀以降のソース・ロベールは多かれ少なかれいずれもマスタードを加える点が共通しているので、名称が先にあり、内容が時代とともにはっきりしたものになっていたのだろう。}}{ソース・ロベール}}\label{sauce-robert}}

\frsub{Sauce Robert}

\index{そーす@ソース!ろへーる@---・ロベール}
\index{ろへーる@ロベール!そーす@ソース・---}
\index{sauce@sauce!robert@--- Robert}
\index{robert@Robert!sauce@Sauce ---} \index[src]{robert@Robert}
\index[src]{ろへーる@ロベール}

(仕上がり5 dL分)

大きめの玉ねぎを細かくみじん切りにし、バターで色付かないよう強火でさっと炒める。

白ワイン2
dLを注ぎ、\(\frac{1}{3}\)量になるまで煮詰める。\protect\hyperlink{sauce-demi-glace}{ソース・ドゥミグラス}
3 dLを加え、弱火で10分間煮る。

シノワ\footnote{主として金属製で円錐形に取っ手の付いた漉し器。清朝の高級役人がかぶっていた帽子の形状から「中国の」を意味するchinoisの名称となったと言われている。}で漉し(これは任意。漉さなくてもいい)、火から外して、粉砂糖1つまみとマスタード大さじ1杯を加えて仕上げる。

\atoaki{}

\hypertarget{sauce-robert-escoffier}{%
\subsubsection{ソース・ロベール・エスコフィエ}\label{sauce-robert-escoffier}}

\frsub{Sauce Robert Escoffier}

\index{そーす@ソース!ろへーるえすこふいえ@---・ロベール・エスコフィエ}
\index{ろへーる@ロベール!そーすえすこふいえ@ソース・---・エスコフィエ}
\index{sauce@sauce!robert escoffier@--- Robert Escoffier}
\index{robert@Robert!sauce escoffier@Sauce --- Escoffier}
\index[src]{robert escoffier@Robert Escoffier}
\index[src]{ろへーるえすこふいえ@ロベール・エスコフィエ}

このソースは完成品が市販されている\footnote{\protect\hyperlink{sauce-diable-escoffier}{ソース・ディアーブル・エスコフィエ}訳注参照。}。

温かい料理にも冷たい料理にもよく合う。温かい料理に合わせる場合は、同量の\protect\hyperlink{jus-de-veau-brun}{仔牛の茶色いフォン}と混ぜること。

\ldots{}\ldots{}豚、仔牛、鶏、魚のグリル焼きによく合う。

\atoaki{}

\hypertarget{sauce-romaine}{%
\subsubsection[ローマ風ソース]{\texorpdfstring{ローマ風\footnote{フランス料理における「ローマ風」の名称は「イタリア風」と同様にとくに根拠や由来が見出せないものが多い。このソースの場合は松の実を使うところから、20世紀前半に活躍したイタリアの作曲家レスピーギのローマ三部作のうちの「ローマの松」を想起させるが、残念ながらこの曲が作曲されたのは1924年、つまり本書より後なので関係はない。だが、松の実を採るイタリアカサマツは、アッピア街道の並木などで有名なように、イタリアとりわけローマ近辺において多く見られる(だからこそレスピーギが曲の題材にしたわけだが)。その意味においては、松の実を使っているということがこのソース名の根拠と見ることも不可能ではないだろう。しかしながら、それを証明する文献、史料があるかは不明。}ソース}{ローマ風ソース}}\label{sauce-romaine}}

\frsub{Sauce Romaine}

\index{そーす@ソース!ろーまふう@ローマ風---}
\index{ろーまふう@ローマ風!そーす@---ソース}
\index{sauce@sauce!romain@--- Romaine}
\index{romain@romain(e)!Sauce Romaine} \index[src]{romaine@Romaine}
\index[src]{ろーまふう@ローマ風}

砂糖50 gを火にかけてブロンド色にカラメリゼ\footnote{焦がさないように弱火で混ぜながら熱で砂糖を溶かしていく。}する。これをヴィネガー
1 \(\frac{1}{2}\)
dLでのばす。砂糖を完全に溶かし込めたら、\protect\hyperlink{sauce-espagnole}{ソース・エスパニョル}
6 dLと\protect\hyperlink{fonds-de-gibier}{ジビエのフォン} 3
dLを加える。これを\(\frac{3}{4}\)量弱まで煮詰める。布で漉し、松の実20
gをローストしたものと、大きさが揃るよう選別したスミヌル干しぶどう\footnote{トルコ産の白い干しぶどう。}20
gおよびコリント干しぶとう\footnote{ギリシア産の黒い小粒の干しぶどう(\protect\hyperlink{sauce-moscovite}{モスクワ風ソース}参照)。}20
gを温湯でもどしたものを加えて仕上げる。

\hypertarget{nota-sauce-romaine}{%
\subparagraph{【原注】}\label{nota-sauce-romaine}}

上記のとおり作る場合、このソースは大型ジビエ料理用だが、ジビエのフォンではなく通常の\protect\hyperlink{fonds-brun}{茶色いフォン}を使えば、マリネした牛、羊肉の料理に合わせることも可能。

\atoaki{}

\hypertarget{sauce-rouennaise}{%
\subsubsection[ルーアン風ソース]{\texorpdfstring{ルーアン風\footnote{ルーアンは野生のcolvertコルヴェール、いわゆる青首鴨を家禽化したルーアン鴨の産地として有名。}ソース}{ルーアン風ソース}}\label{sauce-rouennaise}}

\frsub{Sauce Rouennaise}

\index{そーす@ソース!るーあんふう@ルーアン風---}
\index{るーあんふう@ルーアン風!そーす@---ソース}
\index{sauce@sauce!rouennaise@--- Rouennaise}
\index{rouannais@rouennais(e)!sauce@Sauce Rouennaise}
\index[src]{rouennaise@Rouennaise} \index[src]{るーあんふう@ルーアン風}

(仕上がり5 dL分)

\protect\hyperlink{sauce-bordelaise}{ボルドー風ソース} 4 dL
を用意する。ただし、良質な赤ワインを使って作ること。(\protect\hyperlink{sauce-bordelaise}{ボルドー風ソース}参照)。

中位の大きさの鴨のレバー3個を裏漉しする。こうして出来たレバーのピュレをソースに加え、沸騰させない程度の温度で火を通す\footnote{pocher
  ポシェする。}。絶対に沸騰させないこと。沸騰させてしまうと途端にレバーのピュレが粒状になってしまう。

布で漉し、塩こしょうを効かせる。

このソースの特質\ldots{}\ldots{}エシャロットを加えた赤ワインを煮詰めたものに鴨の生レバーのピュレを加えたもの。

\ldots{}\ldots{}ルーアン産鴨のローストには、いわば必須といってもいいソース。

\atoaki{}

\hypertarget{sauce-salmis}{%
\subsubsection[ソース・サルミ]{\texorpdfstring{ソース・サルミ\footnote{語源は「ごった煮」を意味する
  salmigondis
  とするのが定説のようだが、salmigondisがその意味で用いられるようになったのは19世紀以降と考えられ、それ以前はragoûtラグーと同義と見なされていた。ラグーはその語源的意味が「食欲をそそるもの」であり、17世紀に、それまでポタージュと呼ばれていた煮込み料理についてラグーの名称をつけることが流行した。また、salmigondisの古い語形のひとつsalmigondinは16世紀の小説家フランソワ・ラブレー『ガルガンチュアとパンタグリュエル』の「第四の書」において用いられているが、日本語の「ごった煮」のニュアンスとはかなり違う意味で、美味な料理のひとつとして挙げられている。いずれにしても、salmigondin,
  salmigondisというラグーの別称が、ある時期から鳥類を材料にしたものに限定されるようになったことは確かで、カレームの『19世紀フランス料理』ではsalmisの語で、野鳥などのラグーを呼んでいる。例えば「ベカスのサルミ」「ペルドローのサルミ」など。}}{ソース・サルミ}}\label{sauce-salmis}}

\frsub{Sauce Salmis}

\index{そーす@ソース!さるみ@---・サルミ}
\index{さるみ@サルミ!そーす@ソース・---}
\index{sauce@sauce!salmis@--- Salmis}
\index{salmis@salmis!sauce@Sauce ---} \index[src]{salmis@Salmis}
\index[src]{さるみ@サルミ}

ソースというよりはむしろクリ\footnote{coulis \textless{} couler
  クレ「流れる」から派生した語だが、料理用語としては、やや水分の多いピュレと理解するといい。日本では「クーリ」と呼ぶことも多い。ここでは二つの解釈が可能で、ひとつは\protect\hyperlink{}{ポタージュ・クリ}に近いという意味。もうひとつは「昔ながらのソース」の意。後者の場合、エスコフィエが「古典料理」と呼ぶ17、18世紀においてソースのことをクリと呼んでいたのを踏まえていると考えられる。}と呼んだほうがいいこのソースの作り方はどんな場合も一点を除いて変わることがない。それは、このソースを合わせるジビエ(鳥)の種類によって、つまり普通に肉料理として扱えるジビエか、肉断ち\footnote{小斉のこと。カトリックの習慣として(厳密な教義ではない)四旬節(復活祭までの46日間)や毎週金曜などに行なわれる、肉食を断つ行為のこと。}の際の食材として扱えるもの\footnote{ある種の水鳥はイルカと同様に魚と同等のものと見做され、小斉の場合にも食材として認められていた。具体的にはハシヒロ鴨、オナガ鴨、サルセル鴨など。もっとも、水鳥を肉断ちの際の食材として扱うというのは一種の詭弁ともいえなくないわけで、このソースを作る際に\protect\hyperlink{sauce-espagnole-maigre}{魚料理用ソース・エスパニョル}をベースとした\protect\hyperlink{sauce-demi-glace}{ソース・ドゥミグラス}を使うとは考え難く、本文にあるようにフォンの代用としてマッシュルームの茹で汁を用いるという指示を守るだけで、厳密に小斉の料理として成立するレシピと言えるかは疑問の残るところだ。}かで、どんな液体を用いるかということだけだ。

細かく刻んだ\protect\hyperlink{mirepoix}{ミルポワ} 150
gをバターでじっくり色付くまで炒める。そこに、その料理で用いているジビエの手羽と腿の皮、ガラを細かく刻んで加える。

白ワイン3
dLを注ぎ、\(\frac{1}{3}\)量まで煮詰める。\protect\hyperlink{sauce-demi-glace}{ソース・ドゥミグラス}
8
dLを加えて、約45分間弱火で煮込む。漉し器で漉すが、その際に香味野菜とガラのエキス\footnote{原文quintessence(カンテソンス)。本来の意味は錬金術でいう「第五元素」。16世紀の作家フランソワ・ラブレーは存命当時、自著を筆名「カンテサンス抽出をなし遂げたアルコフリバス師」で出版していた時期がある。もっとも、このカンテサンスという語自体は中世以来、料理において「エキス」「美味しさの本質」程度の意味でよく用いられた。}が得られるよう、強く押し絞ってやること。こうして出来たクリを、このソースを合わせる鳥と同種のものでとったフォン4
dLで薄める。

ジビエが肉断ちの食材と見做されるもので、なおかつそれを厳格に守って作らなければならない場合は、このときフォンの代わりにマッシュルームの茹で汁を用いればいい。

約45分〜1時間、弱火にかけて浮いてくる不純物を丁寧に取り除いてやる\footnote{dépouiller
  デプイエ。現代ではécumerエキュメの語を用いる現場が多い。}。さらにソースを\(\frac{2}{3}\)以下の量になるまで煮詰める。これにマッシュルームの茹で汁とトリュフエッセンスを適量加えて丁度いい濃度になるよう調製する。

布で漉し、軽くバターを加えて仕上げる\footnote{原文は légèrement
  beurrerでありそのまま訳したが、現代の調理現場ではmonter au beurre
  バターでモンテする、という表現がよく使われる。}。

\hypertarget{nota-sauce-salmis}{%
\subparagraph{【原注】}\label{nota-sauce-salmis}}

仕上げの際に、ソース1 Lあたりバター約50 gを加えるが、これは任意。

\atoaki{}

\hypertarget{sauce-tortue}{%
\subsubsection[ソース・トルチュ]{\texorpdfstring{ソース・トルチュ\footnote{tortue
  (トルチュ)は海亀のこと。古くは海亀料理用のソースだったが、19世紀以降は仔牛の頭肉料理に合わせるのが一般的になった。なお、
  tortu(e)という形容詞があり「曲がりくねった、(性格が)ひねくれた」という同音異義語があるが、このソースの由来とは無関係。}}{ソース・トルチュ}}\label{sauce-tortue}}

\frsub{Sauce Tortue}

\index{そーす@ソース!とるちゆ@---・トルチュ}
\index{とるちゆ@トルチュ!そーす@ソース・---}
\index{うみかめ@海亀 ⇒ トルチュ!そーすとるちゆ@ソース・トルチュ}
\index{sauce@sauce!tortue@--- Tortue}
\index{tortue@tortue!sauce@Sauce ---} \index[src]{tortue@Tortue}
\index[src]{とるちゆ@トルチュ}

2 \(\frac{1}{2}\)
Lの\protect\hyperlink{jus-de-veau-brun}{仔牛のフォン}を鍋で沸かし、セージ3
g、マジョラム1 g、ローズマリー1 g、バジル2 g、タイム1 g、ローリエの葉1
g、パセリの葉1つまみ、マッシュルームの切りくず25 gを投入する。蓋をして
25分間煎じる。こうして煎じた液体を漉す2分前に大粒のこしょう4個を加える。

布で漉し、\protect\hyperlink{sauce-demi-glace}{ソース・ドゥミグラス} 7
dLに\protect\hyperlink{sauce-tomate}{トマトソース} 3
dLを合わせたものに、上記で煎じた液体を、風味が際立つ程度に適量加える。\(\frac{3}{4}\)
量まで煮詰め、布で漉す。仕上げにマデラ酒1
dLとトリュフエッセンス少々を加え、さらにカイエンヌで風味を引き締める。

\hypertarget{nota-sauce-tortue}{%
\subparagraph{【原注】}\label{nota-sauce-tortue}}

このソースはある程度まとまった量で作る必要がある。カイエンヌを使う指示があるからだ。それでも、カイエンヌはとても気をつけて量を加減する必要がある\footnote{フランス料理において(というよりも伝統的かつ一般的なフランス人にとって)、唐辛子の辛さは嫌われる傾向が非常に強い。}。

\atoaki{}

\hypertarget{sauce-venaison}{%
\subsubsection[ソース・ヴネゾン]{\texorpdfstring{ソース・ヴネゾン\footnote{Venaison(ヴネゾン)とはノロ鹿chevreuilや猪sanglierなどの大型ジビエのこと。なおニホンジカやエゾジカはcerf(セール)に分類され、フランス料理の食材としてはあまり高く評価されない傾向がある。}}{ソース・ヴネゾン}}\label{sauce-venaison}}

\frsub{Sauce Venaison}

\index{そーす@ソース!うねそん@---・ヴネゾン}
\index{うねそん@ヴネゾン!そーす@ソース・---}
\index{おおかたしひえ@大型ジビエ ⇒ ヴネゾン!そーす@ソース・ヴネゾン}
\index{sauce@sauce!venaison@--- Venaison}
\index{venaison@venaison!sauce@Sauce ---} \index[src]{venaison@Venaison}
\index[src]{うねそん@ヴネゾン}

完全に仕上げた\protect\hyperlink{sauce-poivrade-pour-gibier}{ジビエ用ソース・ポワヴラード}
\(\frac{3}{4}\)
Lに、\protect\hyperlink{gelee-de-groseilles-a}{グロゼイユのジュレ}大さじ3杯強を生クリーム1
dLで溶いてから加える。

グロゼイユのジュレと生クリームを加えるのは、鍋を火から外して、提供直前にすること。

\ldots{}\ldots{}大型ジビエ料理用。

\atoaki{}

\hypertarget{sauce-vin-rouge}{%
\subsubsection{赤ワインソース}\label{sauce-vin-rouge}}

\frsub{Sauce au Vin rouge}

\index{そーす@ソース!あかわいん@赤ワイン---}
\index{あかわいん@赤ワイン!そーす@---ソース}
\index{sauce@sauce!vin rouge@--- au Vin rouge}
\index{vin@vin!sauce rouge@Sauce au --- rouge}
\index[src]{vin rouge@Vin rouge (au)} \index[src]{あかわいん@赤ワイン}

「赤ワインソース」という場合、煮詰めてからブールマニエでとろみを付けるブルゴーニュ風の仕立てか、魚を煮るのに用いた赤ワインを使うことが特徴である「ソース・マトロット」のいずれかから派生したものなのは言うまでもない。もっとも、後者の場合はワインの風味は失われてしまっていてソースの水気と味付けの意味しか持っていないと言える。

両者どちらもまさしく「赤ワインソース」だが、\protect\hyperlink{sauce-bourguignonne}{ブルゴーニュ風ソース}と\protect\hyperlink{sauce-matelote}{ソース・マトロット}はそれぞれ作り方も用途も違うから別々の名称として、この「茶色い派生ソース」の節で説明した。

筆者としては、本当の「赤ワインソース」は以下のように作るものと考えている。

ごく細かく刻んだ標準的な\protect\hyperlink{mirepoix}{ミルポワ} 125
gをバターで炒める。良質の赤ワイン \(\frac{1}{2}\)
Lを注ぐ。半量になるまで煮詰める。つぶしたにんにく1片、\protect\hyperlink{sauce-espagnole}{ソース・エスパニョル}
7 \(\frac{1}{2}\) dLを加え、12〜
15分、火にかけて浮いてくる不純物を丁寧に取り除く\footnote{dépouiller
  デプイエ ≒ écumer エキュメ。}。

布で漉し、バター100
gとアンチョビエッセンス小さじ1杯、カイエンヌ1つまみを加えて仕上げる。

\ldots{}\ldots{}魚料理用ソース。

\atoaki{}

\hypertarget{sauce-zingara-a}{%
\subsubsection[ソース・ザンガラ
A]{\texorpdfstring{ソース・ザンガラ\footnote{もとの語形はzingaro
  ザンガロ、またはヂンガロ。ジプシー、ボヘミアンの意。料理ではパプリカ粉末やカイエンヌを用いたものに命名されることが多い。}
A}{ソース・ザンガラ A}}\label{sauce-zingara-a}}

\frsub{Sauce Zingara A}

\index{そーす@ソース!さんからa@---・ザンガラ A}
\index{さんから@ザンガラ!そーすa@ソース・--- A}
\index{しふしーふう@ジプシー風!そーすa@ソース・ザンガラ A}
\index{sauce@sauce!zingara a@--- Zingara A}
\index{zingara@Zingara!sauce a@Sauce --- A}
\index[src]{zingara a@Zingara A} \index[src]{さんからa@ザンガラ A}

このソースは古典料理の\protect\hyperlink{garniture-zingara}{ガルニチュール・ザンガラ}とはまったく関係がない。むしろイギリス料理に由来し、本書でもイギリス風ソースの節において似たようなものをいくつか採り上げている。

ヴィネガー2 \(\frac{1}{2}\)
dLにエシャロットのみじん切り大さじ1杯を加えて半量になるまで煮詰める。\protect\hyperlink{jus-de-veau-lie}{茶色いジュ}
7 dLを注ぎ、バターで揚げたパンの身160
gを加える。弱火で5〜6分間煮る。パセリのみじん切り大さじ1杯とレモン
\(\frac{1}{2}\) 個分の搾り汁を加えて仕上げる。

\atoaki{}

\hypertarget{sauce-zingara-b}{%
\subsubsection{ソース・ザンガラ B}\label{sauce-zingara-b}}

\frsub{Sauce Zingara B}

\index{そーす@ソース!さんからb@---・ザンガラ B}
\index{さんから@ザンガラ!そーすb@ソース・--- B}
\index{しふしーふう@ジプシー風!そーすb@ソース・ザンガラ B}
\index{sauce@sauce!zingara b@--- Zingara B}
\index{zingara@Zingara!sauce b@Sauce --- B}
\index[src]{zingara b@Zingara B} \index[src]{さんからb@ザンガラ B}

白ワイン3 dLとマッシュルームの茹で汁3
dLを合わせて\(\frac{1}{3}\)量になるまで煮詰める。

\protect\hyperlink{sauce-demi-glace}{ソース・ドゥミグラス} 4
dLと\protect\hyperlink{sauce-tomate}{トマトソース} 2 \(\frac{1}{2}\)
dL、\protect\hyperlink{fonds-blanc}{白いフォン} 1
dLを注ぐ。浮いてくる不純物を徹底的に取り除きながら5〜6分火にかける。

仕上げに、カイエンヌ1つまみで風味を引き締め、太さ1〜2 mmの千切りにした
\footnote{julienne
  (ジュリエーヌ)。日本語では「ジュリエンヌ」と言うことが多いが、「ジュリヤン」のように言う調理現場もある。}ハム(脂身のないところ)と赤く漬けた舌肉70
gおよびマッシュルーム 50 g、トリュフ 30 gを加える。

\ldots{}\ldots{}仔牛料理、鶏料理用。

\index{そーす@ソース!ふらうんはせい@\textbf{ブラウン系の派生---}|)}
\index{sauce@sauce!petites brunes composees@\textbf{Petites ---s Brunes Composées}|)}

\end{recette}

\end{document}
