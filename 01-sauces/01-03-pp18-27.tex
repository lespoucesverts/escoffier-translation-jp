\hypertarget{ux8336ux8272ux3044ux6d3eux751fux30bdux30fcux30b9}{%
\section{茶色い派生ソース}\label{ux8336ux8272ux3044ux6d3eux751fux30bdux30fcux30b9}}

\hypertarget{petites-sauces-brunes-composuxe9es}{%
\subsection{Petites Sauces Brunes
Composées}\label{petites-sauces-brunes-composuxe9es}}

\maeaki

\hypertarget{ux30bdux30fcux30b9ux30d3ux30acux30e9ux30fcux30c91}{%
\subsubsection[ソース・ビガラード]{\texorpdfstring{ソース・ビガラード\footnote{ビガラードは本来、南フランスで栽培されるビターオレンジの一種。}}{ソース・ビガラード}}\label{ux30bdux30fcux30b9ux30d3ux30acux30e9ux30fcux30c91}}

\hypertarget{sauce-bigarade}{%
\paragraph{Sauce Bigarade}\label{sauce-bigarade}}

\index{そーす@ソース!びがらーど@---・ビガラード}
\index{びがらーど@ビガラード!そーす@ソース・---}
\index{sauce@sauce!bigarade@--- Bigarade}
\index{bigarade@bigarade!sauce@Sauce ---}

\hypertarget{ux4ed4ux9d28ux306eux30d6ux30ecux30bc2-ux7528}{%
\subparagraph[仔鴨のブレゼ 用]{\texorpdfstring{仔鴨のブレゼ\footnote{ブレゼおよびポワレについては第7章「肉料理」参照。}
用}{仔鴨のブレゼ 用}}\label{ux4ed4ux9d28ux306eux30d6ux30ecux30bc2-ux7528}}

仔鴨をブレゼした際の煮汁を漉してから浮き脂を取り除き\footnote{dégraisser
  デグレセ。}、煮詰める。煮詰まった らさらに目の細かい布で漉し、ソース1
Lあたりオレンジ4個とレモン1個の搾り 汁でのばす。

\hypertarget{ux4ed4ux9d28ux306eux30ddux30efux30ecux7528}{%
\subparagraph{仔鴨のポワレ用}\label{ux4ed4ux9d28ux306eux30ddux30efux30ecux7528}}

仔鴨をポワレのフォン\footnote{ここでのポワレは蒸し焼きの一種であるから、煮汁それ自体は野菜に
  含まれていた水分くらいしかない。実際には、火入れの終わった肉を取り
  出してから、鍋に適量のフォンを注いで火にかけ、残った香味野菜から風
  味を引き出したものを使う。}から浮き脂を取り除き、でんぷんで軽くとろみ付
けする。砂糖20gに大さじ\(\sfrac{1}{2}\)杯のヴィネガーを加えて火にかけカラメル状にし
たものを加える。ブレゼ用と同様に、オレンジとレモンの搾り汁でのばす。

仔鴨のブレゼ用、ポワレ用いずれの場合も、細かい千切りにしてよく下茹でし
ておいたオレンジの皮大さじ2とレモンの皮大さじ1を加えて仕上げる。

\maeaki

\hypertarget{ux30dcux30ebux30c9ux30fcux98a8ux30bdux30fcux30b9}{%
\subsubsection{ボルドー風ソース}\label{ux30dcux30ebux30c9ux30fcux98a8ux30bdux30fcux30b9}}

\hypertarget{sauce-bordelaise}{%
\paragraph{Sauce Bordelaise}\label{sauce-bordelaise}}

\index{そーす@ソース!ぼるどーふう@ボルドー風---}
\index{ぼるどーふう@ボルドー風!そーす@---ソース}
\index{sauce@sauce!bordelaise@--- Bordelaise}
\index{bordelais@bordelais!sauce@Sauce Bordelaise}

赤ワイン3 dl にエシャロットのみじん切り大さじ2、粗く砕いたこしょう、タ
イム、ローリエの葉\(\sfrac{1}{2}\)枚を加えて火にかけ、\(\sfrac{1}{4}\)量になるまで煮詰める。ソー
ス・エスパニョル1dlを加えて火にかけ、浮いてくる夾雑物を丁寧に取り除き
ながら弱火で15分間煮る。目の細かい布で漉す。

溶かしたグラスドヴィアンド大さじ1杯とレモン汁\(\sfrac{1}{4}\)個分、細かいさいの目
か輪切りにしてポシェしておいた牛骨髄を加えて仕上げる。

\ldots{}\ldots{}牛、羊の赤身肉のグリル用

【原注】こんにちではボルドー風ソースをこのように赤ワインを用いて作るが、
本来的には誤りである。もともとは白ワインが用いられていた。白ワインを用
いるものについては「ボルドー風ソース ボヌフォワ」として後述。

\maeaki

\hypertarget{ux30d6ux30ebux30b4ux30fcux30cbux30e5ux98a8ux30bdux30fcux30b9}{%
\subsubsection{ブルゴーニュ風ソース}\label{ux30d6ux30ebux30b4ux30fcux30cbux30e5ux98a8ux30bdux30fcux30b9}}

\hypertarget{sauce-bourguignonne}{%
\paragraph{Sauce Bourguignonne}\label{sauce-bourguignonne}}

\index{そーす@ソース!ぶるごーにゅふう@ブルゴーニュ風---}
\index{ぶるごーにゅふう@ブルゴーニュ風!そーす@---ソース}
\index{sauce@sauce!bourguignonne@--- Bourguignonne}
\index{bourguignon@bourguignon!sauce@Sauce Bourguignonne}

上質の赤ワイン1\(\sfrac{1}{2}\) L
に、エシャロット5個の薄切りとパセリの枝、タイム、
ローリエの葉\(\sfrac{1}{2}\)枚、マッシュルームの切りくず\footnote{料理に使うマッシュルームは通常、トゥルネ(包丁を持った側の手は動
  かさずに材料を回して切ることからついた用語)すなわち螺旋状に切って
  供するが、その際に少なくない量の切りくずが出るのでこれを使う。}25gを加えて、半量になる
まで煮詰める。布で漉し、ブールマニエ80g(バター45gと小麦粉35g)を加え
てとろみを付ける。提供直前にバター150gを溶かし込み、カイエンヌ\footnote{赤唐辛子の粉末だが、カイエンヌは本来、品種名。日本でよく用いられ
  ているタカノツメなどと比べると辛さもややマイルドで、風味も異なる。}ごく
少量で加えて風味よく仕上げる。

\ldots{}\ldots{}いろいろな卵料理や、家庭料理に好適なソース。

\maeaki

\hypertarget{ux30d6ux30ebux30bfux30fcux30cbux30e5ux98a8ux30bdux30fcux30b9}{%
\subsubsection{ブルターニュ風ソース}\label{ux30d6ux30ebux30bfux30fcux30cbux30e5ux98a8ux30bdux30fcux30b9}}

\hypertarget{sauce-bretonne}{%
\paragraph{Sauce Bretonne}\label{sauce-bretonne}}

\index{そーす@ソース!ぶるたーにゅふうちゃいろ@ブルターニュ風--- (茶色)}
\index{ぶるたーにゅふう@ブルターニュ風!そーすちゃいろ@---ソース (茶色)}
\index{sauce@sauce!bretonne brune@--- Bretonne (brune)}
\index{breton@breton!sauce brune@Sauce Bretonne (brune)}

中位の玉ねぎ2個をみじん切りにして、バターでブロンド色になるまで炒める。
白ワイン2\(\sfrac{1}{2}\)dlを注ぎ、半量になるまで煮詰める。ここにソース・
エスパニョル3\(\sfrac{1}{2}\)およびトマトソース同量を加える。7〜8分間煮
立ててから、刻んだパセリを加えて仕上げる。

【原注】このソースは{[}白いんげん豆のブルターニュ風{]}以外にはほとんど使わ
れない。

\maeaki

\hypertarget{ux30bdux30fcux30b9ux30b9ux30eaux30fcux30ba6}{%
\subsubsection[ソース・スリーズ]{\texorpdfstring{ソース・スリーズ\footnote{スリーズ
  cerises はさくらんぼのこと。このレシピでグロゼイユ(す
  ぐり)のジュレを用いるが、古くはさくらんぼを用いていたことからこの
  名称となった。}}{ソース・スリーズ}}\label{ux30bdux30fcux30b9ux30b9ux30eaux30fcux30ba6}}

\hypertarget{sauce-aux-cerises}{%
\paragraph{Sauce aux cerises}\label{sauce-aux-cerises}}

\index{そーす@ソース!すりーず@---・スリーズ}
\index{sauce@sauce!cerise@--- aux Cerises}

ポルト酒2dlにイギリス風ミックススパイスひとつまみと、すりおろしたオレ
ンジの皮を大さじ\(\sfrac{1}{2}\)杯加えて\(\sfrac{2}{3}\)量になるまで煮詰め
る。 \href{}{グロゼイユのジュレ}
2\(\sfrac{1}{2}\)を加え、仕上げにオレンジ果汁を加 える。

\ldots{}\ldots{}大型猟獣肉の料理用だが、鴨のポワレやブレゼにも用いられる。

\maeaki

\hypertarget{ux30bdux30fcux30b9ux30b7ux30e3ux30f3ux30d4ux30cbux30e7ux30f37}{%
\subsubsection[ソース・シャンピニョン]{\texorpdfstring{ソース・シャンピニョン\footnote{champignons
  キノコ全般を意味する語だが、単独で用いられる場合はい
  わゆるマッシュルームを指す。}}{ソース・シャンピニョン}}\label{ux30bdux30fcux30b9ux30b7ux30e3ux30f3ux30d4ux30cbux30e7ux30f37}}

\hypertarget{sauce-aux-champignons}{%
\paragraph{Sauce aux Champignons}\label{sauce-aux-champignons}}

\index{そーす@ソース!まっしゅるーむちゃいろ@マッシュルーム--- (茶色)}
\index{まっしゅるーむ@マッシュルーム!そーすちゃいろ@---ソース (茶色)}
\index{sauce@sauce!champignons brune@--- aux Champignons (brune)}
\index{champignon@champignon!sauce brune@Sauce aux Champignons (brune)}

マッシュルームの煮汁2\(\sfrac{1}{2}\) dl を半量になるまで煮詰める。
\protect\hyperlink{sauce-demi-glace}{ソー ス・ドゥミグラス}8 dl
を加えて数分間煮立てる。布で漉し、バター50gを
投入して味を調え、あらかじめ下茹でしておいた小さめのマッシュルームの笠
100gを加えて仕上げる。

\maeaki

\hypertarget{ux30bdux30fcux30b9ux30b7ux30e3ux30ebux30adux30e5ux30c6ux30a3ux30a8ux30fcux30eb8}{%
\subsubsection[ソース・シャルキュティエール]{\texorpdfstring{ソース・シャルキュティエール\footnote{シャルキュトリ(豚肉加工業)風、の意。Charcutrieの語源はchar(肉)
  +cuite(調理された)+rie(業)。ハムやソーセージなどと定番の組合せ
  であるマスタードをベースとしているソース・ロベールと、おなじく定番
  のつけ合わせであるコルニション(小さいうちに収穫してヴィネガー漬け
  にしたきゅうり。専用品種がある)を使うことから、シャルキュトリ風と
  呼ばれる。}}{ソース・シャルキュティエール}}\label{ux30bdux30fcux30b9ux30b7ux30e3ux30ebux30adux30e5ux30c6ux30a3ux30a8ux30fcux30eb8}}

\hypertarget{sauce-charcutiere}{%
\paragraph{Sauce Charcutière}\label{sauce-charcutiere}}

\index{そーす@ソース!しゃるきゅとりふう@シャルキュトリ風---}
\index{しゃるきゅとりふう@シャルキュトリ風!そーす@---ソース}
\index{sauce@sauce!charcutière@--- Charcutière}
\index{charcutier@charcutier!sauce@Sauce Charcutière}

提供直前に、\href{}{ソース・ロベール}1 L に細さ2mm程度で短かめの
千切り\footnote{1〜2mm程度の細さの千切りにした野菜などをジュリエンヌjulienneと呼ぶ。}にしたものを加える(\href{}{ソース・ロベール}参照)。

\maeaki

\hypertarget{ux30bdux30fcux30b9ux30b7ux30e3ux30b9ux30fcux30eb10}{%
\subsubsection[ソース・シャスール]{\texorpdfstring{ソース・シャスール\footnote{狩人風、の意。古くは猟獣肉をすり潰したものを使った料理を指した
  という説もある。マッシュルームとエシャロット、白ワインを使うのが特
  徴であり、このソースを使った料理にも「シャスール」の名が付けられる。}}{ソース・シャスール}}\label{ux30bdux30fcux30b9ux30b7ux30e3ux30b9ux30fcux30eb10}}

\hypertarget{sauce-chasseur}{%
\paragraph{Sauce Chasseur}\label{sauce-chasseur}}

\index{そーす@ソース!しゃすーる@---・シャスール}
\index{しゃすーる@シャスール!そーす@ソース・---}
\index{sauce@sauce!chasseur@--- Chasseur}
\index{chasseur@chasseur!sauce@Sauce ---}

生のマッシュルームを薄切りにしたもの150gをバターで炒める。エシャロット
\footnote{échalote
  玉ねぎによく似ているが、小ぶりで水分が少なく、香味野菜
  としてよく用いられる。伝統的な品種は種子ではなく種球を植えて栽培す
  る。なお、日本でしばしば「エシャレット」の名称で流通しているものは
  ラッキョウの若どりであり、フランス料理で用いるエシャロットとはまっ
  たく異なる。}のみじん切り大さじ2\(\sfrac{1}{2}\)杯を加えてさらに軽く炒め、白ワイ
ン3 dl
を注ぎ、半量になるまで煮詰める。\protect\hyperlink{sauce-tomate}{ソマトソース}3
dl と\protect\hyperlink{sauce-demi-glace}{ソース・ ドゥミグラス}2
dlを加える。数分間沸騰させたら、バター150 gと、セルフ イユ\footnote{cerfeuil
  日本ではチャービルとも呼ばれるセリ科のハーブ。}とエストラゴン\footnote{estragon
  日本ではタラゴンとも呼ばれるヨモギ科のハーブ。フレンチ
  タラゴンとロシアンタラゴンの2種がある。料理に用いるのはフレンチタ
  ラゴン。}をみじん切りにしたもの大さじ
1\(\sfrac{1}{2}\)杯を加えて仕上げる。

\maeaki

\hypertarget{ux30bdux30fcux30b9ux30b7ux30e3ux30b9ux30fcux30ebux30a8ux30b9ux30b3ux30d5ux30a3ux30a8ux6d41}{%
\subsubsection{ソース・シャスール(エスコフィエ流)}\label{ux30bdux30fcux30b9ux30b7ux30e3ux30b9ux30fcux30ebux30a8ux30b9ux30b3ux30d5ux30a3ux30a8ux6d41}}

\hypertarget{sauce-chasseur-procede-escoffier}{%
\paragraph{Sauce Chasseur (Procédé
Escoffier)}\label{sauce-chasseur-procede-escoffier}}

\index{そーす@ソース!しゃすーるえすこふぃえ@---・シャスール(エスコフィエ流)}
\index{しゃすーる@シャスール!そーすしゃすーるえすこふぃえ@ソース・--- (エスコフィエ流)}
\index{sauce@sauce!chasseur escoffier@--- Chasseur (Procédé Escoffier)}
\index{chasseur@chasseur!sauce escoffier@Sauce --- (Procédé Escoffier)}

生のマッシュルームを薄切りにしたもの150gを、バターと植物油で軽く色付く
まで炒める。みじん切りにしたエシャロット大さじ1杯を加え、なるべくすぐ
に余分な油をきる。白ワイン2dl とコニャック約50ml を注ぎ、半量になるま
で煮詰める。\protect\hyperlink{sauce-demi-glace}{ソース・ドゥミグラス}4
dlと{[}トマトソース{]}2 dl、\protect\hyperlink{glace-de-viande}{グラス
ドヴィアンド}大さじ\(\sfrac{1}{2}\)杯を加える。

5分間沸騰させたら、仕上げにパセリのみじん切り少々を加える。

\maeaki

\hypertarget{ux8336ux8272ux3044ux30bdux30fcux30b9ux30b7ux30e7ux30d5ux30edux30ef15}{%
\subsubsection[茶色いソース・ショフロワ]{\texorpdfstring{茶色いソース・ショフロワ\footnote{chaudショ「熱い、温かい」とfroidフロワ「冷たい」の合成語で、
  火を通した肉や魚を冷まし、表面にこのソース・ショフロワを覆うように
  塗り付け、さらにジュレを覆いかけた料理。料理の発祥については諸説あ
  り、なかでもルイ15世に仕えていた料理長ショフロワChaufroixが考案し
  たという説を支持してなのか、英語ではこの料理をChaufroixと綴ること
  も多い。Chaud-froidの表記は19世紀後半には文献に見られる。なお、複
  数形はchauds-froidsと綴る。トリュフの薄切りやエストラゴンなどのハー
  ブその他で表面に華麗な装飾を施すことが19世紀には盛んに行なわれてい
  た。現代でも装飾に凝った仕立てにするケースは多い。}}{茶色いソース・ショフロワ}}\label{ux8336ux8272ux3044ux30bdux30fcux30b9ux30b7ux30e7ux30d5ux30edux30ef15}}

\hypertarget{sauce-chaud-froid-brune}{%
\paragraph{Sauce Chaud-froid brune}\label{sauce-chaud-froid-brune}}

\index{そーす@ソース!しょふろわちゃいろ@---・ショフロワ(茶色)}
\index{しょふろわ@ショフロワ!そーす(ちゃいろ)@ソース・--- (茶色)}
\index{sauce@sauce!chaud-froid brune@--- Chaud-froid brune}
\index{chaud-froid@chaud-froid!sauce brune@Sauce --- brune}

(仕上がり1L 分)

\protect\hyperlink{sauce-demi-glace}{ソース・ドゥミグラス}\(\sfrac{3}{4}\)
Lとトリュフエッ センス1 dl、ジュレ6〜7 dlを用意する。

ソース・ドゥミグラスにトリュフエッセンスを加えて、強火で煮詰めるが、こ
の時に鍋から離れないこと。煮詰めながらジュレを少量ずつ加えていく。最終
的に\(\sfrac{2}{3}\)量程度まで煮詰める。

味見をして、ソースがショフロワに使うのに丁度いい濃さになっているか確
認すること。

マデラ酒またはポルト酒\(\sfrac{1}{2}\)dlを加える。布で漉し、ショフロワの主素材の表
面に塗り付けるのに丁度いい固さになるまで、丁寧にゆっくり混ぜながら冷ます。

\maeaki

\hypertarget{ux8336ux8272ux3044ux30bdux30fcux30b9ux30b7ux30e7ux30d5ux30edux30efux9d28ux7528}{%
\subsubsection{茶色いソース・ショフロワ(鴨用)}\label{ux8336ux8272ux3044ux30bdux30fcux30b9ux30b7ux30e7ux30d5ux30edux30efux9d28ux7528}}

\hypertarget{sauce-chaud-froid-brune-pour-canards}{%
\paragraph{Sauce Chaud-froid brune pour
Canards}\label{sauce-chaud-froid-brune-pour-canards}}

\index{そーす@ソース!しょふろわちゃいろかもよう@茶色い---・ショフロワ(鴨用)}
\index{しょふろわ@ショフロワ!ちゃいろいそーすしょふろわかもよう@茶色いソース・---(鴨用)}
\index{sauce@sauce!chaud-froid brune pour canards@--- Chaud-froid brune pour Canards}
\index{chaud-froid@chaud-froid!sauce brune pour Canards@Sauce --- brune pour Canards}

作り方は上記、\protect\hyperlink{sauce-chaud-froid-brune}{茶色いソース・ショフロワ}と同様だが、トリュフエッセン
スではなく、鴨のガラでとったフュメ1\(\sfrac{1}{2}\)
dlを用いること。また、 上記のレシピよりややしっかり煮詰めること。

ソースを布で漉したら、オレンジ3個分の搾り汁、とオレンジの皮をごく薄く
剥いて細かい千切りにしたもの\footnote{zeste
  ゼスト。オレンジやレモンの皮の表面を器具を用いてすりおろ
  すか、ナイフでごく薄く表皮を向き、細かい千切りにしたもの。ここでは
  後者を使う指定になっている。}大さじ2杯を加える。オレンジの皮の千切
りはしっかりと下茹でしてよく水気をきっておくころ。

\maeaki

\hypertarget{ux8336ux8272ux3044ux30bdux30fcux30b9ux30b7ux30e7ux30d5ux30edux30efux30b8ux30d3ux30a8ux7528}{%
\subsubsection{茶色いソース・ショフロワ(ジビエ用)}\label{ux8336ux8272ux3044ux30bdux30fcux30b9ux30b7ux30e7ux30d5ux30edux30efux30b8ux30d3ux30a8ux7528}}

\hypertarget{sauce-chaud-froid-brune-pour-gibier}{%
\paragraph{Sauce Chaud-froid brune pour
Gibier}\label{sauce-chaud-froid-brune-pour-gibier}}

\index{そーす@ソース!しょふろわちゃいろじびえよう@茶色い---・ショフロワ(ジビエ用)}
\index{しょふろわ@ショフロワ!そーすしょふろわじびえよう@茶色いソース・---(ジビエ用)}
\index{sauce@sauce!chaud-froid brune pour Gibier@--- Chaud-froid brune pour Gibier}
\index{chaud-froid@chaud-froid!sauce brune pour Gibier@Sauce --- brune pour Gibier}

作り方は上記\protect\hyperlink{sauce-chaud-froid-brune}{標準的なソース・ショフロワ}と同じだが、トリュフエッセンスではなく、ショフロワとして供するジビエのガラでとったフュメ\footnote{XX頁、\protect\hyperlink{fonds-de-gibier}{ジビエのフォン}参照。}2dlを用いること。

\maeaki

\hypertarget{ux30c8ux30deux30c8ux5165ux308aux30bdux30fcux30b9ux30b7ux30e7ux30d5ux30edux30ef}{%
\subsubsection{トマト入りソース・ショフロワ}\label{ux30c8ux30deux30c8ux5165ux308aux30bdux30fcux30b9ux30b7ux30e7ux30d5ux30edux30ef}}

\hypertarget{sauce-chaud-froid-tomatee}{%
\paragraph{Sauce Chaud-froid tomatée}\label{sauce-chaud-froid-tomatee}}

\index{そーす@ソース!しょふろわとまといり@トマト入り---・ショフロワ}
\index{しょふろわ@ショフロワ!そーす(とまといり)@トマト入りソース・---}
\index{sauce@sauce!chaud-froid tomatée@--- Chaud-froid tomatée}
\index{chaud-froid@chaud-froid!sauce tomatée@Sauce --- tomatée}

良質で、既によく煮詰めてあるトマトピュレ1 Lを、さらに煮詰めながら7〜8
dlのジュレを少しずつ加えていく。全体量が1L以下になるまで煮詰めること。

布で漉し、使いやすい固さになるまで、ゆっくり混ぜながら冷ます。

\maeaki

\hypertarget{ux30bdux30fcux30b9ux30b7ux30e5ux30f4ux30ebux30a4ux30e6}{%
\subsubsection{ソース・シュヴルイユ}\label{ux30bdux30fcux30b9ux30b7ux30e5ux30f4ux30ebux30a4ux30e6}}

\hypertarget{sauce-chevreuil}{%
\paragraph{Sauce Chevreuil}\label{sauce-chevreuil}}

\index{しゅうるいゆ@シュヴルイユ!そーす@ソース・---}
\index{そーす@ソース!しゅうるいゆ@---・シュヴルイユ}
\index{のろしか@ノロ鹿!そーすしゅうるいゆ@ソース・シュヴルイユ}
\index{sauce@sauce!chevreuil@--- Chevreuil}
\index{chevreuil@chevreuil!sauce@Sauce ---}

標準的な{[}ソース・ポワヴラード{]}と同様に作るが、

\begin{enumerate}
\def\labelenumi{\arabic{enumi}.}
\item
  マリネした牛・羊肉の料理に添える場合\footnote{chevreuil
    シュヴルイユはノロ鹿のことだが、このように事前にマリ
    ネした牛・羊肉を用いた料理にもこのソースを使い「シュヴルイユ(風)」
    と謳う。1806年刊ヴィアール『帝国料理の本』においてノロ鹿のフィレは
    香辛料を加えたワインヴィネガーで48時間マリネしてから調理すると書か
    れている。オド『女性料理人のための本』では、確認出来た1834年の第4版
    から1900年の第78版に至るまで、ノロ鹿の項において「一週間もヴィネガー
    たっぷりの漬け汁でマリネするのはやりすぎだが、強い味が好みなら1〜4
    日間」香辛料と赤ワインあるいはヴィネガーでマリネするといい、と説明
    されている。つまり、ノロ鹿とは必ずマリネしてから調理するものという
    一種のコンセンサスがあったために、マリネした牛・羊肉の料理にも「シュ
    ヴルイユ(風)」の名称が謳われるようになったと考えられる。}は、ハム入りの\protect\hyperlink{mirepoix}{ミルポワ}\footnote{XX参照。}を加える。
\item
  ジビエ料理に添える場合は、そのジビエの端肉を加える。
\end{enumerate}

素材をヘラなどで強く押し付けるようにして漉す\footnote{シノワ(XX訳注参照)などを用いる。}。良質の赤ワイン
1\(\sfrac{1}{2}\)dlをスプーン1杯ずつ加えながら煮て、浮き上がってくる不純
物を丁寧に取り除いていく\footnote{dépouiller デプイエ。XX訳注XX参照。}。

最後に、カイエンヌ\footnote{XX訳注XX参照。}ごく少量と砂糖1つまみを加えて味を\ruby{調}{ととの}え、布で漉す。

\maeaki

\hypertarget{ux30bdux30fcux30b9ux30b3ux30ebux30d9ux30fcux30eb23}{%
\subsubsection[ソース・コルベール]{\texorpdfstring{ソース・コルベール\footnote{17世紀の政治家、ジャン・バティスト・コルベール(1619〜1683)の名を冠したもの。}}{ソース・コルベール}}\label{ux30bdux30fcux30b9ux30b3ux30ebux30d9ux30fcux30eb23}}

\hypertarget{sauce-colbert}{%
\paragraph{Sauce Colbert}\label{sauce-colbert}}

\index{そーす@ソース!こるべーる@---・コルベール}
\index{こるべーる@コルベール!そーす@ソース・---}
\index{sauce@sauce!colbert@--- Colbert}
\index{colbert@Colbert!sauce@Sauce ---}

\href{}{メートルドテルバター}に\protect\hyperlink{glace-de-viande}{グラスドヴィアンド}を加えたもののことだが、正しくは「\href{}{コルベールバター}」と呼ぶべきものだ\footnote{具体的なレシピは「コルベールバター」p.XX参照のこと。}。

また、コルベールバターと\href{}{ソース・シャトーブリアン}との違いを明確にさせ
ようとして、メートルドテルバターにエストラゴンを加える者もいる。だが、
必ずそうすべきということではない。実際、ブール・コルベールとソース・シャ
トーブリアンは明らかに違うものだからだ。ソース・シャトーブリアンは軽く
仕上げたグラスドヴィアントにバターをパセリのみじん切りを加えたものであ
る。一方、コルベールバターあるいはソース・コルベールと呼ばれているもの
はあくまでもバターが主であって、グラスドヴィアンドは補助的なものに過ぎ
ない。

\maeaki

\hypertarget{ux30bdux30fcux30b9ux30c7ux30a3ux30a2ux30fcux30d6ux30eb25}{%
\subsubsection[ソース・ディアーブル]{\texorpdfstring{ソース・ディアーブル\footnote{悪魔の意。}}{ソース・ディアーブル}}\label{ux30bdux30fcux30b9ux30c7ux30a3ux30a2ux30fcux30d6ux30eb25}}

\hypertarget{sauce-diable}{%
\paragraph{Sauce Diable}\label{sauce-diable}}

\index{そーす@ソース!でぃあーぶる@---・ディアーブル}
\index{でぃあーぶる@ディアーブル!そーす@ソース・---}
\index{sauce@sauce!diable@--- Diable}
\index{diable@diable!sauce@Sauce ---}

このソースはごく少量ずつ作るのが一般的だが、ここではそれを守らずに、仕
上り2\(\sfrac{1}{2}\) dlとして説明する

白ワイン3dlにエシャロット3個分のみじん切りを加え、\(\sfrac{1}{3}\)量以下になるまで煮詰める。

\protect\hyperlink{sauce-demi-glace}{ソース・ドゥミグラス}2
dlを加えて数分間煮立たせ、仕上げにカイエンヌの
粉末をたっぷり効かせる\footnote{唐辛子の品種としてのカイエンヌは日本で一般的なタカノツメよりも
  比較的辛さがマイルドなので、「たっぷり」という表現に惑わされないよ
  う注意。}。

【原注】

白ワインではなくヴィネガーを煮詰め、仕上げにハーブを加えて作る調理現場
もあるが、著者としては上記の作り方がいいと思う。

\maeaki

\hypertarget{ux30bdux30fcux30b9ux30c7ux30a3ux30a2ux30fcux30d6ux30ebux30a8ux30b9ux30b3ux30d5ux30a3ux30a8}{%
\subsubsection{ソース・ディアーブル・エスコフィエ}\label{ux30bdux30fcux30b9ux30c7ux30a3ux30a2ux30fcux30d6ux30ebux30a8ux30b9ux30b3ux30d5ux30a3ux30a8}}

\hypertarget{sauce-diable-escoffier}{%
\paragraph{Sauce Diable Escoffier}\label{sauce-diable-escoffier}}

\index{そーす@ソース!でぃあーぶるえすこふぃえ@---・ディアーブル・エスコフィエ}
\index{でぃあーぶるえすこふぃえ@ディアーブル・エスコフィエ!そーす@ソース・---・エスコフィエ}
\index{sauce@sauce!diable escoffier@--- Diable Escoffier}
\index{diable@diable!sauce escoffier@Sauce --- Escoffier}

このソースは完成品が市販\footnote{現在は市販されていないと思われる。フランスにおいては未確認だが、
  1980年代までアメリカ合衆国ではナビスコが瓶詰めを生産、販売していた。}されている。同量の柔くしたバターを混ぜ合わせるだけでいい。

\maeaki

\hypertarget{ux30bdux30fcux30b9ux30c7ux30a3ux30a2ux30fcux30cc28}{%
\subsubsection[ソース・ディアーヌ]{\texorpdfstring{ソース・ディアーヌ\footnote{ローマ神話の女神ディアーナのこと。ギリシア神話のアルテミスに相
  当し、狩猟、貞潔の女神。また月の女神ルーナ(セレーネー)と同一視さ
  れた。ここでは大型ジビエ料理用のソースであるから、狩猟の女神という
  意味合いが強い。}}{ソース・ディアーヌ}}\label{ux30bdux30fcux30b9ux30c7ux30a3ux30a2ux30fcux30cc28}}

\hypertarget{sauce-diane}{%
\paragraph{Sauce Diane}\label{sauce-diane}}

\index{そーす@ソース!てぃあーぬ@---・ディアーヌ}
\index{てぃあーぬ@ディアーヌ!そーす@ソース・---}
\index{sauce@sauce!diane@--- Diane} \index{diane@Diane!sauce@Sauce ---}

不純物を充分に取り除き、コクと風味ゆたかな{[}ソース・ポワヴラード{]}5
dlを 用意する。提供直前に、泡立てた生クリーム4
dl(生クリーム2dlを泡立てて
倍量にする)と、小さな三日月の形にしたトリュフのスライスと固茹で卵の白
身を加える。

\ldots{}\ldots{}大型ジビエの骨付き背肉および、その中心部を円筒形に切り出したもの\footnote{noisette
  ノワゼット。}、フィレ料理用。

\maeaki

\hypertarget{ux30bdux30fcux30b9ux30c7ux30e5ux30afux30bbux30eb29}{%
\subsubsection[ソース・デュクセル]{\texorpdfstring{ソース・デュクセル\footnote{デュクセル・セッシュ(第2章ガルニチュール参照)を用いることからこの名称が用いられている。}}{ソース・デュクセル}}\label{ux30bdux30fcux30b9ux30c7ux30e5ux30afux30bbux30eb29}}

\hypertarget{sauce-duxelles}{%
\paragraph{Sauce Duxelles}\label{sauce-duxelles}}

\index{そーす@ソース!てゅくせる@---・デュクセル}
\index{てゅくせる@デュクセル!そーす@ソース・---}
\index{sauce@sauce!duxelles@--- Duxelles}
\index{duxelles@duxelles!sauce@Sauce ---}

白ワイン2dlとマッシュルームの煮汁2 dlにエシャロットのみじん切り大さじ2
杯を加えて、\(\sfrac{1}{3}\)量まで煮詰める。\protect\hyperlink{sauce-demi-glace}{ソース・ドゥミグラ
ス}\(\sfrac{1}{2}\) Lとトマトピュレ1\(\sfrac{1}{2}\) dl、
デュクセル・セッシュ大さじ4杯を加える。5分間煮立たせ、パセリのみじん切
り大さじ\(\sfrac{1}{2}\)を加える。

\ldots{}\ldots{}グラタンの他、いろいろな料理に用いられる。

\hypertarget{ux539fux6ce8-1}{%
\subparagraph{【原注】}\label{ux539fux6ce8-1}}

ソース・デュクセルはイタリア風ソースと混同されることが多いが、ソース・
デュクセルにはハムも、赤く漬けた舌肉も入れないので、まったく別のものだ。

\maeaki

\hypertarget{ux30bdux30fcux30b9ux30a8ux30b9ux30c8ux30e9ux30b4ux30f3}{%
\subsubsection{ソース・エストラゴン}\label{ux30bdux30fcux30b9ux30a8ux30b9ux30c8ux30e9ux30b4ux30f3}}

\hypertarget{sauce-estragon}{%
\paragraph{Sauce Estragon}\label{sauce-estragon}}

\index{そーす@ソース!えすとらこんちゃいろ@---・エストラゴン(茶色いソース)}
\index{えすとらこんちゃいろ@エストラゴン!そーす@ソース・---(茶色いソース)}
\index{sauce@sauce!estragonbrune@--- Estragon (sauce brune)}
\index{estragon@estragon!sauce brune@Sauce --- (brune)}

(仕上り2\(\sfrac{1}{2}\)dl分)

白ワイン2dlを沸かし、エストラゴンの枝20gを投入する。蓋をして10分間、煎
じる\footnote{infuserアンフュゼ。}。2\(\sfrac{1}{2}\)dlの\protect\hyperlink{sauce-demi-glace}{ソース・ドゥミグラス}ま
たは、\protect\hyperlink{jus-de-veau-lie}{とろみを付けた仔牛のジュ}を加え、約\(\sfrac{2}{3}\)量になるまで
煮詰める。布で漉し、みじん切りにしたエストラゴン小さじ1杯を加えて仕上
げる。

\ldots{}\ldots{}仔牛や仔羊の背肉の中心を円筒形に切り出した料理や家禽料理用。

\maeaki

\hypertarget{ux30bdux30fcux30b9ux30d5ux30a3ux30caux30f3ux30b7ux30a8ux30fcux30eb34}{%
\subsubsection[ソース・フィナンシエール]{\texorpdfstring{ソース・フィナンシエール\footnote{Financier徴税官(財務官)風の意。フランス革命以前の徴税官は、王
  に代わって徴税を行なう大貴族が就く役職であり、膨大な利権によりきわめて
  裕福であったという。このソースと組み合わせる{[}ガルニチュール・フィナン
  シエール{]}が、雄鶏のとさかと睾丸、仔羊の胸腺肉、トリュフなどの比較的入
  手困難あるいは高級な食材で構成されていることが名称の由来と思われる。ブ
  リヤ=サヴァランは『美味礼讃』(味覚の生理学)において、徴税官たちは旬
  のはしりの食材を真っ先に食べられる、いわば特権階級だと述べている。なお、
  カレーム『19世紀フランス料理』においては、ソースとガルニチュールを分離
  せず、「ラグー・アラ・フィナンシエール」として採りあげられているが、全
  ての素材を別々に加熱調理してソースと合わせるものであり、いわゆる「煮込
  み」とは呼びがたいものとなっている。フランス料理の影響が比較的強かった
  北イタリアにこの原型に近いと思われるラグー「ピエモンテ風フィナンツィエ
  ラ」がある。鶏のとさか、肉垂、睾丸、鶏レバーおよび仔牛の胸腺肉などを煮
  込んだものだが、レシピを読む限りにおいては比較的庶民的あるいは農民的料
  理に変化したものと思われる (cf.~Anna Gosetti della Salda, \emph{Le
  Ricette Regionali Italiane}, Milano, Solares, 1967,
  p.57.)。ちなみに焼き菓子の
  フィナンシエfinancierも同語源だが、何故その名称になったかは不明。}}{ソース・フィナンシエール}}\label{ux30bdux30fcux30b9ux30d5ux30a3ux30caux30f3ux30b7ux30a8ux30fcux30eb34}}

\hypertarget{sauce-financiere}{%
\paragraph{Sauce Financière}\label{sauce-financiere}}

\index{そーす@ソース!ふぃなんしえーる@---・フィナンシエール}
\index{ふぃなんしえーる@フィナンシエール!そーす@ソース・---}
\index{ちょうせいかんふう@徴税官風!そーすふぃなんしえーる@ソース・---}
\index{sauce@sauce!financiere@--- Financière}
\index{financier@financier!sauce@Sauce Financière}

\protect\hyperlink{sauce-madere}{ソース・マデール}1\(\sfrac{1}{4}\)Lを\(\sfrac{3}{4}\)量以下になるまで煮詰
め、火から外してトリュフエッセンス1 dlを加える。布で漉して仕上げる。

\ldots{}\ldots{}\href{}{ガルニチュール・フィナンシエール}用だが、その他の肉料理にも用いられる。

\maeaki

\hypertarget{ux9999ux8349ux30bdux30fcux30b9}{%
\subsubsection{香草ソース}\label{ux9999ux8349ux30bdux30fcux30b9}}

\hypertarget{sauce-aux-fines-herbes}{%
\paragraph{Sauce aux Fines Herbes}\label{sauce-aux-fines-herbes}}

\index{そーす@ソース!こうそう@香草---}
\index{こうそう@香草!そーす@---ソース}
\index{はーぶ@ハーブ!こうそうそーす@香草ソース}
\index{sauce@sauce!fines herbes@--- aux Fines Herbes}
\index{fines herbes@fines herbes!sauce@Sauce aux ---}

白ワイン3dlを沸かし、パセリの葉、セルフイユ、エストラゴン、シブレット
を各1つまみ強、投入する。約20分間煎じる。布で漉し、\protect\hyperlink{sauce-demi-glace}{ソース・ドゥミグラ
ス}または\protect\hyperlink{jus-de-veau-lie}{とろみを付けた仔牛のジュ}6
dlを加える。仕上げに、煎じる
のに使ったのと同じ香草を細かく刻んだもの計、大さじ2\(\sfrac{1}{2}\)杯と
レモンの搾り汁少々を加える。

\hypertarget{ux539fux6ce8-2}{%
\subparagraph{【原注】}\label{ux539fux6ce8-2}}

古典料理ではこの「香草ソース」と\protect\hyperlink{sauce-duxelles}{ソース・デュクセル}が混同されるこ
ともあったが、こんにちではまったく違うものとして扱われている。

\maeaki

\hypertarget{ux30b8ux30e5ux30cdux30fcux30f4ux98a8ux30bdux30fcux30b9}{%
\subsubsection{ジュネーヴ風ソース}\label{ux30b8ux30e5ux30cdux30fcux30f4ux98a8ux30bdux30fcux30b9}}

\hypertarget{sauce-genevoise}{%
\paragraph{Sauce Genevoise}\label{sauce-genevoise}}

\index{そーす@ソース!じゅねーうふう@ジュネーヴ風---}
\index{じゅねーうふう@ジュネーヴ風!そーす@---ソース}
\index{sauce@sauce!genevoise@--- Genevoise}
\index{genevois@genevois!sauce@Sauce Genevoise}

鍋にバターを熱し、細かく刻んだミルポワを色付かないよう強火でさっと炒め
る。ミルポワの材料は、にんじん100 g、玉ねぎ80 g、タイムとローリエ少々、
パセリの枝20 g。そこにサーモンの頭1kgと粗く砕いたこしょう1つまみを入れ、
蓋をして弱火で15分程蒸し煮する。

鍋に残ったバターを捨て、赤ワイン1Lを注ぐ。半量になるまで煮詰める。そこ
に\protect\hyperlink{sauce-espagnole-maigre}{魚料理用ソース・エスパニョル}\(\sfrac{1}{2}\)
Lを加える。弱火で1時間
煮込む。漉し器を使い、材料を押しつけながら漉す。しばらく休ませてから、
表面に浮いた油脂を取り除く\footnote{dégraisser デグレセ。}

さらに赤ワイン\(\sfrac{1}{2}\) Lと、魚のフュメ\(\sfrac{1}{2}\)
Lを加える。ソー スの表面に浮いてくる不純物を徹底的に取り除き\footnote{dépouiller
  デプイエ。}、丁度いい濃さになる まで煮詰める。

これを布で漉し、静かに混ぜながら、アンチョヴィのエッセンス大さじ1杯と
バター150 gを加えて仕上げる。

\ldots{}\ldots{}サーモン、鱒料理用。

\hypertarget{ux539fux6ce8-3}{%
\subparagraph{【原注】}\label{ux539fux6ce8-3}}

このソースはもともとカレームが「ジェノヴァ風」と名付けたものだが、その
後ルキュレ、グフェと立て続けに「ジュネーヴ風」の名称を用いた。だが、ジュ
ネーヴは赤ワインの産地ではないから理屈としてはおかしい。

間違っているとはいえ、ジュネーヴ風という名称で定着してしまっているので、
本書でもそのままにしている。だが、ジュネーヴ風であれジェノヴァ風であれ、
カレーム、ルキュレ、デュボワ、グフェはいずれもこのソースに赤ワインを用
いるよう指示している。つまり赤ワインを用いることがこのソースのポイント。

\maeaki

\hypertarget{ux30bdux30fcux30b9ux30b4ux30c0ux30fcux30eb37}{%
\subsubsection[ソース・ゴダール]{\texorpdfstring{ソース・ゴダール\footnote{ファーヴル『料理および食品衛生事典』およびモンタニェ『ラルース・
  ガストロノミック』初版でも由来については述べられていないが、ガルニ
  チュール・ゴダールの構成要素がガルニチュール・フィナンシエールとよ
  く似ている点などから、おそらくは18世紀の徴税官(つまりフィナンシエ)
  であり作家としても活動したクロード・ゴダール・ドクール Claude Godard
  d'Aucour(1716〜1795)の名を冠したものと考えられる。}}{ソース・ゴダール}}\label{ux30bdux30fcux30b9ux30b4ux30c0ux30fcux30eb37}}

\hypertarget{sauce-godard}{%
\paragraph[Sauce Godard]{\texorpdfstring{Sauce Godard\footnote{底本とした現行版(第四版)では最後がdではなくtとなっているが、初版から第三
  版にいたるまでdとなっており、現行版は明らかな誤植。}}{Sauce Godard}}\label{sauce-godard}}

\index{そーす@ソース!ごだーる@---・ゴダール}
\index{ごだーる@ゴダール!そーす@ソース・---}
\index{sauce@sauce!godart@--- Godart}
\index{godard@Godard!sauce@Sauce ---}

シャンパーニュまたは辛口の白ワイン4 dlにハム入りの細かく刻んだ{[}ミルポ
ワ{]}。{[}ソース・ドゥミグラス{]}1
Lとマッシュルームのエッセンス2dlを加える。
弱火に10分かけ、シノワ\footnote{XXページ訳注参照。}で漉す。

\(\sfrac{2}{3}\)量になるまで煮詰め、布で漉す。

\ldots{}\ldots{}\href{}{ガルニチュール ゴタール}用。

\maeaki

\hypertarget{ux30bdux30fcux30b9ux30b0ux30e9ux30f3ux30f4ux30ccux30fcux30eb40}{%
\subsubsection[ソース・グランヴヌール]{\texorpdfstring{ソース・グランヴヌール\footnote{王家や貴族に仕える狩猟長のことをグランヴヌールと呼ぶ。}}{ソース・グランヴヌール}}\label{ux30bdux30fcux30b9ux30b0ux30e9ux30f3ux30f4ux30ccux30fcux30eb40}}

\hypertarget{sauce-grand-veneur}{%
\paragraph{Sauce Grand-Veneur}\label{sauce-grand-veneur}}

\index{そーす@ソース!くらんうぬーる@---・グランヴヌール}
\index{くらんうぬーる@グランヴヌール!そーす@ソース・---}
\index{sauce@sauce!grand-veneur@--- Grand-Veneur}
\index{grand-veneur@grand-veneur!sauce@Sauce ---}

大型猟獣肉のフュメで澄んだ色合いに作った{[}ソース・ポワヴラード{]}に、ソー
ス1Lあたり野うさぎの血1dlをマリネ液1dlで薄めたものを加える。

火をごく弱くして、血が沸騰しないよう気をつけながら数分間煮る。布で漉す。

\maeaki

\hypertarget{ux30bdux30fcux30b9ux30b0ux30e9ux30f3ux30f4ux30ccux30fcux30ebux30a8ux30b9ux30b3ux30d5ux30a3ux30a8ux6d41}{%
\subsubsection{ソース・グランヴヌール(エスコフィエ流)}\label{ux30bdux30fcux30b9ux30b0ux30e9ux30f3ux30f4ux30ccux30fcux30ebux30a8ux30b9ux30b3ux30d5ux30a3ux30a8ux6d41}}

\hypertarget{sauce-grand-veneur-procede-escoffier}{%
\paragraph{Sauce Grand-Veneur (Procédé
Escoffier)}\label{sauce-grand-veneur-procede-escoffier}}

\index{そーす@ソース!くらんうぬーるえすこふぃえ@---・グランヴヌール(エスコフィエ)}
\index{くらんうぬーるえすこふぃえ@グランヴヌール(エスコフィエ)!そーす@ソース・---}
\index{sauce@sauce!grand-veneur escoffier@--- Grand-Veneur (Procédé Escoffier)}
\index{grand-veneur@grand-veneur!sauce escoffier@Sauce --- (Procédé Escoffier)}

軽く仕上げた{[}ソース・ポワヴラード{]}1
Lあたり{[}グロゼイユのジュレ{]}大さじ2
杯と生クリーム2\(\sfrac{1}{2}\)dlを加える。

\ldots{}\ldots{}上記2つのソースは鹿、猪などの大きな塊肉の料理に用いる。

\maeaki

\hypertarget{ux30bdux30fcux30b9ux30b0ux30e9ux30bfux30f345}{%
\subsubsection[ソース・グラタン]{\texorpdfstring{ソース・グラタン\footnote{魚のグラタン用ソースだが、グラタンの技術的ポイントについては第7章「肉料理」参照。}}{ソース・グラタン}}\label{ux30bdux30fcux30b9ux30b0ux30e9ux30bfux30f345}}

\hypertarget{sauce-gratin}{%
\paragraph{Sauce Gratin}\label{sauce-gratin}}

\index{そーす@ソース!くらたん@---・グラタン}
\index{くらたん@グラタン!そーす@ソース・---}
\index{sauce@sauce!gratin@--- Gratin}
\index{gratin@gratin!sauce@Sauce ---}

白ワインと、このソースを合わせる魚のアラなどでとった\href{fumet-de-poisson}{魚のフュ
メ}各3 dlにエシャロットのみじん切り大さじ
1\(\sfrac{1}{2}\)杯を加え、半量以下になるまで煮詰める。

\href{}{デュクセル・セッシュ}大さじ3杯と、\protect\hyperlink{sauce-espagnole-maigre}{魚料理用ソース・エスパニョ
ル}または\protect\hyperlink{sauce-demi-glace}{ソース・ドゥミグラ ス}5
dlを加える。5〜6分間煮立たせる。提供直前に、パ
セリのみじん切り大さじ\(\sfrac{1}{2}\)を加えて仕上げる。

\ldots{}\ldots{}舌びらめ、メルラン\footnote{タラの近縁種。}、バルビュ\footnote{鰈の近縁種。この場合のフィレはいわゆる「五枚おろし」にしたもの。}のフィレなどのグラタン用。

\maeaki

\hypertarget{ux30bdux30fcux30b9ux30a2ux30b7ux30a743}{%
\subsubsection[ソース・アシェ]{\texorpdfstring{ソース・アシェ\footnote{細かく刻んだもの、の意。}}{ソース・アシェ}}\label{ux30bdux30fcux30b9ux30a2ux30b7ux30a743}}

\hypertarget{sauce-hachee}{%
\paragraph{Sauce Hachée}\label{sauce-hachee}}

\index{そーす@ソース!あしぇ@---・アシェ}
\index{sauce@sauce!hachee@--- Hach\'ee}

玉ねぎの細かいみじん切り100gと、エシャロットの細かいみじん切り大さじ
1\(\sfrac{1}{2}\)杯をバターで色付かないよう炒める。ヴィネガー3
dlを注ぎ、 半量まで煮詰める。{[}ソース・エスパニョル{]}4 dlと{[}トマト
ソース{]}1\(\sfrac{1}{2}\) dlを加える。5〜6分煮立たせる。

ハムの脂身のない部分を細かく刻んだもの大さじ1\(\sfrac{1}{2}\)杯と小ぶり
のケイパー大さじ1\(\sfrac{1}{2}\)杯、{[}デュクセル・セッシュ{]}大さじ
1\(\sfrac{1}{2}\)杯、パセリのみじん切り大さじ\(\sfrac{1}{2}\)杯を加えて仕
上げる

\ldots{}\ldots{}このソースは{[}ソース・ピカント{]}と等価のものと考えていい。用途も同じ。

\maeaki

\hypertarget{ux9b5aux6599ux7406ux7528ux30bdux30fcux30b9ux30a2ux30b7ux30a7}{%
\subsubsection{魚料理用ソース・アシェ}\label{ux9b5aux6599ux7406ux7528ux30bdux30fcux30b9ux30a2ux30b7ux30a7}}

\hypertarget{sauce-hachee-maigre}{%
\paragraph{Sauce Hachée maigre}\label{sauce-hachee-maigre}}

上記と同様に、玉ねぎとエシャロットを色付かないようバターで炒め、ヴィネ
ガーを注いで煮詰める。

魚の\href{}{クールブイヨン}5
dlを注ぎ、\protect\hyperlink{roux-brun}{茶色いルー}45 gまたはブー
ルマニエ50 gでとろみを付ける。弱火で8〜10分間煮込む。

提供直前に、細かく刻んだハーブミックス大さじ1杯と{[}デュクセル・セッシュ{]}大
さじ1\(\sfrac{1}{2}\)杯、小粒のケイパー大さじ1\(\sfrac{1}{2}\)杯、アンチョ
ヴィソース大さじ\(\sfrac{1}{2}\)杯とバター60 g、または80〜100
gのアンチョ ヴィバターを加えて仕上げる。

\ldots{}\ldots{}エイのような、あまり高級ではない魚のブイイ\footnote{茹でた肉、魚のこと。}用。

\maeaki

\hypertarget{ux30bdux30fcux30b9ux30e6ux30b5ux30ebux30c951}{%
\subsubsection[ソース・ユサルド]{\texorpdfstring{ソース・ユサルド\footnote{もとはハンガリーで農家20戸につき1人の割合で招集された騎兵
  hussard を指す。この語は16世紀まで遡ることが出来るが、のちに「乱暴
  者」といったニュアンスでも使われるようになった。à la hussarde は
  「乱暴に、粗野に」の意味でも用いられるが、料理においてはレフォール
  を使ったものに名付けられることが多い。}}{ソース・ユサルド}}\label{ux30bdux30fcux30b9ux30e6ux30b5ux30ebux30c951}}

\hypertarget{sauce-hussarde}{%
\paragraph{Sauce Hussarde}\label{sauce-hussarde}}

\index{そーす@ソース!ゆさるど@---・ユサルド}
\index{ゆさるど@ユサルド!そーす@ソース・---}
\index{sauce@sauce!hussarde@--- Hussarde}
\index{hussarde@Hussarde!sauce@Sauce ---}

玉ねぎ2個とエシャロット2個を細かくみじん切りにして、バターで色よく炒め
る。白ワイン4
dlを注ぎ、半量になるまで煮詰める。\protect\hyperlink{sauce-demi-glace}{ソース・ドゥミグラ
ス}4
dlとトマトピュレ大さじ2杯、\protect\hyperlink{fonds-blanc-ordinaire}{白いフォ
ン}2 dl、生ハムの脂身のないところ80 g、潰したに
んにく1片、ブーケガルニを加える。弱火で25〜30分煮込む。

ハムを取り出して、ソースをスプーンで押すようにして布で漉す。

火にかけて温め、小さなさいの目\footnote{brunoise ブリュノワーズ。}に刻んだハムと、おろしたレフォール
\footnote{raifort いわゆる西洋わさび。}少々、パセリのみじん切りをたっぷり1つまみ加えて仕上げる。

\ldots{}\ldots{}牛、羊肉のグリルまたは串を刺してローストしてアントレ\footnote{通常、大きな塊肉などに串を刺してローストするのは料理区分として
  アントレに含められることはないが、このソースを用いる「牛フィレ ユ
  サルド」は牛フィレの塊に串を刺してローストし、ポム・デュシェスとマッ
  シュルームを合わせるが、本書ではアントレに分類されている。}として供する際に用いる。

\maeaki

\hypertarget{ux30a4ux30bfux30eaux30a2ux98a8ux30bdux30fcux30b9}{%
\subsubsection{イタリア風ソース}\label{ux30a4ux30bfux30eaux30a2ux98a8ux30bdux30fcux30b9}}

\hypertarget{sauce-italienne}{%
\paragraph{Sauce Italienne}\label{sauce-italienne}}

\index{そーす@ソース!いたりあふう@イタリア風---}
\index{いたりあん@イタリアン!いたりあふうそーす@イタリア風ソース}
\index{いたりあふう@イタリア風!そーす@---ソース}
\index{sauce@sauce!Italienne@--- Italienne}
\index{italien@italien!sauce italienne@Sauce Italienne}

トマトの風味の効いた\protect\hyperlink{sauce-demi-glace}{ソース・ドゥミグラス}\(\sfrac{3}{4}\)
Lに、\href{}{デュ
クセル・セッシュ}大さじ4杯と、加熱ハムの脂身のないところを小さなさいの
目に切ったもの125 gを加える。5〜6分間煮る。提供直前に、パセリとセルフ
イユ、エスゴラゴンのみじん切り大さじ1杯を加えて仕上げる。

\ldots{}\ldots{}いろいろな肉料理に合わせる。

\hypertarget{ux539fux6ce8-4}{%
\subparagraph{【原注】}\label{ux539fux6ce8-4}}

このソースを魚料理に合わせる場合、ハムは使わずに\protect\hyperlink{fumet-de-poisson}{魚のフュメ}を煮詰めて
加える。

\maeaki

\hypertarget{ux3068ux308dux307fux3092ux4ed8ux3051ux305fux30b8ux30e5ux30a8ux30b9ux30c8ux30e9ux30b4ux30f3ux98a8ux5473}{%
\subsubsection{とろみを付けたジュ エストラゴン風味}\label{ux3068ux308dux307fux3092ux4ed8ux3051ux305fux30b8ux30e5ux30a8ux30b9ux30c8ux30e9ux30b4ux30f3ux98a8ux5473}}

\hypertarget{jus-lie-a-lestragon}{%
\paragraph{Jus lié à l'Estragon}\label{jus-lie-a-lestragon}}

\index{そーす@ソース!とろみをつけたしゅえすとらごん@とろみを付けたジュ エストラゴン風味}
\index{しゅ@ジュ!えすとらごん@とろみを付けた--- エストラゴン風味}
\index{sauce@sauce!jus lie a l'estragon@Jus lié à  l'Estragon}
\index{estragon@estragon!jus lie a l'estragon@Jus lié à l'Estragon}
\index{jus@jus!estragon@--- lié à l'Estragon}

\protect\hyperlink{fonds-de-veau-brun}{仔牛のフォン}または\protect\hyperlink{fonds-de-volaille}{鶏のフォン}に、エストラゴン50gを加えて香りを煮出
し\footnote{imfuser アンフュゼ。}たもの。

布で漉してから、アロールート\footnote{コーンスターチで代用する。}または、でんぷん30
gでとろみを付ける。

\ldots{}\ldots{}白身肉のノワゼットや家禽のフィレなどに添える。

\maeaki

\hypertarget{ux3068ux308dux307fux3092ux4ed8ux3051ux305fux30b8ux30e5ux30c8ux30deux30c8ux98a8ux5473}{%
\subsubsection{とろみを付けたジュ トマト風味}\label{ux3068ux308dux307fux3092ux4ed8ux3051ux305fux30b8ux30e5ux30c8ux30deux30c8ux98a8ux5473}}

\hypertarget{jus-lie-tomate}{%
\paragraph{Jus lié tomaté}\label{jus-lie-tomate}}

\index{そーす@ソース!じゅとまといり@とろみを付けたジュ トマト入り}
\index{じゅ@ジュ!とまといり@とろみを付けた--- トマト入り}
\index{sauce@sauce!jus lie tomateq@Jus lié tomaté}
\index{tomate@tomate!jus lie tomate@Jus lié tomaté}
\index{jus@jus!tomate@--- lié tomaté}

\protect\hyperlink{fonds-de-veau-brun}{仔牛のフォン}1
Lあたりトマトエッセンス3 dlを加え、\(\sfrac{4}{5}\)量ま で煮詰める。

\ldots{}\ldots{}牛、羊肉料理用。

\maeaki

\hypertarget{ux30eaux30e8ux30f3ux98a8ux30bdux30fcux30b9}{%
\subsubsection{リヨン風ソース}\label{ux30eaux30e8ux30f3ux98a8ux30bdux30fcux30b9}}

\hypertarget{sauce-lyonnaise}{%
\paragraph{Sauce Lyonnaise}\label{sauce-lyonnaise}}

\index{そーす@ソース!りよんふう@リヨン風---}
\index{りよんふう@リヨン風!りよんふうそーす@---ソース}
\index{sauce@sauce!lyonnaise@--- Lyonnaise}
\index{liyonnais@lyonnais!sauce lyonnaise@Sauce Lyonnaise}

中位の大きさの玉ねぎ3個をみじん切りにし、バターでじっくり、ごく弱火で
ブロンド色になるまで炒める。白ワイン2 dlとヴィネガー2 dlを注ぐ。
\(\sfrac{1}{3}\)量まで煮詰め、\protect\hyperlink{sauce-demi-glace}{ソース・ドゥミグラ
ス}\(\sfrac{3}{4}\) Lを加える。
5〜6分かけて表面に浮いてくる不純物を丁寧に取り除き\footnote{dépouiller
  デプイエ。現代ではエキュメと呼ぶ現場が多い。}、布で漉す。

\hypertarget{ux539fux6ce8-5}{%
\subparagraph{【原注】}\label{ux539fux6ce8-5}}

このソースを合わせる料理によっては、ソースを布で漉さずに玉ねぎを残してもいい。

\maeaki

\hypertarget{ux30bdux30fcux30b9ux30deux30c7ux30fcux30eb}{%
\subsubsection{ソース・マデール}\label{ux30bdux30fcux30b9ux30deux30c7ux30fcux30eb}}

\hypertarget{sauce-madere}{%
\paragraph{Sauce Madère}\label{sauce-madere}}

\index{そーす@ソース!までーる@---・マデール}
\index{までいら@マデイラ!そーすまでーる@ソース・マデール}
\index{sauce@sauce!madere@--- Madère}
\index{madere@madère!sauce madere@Sauce Madère}

\protect\hyperlink{sauce-demi-glace}{ソース・ドゥミグラス}を煮詰め\footnote{ソース・ドゥミグラスは既に煮詰めて仕上がった状態のものなので、9割程度
  にまでしか煮詰めないことに注意。}、火から外して、ソー ス1
Lあたりマデラ酒1 dlの割合で加え、普通の濃度にする。

\maeaki

\hypertarget{ux30bdux30fcux30b9ux30deux30c8ux30edux30c3ux30c854}{%
\subsubsection[ソース・マトロット]{\texorpdfstring{ソース・マトロット\footnote{水夫風、船員風、の意。}}{ソース・マトロット}}\label{ux30bdux30fcux30b9ux30deux30c8ux30edux30c3ux30c854}}

\hypertarget{sauce-matelote}{%
\paragraph{Sauce Matelote}\label{sauce-matelote}}

\index{そーす@ソース!まとろつと@---・マトロット}
\index{まとろつと@マトロット!そーすまとろつと@ソース・---}
\index{sauce@sauce!matelote@--- Matelote}
\index{matelote@matelote!sauce matelote@Sauce Matelote}

魚をポシェするのに使った\href{}{赤ワイン入りの魚用クールブイヨン}3
dlにマッシュルームの切りくず25 g
を加え、\(\sfrac{1}{3}\)量になるまで煮詰める。

煮詰めたら\protect\hyperlink{sauce-espagnole-maigre}{魚料理用ソース・エスパニョル}8
dl を加えてひと煮立ちさせる。布で漉し、バター150 gとごく少量のカイエンヌ
の粉末を加えて仕上げる。

\maeaki

\hypertarget{ux30bdux30fcux30b9ux30e2ux30efux30eb}{%
\subsubsection{ソース・モワル}\label{ux30bdux30fcux30b9ux30e2ux30efux30eb}}

\hypertarget{sauce-moelle}{%
\paragraph[Sauce Moelle]{\texorpdfstring{Sauce Moelle\footnote{骨髄のこと。}}{Sauce Moelle}}\label{sauce-moelle}}

\index{そーす@ソース!もわる@---・モワル}
\index{こつずい@骨髄!そーすもわる@ソース・モワル}
\index{sauce@sauce!moelle@--- Moelle}
\index{moelle@moelle!sauce moelle@Sauce ---}

ソースの作り方は\protect\hyperlink{sauce-bordelaise}{ボルドー風ソース}とまったく同じだ
が、バターを加えるのは何らかの野菜料理に添える場合のみであり、その場合
のバターの量は通常どおりとするこ。

どんな場合にせよ、仕上げに、小さなさいの目に切ってポシェしておいた骨髄
をソース1 Lあたり150〜180 gおよび刻んで下茹でしたパセリの葉小さじ1杯を
加える。

\maeaki

\hypertarget{ux30e2ux30b9ux30afux30efux98a856ux30bdux30fcux30b9}{%
\subsubsection[モスクワ風ソース]{\texorpdfstring{モスクワ風\footnote{モスクワ風の名称を持つ料理や菓子は多い。
  18世紀後半から19世紀前
  半にかけて、ロシアの宮廷や貴族らの間でフランスの食文化が流行し、多
  くのフランス人料理人が招聘され、彼らはロシア料理のレシピをフランス
  に持ち帰った。クーリビヤックなどが代表的な例だろう。また、19世紀後
  半になると、とりわけフランス料理においてもロシア料理からの影響が多
  く見られるようになる。キャビアとウォトカを食前に愉しむのが流行した
  のもその時代からである。フランスとロシアの食文化は相互に影響関係に
  あったと言えよう。}ソース}{モスクワ風ソース}}\label{ux30e2ux30b9ux30afux30efux98a856ux30bdux30fcux30b9}}

\hypertarget{sauce-moscovite}{%
\paragraph{Sauce Moscovite}\label{sauce-moscovite}}

\index{そーす@ソース!もすくわふう@モスクワ風---}
\index{もすくわふう@モスクワ風!そーす@---ソース}
\index{sauce@sauce!moscovite@--- Moscovite}
\index{moscovite@moscovite!sauce moscovite@Sauce ---}
