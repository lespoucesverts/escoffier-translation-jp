\setlength{\parindent}{1\zw}

\vspace*{4\zw}

\begin{main}

\hypertarget{preface-heston-blumenthal}{%
\section{ヘストン・ブルメンタールによる英訳第5版への序文}\label{preface-heston-blumenthal}}

\vspace{2\zw}
\thispagestyle{empty}

僕は独学でシェフになった。16才の頃、フランスを旅してミシュランの星付きレストランを訪れ、そのときの体験にすっかり魅せられた\ldots{}\ldots{}それは美味しい料理だけじゃなく、レストランの情景やいろんな音、匂い、それに料理をまるで芝居の情景のごとくプレゼンテーションする様子にもだ。僕は一瞬にして、自分がやりたいのはこういうことなんだと理解した。

そんなわけで、学校を辞めて、僕独自の奇妙ともいえる料理修業をはじめた。ミシュランのガイドブックやゴミヨの本を夢中になって読み、年に一度はフランスに行って、それらの本で賞賛されている店に行って食事をし、『フランスグルメツアー』や『ラルース・ガストロノミック』に出ているクラシカルなフランス料理を何度も何度もくりかえし作ってみた。

そしてもちろん、このエスコフィエ『料理の手引き』も。

オーギュスト・エスコフィエが19世紀末から20世紀初頭にかけて、もっとも偉大かつ急進派のシェフだったのは疑うべくもない。\protect\hyperlink{homard-americaine}{オマール・アメリケーヌ}\footnote{オマール・アメリケーヌという料理そのものはエスコフィエの創作によるものではないという点では事実誤認と言える。ただし、これをソースとガルニチュールに分離して舌びらめ等の魚料理に添えるものと位置付けを変えたという点でエスコフィエは画期的といもいえる改良を加えている。\protect\hyperlink{sauce-americaine}{ソース・アメリケーヌ}、\protect\hyperlink{garniture-americaine}{ガルニチュール・アメリケーヌ}および各訳注参照。}や\protect\hyperlink{peches-melba}{ピーチメルバ}のような新しい料理を無数に創作すると同時に、オートキュイジーヌの原則全体を見直して再解釈し、決まりごとのやかましいガルニチュールや重たいソース、やたらと仕事の手間のかかる派手な盛り付けなどは廃したり正していったわけだ。(エスコフィエの口癖のひとつに「シンプルに作れ」というのがあった。そう、まさにその通りの意味にすべきなんだ\footnote{{[}「第二版序文」とりわけp.vii{]}参照。})。エスコフィエの関心が厨房の中にとどまらずすごい広がりを持っていたことに、僕はいつもインスパイアされていると言っていい。エスコフィエは食品科学、とりわけ食品保存について関心を持っていたんだ。トマト缶詰製造やマギーブイヨンの開発にも関わり、自らのブランドで瓶詰めのソースやピクルスも開発した。

エスコフィエという人物は長い影響を及ぼしている。エスコフィエが最初に確立したやりかたで、現代のレストランはいまも運営されている。ブリガードと呼ばれる明確な命令系統のシステムを仕上げたことで、調理場での料理の作り方はまるっきり変わった。料理長の指示は部門シェフに伝わり、彼らから各料理人に伝えられる。それぞれの料理人は自分の担当している作業に責任を持つ、というわけだ。エスコフィエはまた、料理を食卓にお出しする方法も変えた。全部の料理をずらりと一度に食卓に並べる「フランス式サービス」をやめて、ロシア式サービスつまりコース料理が順を追って提供される方式を支持した。このやりかたはこんにちまでずっと続けられているんだ。僕たちは、オーギュストのおかげで食事をしているみたいなものなんだ。

エスコフィエの正確な指示と創意工夫は彼の『料理の手引き』のすべてのページに溢れている。エスコフィエは『料理の手引き』が「ただのレシピ集である以上に役に立つ本」となることを期待していると書いた。実際、まさしくその通りになっているわけだ。エスコフィエの細部にいたるまでの注意深さは伝説になっている。自分の顧客それぞれの食の好き嫌いを書き留め続けていたという。この『料理の手引き』の制作にあたって、エスコフィエ自身が助手に、わざわざこのために用意させた計量秤を送り付け、それぞれの材料の分量を実際に計らせたという。こんなにもレシピに対して綿密であることにこだわったということは、それらのレシピが実際に作ってとても楽しいものになった(これこそがフランス料理の基礎教育の凄いところだ)のはもちろんだが、そのおかげで、世界の超一流シェフのひとりに創造力=想像力についての深い洞察力を与えてくれたのは事実だ。

シェフたちのなかには、クラシカルな伝統なんてすっとばして、エスコフィエみたいな保守的で時代遅れのくせに偉そうな人物なんて否定したいという思いに駆られる者もいる。それはとんでもない間違いだ。エスコフィエの科学、テクノロジー、マーケティング、厨房の改革への関心、それに料理のプレゼンテーションをシンプルにすること、どれも僕としては、非常に現代的な問題だと思う。知れば知るほど、素晴しい料理というものは伝統をぶち壊すことによってではなく、伝統を新たな方向へ導くことによって生み出されるのだと分かったんだ。\ruby{革命}{レボリューション}じゃなく、むしろ\ruby{進化}{エボリューション}だよ。

\thispagestyle{empty}

そういう意味で、エスコフィエのこの『料理の手引き』は最高のスターティングポイントだと思う。

\vspace{1\zw}
\begin{flushright}
ヘストン・ブルメンタール 2011年
\end{flushright}

\end{main}
