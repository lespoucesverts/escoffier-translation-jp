\newpage

\begin{main}

\hypertarget{gelees-diverses}{%
\section{ジュレ}\label{gelees-diverses}}

\begin{frsecenv}

Gelées diverses

\end{frsecenv}

\index{gelee@gelée} \index{しゆれ@ジュレ}

どんなジュレも、ベースとなっているのはほぼ全てフォンだ。だから、フォンのメインとなっている素材によってジュレの風味が決まるわけだ。その結果としてジュレの用途も\ruby{自}{おの}ずと決まってくる。

人工的な凝固剤を使わずにジュレを確実に固めるためには、フォンのメインとなる素材に、仔牛の足や豚皮のようなゼラチン質の量を計算して加えることになる。仔牛の足や豚の皮を使えば、ジュレを確実に凝固させられるし、しかも柔らかな口あたりに仕上げられる。

そうはいっても、とりわけ夏季には、クラリフィエ\footnote{clarifier
  \textgreater{} clarification 次項参照。}の作業を行なう前に必ず、フォンを氷の上に垂らしてみて、固さと濃度を確認し、必要があれば板ゼラチンを何枚か加えてやること。

追加する板ゼラチンの量は、どんな場合でも、フォン1 Lあたり9
g(6枚)を越えないこと。板ゼラチンは、透き通っていてぱりぱりと割れやすく、
\ruby{膠}{にかわ}っぽい味のしないものを選ぶこと。必ず冷水でもどしてから使うか、せめてよく洗ってから用いること。

標準的なジュレを作る際に人工着色料を使うことはお勧め出来ない。標準的なジュレは充分に色よく仕上がるものだ。さらに、最後にマデイラ酒を加えてやれば充分に、標準的なジュレの特徴ともいえる淡い琥珀色に仕上がる。

\end{main}

\begin{recette}

\hypertarget{fonds-pour-gelee-ordinaire}{%
\subsubsection[標準的なジュレ用のフォン]{\texorpdfstring{標準的なジュレ用のフォン\footnote{この項および次の「白いジュレ用のフォン」は初版と第二版以降の異同が大きい。この「標準的なジュレ用のフォン」は初版では使用する液体が
  8 litres et demi de remouillage
  いわゆる「二番のフォン」であり、加熱時間も6時間と短かい。第二版は「水8.5
  L」になるが、加熱時間は6時間のままで、作業手順が「ソース用の白いフォンと同じ」となっている。第三版で現在の記述となった。}}{標準的なジュレ用のフォン}}\label{fonds-pour-gelee-ordinaire}}

\begin{frsubenv}

Fonds pour gelée ordinaire

\end{frsubenv}

\index{gelee@gelée!fonds ordinaire@Fonds pour gelée ordinaire}
\index{しゆれ@ジュレ!ひようひゆんてきなしゆれようのふおん@標準的なジュレ用のフォン}

(仕上がり5 L分)

\begin{itemize}
\item
  主素材\ldots{}\ldots{}仔牛のすね肉とバモルソー\footnote{原文 bas
    morceaux 煮込みなどに用いる部位の総称。bas
    は「低い」が原義であり、食材として低級な部位というニュアンス。}2
  kg、細かく砕いた仔牛の骨1.5 kg、牛の脚肉1.5
  kg\ldots{}\ldots{}これらの肉と骨はオーブンで軽く色付けておくこと。
\item
  ゼラチン質\ldots{}\ldots{}骨を取り除いて\footnote{原文 désosser
    (デゾセ)骨を取り除く。}下茹でした\footnote{原文 blanchir
    (ブランシール)。下茹ですることがだ、原義は「白くする」。もとは中世において肉を調理する際にはローストであれ煮込みであれ、ほぼ必ず下茹でしていた。赤い肉を茹でると表面が白くなることからこの用語が定着することになったが、現在ではもっぱら野菜の下茹でなどについて言うことがほとんど。「ブランシェ」と言う現場もあるようだが、もとのフランス語からやや離れているので「ブランシール」で覚えるといいだろう。}仔牛の足3本、背脂を付けたままの生の豚皮\footnote{塩漬けなどの加工をしていない、ということ。}250
  g。
\item
  香味素材\ldots{}\ldots{}にんじん200 g、玉ねぎ200 g、ポワロー\footnote{poireau(x)
    ポロねぎ。日本の長葱とは異なり、植物としてはむしろ、にんにくに近い。葱と風味がかなり異なるため、見た目が似ているからと下仁田葱で「代用」するのはあまり好ましいとはいえないだろう。ポワローは古代ローマ時代からヨーロッパで広く親しまれてきた野菜のひとつであり、ローマ皇帝ネロが演説で大きな声を出すために、ポワローの蜂蜜漬けを好んだという逸話もある。伝統的な栽培方法の場合、旬は秋〜冬。播種から収穫まで10ヶ月以上かかる品種も多い。太さ3〜5
    cm、軟白部が20〜 40
    cmくらいのものが多い。フランスの標準的な規格では軟白部20
    cm以上。かつては日本の長葱と同様に成長に応じて「土寄せ」して栽培していたが、その方法では内部に土砂が入りやすい。現代のフランスではサヴォイキャベツやプチポワ同様に、大規模、機械化栽培が進んでいる品目のひとつ。また、太さ1
    cm程のミニ・ポワローも付け合わせ用の高級野菜として人気がある。元来ミニ・ポワローは苗の「間引き」を利用したものだったが、現在ではミニ・ポワローむけの品種も開発されている。日本にも秋〜冬季はヨーロッパ産が、春〜夏季はオーストラリア産が安定的に輸入されている。日本国内での生産も明治以降、試みられてはいるが、需給バランスとコスト的な問題から断念せざるを得ないケースも少なくないようだ。なお、第二次大戦前は八丈島などでこうした西洋野菜の栽培が行なわれ、船便で東京まで運ばれていたという(cf.~大木健二『大木健二の洋菜ものがたり』日本デシマル、1997年)。なお、現代フランス語でブレット(ふだんそう)のことを
    poirée (ポワレ)とも呼ぶが、これは ポワロー poireau
    と同語源。中世の料理書にはしばしば、野菜をペースト状になるまで煮込んだポタージュとして
    porée
    (ポレ)というものが出てくるが、どちらを材料として用いているか判別できないケースもある。ブレットはビーツともとは同じもので、16世紀頃に品種分化されたといわれており、bette(ベット=ビーツ)のrave(ラーヴ=根)がbetterave(ベトラーヴ)つまり現代フランス語でビーツを意味する語となり、betteはbête(獣、愚かな)と同じ発音であることが嫌われてblette(ブレット)と日常的に呼ばれるようになった。}50
  g、セロリ 50 g、充分な香りと量のブーケガルニ。
\item
  使用する液体\ldots{}\ldots{}水 8.5 L。
\item
  加熱時間\ldots{}\ldots{}6時間。
\item
  作業手順\ldots{}\ldots{}ソース用の\protect\hyperlink{fonds-brun}{茶色いフォン}とまったく同じ。ただし、ジュレ用のフォンの色合いはソース用のフォンよりも薄くしておくこと。
\end{itemize}

\atoaki{}

\hypertarget{fonds-pour-gelee-blanche}{%
\subsubsection[白いジュレ用のフォン]{\texorpdfstring{白いジュレ用のフォン\footnote{初版全文は「主素材、ゼラチン質、香味素材は上記のとおり。注ぐ液体は水(原文
  mouillage à
  blanc)、作業手順は基本の白いフォンと同様」。第二版で現在の記述となっている。この文脈からすると、白いフォンの二番を使うとも解釈され得るが、前項の「標準的なジュレ用のフォン」が最終的に水を用いて作ることになっているのと比較すると、加熱時間および作業手順が何と同様なのか曖昧になってしまうため、ここでは液体、加熱時間、作業手順を\protect\hyperlink{fonds-blanc}{標準的な白いフォン}と同じと解釈した。なお、英訳第5版では、but
  use very white stock instead of
  water「水ではなく白いフォン」を注ぐとなっている。}}{白いジュレ用のフォン}}\label{fonds-pour-gelee-blanche}}

\begin{frsubenv}

Fonds pour gelée blanche

\end{frsubenv}

\index{gelee@gelée!fonds ordinaire@Fonds pour gelée ordinaire}
\index{しゆれ@ジュレ!しろいしゆれようのふおん@白いジュレ用のフォン}

主素材、ゼラチン質、香味素材の種類と分量は前記の\protect\hyperlink{fonds-pour-gelee-ordinaire}{標準的なジュレ用のフォン}を参照。

使用する液体の量は\protect\hyperlink{fonds-blanc}{標準的な白いフォン}とまったく同じにすること。

加熱時間も作業手順も同様。

\atoaki{}

\hypertarget{fonds-pour-gelee-de-volaille}{%
\subsubsection{鶏のジュレ用のフォン}\label{fonds-pour-gelee-de-volaille}}

\begin{frsubenv}

Fonds pour gelée de volaille

\end{frsubenv}

\index{gelee@gelée!fonds volaille@Fonds pour gelée de volaille}
\index{しゆれ@ジュレ!とりのしゆれようのふおん@鶏のジュレ用のフォン}
\index{しゆれ@ジュレ!うおらいゆのしゆれようのふおん@ヴォライユのジュレ用のフォン ⇒ 鶏のジュレ用のフォン}
\index{とり@鶏!しゆれようのふおん@---のジュレ用のフォン}
\index{うおらいゆ@ヴォライユ!しゆれようのふおん@ヴォライユのジュレ用のフォン ⇒ 鶏のジュレ用のフォン}

(仕上がり5 L分)

\begin{itemize}
\item
  主素材\ldots{}\ldots{}仔牛のすね肉1.5 kg、牛の脚肉1.5
  kg、細かく砕いた仔牛の骨1.5
  kg、鶏ガラ、とさか、手羽先、足など(とりわけ湯通しした手羽と足)、1.5
  kg。
\item
  ゼラチン質\ldots{}\ldots{}骨を取り除いて下茹でした仔牛の足(小)3本。
\item
  香味素材\ldots{}\ldots{}材料の種類は標準的なジュレ用のフォンと同じだが、量はやや少なめにすること。
\item
  使用する液体\ldots{}\ldots{}軽く仕上げた\protect\hyperlink{fonds-blanc}{白いフォン}
  8 L。
\item
  加熱時間\ldots{}\ldots{}4時間半。
\item
  作業手順\ldots{}\ldots{}ソース用の\protect\hyperlink{fonds-de-volaille}{鶏のフォン}とまったく同じ。
\end{itemize}

\atoaki{}

\hypertarget{fonds-pour-gelee-de-gibier}{%
\subsubsection{ジビエのジュレ用のフォン}\label{fonds-pour-gelee-de-gibier}}

\begin{frsubenv}

Fonds pour gelée de gibier

\end{frsubenv}

\index{gelee@gelée!fonds gibier@Fonds pour gelée de gibier}
\index{gibierl@gibier!gelee@gelée!fonds@Fonds pour gelée de ---}
\index{しゆれ@ジュレ!しひえのしゆれようのふおん@ジビエのジュレ用のフォン}
\index{しひえ@ジビエ!しゆれ@ジュレ!ふおん@---のジュレ用のフォン}

(仕上がり5 L分)

\begin{itemize}
\item
  主素材\ldots{}\ldots{}仔牛のすね肉1 kg、牛の脚肉2 kg、仔牛の骨750
  g、ジビエのガラやバモルソー\footnote{\protect\hyperlink{fonds-pour-gelee-ordinaire}{標準的なジュレ用のフォン}訳注参照。}1.75
  kg。これらはすべてオーブンで焼いて色付けておくこと。
\item
  ゼラチン質\ldots{}\ldots{}\protect\hyperlink{fonds-pour-gelee-de-volaille}{鶏のジュレ用のフォン}と同じ。
\item
  香味素材\ldots{}\ldots{}材料の種類は標準的なジュレ用のフォンと同じだが、セロリとタイムを
  \(\frac{1}{3}\)量多くすること。ジュニパーベリー\footnote{セイヨウネズの実。ジンの香りを特徴付けているもの。}7〜8粒を追加すること。
\item
  使用する液体\ldots{}\ldots{}水8 L。
\item
  加熱時間\ldots{}\ldots{}4時間。
\item
  作業手順\ldots{}\ldots{}ソース用の\protect\hyperlink{fonds-de-gibier}{ジビエのフォン}とまったく同じ。
\end{itemize}

\atoaki{}

\hypertarget{fonds-de-poisson-pour-gelee-ordinaire}{%
\subsubsection{標準的なジュレ用の魚のフォン}\label{fonds-de-poisson-pour-gelee-ordinaire}}

\begin{frsubenv}

Fonds de poisson pour gelée ordinaire

\end{frsubenv}

\index{gelee@gelée!fonds poisson@Fonds de poisson pour gelée ordinaire}
\index{poisson@poisson!gelee@gelée!fonds@Fonds de --- pour gelée ordinaire}
\index{しゆれ@ジュレ!ひようしゆんてきなしゆれようのさかなのふおん@標準的なジュレ用の魚のフォン}
\index{さかな@魚!しゆれようのふおん@標準的なのジュレ用の---のフォン}

(仕上がり5 L分)

\begin{itemize}
\item
  主素材\ldots{}\ldots{}グロンダン\footnote{ホウボウ科の魚。和名カナガシラ。}、ヴィーヴ\footnote{ハチミシカ科の海水魚の総称。}、メルラン\footnote{鱈の近縁種。}などの安い魚750
  g、舌びらめのアラと端肉750 g。
\item
  香味素材\ldots{}\ldots{}薄切りにした\footnote{émincer エマンセ。}玉ねぎ200
  g、パセリの根2本、フレッシュなマッシュルームの切りくず100 g。
\item
  使用する液体\ldots{}\ldots{}やや薄めで透き通った仕上がりの魚のフュメ6
  L。
\item
  加熱時間\ldots{}\ldots{}45分間。
\item
  作業手順\ldots{}\ldots{}\protect\hyperlink{fumet-de-poisson}{魚のフュメ}と同じ。
\end{itemize}

\atoaki{}

\hypertarget{fonds-pour-gelee-de-poisson-au-vin-rouge}{%
\subsubsection{赤ワインを用いた魚のジュレ用のフォン}\label{fonds-pour-gelee-de-poisson-au-vin-rouge}}

\begin{frsubenv}

Fonds pour gelée de poisson au vin rouge

\end{frsubenv}

\index{gelee@gelée!fonds poisson rouge@Fonds pour gelée de poisson au vin rouge}
\index{poisson@poisson!gelee@gelée!fonds rouge@Fonds pour gelée de poisson au vin rouge}
\index{しゆれ@ジュレ!あかわいんをもちいたさかなのしゆれようのふおん@赤ワインを用いた魚のジュレ用のフォン}
\index{さかな@魚!しゆれようのふおんあかわいん@赤ワインを用いた---のジュレ用のフォン}

このフォンは通常、鯉やトラウトなどの魚料理に用いられる。

このフォンに使用する液体は、良質なブルゴーニュ産赤ワインと\protect\hyperlink{fumet-de-poisson}{魚のフュメ}を同量ずつにする。魚のフュメは、ジュレが確実に固まるよう、ゼラチン質が多めのものを用いること。

風味付けは、魚に火を通すのに使った香味野菜によるもので充分だ。

\atoaki{}

\hypertarget{observation-sur-l-emplois-des-fonds-destines-aux-gelees}{%
\subsubsection{ジュレ用のフォンについての注意}\label{observation-sur-l-emplois-des-fonds-destines-aux-gelees}}

\ldots{}\ldots{}ジュレ用のフォンは出来るだけ、使用する前日に仕込んでおくこと。いい具合に煮込んだら、浮き脂を取り除き\footnote{dégraisser
  デグレセ。}、漉してから陶製の容器に入れて冷ます。

冷めるとフォンは凝固する。取り除ききれなかったごくわずかな脂が表面に浮いてくるが、板状に固まるので容易に取り除くことが出来る。布あるいは漉し器でフォンを漉した際にすり抜けてしまった堆積物も自重で容器の底に沈むので、フォンを完全に澄ませることが出来る。

\end{recette}

\newpage

\begin{main}

\hypertarget{clarfication-des-gelees}{%
\subsection[ジュレのクラリフィエ]{\texorpdfstring{ジュレのクラリフィエ\footnote{clarification
  (クラリフィカスィオン)澄ませること、透明にさせること、の意の名詞だが、(1)本文にあるように、ただ単に「澄ませる」だけではなく、風味を補ったり強化し、色合いを調節する作業も兼ねていること、(2)現代日本の調理現場ではフランス語の動詞
  clarifier
  をカタカナにして「クラリフィエ」と呼ぶケースが多いことなどを考慮して、カタカナで動詞形のクラリフィエとした。なお、「クラリフェ」と呼ぶ現場もあるようだが、もとのフランス語がclarif\textbf{i}erとiの音があるのでこれは許容しがたい。}}{ジュレのクラリフィエ}}\label{clarfication-des-gelees}}

\begin{frsecbenv}

Clarification des Gelées

\end{frsecbenv}

\index{gelee@gelée!clarification@Clarification des ---s}
\index{clarification@clarification!gelee@--- des gelées}
\index{しゆれ@ジュレ!くらりふいえ@---のクラリフィエ}
\index{くらりふいえ@クラリフィエ!しゆれ@ジュレの---}

\end{main}

\begin{recette}

\hypertarget{gelees-ordinaires}{%
\subsubsection[標準的なジュレ]{\texorpdfstring{標準的なジュレ\footnote{このgrasse
  \textless{} gras
  は「脂気のある、太った」の意ではなく、カトリックにおける「小斉」の食事を
  maigre
  と表現することと対になっているもの。すなわち「小斉ではない通常の」の意であることに注意。小斉については\protect\hyperlink{sauce-espagnole-maigre}{魚料理用ソース・エスパニョル}訳注および\protect\hyperlink{sauce-laguipiere}{ソース・ラギピエール}訳注参照。}}{標準的なジュレ}}\label{gelees-ordinaires}}

\begin{frsubenv}

Gelées grasses ordinaires

\end{frsubenv}

\index{gelee@gelée!grasses ordinaire@---s grasses ordinaires}
\index{clarification@clarification!gelees grasses ordinaire@gelées  grasses ordinaires}
\index{しゆれ@ジュレ!ひようしゆんてきな@標準的な---}
\index{くらりふいえ@クラリフィエ!ひようしゆんてきな@標準的なジュレ}

(仕上がり5 L分)

\begin{enumerate}
\def\labelenumi{\arabic{enumi}.}
\item
  まずフォンの濃度を確認する。必要に応じて追加すべきゼラチンの量を調整する。
\item
  ジュレ用のフォンは充分に浮き脂を取り除き\footnote{dégraisser
    デグレセ。}、沈殿物も取り除い\footnote{décanter デカンテ。}てあること。
\item
  厚手で適切な大きさの片手鍋\footnote{casserole カスロール。}に、細挽き\footnote{ミートチョッパーやフードプロセッサが一般化する以前はアショワールhachoirという、両側に柄の付いた刃が湾曲した専用の包丁で細かく刻んでいた。}にした脂身のない\footnote{ここで原文はmaigreを用いているが、これはもちろん「脂気のない」の意。}赤身の牛肉
  500 gとセルフイユとエストラゴン計10 g、卵白3個分を入れる。
\item
  冷たい、あるいは生温い状態のジュレ用のフォンを挽肉の上から入れ、泡立て器かヘラで混ぜる。\\
  ゆっくり混ぜながら、強過ぎない程度の火加減で沸騰させる。卵白に含まれるアルブミンの分子が澄ませる作用を持っているので\footnote{やや大雑把な説明になるが、液体中に浮遊している不純物を抱き込むかたちで卵白が熱変性により凝固する、その結果として液体を「澄ませる」ことになる。ただし、これだけだと液体の味そのものや風味が薄くなってしまうために、それを補うあるいは強化する意味で挽肉や香草、香り付けの酒類を加える、ということ。}、混ぜることで卵白がまんべんなく広がるようにするわけだ。\\
  15分程、微沸騰の状態を保ち、目の詰まった布で漉す。
\end{enumerate}

\hypertarget{nota-gelees-grasses-ordinaires}{%
\subparagraph{【原注】}\label{nota-gelees-grasses-ordinaires}}

ジュレに酒類を添加するのは、ほぼ冷めた状態になってからにするのがいい。クラリフィエの作業中に酒類を加えるのは、沸騰しているために味が悪くなってしまうので、致命的な誤りでさえある。

そうではなく、ほぼ冷めた状態のジュレに酒類を添加すれば、その香気はそのまま保たれることになる。

作業の最後に酒類をジュレに添加すればジュレを薄めてしまう結果になるわけだからそれを考慮して、添加する酒類の量によっては、あらかじめジュレを充分に固めに作っておくのがいい。そうすれば、ジュレが固まるのに充分なゼラチンの濃度を保てるわけだ。

マデイラ酒、マルサラ酒、シェリー酒を加える場合の分量はジュレ1 Lあたり1
dLとすること。

ライン産のワインやシャンパーニュ、銘醸白ワインを加える場合は、ジュレ1
Lあたり2
dLとすること。加える酒類がどんなものであっても、文句ない程に良質のものを用いるべきだ。質の悪い酒類を加えてジュレの仕上がりを台無しにしてしまうくらいなら、加えないほうがまだましと言える。

\atoaki{}

\hypertarget{gelee-de-volaille}{%
\subsubsection{鶏のジュレ}\label{gelee-de-volaille}}

\begin{frsubenv}

Gelée de volaille

\end{frsubenv}

\index{gelee@gelée!volaille@--- de volaille}
\index{clarification@clarification!gelee volaille@gelée de volaille}
\index{しゆれ@ジュレ!とりのしゆれ@鶏の---}
\index{くらりふいえ@クラリフィエ!とりのしゆれ@鶏のジュレ}
\index{しゆれ@ジュレ!うおらいゆの@ヴォライユの---}
\index{くらりふいえ@クラリフィエ!うおらいゆのしゆれ@ヴォライユのジュレ}

鶏のジュレのクラリフィエは標準的なジュレの場合とまったく同じに行なう。香味素材(セルフイユとエストラゴン)、澄ませるための材料(卵白)も同様にする。

ただし、味の補強に用いる肉については変更すること。すなわち牛の赤身肉を半量にして、残り半量は鶏の首肉にする。つまり、牛肉250
gと鶏の首肉250 g の挽肉を用いる。

\hypertarget{nota-gelee-de-volaile}{%
\subparagraph{【原注】}\label{nota-gelee-de-volaile}}

鶏のローストのガラを粗く砕いてエチューヴ\footnote{食品の乾燥などに主に用いられる低温のオーブンの一種。}でよく乾燥させて脂気を抜いたものを、このクラリフィエの際に加えると、素晴しい結果が得られる。

\atoaki{}

\hypertarget{gelee-de-gibier}{%
\subsubsection{ジビエのジュレ}\label{gelee-de-gibier}}

\begin{frsubenv}

Gelée de gibier

\end{frsubenv}

\index{gelee@gelée!gibier@--- de gibier}
\index{clarification@clarification!gelee gibier@gelée de gibier}
\index{しゆれ@ジュレ!しひえ@ジビエの---}
\index{くらりふいえ@クラリフィエ!しひえのしゆれ@ジビエのジュレ}

クラリフィエの作業のやり方はまったく同様。ただし、このジュレを作る際には、いくつか留意すべきポイントがある。

標準的なジビエのジュレ、つまり特有の風味を持たせないものの場合は、味の補強には牛の挽肉250
gとジビエの赤身の挽肉250 gを用いること。

ジュレに独特の香りを持たせる必要がある場合には、必ず、肉それ自体に香気のあるジビエの肉、すなわち、ペルドロー、雉、ジェリノット\footnote{gélinotte
  雷鳥の一種。}などをクラリフィエの際に用いること。

どんなジビエのジュレでも仕上げに、ジュレ1
Lあたり大さじ2杯の上等なコニャックを加える。ただし、コニャックは絶対に良質のものでなければいけない。平凡なコニャックしか使えないのなら、これは省いたほうがいい。

この香り付けをしなくても、ジュレは不完全なものとはいえ、一応使えるものになる。いっぽうで、ありきたりのコニャックで香り付けすると、美味しくは仕上がらない。

\atoaki{}

\hypertarget{gelee-de-poisson-blanche}{%
\subsubsection{魚の白いジュレ}\label{gelee-de-poisson-blanche}}

\begin{frsubenv}

Gelée de poisson blanche

\end{frsubenv}

\index{gelee@gelée!poisson blanche@--- de poisson blanche}
\index{clarification@clarification!gelee poisson blanche@gelée de poisson blanche}
\index{しゆれ@ジュレ!さかなのしろい@魚の白い---}
\index{くらりふいえ@クラリフィエ!さかなのしろいしゆれ@魚の白いジュレ}

魚のジュレのクラリフィエは以下のとおり\footnote{(1)または(2)の方法をとる、と解釈していいだろう。}。

\begin{enumerate}
\def\labelenumi{\arabic{enumi}.}
\item
  卵白を使う場合、ジュレ5
  Lあたり卵白3個分に、クラリフィエによって薄まってしまうのを補うためにメルランの身を細かく刻んだもの250
  gを加える。
\item
  もし可能なら新鮮なキャビア、なければ圧縮キャビア\footnote{\protect\hyperlink{beurre-de-caviar}{キャビアバター}訳注参照。}をジュレ1
  Lあたり50
  g用いる。方法は魚のコンソメのクラリフィエで説明している\footnote{概要は、キャビアをピュレ状にすり潰し、冷たい魚のコンソメでのばして加える。火にかけて絶えず混ぜながら沸かし、微沸騰の状態を20分保った後、布で漉す、という方法。}
  (ポタージュの章を参照)。
\end{enumerate}

魚のジュレの香り付けには、辛口のシャンパーニュもしくはブルゴーニュの銘醸白ワインを用いるといいが、\protect\hyperlink{gelees-ordinaires}{標準的なジュレ}の注において説明した酒類を加える場合の注意事項を勘案すること。

\hypertarget{nota-gelee-de-poisson-blanche}{%
\subparagraph{【原注】}\label{nota-gelee-de-poisson-blanche}}

場合によっては、ジュレ1
Lあたり4尾のエクルヴィスを用いることで、魚のジュレに独特の風味付けをすることも出来る。エクルヴィスをソテーしてビスクを作る要領で煮てから、鉢に入れて細かくすり潰し、最後に漉す作業の10分前に魚のフォンに加える。

\atoaki{}

\hypertarget{gelee-de-poisson-au-vin-rouge}{%
\subsubsection{赤ワインを用いた魚のジュレ}\label{gelee-de-poisson-au-vin-rouge}}

\begin{frsubenv}

Gelée de poisson au vin rouge

\end{frsubenv}

\index{gelee@gelée!poisson vin rouge@--- de poisson au vin rouge}
\index{clarification@clarification!gelee poisson vin rouge@gelée de poisson au vin rouge}
\index{しゆれ@ジュレ!あかわいんをもちいたさかなの@赤ワインを用いた魚の---}
\index{くらりふいえ@クラリフィエ!あかわいんをもちいたさかなのしゆれ@赤ワインを用いた魚のジュレ}

このジュレのクラリフィエには、ジュレ5 Lあたり卵白4個分を用いる。

赤ワインで魚を煮ている途中や、ジュレのクラリフィエ作業の際に、タンニン由来の色素にすぐ変化してしまうことがしばしば、というかほぼ必ず起こる。ワインが分解してしまうのは魚のフュメに含まれているゼラチン質と接触して反応するためのようだ。こんにちに至るまで、これを避ける方法は見つかっていない。

そのため、色合いの不足を補うには人工色素(液体のカルミン\footnote{コチニール色素。\protect\hyperlink{beurres-composes}{合わせバター}本文および訳注参照。}か別の植物由来の色素)を加える必要がある。ただし、使用量にはごく細心の注意を払い、ジュレがやや深みをおびたバラ色を越えてしまわないようにすること。

\end{recette}
