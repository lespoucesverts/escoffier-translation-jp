% !TEX program = LuaLaTeX

\documentclass[openany,14Q,a4paper]{ltjsbook}
\usepackage{amsmath}
  \let\equation\gather
  \let\endequation\endgather
\usepackage{amssymb}
\usepackage[no-math]{fontspec}
\usepackage{geometry}
\usepackage{luaotfload}
\usepackage{graphicx}

\usepackage{setspace}


%% 欧文フォント設定 Libertine/Biolinum
\setmainfont[Ligatures=Historic,Scale=1.0]{Linux Libertine O}
\setsansfont[Ligatures=TeX, Scale=MatchLowercase]{Linux Biolinum O} 
\usepackage{unicode-math}
\setmathfont[Scale=1.2]{libertinusmath-regular.otf}
%%%和文フォント、ルビ
\usepackage{luatexja}
\usepackage{luatexja-fontspec}
%\ltjdefcharrange{8}{"2000-"2013, "2015-"2025, "2027-"203A, "203C-"206F}
%\ltjsetparameter{jacharrange={-2, +8}}
\usepackage{luatexja-ruby}

\newopentypefeature{PKana}{On}{pkna} % "PKana" and "On" can be arbitrary string
%%%%明朝にIPAexMincho、ゴチ(太字)にMoboGoBを使う設定。和文カナプロプーショナル使用可能だが読みづらくなる。
\setmainjfont[%
     %YokoFeatures={JFM=prop,PKana=On},%
     %CharacterWidth=AlternateProportional,%
%    CharacterWidth=Proportional,%Mogo, IPAExMinchoには不可
     %Kerning=On,%
     BoldFont={ MoboGoB },%
     ItalicFont={ MoboGoB },%
     BoldItalicFont={ MoboGoExB }%
     % ]{ MogaHMin }
     ]{ IPAExMincho }
     % ]{ IPAmjMincho }
\setsansjfont[%
     %YokoFeatures={JFM=prop,PKana=On},%
     %CharacterWidth=AlternateProportional,%
     % CharacterWidth=Proportional,%Mobo, IPAExGOthicには不可
     %Kerning=On,
     BoldFont={ MoboGoB },%
     ItalicFont={ MoboGoB },%
     BoldItalicFont={ MoboGoExB }%
     % ]{ MoboGo}
     ]{ IPAExGothic }
     % %  %%%% 和文仮名プロプーショナルここまで

\renewcommand{\bfdefault}{bx}%和文ボールドを有効にする
\renewcommand{\headfont}{\gtfamily\sffamily\bfseries}%和文ボールドを有効にする


%文字サイズ、見出しなどの再定義

\newcommand{\medlarge}{\fontsize{11}{13}\selectfont}
\newcommand{\medsmall}{\fontsize{9.23}{9.5}\selectfont}
%\newcommand{\twelveq}{\jsc@setfontsize\twelveq{9.230769}{9.75}\selectfont}
%\newcommand{\thirteenq}{\jsc@setfontsize\fourteenq{10}{11}\selectfont}
%\newcommand{\fourteenq}{\jsc@setfontsize\fourteenq{10.7692}{13}\selectfont}
%\newcommand{\fifteenq}{\jsc@setfontsize\fifteenq{11.53846}{14}\selectfont}
\makeatletter
\renewcommand{\chapter}{%
  \if@openleft\cleardoublepage\else
  \if@openright\cleardoublepage\else\clearpage\fi\fi
  \plainifnotempty % 元: \thispagestyle{plain}
  \global\@topnum\z@
  \if@english \@afterindentfalse \else \@afterindenttrue \fi
  \secdef
    {\@omit@numberfalse\@chapter}%
    {\@omit@numbertrue\@schapter}}
\def\@chapter[#1]#2{%
  \ifnum \c@secnumdepth >\m@ne
    \if@mainmatter
      \refstepcounter{chapter}%
      \typeout{\@chapapp\thechapter\@chappos}%
      \addcontentsline{toc}{chapter}%
        {\protect\numberline
        % {\if@english\thechapter\else\@chapapp\thechapter\@chappos\fi}%
        {\@chapapp\thechapter\@chappos}%
        #1}%
    \else\addcontentsline{toc}{chapter}{#1}\fi
  \else
    \addcontentsline{toc}{chapter}{#1}%
  \fi
  \chaptermark{#1}%
  \addtocontents{lof}{\protect\addvspace{10\jsc@mpt}}%
  \addtocontents{lot}{\protect\addvspace{10\jsc@mpt}}%
  \if@twocolumn
    \@topnewpage[\@makechapterhead{#2}]%
  \else
    \@makechapterhead{#2}%
    \@afterheading
  \fi}
\def\@makechapterhead#1{%
  \vspace*{0\Cvs}% 欧文は50pt
  {\parindent \z@ \centering \normalfont
    \ifnum \c@secnumdepth >\m@ne
      \if@mainmatter
        \huge\headfont \@chapapp\thechapter\@chappos%変更
        \par\nobreak
        \vskip \Cvs % 欧文は20pt
      \fi
    \fi
    \interlinepenalty\@M
    \huge \headfont #1\par\nobreak
    \vskip 1\Cvs}} % 欧文は40pt%変更

\renewcommand{\section}{%
    \if@slide\clearpage\fi
    \@startsection{section}{1}{\z@}%
    {\Cvs \@plus.5\Cdp \@minus.2\Cdp}% 前アキ
    % {.5\Cvs \@plus.3\Cdp}% 後アキ
    {.5\Cvs}
    {\normalfont\Large\headfont\bfseries\centering}}%変更

\renewcommand{\subsection}{\@startsection{subsection}{2}{\z@}%
    {\Cvs \@plus.5\Cdp \@minus.2\Cdp}% 前アキ
    % {.5\Cvs \@plus.3\Cdp}% 後アキ
    {.5\Cvs}
  %  {\normalfont\large\headfont\bfseries\centering}} %変更
    {\normalfont\large\headfont\centering}} %変更

\renewcommand{\subsubsection}{\@startsection{subsubsection}{3}{\z@}%
  % {0\Cvs \@plus.8\Cdp \@minus.6\Cdp}%変更
    {1sp \@plus.5\Cdp \@minus.5\Cdp}%変更
    {\if@slide .5\Cvs \@plus.3\Cdp \else \z@ \fi}%
    % {\normalfont\medlarge\headfont\leftskip -1\zw}}
    {\normalfont\medlarge\headfont\leftskip -1\zw}}

\renewcommand{\paragraph}{\@startsection{paragraph}{4}{\z@}%
    {0.5\Cvs \@plus.5\Cdp \@minus.2\Cdp}%
    % {\if@slide .5\Cvs \@plus.3\Cdp \else -1\zw\fi}% 改行せず 1\zw のアキ
    {1sp}%後アキ
    {\normalfont\normalsize\headfont}}
\renewcommand{\subparagraph}{\@startsection{subparagraph}{5}{\z@}%
    {\z@}{\if@slide .5\Cvs \@plus.3\Cdp \else -.5\zw\fi}%
    {\normalfont\normalsize\headfont\hskip-.5\zw\noindent}}  

\newenvironment{frchapenv}%
{\begin{center}\vskip-2ex\normalfont\headfont%
    \LARGE\scshape\setstretch{0.8}}%
  {\end{center}\vspace{0.5\zw}\setstretch{1.0}}

\newenvironment{frsecenv}%
{\begin{center}\vskip-2ex\normalfont\headfont%
    \normalsize\scshape\large\setstretch{0.8}}%
  {\end{center}\vspace{0.5\zw}\setstretch{1.0}}

\newenvironment{frsecbenv}%
{\begin{center}\vskip-2\zw\normalfont\headfont%
    \medlarge\scshape\setstretch{0.8}\hspace{1em}}%
  {\end{center}\vspace{0.5\zw}\setstretch{1.0}}

\newenvironment{frsubenv}%
{\begin{spacing}{0.2}\setlength{\leftskip}{-1\zw}\bfseries}%
  {\end{spacing}\normalfont\normalsize\setlength{%
    \leftskip}{0pt}\par\vspace{1.1\zw}}

\renewcommand{\thechapter}{}
\renewcommand{\thesection}{\hskip-1\zw}
\renewcommand{\thesubsection}{}
\renewcommand{\thesubsubsection}{}
\renewcommand{\theparagraph}{}

\renewcommand{\prepartname}{\if@english Part~\else {}\fi}
\renewcommand{\postpartname}{\if@english\else {}\fi}
\renewcommand{\prechaptername}{\if@english Chapter~\else {}\fi}
\renewcommand{\postchaptername}{\if@english\else {}\fi}
\renewcommand{\presectionname}{}%  第
\renewcommand{\postsectionname}{}% 節

\makeatletter
\def\ps@headings{%
  \let\@oddfoot\@empty
  \let\@evenfoot\@empty
  \def\@evenhead{%
    \if@mparswitch \hss \fi
    \underline{\hbox to \fullwidth{\ltjsetparameter{autoxspacing={true}}
%      \textbf{\thepage}\hfil\leftmark}}%
       \normalfont\thepage\hfill\scshape\small\leftmark\normalfont}}%
    \if@mparswitch\else \hss \fi}%
  \def\@oddhead{\underline{\hbox to \fullwidth{\ltjsetparameter{autoxspacing={true}}
        {\if@twoside\scshape\small\rightmark\else\scshape\small\leftmark\fi}\hfil\thepage\normalfont}}\hss}%
  \let\@mkboth\markboth
  \def\chaptermark##1{\markboth{%
    \ifnum \c@secnumdepth >\m@ne
      \if@mainmatter
        \if@omit@number\else
          \@chapapp\thechapter\@chappos\hskip1\zw
        \fi
      \fi
    \fi
    ##1}{}}%
  \def\sectionmark##1{\markright{%
%    \ifnum \c@secnumdepth >\z@ \thesection \hskip1\zw\fi
    \ifnum \c@secnumdepth >\z@ \thesection \hskip-1\zw\fi
    ##1}}}%
\makeatother

\makeatletter
%%%%%%%% Lua GC
\patchcmd\@outputpage{\stepcounter{page}}{%
  \directlua{%
	if jit then
      local k = collectgarbage("count")
      if k>900000 then 
        collectgarbage("collect")
        texio.write_nl("term and log", "GC: ", math.floor(k), math.floor(collectgarbage("count")))
      end
	end
  }%
  \stepcounter{page}%
}{}{}
\makeatother

%リスト環境
\def\tightlist{\itemsep1pt\parskip0pt\parsep0pt}%pandoc対策

\makeatletter
  \parsep   = 0pt
  \labelsep = .5\zw
  \def\@listi{%
     \leftmargin = 0pt \rightmargin = 0pt
     \labelwidth\leftmargin \advance\labelwidth-\labelsep
     \topsep     = 0pt%\baselineskip
     %\topsep -0.1\baselineskip \@plus 0\baselineskip \@minus 0.1 \baselineskip
     \partopsep  = 0pt \itemsep       = 0pt
     \itemindent = -.5\zw \listparindent = 0\zw}
  \let\@listI\@listi
  \@listi
  \def\@listii{%
     \leftmargin = 1.8\zw \rightmargin = 0pt
     \labelwidth\leftmargin \advance\labelwidth-\labelsep
     \topsep     = 0pt \partopsep     = 0pt \itemsep   = 0pt
     \itemindent = 0pt \listparindent = 1\zw}
  \let\@listiii\@listii
  \let\@listiv\@listii
  \let\@listv\@listii
  \let\@listvi\@listii
\makeatother


%%%% 本文中の参照ページ番号表示 %%%%%%%

\makeatletter

%\AtBeginDocument{%
%  \DeclareRobustCommand\ref{\@ifstar\@refstar\@refstar}%
%  \DeclareRobustCommand\pageref{\@ifstar\@pagerefstar\@pagerefstar}}
\let\orig@Hy@EveryPageAnchor\Hy@EveryPageAnchor
\def\Hy@EveryPageAnchor{%
    \begingroup
    \hypersetup{pdfview=Fit}%
    \orig@Hy@EveryPageAnchor
    \endgroup
  }
\makeatother

%%%% pandoc が三点リーダーを勝手に変える対策
\renewcommand{\ldots}{\noindent…}

%%% 脚注番号のページ毎のリセットと脚注位置の調整
%\renewcommand{\footnotesize}{\small}

\makeatletter
\usepackage[bottom,perpage,stable]{footmisc}%
%\setlength{\skip\footins}{4mm plus 4mm}
%\usepackage{footnpag}
\renewcommand\@makefntext[1]{%
  \advance\leftskip 0\zw
  \parindent 1\zw
  \noindent
  \llap{\@thefnmark\hskip0.5\zw}#1}
\let\footnotes@ve=\footnote
\def\footnote{\inhibitglue\footnotes@ve}
\let\footnotemarks@ve=\footnotemark
%\def\footnotemark{\inhibitglue\footnotemarks@ve}
\renewcommand{\footnotemark}{\footnotemarks@ve}%変更
 % \renewcommand{\thefootnote}{\ifnum\c@footnote>\z@%
 %   \leavevmode\hbox{}\@arabic\c@footnote\hbox{)}\fi}
\renewcommand\thefootnote{\hskip.1em\arabic{footnote})}

\makeatother

%%%%%%%%%レシピと本文%%%%%%%%%%%%
\usepackage{multicol}
\setlength{\columnsep}{3\zw}

%%% 本文
\newenvironment{main}{}{}
%%% レシピ
%本文ナミ(無指定)
\newenvironment{recette}%
{\setlength{\parindent}{0pt}\begin{spacing}{0.8}\begin{multicols}{2}%
      \setlength\topskip{.8\baselineskip}}%
    {\end{multicols}\end{spacing}}




%% レイアウト調整(A4Paper,14Q,twoside,ltjsbook.cls) 
%%
\setlength{\hoffset}{0\zw}
\setlength{\oddsidemargin}{0\zw}
\setlength{\evensidemargin}{-1\zw}
% \setlength{\oddsidemargin}{1\zw}%製本時に右ページのみをオフセット
%\setlength{\evensidemargin}{0pt}%
\setlength{\fullwidth}{47\zw}
\setlength{\textwidth}{47\zw}%%ltjsclassesのみ有効
%\setlength{\fullwidth}{159mm}
%\setlength{\textwidth}{159mm}
\setlength{\marginparsep}{0pt}
\setlength{\marginparwidth}{0pt}
\setlength{\footskip}{0pt}
\setlength{\voffset}{-17mm}
\setlength{\textheight}{260mm}
\setlength{\parskip}{0pt}
\setlength{\parindent}{0pt}

\newcommand{\atoaki}{\vspace{1.25mm}}

%%分数の表記
\usepackage{xfrac}
\let\frac\sfrac







%%%%%%%%% hyperref %%%%%%%%%%%%%
\usepackage{refcount}
\usepackage[unicode=true,hyperindex=true,pageanchor]{hyperref}
\hypersetup{hyperindex=true,%
             breaklinks=true,%
             bookmarks=true,%
             pdfauthor={五 島 学(責任編集・訳・注釈),河井 健司(訳・注釈),春野 裕征(訳),山 本 学(訳),高 橋 昇(校正)},%
             pdftitle={エスコフィエ『料理の手引き』全注解},%
            % colorlinks=false%true,%
             colorlinks=true,%
             citecolor=blue,%
             urlcolor=magenta,%
             linkcolor=magenta,%
             bookmarksdepth=subsubsection,%
             pdfborder={0 0 0},%
%             hyperfootnotes=false,%
             plainpages=false,
             }
             \urlstyle{same}
\makeatletter
\usepackage{etoolbox}
\if@mainmatter{\let\myhyperlink\hyperlink%
\renewcommand{\hyperlink}[2]{\myhyperlink{#1}{#2} [p.\getpagerefnumber{#1}{}] }}
  \AtBeginEnvironment{recette}{%
\let\myhyperlink\hyperlink%
\renewcommand{\hyperlink}[2]{\myhyperlink{#1}{#2} [p.\getpagerefnumber{#1}{}] }}
  \AtBeginEnvironment{main}{%
\let\myhyperlink\hyperlink%
\renewcommand{\hyperlink}[2]{\myhyperlink{#1}{#2} [p.\getpagerefnumber{#1}{}] }}
\makeatother

\title{\Huge{オーギュスト・エスコフィエ}\\\HUGE{『料理の手引き』全注解}}
\author{\LARGE{五 島 学(責任編集・訳・注釈)}\and \LARGE{河井 健司(訳・注釈)} \and \large{春野 裕征(訳)} \and \large{山 本 学(原文対照)} \and \large{高 橋 昇(校正)}}
\date{}

%%索引関連
\usepackage{sauceindexdef}
\usepackage{index}
%\usepackage[useindex]{spliidx}

\makeindex

\newindex{src}{idx1}{ind1}{ソース名から料理を探す}
% \renewindex{default}{idx}{ind}{総合索引}


 \makeatletter
\renewenvironment{theindex}{% 索引を3段組で出力する環境
    \if@twocolumn
      \onecolumn\@restonecolfalse
    \else
      \clearpage\@restonecoltrue
    \fi
    \columnseprule.4pt \columnsep 2\zw
    \ifx\multicols\@undefined
      \twocolumn[\@makeschapterhead{\indexname}%
      \addcontentsline{toc}{chapter}{\indexname}]%
    \else
      \ifdim\textwidth=\fullwidth
        \setlength{\evensidemargin}{\oddsidemargin}
        \setlength{\textwidth}{\fullwidth}
        \setlength{\linewidth}{\fullwidth}
        \begin{multicols}{3}[\chapter*{\indexname}%
        \addcontentsline{toc}{chapter}{\indexname}]%
      \else
        \begin{multicols}{2}[\chapter*{\indexname}%
        \addcontentsline{toc}{chapter}{\indexname}]%
      \fi
    \fi
    \@mkboth{\indexname}{}%
    \plainifnotempty % \thispagestyle{plain}
    \parindent\z@
    \parskip\z@ \@plus .3\jsc@mpt\relax
    \let\item\@idxitem
    \raggedright
    \footnotesize\narrowbaselines
  }{
    \ifx\multicols\@undefined
      \if@restonecol\onecolumn\fi
    \else
      \end{multicols}
    \fi
    \clearpage
  }
 \makeatother



\begin{document}


\maketitle

%%後援者一覧
\input{00-avant-propos/00-benefactors}


%空白ページ
\newpage
\thispagestyle{empty}
 %←全角スペース
\newpage
\newpage
\thispagestyle{empty}
 %←全角スペース
\newpage


%%% 序文開始
%\layout%レイアウト数値確認用
%%%%hston blumenthal
\frontmatter

\setlength{\parindent}{1\zw}

\vspace*{4\zw}

\begin{main}

\hypertarget{preface-heston-blumenthal}{%
\section{ヘストン・ブルメンタールによる英訳第5版への序文}\label{preface-heston-blumenthal}}

\vspace{2\zw}
\thispagestyle{empty}

僕は独学でシェフになった。16才の頃、フランスを旅してミシュランの星付きレストランを訪れ、そのときの体験にすっかり魅せられた\ldots{}\ldots{}それは美味しい料理だけじゃなく、レストランの情景やいろんな音、匂い、それに料理をまるで芝居の情景のごとくプレゼンテーションする様子にもだ。僕は一瞬にして、自分がやりたいのはこういうことなんだと理解した。

そんなわけで、学校を辞めて、僕独自の奇妙ともいえる料理修業をはじめた。ミシュランのガイドブックやゴミヨの本を夢中になって読み、年に一度はフランスに行って、それらの本で賞賛されている店に行って食事をし、『フランスグルメツアー』や『ラルース・ガストロノミック』に出ているクラシカルなフランス料理を何度も何度もくりかえし作ってみた。

そしてもちろん、このエスコフィエ『料理の手引き』も。

オーギュスト・エスコフィエが19世紀末から20世紀初頭にかけて、もっとも偉大かつ急進派のシェフだったのは疑うべくもない。\protect\hyperlink{homard-americaine}{オマール・アメリケーヌ}\footnote{オマール・アメリケーヌという料理そのものはエスコフィエの創作によるものではないという点では事実誤認と言える。ただし、これをソースとガルニチュールに分離して舌びらめ等の魚料理に添えるものと位置付けを変えたという点でエスコフィエは画期的といもいえる改良を加えている。\protect\hyperlink{sauce-americaine}{ソース・アメリケーヌ}、\protect\hyperlink{garniture-americaine}{ガルニチュール・アメリケーヌ}および各訳注参照。}や\protect\hyperlink{peches-melba}{ピーチメルバ}のような新しい料理を無数に創作すると同時に、オートキュイジーヌの原則全体を見直して再解釈し、決まりごとのやかましいガルニチュールや重たいソース、やたらと仕事の手間のかかる派手な盛り付けなどは廃したり正していったわけだ。(エスコフィエの口癖のひとつに「シンプルに作れ」というのがあった。そう、まさにその通りの意味にすべきなんだ\footnote{{[}「第二版序文」とりわけp.vii{]}参照。})。エスコフィエの関心が厨房の中にとどまらずすごい広がりを持っていたことに、僕はいつもインスパイアされていると言っていい。エスコフィエは食品科学、とりわけ食品保存について関心を持っていたんだ。トマト缶詰製造やマギーブイヨンの開発にも関わり、自らのブランドで瓶詰めのソースやピクルスも開発した。

エスコフィエという人物は長い影響を及ぼしている。エスコフィエが最初に確立したやりかたで、現代のレストランはいまも運営されている。ブリガードと呼ばれる明確な命令系統のシステムを仕上げたことで、調理場での料理の作り方はまるっきり変わった。料理長の指示は部門シェフに伝わり、彼らから各料理人に伝えられる。それぞれの料理人は自分の担当している作業に責任を持つ、というわけだ。エスコフィエはまた、料理を食卓にお出しする方法も変えた。全部の料理をずらりと一度に食卓に並べる「フランス式サービス」をやめて、ロシア式サービスつまりコース料理が順を追って提供される方式を支持した。このやりかたはこんにちまでずっと続けられているんだ。僕たちは、オーギュストのおかげで食事をしているみたいなものなんだ。

エスコフィエの正確な指示と創意工夫は彼の『料理の手引き』のすべてのページに溢れている。エスコフィエは『料理の手引き』が「ただのレシピ集である以上に役に立つ本」となることを期待していると書いた。実際、まさしくその通りになっているわけだ。エスコフィエの細部にいたるまでの注意深さは伝説になっている。自分の顧客それぞれの食の好き嫌いを書き留め続けていたという。この『料理の手引き』の制作にあたって、エスコフィエ自身が助手に、わざわざこのために用意させた計量秤を送り付け、それぞれの材料の分量を実際に計らせたという。こんなにもレシピに対して綿密であることにこだわったということは、それらのレシピが実際に作ってとても楽しいものになった(これこそがフランス料理の基礎教育の凄いところだ)のはもちろんだが、そのおかげで、世界の超一流シェフのひとりに創造力=想像力についての深い洞察力を与えてくれたのは事実だ。

シェフたちのなかには、クラシカルな伝統なんてすっとばして、エスコフィエみたいな保守的で時代遅れのくせに偉そうな人物なんて否定したいという思いに駆られる者もいる。それはとんでもない間違いだ。エスコフィエの科学、テクノロジー、マーケティング、厨房の改革への関心、それに料理のプレゼンテーションをシンプルにすること、どれも僕としては、非常に現代的な問題だと思う。知れば知るほど、素晴しい料理というものは伝統をぶち壊すことによってではなく、伝統を新たな方向へ導くことによって生み出されるのだと分かったんだ。\ruby{革命}{レボリューション}よりもむしろ\ruby{進化}{エボリュー
ション}すべきなんだ。

\thispagestyle{empty}

そういう意味で、エスコフィエのこの『料理の手引き』は最高のスターティングポイントだと思う。

\vspace{1\zw}
\begin{flushright}
ヘストン・ブルメンタール 2011年
\end{flushright}

\end{main}


\frontmatter

\include{00-avant-propos/02-avant-propos}

%%% 本文開始
\mainmatter

%%% chapitre i. saucesa
%%%%%I. Sauces 
\hypertarget{sauces}{%
\chapter{I. ソース Sauces}\label{sauces}}

\hypertarget{les-fonds-de-cuisine}{%
\section{フォン、その他のストック}\label{les-fonds-de-cuisine}}

\frsec{Les Fonds de Cuisine}

\index{fonds@fonds} \index{ふおん@フォン}

\normalsize
\setstretch{1.0}

本書は実際に厨房で働く料理人を対象としたものだが、まず最初に料理のベースとして仕込んでストックしておくもの\footnote{本書での
  fonds の語は fond (基礎、土台)、fonds
  (資産、資本)、そして料理用語として一般に用いられているフォン、のトリプルミーニングになっている。そのまま「フォン」と訳したいところだが、日本語の場合「出汁」としての意味合いが強いため、本文中では分りやすさを重視してやや冗長に「料理のベースとして仕込んでストックしておくもの」のように訳している。}について少々述べておきたい\footnote{この部分は経営者に向けて書かれているようにも読めるが、エスコフィエの時代以降、料理人がオーナーシェフとして経営に携わるケースが激増したことを考えると、その先見の明に驚かざるを得ない。}。我々料理人にとって重要なものだからだ。

ここで述べる料理のベースとして仕込んでストックしておくものは、実際、料理の土台そのものであり、それなしでは美味しい料理を作ることの出来ない、まず最初に必要なものだ。だからこそ、料理のベースとして仕込んでおくストックはとても重要であり、いい仕事をしたいと努めている料理人ほどこれらを重視している。

これらは、料理において常に立ち戻るべき出発点となるものだが、料理人がいい仕事をしたいと望んでも、才能があっても、それだけでいいものを作ることは出来ない。料理のベースを作るにも材料が必要なのだ。だから、必要な材料は良質のものを自由に使えるようにしなければならない。

筆者としては、むやみな贅沢には反対だが、それと同じくらい、食材コストを抑え過ぎるのも良くないと考えている。そんなことをしていては、伸びる筈の才能の芽を摘んでしまうばかりか、意識の高い料理人ならモチベーションの維持すら出来ないだろう。

どんなに優秀な料理人だって、無から何かを作り出すことは不可能だ。期待される結果に対して、素材の質が劣っていたり量が足りないことがあれば、それでも料理人にいい仕事をしろと要求するなど言語道断である。

料理のベースとして仕込んでおくストックに関するの重要ポイントは、必要な材料は質、量ともに充分に、惜しげもなく使えるようにすることだ。

ある調理現場で可能なことが、別の調理現場では不可能な場合があるのは言うまでもない。料理人の仕事内容は顧客層によっても変わる。到達すべき目標によって手段も変わるということだ。

そういう意味で、何事も相対的なものであるとはいえ、こと料理のベースとして仕込んでストックすべきものに関しては絶対に外してはならないポイントがあるわけだ。組織のトップがこの点で出費を惜しんだり、コスト面で過度に目くじらを立てるようでは、美味しい料理なんて出来るわけがないのだから、現実に厨房を仕切っている料理長を批判する資格もない。そんなのが根拠のない言い掛かりなのは明らかだ。素材の質が悪かったり、量が足りないのであれば、料理長が素晴しい料理を出せないのは言うまでもあるまい。ぶどうの搾りかすに水を加えて醗酵させた安ワインを立派な瓶に詰めてしまえば高級ワインになると思う程に馬鹿げたことはないのだ。

料理人は、必要なものを何でも使っていいなら、料理のベースとして仕込んでおくストックにとりわけ力を入れるべきであり、文句のつけようのない出来になるよう気を使うべきだ。そこに手間隙かけていればそれだけ厨房全体の仕事がきちんと進むのだから、注文を受けた料理をきちんと作れるかどうかは、結局のところ、料理のベースとなる仕込み類にどれだけ手間\ruby{隙}{ひま}をかけるかということなのだ。

\newpage

\hypertarget{principaux-fonds-de-cuisine}{%
\section{主要なフォンとストック}\label{principaux-fonds-de-cuisine}}

\frsec{Principaux Fonds de Cuisine}

料理のベースとして仕込んでおくべきものは主として\ldots{}\ldots{}

\begin{itemize}
\tightlist
\item
  \textbf{コンソメ・サンプルとコンソメ・ドゥーブル}
\item
  \textbf{茶色いフォン、白いフォン、鶏のフォン、ジビエのフォン、魚のフォン}\ldots{}\ldots{}これらはとろみを付けたジュ、基本ソースのベースになる
\item
  \textbf{フュメ、エッセンス}\ldots{}\ldots{}派生ソースに用いる
\item
  \textbf{グラスドヴィアンド、鶏のグラス、ジビエのグラス}
\item
  \textbf{茶色いルー、ブロンドのルー、白いルー}
\item
  \textbf{基本ソース}\ldots{}\ldots{}エスパニョル、ヴルテ、ベシャメル、トマト
\item
  \textbf{肉料理用ジュレ、魚料理用ジュレ}
\end{itemize}

\vspace{1\zw}

以下も日常的に使う料理のベースとして仕込んでおくものとして扱う。

\begin{itemize}
\tightlist
\item
  \textbf{ミルポワ、マティニョン}
\item
  \textbf{クールブイヨン、肉および野菜用のブラン}
\item
  \textbf{マリナード、ソミュール}
\item
  \textbf{肉料理用ファルス、魚料理用ファルス}
\item
  \textbf{ガルニチュールに用いるアパレイユ}、など\ldots{}\ldots{}
\end{itemize}

\vspace{1\zw}

本書は上記を順に説明していく構成にはなっていない。グリル、ロースト、グラタン等の調理技法についても順を追っていくわけではない。料理の種類ごとに一定の位置、つまりは関連の深い料理の章の冒頭において説明していくことになる。

\vspace{1\zw}

そのようなわけで、本書においては以下のようになる\ldots{}\ldots{}

\begin{itemize}
\tightlist
\item
  フォン、フュメ、エッセンス、グラス、マリナード、ジュレの説明\ldots{}\ldots{}
  \textbf{ 第1章 ソース}
\item
  コンソメおよびそのクラリフィエ、ポタージュの浮き実についての説明\ldots{}\ldots{}\textbf{第3章
  ポタージュ}
\item
  ファルスとガルニチュール用アパレイユの作り方\ldots{}\ldots{}\textbf{第2章
  ガルニチュール}
\item
  クールブイヨン、魚料理用ファルス等\ldots{}\ldots{}\textbf{第6章
  魚料理}
\item
  グリル、ブレゼ、ポワレの調理理論\ldots{}\ldots{}\textbf{第7章 肉料理}
\end{itemize}

\newpage

\hypertarget{section-grandes-sauces-de-base}{%
\section{基本ソース}\label{section-grandes-sauces-de-base}}

\frsec{Grandes Sauces de Base}

\index{そーす@ソース!きほん@基本---}
\index{sauce@sauce!00grandes@*Grandes ---s de Base}

\begin{itemize}
\item
  \textbf{およびそれらを組み合せたり煮詰めるなどの方法で作る派生ソース}
\item
  \textbf{イギリス風ソース(温製および冷製)}
\item
  \textbf{いろいろな冷製ソース}
\item
  \textbf{ブール・コンポゼ(ミックスバター)}
\item
  \textbf{マリナード}
\item
  \textbf{ジュレ}
\end{itemize}

\hypertarget{osbservation-sur-la-sauce}{%
\section{概説}\label{osbservation-sur-la-sauce}}

ソースは料理においてもっとも主要な位置にある。フランス料理が世界に冠たるものであるのもひとえにソースの存在によるのだ。だから、ソースは出来るかぎり手間をかけ、細心の注意を払って作るようにしなければならない。

ソースを作るうえでその基礎となるのが何らかの「ジュ」である\footnote{ここではジュといわゆるフォンが同じ意味で使われている。}。すなわち、茶色いソースは「茶色いジュ」(エストゥファード)から作る。ヴルテには「澄んだジュ(白いフォン\footnote{日本の調理現場で「白いフォン」を意味する「フォン・ブラン」は主として鶏のフォンを指すことが多いが、本書で扱われている白いフォンのうち標準的なものは仔牛肉、家禽類をベースとしており、鶏のフォンは別途説明されている。})を使う。ソースを担当する料理人はまず第一に、完璧なジュを作るところから始めなければならない。キュシー侯爵
\footnote{1767-1841。19世紀の著名な美食家。
  著書に『食卓の古典』(1843)がある。料理名にキュシーの名を冠したものも多い。}が言うように、ソース担当の料理人は「頭脳明晰な化学者\footnote{原文
  chimiste。現代は分子ガストロノミーが盛んだが、料理を作る過程で起きる現象や結果を「化学」で説明しようとする試みは少なくともカレームまで遡ることが出来る。\protect\hyperlink{fonds-brun}{茶色いフォン}のレシピにおいて言及されるオスマゾームという想像上の物質もその範疇に含まれるだろう。また、化学の前身たる「錬金術」的概念は中世以来いくつかの料理書において散見される。}でありかつ天才的なクリエイターで、卓越した料理という建造物のいわば大黒柱たる存在」なのだ。

昔のフランス料理\footnote{本書において「昔の料理」と表現される場合は概ね17〜18世紀末と考えていい。}では、素材に串を刺してあぶり焼きするローストを別にすれば、どんな料理も「ブレゼ」か「エチュヴェ」のようなものばかりだった。だが、その時代には既に、フォンが料理という大建築の丸天井の\ruby{要}{か
なめ}だったし、材料コストが重視されるこんにちの我々と比べたら想像も出来ないくらい贅沢に材料を使ってフォンをとっていたのだ。実際、アンヌ・ドートリッシュ\footnote{17世紀に絶対王政を確立したルイ14世の母。}がスペインからルイ13世に嫁いだ際に随行してきたスペインの料理人たちによってフランス料理にルーを用いる方法が伝えられたが\footnote{ルーがスペインからもたらされたというのは逸話、伝承の域を出ない。}、当時はほとんど看過された。ジュそれ自体で充分だったからだ。ところが時代が下り、料理におけるコストの問題が重視されるようになった。ジュはその結果、貧相なものになってしまった。その美味しさを補うものとして、ルーを用いて作るソース・エスパニョルが欠くべからざる存在となった。

ソース・エスパニョルはその完成度の高さゆえに成功をおさめたわけだ。だが、すぐに当初の目的を越えた使い方をされるようになった。19世紀末には本当にこのソースが必要な場合以外にも使われたわけだ。ソース・エスパニョルの濫用によって、どんな料理も固有の香りのない、全部の風味の混ざりあったのっぺりとした調子のものばかりになってしまった。

ようやく近年になって、料理の風味がどれも同じようなものであることに批判が集まってきて、その結果として激しい揺り戻しが起きたのだった。グランドキュイジーヌでは、透き通ったような薄い色合いでしかも風味のしっかりした仔牛のフォンが見直されつつある。そのようなわけで、ソース・エスパニョルそれ自体の重要性はだんだん減っていくだろうと思われる。

ソース・エスパニョルが基本ソースとして扱われるべき理由は何か? ソース・エスパニョルそれ自体に固有の色合いや風味というものはなく、これらはどんなフォンを用いて作るかで決まる。まさにこの点にソース・エスパニョルの長所が存するのだ。補助材料としてルーを加えるが、ルーにはとろみを付けるという意味しかなく、風味にはまったく寄与しない。そもそも、ソースを完璧に仕上げるためには、とろみ以外のルーに含まれる成分はソースからほぼ完全に取り除いてしまっても差し支えはない。不純物を丁寧に取り除いたソースにはルーに含まれていたでんぷん質だけが残っているわけだ。だから、ソースの口あたりを滑らかなものにするために必要なのがでんぷん質だけなら、純粋なでんぷんだけを用いる方がずっと簡単で、作業時間も大幅に短縮されるし、その結果として、ソースを火にかけ過ぎてしまうようなミスも防げる。将来的には、小麦粉ではなく純粋なでんぷんでルーを作るようになるかも知れない。

料理界の現状を\ruby{鑑}{かんが}みるに、\ul{ソース・エスパニョル}と
\ul{とろみを付けたジュ}をそれぞれ使い分けざるを得ない。これにはさまざまな理由があるが、大きな仕立てのブレゼや、羊や仔羊以外を材料にしたラグーでは、肉汁が煮汁に染み出してきて美味しくなるわけだから、トマトを加えたソース・エスパニョルを用いるのがいい。なお、ソース・エスパニョルをさらに丁寧に仕上げるとソース・ドゥミグラスとなる。これはいろいろなソテーに不可欠なもので、今後も変わることはないだろう。

一方、牛や羊、家禽を使った繊細で軽い仕立ての料理にはとろみを付けたジュの方が好まれる。デグラセの際に少量だけ、料理の主素材と同じものからとったジュを用いる。

こんにちのフランス料理においては、肉とソースの調和がとれているべきという、まことに理に適った厳守すべき決まりがある。

だから、ジビエ料理にはジビエのフォンを用いるか、とりたてて際立った個性を持たないフォンを用いて作ったソースを添える。牛や羊のフォンは用いない。ジビエのフォンというのは、さほど濃厚なものを作ることは出来ないが、素材の個性的な風味を表現するには最適だ。こういった事情は魚料理にも当て
\ruby{嵌}{はま}る。ソースそれ自体が際だった風味を持たないものの場合には必ず魚のフュメを加えてやるのだ。このようにしてそれぞれの料理に個性的な風味を実現させることになる。

もちろん、ここまで述べた原則を実現しようにも、コストの問題がしばしば起こることは承知している。けれども、熱意のある、他者の評価を意識している料理人なら問題点を熟考して、完璧とは言わぬまでも満足のいく結果を得ることが出来るだろう。\newpage

\normalsize
\setstretch{1.0}

\hypertarget{traitement-des-elements-de-base}{%
\section{ソースのベース作り}\label{traitement-des-elements-de-base}}

\frsec{Traitement des Éléments de Base dans le Travail des Sauces}

\index{そーす@ソース!そーすつくりのべーす@---のベース作り}
\index{sauce@sauce!Traitement des elements de base dans le travail des sauces@Traitement des Éléments de Base dans le Travail des ---s}
\begin{recette}
\hypertarget{fonds-brun}{%
\subsubsection{茶色いフォン(エストゥファード)}\label{fonds-brun}}

\frsub{Fonds brun ou Estouffade}

\index{ふおん@フォン!ちやいろいふおん@茶色い---}
\index{えすとうふあーと@エストゥファード}
\index{fonds@fonds!brun@--- brun}
\index{fonds@fonds!estouffade@estouffade (fonds brun)}
\index{estouffade@estouffade!fonds brun@ --- (fonds brun)}

(仕上がり10 L分)

\begin{itemize}
\item
  主素材\ldots{}\ldots{}牛すね6
  kg、仔牛のすね6kgまたは仔牛の端肉で脂身を含まないもの6
  kg、骨付きハムのすねの部分1本(前もって下茹でしておくこと)、塩漬けしていない豚皮を下茹でしたもの650
  g。
\item
  香味素材\ldots{}\ldots{}にんじん650 g、玉ねぎ650
  g、ブーケガルニ(パセリの枝100 g、タイム10 g、ローリエ5
  g、にんにく1片)。
\item
  作業手順\ldots{}\ldots{}肉を骨から外す。
\end{itemize}

骨は細かく砕き、オーブンに入れて軽く焼き色を付ける。野菜は焼き色が付くまで炒める。これらを鍋に入れて14
Lの水を注ぎ、ゆっくりと、最低12時間煮込む。水位が下がらぬように、適宜沸騰した湯を足すこと。

大きめのさいの目に切った牛すね肉を別鍋で焼き色が付くまで炒める。先に煮込んでいたフォンを少量加えて煮詰める。この作業を2〜3回行ない、フォンの残りを注ぐ。

鍋を沸騰させて、浮いてくる泡を取り除く。浮き脂も丁寧に取り除く。蓋をして弱火で完全に火が通るまで煮込んだら、布で漉してストックしておく。

\hypertarget{nota-fonds-brun}{%
\subparagraph{【原注】}\label{nota-fonds-brun}}

フォンの材料に牛の骨などが含まれている場合には、事前にその骨だけで12〜
15時間かけてとろ火でフォンをとるといい。

フォンの材料を鍋に焦げ付くくらいまで強く焼き色を付ける\footnote{パンセ
  pincer
  と呼ばれる手法。原義は「抓む」。材料が鍋底に張り付いて、トングなどでしっかり「抓ま」ないと取れないくらい強く焼き付けることからそう呼ばれるようになった。古い料理書では推奨するものも多かった。}のはよろしくない。経験からいって、丁度いい色合いのフォンに仕上げるには、肉に含まれているオスマゾーム\footnote{19世紀頃、赤身肉の美味しさの本質であると考えられていた想像上の物質。赤褐色をした窒素化合物の一種で水に溶ける性質があるとされた。なお、当時のヨーロッパではグルタミン酸はもとよりイノシン酸が「うま味」の要素であるという概念すらなく、「コクがある」corsé
  とか「肉汁たっぷり」onctueux (オンクチュー)や succulent
  (スュキロン)などの表現で肉料理やソースの美味しさが表現された。}の働きだけで充分だ。

\hypertarget{fonds-blanc}{%
\subsubsection{白いフォン}\label{fonds-blanc}}

\frsub{Fonds blanc ordinaire}

\index{ふおん@フォン!しろい@白い---}
\index{fonds@fonds!blanc ordinaire@--- blanc ordinaire}

(仕上がり10 L分)

\begin{itemize}
\item
  主素材\ldots{}\ldots{}仔牛のすね、および端肉10k
  g、鶏の手羽やとさか、足など、または鶏がら4羽分、
\item
  香味素材\ldots{}\ldots{}にんじん800 g、玉ねぎ400 g、ポワロー300
  g、セロリ100 g、ブーケガルニ(パセリの枝100
  g、タイム1枝、ローリエの葉1枚、クローブ4本)。
\item
  使用する液体と味付け\ldots{}\ldots{}水12 L、塩60 g。
\item
  作業手順\ldots{}\ldots{}肉は骨を外し、紐で縛る。骨は細かく砕く。鍋に肉と骨を入れ、水を注ぎ塩を加える。火にかけ、浮いてくるアクを取り除き香味素材を加える。
\item
  加熱時間\ldots{}\ldots{}弱火で3時間。
\end{itemize}

\hypertarget{nota-fonds-blanc}{%
\subparagraph{【原注】}\label{nota-fonds-blanc}}

このフォンは火加減を抑えて、出来るだけ澄んだ仕上がりにすること。アクや浮き脂は丁寧に取り除くこと。

茶色いフォンの場合と同様に、始めに細かく砕いた骨だけを煮てから指定量の水を注ぎ、弱火で5時間煮る方法もある。

この骨を煮た汁で肉を煮るわけだ。その作業内容は上記茶色いフォンの場合と同様。この方法は、骨からゼラチン質を完全に抽出出来るという利点がある。当然のことだが、煮ている間に蒸発して失なわれてしまった分は湯を足してやり、全体量を12
Lにしてから肉を煮ること。

\hypertarget{fonds-de-volaille}{%
\subsubsection{鶏のフォン(フォンドヴォライユ)}\label{fonds-de-volaille}}

\frsub{Fonds de volaille}

\index{ふおん@フォン!とりのふおん@鶏の---}
\index{fonds@fonds!volaille@--- de volaille}
\index{かきん@家禽!とりのふおん@鶏のフォン}
\index{うおらいゆ@ヴォライユ!ふおんとうおらいゆ@フォンドヴォライユ}

白いフォンと同じ主素材、香味素材、水の量で、さらに鶏のとさかや手羽、ガラを適宜増量し、廃鶏3羽を加えて作る。

\hypertarget{jus-de-veau-brun}{%
\subsubsection{仔牛の茶色いフォン(仔牛の茶色いジュ)}\label{jus-de-veau-brun}}

\frsub{Fonds, ou Jus de veau brun}

\index{ふおん@フォン!こうしのちやいろい@仔牛の茶色い---}
\index{しゆ@ジュ!こうしのちやいろいしゆ@仔牛の茶色い---}
\index{fonds@fonds!fonds de veau brun@--- de veau brun}
\index{jus@jus!jus de veau brun@--- de veau brun}
\index{こうし@仔牛!こうしのちやいろいふおん@---の茶色いフォン(ジュ)}
\index{veau@veau!fonds brun@fonds ou jus de --- brun}

(仕上がり10 L分)

\begin{itemize}
\item
  主素材\ldots{}\ldots{}骨を取り除いた仔牛のすね肉と肩肉(紐で縛っておく)6kg、細かく砕いた仔牛の骨5
  kg。
\item
  香味素材\ldots{}\ldots{}にんじん600 g、玉ねぎ400 g、パセリの枝100
  g、ローリエの葉 2枚、タイム2枝。
\item
  使用する液体\ldots{}\ldots{}白いフォンまたは水12 L。水を用いる場合は1
  Lあたり3 gの塩を加える。
\item
  作業手順\ldots{}\ldots{}厚手の片手鍋または寸胴鍋の底に輪切りにしたにんじんと玉ねぎを敷きつめる。その他の香味素材と、あらかじめオーブンで焼き色を付けておいた骨と肉を鍋に加える。
\end{itemize}

蓋をして約10分間、蓋をして弱火にかけた野菜から水分が汗をかくように出るイメージで蒸し焼き状態にし、素材の味を引き出す\footnote{suer
  (スュエ)シュエ。}。フォンまたは水少量を加え、煮詰める。この作業をさらに1〜2回行なう。残りのフォンまたは水を注ぎ、蓋をし、沸騰させる。アクを丁寧に取る。微沸騰の状態で6時間煮る。

布で漉し、ストックしておく。使用目的や必要に応じて、さらに煮詰めてからストックしてもいい。

\hypertarget{fonds-de-gibier}{%
\subsubsection{ジビエのフォン}\label{fonds-de-gibier}}

\frsub{Fonds de gibier}

\index{ふおん@フォン!しひえ@ジビエの---}
\index{fonds@fonds!fonds de gibier@--- de gibier}
\index{しひえ@ジビエ!ふおん@---のフォン}
\index{gibier@gibier!fonds@fonds de ---}

(仕上がり5 L分)

\begin{itemize}
\item
  主素材\ldots{}\ldots{}ノロ鹿の頸、胸肉および端肉3
  kg(老いたノロ鹿がいいが、新鮮なものを使うこと)、野うさぎ\footnote{lièvre
    (リエーヴル)。}の端肉1 kg、老うさぎ2羽、山うずら2羽、老きじ1羽。
\item
  香味素材\ldots{}\ldots{}にんじん250 g、玉ねぎ250
  g、セージ1枝、ジュニパーベリー \footnote{セイヨウネズの樹の実。}15粒、標準的なブーケガルニ。
\end{itemize}

\begin{itemize}
\item
  使用する液体\ldots{}\ldots{}水6 Lおよび白ワイン1瓶。
\item
  加熱時間\ldots{}\ldots{}3時間。
\item
  作業手順\ldots{}\ldots{}ジビエは事前にオーブンで焼き色を付けておき、野菜と香草を敷き詰めた鍋に入れる。野菜類も事前に焼き色を付けておくこと。ジビエを焼くのに用いた天板を白ワインでデグラセし、これを鍋に注ぐ。同量の水も加え、ほぼ水分がなくなるまで煮詰める。
\end{itemize}

この作業の後で、残りの水全量を注ぎ、沸騰させる。丁寧にアクを引きながらごく弱火で煮る\footnote{最後に布で漉す必要があるが、当然のこととして明記されていないので注意。}。

\hypertarget{fumet-de-poisson}{%
\subsubsection[魚のフュメ(フュメドポワソン)]{\texorpdfstring{魚のフュメ(フュメドポワソン)\footnote{本質的には前出の「フォン」と同様のものだが、魚(およびジビエ)を素材としたフォンは香りがポイントとなるため、フュメ
  fumet (香気、良い香りの意)の名称のほうが一般的に使われている。}}{魚のフュメ(フュメドポワソン)}}\label{fumet-de-poisson}}

\frsub{Fonds, ou Fumet de poisson}

\index{ふおん@フォン!さかな@魚の---}
\index{ふゆめ@フュメ!さかな@魚の---}
\index{ふゆめ@フュメ!ほわそん@フュメドポワソン}
\index{fumet@fumet!fumet de poisson@--- de poisson}
\index{fonds@fonds!fumet de poisson@fumet de poisson}

(仕上がり10L分)

\begin{itemize}
\item
  主素材\ldots{}\ldots{}舌びらめ、メルラン\footnote{タラの近縁種。}やバルビュ\footnote{ヒラメの近縁種。}のあら10
  kg。
\item
  香味素材\ldots{}\ldots{}薄切りにした玉ねぎ500 g、パセリの根\footnote{パセリには根がにんじん形に肥大する品種もある(persil
    tubéreux 根パセリ。葉は平らでイタリアンパセリのように使う)。}と茎100
  g、マッシュルームの切りくず250 g、レモンの搾り汁1個分、粒こしょう15
  g(これはフュメを漉す10分前に投入する)。
\item
  使用する液体と調味料\ldots{}\ldots{}水10 L、白ワイン1瓶。液体1
  Lあたり3〜4 gの塩。
\item
  加熱時間\ldots{}\ldots{}30分。
\item
  作業手順\ldots{}\ldots{}鍋底に香味野菜を敷き詰め、魚のあらを入れる。水と白ワインを注ぎ、強火にかける。丁寧にアクを引き、微沸騰の状態を保つようにする。
  30分煮たら目の細かい網で漉す。
\end{itemize}

\hypertarget{nota-fumet-de-poisson}{%
\subparagraph{【原注】}\label{nota-fumet-de-poisson}}

質の悪い白ワインを使うと灰色がかったフュメになってしまう。品質の疑わしいワインは使わないほうがいい。

このフュメはソースを作る際に加える液体として用いる。魚料理用ソース・エスパニョルを作ることを想定する場合には、魚のあらをバターでエチュベしてから水と白ワインを注いで煮るといい。

\hypertarget{fonds-de-poisson-au-vin-rouge}{%
\subsubsection{赤ワインを用いた魚のフォン}\label{fonds-de-poisson-au-vin-rouge}}

\frsub{Fonds de poisson au vin rouge}

\index{ふおん@フォン!あかわいんをもちいたさかなのふおん@赤ワインを用いた魚の---}
\index{fonds@fonds!fonds de poisson au vin rouge@--- de poisson au vin rouge}

このフォンそれ自体を用意することは滅多にない。というのも、例えばマトロットのような料理の魚の煮汁そのものだからだ。

とはいえ、こんにちでは魚のアラをすっかり取り除いた状態で料理を提供する必要がますます高まってきているので、ここでそのレシピを記しておくべきだろう。このフォンの必要性と有用さはどんどん高まっていくと思われる。

原則として、このフォンの仕込みには、料理として提供するのと同じ種類の魚のアラを用いて、その香りの特徴を生かす必要がある。だが、どんな種類の魚を使う場合でも作り方は同じだ。

(仕上がり5 L分)

\begin{itemize}
\item
  主素材\ldots{}\ldots{}料理に用いるのと同じ魚種の頭とアラ2.5 kg。
\item
  香味素材\ldots{}\ldots{}薄切りにして下茹でした玉ねぎ300
  g、パセリの枝100
  g、タイムの小枝1本、小さめのローリエの葉2枚、にんにく5片、マッシュルームの切りくず100
  g。
\item
  使用する液体と調味料\ldots{}\ldots{}水3.5 L、良質の赤ワイン2 L、塩15
  g。
\item
  加熱時間\ldots{}\ldots{}30分。
\item
  作業手順\ldots{}\ldots{}「魚の白いフォン\footnote{前項のフュメドポワソンのこと。}」と同様にする。
\end{itemize}

\hypertarget{nota-fonds-de-poisson-au-vin-rouge}{%
\subparagraph{【原注】}\label{nota-fonds-de-poisson-au-vin-rouge}}

このフォンは魚の白いフォンよりも濃く煮詰めることが可能。とはいえ、保存のために煮詰めないでいいように、その都度、必要な量だけ仕込むことを勧める。

\hypertarget{essence-de-poisson}{%
\subsubsection{魚のエッセンス}\label{essence-de-poisson}}

\frsub{Essence de poisson}

\index{えつせんす@エッセンス!さかな@魚の---}
\index{essence@essence!poisson@--- de poisson}

\begin{itemize}
\item
  主素材\ldots{}\ldots{}メルラン\footnote{タラの近縁種。}および舌びらめの頭、アラ2
  kg。
\item
  香味素材\ldots{}\ldots{}薄切りにした玉ねぎ125
  g、マッシュルームの切りくず300 g、パセリの枝50
  g、レモンの搾り汁1個分。
\item
  使用する液体\ldots{}\ldots{}煮詰めていないフュメドポワソン1
  \(\frac{1}{2}\) L、良質の白ワイン3 dL。
\item
  所要時間\ldots{}\ldots{}45分。
\item
  作業手順\ldots{}\ldots{}鍋にバター100
  gと玉ねぎ、パセリの枝、マッシュルームの切りくずを入れ、強火で色づかないようさっと炒める。アラと端肉を加える。蓋をして約15分弱火で蒸し煮する\footnote{素材を入れた鍋に蓋をして弱火にかけ、少量の水分で蒸し煮状態にすることを
    étuver
    エチュベという。このフランス語をそのまま用いている調理現場も少なくない。}。その間、小まめに混ぜてやること。白ワインを注ぎ、半量になるまで煮詰める。最後にフュメドポワソンを注ぎ、レモン汁と塩2
  gを加える。
\end{itemize}

再び火にかけて、とろ火で15分程煮込んだら、布で漉す。

\hypertarget{nota-essence-de-poisson}{%
\subparagraph{【原注】}\label{nota-essence-de-poisson}}

魚のエッセンスは、舌びらめやチュルボ、チュルボタン、バルビュ\footnote{いずれも鰈、ひらめの近縁種。チュルボタンはチュルボの小さいものを言う。}
などのフィレ\footnote{3枚おろし、または5枚おろしにして、頭とアラを取り除いた状態。}をポシェする際に用いる。

さらに、このエッセンスを煮詰めて、上記でポシェした魚のソースに加えて風味を強くするのに使う。

\hypertarget{essences-diverses}{%
\subsubsection{エッセンスについて}\label{essences-diverses}}

\frsub{Essences diverses}

\index{えつせんす@エッセンス!01えつせんすについて@---について(フォン)}
\index{essence@essence!01 diverses@---s diverses (fonds)}

その名のとおり、エッセンスとはごく少量になるまで煮詰めて非常に強い風味を持たせたフォンのこと。

エッセンスは普通のフォンと本質的には同じものだが、素材の風味をしっかり出すために、使用する液体の量はずっと少ない。したがって、仕上げにエッセンスを加える指示がある料理の場合でも、そもそも充分に風味ゆたかなフォンを用いていれば、エッセンスは必要ないことが分かるだろう。

まず最初に、美味しく風味ゆたかなフォンを用いるほうが、あまり出来のよくないフォンで調理し、後からエッセンスで欠点を補うよりもずっと簡単なのだ。その方がいい結果が得られるし、時間と材料の節約にもなる。

セロリ、マッシュルーム、モリーユ\footnote{morille
  キノコの一種。和名アミガサタケ。}、トリュフなど、とりわけ明確な風味の素材のエッセンスを、必要に応じて用いるにとどめるのがいい。

また、十中八九、フォンを仕込む際に素材そのものを加えた方が、エッセンスを仕込むよりもいい結果が得られることは頭に入れておくこと。

そのようなわけで、エッセンスについてこれ以上長々と述べる必要もないと思われる。ベースとなるフォンがコクと風味がゆたかなものならであるなら、エッセンスはまったく無用の長物と言える。

\hypertarget{glaces-diverses}{%
\subsubsection{グラスについて}\label{glaces-diverses}}

\frsub{Glaces diverses}

\index{くらす@グラス!01くらすについて@---について}
\index{glace@glace!01 diverses@---s diverses}

グラスドヴィアンド、鶏のグラス(グラスドヴォライユ)、ジビエのグラス、魚のグラスの用途は多岐にわたる。これらは、上記いずれかの素材でとったフォンをシロップ状になるまで煮詰めたもののことだ。

これらの使い途は、料理の仕上げに表面に塗ってしっとりとした艶を出させるのに用いる場合もあれば、ソースの味や色合いを濃くするために用いたり、あるいは、あまりに出来のよくないフォンで作った料理の場合にはコクを与えるために使うこともある。また、料理によっては適量のバターやクリームを加えてグラスそのものをソースとして用いることもある。

グラスとエッセンスの違いだが、エッセンスが料理の風味そのものを強くすることだけが目的であるのに対して、グラスは素材の持つコクと風味をごく少量にまで濃縮したものだ。

だからほとんどの場合、エッセンスよりもグラスを使うほうがいい。

とはいえ昔の料理長たちの中には、グラスの使用を絶対に認めない者もいた。その理由は、料理を作る度に毎回その料理のためのフォンをとるべきであり、それだけで料理として充分なものにすべき、ということだった。


確かに時間と費用の点で制限がなければその理屈は正しい。だが、こんにちでは、そのようなことの出来る調理現場はほとんどない。そもそもグラスは、正しく適量を用いるのであれば、そのグラスが丁寧に作られたものであるならな、素晴しい結果が得られる。
だから多くの場合、グラスはまことに有用なものと言える。

\hypertarget{glace-de-viande}{%
\subsubsection{グラスドヴィアンド}\label{glace-de-viande}}


\frsub{Glace de viande}

\index{くらす@グラス!くらすとういあんと@---ドヴィアンド}
\index{glace@glace!viande@--- de viande}

茶色いフォン(エストゥファード)を煮詰めて作る。

煮詰めて濃くなっていく途中、何度か布で漉して、より小さな鍋に移しかえていく。煮詰めている際に、丁寧にアクを引くことが、澄んだグラスを作るポイント。

煮詰めている際には、フォンの濃縮具合に応じて、火加減を弱めていくこと。最初は強火でいいが、作業の最後の方は弱火にしてゆっくり煮詰めてやること。

スプーンを入れてみて、引き上げた際に、艶のあるグラスの層でスプーンが覆われ、しっかり張り付いているくらいが丁度いい。要するに、スプーンがグラスでコーティングされた状態になればいいということだ。

\hypertarget{nota-glace-de-viande}{%
\subparagraph{【原注】}\label{nota-glace-de-viande}}

色が薄くて軽い仕上がりのグラスが必要な場合には、茶色いフォンではなく、標準的な仔牛のフォンを用いる。

\hypertarget{glace-de-volaille}{%
\subsubsection{鶏のグラス(グラスドヴォライユ)}\label{glace-de-volaille}}

\frsub{Glace de volaille}

\index{くらす@グラス!とり@鶏の---(---ドヴォライユ)}
\index{くらす@グラス!うおらいゆ@---ドヴォライユ}
\index{glace@glace!volaille@--- de volaille}

鶏のフォン(フォンドヴォライユ)を用いて、グラスドヴィアンドと同様にして作る。

\hypertarget{glace-de-gibier}{%
\subsubsection{ジビエのグラス}\label{glace-de-gibier}}

\frsub{Glace de gibier}

\index{くらす@グラス!しひえ@ジビエの---}
\index{glace@glace!glace de gibier@--- de gibier}
\index{しひえ@ジビエ!くらす@---のグラス}
\index{gibier@gibier!gibier@glace de ---}

ジビエのフォンを煮詰めて作る。ある特定のジビエの風味を生かしたグラスを作るには、そのジビエだけでとったフォンを用いること。

\hypertarget{glace-de-poisson}{%
\subsubsection{魚のグラス}\label{glace-de-poisson}}

\frsub{Glace de poisson}

\index{くらす@グラス!さかな@魚の---}
\index{glace@glace!poisson@--- de poisson}

このグラスを用いることはあまり多くない。日常的な業務においては「魚のエッセンス」を用いることが好まれる。そのほうが魚の風味も繊細になる。魚のエッセンスで魚をポシェした後に煮詰めてソースに加える。
\end{recette}
\hypertarget{roux}{%
\section{ルー}\label{roux}}

\frsec{Roux}

\index{るー@ルー} \index{roux@roux}

ルーはいろいろな派生ソースのベースとなる基本ソースにとろみを付ける役目を持つ。ルーの仕込みは、一見したところさほど重要に思われぬだろうが、実際には正反対だ。丁寧に注意深く作業すること。

茶色いルーは加熱に時間がかかるので、大規模な調理現場では前もって仕込んでおく。ブロンドのルーと白いルーはその都度用意すればいい。
\begin{recette}
\hypertarget{roux-brun}{%
\subsubsection{茶色いルー}\label{roux-brun}}

\frsub{Roux brun}

\index{るー@ルー!ちやいろ@茶色い---} \index{roux@roux!brun@--- brun}

(仕上がり1 kg分)

\begin{enumerate}
\def\labelenumi{\arabic{enumi}.}
\tightlist
\item
  澄ましバター\ldots{}\ldots{}500 g
\item
  ふるった小麦粉\ldots{}\ldots{}600 g
\end{enumerate}

\hypertarget{cuisson-des-roux}{%
\subsubsection{ルーの火入れについて}\label{cuisson-des-roux}}

\index{るー@ルー!ひいれについて@---の火入れについて}
\index{roux@roux!cuisson@cuisson du ---}

加熱時間は使用する熱源の強さで変わってくる。だから数字で何分とは言えない。ただし、火力が強過ぎるよりは弱いくらいの方がいい。というのも、温度が高すぎると小麦粉の細胞が硬化して中身を閉じ込めてしまい、そうなると後でフォンなどの液体を加えた際に上手く混ざらず、滑らかなとろみの付いたソースにならない。乾燥豆をいきなり熱湯で茹でるのと同じようなことが起きるわけだ。低い温度から始めてだんだんと熱くしていけば、小麦粉の細胞壁がゆるんで細胞中のでんぷんが膨張し、熱によって発酵状態の初期のようになる。このようにして、でんぷんをデキストリンに変化させる\footnote{現代の科学的見地からすると必ずしも正確な記述ではないので注意。}。デキストリンは水溶性の物質で、これが「とろみ」の主な要素なのだ。茶色いルーは淡褐色の美しい色合いで滑らかな仕上がりにする。だまがあってはいけない。

ルーを作る際には必ず、澄ましバターを使うこと\footnote{初版〜第三版では「澄ましバターまたは充分に澄ましたグレスドマルミット」となっている。グレスドマルミットとは、コンソメなどを作る際に、浮いてくる油脂を取り除く必要があるが、それを捨てずにまとめてから漉して澄ませたもののこと。基本的に獣脂と考えていい。なお、同時代の料理書
  --- 例えばペラプラ『近代料理技術』(1935年)---
  には、ルーを作るのにバターを使う必要はなく、グレスドマルミットで充分、としているものもある。}。
生のバターには相当量のカゼインが含まれている。カゼインがあると火を均質に通すことが出来なくなってしまう。とはいえ、以下を覚えておくといい。ソースとして仕上げた段階で、ルーで使ったバターは風味という点ではほとんど意味が失なわれている。そもそもソースの仕上げに不純物を取り除く\footnote{dépouiller
  デプイエ。ソースや煮込み料理を仕上げる際に、浮き上がってくる不純物を徹底的に取り除き、目の細かい布などで漉すこと。原義は動物などの皮を剥ぐ、剥くことの意で、野うさぎの皮を剥ぐ、うなぎの皮を剥く、という意味で現代の厨房でも用いられているる。ソースの場合は表面に凝固した蛋白質や油脂の膜が出来、それを「剥ぐように」取り除くことから、あるいは表面に浮いてくる不純物を徹底的に取り除いてきれいなソースに仕上げることを、動物の皮を剥いてきれいな身だけにすることになぞらえて、この用語が用いられるようになったようだ。なお、本書においてécumer(エキュメ)が単に浮いてくる泡やアクを取る、という作業であるのに対して、dépouiller(デプイエ)は「徹底的に不純物を取り除いて美しく仕上げる」という意味合いが込められている。現代では品種改良や農法の変化によって野菜のアクも少なくなり、小麦粉も精製度の高いものを利用出来るなど、食材および調味料の多くで純度の高いものを使用する場合がほとんどであり、このデプイエという作業は20世紀後半にはほとんど行なわれなくなり、écumer(エキュメ)という用語だけで済ませることがほとんど(cf.辻静雄監訳『オリヴェ
  ソースの本』柴田書店、 1970年、27〜28頁)。}段階でバターも完全に取り除かれてしまうわけだ。だからルーに用いるバターは小麦粉に熱を通すためだけのものと考えていい。

ルーはソース作りの出発点だ。だから次の点も記憶に\ruby{留}{とど}めること。小麦粉にでんぷんが含まれているからこそソースに「とろみ」が付く。だから純粋なでんぷん(特性が小麦のでんぷんと同じでも異なったものでも)でルーを作っても、小麦粉の場合と同様の結果が得られるだろう。ただしその場合は小麦粉でルーを作る場合より注意して作業する必要がある。また、小麦粉と違って余計な物質が含まれていないために、全体の分量比率を考え直すことになる。

\hypertarget{nota-roux}{%
\subparagraph{【原注】}\label{nota-roux}}

本文で述べたように、茶色いルーを作る際には澄ましバターを用いる。他の動物性油脂はよほど経済的事情が逼迫していない限り使わないこと。材料コストが問題になる場合でも、ソースの仕上げに不純物を取り除く際に多少の注意を払えば、ルーに用いたバターを回収するのはさして難しいことではない\footnote{既に述べたように初版〜第三版まではバターまたはグレスドマルミットという指示であったことに留意する必要はあるだろう。実際のところ、良質のバターを用いてルーを作ったほうが、軽やかな仕上りのソースになる傾向があることは言うまでもない。}。それを後で他の用途で使えばいいだろう。

\hypertarget{roux-blond}{%
\subsubsection{ブロンドのルー}\label{roux-blond}}

\frsub{Roux blond}

\index{るー@ルー!ふろんと@ブロンドの---}
\index{roux@roux!blond@--- blond}

(仕上がり1 kg分)

材料の比率は茶色いルーと同じ。すなわちバター500 gと、ふるった小麦粉600
g。

火入れは、ルーがほんのりブロンド色になるまで、ごく弱火で行なう。

\hypertarget{roux-blanc}{%
\subsubsection{白いルー}\label{roux-blanc}}

\frsub{Roux blanc}

\index{るー@ルー!しろい@白い---} \index{roux@roux!blanc@--- blanc}

500 gのバターと、ふるった小麦粉600 g。

このルーの火入れは数分、つまり粉っぽさがなくなるまでの時間でいい。
\end{recette}
\PassOptionsToPackage{unicode=true}{hyperref} % options for packages loaded elsewhere
\PassOptionsToPackage{hyphens}{url}
%
\documentclass[14Q,a4paperpaper,]{ltjsbook}
\usepackage{lmodern}
\usepackage{amssymb,amsmath}
\usepackage{ifxetex,ifluatex}
\usepackage{fixltx2e} % provides \textsubscript
\ifnum 0\ifxetex 1\fi\ifluatex 1\fi=0 % if pdftex
  \usepackage[T1]{fontenc}
  \usepackage[utf8]{inputenc}
  \usepackage{textcomp} % provides euro and other symbols
\else % if luatex or xelatex
  \usepackage{unicode-math}
  \defaultfontfeatures{Ligatures=TeX,Scale=MatchLowercase}
\fi
% use upquote if available, for straight quotes in verbatim environments
\IfFileExists{upquote.sty}{\usepackage{upquote}}{}
% use microtype if available
\IfFileExists{microtype.sty}{%
\usepackage[]{microtype}
\UseMicrotypeSet[protrusion]{basicmath} % disable protrusion for tt fonts
}{}
\IfFileExists{parskip.sty}{%
\usepackage{parskip}
}{% else
\setlength{\parindent}{0pt}
\setlength{\parskip}{6pt plus 2pt minus 1pt}
}
\usepackage{hyperref}
\hypersetup{
            pdftitle={エスコフィエ『料理の手引き』全注解},
            pdfauthor={五 島 学},
            pdfborder={0 0 0},
            breaklinks=true}
\urlstyle{same}  % don't use monospace font for urls
\setlength{\emergencystretch}{3em}  % prevent overfull lines
\providecommand{\tightlist}{%
  \setlength{\itemsep}{0pt}\setlength{\parskip}{0pt}}
\setcounter{secnumdepth}{0}
% Redefines (sub)paragraphs to behave more like sections
\ifx\paragraph\undefined\else
\let\oldparagraph\paragraph
\renewcommand{\paragraph}[1]{\oldparagraph{#1}\mbox{}}
\fi
\ifx\subparagraph\undefined\else
\let\oldsubparagraph\subparagraph
\renewcommand{\subparagraph}[1]{\oldsubparagraph{#1}\mbox{}}
\fi

% set default figure placement to htbp
\makeatletter
\def\fps@figure{htbp}
\makeatother


\title{エスコフィエ『料理の手引き』全注解}
\author{五 島 学}
\date{}
%%%%%%%%%%%%%%%%%%%%%%% added by mgoto
\usepackage{setspace}


% %%%%%%%%% hyperref %%%%%%%%%%%%%
% \usepackage{refcount}
% \usepackage[unicode=true,hyperfootnotes=false,pageanchor]{hyperref}
% \hypersetup{hyperindex=false,%
%              breaklinks=true,%
%              bookmarks=true,%
%              pdfauthor={五島 学},%
%              pdftitle={エスコフィエ『料理の手引き』全注解},%
%              colorlinks=false%true,%
%              %colorlinks=true,%
%              citecolor=blue,%
%              urlcolor=cyan,%
%              linkcolor=magenta,%
%              bookmarksdepth=subsubsection,%
%              pdfborder={0 0 0},%
%              hyperfootnotes=false,%
%              plainpages=false,
%              }
% \urlstyle{same}





%% 欧文フォント設定
% Libertine/Biolinum
\setmainfont[Ligatures=Historic,Scale=1.0]{Linux Libertine O}
\setsansfont[Ligatures=TeX, Scale=MatchLowercase]{Linux Biolinum O} 
%\usepackage{libertine}
\usepackage{unicode-math}
\setmathfont[Scale=1.2]{libertinusmath-regular.otf}
\usepackage{luatexja}
\usepackage{luatexja-fontspec}
%\ltjdefcharrange{8}{"2000-"2013, "2015-"2025, "2027-"203A, "203C-"206F}
%\ltjsetparameter{jacharrange={-2, +8}}
\usepackage{luatexja-ruby}

\newopentypefeature{PKana}{On}{pkna} % "PKana" and "On" can be arbitrary string
%%%%明朝にIPAexMincho、ゴチ(太字)にMoboGoBを使う設定。和文カナプロプーショナル使用可能だが読みづらくなる。
\setmainjfont[%
     %YokoFeatures={JFM=prop,PKana=On},%
     %CharacterWidth=AlternateProportional,%
%    CharacterWidth=Proportional,%Mogo, IPAExMinchoには不可
     %Kerning=On,%
     BoldFont={ MoboGoB },%
     ItalicFont={ MoboGoB },%
     BoldItalicFont={ MoboGoExB }%
     % ]{ MogaHMin }
     ]{ IPAExMincho }
     % ]{ IPAmjMincho }
\setsansjfont[%
     %YokoFeatures={JFM=prop,PKana=On},%
     %CharacterWidth=AlternateProportional,%
     % CharacterWidth=Proportional,%Mobo, IPAExGOthicには不可
     %Kerning=On,
     BoldFont={ MoboGoB },%
     ItalicFont={ MoboGoB },%
     BoldItalicFont={ MoboGoExB }%
     % ]{ MoboGo}
     ]{ IPAExGothic }
     % %  %%%% 和文仮名プロプーショナルここまで

     
\renewcommand{\bfdefault}{bx}%和文ボールドを有効にする
\renewcommand{\headfont}{\gtfamily\sffamily\bfseries}%和文ボールドを有効にする

%リスト環境
\def\tightlist{\itemsep1pt\parskip0pt\parsep0pt}%pandoc対策

\makeatletter
  \parsep   = 0pt
  \labelsep = .5\zw
  \def\@listi{%
     \leftmargin = 0pt \rightmargin = 0pt
     \labelwidth\leftmargin \advance\labelwidth-\labelsep
     \topsep     = 0pt%\baselineskip
     %\topsep -0.1\baselineskip \@plus 0\baselineskip \@minus 0.1 \baselineskip
     \partopsep  = 0pt \itemsep       = 0pt
     \itemindent = -.5\zw \listparindent = 0\zw}
  \let\@listI\@listi
  \@listi
  \def\@listii{%
     \leftmargin = 1.8\zw \rightmargin = 0pt
     \labelwidth\leftmargin \advance\labelwidth-\labelsep
     \topsep     = 0pt \partopsep     = 0pt \itemsep   = 0pt
     \itemindent = 0pt \listparindent = 1\zw}
  \let\@listiii\@listii
  \let\@listiv\@listii
  \let\@listv\@listii
  \let\@listvi\@listii
\makeatother

%%%%インデックス準備
%\usepackage{makeidx}
\usepackage{index}
%\usepackage[useindex]{splitidx}
\newindex{src}{idx1}{ind1}{ソース名から料理を探す}
\makeindex
 
 \makeatletter
\renewenvironment{theindex}{% 索引を3段組で出力する環境
    \if@twocolumn
      \onecolumn\@restonecolfalse
    \else
      \clearpage\@restonecoltrue
    \fi
    \columnseprule.4pt \columnsep 2\zw
    \ifx\multicols\@undefined
      \twocolumn[\@makeschapterhead{\indexname}%
      \addcontentsline{toc}{chapter}{\indexname}]%
    \else
      \ifdim\textwidth=\fullwidth
        \setlength{\evensidemargin}{\oddsidemargin}
        \setlength{\textwidth}{\fullwidth}
        \setlength{\linewidth}{\fullwidth}
        \begin{multicols}{3}[\chapter*{\indexname}%
        \addcontentsline{toc}{chapter}{\indexname}]%
      \else
        \begin{multicols}{2}[\chapter*{\indexname}%
        \addcontentsline{toc}{chapter}{\indexname}]%
      \fi
    \fi
    \@mkboth{\indexname}{}%
    \plainifnotempty % \thispagestyle{plain}
    \parindent\z@
    \parskip\z@ \@plus .3\jsc@mpt\relax
    \let\item\@idxitem
    \raggedright
    \footnotesize\narrowbaselines
  }{
    \ifx\multicols\@undefined
      \if@restonecol\onecolumn\fi
    \else
      \end{multicols}
    \fi
    \clearpage
  }
 \makeatother


%%%% 本文中の参照ページ番号表示 %%%%%%%

\makeatletter

%\AtBeginDocument{%
%  \DeclareRobustCommand\ref{\@ifstar\@refstar\@refstar}%
%  \DeclareRobustCommand\pageref{\@ifstar\@pagerefstar\@pagerefstar}}
\let\orig@Hy@EveryPageAnchor\Hy@EveryPageAnchor
\def\Hy@EveryPageAnchor{%
    \begingroup
    \hypersetup{pdfview=Fit}%
    \orig@Hy@EveryPageAnchor
    \endgroup
  }
  \usepackage{etoolbox}
\if@mainmatter{\let\myhyperlink\hyperlink%
\renewcommand{\hyperlink}[2]{\myhyperlink{#1}{#2} [p.\getpagerefnumber{#1}{}] }}
  \AtBeginEnvironment{recette}{%
\let\myhyperlink\hyperlink%
\renewcommand{\hyperlink}[2]{\myhyperlink{#1}{#2} [p.\getpagerefnumber{#1}{}] }}
  \AtBeginEnvironment{Main}{%
\let\myhyperlink\hyperlink%
\renewcommand{\hyperlink}[2]{\myhyperlink{#1}{#2} [p.\getpagerefnumber{#1}{}] }}


%%%% pandoc が三点リーダーを勝手に変える対策
\renewcommand{\ldots}{\noindent…}

%%% 脚注番号のページ毎のリセットと脚注位置の調整
%\renewcommand{\footnotesize}{\small}

\makeatletter

\usepackage[bottom,perpage,stable]{footmisc}%
%\setlength{\skip\footins}{4mm plus 4mm}
%\usepackage{footnpag}
\renewcommand\@makefntext[1]{%
  \advance\leftskip 0\zw
  \parindent 1\zw
  \noindent
  \llap{\@thefnmark\hskip0.5\zw}#1}


\let\footnotes@ve=\footnote
\def\footnote{\inhibitglue\footnotes@ve}
\let\footnotemarks@ve=\footnotemark
%\def\footnotemark{\inhibitglue\footnotemarks@ve}
\renewcommand{\footnotemark}{\footnotemarks@ve}%変更
% %\def\thefootnote{\ifnum\c@footnote>\z@\leavevmode\lower.5ex\hbox{(}\@arabic\c@footnote\hbox{)}\fi}
\renewcommand{\thefootnote}{\ifnum\c@footnote>\z@\leavevmode\hbox{}\@arabic\c@footnote\hbox{)}\fi}

%%%%%%%%%レシピと本文%%%%%%%%%%%%
\usepackage{multicol}
\setlength{\columnsep}{3\zw}

%%% 本文
\newenvironment{Main}{}{}
%%% レシピ
% \setlength{\columnwidth}{24\zw}
%本文ヨリ小%\small
%\newenvironment{recette}{\setlength{\parindent}{0pt}\begin{small}\begin{spaceing}{0.8}\begin{multicols}{2}}{\end{multicols}\end{spacing}\end{small}}
%本文やや小%\medsmall
%\newenvironment{recette}{\setlength{\parindent}{0pt}\begin{medsmall}\begin{spacing}{0.75}\begin{multicols}{2}}{\end{multicols}\end{spacing}\end{medsmall}}
%本文ナミ(無指定)
\newenvironment{recette}{\setlength{\parindent}{0pt}\begin{spacing}{0.8}\begin{multicols}{2}\setlength\topskip{.8\baselineskip}}{\end{multicols}\end{spacing}}

%文字サイズ、見出しなどの再定義
\makeatletter
%\renewcommand{\large}{\jsc@setfontsize\large\@xipt{14}}
%\renewcommand{\Large}{\jsc@setfontsize\Large{13}{15}}

\newcommand{\medlarge}{\fontsize{11}{13}\selectfont}
\newcommand{\medsmall}{\fontsize{9.23}{9.5}\selectfont}
\newcommand{\twelveq}{\jsc@setfontsize\twelveq{9.230769}{9.75}\selectfont}
\newcommand{\thirteenq}{\jsc@setfontsize\fourteenq{10}{11}\selectfont}
\newcommand{\fourteenq}{\jsc@setfontsize\fourteenq{10.7692}{13}\selectfont}
\newcommand{\fifteenq}{\jsc@setfontsize\fifteenq{11.53846}{14}\selectfont}
\makeatletter
\renewcommand{\chapter}{%
  \if@openleft\cleardoublepage\else
  \if@openright\cleardoublepage\else\clearpage\fi\fi
  \plainifnotempty % 元: \thispagestyle{plain}
  \global\@topnum\z@
  \if@english \@afterindentfalse \else \@afterindenttrue \fi
  \secdef
    {\@omit@numberfalse\@chapter}%
    {\@omit@numbertrue\@schapter}}
\def\@chapter[#1]#2{%
  \ifnum \c@secnumdepth >\m@ne
    \if@mainmatter
      \refstepcounter{chapter}%
      \typeout{\@chapapp\thechapter\@chappos}%
      \addcontentsline{toc}{chapter}%
        {\protect\numberline
        % {\if@english\thechapter\else\@chapapp\thechapter\@chappos\fi}%
        {\@chapapp\thechapter\@chappos}%
        #1}%
    \else\addcontentsline{toc}{chapter}{#1}\fi
  \else
    \addcontentsline{toc}{chapter}{#1}%
  \fi
  \chaptermark{#1}%
  \addtocontents{lof}{\protect\addvspace{10\jsc@mpt}}%
  \addtocontents{lot}{\protect\addvspace{10\jsc@mpt}}%
  \if@twocolumn
    \@topnewpage[\@makechapterhead{#2}]%
  \else
    \@makechapterhead{#2}%
    \@afterheading
  \fi}
\def\@makechapterhead#1{%
  \vspace*{0\Cvs}% 欧文は50pt
  {\parindent \z@ \centering \normalfont
    \ifnum \c@secnumdepth >\m@ne
      \if@mainmatter
        \huge\headfont \@chapapp\thechapter\@chappos%変更
        \par\nobreak
        \vskip \Cvs % 欧文は20pt
      \fi
    \fi
    \interlinepenalty\@M
    \huge \headfont #1\par\nobreak
    \vskip 1\Cvs}} % 欧文は40pt%変更

\renewcommand{\section}{%
    \if@slide\clearpage\fi
    \@startsection{section}{1}{\z@}%
    {\Cvs \@plus.5\Cdp \@minus.2\Cdp}% 前アキ
    % {.5\Cvs \@plus.3\Cdp}% 後アキ
    {.5\Cvs}
    {\normalfont\Large\headfont\bfseries\centering}}%変更

\renewcommand{\subsection}{\@startsection{subsection}{2}{\z@}%
    {\Cvs \@plus.5\Cdp \@minus.2\Cdp}% 前アキ
    % {.5\Cvs \@plus.3\Cdp}% 後アキ
    {.5\Cvs}
  %  {\normalfont\large\headfont\bfseries\centering}} %変更
    {\normalfont\large\headfont\centering}} %変更

\renewcommand{\subsubsection}{\@startsection{subsubsection}{3}{\z@}%
  % {0\Cvs \@plus.8\Cdp \@minus.6\Cdp}%変更
    {1sp \@plus.5\Cdp \@minus.5\Cdp}%変更
    {\if@slide .5\Cvs \@plus.3\Cdp \else \z@ \fi}%
    % {\normalfont\medlarge\headfont\leftskip -1\zw}}
    {\normalfont\medlarge\headfont\leftskip -1\zw}}

\renewcommand{\paragraph}{\@startsection{paragraph}{4}{\z@}%
    {0.5\Cvs \@plus.5\Cdp \@minus.2\Cdp}%
    % {\if@slide .5\Cvs \@plus.3\Cdp \else -1\zw\fi}% 改行せず 1\zw のアキ
    {1sp}%後アキ
    {\normalfont\normalsize\headfont}}
\renewcommand{\subparagraph}{\@startsection{subparagraph}{5}{\z@}%
    {\z@}{\if@slide .5\Cvs \@plus.3\Cdp \else -.5\zw\fi}%
    {\normalfont\normalsize\headfont\hskip-.5\zw\noindent}}  

\newcommand{\frchap}[1]{\vspace*{-2ex}%
 \begin{center}\normalfont\headfont\LARGE\setstretch{0.8}
 \scshape#1\normalfont\normalsize
\end{center}\vspace{0.5\zw}\setstretch{1.0}}

\newcommand{\frsec}[1]{\vspace*{-2ex}%
 \begin{center}\normalfont\headfont\large\setstretch{0.8}
 \scshape#1\normalfont\normalsize
\end{center}\vspace{0.5\zw}\setstretch{1.0}}
  
\newcommand{\frsecb}[1]{\vspace*{-2ex}%
\begin{center}\normalfont\headfont\medlarge\setstretch{0.8}%
  \hspace{1em}\scshape#1\normalfont\normalsize%
\end{center}\vspace{0.5\zw}\setstretch{1.0}}
\makeatother

\newenvironment{frchapenv}{\vspace*{-2ex}\begin{center}\normalfont\headfont%
    \LARGE\setstretch{0.8}\normalfont\normalsize\scshape%
    }{\end{center}\vspace{0.5\zw}\setstretch{1.0}}

\newenvironment{frsecenv}{\vspace*{-2ex}%
\begin{center}\normalfont\headfont\medlarge\setstretch{0.8}%
  \hspace{1em}\normalfont\normalsize\scshape%
}{\end{center}\vspace{0.5\zw}\setstretch{1.0}}

\newenvironment{frsecbenv}{\vspace*{-2ex}%
\begin{center}\normalfont\headfont\medlarge\setstretch{0.8}%
  \hspace{1em}\normalfont\normalsize\scshape%
}{\end{center}\vspace{0.5\zw}\setstretch{1.0}}

\newenvironment{frsecenv}{\vspace*{-2ex}%
  \begin{center}\normalfont\headfont\large\setstretch{0.8}\normalfont\normalsize\scshape}%
  {\end{center}\vspace{0.5\zw}\setstretch{1.0}}

\newenvironment{frsubenv}{\begin{spacing}{0.2}\setlength{\leftskip}{-1\zw}\bfseries}{\end{spacing}\normalfont\normalsize\setlength{\leftskip}{0pt}\par\vspace{1.1\zw}}

\newcommand{\frsub}[1]{\begin{frsubenv}#1\end{frsubenv}\par\vspace{1.1\zw}}


\renewcommand{\thechapter}{}
\renewcommand{\thesection}{\hskip-1\zw}
\renewcommand{\thesubsection}{}
\renewcommand{\thesubsubsection}{}
\renewcommand{\theparagraph}{}


% \makeatletter
% \@removefromreset{subsubsection}{subsection}
% \def\thesubsubsection{\arabic{subsubsection}.}
% \newcounter{rnumber}
% \renewcommand{\thernumber}{\refstepcounter{rnumber} }
\makeatother
\renewcommand{\prepartname}{\if@english Part~\else {}\fi}
\renewcommand{\postpartname}{\if@english\else {}\fi}
\renewcommand{\prechaptername}{\if@english Chapter~\else {}\fi}
\renewcommand{\postchaptername}{\if@english\else {}\fi}
\renewcommand{\presectionname}{}%  第
\renewcommand{\postsectionname}{}% 節

%\makeatletter
\def\ps@headings{%
  \let\@oddfoot\@empty
  \let\@evenfoot\@empty
  \def\@evenhead{%
    \if@mparswitch \hss \fi
    \underline{\hbox to \fullwidth{\ltjsetparameter{autoxspacing={true}}
%      \textbf{\thepage}\hfil\leftmark}}%
       \normalfont\thepage\hfill\scshape\small\leftmark\normalfont}}%
    \if@mparswitch\else \hss \fi}%
  \def\@oddhead{\underline{\hbox to \fullwidth{\ltjsetparameter{autoxspacing={true}}
        {\if@twoside\scshape\small\rightmark\else\scshape\small\leftmark\fi}\hfil\thepage\normalfont}}\hss}%
  \let\@mkboth\markboth
  \def\chaptermark##1{\markboth{%
    \ifnum \c@secnumdepth >\m@ne
      \if@mainmatter
        \if@omit@number\else
          \@chapapp\thechapter\@chappos\hskip1\zw
        \fi
      \fi
    \fi
    ##1}{}}%
  \def\sectionmark##1{\markright{%
%    \ifnum \c@secnumdepth >\z@ \thesection \hskip1\zw\fi
    \ifnum \c@secnumdepth >\z@ \thesection \hskip-1\zw\fi
    ##1}}}%
\makeatother

\makeatletter
%%%%%%%% Lua GC
\patchcmd\@outputpage{\stepcounter{page}}{%
  \directlua{%
	if jit then
      local k = collectgarbage("count")
      if k>900000 then 
        collectgarbage("collect")
        texio.write_nl("term and log", "GC: ", math.floor(k), math.floor(collectgarbage("count")))
      end
	end
  }%
  \stepcounter{page}%
}{}{}
\makeatother

%\input{preamble/sources-index}%ソース名からの逆引きインデックス用コマンド

\makeatletter

%%%%%%基本ソース

\def\srcEspagnole#1#2#3#4{%
  \index[src]{espagnole@Espagnole!{#1}@{#2}}
  \index[src]{えすはによる@エスパニョル!{#3}@{#4}}}
 
\def\srcEspagnoleMaigre#1#2#3#4{%
  \index[src]{espagnole maigre@Espagnole maigre!{#1}@{#2}}%
  \index[src]{えすはによるさかな@エスパニョル(魚料理用)!{#3}@{#4}}}

\def\srcDemiGlace#1#2#3#4{%
  \index[src]{demi-glace@Demi-glace!{#1}@{#2}}%
  \index[src]{とうみくらす@ドゥミグラス!{#3}@{#4}}}

\def\srcJusDeVeauLie#1#2#3#4{%
  \index[src]{jus veau lie@Jus de veau lié!{#1}@{#2}}%
  \index[src]{とろみをつけたこうしのしゆ@とろみを付けた仔牛のジュ!{#3}@{#4}}}

\def\srcVeoute#1#2#3#4{%
  \index[src]{veloute@Velouté!{#1}@{#2}}%
  \index[src]{うるて@ヴルテ!{#3}@{#4}}}

\def\srcVeouteDeVolaille#1#2#3#4{%
  \index[src]{veloute de volaille@Velouté de volaille!{#1}@{#2}}%
  \index[src]{とりのうるて@鶏のヴルテ!{#3}@{#4}}}

\def\srcVeouteDePoisson#1#2#3#4{%
  \index[src]{veloute de poisson@Velouté de poisson!{#1}@{#2}}%
  \index[src]{さかなのうるて@魚のヴルテ!{#3}@{#4}}}

\def\srcAllemande#1#2#3#4{%
  \index[src]{allemande@Allemande!{#1}@{#2}}%
  \index[src]{あるまんと@アルマンド!{#3}@{#4}}}

\def\srcSupreme#1#2#3#4{%
  \index[src]{supreme@Suprême!{#1}@{#2}}%
  \index[src]{しゆふれーむ@シュプレーム!{#3}@{#4}}}

\def\srcBechamel#1#2#3#4{%
  \index[src]{bechamel@Bechamel!{#1}@{#2}}%
  \index[src]{へしやめる@ベシャメル!{#3}@{#4}}}

\def\srcTomate#1#2#3#4{%
  \index[src]{tomate@Tomate!{#1}@{#2}}%
  \index[src]{とまと@トマト!{#3}@{#4}}}

%%%%%%%%ブラウン系の派生ソース

\def\srcBigarade#1#2#3#4{%
  \index[src]{bigarade@Bigarade!{#1}@{#2}}%
  \index[src]{ひからーと@ビガラード!{#3}@{#4}}}

\def\srcBordelaise#1#2#3#4{%
  \index[src]{bordelaise@Bordelaise!{#1}@{#2}}%
  \index[src]{ほるとーふう@ボルドー風!{#1}@{#2}}}

\def\srcBourguignonne#1#2#3#4{%
  \index[src]{bourguignonne@Bourguignonne!{#1}@{#2}}%
  \index[src]{ふるこーにゆふう@ブルゴーニュ風!{#3}@{#4}}}

\def\srcBretonne#1#2#3#4{%
  \index[src]{bretonne@Bretonne!{#1}@{#2}}%
  \index[src]{ふるたーにゆふう@ブルターニュ風!{#3}@{#4}}}

\def\srcCerises#1#2#3#4{%
  \index[src]{cerises@Cerises (aux)!{#1}@{#2}}%
  \index[src]{すりーす@スリーズ!{#3}@{#4}}}

\def\srcChampignons#1#2#3#4{%
  \index[src]{champignons@Champignons (aux)!{#1}@{#2}}%
  \index[src]{しやんひによん@シャンピニョン!{#3}@{#4}}}

\def\srcChampignons#1#2#3#4{%
  \index[src]{charcutiere@Charcutière!{#1}@{#2}}%
  \index[src]{しやるきゆていえーる@シャルキュティエール!{#3}@{#4}}}

\def\srcChasseur#1#2#3#4{%
  \index[src]{chasseur@Chasseur!{#1}@{#2}}%
  \index[src]{しやすーる@シャスール!{#3}@{#4}}}

\def\srcChasseurEscoffier#1#2#3#4{%
  \index[src]{しやすーるえすこふいえ@シャスール(エスコフィエ)!{#1}@{#2}}%
  \index[src]{chasseur escoffier@Chasseur (Escoffier)!{#3}@{#4}}}

\def\srcChaudFroidBrune#1#2#3#4{%
  \index[src]{chaud-froid brune@Chaud-froid brune!{#1}@{#2}}%
  \index[src]{しよふろわしやいろ@ショフロワ(茶色)!{#3}@{#4}}}

\def\srcChaudFroidBruneCanard#1#2#3#4{%
  \index[src]{chaud-froid brune canards@Chaud-froid brune pour Canards!{#1}@{#2}}%
  \index[src]{しよふろわちやいろかも@ショフロワ(茶色、鴨用)!{#3}@{#4}}}

\def\srcChaudFroidBruneGibier#1#2#3#4{%
  \index[src]{chaud-froid brune gibier@Chaud-froid brune pour Gibier!{#1}@{#2}}%
  \index[src]{しよふろわちやいろしひえ@ショフロワ(茶色、ジビエ用)!{#3}@{#4}}}

\def\srcChaudFroidTomatee#1#2#3#4{%
  \index[src]{chaud-froid tometee@Chaud-froid tomatée!{#1}@{#2}}%
  \index[src]{しよふろわとまて@トマト入りショフロワ!{#3}@{#4}}}

\def\srcChevreuil#1#2#3#4{%
  \index[src]{chevreuil@Chevreuil!{#1}@{#2}}%
  \index[src]{しゆうるいゆ@シュヴルイユ!{#3}@{#4}}}

\def\srcColbert#1#2#3#4{%
  \index[src]{colbert@Colbert!{#1}@{#2}}%
  \index[src]{こるへーる@コルベール!{#3}@{#4}}}

\def\srcDiable#1#2#3#4{%
  \index[src]{diable@Diable!{#1}@{#2}}%
  \index[src]{ていあーふる@ディアーブル!{#3}@{#4}}}

\def\srcDiableEscoffier#1#2#3#4{%
  \index[src]{diable escoffier@Diable Escoffier!{#1}@{#2}}%
  \index[src]{ていあーふるえすこふいえ@ディアーブル・エスコフィエ!{#3}@{#4}}}

\def\srcDiane#1#2#3#4{%
  \index[src]{diane@Diane!{#1}@{#2}}%
  \index[src]{ていあーぬ@ディアーヌ!{#3}@{#4}}}

\def\srcDuxelles#1#2#3#4{%
  \index[src]{duxelles@Duxelles!{#1}@{#2}}%
  \index[src]{てゆくせる@デュクセル!{#3}@{#4}}}

\def\srcEstragon#1#2#3#4{%
  \index[src]{estragon@Estragon!{#1}@{#2}}%
  \index[src]{えすとらこん@エストラゴン!{#3}@{#4}}}

\def\srcFinanciere#1#2#3#4{%
  \index[src]{financierel@Financière!{#1}@{#2}}%
  \index[src]{ふいなんしえーる@フィナンシエール!{#3}@{#4}}}

\def\srcFinesHerbes#1#2#3#4{%
  \index[src]{fines herbes@Fines herbes (aux)!{#1}@{#2}}%
  \index[src]{こうそう@香草!{#3}@{#4}}}

\def\srcGenevoise#1#2#3#4{%
  \index[src]{genevoise@Genevoise!{#1}@{#2}}%
  \index[src]{しゆねーうふう@ジュネーヴ風!{#3}@{#4}}}

\def\srcGodard#1#2#3#4{%
  \index[src]{godard@Godard!{#1}@{#2}}%
  \index[src]{こたーる@ゴダール!{#3}@{#4}}}

\def\srcGrandVeneur#1#2#3#4{%
  \index[src]{grand-veneur@Grand-Veneur!{#1}@{#2}}%
  \index[src]{くらんうぬーる@グランヴヌール!{#3}@{#4}}}

\def\srcGrandVeneurEscoffier#1#2#3#4{%
  \index[src]{grand-veneur escoffier@Grand-Veneur Escoffier!{#1}@{#2}}%
  \index[src]{くらんうぬーるえすこふぃえ@グランヴヌール(エスコフィエ)!{#3}@{#4}}}

\def\srcGratin#1#2#3#4{%
  \index[src]{gratin@Gratin!{#1}@{#2}}%
  \index[src]{くらたん@グラタン!{#3}@{#4}}}

\def\srcHachee#1#2#3#4{%
  \index[src]{hachee@Hachée!{#1}@{#2}}%
  \index[src]{あしえ@アシェ!{#3}@{#4}}}

\def\srcHacheeMaigre#1#2#3#4{%
  \index[src]{hachee maigrel@Hachée maigre!{#1}@{#2}}%
  \index[src]{あしえさかな@アシェ(魚料理用)!{#3}@{#4}}}

\def\srcHussarde#1#2#3#4{%
  \index[src]{hussarde@Hussarde!{#1}@{#2}}%
  \index[src]{ゆさると@ユサルド!{#3}@{#4}}}

\def\srcItalienne#1#2#3#4{%
  \index[src]{italienne@Italienne!{#1}@{#2}}%
  \index[src]{いたりあふう@イタリア風!{#3}@{#4}}}

\def\srcJusLieEstragon#1#2#3#4{%
  \index[src]{jus lie a l'estragon@Jus lié à l'Estragon!{#1}@{#2}}%
  \index[src]{とろみをつけたしゆえすとらこん@とろみを付けたジュ・エストラゴン風味!{#3}@{#4}}}

\def\srcJusLieTomate#1#2#3#4{%
  \index[src]{jus lie tomate@Jus lié tomaté!{#1}@{#2}}%
  \index[src]{とろみをつけたしゆとまと@とろみを付けたジュ・トマト風味!{#3}@{#4}}}

\def\srcLyonnaise#1#2#3#4{%
  \index[src]{lyonnaise@Lyonnaise!{#1}@{#2}}%
  \index[src]{りよんふう@リヨン風!{#3}@{#4}}}

\def\srcMadere#1#2#3#4{%
  \index[src]{madere@Madère!{#1}@{#2}}%
  \index[src]{まてーる@マデール!{#3}@{#4}}}

\def\srcMatelote#1#2#3#4{%
  \index[src]{matelote@Matelote!{#1}@{#2}}%
  \index[src]{まとろつと@マトロット!{#3}@{#4}}}

\def\srcMoelle#1#2#3#4{%
  \index[src]{moelle@Moelle!{#1}@{#2}}%
  \index[src]{もわる@モワル!{#3}@{#4}}}

\def\srcMoscovite#1#2#3#4{%
  \index[src]{moscovite@Moscovite!{#1}@{#2}}%
  \index[src]{もすくわふう@モスクワ風!{#3}@{#4}}}

\def\srcPerigueux#1#2#3#4{%
  \index[src]{perigueux@Périgueux!{#1}@{#2}}%
  \index[src]{へりくー@ペリグー!{#3}@{#4}}}

\def\srcPerigourdine#1#2#3#4{%
  \index[src]{perigourdine@Périgourdine!{#1}@{#2}}%
  \index[src]{へりくるていーぬ@ペリグルディーヌ!{#3}@{#4}}}

\def\srcPiquante#1#2#3#4{%
  \index[src]{piquante@Piquante!{#1}@{#2}}%
  \index[src]{ひかんと@ピカント!{#3}@{#4}}}

\def\srcPoivradeOrdinaire#1#2#3#4{%
  \index[src]{poivrade ordinaire@Poivrade ordinaire!{#1}@{#2}}%
  \index[src]{ほわふらーとひようしゆん@ポワヴラード(標準)!{#3}@{#4}}}

\def\srcPoivradeGibier#1#2#3#4{%
  \index[src]{poivrade gibier@Poivrade pour Gibier!{#1}@{#2}}%
  \index[src]{ほわふらーとしひえ@ポワヴラード(ジビエ用)!{#3}@{#4}}}

\def\srcPorto#1#2#3#4{%
  \index[src]{porto@Porto!{#1}@{#2}}%
  \index[src]{ほると@ポルト!{#3}@{#4}}}

\def\srcPortugaise#1#2#3#4{%
  \index[src]{portugaise@Portugaise!{#1}@{#2}}
  \index[src]{ほるとかるふう@ポルトガル風!{#3}@{#4}}}

\def\srcProvencale#1#2#3#4{%
  \index[src]{provencale@Provençale!{#1}@{#2}}
  \index[src]{ふろうあんすふう@プロヴァンス風!{#3}@{#4}}}

\def\srcRegence#1#2#3#4{%
  \index[src]{regence@Régence!{#1}@{#2}}
  \index[src]{れしやんす@レジャンス!{#3}@{#4}}}

\def\srcRobert#1#2#3#4{%
  \index[src]{robert@Robert!{#1}@{#2}}
  \index[src]{ろへーる@ロベール!{#3}@{#4}}}

\def\srcRobertEscoffier#1#2#3#4{%
  \index[src]{robert escoffier@Robert Escoffier!{#1}@{#2}}
  \index[src]{ろへーるえすこふいえ@ロベール・エスコフィエ!{#3}@{#4}}}

\def\srcRomaine#1#2#3#4{%
  \index[src]{romaine@Romaine!{#1}@{#2}}
  \index[src]{ろーまふう@ローマ風!{#3}@{#4}}}

\def\srcRouennaise#1#2#3#4{%
  \index[src]{rouennaise@Rouennaise!{#1}@{#2}}
  \index[src]{るーあんふう@ルーアン風!{#3}@{#4}}}

\def\srcSalmis#1#2#3#4{%
  \index[src]{salmis@Salmis!{#1}@{#2}}
  \index[src]{さるみ@サルミ!{#3}@{#4}}}

\def\srcTortue#1#2#3#4{%
  \index[src]{tortue@Tortue!{#1}@{#2}}
  \index[src]{とるちゆ@トルチュ!{#3}@{#4}}}

\def\srcVenaison#1#2#3#4{%
  \index[src]{venaison@Venaison!{#1}@{#2}}
  \index[src]{うねそん@ヴネゾン!{#3}@{#4}}}

\def\srcVinRouge#1#2#3#4{%
  \index[src]{vin rouge@Vin rouge (au)!{#1}@{#2}}
  \index[src]{あかわいん@赤ワイン!{#3}@{#4}}}


\def\srcZingaraA#1#2#3#4{%
  \index[src]{zingara a@Zingara A!{#1}@{#2}}
  \index[src]{さんからa@ザンガラ A!{#3}@{#4}}}

\def\srcZingaraB#1#2#3#4{%
  \index[src]{zingara b@Zingara B!{#1}@{#2}}
  \index[src]{さんからb@ザンガラ B!{#3}@{#4}}}


%%%%%%% ホワイト系の派生ソース

\def\srcAlbufera#1#2#3#4{%
  \index[src]{albufera@Albuféra!{#1}@{#2}}
  \index[src]{あるひゆふえら@アルビュフェラ!{#3}@{#4}}}

\def\srcAmericaine#1#2#3#4{%
  \index[src]{americaine@Américaien!{#1}@{#2}}
  \index[src]{あめりけーぬ@アメリケーヌ!{#3}@{#4}}}

\def\srcAnchois#1#2#3#4{%
  \index[src]{anchois@Anchois!{#1}@{#2}}%
  \index[src]{あんちよひ@アンチョビ!{#3}@{#4}}}

\def\srcAurore#1#2#3#4{%
  \index[src]{aurore@Aurore!{#1}@{#2}}%
  \index[src]{おーろーる@オーロール!{#3}@{#4}}}

\def\srcAuroreMaigre#1#2#3#4{%
  \index[src]{aurore maigre@Aurore maigre!{#1}@{#2}}%
  \index[src]{おーろーるさかな@オーロール(魚料理用)!{#3}@{#4}}}

\def\srcBavaroise#1#2#3#4{%
  \index[src]{bavaroise@Bavaroise!{#1}@{#2}}%
  \index[src]{はいえるんふう@バイエルン風!{#3}@{#4}}}

\def\srcBearnaise#1#2#3#4{%
  \index[src]{bearnaise@Béarnaise!{#1}@{#2}}%
  \index[src]{へあるねーす@ベアルネーズ!{#3}@{#4}}}

\def\srcBearnaiseTomatee#1#2#3#4{%
  \index[src]{bearnaise tomatee@Béarnaise tomatée!{#1}@{#2}}%
  \index[src]{へあるねーすとまと@ベアルネース(トマト入り)!{#3}@{#4}}}

\def\srcChoron#1#2#3#4{%
  \index[src]{choron@Choron!{#1}@{#2}}%
  \index[src]{しよろん@ショロン!{#3}@{#4}}}

\def\srcBearnaiseGlaceDeViande#1#2#3#4{%
  \index[src]{bearnaise glace de viande@Bearnaise à la glace de viande!{#1}@{#2}}%
  \index[src]{へあるねーすくらすとういあんと@ベアルネーズ(グラスドヴィアンド入り)!{#3}@{#4}}}

\def\srcFoyot#1#2#3#4{%
  \index[src]{foyot@Foyot!{#1}@{#2}}%
  \index[src]{ふおいよ@フォイヨ!{#3}@{#4}}}

\def\srcValois#1#2#3#4{%
  \index[src]{valois@Valois!{#1}@{#2}}%
  \index[src]{うあろわ@ヴァロワ!{#3}@{#4}}}

\def\srcBercy#1#2#3#4{%
  \index[src]{bercy@Bercy!{#1}@{#2}}%
  \index[src]{へるしー@ベルシー!{#3}@{#4}}}

\def\srcBeurre#1#2#3#4{%
  \index[src]{beurre@Beurre (au)!{#1}@{#2}}%
  \index[src]{ふーる@オ・ブール!{#3}@{#4}}}

\def\srcBatarde#1#2#3#4{%
  \index[src]{batarde@Batarde!{#1}@{#2}}%
  \index[src]{はたると@バタルド!{#3}@{#4}}}

\def\srcBonnefoy#1#2#3#4{%
  \index[src]{bonnefoy@Bonnefoy!{#1}@{#2}}%
  \index[src]{ほぬふおわ@ボヌフォワ!{#3}@{#4}}}

\def\srcBordelaiseVinBlanc#1#2#3#4{%
  \index[src]{bordelaise vin blanc@Bordelaise au vin blanc!{#1}@{#2}}%
  \index[src]{ほるとーふうしろわいん@ボルドー風(白ワイン)!{#3}@{#4}}}

\def\srcBretonneBlanche#1#2#3#4{%
  \index[src]{bretonne blanche@Bretonne (blanche)!{#1}@{#2}}%
  \index[src]{ふるたーにゆふうしろ@ブルターニュ風(ホワイト系)!{#3}@{#4}}}

\def\srcCanotiere#1#2#3#4{%
  \index[src]{canotiere@Canotière!{#1}@{#2}}%
  \index[src]{かのていえーる@カノティエール!{#3}@{#4}}}

\def\srcCapres#1#2#3#4{%
  \index[src]{capres@Câpres (aux)!{#1}@{#2}}%
  \index[src]{けいはー@ケイパー!{#3}@{#4}}}

\def\srcCardinal#1#2#3#4{%
  \index[src]{cardinl@Cardinal!{#1}@{#2}}%
  \index[src]{かるていなる@カルディナル!{#3}@{#4}}}

\def\src#1#2#3#4{%
  \index[src]{champignons blanche@Champignons (aux)(blanche)!{#1}@{#2}}%
  \index[src]{まつしゆるーむしろ@マッシュルーム(ホワイト系)!{#3}@{#4}}}

\def\srcChantilly#1#2#3#4{%
  \index[src]{chantilly@Chantilly!{#1}@{#2}}%
  \index[src]{しやんていい@シャンティイ!{#3}@{#4}}}

\def\srcChateaubriand#1#2#3#4{%
  \index[src]{chateaubriand@Chateaubriand!{#1}@{#2}}
  \index[src]{しやとーふりやん@シャトーブリヤン!{#3}@{#4}}}

\def\srcChaudFroidBlancheOrdinaire#1#2#3#4{%
  \index[src]{choud-froid blanche ordinaire@Chaud-froid blanche ordinaire!{#1}@{#2}}
  \index[src]{しよふろわしろひようしゆん@ショフロワ(白)(標準)!{#3}@{#4}}}

\def\srcChaudFroidBlonde#1#2#3#4{%
  \index[src]{choud-froid blonde@Chaud-froid blonde!{#1}@{#2}}%
  \index[src]{しよふろわふろんと@ショフロワ(ブロンド)!{#3}@{#4}}}

\def\srcChaudFroidAurore#1#2#3#4{%
  \index[src]{chaud-froid aurore@Chaud-froid Aurore!{#1}@{#2}}%
  \index[src]{しよふろわおーろーる@ショフロワ・オーロール!{#3}@{#4}}}

\def\srcChaudFroidVertPre#1#2#3#4{%
  \index{chaud-froid vert-pre@Chaud-froid Vert-pré!{#1}@{#2}}%
  \index[src]{しよふろわうえーるふれ@ショフロワ・ヴェールプレ!{#3}@{#4}}}

\def\srcChaudFroidMaigre#1#2#3#4{%
  \index[src]{chaud-froid maigre@Chaud-froid maigre!{#1}@{#2}}%
  \index[src]{しよふろわさかな@ショフロワ(魚料理用)!{#3}@{#4}}}

\def\srcChivry#1#2#3#4{%
  \index[src]{chivry@Chivry!{#1}@{#2}}%
  \index[src]{しうり@シヴリ!{#3}@{#4}}}

\def\srcCreme#1#2#3#4{%
  \index[src]{creme@Crème (à la)!{#1}@{#2}}%
  \index[src]{くれーむ@クレーム!{#3}@{#4}}}

\def\srcCrevettes#1#2#3#4{%
  \index[src]{crevettes@Crevettes (aux)!{#1}@{#2}}%
  \index[src]{くるうえつと@クルヴェット!{#3}@{#4}}}

\def\srcCurrie#1#2#3#4{%
  \index[src]{currie@Currie!{#1}@{#2}}%
  \index[src]{かれー@カレー!{#3}@{#4}}}

\def\srcCurrieIndienne#1#2#3#4{%
  \index[src]{currie indienne@Currie à l'Indienne!{#1}@{#2}}%
  \index[src]{いんとふうかれー@インド風カレー!{#3}@{#4}}}

\def\srcDiplomate#1#2#3#4{%
  \index[src]{diplomate@Diplomate!{#1}@{#2}}%
  \index[src]{ていふろまつと@ディプロマット!{#3}@{#4}}}

\def\srcEcossaise#1#2#3#4{%
  \index[src]{ecossaise@Ecossaise!{#1}@{#2}}%
  \index[src]{すこつとらんとふう@スコットランド風!{#3}@{#4}}}

\def\srcEstragon#1#2#3#4{%
  \index[src]{estragon@Estragon!{#1}@{#2}}%
  \index[src]{えすとらこん@エストラゴン!{#3}@{#4}}}

\def\srcFinesHerbes#1#2#3#4{%
  \index[src]{fines herbes blanche@Fines herbes blanche (aux)!{#1}@{#2}}%
  \index[src]{こうそうしろ@香草(ホワイト系)!{#3}@{#4}}}

\def\srcGroseilles#1#2#3#4{%
  \index[src]{groseilles@Groseilles!{#1}@{#2}}%
  \index[src]{くろせいゆ@グロゼイユ!{#3}@{#4}}}

\def\srcHollandaise#1#2#3#4{%
  \index[src]{hollandaise@Hollandaise!{#1}@{#2}}%
  \index[src]{おらんてーす@オランデーズ!{#3}@{#4}}}

\def\srcHomard#1#2#3#4{%
  \index[src]{homard@Homard!{#1}@{#2}}%
  \index[src]{おまーる@オマール!{#3}@{#4}}}

\def\srcHongroise#1#2#3#4{%
  \index[src]{hongroise@Hongroise!{#1}@{#2}}%
  \index[src]{はんかりーふう@ハンガリー風!{#3}@{#4}}}

\def\srcHuitres#1#2#3#4{%
  \index[src]{huitres@Huîtres (aux)!{#1}@{#2}}%
  \index[src]{かきいり@牡蠣入り!{#3}@{#4}}}

\def\srcIndienne#1#2#3#4{%
  \index[src]{indienne@Indienne!{#1}@{#2}}%
  \index[src]{いんとふう@インド風!{#3}@{#4}}}

\def\srcIvoire#1#2#3#4{%
  \index[src]{ivoire@Ivoire!{#1}@{#2}}%
  \index[src]{いうおわーる@イヴォワール!{#3}@{#4}}}

\def\srcJoinville#1#2#3#4{%
  \index[src]{joinville@Joinville!{#1}@{#2}}%
  \index[src]{しよわんういる@ジョワンヴィル!{#3}@{#4}}}

\def\srcLaguipiere#1#2#3#4{%
  \index[src]{laguipiere@Laguipière!{#1}@{#2}}%
  \index[src]{らきひえーる@ラギピエール!{#3}@{#4}}}

\def\srcLivonienne#1#2#3#4{%
  \index[src]{livonienne@Livonienne!{#1}@{#2}}%
  \index[src]{りうおにあふう@リヴォニア風!{#3}@{#4}}}

\def\srcMaltaise#1#2#3#4{%
  \index[src]{maltaise@maltaise!{#1}@{#2}}%
  \index[src]{まるたふう@マルタ風!{#3}@{#4}}}

\def\srcMariniere#1#2#3#4{%
  \index[src]{mariniere@Marinière!{#1}@{#2}}%
  \index[src]{まりにえーる@マリニエール!{#3}@{#4}}}

\def\srcMateloteBlanche#1#2#3#4{%
  \index[src]{matelote blanche@Matelote blanche!{#1}@{#2}}%
  \index[src]{まとろつとしろ@マトロット(白)!{#3}@{#4}}}

\def\srcMornay#1#2#3#4{%
  \index[src]{mornay@Mornay!{#1}@{#2}}%
  \index[src]{もるねー@モルネー!{#3}@{#4}}}

\def\srcMousseline#1#2#3#4{%
  \index[src]{mousseline@Mousseline!{#1}@{#2}}%
  \index[src]{むすりーぬ@ムスリーヌ!{#3}@{#4}}}

\def\srcMousseuse#1#2#3#4{%
  \index[src]{mousseuse@Mousseuse!{#1}@{#2}}%
  \index[src]{むすーす@ムスーズ!{#3}@{#4}}}

\def\srcMoutarde#1#2#3#4{%
  \index[src]{moutarde@Moutarde!{#1}@{#2}}%
  \index[src]{むたると@ムタルド!{#3}@{#4}}}

\def\srcNantua#1#2#3#4{%
  \index[src]{nantua@Nantua!{#1}@{#2}}%
  \index[src]{なんちゆあ@ナンチュア!{#3}@{#4}}}

\def\srcNewBurgCru#1#2#3#4{%
  \index[src]{new-burg cru@New-burg avec le homard cru!{#1}@{#2}}%
  \index[src]{にゆーはーくいけ@ニューバーグ(活けオマール)!{#3}@{#4}}}

\def\srcNewBurgCuit#1#2#3#4{%
  \index[src]{new-burg cuit@New-burg avec le homard cuit!{#1}@{#2}}%
  \index[src]{にゆーはーくゆて@ニューバーグ(茹でオマール)!{#3}@{#4}}}

\def\srcNoisette#1#2#3#4{%
  \index[src]{noisette@Noisette!{#1}@{#2}}%
  \index[src]{のわせつと@ノワゼット!{#3}@{#4}}}

\def\srcNormande#1#2#3#4{%
  \index[src]{normande@Normande!{#1}@{#2}}%
  \index[src]{のるまんていふう@ノルマンディ風!{#3}@{#4}}}

\def\srcOrientale#1#2#3#4{%
  \index[src]{orientale@Orientale!{#1}@{#2}}%
  \index[src]{おりえんとふう@オリエント風!{#3}@{#4}}}

\def\srcPaloise#1#2#3#4{%
  \index[src]{paloise@Paloise!{#1}@{#2}}%
  \index[src]{ほーふう@ポー風!{#3}@{#4}}}  

\def\srcPoulette#1#2#3#4{%
  \index[src]{poulette@Poulette!{#1}@{#2}}%
  \index[src]{ふれつと@プレット!{#3}@{#4}}}

\def\srcRavigote#1#2#3#4{%
  \index[src]{ravigote@Ravigote!{#1}@{#2}}%
  \index[src]{らういこつと@ラヴィゴット!{#3}@{#4}}}

\def\srcRegencePoisson#1#2#3#4{%
  \index[src]{regence poisson@Régence pour Poissons!{#1}@{#2}}%
  \index[src]{れしやんすさかなよう@レジャンス(魚料理用)!{#3}@{#4}}}

\def\srcRegenceGarnituresVolaille#1#2#3#4{%
  \index[src]{regence garnitures volaille@Régence pour garnitures de Volaille!{#1}@{#2}}%
  \index[src]{れしやんすとりりようりのかるにちゆーるよう@レジャンス(鶏料理のガルニチュール用)!{#3}@{#4}}}

\def\srcRiche#1#2#3#4{%
  \index[src]{riche@Riche!{#1}@{#2}}%
  \index[src]{りつしゆ@リッシュ!{#3}@{#4}}}

\def\srcRubens#1#2#3#4{%
  \index[src]{rubens@Rubens!{#1}@{#2}}%
  \index[src]{るーへんす@ルーベンス!{#3}@{#4}}}

\def\srcSaintMalo#1#2#3#4{%
  \index[src]{saint-malo@Saint-Malo!{#1}@{#2}}%
  \index[src]{さんまろふう@サンマロ風!{#3}@{#4}}}

\def\srcSmitane#1#2#3#4{%
  \index[src]{smitane@Smitane!{#1}@{#2}}%
  \index[src]{すみたーぬ@スミターヌ!{#3}@{#4}}}

\def\srcSolferino#1#2#3#4{%
  \index[src]{solferino@Solférino!{#1}@{#2}}%
  \index[src]{そるふえりの@ソルフェリノ!{#3}@{#4}}}

\def\srcSoubise#1#2#3#4{%
  \index[src]{soubise@Soubise!{#1}@{#2}}
  \index[src]{すひーす@スビーズ!{#3}@{#4}}}

\def\srcSoubiseTomatee#1#2#3#4{%
  \index[src]{soubise tomatee@Soubise tomatée!{#1}@{#2}}%
  \index[src]{すひーすとまといり@スビース(トマト入り)!{#3}@{#4}}}

\def\srcSouchet#1#2#3#4{%
  \index[src]{souchet@Souchet!{#1}@{#2}}%
  \index[src]{すーしえ@スーシェ!{#3}@{#4}}}

\def\srcTyrolienne#1#2#3#4{%
  \index[src]{tyrolienne@Tyrolienne!{#1}@{#2}}%
  \index[src]{ちろるふう@チロル風!{#3}@{#4}}}

\def\srcTyrolienneAncienne#1#2#3#4{%
  \index[src]{tyroienne ancienne@!Tyrolienne à l'ancienne!{#1}@{#2}}%
  \index[src]{ちろるふうくらしつく@チロル風・クラシック!{#3}@{#4}}}

\def\srcValois#1#2#3#4{%
  \index[src]{valois@Valois!{#1}@{#2}}%
  \index[src]{うあろわ@ヴァロワ!{#3}@{#4}}}

\def\srcVenitienne#1#2#3#4{%
  \index[src]{venitienne@Vénitienne!{#1}@{#2}}%
  \index[src]{うえねついあふう@ヴェネツィア風!{#3}@{#4}}}

\def\srcVeron#1#2#3#4{%
  \index[src]{veron@Véron!{#1}@{#2}}%
  \index[src]{うえろん@ヴェロン!{#3}@{#4}}}

\def\srcVillageoise#1#2#3#4{%
  \index[src]{villageoise@Villageoise!{#1}@{#2}}%
  \index[src]{むらひとふう@村人風!{#3}@{#4}}}

\def\srcVilleroy#1#2#3#4{%
  \index[src]{villeroy@Villeroy!{#1}@{#2}}%
  \index[src]{ういるろわ@ヴィルロワ!{#3}@{#4}}}

\def\srcVilleroySoubisee#1#2#3#4{%
  \index[src]{villeroy soubisee@Villeroy soubisée!{#1}@{#2}}%
  \index[src]{ういるろわすひーすいり@ヴィルロワ(スビーズ入り)!{#3}@{#4}}}

\def\srcVilleroyTomatee#1#2#3#4{%
  \index[src]{villeroy tomatee@Villeroy tomatée!{#1}@{#2}}%
  \index[src]{ういるろわとまといり@ヴィルロワ(トマト入り)!{#3}@{#4}}}

\def\srcVinBlanc#1#2#3#4{%
  \index[src]{vin blanc@Vin blanc!{#1}@{#2}}%
  \index[src]{しろわいん@白ワイン!{#3}@{#4}}}

%%%%イギリス風ソース(温製)



\makeatother


%% レイアウト調整(A4Paper,14Q,twoside,ltjsbook.cls) 
%%
\setlength{\hoffset}{0\zw}
\setlength{\oddsidemargin}{0\zw}%タブレット前提の中央配置
\setlength{\evensidemargin}{\oddsidemargin}
% \setlength{\oddsidemargin}{1\zw}%製本時に右ページのみをオフセット
%\setlength{\evensidemargin}{0pt}%
\setlength{\fullwidth}{45\zw}
\setlength{\textwidth}{45\zw}%%ltjsclassesのみ有効
%\setlength{\fullwidth}{159mm}
%\setlength{\textwidth}{159mm}
\setlength{\marginparsep}{0pt}
\setlength{\marginparwidth}{0pt}
\setlength{\footskip}{0pt}
\setlength{\voffset}{-17mm}
\setlength{\textheight}{265mm}
\setlength{\parskip}{0pt}
\setlength{\parindent}{0pt}

\newcommand{\atoaki}{\vspace{1.25mm}}

%%分数の表記Obsolete
\usepackage{xfrac}
\let\frac\sfrac





\begin{document}
\maketitle

\begin{Main}

\hypertarget{grandes-sauces-de-base}{%
\section{基本ソース}\label{grandes-sauces-de-base}}

\begin{frsecenv}

Grandes Sauces de Base

\end{frsecenv}

\index{そーす@ソース!きほん@\textbf{基本---}|(}
\index{sauce@sauce!grandes@\textbf{Grandes ---s de Base}|(}

\end{Main}

\begin{recette}

\hypertarget{sauce-espagnole}{%
\subsubsection[ソース・エスパニョル]{\texorpdfstring{ソース・エスパニョル\footnote{本節冒頭では、ルーがスペインの料理人によってもたらされ、その結果としてソース・エスパニョルが作られるようになったと読める記述があるが、これはむしろ誤りと考えるべき。エスパニョル
  espagnol(e)は「スペイン(風)の」意だが、スペイン料理起源というわけでもない。スペインを想起させるトマトを使うから、あるいは、ソースが茶褐色なのがムーア系スペイン人を想起させるから、など定説はない。カレーム『19世紀フランス料理』第3巻に収められたソース・エスパニョルの作り方は、フォンをとるところから始まり4ページにわたって詳細なものとなっている(pp.8-11)。その中で、肉を入れた鍋に少量のブイヨンを注いで煮詰めることを繰り返す。ここまでは18世紀の料理書で一般的な手法であるが、その後に大量のブイヨンを注いだ後、いきなり強火にかけるのではなく、弱火で加熱していくやり方を「スペイン式の方法」と述べている。カレームにおいては、これがソースの名称の根拠のひとつになっていると考えていいだろう。もちろん、ソース・エスパニョルという名称のソースはカレーム以前からあり、1806年刊のヴィアール『帝国料理の本』にもカレームのレシピより簡単だが、ほぼ同様のものが基本ソースとして収録されている。また、それ以前にもソース・エスパニョルに類する名称のソースはあったが、たとえば1739年刊ムノン『新料理研究』第2巻にある「スペイン風ソース」はかなり趣きが異なる(コリアンダーひと把みを加えるのが特徴的)。同じ料理名でも時代や料理書の著者によってまったく違う料理になっていることは、食文化史において珍しいことではない。また、とりわけ料理名に地名、国名が冠されているものの中には根拠や由来のはっきりしないものも多い。いずれにしても、本書のソース・エスパニョルの源流は19世紀初頭のヴィアールあたりからと考えられる。ソース・エスパニョルは19世紀を通して普及し、茶色いソースの代表的な存在となった。こんにちでもフォンドヴォーをベースとしたソースは、ルーでとろみ付けこそしないが、仔牛の骨などから出るコラーゲンによるとろみを利用したもので、仕上がりの色合いや、ごく標準的ともいえる風味付けの方法などが引継がれ続けている調理現場も少なくない。もっとも、上述のように本書では「茶色いルー」を使うところに「エスパニョル」であることの理由を見い出そうとしていると解釈される。}}{ソース・エスパニョル}}\label{sauce-espagnole}}

\begin{frsubenv}

Sauce espagnole

\end{frsubenv}

\index{そーす@ソース!えすはによる@---・エスパニョル}
\index{えすはによる@エスパニョル!そーす@ソース・---}
\index{すへいんふう@スペイン風(エスパニョル)!そーすえすはによる@ソース・エスパニョル}
\index{sauce@sauce!espagnole@--- Espagnole}
\index{espagnol@espagnol!sauce@Sauce ---e}
\index[src]{えすはによる@エスパニョル} \index[src]{espagnole@Espagnole}

(仕上がり5 L分)

\begin{itemize}
\item
  とろみ付けのための\protect\hyperlink{roux-brun}{ルー}\ldots{}\ldots{}625
  g。
\item
  \protect\hyperlink{fonds-brun}{茶色いフォン}(ソースを仕上げるのに必要な全量)\ldots{}\ldots{}12
  L。
\item
  \protect\hyperlink{mirepoix}{ミルポワ}\footnote{mirepoix
    (ミルポワ)。ソースやフォンにコクを与える目的で、細かいさいの目に切った香味野菜や塩漬け豚ばら肉を合わせたもの。18世紀にミルポワ公爵の料理人が考案したといわれているが真偽は不明。同様のものにmatignon(マティニョン)がある。ミルポワより大きめのさいの目に切るのが一般的とされるが、調理現場によってはあまり区別せずミルポワとのみ呼称するケースも多い。第2章ガルニチュール、\protect\hyperlink{mirepoix}{ミルポワ}訳注参照。}(香味素材)\ldots{}\ldots{}小さなさいの目に切った塩漬け豚ばら肉150
  g、2 mm程度のさいの目\footnote{brunoise (ブリュノワーズ)。1〜2 mm
    のさいの目に切ること。 couper en
    mirepoix(クゥペオンミルポワ)ミルポワに切るとも言う。}に切ったにんじん250
  gと玉ねぎ 150
  g、タイム2枝、ローリエの葉2枚。\index{みるほわ@ミルポワ}\index{mirepoix}
\item
  作業手順
\end{itemize}

\begin{enumerate}
\def\labelenumi{\arabic{enumi}.}
\item
  フォン8
  Lを鍋で沸かす。あらかじめ柔らかくしておいたルーを加え、木杓子か泡立て器で混ぜながら沸騰させる。

  弱火にして\footnote{原文から直訳すると「鍋を火の脇に置く」だが、現代の調理環境では単純に「弱火にする」と解釈していい。}微沸騰の状態を保つ。
\item
  以下のようにしてあらかじめ用意しておいたミルポワを投入する。ソテー鍋に塩漬け豚ばら肉を入れて火にかけて脂を溶かす。そこに、細かく刻んだにんじんと玉ねぎ、タイム、ローリエの葉を加える。野菜が軽く色づくまで強火で炒める。丁寧に、余分な脂を捨てる。これをソースに加える。野菜を炒めたソテー鍋に白ワイン約100
  mL\footnote{原文 un verre de vin blanc
    (アンヴェールドヴァンブロン)。直訳すると「グラス1杯の白ワイン」だが、本書において
    un verre de 〜は「約1 dL=100 mL」と覚えておくといいだろう。}を加えてデグラセ
  \footnote{dégrasser
    鍋に粘液状になって付着している肉汁を酒類あるいは水で溶かし出してソースなどに利用すること。}し、それを半量まで煮詰める。これも同様にソースの鍋に加える。こまめに浮いてくる夾雑物を徹底的に取り除き\footnote{dépouiller
    デプイエ。前節「ルーの火入れについて」訳注参照。}ながら弱火で約1
  時間煮込む。
\item
  ソースをシノワ\footnote{小さな穴が多く空けられた円錐形で、取っ手の付いた漉し器の一種。金属製のものが主流。}で、ミルポワ野菜を軽く押しながら漉し、別の片手鍋に移す。フォン2
  Lを注ぎ足す。さらに2時間、微沸騰の状態を保ちながら煮込む。その後、陶製の鍋に移し、ゆっくり混ぜながら冷ます。
\item
  翌日、再び厚手の片手鍋に移してから、フォン2 Lとトマトピュレ1
  Lまたは同等の生のトマトつまり2
  kgを加える。トマトピュレを用いる場合は、あらかじめオーブンでほとんど茶色になるまで焼いておくといい。そうするとトマトピュレの酸味を抜くことが出来る。そうすればソースを澄ませる作業が楽になるし、ソースの色合いも温かそうで美しいものになる。ソースをヘラか泡立て器で混ぜながら強火で沸騰させる。弱火にして1時間微沸騰の状態を保つ。最後に、表面に浮いている不純物を、細心の注意を払いながら徹底的に取り除く。布で漉し、完全に冷めるまで、ゆっくり混ぜ続けること\footnote{この、ヘラなどでゆっくり混ぜながら冷ます作業を
    vanner
    (ヴァネ)すると呼ぶが、日本の調理現場ではあまり用いられていない。}。
\end{enumerate}

\hypertarget{nota-sauce-espagnole}{%
\subparagraph{【原注】}\label{nota-sauce-espagnole}}

ソース・エスパニョルで仕上げに不純物を取り除くのにかかる時間はいちがいには言えない。これは、ソースに用いるフォンの質次第で変わるからだ。

ソースにするフォンが上質なものであればある程、仕上げに不純物を取り除く作業は早く済む。そういう場合には、ソース・エスパニョルを5時間で作ることも無理ではない。

\atoaki{}

\hypertarget{sauce-espagnole-maigre}{%
\subsubsection[魚料理用ソース・エスパニョル]{\texorpdfstring{魚料理用ソース・エスパニョル\footnote{フランス語のソース名にあるmaigre(メーグル)はこの場合、一般的に「魚用、魚料理用」と訳すが、厳密には「小斉の際の料理用」となろう。小斉とは、カトリックで古くから特定の期間、曜日に肉類を断つ食事をする宗教的食習慣。日本の「お精進」とニュアンスは近いが、小斉においては忌避されるのは鳥獣肉のみであり、魚介や乳製品はいいとされた。こじつけのように、水鳥は水のものだから魚介扱いであり、またイルカも魚類として扱われていた。小斉が行なわれるのは復活祭の前46日間(四旬節、逆に言えばカーニバルの最終日マルディグラの翌日から46日)と、週に一度(多くの場合は金曜)であった。合計すると小斉が行なわれるのは年間100日近くもあり、中世から18世紀の料理人たちは小斉の宴席に供する料理に工夫を凝らしていた。この習慣は19世紀になるとだんだん廃れていき、エスコフィエの時代には、料理人に対して小斉のための料理を要求することは少なくなっていった。}}{魚料理用ソース・エスパニョル}}\label{sauce-espagnole-maigre}}

\begin{frsubenv}

Sauce espagnole maigre

\end{frsubenv}

\index{そーす@ソース!えすはによるるさかな@---・エスパニョル (魚料理用)}
\index{えすはによる@エスパニョル!そーすさかなよう@ソース・--- (魚料理用)}
\index{すへいんふう@スペイン風(エスパニョル)!そーすえすはによるさかな@ソース・エスパニョル(魚料理用)}
\index{sauce@sauce!espagnole maigre@--- Espagnole maigre}
\index{espagnol@espagnol!sauce maigre@Sauce Espagnole maigre}
\index[src]{さかなりようりようえすはによる@魚料理用エスパニョル}
\index[src]{espagnole maigre@Espagnole maigre}

(仕上がり5 L分)

\begin{itemize}
\item
  バターを用いて\footnote{初版〜第三版にかけては、茶色いルーを作るのに「バターまたは、きれいなグレスドマルミット(コンソメなどを作る際に表面に浮いてくる脂をすくい取って、不純物を漉し取ったものであり、基本的に獣脂)」を用いる、とある。上述のように、カトリックにおける「小斉」の場合、獣脂は忌避されたがバターなどの乳製品は許容された。そのため特に「バターを用いて作ったルー」という指定がなされ、第四版では茶色いルーに澄ましバターのみを使う旨が強調されたが、ここでは初版以来の記述がそのまま残っているために、やや冗長に思われる表現となっている。}作った\protect\hyperlink{roux-brun}{ルー}\ldots{}\ldots{}500
  g。
\item
  \protect\hyperlink{fumet-de-poisson}{魚のフュメ(フュメドポワソン)}(ソースを仕上げるために必要な全量)\ldots{}\ldots{}10
  L。
\item
  ミルポワ\ldots{}\ldots{}標準的なソース・エスパニョルと同じ\protect\hyperlink{mirepoix}{ミルポワ}野菜を同量と、塩漬け豚ばら肉の代わりにバターを用い、マッシュルームまたはマッシュルームの切りくず\footnote{champignons
    de Paris
    (シャンピニョンドパリ)いわゆるマッシュルームは、ガルニチュールなど料理の一部として提供する際に、トゥルネ
    tourner
    といって螺旋(らせん)状の切れ込みを入れて装飾したものを使う。その際に少なくない量、具体的には重量で15〜20%程度が「切りくず」として発生するのでこれを利用する。なお、tourner(トゥルネ)の原義は「回す」であり、包丁を持った側の手は動かさずに、材料のほうを回すようにして切れ目を入れたり、アーティチョークや果物などの皮を剥くことを意味する。}250
  gを加える。
\item
  作業手順\ldots{}\ldots{}標準的なソース・エスパニョルとまったく同様に作る。
\item
  加熱時間と不純物を取り除くのに必要な時間\ldots{}\ldots{}5時間。
\end{itemize}

仕上げに漉してから、標準的なソース・エスパニョルとまったく同様に、完全に冷めるまでゆっくり混ぜ続けること。

\atoaki{}

\hypertarget{observation-sauce-espagnole-maigre}{%
\subsubsection{魚料理用ソース・エスパニョル補足}\label{observation-sauce-espagnole-maigre}}

このソースを日常的な料理のベースとなる仕込みに含めるかどうかについては意見が分れるところだ。

普通のソース・エスパニョルは、つまるところ風味の点ではほとんどニュートラルなものだから、それに魚のフュメを加えれば、魚料理用ソース・エスパニョルとして充分に通用するだろう。どうしても上で挙げた魚料理用ソース・エスパニョルが必要になるのは、宗教的に厳格に小斉の決まりを守って料理を作る場合のみで、さすがにその場合は代用品などない。

\atoaki{}

\hypertarget{sauce-demi-glace}{%
\subsubsection[ソース・ドゥミグラス]{\texorpdfstring{ソース・ドゥミグラス\footnote{日本の洋食などで一般的な「デミグラス」あるいは「ドミグラス」」とはかなり異なった仕上りのソースであることに注意。ソース・エスパニョルの仕上げにあたって、徹底的に不純物を取り除くことを何度も強調しているのは、透き通った茶色がかった色合いの、なめらかなソースを目指すからであり、それをさらに徹底させるということは、透明度、なめらかさの面でさらに上を目指すということを意味するからだ。ちなみに、アメリカに本社のあるメーカーの「デミグラスソース」の缶詰はもっぱら日本で販売されている製品であり、ヨーロッパおよびアメリカでは同一ブランドに該当する商品は存在しないようだ。}}{ソース・ドゥミグラス}}\label{sauce-demi-glace}}

\begin{frsubenv}

Sauce demi-glace

\end{frsubenv}

\index{そーす@ソース!とうみくらす@---・ドゥミグラス}
\index{とうみくらす@ドゥミグラス!そーす@ソース・---}
\index{sauce@sauce!demi-glace@--- Demi-glace}
\index{demi-glace@demi-glace!sauce@sauce ---}
\index[src]{demi-glace@Demi-glace}
\index[src]{とうみくらす@ドゥミグラス}

一般に「ドゥミグラス」と呼ばれているものは、いったん仕上がった\protect\hyperlink{sauce-espagnole}{ソース・エスパニョル}をさらに、もうこれ以上は無理という位に徹底的に不純物を取り除いたもののことだ。

最後の仕上げに\protect\hyperlink{glace-de-viande}{グラスドヴィアンド}などを加える。風味付けに何らかの酒類\footnote{本書ではマデイラ酒(マデイラワイン、ポルトガルの酒精強化ワイン、すなわちブドウ果汁が酵母により醗酵している途中で蒸留酒を加えて醗酵を止める製法のもので、甘口のものが多い)が用いられることが多い。}を加えれば、当然ながらソースの性格も変わるので、最終的な使い途に応じて決めること。

\hypertarget{nota-sauce-demi-glace}{%
\subsubsection{【原注】}\label{nota-sauce-demi-glace}}

ソースの色合いを決めるワインを仕上げに加える際には、「火から外して」行なうこと。沸騰しているとワインの香りがとんでしまうからだ。

\atoaki{}

\hypertarget{jus-de-veau-lie}{%
\subsubsection{とろみを付けた仔牛のジュ}\label{jus-de-veau-lie}}

\begin{frsubenv}

Jus de veau lié

\end{frsubenv}

\index{しゆ@ジュ!こうしのしゆ@仔牛の---(とろみを付けた)}
\index{そーす@ソース!とろみをつけたこうしのしゆ@とろみを付けた仔牛のジュ}
\index{こうし@仔牛!とろみをつけたこうしのしゆ@とろみを付けた---のジュ}
\index{jus@jus!jus veau lie@--- de veau lié}
\index{veau@veau!jus lie@jus de --- lié}
\index{jus de veau lie@Jus de veau lié}
\index[src]{とろみをつけたこうしのしゆ@とろみを付けた仔牛のジュ}

(仕上がり1 L分)

\begin{itemize}
\item
  仔牛のフォン\ldots{}\ldots{}\protect\hyperlink{jus-de-veau-brun}{仔牛の茶色いフォン}
  4 L。
\item
  とろみ付け材料\ldots{}\ldots{}アロールート\footnote{allow-root
    南米産のクズウコンを原料とした良質のでんぷん。日本では入手が難しいこともあり、コーンスターチが用いられることがほとんど。}30
  g。
\item
  作業手順\ldots{}\ldots{}よく澄んだ仔牛のフォン4
  Lを強火にかけ、\(\frac{1}{4}\) 量つまり1 Lになるまで煮詰める。
\end{itemize}

大さじ数杯分の冷たいフォンでアロールートを溶く。これを沸騰している鍋に加える。1分程度だけ火にかけ続けたら、布で漉す。

\hypertarget{nota-jus-de-veau-lie}{%
\subparagraph{【原注】}\label{nota-jus-de-veau-lie}}

この、とろみを付けた仔牛のジュは、本書では頻繁に使う指示をしているが、必ず、しっかりした味で透き通った、きれいな薄茶色に仕上げること。

\atoaki{}

\hypertarget{veloute}{%
\subsubsection[ヴルテ(標準的な白いソース)]{\texorpdfstring{ヴルテ\footnote{velouté
  (ヴルテ)原義は「ビロードのように柔らかな、なめらかな」。日本ではベシャメルソースと混同されやすいが、内容がまったく異なるソースなので注意。}(標準的な白いソース)}{ヴルテ(標準的な白いソース)}}\label{veloute}}

\begin{frsubenv}

Velouté, ou sauce blanche graisse

\end{frsubenv}

\index{うるて@ヴルテ!ひようひゆんてきなそーすうるて@標準的なソース・---}
\index{そーす@ソース!うるてひようひゆん@ヴルテ(標準的な)}
\index{ふるーて@ブルーテ ⇒ ヴルテ} \index{veloute@velouté}
\index{veloute@velouté!sauce blanche grasse@sauce blanche grasse}
\index[src]{veloute@Velouté} \index[src]{うるて@ヴルテ}

(仕上がり5 L分)

\begin{itemize}
\item
  とろみ付けの材料\ldots{}\ldots{}バターを用いて作った\footnote{\protect\hyperlink{sauce-espagnole-maigre}{魚料理用ソース・エスパニョル}、訳注参照。}\protect\hyperlink{roux-blond}{ブロンドのルー}
  625 g。
\item
  よく澄んだ\protect\hyperlink{fonds-blanc-ordinaire}{仔牛の白いフォン}\ldots{}\ldots{}5
  L。
\item
  作業手順\ldots{}\ldots{}ルーをフォンに溶かし込む。フォンは冷たくても熱くてもいいが、フォンが熱い場合にはソースが充分なめらかになるよう注意して溶かすこと。混ぜながら沸騰させる。微沸騰の状態を保ちながら、浮いてくる不純物を完全に取り除いていく\footnote{dépouiller
    (デプイエ)。\protect\hyperlink{sauce-espagnole}{ソース・エスパニョル}、訳注参照。}。この作業はとりわけ細心の注意を払って行なうこと。
\item
  加熱時間と不純物を取り除く作業に必要な時間\ldots{}\ldots{}1時間半。
\end{itemize}

その後、ヴルテを布で漉す\footnote{ある程度濃度のある液体やピュレを布で漉す場合、昔は「二人がかりで行なう必要があり、それぞれが巻いた布の端を左手に持ち、右手に持った木杓子を使って圧し搾る」(『ラルース・ガストロノミーク』初版、
  1938年)という方法が一般的だった。}。陶製の鍋に移してゆっくり混ぜながら完全に冷ます。

\atoaki{}

\hypertarget{veloute-de-volaille}{%
\subsubsection{鶏のヴルテ}\label{veloute-de-volaille}}

\begin{frsubenv}

Velouté de volaille

\end{frsubenv}

\index{うるて@ヴルテ!とりのうるて@鶏の---(ヴルテドヴォライユ)}
\index{そーす@ソース!うるてとり@ヴルテ(鶏)}
\index{ふるーて@ブルーテ ⇒ ヴルテ}
\index{うおらいゆ@ヴォライユ!うるてとうおらいゆ@ヴルテドヴォライユ(鶏のヴルテ)}
\index{かきん@家禽!とりのうるて@鶏のヴルテ}
\index{veloute@velouté!volaille@--- de Volaille}
\index{sauce@sauce!veloute volaille@Velout\'e de Volaille}
\index[src]{veloute de volaille@Velouté de volaille}
\index[src]{とりのうるて@鶏のヴルテ}

このヴルテの作り方だが、上述の\protect\hyperlink{veloute}{標準的なヴルテ}と、材料比率と作業はまったく同じ。使用する液体として\protect\hyperlink{fonds-de-volaille}{鶏の白いフォン(フォンドヴォライユ)}を使う。

\atoaki{}

\hypertarget{veloute-de-poisson}{%
\subsubsection{魚料理用ヴルテ}\label{veloute-de-poisson}}

\begin{frsubenv}

Velouté de poisson

\end{frsubenv}

\index{うるて@ヴルテ!さかなうるて@魚料理用---}
\index{そーす@ソース!うるてさかな@ヴルテ(魚料理用)}
\index{veloute@velouté!poisson@--- de Poisson}
\index{sauce@sauce!veloute poisson@Velouté de Poisson}
\index[src]{veloute de poisson@Velouté de poisson}
\index[src]{さかなのうるて@魚のヴルテ}

ルーと液体の分量は標準的なヴルテとまったく同じだが、仔牛のフォンではなく\protect\hyperlink{fumet-de-poisson}{魚のフュメ}を用いて作る。

ただし、魚を素材として用いるストックはどれもそうだが、手早く作業すること。不純物を取り除く作業も20分程度にとどめること。その後、布で漉し、陶製の鍋に移してゆっくり混ぜながら完全に冷ます。

\atoaki{}

\hypertarget{sauce-allemande}{%
\subsubsection[ソース・アルマンド(パリ風ソース)]{\texorpdfstring{ソース・アルマンド(パリ風ソース\footnote{原書では「パリ風ソース(元ソース・アルマンド)」となっているが、後述のように、こんにちでもソース・アルマンドの名称のほうが一般的であるため、ここではSauce
  Parisienneの「訳語」としてソース・アルマンドをあてることとした。})}{ソース・アルマンド(パリ風ソース)}}\label{sauce-allemande}}

\begin{frsubenv}

Sauce parisienne (ex-Allemande)

\end{frsubenv}

\index{そーす@ソース!ぱりふう@パリ風--- ⇒ ---・アルマンド}
\index{はりふう@パリ風!そーす@---ソース ⇒ ---・アルマンド}
\index{といつふう@ドイツ風!そーす@ソース・アルマンド}
\index{あるまん@アルマン(ド)!そーす@ソース・アルマンド}
\index{sauce@sauce!parisienne@--- parisienne (ex-Allemande)}
\index{parisien@parisien!sauce@Sauce Parisienne = Sauce Allemande}
\index{allemand@allemand!sauce@Sauce allemande (--- Parisienne)}
\index[src]{allemande@Allemande} \index[src]{あるまんと@アルマンド}

(仕上がり1 L分)

これは、\protect\hyperlink{veloute}{標準的なヴルテ}に卵黄でとろみを付けたソース。

\begin{itemize}
\item
  標準的なヴルテ\ldots{}\ldots{}1 L。
\item
  追加素材\ldots{}\ldots{}卵黄5個、白いフォン(冷たいもの)
  \(\frac{1}{2}\)
  L、粗く砕いたこしょう1ひとつまみ、すりおろしたナツメグ少々、マッシュルームの煮汁2
  dL、レモン汁少々。
\item
  作業手順\ldots{}\ldots{}厚手のソテー鍋にマッシュルームの茹で汁と白いフォン、卵黄、粗く砕いたこしょう、ナツメグ、レモン汁を入れる。泡立て器でよく混ぜ、そこにヴルテを加える。火にかけて沸騰させ、強火で
  \$\frac{2}{3} 量になるまで、ヘラで混ぜながら煮詰める。
\end{itemize}

ヘラの表面がソースでコーティングされる状態になるまで煮詰めたら、布で漉す。

膜が張らないよう、表面にバターのかけらをいくつか載せてやり、湯煎にかけておく。

\begin{itemize}
\tightlist
\item
  仕上げ\ldots{}\ldots{}提供直前に、バター100 gを加えて仕上げる。
\end{itemize}

\hypertarget{nota-sauce-allemande}{%
\subparagraph{【原注】}\label{nota-sauce-allemande}}

ソース・アルマンド(ドイツ風)とも呼ばれるが、本書では「パリ風」の名称を採用した。そもそも「アルマンド」というの名称に正当性がないからだ。習慣としてそう呼ばれてきただけであって、明らかに理屈に合わない名称だ
\footnote{エスコフィエは普仏戦争に従軍した経歴があり、ドイツ嫌いとして知られていた。}。1883年に雑誌「料理技術」に某タヴェルネ氏が寄せた記事には、当時ある優秀な料理人がアルマンドなどという理屈に合わない名称を使うのはやめたという話が出ている。

こんにち既に「パリ風ソース」の名称を採用している料理長もいる。そう呼んだほうが好ましいわけだが、残念なことにまだ一般的にはなっていない
\footnote{エスコフィエの願いもむなしく、現代においてもソース・アルマンドの名称で定着している。この「全注解」においても以後は「ソース・アルマンド」と訳しているので注意されたい。なお、「ドイツ風」というソース名の由来について、ソースの淡い黄色がドイツ人に多い金髪を想起させるからだとカレームは述べている。}。

\atoaki{}

\hypertarget{sauce-supreme}{%
\subsubsection[ソース・シュプレーム]{\texorpdfstring{ソース・シュプレーム\footnote{suprême
  原義は「至高の」だが、料理においてはしばしば鶏や鴨の胸肉、白身魚のフィレなどを意味する。また、このソースのように、とくに意味もなくこの名を料理につけられているケースも多い。}}{ソース・シュプレーム}}\label{sauce-supreme}}

\begin{frsubenv}

Sauce supême

\end{frsubenv}

\index{そーす@ソース!そーすしゆふれーむ@---・シュプレーム}
\index{しゆふれーむ@シュプレーム!そーす@ソース・---}
\index{sauce@sauce!supreme@--- Suprême}
\index{supreme@suprême!sauce@Sauce ---} \index[src]{supreme@Suprême}
\index[src]{しゆふれーむ@シュプレーム}

\protect\hyperlink{veloute-de-volaille}{鶏のヴルテ}に生クリーム\footnote{フランスの生クリームのうち、料理でよく使われるのは、日本の生クリームにやや近い「クレーム・フレッシュ・パストゥリゼ」(低温殺菌した生クリームで乳脂肪分30〜38%)のほか、「クレーム・フレッシュ・エペス」(低温殺菌後に乳酸醗酵させたもので日本で一般的な生クリームより濃度がある)、「クレーム・ドゥーブル」(殺菌後に乳酸醗酵させたもので乳脂肪分40%程度でかなり濃度がある)などがある。}を加えてなめらかに仕上げ\footnote{monter
  モンテ。原義は「上げる、ホイップする」だが、ソースの仕上げの際などに、バターや生クリームを加えて、なめらかに仕上げることも「モンテ」の語を使用する場合が多い。}たもの。ソース・シュプレームは、正しく作った場合「白さの\ruby{際}{きわ}だったとても繊細な」仕上がりのものでなくてはいけない。

(仕上がり1 L分)

\begin{itemize}
\item
  鶏のヴルテ\ldots{}\ldots{}1 L。
\item
  追加素材\ldots{}\ldots{}\protect\hyperlink{fonds-de-volaille}{鶏の白いフォン}
  1 L、マッシュルームの茹で汁1 dL、良質な生クリーム2 \(\frac{1}{2}\)
  dL。
\item
  作業手順\ldots{}\ldots{}鍋に鶏のフォンとマッシュルームの茹で汁、鶏のヴルテを入れて強火にかけ、ヘラで混ぜながら、生クリームを少しずつ加え、煮詰めていく。このヴルテと生クリームを煮詰めたものの分量は、上で示した仕上がり
  1 Lのソース・シュプレームを作るには、 \(\frac{1}{3}\)
  量まで煮詰まっていなくてはならない。
\end{itemize}

布で漉し、仕上げに1 dLの生クリームとバター80
gを加えてゆっくり混ぜながら冷ますと、丁度最初のヴルテと同量になる。

\atoaki{}

\hypertarget{sauce-bechamel}{%
\subsubsection[ベシャメルソース]{\texorpdfstring{ベシャメルソース\footnote{17世紀にルイ14世のメートルドテルを務めたこともあるルイ・ベシャメイユLouis
  Béchameil(1630〜1703)の名が冠されているこのソースは、彼自身の創案あるいは彼に仕えていた料理人によるものという説もあったが真偽は疑わしい。17世紀頃の成立であることは確かだが、おそらくは古くからあったソースを改良したものに過ぎず、また、19世紀前半のカレームのレシピはヴルテを煮詰め、卵黄と煮詰めた生クリームでとろみを付けるというものだった。同様に1867年刊グフェ『料理の本』のレシピも、炒めた仔牛肉と野菜に小麦粉を振りかけてからブイヨン注ぎ、これを煮詰め、漉してから生クリームを加えるというものだった。}}{ベシャメルソース}}\label{sauce-bechamel}}

\begin{frsubenv}

Sauce Béchamel

\end{frsubenv}

\index{そーす@ソース!へしやめる@ベシャメル---}
\index{へしやめる@ベシャメル!そーす@---ソース}
\index{sauce@sauce!bechamel@--- Béchamel}
\index{bechamel@Béchamel (sauce)} \index[src]{bechamel@Bechamel}
\index[src]{へしやめる@ベシャメル}

(仕上がり 5 L分)

\begin{itemize}
\item
  \protect\hyperlink{roux-blanc}{白いルー}\ldots{}\ldots{}650 g。
\item
  使用する液体\ldots{}\ldots{}沸かした牛乳5 L。
\item
  追加素材\ldots{}\ldots{}白身で脂肪のない仔牛肉300
  gをさいの目に切り、みじん切りにした玉ねぎ(小)2個分とタイム1枝、粗く砕いたこしょう1つまみ、塩25
  g とバターを鍋に入れて蓋をし、色付かないように弱火で蒸し煮したもの。
\item
  作業手順\ldots{}\ldots{}沸かした牛乳でルーを溶く。混ぜながら沸騰させる。ここに、先に蒸し煮しておいた野菜と調味料、仔牛肉を加える。弱火で1時間煮込む。布で漉し\footnote{\protect\hyperlink{veloute}{ヴルテ}訳注参照。}、表面にバターのかけらをいくつか載せて膜が張らないようにする。肉類を絶対に使わない\footnote{小斉のこと。\protect\hyperlink{sauce-espagnole-maigre}{魚料理用ソース・エスパニョル}訳注参照。}で調理する必要がある場合は、仔牛肉を省き、香味野菜などは上記のとおりに作ること。
\end{itemize}

このソースは次のようなやり方をすると手早く作ることも出来る。沸かした牛乳に塩、薄切りにした玉ねぎ、タイム、粗く砕いたこしょう、ナツメグを加える。蓋をして弱火で10分煮る。これを漉してルーを入れた鍋の中に入れ、強火にかけて沸騰させる。その後15〜20分だけ煮込めばいい。

\atoaki{}

\hypertarget{sauce-tomate}{%
\subsubsection{トマトソース}\label{sauce-tomate}}

\begin{frsubenv}

Sauce tomate

\end{frsubenv}

\index{そーす@ソース!とまとそーす@トマト---}
\index{とまと@トマト!ソース@---ソース}
\index{sauce@sauce!tomate@--- tomate}
\index{tomate@tomate!sauce@Sauce ---} \index[src]{tomate@Tomate}
\index[src]{とまと@トマト}

(仕上がり5 L分)

\begin{itemize}
\item
  主素材\ldots{}\ldots{}トマトピュレ4 L、または生のトマト6 kg。
\item
  \protect\hyperlink{mirepoix}{ミルポワ}\ldots{}\ldots{}さいの目に切って下茹でしておいた塩漬け豚ばら肉140
  g 、1〜2 mm 角のさいの目に刻んだにんじん200 gと玉ねぎ150
  g、ローリエの葉 1枚、タイム1枝、バター100 g。
\item
  追加素材\ldots{}\ldots{}小麦粉150 g、白いフォン2 L、にんにく2片。
\item
  調味料\ldots{}\ldots{}塩20 g、砂糖30 g、こしょう1つまみ。
\item
  作業手順\ldots{}\ldots{}厚手の片手鍋で、塩漬け豚ばら肉を軽く色付くまで炒める。ミルポワの野菜を加え、野菜も色よく炒める。小麦粉を振りかける。ブロンド色になるまで炒めてから、トマトピュレまたは潰した生トマトと白いフォン、砕いたにんにく、塩、砂糖、こしょうを加える。
\end{itemize}

火にかけて混ぜながら沸騰させる。鍋に蓋をして弱火のオーブンに入れ1時間半〜2時間加熱する。

目の細かい漉し器または布で漉す。再度、火にかけて数分間沸騰させる。保存用の器に移し、ソースが空気に触れて表面に膜が張らないよう、バターのかけらを載せてやる。

\hypertarget{nota-sauce-tomate}{%
\subparagraph{【原注】}\label{nota-sauce-tomate}}

トマトピュレを使い、小麦粉は使わず、その他は上記のとおりに作ってもいい。漉し器か布で漉してから、充分な濃度になるまでしっかり煮詰めてやること。

\index{そーす@ソース!きほん@\textbf{基本---}|)}
\index{sauce@sauce!grandes@\textbf{Grandes ---s de Base}|)}

\end{recette}

\end{document}

\hypertarget{petites-sauces-brunes-composees}{%
\section{ブラウン系の派生ソース}\label{petites-sauces-brunes-composees}}

\frsec{Petites Sauces Brunes Composées}

\index{そーす@ソース!ふらうんはせい@ブラウン系の派生---}
\index{sauce@sauce!petites brunes composees@Petites ---s Brunes Composées}
\begin{recette}
\hypertarget{sauce-bigarade}{%
\subsubsection{ソース・ビガラード}\label{sauce-bigarade}}

\frsub{Sauce Bigarade}\footnote{ビガラードは本来、南フランスで栽培されるビターオレンジの一種。}

\index{そーす@ソース!ひからーと@---・ビガラード}
\index{そーす@ソース!ふらうんはせい@ブラウン系の派生---!ひからーと@---・ビガラード}
\index{ひからーと@ビガラード!そーす@ソース・---}
\index{sauce@sauce!petites brunes composees@Petites ---s Brunes Composées!bigarade@--- Bigarade}
\index{sauce@sauce!bigarade@--- Bigarade}
\index{bigarade@bigarade!sauce@Sauce ---}

\hypertarget{sauce-bigarade-pour-caneton-braise}{%
\subparagraph{仔鴨のブレゼ用}\label{sauce-bigarade-pour-caneton-braise}}

\ldots{}\ldots{}仔鴨をブレゼ\footnote{料理の仕立てとしてのブレゼはたんに「蒸し煮」することではない。原
  則的な手順をごく簡単に述べておく。厚めに輪切りにしたにんじんと玉ね
  ぎをバターまたはラードで炒め、ブーケガルニとともに鍋に入れる。表面
  を色よく焼き固めた肉を、脂身の少ない肉の場合には豚背脂のシートで包
  んで素材がぴったり入る大きさ鍋に入れ、\protect\hyperlink{fonds-brun}{茶色いフォン}
  を注ぎ、蓋をしてオーブンに入れ、微沸騰の状態を保つようにして煮込む。
  火が通ったら肉を取り出し、鍋に残った煮汁でソースを作る。詳細につい
  ては\protect\hyperlink{}{第7章 肉料理}参照。}した際の煮汁を漉してから浮き脂を取り除き\footnote{dégraisser
  デグレセ。}、煮詰める。 煮詰まったらさらに目の細かい布で漉し、ソース1
Lあたりオレンジ4個とレモ ン1個の搾り汁でのばす。

\hypertarget{sauce-bigarade-pour-caneton-poele}{%
\subparagraph{仔鴨のポワレ用}\label{sauce-bigarade-pour-caneton-poele}}

\ldots{}\ldots{}仔鴨をポワレ\footnote{ポワレについても簡単に述べておく。本書においてポワレは「フライパ
  ンで焼く」という意味で用いられることは決してない(フライパンで魚な
  どを焼くことをポワレと呼ぶようになったのは20世紀後半のこと)。本書
  では「ローストの一種」と定義されており(この点がカレームとはまった
  く異なる)、3〜4mm角に切った香味野菜(マティニョン)を生のまま鍋の
  底に入れ、その上に味付けをした肉を置く。溶かしバターをかけてから、
  蓋をして中火のオーブンに入れて蒸し焼きにする。時折様子を見て溶かし
  バターをかけてやること。肉に火が通ったら鍋から取り出し、\protect\hyperlink{jus-de-veau-brun}{茶色い仔
  牛のフォン}を注いで弱火にかけて10分程煮込み、
  マティニョンとして用いた野菜から風味を引き出してソースにする。これ
  がレシピにある「ポワレのフォン」となる。}のフォンから浮き脂を取り除き、でんぷんで軽くとろみ付け
する。砂糖20gに大さじ\undemi{}杯のヴィネガーを加えて火にかけカラメル状
にしたものを加える。ブレゼ用と同様に、オレンジとレモンの搾り汁でのばす。

仔鴨のブレゼ用、ポワレ用いずれの場合も、細かい千切りにしてよく下茹でし
ておいたオレンジの皮大さじ2とレモンの皮\footnote{柑橘類の表皮を薄く剥いてごく細い千切りにしたり、器具を用いてお
  ろしたものをzeste(ゼスト)と呼ぶ。千切りにしたものは苦味を取り除く
  ために下茹ですることが多い。}大さじ1を加えて仕上げる。

\hypertarget{sauce-bordelaise}{%
\subsubsection{ボルドー風ソース}\label{sauce-bordelaise}}

\frsub{Sauce Bordelaise}

\index{そーす@ソース!ほるとーふう@ボルドー風---}
\index{そーす@ソース!ふらうんはせい@ブラウン系の派生---!ほるとーふう@ボルドー風---}
\index{ほるとーふう@ボルドー風!そーす@---ソース}
\index{sauce@sauce!petites brunes composees@Petites ---s Brunes Composées!bordelaise@--- Bordelaise}
\index{sauce@sauce!bordelaise@--- Bordelaise}
\index{bordelais@bordelais(e)!sauce@Sauce Bordelaise}

赤ワイン3 dlにエシャロットのみじん切り大さじ2、粗く砕いたこしょう、タ
イム、ローリエの葉\undemi{}枚を加えて火にかけ、\unquart{}量になるまで
煮詰める。ソース・エスパニョル1 dlを加えて火にかけ、浮いてくる夾雑物を
丁寧に取り除きながら弱火で15分間煮る。目の細かい布で漉す。

溶かした\protect\hyperlink{glace-de-viande}{グラスドヴィアンド}大さじ1杯とレモン汁\unquart{}個分、細かいさ
いの目か輪切りにしてポシェしておいた牛骨髄を加えて仕上げる。

\ldots{}\ldots{}牛、羊の赤身肉のグリル用

【原注】こんにちではボルドー風ソースをこのように赤ワインを用いて作るが、
本来的には誤りである。元来は白ワインが用いられていた。これは\protect\hyperlink{sauce-bonnefoy}{ボルドー
風ソース・ボヌフォワ}として後述。

\hypertarget{sauce-bourguignonne}{%
\subsubsection{ブルゴーニュ風ソース}\label{sauce-bourguignonne}}

\frsub{Sauce Bourguignonne}

\index{そーす@ソース!ふるこーにゆふう@ブルゴーニュ風---}
\index{そーす@ソース!ふらうんはせい@ブラウン系の派生---!ふるこーにゆふう@ブルゴーニュ風---}
\index{ふるこーにゆふう@ブルゴーニュ風!そーす@---ソース}
\index{sauce@sauce!petites brunes composees@Petites ---s Brunes Composées!bourguignonne@--- bourguignonne}
\index{sauce@sauce!bourguignonne@--- Bourguignonne}
\index{bourguignon@bourguignon(ne)!sauce@Sauce Bourguignonne}

上質の赤ワイン1\undemi{} L に、エシャロット5個の薄切りとパセリの枝、タ
イム、ローリエの葉\undemi{}枚、マッシュルームの切りくず\footnote{料理に使うマッシュルームは通常、トゥルネ(包丁を持った側の手は動
  かさずに材料を回して切ることからついた用語)すなわち螺旋状に切って
  供するが、その際に少なくない量の切りくずが出るのでこれを使う。}25gを加えて、
半量になるまで煮詰める。布で漉し、ブールマニエ80g(バター45gと小麦粉
35g)を加えてとろみを付ける。提供直前にバター150gを溶かし込み、カイエ
ンヌ\footnote{赤唐辛子の粉末だがカイエンヌは本来、品種名。日本のタカノツメと
  比べると辛さもややマイルドで、風味も異なる。}ごく少量で加えて風味よく仕上げる。

\ldots{}\ldots{}いろいろな卵料理や、家庭料理に好適なソース。

\hypertarget{sauce-bretonne}{%
\subsubsection{ブルターニュ風ソース}\label{sauce-bretonne}}

\frsub{Sauce Bretonne}

\index{そーす@ソース!ふるたーにゆふうふらうんけい@ブルターニュ風---(ブラウン系)}
\index{そーす@ソース!ふらうんはせい@ブラウン系の派生---!ふるたーにゆふう@ブルターニュ風---}
\index{ふるたーにゆふう@ブルターニュ風!そーすふらうんけい@---ソース(ブラウン系)}
\index{sauce@sauce!petites brunes composees@Petites ---s Brunes Composées!bretonne@--- Bretonne}
\index{sauce@sauce!bretonne brune@--- Bretonne (brune)}
\index{breton@breton(ne)!sauce brune@Sauce Bretonne (brune)}

中位の玉ねぎ2個をみじん切りにして、バターでブロンド色になるまで炒める。
白ワイン2\undemi{} dlを注ぎ、半量になるまで煮詰める。ここにソース・エス
パニョル3\undemi{}
dlおよびトマトソース同量を加える。7〜8分間煮立ててから、
刻んだパセリを加えて仕上げる。

【原注】このソースは\protect\hyperlink{haricots-blancs-bretonne}{白いんげん豆のブルターニュ風}以外にはほとんど使わ
れない。

\hypertarget{sauce-aux-cerises}{%
\subsubsection{ソース・スリーズ}\label{sauce-aux-cerises}}

\frsub{Sauce aux Cerises}\footnote{さくらんぼのこと。このレシピでグロゼイユ(すぐり)のジュレを用い
  るが、古くはさくらんぼを用いていたことからこの名称となったと言われ
  ている。}

\index{そーす@ソース!すりーす@---・スリーズ}
\index{そーす@ソース!ふらうんはせい@ブラウン系の派生---!すりーす@---・スリーズ}
\index{くろせいゆ@グロゼイユ!そーす@ソース!すりーす@---・スリーズ}
\index{さくらんほ@サクランボ!そーす@ソース!すりーす@---・スリーズ}
\index{sauce@sauce!petites brunes composees@Petites ---s Brunes Composées!cerises@--- aux Cerises}
\index{sauce@sauce!cerise@--- aux Cerises}

ポルト酒2 dlにイギリス風ミックススパイス\footnote{Mixed
  spiceのこと。Pudding spiceとも呼ばれる。シナモン、ナツメ
  グ、オールスパイスの組み合わせが典型的。これにクローブ、生姜、コリ
  アンダーシード、キャラウェイシードなどが加わっていることも多い。}1つまみと、すりおろしたオレ
ンジの皮を大さじ\undemi{}杯加えて\deuxtiers{}量になるまで煮詰める。
\protect\hyperlink{gelee-de-groseilles-a}{グロゼイユのジュレ} 2\undemi{}
dlを加え、仕上げにオレンジ果汁を加える。

\ldots{}\ldots{}大型ジビエの料理用だが、鴨のポワレやブレゼにも用いられる。

\hypertarget{sauce-aux-champignons}{%
\subsubsection{ソース・シャンピニョン}\label{sauce-aux-champignons}}

\frsub{Sauce aux Champignons}\footnote{champignons
  キノコ全般を意味する語だが、単独で用いられる場合はい
  わゆるマッシュルームを指す。}

\index{そーす@ソース!まつしゆるーむ@マッシュルーム ⇒ ---・シャンピニョン}
\index{そーす@ソース!ふらうんはせい@ブラウン系の派生---!しやんひによん@---・シャンピニョン}
\index{そーす@ソース!ふらうんはせい@ブラウン系の派生---!まつしゆるーむ@マッシュルーム ⇒ ---・シャンピニョン}
\index{まっしゅるーむ@マッシュルーム ⇒ シャンピニョン}
\index{しやんひによん@シャンピニョン!そーすふらうんけいはせい@ソース・---(ブラウン系)}
\index{sauce@sauce!petites brunes composees@Petites ---s Brunes Composées!champignons@--- aux Champignons}
\index{sauce@sauce!champignons brune@--- aux Champignons (brune)}
\index{champignon@champignon!sauce brune@Sauce aux Champignons (brune)}

マッシュルームの茹で汁2\undemi{} dl を半量になるまで煮詰める。
\protect\hyperlink{sauce-demi-glace}{ソース・ ドゥミグラス}8
dlを加えて数分間煮立てる。布で漉し、 バター50
gを投入して味を調え、あらかじめ下茹でしておいた小さめのマッシュ
ルームの笠100 gを加えて仕上げる。

\hypertarget{sauce-charcutiere}{%
\subsubsection{ソース・シャルキュティエール}\label{sauce-charcutiere}}

\frsub{Sauce Charcutière}\footnote{シャルキュトリ(豚肉加工業)風、の意。Charcutrieの語源はchar(肉)
  +cuite(調理された)+rie(業)。ハムやソーセージなどと定番の組合せ
  であるマスタードを使う\protect\hyperlink{sauce-robert}{ソース・ロベール}と、おなじ
  く定番のつけ合わせであるコルニション(小さいうちに収穫してヴィネガー
  漬けにしたきゅうり。専用品種がある)を使うことに由来。}

\index{そーす@ソース!しやるききゆとりふう@シャルキュトリ風 ⇒ ---・シャルキュティエール}
\index{そーす@ソース!ふらうんはせい@ブラウン系の派生---!しやるきゆていえーる@---・シャルキュティエール}
\index{しやるきゆとりふう@シャルキュトリ風!そーす@---ソース ⇒ ソース・シャルキュティエール}
\index{sauce@sauce!petites brunes composees@Petites ---s Brunes Composées!charcutiere@--- Charcutière}
\index{sauce@sauce!charcutiere@--- Charcutière}
\index{charcutier@charcutier(ère)!sauce@Sauce Charcutière}

提供直前に、\protect\hyperlink{sauce-robert}{ソース・ロベール}1 L
に細さ2 mm程度で短かめの千切り\footnote{1〜2mm程度の細さの千切りにした野菜などをジュリエンヌjulienneと呼
  ぶ。} にしたもの100
gを加える(\protect\hyperlink{sauce-robert}{ソース・ロベール}参照)。

\hypertarget{sauce-chasseur}{%
\subsubsection{ソース・シャスール}\label{sauce-chasseur}}

\frsub{Sauce Chasseur}\footnote{狩人風、の意。古くは猟獣肉をすり潰したものを使った料理を指した
  という説もある。マッシュルームとエシャロット、白ワインを使うのが特
  徴であり、このソースを使った料理にも「シャスール」の名が付けられる。}

\index{そーす@ソース!しやすーる@---・シャスール}
\index{そーす@ソース!ふらうんはせい@ブラウン系の派生---!しやすーる@---・シャスール}
\index{しやすーる@シャスール!そーす@ソース・---}
\index{かりうとふう@狩人風 ⇒ ソース・シャスール}
\index{sauce@sauce!petites brunes composees@Petites ---s Brunes Composées!chasseur@--- Chasseur}
\index{sauce@sauce!chasseur@--- Chasseur}
\index{chasseur@chasseur!sauce@Sauce ---}

生のマッシュルームを薄切りにしたもの150gをバターで炒める。エシャロット
\footnote{échalote
  玉ねぎによく似ているが、小ぶりで水分が少なく、香味野菜
  としてよく用いられる。伝統的な品種は種子ではなく種球を植えて栽培す
  る。なお、日本でしばしば「エシャレット」の名称で流通しているものは
  ラッキョウの若どりであり、フランス料理で用いるエシャロットとはまっ
  たく異なる。}のみじん切り大さじ2\undemi{}杯を加えてさらに軽く炒め、白ワイン3
dl
を注ぎ、半量になるまで煮詰める。\protect\hyperlink{sauce-tomate}{トマトソース}3
dl と\protect\hyperlink{sauce-demi-glace}{ソース・ドゥミグラス}2
dlを加える。数分間沸騰さ せたら、バター150 gと、セルフイユ\footnote{cerfeuil
  日本ではチャービルとも呼ばれるセリ科のハーブ。}とエストラゴン\footnote{estragon
  日本ではタラゴンとも呼ばれるヨモギ科のハーブ。フランス
  料理ではとても好まれる重要なハーブのひとつ。フレンチタラゴンとロシ
  アンタラゴンの2種がある。料理に用いるのはフレンチタラゴンであり、
  この品種は種子ではなく株分けや挿し芽で殖やして栽培される。寒さには
  比較的強いが、日本の梅雨の湿度や夏の暑さには弱い。}をみじん切り
にしたもの大さじ1\undemi{}杯を加えて仕上げる。

\hypertarget{sauce-chasseur-procede-escoffier}{%
\subsubsection{ソース・シャスール(エスコフィエ流)}\label{sauce-chasseur-procede-escoffier}}

\frsub{Sauce Chasseur (Procédé Escoffier)}

\index{そーす@ソース!しやすーるえすこふいえ@---・シャスール(エスコフィエ流)}
\index{そーす@ソース!ふらうんはせい@ブラウン系の派生---!しやすーるえすこふいえ@---・シャスール(エスコフィエ流)}
\index{しやすーる@シャスール!そーすしゃすーるえすこふぃえ@ソース・---(エスコフィエ流)}
\index{かりうとふう@狩人風 ⇒ ソース・シャスール(エスコフィエ流)}
\index{sauce@sauce!petites brunes composees@Petites ---s Brunes Composées!chasseur escoffier@--- Chasseur (Procédé Escoffier)}
\index{sauce@sauce!chasseur escoffier@--- Chasseur (Procédé Escoffier)}
\index{chasseur@chasseur!sauce escoffier@Sauce --- (Procédé Escoffier)}

生のマッシュルームを薄切りにしたもの150gを、バターと植物油で軽く色付く
まで炒める。みじん切りにしたエシャロット大さじ1杯を加え、なるべくすぐ
に余分な油をきる。白ワイン2 dl とコニャック約50 ml を注ぎ、半量になるま
で煮詰める。\protect\hyperlink{sauce-demi-glace}{ソース・ドゥミグラス}4
dlと\protect\hyperlink{sauce-tomate}{トマトソー ス}2
dl、\protect\hyperlink{glace-de-viande}{グラスドヴィアンド}大さじ\undemi{}杯を加え
る。

5分間沸騰させたら、仕上げにパセリのみじん切り少々を加える。

\hypertarget{sauce-chaud-froid-brune}{%
\subsubsection{茶色いソース・ショフロワ}\label{sauce-chaud-froid-brune}}

\frsub{Sauce Chaud-froid brune}\footnote{chaudショ「熱い、温かい」とfroidフロワ「冷たい」の合成語で、火
  を通した肉や魚を冷まし、表面にこのソース・ショフロワを覆うように塗
  り付け、さらにジュレを覆いかけた料理。料理の発祥については諸説あり、
  なかでもルイ15世に仕えていた料理長ショフロワChaufroixが考案したと
  いう説を支持してなのか、英語ではこの料理をChaufroixと綴ることも多
  い。Chaud-froidの表記は19世紀後半には文献に見られる。なお、複数形
  はchauds-froidsと綴る。トリュフの薄切りやエストラゴンなどのハーブ
  その他で表面に華麗な装飾を施すことが19世紀には盛んに行なわれていた。
  現代でも装飾に凝った仕立てにするケースは多い。}

\index{そーす@ソース!ふらうんはせい@ブラウン系の派生---!しよふろわちやいろ@茶色い---・ショフロワ(エスコフィエ流)}
\index{そーす@ソース!しよふろわちやいろ@茶色い---・ショフロワ}
\index{そーす@ソース!ふらうんはせい@ブラウン系の派生---!しよふろわふらうんけい@茶色い---・ショフロワ}
\index{しよふろわ@ショフロワ!そーすふらうんけい@茶色いソース・---}
\index{sauce@sauce!petites brunes composees@Petites ---s Brunes Composées!chaud-froid brune@--- Chaud-froid brune}
\index{sauce@sauce!chaud-froid brune@--- Chaud-froid brune}
\index{chaud-froid@chaud-froid!sauce brune@Sauce --- brune}

(仕上がり1 L分)

\protect\hyperlink{sauce-demi-glace}{ソース・ドゥミグラス}\troisquarts{}
Lとトリュフエッ センス1 dl、ジュレ6〜7 dlを用意する。

ソース・ドゥミグラスにトリュフエッセンスを加えて、強火で煮詰めるが、こ
の時に鍋から離れないこと。煮詰めながらジュレを少量ずつ加えていく。最終
的に\deuxtiers{}量程度まで煮詰める。

味見をして、ソースがショフロワに使うのに丁度いい濃さになっているか確認
すること。

マデラ酒またはポルト酒\undemi{}dlを加える。布で漉し、ショフロワの主素
材の表面に塗り付けるのに丁度いい固さになるまで、丁寧にゆっくり混ぜなが
ら冷ます。

\hypertarget{sauce-chaud-froid-brune-pour-canards}{%
\subsubsection{茶色いソース・ショフロワ(鴨用)}\label{sauce-chaud-froid-brune-pour-canards}}

\frsub{Sauce Chaud-froid brune pour Canards}

\index{そーす@ソース!ふらうんはせい@ブラウン系の派生---!しよふろわちやいろかもよう@茶色い---・ショフロワ(鴨用)}
\index{そーす@ソース!しよふろわちやいと@茶色い---・ショフロワ(鴨用)}
\index{しよふろわ@ショフロワ!ちやいろいそーすかもよう@茶色いソース・---(鴨用)}
\index{sauce@sauce!petites brunes composees@Petites ---s Brunes Composées!chaud-frois canard@--- Chaud-froid pouir Canards}
\index{sauce@sauce!chaud-froid brune pour canards@--- Chaud-froid brune pour Canards}
\index{chaud-froid@chaud-froid!sauce brune pour Canards@Sauce --- brune pour Canards}

作り方は上記、\protect\hyperlink{sauce-chaud-froid-brune}{茶色いソース・ショフロワ}と同
様だが、トリュフエッセンスではなく、鴨のガラでとったフュメ1\undemi{}
dlを用いること。また、上記のレシピよりややしっかり煮詰めること。

ソースを布で漉したら、オレンジ3個分の搾り汁、とオレンジの皮をごく薄く
剥いて細かい千切りにしたもの\footnote{zeste
  ゼスト。オレンジやレモンの皮の表面を器具を用いてすりおろ
  すか、ナイフでごく薄く表皮を向き、細かい千切りにしたもの。ここでは
  後者を使う指定になっている。}大さじ2杯を加える。オレンジの皮の千切
りはしっかりと下茹でしてよく水気をきっておくこと。

\hypertarget{sauce-chaud-froid-brune-pour-gibier}{%
\subsubsection{茶色いソース・ショフロワ(ジビエ用)}\label{sauce-chaud-froid-brune-pour-gibier}}

\frsub{Sauce Chaud-froid brune pour Gibier}

\index{そーす@ソース!ふらうんはせい@ブラウン系の派生---!しよふろわちやいろしひえよう@茶色い---・ショフロワ(ジビエ用)}
\index{そーす@ソース!しよふろわちやいろじびえよう@茶色い---・ショフロワ(ジビエ用)}
\index{しよふろわ@ショフロワ!そーすしよふろわちやいろじびえよう@茶色いソース・---(ジビエ用)}
\index{sauce@sauce!petites brunes composees@Petites ---s Brunes Composées!chaud-frois gibier@--- Chaud-froid pour Gibier}
\index{sauce@sauce!chaud-froid brune pour Gibier@--- Chaud-froid brune pour Gibier}
\index{chaud-froid@chaud-froid!sauce brune pour Gibier@Sauce --- brune
pour Gibier}

作り方は上記\protect\hyperlink{sauce-chaud-froid-brune}{標準的なソース・ショフロワ}と同
じだが、トリュフエッセンスではなく、ショフロワとして供するジビエのガラ
でとったフュメ\footnote{\protect\hyperlink{fonds-de-gibier}{ジビエのフォン}参照。}2dlを用いること。

\hypertarget{sauce-chaud-froid-tomatee}{%
\subsubsection{トマト入りソース・ショフロワ}\label{sauce-chaud-froid-tomatee}}

\frsub{Sauce Chaud-froid tomatée}

\index{そーす@ソース!ふらうんはせい@ブラウン系の派生---!とまといりしよふろわ@トマト入り---・ショフロワ}
\index{そーす@ソース!しよふろわとまといり@トマト入り---・ショフロワ}
\index{しよふろわ@ショフロワ!そーすちやいろとまといり@トマト入りソース・---}
\index{sauce@sauce!petites brunes composees@Petites ---s Brunes Composées!chaud-froid tomatee@--- Chaud-froid tomatée}
\index{sauce@sauce!chaud-froid tomatée@--- Chaud-froid tomatée}
\index{chaud-froid@chaud-froid!sauce tomatée@Sauce --- tomatée}

良質で、既によく煮詰めてあるトマトピュレ1 Lを、さらに煮詰めながら7〜8
dlのジュレを少しずつ加えていく。全体量が1L以下になるまで煮詰めること。

布で漉し、使いやすい固さになるまで、ゆっくり混ぜながら冷ます。

\hypertarget{sauce-chevreuil}{%
\subsubsection{ソース・シュヴルイユ}\label{sauce-chevreuil}}

\frsub{Sauce Chevreuil}

\index{そーす@ソース!ふらうんはせい@ブラウン系の派生---!しゆうるいゆ@---・シュヴルイユ}
\index{しゆうるいゆ@シュヴルイユ!そーす@ソース・---}
\index{そーす@ソース!しゅうるいゆ@---・シュヴルイユ}
\index{のろしか@ノロ鹿 ⇒ シュヴルイユ!そーす@ソース!しゆうるいゆ@ソース・シュヴルイユ}
\index{sauce@sauce!petites brunes composees@Petites ---s Brunes Composées!chevreuil@--- Chevreuil}
\index{sauce@sauce!chevreuil@--- Chevreuil}
\index{chevreuil@chevreuil!sauce@Sauce ---}

\protect\hyperlink{sauce-poivrade}{標準的なソース・ポワヴラード}と同様に作るが、

\begin{enumerate}
\def\labelenumi{\arabic{enumi}.}
\item
  マリネした牛・羊肉の料理に添える場合\footnote{chevreuil
    シュヴルイユはノロ鹿のことだが、このように事前にマリ
    ネした牛・羊肉を用いた料理にもこのソースを使い「シュヴルイユ(風)(仕立て)」
    と\ruby{謳}{うた}う。1806年刊ヴィアール『帝国料理の本』においてノ
    ロ鹿のフィレは香辛料を加えたワインヴィネガーで48時間マリネしてから
    調理すると書かれている。オド『女性料理人のための本』では、確認出来
    た1834年の第4版から1900年の第78版に至るまで、ノロ鹿の項において
    「一週間もヴィネガーたっぷりの漬け汁でマリネするのはやりすぎだが、
    強い味が好みなら1〜4日間」香辛料と赤ワインあるいはヴィネガーでマリ
    ネするといい、と説明されている。つまり、ノロ鹿とは必ずマリネしてか
    ら調理するものという一種のコンセンサスがあったために、マリネした牛・
    羊肉の料理にも「シュヴルイユ(風)」の名称が謳われるようになったと考
    えられる。}は、ハム入りの\protect\hyperlink{mirepoix}{ミルポ
  ワ}を加える。
\item
  ジビエ料理に添える場合は、そのジビエの端肉を加える。
\end{enumerate}

素材をヘラなどで強く押し付けるようにして漉す\footnote{シノワ(\protect\hyperlink{sauce-espagnole}{ソース・エスパニョル}訳注参照)などを用いる。}。良質の赤ワイン
1\undemi{}dlをスプーン1杯ずつ加えながら煮て、浮き上がってくる不純物を
丁寧に取り除いていく\footnote{dépouiller デプイエ ≒ écumer エキュメ。}。

最後に、カイエンヌごく少量と砂糖1つまみを加えて味を\ruby{調}{とと
の}え、布で漉す。

\hypertarget{sauce-colbert}{%
\subsubsection{ソース・コルベール}\label{sauce-colbert}}

\frsub{Sauce Colbert}\footnote{17世紀の政治家、ジャン・バティスト・コルベール(1619〜1683)の
  名を冠したもの。}

\index{そーす@ソース!ふらうんはせい@ブラウン系の派生---!こるへーる@---・コルベール}
\index{そーす@ソース!こるへーる@---・コルベール}
\index{こるへーる@コルベール!そーす@ソース・---}
\index{sauce@sauce!petites brunes composees@Petites ---s Brunes Composées!colbert@--- Colbert}
\index{sauce@sauce!colbert@--- Colbert}
\index{colbert@Colbert!sauce@Sauce ---}

\protect\hyperlink{beurre-maitre-d-hotel}{メートルドテルバター}に\protect\hyperlink{glace-de-viande}{グラスドヴィアンド}を加
えたもののことだが、正しくは「\protect\hyperlink{beurre-colbert}{ブール・コルベール}」と
呼ぶべきものだ\footnote{具体的なレシピは\protect\hyperlink{beurre-colbert}{ブール・コルベール}参照のこと。}。

また、ブール・コルベールと\protect\hyperlink{sauce-chateaubriand}{ソース・シャトーブリアン}との違いを明確にさ
せようとして、メートルドテルバターにエストラゴンを加える者もいる。だが、
必ずそうすべきということではない。実際、ブール・コルベールとソース・シャ
トーブリアンは明らかに違うものだからだ。ソース・シャトーブリアンは軽く
仕上げたグラスドヴィアントにバターとパセリのみじん切りを加えたものであ
る。一方、ブール・コルベールあるいはソース・コルベールと呼ばれているもの
はあくまでもバターが主であって、グラスドヴィアンドは補助的なものに過ぎ
ない。

\hypertarget{sauce-diable}{%
\subsubsection{ソース・ディアーブル}\label{sauce-diable}}

\frsub{Sauce Diable}\footnote{悪魔の意。}

\index{そーす@ソース!ふらうんはせい@ブラウン系の派生---!ていあーふる@---・ディアーブル}
\index{そーす@ソース!ていあーふる@---・ディアーブル}
\index{ていあーふる@ディアーブル!そーす@ソース・---}
\index{あくま@悪魔 ⇒ ディアーブル!そーす@ソース!そーすていあーふる@ソース・ディアーブル}
\index{sauce@sauce!petites brunes composees@Petites ---s Brunes Composées!diable@--- Diable}
\index{sauce@sauce!diable@--- Diable}
\index{diable@diable!sauce@Sauce ---}

このソースはごく少量ずつ作るのが一般的だが、ここではそれを守らずに、仕
上り2\undemi{} dlとして説明する

白ワイン3 dlにエシャロット3個分のみじん切りを加え、\untiers{}量以下にな
るまで煮詰める。

\protect\hyperlink{sauce-demi-glace}{ソース・ドゥミグラス}2
dlを加えて数分間煮立たせ、
仕上げにカイエンヌの粉末をたっぷり効かせる\footnote{「たっぷり」という表現に惑わされないよう注意。}。

\ldots{}\ldots{}鶏と鳩のグリルに合わせる。

\hypertarget{nota-sauce-diable}{%
\subparagraph{【原注】}\label{nota-sauce-diable}}

白ワインではなくヴィネガーを煮詰め、仕上げにハーブを加えて作る調理現場
もあるが、著者としては本書で示しているの作り方がいいと思う。

\hypertarget{sauce-diable-escoffier}{%
\subsubsection{ソース・ディアーブル・エスコフィエ}\label{sauce-diable-escoffier}}

\frsub{Sauce Diable Escoffier}

\index{そーす@ソース!ふらうんはせい@ブラウン系の派生---!ていあーふるえすこふいえ@---・ディアーブル・エスコフィエ}
\index{そーす@ソース!ていあーふるえすこふいえ@---・ディアーブル・エスコフィエ}
\index{ていあーふる@ディアーブル!そーす@ソース!えすこふいえ@ソース・---・エスコフィエ}
\index{あくま@悪魔 ⇒ ディアーブル!そーす@ソース!エスコフイエ@ソース・ディアーブル・エスコフィエ}
\index{sauce@sauce!petites brunes composees@Petites ---s Brunes Composées!diable escoffier@--- Diable Escoffier}
\index{sauce@sauce!diable escoffier@--- Diable Escoffier}
\index{diable@diable!sauce escoffier@Sauce --- Escoffier}

このソースは完成品が市販\footnote{現在は市販されていないと思われる。フランスにおいては未確認だが、
  1980年代までアメリカ合衆国ではナビスコがソース・ロベール・エスコフィ
  エとともに瓶詰めを生産、販売していた。初版ではこれら2つの製品への
  言及がなく、第二版で追加されたことから、1903年〜1907年の間に製品化
  された可能性もある。また、第二版(1907年)と同年の英訳版、第三版
  (1912年)にはソース・スリーズ・エスコフィエの記述が見られるが、こ
  れは第四版で削除されており、生産中止になったと思われる。エスコフィ
  エ・ブランドの既製品ソースはさらに他にもあったようだが詳細は不明。なお、
  エスコフィエは1922年頃、ジュリユス・マジがブイヨンキューブ(日本で
  は「マギーブイヨン」の商品名)を開発する際にも協力した。}されている。同量の柔くしたバターを混ぜ合
わせるだけでいい。

\hypertarget{sauce-diane}{%
\subsubsection{ソース・ディアーヌ}\label{sauce-diane}}

\frsub{Sauce Diane}\footnote{ローマ神話の女神ディアーナのこと。ギリシア神話のアルテミスに相
  当し、狩猟、貞潔の女神。また月の女神ルーナ(セレーネー)と同一視さ
  れた。ここでは大型ジビエ料理用のソースであるから、狩猟の女神という
  意味合いが強い。}

\index{そーす@ソース!ふらうんはせい@ブラウン系の派生---!ていあーぬ@---・ディアーヌ}
\index{そーす@ソース!ていあーぬ@---・ディアーヌ}
\index{ていあーぬ@ディアーヌ!そーす@ソース・---}
\index{sauce@sauce!petites brunes composees@Petites ---s Brunes Composées!diane@--- Diane}
\index{sauce@sauce!diane@--- Diane} \index{diane@Diane!sauce@Sauce ---}

不純物を充分に取り除き、コクと風味ゆたかな\protect\hyperlink{sauce-poivrade}{ソース・ポワヴラー
ド}5 dlを用意する。提供直前に、泡立てた生クリーム4 dl
(生クリーム2dlを泡立てて倍量にする)と、小さな三日月の形にしたトリュ
フのスライスと固茹で卵の白身を加える。

\ldots{}\ldots{}大型ジビエの骨付き背肉および、その中心部を円筒形に切り出したもの
\footnote{noisette ノワゼット。}、フィレ料理用。

\hypertarget{sauce-duxelles}{%
\subsubsection{ソース・デュクセル}\label{sauce-duxelles}}

\frsub{Sauce Duxelles}{[}\^{}

\index{そーす@ソース!ふらうんはせい@ブラウン系の派生---!てゆくせる@---・デュクセル}
\index{そーす@ソース!てゆくせる@---・デュクセル}
\index{てゆくせる@デュクセル!そーす@ソース・---}
\index{sauce@sauce!petites brunes composees@Petites ---s Brunes Composées!duxelles@--- Duxelles}
\index{sauce@sauce!duxelles@--- Duxelles}
\index{duxelles@duxelles!sauce@Sauce ---}

白ワイン2dlとマッシュルームの茹で汁2 dlにエシャロットのみじん切り大さじ2
杯を加えて、\untiers{}量まで煮詰める。\protect\hyperlink{sauce-demi-glace}{ソース・ドゥミグラ
ス}\undemi{} Lとトマトピュレ1\undemi{}
dl、\protect\hyperlink{duxelles-seche}{デュク
セル・セッシュ}大さじ4杯を加える。5分間煮立たせ、パセリのみじん切り
大さじ\undemi{}を加える。

\ldots{}\ldots{}グラタンの他、いろいろな料理に用いられる。

\hypertarget{ux539fux6ce8}{%
\subparagraph{【原注】}\label{ux539fux6ce8}}

ソース・デュクセルはイタリア風ソースと混同されることが多いが、ソース・
デュクセルにはハムも、赤く漬けた舌肉も入れないので、まったく別のものだ。

\hypertarget{sauce-estragon}{%
\subsubsection{ソース・エストラゴン}\label{sauce-estragon}}

\frsub{Sauce Estragon}\footnote{ヨモギ科のハーブ。\protect\hyperlink{sauce-chasseur}{ソース・シャスール}訳注参照。}

\index{そーす@ソース!ふらうんはせい@ブラウン系の派生---!えすとらこん@---・エストラゴン}
\index{そーす@ソース!えすとらこんちゃいろ@---・エストラゴン(ブラウン系)}
\index{えすとらこん@エストラゴン!そーすふらうんけい@ソース・---(ブラウン系)}
\index{sauce@sauce!petites brunes composees@Petites ---s Brunes Composées!estragon@--- Estragon}
\index{sauce@sauce!estragon brune@--- Estragon (brune)}
\index{estragon@estragon!sauce brune@Sauce --- (brune)}

(仕上がり2\undemi{} dl分)

白ワイン2dlを沸かし、エストラゴンの枝20 gを投入する。蓋をして10分間、煎
じる\footnote{infuser(アンフュゼ)。}。2\undemi{}
dlの\protect\hyperlink{sauce-demi-glace}{ソース・ドゥミグラス}また
は、\protect\hyperlink{jus-de-veau-lie}{とろみを付けた仔牛のジュ}を加え、約\deuxtiers{}
量になるまで煮詰める。布で漉し、みじん切りにしたエストラゴン小さじ1杯
を加えて仕上げる。

\ldots{}\ldots{}仔牛や仔羊の背肉の中心を円筒形に切り出した料理や家禽料理用。

\hypertarget{sauce-financiere}{%
\subsubsection{ソース・フィナンシエール}\label{sauce-financiere}}

\frsub{Sauce Financière}\footnote{Financier徴税官(財務官)風の意。フランス革命以前の徴税官は、王
  に代わって徴税を行なう大貴族が就く役職であり、膨大な利権によりきわめて
  裕福であったという。このソースと組み合わせる\protect\hyperlink{garniture-financiere}{ガルニチュール・フィナン
  シエール}が、雄鶏のとさかと睾丸、仔羊の胸腺肉、トリュフなどの比較的
  入手困難あるいは高級とされる食材で構成されていることが名称の由来と思わ
  れる。ブリヤ=サヴァランは『美味礼讃』(味覚の生理学)において、徴税官
  たちは旬のはしりの食材を真っ先に食べられる、いわば特権階級だと述べてい
  る。なお、カレーム『19世紀フランス料理』においては、ソースとガルニチュー
  ルを分離せず、「ラグー・アラ・フィナンシエール」として採りあげられてい
  るが、全ての素材を別々に加熱調理してソースと合わせるものであり、いわゆ
  る「煮込み」とは呼びがたいものとなっている。フランス料理の影響が比較的
  強かった北イタリアにこの原型に近いと思われるラグー「ピエモンテ風フィナ
  ンツィエラ」がある。鶏のとさか、肉垂、睾丸、鶏レバーおよび仔牛の胸腺肉
  などを煮込んだものだが、レシピを読む限りにおいては比較的庶民的あるいは
  農民的料理に変化したものと思われる (cf.~Anna Gosetti della Salda,
  \emph{Le Ricette Regionali Italiane}, Milano, Solares, 1967,
  p.57.)。ちなみに焼
  き菓子のフィナンシエfinancierも同語源だが、何故その名称になったかは不
  明。}

\index{そーす@ソース!ふらうんはせい@ブラウン系の派生---!ふいなんしえーる@---・フィナンシエール}
\index{そーす@ソース!ふいなんしえーる@---・フィナンシエール}
\index{ふいなんしえーる@フィナンシエール!そーす@ソース・---}
\index{ちょうせいかんふう@徴税官風 ⇒ フィナンシエール!そーすふぃなんしえーる@ソース・フィナンシエール}
\index{sauce@sauce!petites brunes composees@Petites ---s Brunes Composées!financiere@--- Financière}
\index{sauce@sauce!financiere@--- Financière}
\index{financier@financier(ère)!sauce@Sauce Financière}

\protect\hyperlink{sauce-madere}{ソース・マデール}1\unquart{}Lを\troisquarts{}量以下に
なるまで煮詰め、火から外してトリュフエッセンス1 dlを加える。布で漉して
仕上げる。

\ldots{}\ldots{}\protect\hyperlink{garniture-financiere}{ガルニチュール・フィナンシエール}用だが、その他の肉料理にも用い
られる。

\hypertarget{sauce-aux-fines-herbes}{%
\subsubsection{香草ソース}\label{sauce-aux-fines-herbes}}

\frsub{Sauce aux Fines Herbes}\footnote{料理名では、いわゆる「ハーブ」についてかつてfines
  herbesの表
  現が多く用いられた。だが、こんにちでは特定のハーブ名をソースや料理名
  に添えて言うことが多い。例えばCôtelette de veau au thymコトレット
  ドヴォオタン(仔牛の骨付き背肉、タイム風味)、やFilet de bar poêlé,
  compote de tomate au basilicフィレドバールポワレ コンポットートド
  トマトバジリック(スズキのフィレとトマトのコンポート、バジル風味)
  など。また、栽培レベルで「香草、ハーブ」の総称としては herbes
  aromatiques
  (エルブザロマティック)、あるいはたんにaromatiques(アロマ
  ティック)が一般的。}

\index{そーす@ソース!ふらうんはせい@ブラウン系の派生---!こうそう@香草---}
\index{そーす@ソース!こうそうふらうんけい@香草---(ブラウン系)}
\index{こうそう@香草!そーすふらうんけい@---ソース(ブラウン系)}
\index{はーぶ@ハーブ ⇒ 香草!こうそうそーすふらうんけい@香草ソース(ブラウン系)}
\index{sauce@sauce!petites brunes composees@Petites ---s Brunes Composées!fines herbes@--- aux Fines Herbes}
\index{sauce@sauce!fines herbes@--- aux Fines Herbes} \index{fines
herbes@fines herbes!sauce@Sauce aux ---}

白ワイン3 dlを沸かし、パセリの葉、セルフイユ、エストラゴン、シブレット
を各1つまみ強、投入する。約20分間煎じる。布で漉し、\protect\hyperlink{sauce-demi-glace}{ソース・ドゥミグラ
ス}または\protect\hyperlink{jus-de-veau-lie}{とろみを付けた仔牛の ジュ}6
dlを加える。仕上げに、煎じるのに使ったのと同
じ香草を細かく刻んだもの計、大さじ2\undemi{}杯とレモンの搾り汁少々を加
える。

\hypertarget{ux539fux6ce8nota-sauce-aux-fines-herbes}{%
\subparagraph{【原注】\{nota-sauce-aux-fines-herbes\}}\label{ux539fux6ce8nota-sauce-aux-fines-herbes}}

古典料理ではこの「香草ソース」と\protect\hyperlink{sauce-duxelles}{ソース・デュクセル}
が混同されることもあったが、こんにちではまったく違うものとして扱われて
いる。

\hypertarget{sauce-genevoise}{%
\subsubsection{ジュネーヴ風ソース}\label{sauce-genevoise}}

\frsub{Sauce Genevoise}

\index{そーす@ソース!ふらうんはせい@ブラウン系の派生---!しゆねーうふう@ジュネーヴ風---}
\index{そーす@ソース!しゆねーうふう@ジュネーヴ風---}
\index{しゆねーうふう@ジュネーヴ風!そーす@---ソース}
\index{sauce@sauce!petites brunes composees@Petites ---s Brunes Composées!genevoise@--- Genevoise}
\index{sauce@sauce!genevoise@--- Genevoise}
\index{genevois@genevois(e)!sauce@Sauce Genevoise}

鍋にバターを熱し、細かく刻んだミルポワを色付かないよう強火でさっと炒め
る。ミルポワの材料は、にんじん100 g、玉ねぎ80 g、タイムとローリエ少々、
パセリの枝20 g。そこにサーモンの頭1kgと粗く砕いたこしょう1つまみを入れ、
蓋をして弱火で15分程蒸し煮する。

鍋に残ったバターを捨て、赤ワイン1Lを注ぐ。半量になるまで煮詰める。そこ
に\protect\hyperlink{sauce-espagnole-maigre}{魚料理用ソース・エスパニョル}\undemi{}
Lを
加える。弱火で1時間煮込む。漉し器を使い、材料を押しつけながら漉す。し
ばらく休ませてから、表面に浮いた油脂を取り除く\footnote{dégraisser
  デグレセ。レードルなどを用いて浮いてきた余計な油脂を取り除く作業。}

さらに赤ワイン\undemi{} Lと、魚のフュメ\undemi{} Lを加える。ソースの表
面に浮いてくる不純物を徹底的に取り除き\footnote{dépouiller デプイエ ≒
  écumer エキュメ。}、丁度いい濃さになるまで煮 詰める。

これを布で漉し、静かに混ぜながら、アンチョヴィのエッセンス大さじ1杯と
バター150 gを加えて仕上げる。

\ldots{}\ldots{}サーモン、鱒料理用。

\hypertarget{nota-sauce-genevoise}{%
\subparagraph{【原注】}\label{nota-sauce-genevoise}}

このソースはもともとカレームが「ジェノヴァ風」\footnote{Sauce à la
  génoise au vin de Bordeaux ボルドー産ワインを用いた
  ジェノヴァ風ソース(『19世紀フランス料理』第3巻、80頁)。本書のこのレシ
  ピと同様に魚料理用ソースだ。ボルドーの赤ワインにみじん切りにして下茹で
  したマッシュルーム、トリュフ、エシャロットを加えてオールスパイスとこしょ
  う少々を入れ、適度に煮詰める。ソース・エスパニョルと赤ワインを加え、湯
  煎にかけておく。提供直前にバター少量を加えて仕上げる、というもの。本書
  においてこのソースを「原型」とするのには疑問が残るところだろう。}と名付けたものだが、その
後ルキュレ、グフェ\footnote{グフェ『料理の本』(1867年)の420ページにあるジュネーヴ風ソー
  スは、薄切りにした玉ねぎ、エシャロット、粗挽きこしょう、にんにく、
  バターを鍋に入れて色付くまで炒め、そこにブルゴーニュ産赤ワインを注
  ぐ。弱火で玉ねぎに火が通るまで煮る。ソース・エスパニョルと仔牛のブ
  ロンドのジュを加えて煮詰め、布で漉す。提供直前にマデラ酒の風味を加
  えて茹でたトリュフのみじん切りとアンチョビバターを加える、というも
  の。赤ワインと玉ねぎ、仕上げにアンチョビを加える点は共通しているが、
  グフェのが肉料理用であるのに対して、本書のこのソースは明らかに魚料
  理用であり、まったく同じソースと呼べるとは言い難い。}が立て続けに「ジュネーヴ風」の名称を用いた。だが、ジュ
ネーヴは赤ワインの産地ではないから理屈としてはおかしい\footnote{料理名に冠された地名は、由来が明確にあるものがある一方で、まっ
  たく意味不明か、あるいはいい加減な思い付きで付けられたのではないか
  とさえ思われるものも少なくない。(à la) russe「ロシア風」や (à la)
  moscovite「モスクワ風」などはロシア料理起源か、あるいは18世紀末〜
  19世紀前半にかけてロシア帝国の宮廷や貴族がこぞってフランスから料理
  人を招聘し、帰国した彼らが創案した料理などはある程度しっかりとした
  由来がわかるものも多い。一方で、(à l')espagnole「スペイン風」(à
  l')italienne「イタリア風」(à la) romaine「ローマ風」(à la grecque)
  「ギリシア風」(à l')allemande「ドイツ風」(à l')hollandaise「オラン
  ダ風」などは由来の不明なケースが非常に多い。\protect\hyperlink{sauce-espagnole}{ソース・エスパニョ
  ル}などはその典型例とも言うべきものだろう。\\
  この原注では由来に非常にこだわっているが、そもそもカレームのレシピ
  は上述のように「ボルドー産ワインを用いたジェノバ風ソース」であるか
  ら、赤ワインの産地かどうかということは実はさしたる問題にはならない。
  重要なのは後半の、赤ワインを用いることがこのソースのポイントという
  こと。}。

間違っているとはいえ、ジュネーヴ風という名称で定着してしまっているので、
本書でもそのままにしている。だが、ジュネーヴ風であれジェノヴァ風であれ、
カレーム、ルキュレ、デュボワ、グフェはいずれもこのソースに赤ワインを用
いるよう指示している。つまり赤ワインを用いることがこのソースのポイント。

\hypertarget{sauce-godard}{%
\subsubsection{ソース・ゴダール}\label{sauce-godard}}

\frsub{Sauce Godard}\footnote{ガルニチュール・ゴダールの構成要素がガルニチュール・フィナンシ
  エールとよく似ている点などから、おそらくは18世紀の徴税官(つまりフィ
  ナンシエ)であり作家としても活動したクロード・ゴダール・ドクール
  Claude Godard d'Aucour(1716〜1795)の名を冠したものと考えられる。
  なお、底本とした現行版(第四版)では最後がdではなくtとなっているが、
  初版から第三版にいたるまでdとなっており、現行版は明らかな誤植。}

\index{そーす@ソース!ふらうんはせい@ブラウン系の派生---!こたーる@---・ゴダール}
\index{そーす@ソース!こたーる@---・ゴダール}
\index{こたーる@ゴダール!そーす@ソース・---}
\index{sauce@sauce!petites brunes composees@Petites ---s Brunes Composées!godard@--- Godard}
\index{sauce@sauce!godard@--- Godard}
\index{godard@Godard!sauce@Sauce ---}

シャンパーニュまたは辛口の白ワイン4
dlにハム入りの細かく刻んだ\protect\hyperlink{mirepoix}{ミルポ
ワ}、\protect\hyperlink{sauce-demi-glace}{ソース・ドゥミグラス}1
Lとマッシュルームのエッセンス2dlを加える。
弱火に10分かけ、シノワ\footnote{\protect\hyperlink{sauce-espagnole}{ソース・エスパニョル}訳注参照。}で漉す。

\deuxtiers{}量になるまで煮詰め、布で漉す。

\ldots{}\ldots{}\protect\hyperlink{garniture-godard}{ガルニチュール ゴタール}用。

\hypertarget{sauce-grand-veneur}{%
\subsubsection{ソース・グランヴヌール}\label{sauce-grand-veneur}}

\frsub{Sauce Grand-Veneur}\footnote{王家や貴族に仕える狩猟長のことをグランヴヌールと呼ぶ。}

\index{そーす@ソース!ふらうんはせい@ブラウン系の派生---!くらんうぬーる@---・グランヴヌール}
\index{そーす@ソース!くらんうぬーる@---・グランヴヌール}
\index{くらんうぬーる@グランヴヌール!そーす@ソース・---}
\index{sauce@sauce!petites brunes composees@Petites ---s Brunes Composées!grand-veneur@--- Grand-Veneur}
\index{sauce@sauce!grand-veneur@--- Grand-Veneur}
\index{grand-veneur@grand-veneur!sauce@Sauce ---}

\protect\hyperlink{fonds-de-gibier}{大型ジビエのフュメ}で澄んだ色合いに作った\protect\hyperlink{sauce-poivrade}{ソース・
ポワヴラード}に、ソース1Lあたり野うさぎの血1dlをマリ
ネ液1dlで薄めたものを加える。

火をごく弱くして、血が沸騰しないよう気をつけながら数分間煮る。布で漉す。

\hypertarget{sauce-grand-veneur-procede-escoffier}{%
\subsubsection{ソース・グランヴヌール(エスコフィエ流)}\label{sauce-grand-veneur-procede-escoffier}}

\frsub{Sauce Grand-Veneur (Procédé Escoffier)}

\index{そーす@ソース!ふらうんはせい@ブラウン系の派生---!くらんうぬーるえすこふいえ@---・グランヴヌール(エスコフィエ流)}
\index{そーす@ソース!くらんうぬーるえすこふいえ@---・グランヴヌール(エスコフィエ流)}
\index{くらんうぬーる@グランヴヌール!そーすえすこふいえ@ソース・---(エスコフィエ流)}
\index{sauce@sauce!petites brunes composees@Petites ---s Brunes Composées!grand-veneur escoffier@--- Grand-Veneur (Procédé Escoffier)}
\index{sauce@sauce!grand-veneur escoffier@--- Grand-Veneur (Procédé Escoffier)}
\index{grand-veneur@grand-veneur!sauce escoffier@Sauce --- (Procédé Escoffier)}

軽く仕上げた\protect\hyperlink{sauce-poivrade}{ソース・ポワヴラード}1
Lあたり{[}グロゼイ ユのジュレ{]}大さじ2杯と生クリーム2\undemi{}
dlを加える。

\ldots{}\ldots{}上記2つのソースは鹿、猪などの大きな塊肉の料理に用いる。

\hypertarget{sauce-gratin}{%
\subsubsection{ソース・グラタン}\label{sauce-gratin}}

\frsub{Sauce Gratin}\footnote{魚のグラタン用ソースだが、グラタンの技術的ポイントについては\protect\hyperlink{gratins}{第
  7章「肉料理」のグタランの項目}参照。}

\index{そーす@ソース!ふらうんはせい@ブラウン系の派生---!くらたん@---・グラタン}
\index{そーす@ソース!くらたん@---・グラタン}
\index{くらたん@グラタン!そーす@ソース・---}
\index{sauce@sauce!petites brunes composees@Petites ---s Brunes Composées!gratin@--- Gratin}
\index{sauce@sauce!gratin@--- Gratin}
\index{gratin@gratin!sauce@Sauce ---}

白ワインと、このソースを合わせる魚のアラなどでとった\protect\hyperlink{fumet-de-poisson}{魚のフュ
メ}各3 dlにエシャロットのみじん切り大さじ1\undemi{}
杯を加え、半量以下になるまで煮詰める。

\protect\hyperlink{duxelles-seche}{デュクセル・セッシュ}大さじ3杯と、\protect\hyperlink{sauce-espagnole-maigre}{魚料理用ソース・エスパニョ
ル}または\protect\hyperlink{sauce-demi-glace}{ソース・ドゥミグラ ス}5
dlを加える。5〜6分間煮立たせる。提供直前に、パ
セリのみじん切り大さじ\undemi{}を加えて仕上げる。

\ldots{}\ldots{}舌びらめ、メルラン\footnote{タラの近縁種。}、バルビュ\footnote{鰈の近縁種。この場合のフィレはいわゆる「五枚おろし」にしたもの。}のフィレなどのグラタン用。

\hypertarget{sauce-hachee}{%
\subsubsection{ソース・アシェ}\label{sauce-hachee}}

\frsub{Sauce Hachée}\footnote{細かく刻んだもの、の意。}

\index{そーす@ソース!ふらうんはせい@ブラウン系の派生---!あしえ@---・アシェ}
\index{そーす@ソース!あしえ@---・アシェ}
\index{sauce@sauce!petites brunes composees@Petites ---s Brunes Composées!hachee@--- Hachée}
\index{sauce@sauce!hachee@--- Hachée}
\index{hache@haché(e)!sauce@Sauce Hachée}

玉ねぎの細かいみじん切り100gと、エシャロットの細かいみじん切り大さじ
1\undemi{}杯をバターで色付かないよう炒める。ヴィネガー3 dlを注ぎ、半量
まで煮詰める。\protect\hyperlink{sauce-espagnole}{ソース・エスパニョル}4
dlと\protect\hyperlink{sauce-tomate}{トマトソース}1\undemi{} dl
を加える。5〜6分煮立たせる。

ハムの脂身のない部分を細かく刻んだもの大さじ1\undemi{}杯と小ぶりのケイ
パー大さじ1\undemi{}杯、{[}デュクセル・セッシュ{]}大さじ1\undemi{}杯、パセ
リのみじん切り大さじ\undemi{}杯を加えて仕上げる

\ldots{}\ldots{}このソースは\protect\hyperlink{sauce-piquante}{ソース・ピカント}と等価のものと考えていい。用途も同じ。

\hypertarget{sauce-hachee-maigre}{%
\subsubsection{魚料理用ソース・アシェ}\label{sauce-hachee-maigre}}

\frsub{Sauce Hachée maigre}

\index{そーす@ソース!ふらうんはせい@ブラウン系の派生---!あしえさかな@魚料理用---・アシェ}
\index{そーす@ソース!あしえ@魚料理用---・アシェ}
\index{sauce@sauce!petites brunes composees@Petites ---s Brunes Composées!hachee maigre@--- Hachée maigre}
\index{sauce@sauce!hachee maigre@--- Hachée maigre}
\index{hache@haché(e)!sauce maigre@Sauce Hachée maigre}

上記と同様に、玉ねぎとエシャロットを色付かないようバターで炒め、ヴィネ
ガーを注いで煮詰める。

魚の\protect\hyperlink{courts-bouillons-de-poisson}{クールブイヨン}5
dlを注ぎ、\protect\hyperlink{roux-brun}{茶色いルー}45 gまたはブー
ルマニエ50 gでとろみを付ける。弱火で8〜10分間煮込む。

提供直前に、細かく刻んだハーブミックス大さじ1杯と\protect\hyperlink{duxelles-seche}{デュクセル・セッシュ}大
さじ1\undemi{}杯、小粒のケイパー大さじ1\undemi{}杯、アンチョヴィソース
大さじ\undemi{}杯とバター60 g、または80〜100 gのアンチョヴィバターを加
えて仕上げる。

\ldots{}\ldots{}エイのような、あまり高級ではない茹でた魚\footnote{原文
  poissons bouillis。このフランス語の表現だと加熱する際に沸
  騰させているニュアンスがあるが、本書の「魚料理」の章において、魚を
  塩を加えて茹でる、あるいはクールブイヨンで煮る際に、沸騰しない程度
  の温度で加熱(ポシェ pocher)すべきと強調されている。この表現は初
  版からのものであり、恐らくはこのソースの部分を実際に執筆した者と、
  魚料理の説明部分を執筆した者が異なることによるわかりにくさ、という
  可能性も排除出来ない。いずれにしても、このソースの場合は、合わせる
  魚をクールブイヨンで沸騰しない程度の温度で加熱(ポシェ)し、そのクー
  ルブイヨンの一部をソースに加えていることから、単に「茹でた魚」と言っ
  ても、本書における魚の加熱方法に則った調理をすべきと解されよう。}用。

\hypertarget{sauce-hussarde}{%
\subsubsection{ソース・ユサルド}\label{sauce-hussarde}}

\frsub{Sauce Hussarde}\footnote{もとはハンガリーで農家20戸につき1人の割合で招集された騎兵
  hussard を指す。この語は16世紀まで遡ることが出来るが、のちに「乱暴
  者」といったニュアンスでも使われるようになった。à la hussarde は
  「乱暴に、粗野に」の意味でも用いられるが、料理においてはレフォール
  を使ったものに名付けられることが多い。}

\index{そーす@ソース!ふらうんはせい@ブラウン系の派生---!ゆさると@---・ユサルド}
\index{そーす@ソース!ゆさると@---・ユサルド}
\index{ゆさると@ユサルド!そーす@ソース・---}
\index{sauce@sauce!petites brunes composees@Petites ---s Brunes Composées!hussarde@--- Hussarde}
\index{sauce@sauce!hussarde@--- Hussarde}
\index{hussard@Hussard(e)!sauce@Sauce Hussarde}

玉ねぎ2個とエシャロット2個を細かくみじん切りにして、バターで色よく炒め
る。白ワイン4
dlを注ぎ、半量になるまで煮詰める。\protect\hyperlink{sauce-demi-glace}{ソース・ドゥミグラ
ス}4 dlとトマトピュレ大さじ2杯、\protect\hyperlink{fonds-blanc}{白いフォ
ン}2 dl、生ハムの脂身のないところ80 g、潰した
にんにく1片、ブーケガルニを加える。弱火で25〜30分煮込む。

ハムを取り出して、ソースをスプーンで押すようにして布で漉す。

火にかけて温め、小さなさいの目\footnote{brunoise ブリュノワーズ。}に刻んだハムと、おろしたレフォール
\footnote{raifort (レフォール)いわゆる西洋わさび、ホースラディッシュ。}少々、パセリのみじん切りをたっぷり1つまみ加えて仕上げる。

\ldots{}\ldots{}牛、羊肉のグリルまたは串を刺してローストしてアントレ\footnote{通常、ローストは料理区分としてアントレに含められることはないが、
  牛フィレは牛の部位のなかでも比較的小さいものとして、まるごと1本の
  ローストであっても原則的にはアントレに分類される。このソースを用い
  る「牛フィレ ユサルド」は牛フィレの塊に串を刺してローストし、ポム・
  デュシェスとマッシュルームを合わせる。}として供 する際に用いる。

\hypertarget{sauce-italienne}{%
\subsubsection{イタリア風ソース}\label{sauce-italienne}}

\frsub{Sauce Italienne}\footnote{この「イタリア風」には根拠も由来も見出すことが出来ない。地名、
  国名を料理名に冠した代表例のひとつ。}

\index{そーす@ソース!ふらうんはせい@ブラウン系の派生---!イタリア風@---}
\index{そーす@ソース!いたりあふう@イタリア風---}
\index{いたりあふう@イタリア風!そーす@---ソース}
\index{sauce@sauce!petites brunes composees@Petites ---s Brunes Composées!italienne@--- Italienne}
\index{sauce@sauce!Italienne@--- Italienne}
\index{italien@italien(ne)!sauce@Sauce Italienne}

トマトの風味の効いた\protect\hyperlink{sauce-demi-glace}{ソース・ドゥミグラ
ス}\troisquarts{}
Lに、\protect\hyperlink{duxelles-seche}{デュクセル・セッシュ}大さ
じ4杯と、加熱ハムの脂身のないところを小さなさいの目に切ったもの125 gを
加える。5〜6分間煮る。提供直前に、パセリとセルフイユ、エスゴラゴンのみ
じん切り大さじ1杯を加えて仕上げる。

\ldots{}\ldots{}いろいろな肉料理に合わせる。

\hypertarget{nota-sauce-italienne}{%
\subparagraph{【原注】}\label{nota-sauce-italienne}}

このソースを魚料理に合わせる場合、ハムは使わずに\protect\hyperlink{fumet-de-poisson}{魚のフュ
メ}を煮詰めて加える。

\hypertarget{jus-lie-a-lestragon}{%
\subsubsection{とろみを付けたジュ エストラゴン風味}\label{jus-lie-a-lestragon}}

\frsub{Jus lié à l'Estragon}

\index{そーす@ソース!ふらうんはせい@ブラウン系の派生---!とろみをつけたしゆえすとらこん@とろみを付けたジュ エストラゴン風味}
\index{そーす@ソース!とろみをつけたしゆえすとらこん@とろみを付けたジュエストラゴン風味}
\index{しゆ@ジュ!とろみをつけたえすとらごん@とろみを付けた--- エストラゴン風味}
\index{sauce@sauce!petites brunes composees@Petites ---s Brunes Composées!jus lie estragon@Jus lié à l'estragon}
\index{sauce@sauce!jus lie estragon@Jus lié à l'Estragon}
\index{estragon@estragon!jus lie estragon@Jus lié à l'Estragon}
\index{jus@jus!lie estragon@--- lié à l'Estragon}

\protect\hyperlink{jus-de-veau-brun}{仔牛のフォン}または\protect\hyperlink{fonds-de-volaille}{鶏のフォ
ン}に、エストラゴン50gを加えて香りを煮出し\footnote{imfuser アンフュゼ。}た
もの。

布で漉してから、アロールート\footnote{コーンスターチで代用する。}または、でんぷん30
gでとろみを付ける。

\ldots{}\ldots{}白身肉のノワゼットや家禽のフィレなどに添える。

\hypertarget{jus-lie-tomate}{%
\subsubsection{とろみを付けたジュ トマト風味}\label{jus-lie-tomate}}

\frsub{Jus lié tomaté}

\index{そーす@ソース!ふらうんはせい@ブラウン系の派生---!とろみをつけたしゆとまとふうみ@とろみを付けたジュ トマト風味}
\index{そーす@ソース!とろみをつけたしゆとまと@とろみを付けたジュ トマト風味}
\index{しゆ@ジュ!とろみをつけたとまとふうみ@とろみを付けた--- トマト風味}
\index{sauce@sauce!petites brunes composees@Petites ---s Brunes Composées!jus lie tomatee@Jus lié tomatée}
\index{sauce@sauce!jus lie tomatee@Jus lié tomaté}
\index{tomate@tomate!jus lie tomate@Jus lié tomaté}
\index{jus@jus!lie tomate@--- lié tomaté}

\protect\hyperlink{jus-de-veau-brun}{仔牛のフォン}1
Lあたりトマトエッセンス3 dlを加え、 \quatrecinquiemes{}量まで煮詰める。

\ldots{}\ldots{}牛、羊肉料理用。

\hypertarget{sauce-lyonnaise}{%
\subsubsection{リヨン風ソース}\label{sauce-lyonnaise}}

\frsub{Sauce Lyonnaise}

\index{そーす@ソース!ふらうんはせい@ブラウン系の派生---!りよんふう@リヨン風---}
\index{そーす@ソース!りよんふう@リヨン風---}
\index{りよんふう@リヨン風!そーす@---ソース}
\index{sauce@sauce!petites brunes composees@Petites ---s Brunes Composées!lyonnaise@--- Lyonnaise}
\index{sauce@sauce!lyonnaise@--- Lyonnaise}
\index{liyonnais@lyonnais(e)!sauce lyonnaise@Sauce Lyonnaise}

中位の大きさの玉ねぎ3個をみじん切りにし、バターでじっくり、ごく弱火で
ブロンド色になるまで炒める。白ワイン2 dlとヴィネガー2 dlを注ぐ。
\untiers{}量まで煮詰め、\protect\hyperlink{sauce-demi-glace}{ソース・ドゥミグラ
ス}\troisquarts{} Lを加える。5〜6分かけて表面に浮い
てくる不純物を丁寧に取り除き\footnote{dépouiller
  デプイエ。現代ではエキュメと呼ぶ現場が多い。}、布で漉す。

\hypertarget{ux539fux6ce8-nota-sauce-lyonnaise}{%
\subparagraph{【原注】
\{nota-sauce-lyonnaise\}}\label{ux539fux6ce8-nota-sauce-lyonnaise}}

このソースを合わせる料理によっては、ソースを布で漉さずに玉ねぎを残して
もいい。

\hypertarget{sauce-madere}{%
\subsubsection{ソース・マデール}\label{sauce-madere}}

\frsub{Sauce Madère}

\index{そーす@ソース!ふらうんはせい@ブラウン系の派生---!まてーる@---・マデール}
\index{そーす@ソース!までーる@---・マデール}
\index{までらしゆ@マデラ酒 ⇒ マデール!そーす@ソース・マデール}
\index{sauce@sauce!petites brunes composees@Petites ---s Brunes Composées!madere@--- Madère}
\index{sauce@sauce!madere@--- Madère}
\index{madere@madère!sauce@Sauce ---}

\protect\hyperlink{sauce-demi-glace}{ソース・ドゥミグラス}を煮詰め\footnote{ソース・ドゥミグラスは既に煮詰めて仕上がった状態のものなので、9
  割程度にまでしか煮詰めないことに注意。}、火から外して、 ソース1
Lあたりマデラ酒1 dlの割合で加え、普通の濃度にする。

\hypertarget{sauce-matelote}{%
\subsubsection{ソース・マトロット}\label{sauce-matelote}}

\frsub{Sauce Matelote}\footnote{水夫風、船員風、の意。トゥーレーヌ地方の郷土料理Matelote
  d'anguille(マトロットダンギーユ)うなぎの赤ワイン煮込み、が有名。
  とはいえ本書にも数種のレシピが収録されているように、赤ワイン煮込み
  にとどまらず、マトロットの名称を持つ料理は他にも複数存在する。}

\index{そーす@ソース!ふらうんはせい@ブラウン系の派生---!まとろつと@---・マトロット}
\index{そーす@ソース!まとろつと@---・マトロット}
\index{まとろつと@マトロット!そーす@ソース・---}
\index{sauce@sauce!petites brunes composees@Petites ---s Brunes Composées!matelote@--- Matelote}
\index{sauce@sauce!matelote@--- Matelote}
\index{matelote@matelote!sauce@Sauce ---}

魚をポシェするのに使った\protect\hyperlink{court-bouillon-c}{赤ワイン入りの魚用クールブイヨン}3
dlにマッ シュルームの切りくず25
gを加え、\untiers{}量になるまで煮詰める。

煮詰めたら\protect\hyperlink{sauce-espagnole-maigre}{魚料理用ソース・エスパニョル}8
dl を加えてひと煮立ちさせる。布で漉し、バター150 gとごく少量のカイエンヌ
の粉末を加えて仕上げる。

\hypertarget{sauce-moelle}{%
\subsubsection{ソース・モワル}\label{sauce-moelle}}

\frsub{Sauce Moelle}\footnote{moelle 骨髄のこと。}

\index{そーす@ソース!ふらうんはせい@ブラウン系の派生---!もわる@---・モワル}
\index{そーす@ソース!もわる@---・モワル}
\index{こつずい@骨髄 ⇒ モワル!そーすもわる@ソース・モワル}
\index{sauce@sauce!petites brunes composees@Petites ---s Brunes Composées!moelle@--- Moelle}
\index{sauce@sauce!moelle@--- Moelle}
\index{moelle@moelle!sauce@Sauce ---}

ソースの作り方は\protect\hyperlink{sauce-bordelaise}{ボルドー風ソース}とまったく同じだ
が、バターを加えるのは何らかの野菜料理に添える場合のみであり、その場合
のバターの量は通常どおりとするこ。

どんな場合にせよ、仕上げに、小さなさいの目に切ってポシェしておいた骨髄
をソース1 Lあたり150〜180 gおよび刻んで下茹でしたパセリの葉小さじ1杯を
加える。

\hypertarget{sauce-moscovite}{%
\subsubsection{モスクワ風ソース}\label{sauce-moscovite}}

\frsub{Sauce Moscovite}\footnote{moscovite(モスコヴィット)すなわちモスクワ風の名称を持つ料理や
  菓子は多い。 18世紀後半から19世紀前半にかけて、ロシアの宮廷や貴族
  らの間でフランスの食文化が流行し、多くのフランス人料理人が招聘され、
  彼らはロシア料理のレシピをフランスに持ち帰った。クーリビヤックなど
  が代表的な例だろう。また、19世紀後半になると、とりわけフランス料理
  においてもロシア料理からの影響が多く見られるようになる。キャビアと
  ウォトカを食前に愉しむのが流行したのもその時代からである。フランス
  とロシアの食文化は相互に影響関係にあったと言えよう。}

\index{そーす@ソース!ふらうんはせい@ブラウン系の派生---!もすくわふう@モスクワ---}
\index{そーす@ソース!もすくわふう@モスクワ風---}
\index{もすくわふう@モスクワ風!そーす@---ソース}
\index{sauce@sauce!petites brunes composees@Petites ---s Brunes Composées!moscovite@--- Moscovite}
\index{sauce@sauce!moscovite@--- Moscovite}
\index{moscovite@moscovite!sauce@Sauce ---}

\protect\hyperlink{fonds-de-gibier}{大型ジビエのフュメ}で作った\protect\hyperlink{sauce-poivrade}{ソース・ポワヴラー
ド}を\troisquarts{} L用意する。提供直前にマラガ酒1 dl
とジェニパーベリーを煎じた汁7 cl\footnote{1 cl = 10
  ml、つまりこの場合は70 ml。}、焼いた松の実かスライスして焼いたアー
モンド40 g、大きさを揃えてぬるま湯でもどしておいたコリント産干しぶどう
\footnote{小粒で黒いギリシア産干しぶどう。}40 gを加えて仕上げる。

\ldots{}\ldots{}大型ジビエ\footnote{venaison
  ヴネゾン。ジビエのうち大型のものを指す。実際は
  ノロ鹿や猪を指すことがほとんど。}の塊肉の料理用。

\hypertarget{sauce-perigueux}{%
\subsubsection{ソース・ペリグー}\label{sauce-perigueux}}

\frsub{Sauce Périgueux}\footnote{トリュフの産地として有名なペリゴール地方の町の名。}

\index{そーす@ソース!ふらうんはせい@ブラウン系の派生---!へりくー@---・ペリグー}
\index{そーす@ソース!へりくー@---・ペリグー}
\index{へりくー@ペリグー!そーす@ソース・---}
\index{sauce@sauce!petites brunes composees@Petites ---s Brunes Composées!perigueux@--- Périgueux}
\index{sauce@sauce!perigueux@--- Périgueux}
\index{perigueux@Périgueux!sauce@Sauce ---}

やや濃いめに煮詰めた\protect\hyperlink{sauce-demi-glace}{ソース・ドゥミグラ
ス}\troisquarts{} Lに、トリュフエッセンス1 \undemi{}
dlと細かく刻んだトリュフ100 gを加える。

\ldots{}\ldots{}いろいろな肉料理、\protect\hyperlink{}{タンバル}、\protect\hyperlink{}{温製パテ}に合わせる。

\hypertarget{sauce-perigourdine}{%
\subsubsection{ソース・ペリグルディーヌ}\label{sauce-perigourdine}}

\frsub{Sauce Périgourdine}\footnote{ペリゴール地方風の意。}

\index{そーす@ソース!ふらうんはせい@ブラウン系の派生---!へりくるていーぬ@---・ペリグルディーヌ}
\index{そーす@ソース!へりくるていーぬ@---・ペリグゥルディーヌ}
\index{へりこーるふう@ペリゴール風 ⇒ ペリグルダン/ペリグルディーヌ!そーす@ソース・ペリグルディーヌ}
\index{sauce@sauce!petites brunes composees@Petites ---s Brunes Composées!perigourdine@--- Périgourdine}
\index{sauce@sauce!perigourdine@--- Périgourdine}
\index{perigourdin@périgourdin(e)!sauce@Sauce Périgourdine}

ソース・ペリグーのバリエーション。トリュフを細かく刻むのではなく、オリー
ブ形か小さな真珠のような形状にナイフで成形\footnote{tourner
  トゥルネ。包丁を持っている側の手は動かさずに材料を回す
  ようにして形を整えること。}したものを加える。トリュ
フを厚めにスライスして加える場合もある。

\hypertarget{sauce-piquante}{%
\subsubsection{ソース・ピカント}\label{sauce-piquante}}

\frsub{Sauce Piquante}\footnote{piquant(e) (ピカン、ピカント)
  一般的には唐辛子などが「辛い」
  の意だが、このソースでは唐辛子の類は使われておらず、むしろ酸味の効
  いたソースと言えよう。古くからのソース名。}

\index{そーす@ソース!ふらうんはせい@ブラウン系の派生---!ひかんと@---・ピカント}
\index{そーす@ソース!ひかんと@---・ピカント}
\index{sauce@sauce!petites brunes composees@Petites ---s Brunes Composées!piquante@--- piquante}
\index{sauce@sauce!piquante@--- Piquante}
\index{piquant@piquant(e)!sauce@Sauce Piquante}

白ワイン3 dlと良質のヴィネガー3 dlにエシャロットのみじん切り大さじ2
\undemi{}杯を合わせて半量に煮詰める。

\protect\hyperlink{sauce-espagnole}{ソース・エスパニョル}6
dlを加え、浮いてくる不純物を 取り除きながら\footnote{dépouiller
  デプイエ。エキュメécumerと呼ぶ現場も多い。}10分間煮る。

火から外し、コルニション\footnote{専用品種のきゅうりを小さなうちに収穫して酢漬けにしたもの。同様
  のピクルス用きゅうりとしてガーキンスという品種系統があるがもっぱら
  アメリカのハンバーガーに挟まれるようなサイズで収穫して漬けたもので
  あり、フランス料理では用いない。}、パセリ、セルフイユ、エストラゴンを細か
く刻んだもの大さじ2杯を加えて仕上げる。

\ldots{}\ldots{}豚肉のグリル焼き、ブイイ\footnote{bouilli
  茹で肉。もとはブイヨンをとった後の茹で肉のことを指した。
  単純に「茹でた肉」としてもいいのだが、17世紀にはこの食べ方が流行し
  たという歴史もあり、野菜などと共に、あるいは他の素材なしに茹でた肉
  はたんに「ブイイ」bouilli と呼ばれる。}、ローストによく合わせるソース。牛肉
のブイイや牛や羊の\protect\hyperlink{}{エマンセ}にも合わせることが出来る。

\hypertarget{sauce-poivrade}{%
\subsubsection{ソース・ポワヴラード (標準)}\label{sauce-poivrade}}

\frsub{Sauce Poivrade ordinaire}\footnote{このソースは遅くとも16世紀まで遡ることが出来る。1505年に出版さ
  れた\href{http://gallica.bnf.fr/ark:/12148/bpt6k792720}{『フランス語版プラティ
  ナ』}がpoivradeとい
  うフランス語の初出。この本において「ジビエ用こしょうのソース、ポワ
  ヴラード」Saulce de poyvre ou poyvrade pour saulvagieとしてレシピ
  が見られる。パンをよく焼いてヴィネガーに浸してすり潰す。水でもどし
  た干しぶどうと獣の血を加えて混ぜ、玉ねぎと未熟ぶどう果汁、パンを浸
  した残りのヴィネガーを加えて漉し器か布で漉す。これを鍋に入れ、こしょ
  う、生姜、シナモンを入れて炭火の上で30分程煮込む。獣の肉を獣脂を熱
  したフライパンで焼き、皿に盛る。上からポワヴラードをかけて供する、
  という内容(f.LXII)。またこの本には、魚料理用のポワヴラードも掲載
  されている。ただし、これが現代まで続くソース・ポワヴラードの原型と
  捉えるのは早計に過ぎる。ここで注目すべきは、最終的に肉あるいは魚の
  ような主素材とソースが一体化したものは中世〜ルネサンス期にはポター
  ジュと呼ばれていたのに対し、ここではソースを別のものと捉えている点
  である。ポワヴラードという語そのものは「こしょうを効かせたもの」と
  いう意味に過ぎず、1660年刊ピエール・ド・リュヌPierre de Lune『新フ
  ランス料理』におけるPoivrade de pigeonneaux 若鳩のポワヴラードは、
  背開きにした若鳩を平たくのばし、塩、こしょう をして弱火でグリルす
  る。薔薇の香りもしくはにんにく風味のヴィネガーを添えて供する、とい
  うもの(p.190)。ピエール・ド・リュヌのレシピにおいてソースに相当す
  るものはヴィネガーであり、むしろ味付けでこしょうを効かせているとい
  うことが料理名の根拠となっているに過ぎない。ちなみに、生食可能な小
  さなサイズのアーティチョークも古くからポワヴラードと呼ばれている。}

\index{そーす@ソース!ふらうんはせい@ブラウン系の派生---!ほわうらーと@---・ポワヴラード(標準)}
\index{そーす@ソース!ほわうらーと@---・ポワヴラード(標準)}
\index{ほわうらーと@ポワヴラード!そーす@ソース・---(標準)}
\index{sauce@sauce!petites brunes composees@Petites ---s Brunes Composées!poivrade ordinaire@--- Poivrade}
\index{sauce@sauce!poivrade ordinaire@--- Poivrade ordinaire}
\index{poivrade@poivrade!sauce ordinaire@Sauce --- ordinaire}

細かいさいの目に切ったにんじん100 gと玉ねぎ80 g、刻んだパセリの茎、タ
イム少々、ローリエの葉少々からなる\protect\hyperlink{mirepoix}{ミルポワ}を油で色付くま
で炒める。

ヴィネガー1 dlとマリナード2 dlを注ぎ、\untiers{}量になるまで煮詰める。
\protect\hyperlink{sauce-espagnole}{ソース・エスパニョル}1
Lを注ぎ、約45分間煮込む。

ソースを漉す10分前に、大粒のこしょう8個を叩きつぶして加える。ソースに
こしょうを入れてからの時間がこれ以上少しでも長いと、こしょうの風味が支
配的になり過ぎることになるので注意。

漉し器で香味素材を軽く押すようにして漉す。\protect\hyperlink{marinades-et-saumuresux5cux257D}{マリナード}\footnote{ヴィネガーやワイン、香味素材、塩などを合わせて肉を漬け込む液体。
  マリネ液と呼ぶこともある。}2 dlでソー
スをのばす。火にかけて35分間、所定の量\footnote{明記されていないが、ここでは約1
  L。}になるまで煮詰めながら、表
面に浮いてくる不純物を徹底的に取り除く\footnote{dépouiller
  デプイエ。現代ではécumerエキュメの語を使う現場が多い。}。

さらに布で漉し、バター50 gを加えて仕上げる\footnote{現代では、バターでモンテするmonter
  au beurreという表現を用いる 現場も多い。}。

\hypertarget{sauce-poivrade-pour-gibier}{%
\subsubsection{ソース・ポワヴラード(ジビエ用)}\label{sauce-poivrade-pour-gibier}}

\frsub{Sauce Poivrade pour Gibier}

\index{そーす@ソース!ふらうんはせい@ブラウン系の派生---!ほわうらーとしひえ@---・ポワヴラード(ジビエ用)}
\index{そーす@ソース!ほわうらーとしひえ@---・ポワヴラード(ジビエ用)}
\index{ほわうらーと@ポワヴラード!そーすしひえ@ソース・---(ジビエ用)}
\index{sauce@sauce!petites brunes composees@Petites ---s Brunes Composées!poivrade gibier@--- Poivrade pour Gibier}
\index{sauce@sauce!poivrade gibier@--- Poivrade pour Gibier}
\index{poivrade@poivrade!sauce  gibier@Sauce --- pour Gibier}

細かいさいの目に切ったにんじん125 gと玉ねぎ125 g、タイムの枝と鳥類では
ないジビエ\footnote{gibier à poil
  逐語訳すると「毛の生えているジビエ」すなわち」鹿、
  猪、野うさぎなどを指す。}の端肉1
kgからなる\protect\hyperlink{mirepoix}{ミルポワ}を油で色よく炒め る。

ミルポワが色付いてきたら、鍋の油を捨てる。ヴィネガー3 dlと白ワイン2 dl
を注ぎ、完全に煮詰める。

\protect\hyperlink{sauce-espagnole}{ソース・エスパニョル}1
Lと\protect\hyperlink{fonds-de-gibier}{ジビエの茶色いフォン}2 L、
\protect\hyperlink{marinades-et-saumures}{マリナード}1 Lを加える。

鍋に蓋をして弱火にかける。可能ならオーブンがいい。3時間半〜4時間加熱す
る。

ソースを漉す8分前に、大粒のこしょう12個を叩きつぶして加える。

漉し器で材料を押すようにして漉す。

これをジビエのフォン\unquart{} Lとマリナード\unquart{} Lでのばし、再び
火にかけて40分間、表面に浮いてくる不純物を丁寧に取り除きながら、1 Lに
なるまで煮詰める。

これを布で漉し、バター75gを加えて仕上げる。

\hypertarget{sauce-poivrade-pour-gibier}{%
\subparagraph{【原注】}\label{sauce-poivrade-pour-gibier}}

一般的にはジビエ料理のソースにはバターを加えないことになっているが、本
書では軽くバターを加えることを推奨する。そうすると、ソースの色の赤みは
薄まるが、繊細で滑らかな口あたりに仕上がる。

\hypertarget{sauce-au-porto}{%
\subsubsection{ソース・ポルト}\label{sauce-au-porto}}

\frsub{Sauce au Porto}

\index{そーす@ソース!ふらうんはせい@ブラウン系の派生---!ほると@---・ポルト}
\index{そーす@ソース!ほると@---・ポルト}
\index{ほるとしゆ@ポルト酒 ⇒ ポルト!そーす@ソース・---}
\index{sauce@sauce!petites brunes composees@Petites ---s Brunes Composées!porto@--- Porto}
\index{sauce@sauce!porto@--- au Porto}
\index{porto@Porto!sauce@Sauce au ---}

マデラ酒ではなくポルト酒を用いて、\protect\hyperlink{sauce-madere}{ソース・マデール}と
同様に作る。

\hypertarget{sauce-portugaise}{%
\subsubsection{ポルトガル風ソース}\label{sauce-portugaise}}

\frsub{Sauce Portugaise}\footnote{日本でもフランス語のままソース・ポルチュゲーズと呼ばれることは
  多い。フランス料理においてポルトガル風の名称を付けた料理はトマトをベー
  スとしたものがほとんど。ただし、トマトを使うからといってポルトガル風の
  名が必ず付くというわけではない。}

\index{そーす@ソース!ふらうんはせい@ブラウン系の派生---!ほるとかるふう@ポルトガル風---}
\index{そーす@ソース!ほるとかるふう@ポルトガル風---}
\index{ほるとかるふう@ポルトガル風!そーす@---ソース}
\index{sauce@sauce!petites brunes composees@Petites ---s Brunes Composées!portugaise@--- Portugaise}
\index{sauce@sauce!porugaise@--- Portugaise}
\index{portugais@portugais(e)!sauce@Sauce Portugaise}

(仕上がり1 L分)

大きめの玉ねぎ1個を細かくみじん切りにする。鍋に油を熱し、強火で玉ねぎ
を炒める。玉ねぎがブロンド色になったら、皮を剥いて種子を取り除き、粗み
じん切りにしたトマト750 gと、つぶしたにんにく1片、塩、こしょうを加える。
トマトの酸味が強い場合は砂糖少々も加える。鍋に蓋をして、弱火で煮る。
\protect\hyperlink{essences-diverses}{トマトエッセンス}少々と、薄めに作ったトマトソース
を適量\footnote{仕上がりの全体量が1
  Lなので、トマトソースを加える量は、グラスドヴィアンド
  を加える前の段階で0.9 L程度になるよう調整する。}、温めて溶かした\protect\hyperlink{glace-de-viande}{グラスドヴィアンド}1
dl、 新鮮なパセリの葉のみじん切り大さじ1杯を加えて仕上げる。

\hypertarget{sauce-provencal}{%
\subsubsection{プロヴァンス風ソース}\label{sauce-provencal}}

\frsub{Sauce Provençale}

\index{そーす@ソース!ふらうんはせい@ブラウン系の派生---!ふろうあんすふう@プロヴァンス風---}
\index{そーす@ソース!ふろうあんすふう@プロヴァンス風---}
\index{ふろうあんすふう@プロヴァンス風!そーす@---ソース}
\index{sauce@sauce!petites brunes composees@Petites ---s Brunes Composées!provencale@--- Provençale}
\index{sauce@sauce!provencale@--- Provençale}
\index{provencal@provençal(e)!sauce@Sauce Provençale}

大ぶりのトマト12個の皮を剥き、つぶして種子は取り除いて、粗く刻む\footnote{concasser
  コンカセ。}。 ソテー鍋に2\undemi{}
dlの油を熱し、そこにトマトを入れる。塩、こしょう、
粉砂糖1つまみで味を調える。しっかりつぶしたにんにく(小)1片と細かく刻
んだパセリ小さじ1杯を加える。

蓋をして弱火で30分間程、煮溶かす。

\hypertarget{nota-sauce-provencale}{%
\subparagraph{【原注】}\label{nota-sauce-provencale}}

このソースについてはさまざまな解釈があるが、本書ではブルジョワ料理にお
ける本物の「プロヴァンス風ソース」のレシピ、つまりはトマトの「フォン
デュ」\footnote{加熱によって溶かしたもの、の意。トマトフォンデュと呼ぶ調理現場も多い。}、を収録した。

\hypertarget{sauce-regence}{%
\subsubsection{ソース・レジャンス}\label{sauce-regence}}

\frsub{Sauce Régence}\footnote{Régence(レジョンス)とは「摂政時代」、すなわちオルレアン公フィリップが幼少だったルイ15世の
  摂政を務めた時代(1715〜1723年)のこと。オルレアン公は美食家として
  有名で、とりわけシャンパーニュを好んだという。この時代はフランス宮
  廷料理の絶頂期でもあった。}

\index{そーす@ソース!ふらうんはせい@ブラウン系の派生---!れしやんす@---・レジャンス}
\index{そーす@ソース!れしやんす@---・レジャンス}
\index{れしやんす@レジャンス!そーす@ソース・---}
\index{sauce@sauce!petites brunes composees@Petites ---s Brunes Composées!regence@--- Régence}
\index{sauce@sauce!regence@--- Régence}
\index{regence@Régence!sauce@Sauce ---}

ライン産ワイン3
dlに、細かく刻んであらかじめ火を通しておいた\protect\hyperlink{mirepoix}{ミルポ
ワ}1 dlと生トリュフの切りくず25gを加え、半量になるまで煮詰
める。トリュフのシーズンでない時季はトリュフエッセンスを使う。\protect\hyperlink{sauce-demi-glace}{ソース・
ドゥミグラス}8 dlを加え、数分間弱火にかけて浮いてく
る不純物を丁寧に取り除き\footnote{dépouiller デプイエ ≒ écumer
  エキュメ。}、布で漉す。

\ldots{}\ldots{}牛、羊の大きな塊肉の料理用。

\hypertarget{sauce-robert}{%
\subsubsection{ソース・ロベール}\label{sauce-robert}}

\frsub{Sauce Robert}\footnote{この名称のソースは古くからある。文献で初めて出てくるのは16世紀
  フランソワ・ラブレーの小説『ガルガンチュアとパンタグリュエル』。そ
  の「第四の書」で料理人の名が大量に列挙される章がある。そのうちの多
  くは架空の人名だが、その中のロベールという料理人がこのソースを考案
  したと書いている。ただし、具体的にどのようなソースかまでは描写され
  ておらず「うさぎのロースト、鴨、加工していない豚肉、卵のポシェ、塩
  漬けのメルラン{[}鱈の近縁種{]}、その他まことに多くの料理に欠かせない
  ソース」と書いてあるのみ(第40章)。どんな料理にも合うと書かれてし
  まうとむしろ特徴を捉え難くなってしまう。いずれにせよ、遅くとも16世
  紀には「ソース」として成立していたと考えられる。また、17世紀のシャ
  ルル・ペロー著『物語集』の「眠れる森の美女」においても、このソース
  名が登場する一節がある。このように16世紀以降多くの文学作品をはじめ
  とする文献にこのソース名は見られる。レシピとしては、1651年刊ラ・ヴァ
  レーヌ『フランス料理の本』における「豚腰肉 ソース・ロベール添え」
  がもっとも古いもののひとつだろう。概略は、豚腰肉を、ヴェルジュ{[}未
  熟ぶどう果汁、中世料理においてよく用いられた{]}とヴィネガー、セージ
  を振り掛けながらローストする。下に置いた脂受け皿に焼いた豚肉から流
  れ落ちた脂がたまるので、これを使って玉ねぎをこんがり炒める。炒めた
  玉ねぎの上に豚後ろ身を載せ、豚腰肉をローストする際にかけたのと同じ
  ソースをかける。このソースはソースロベールと呼ばれている(p.51)。ま
  た、干鱈のソース・ロベール添えの場合は、バターとヴェルジュ少々、マ
  スタードで作るが、ケイパーやシブール{[}葱{]}を加えてもいい(p.202)と
  あり、同じ名称のソースとは見做しがたい。18世紀以降のソース・ロベー
  ルは多かれ少なかれいずれもマスタードを加える点が共通しているので、
  名称が先にあり、内容が時代とともにはっきりしたものになっていたのだ
  ろう。}

\index{そーす@ソース!ふらうんはせい@ブラウン系の派生---!ろへーる@---・ロベール}
\index{そーす@ソース!ろへーる@---・ロベール}
\index{ろへーる@ロベール!そーす@ソース・---}
\index{sauce@sauce!petites brunes composees@Petites ---s Brunes Composées!Robert@--- Robert}
\index{sauce@sauce!robert@--- Robert}
\index{robert@Robert!sauce@Sauce ---}

(仕上がり5 dl分)

大きめの玉ねぎを細かくみじん切りにし、バターで色付かないよう強火でさっ
と炒める。

白ワイン2
dlを注ぎ、\untiers{}量になるまで煮詰める。\protect\hyperlink{sauce-demi-glace}{ソース・ドゥミグ
ラス}3 dlを加え、弱火で10分間煮る。

シノワ\footnote{主として金属製で円錐形に取っ手の付いた漉し器。清朝の高級役人が
  かぶっていた帽子の形状から「中国の」を意味するchinoisの名称となっ
  たと言われている。}で漉し(これは任意。漉さなくてもいい)、火から外して、粉砂
糖1つまみとマスタード大さじ1杯を加えて仕上げる。

\hypertarget{sauce-robert-escoffier}{%
\subsubsection{ソース・ロベール・エスコフィエ}\label{sauce-robert-escoffier}}

\frsub{Sauce Robert Escoffier}

\index{そーす@ソース!ふらうんはせい@ブラウン系の派生---!ろへーるえすこふいえ@---・ロベール・エスコフィエ}
\index{そーす@ソース!ろへーるえすこふいえ@---・ロベール・エスコフィエ}
\index{ろへーる@ロベール!そーすえすこふいえ@ソース・---・エスコフィエ}
\index{sauce@sauce!petites brunes composees@Petites ---s Brunes Composées!robert escoffier@--- Robert Escoffier}
\index{sauce@sauce!robert escoffier@--- Robert Escoffier}
\index{robert@Robert!sauce escoffier@Sauce --- Escoffier}

このソースは完成品が市販されている\footnote{\protect\hyperlink{sauce-diable-escoffier}{ソース・ディアーブル・エスコフィエ}訳注参照。}。

温かい料理にも冷たい料理にもよく合う。温かい料理に合わせる場合は、同量
の\protect\hyperlink{jus-de-veau-brun}{仔牛の茶色いフォン}と混ぜること。

\ldots{}\ldots{}豚、仔牛、鶏、魚のグリル焼きに特によく合う。

\hypertarget{sauce-romaine}{%
\subsubsection{ローマ風ソース}\label{sauce-romaine}}

\frsub{Sauce Romaine}\footnote{フランス料理における「ローマ風」の名称は「イタリア風」と同様に
  とくに根拠や由来が見出せないものが多い。このソースの場合は松の実を
  使うところから、20世紀前半に活躍したイタリアの作曲家レスピーギのロー
  マ三部作のうちの「ローマの松」を想起させるが、残念ながらこの曲が作
  曲されたのは1924年、つまり本書より後なので関係はない。だが、松の実
  を採るイタリアカサマツは、アッピア街道の並木などで有名なように、イ
  タリアとりわけローマ近辺において多く見られる(だからこそレスピーギ
  が曲の題材にしたわけだが)。その意味においては、松の実を使っている
  ということがこのソース名の根拠と見ることも不可能ではないだろう。し
  かしながら、それを証明する文献、史料があるかは不明。}

\index{そーす@ソース!ふらうんはせい@ブラウン系の派生---!ろーまふう@ローマ風---}
\index{そーす@ソース!ろーまふう@ローマ風---}
\index{ろーまふう@ローマ風!そーす@---ソース}
\index{sauce@sauce!petites brunes composees@Petites ---s Brunes Composées!romaine@--- Romaine}
\index{sauce@sauce!romain@--- Romaine}
\index{romain@romain(e)!Sauce Romaine}

砂糖50 gを火にかけてブロンド色にカラメリゼ\footnote{焦がさないように弱火で混ぜながら熱で砂糖を溶かしていく。}する。これをヴィネガー
1\undemi{}
dlでのばす。砂糖を完全に溶かし込めたら、\protect\hyperlink{sauce-espagnole}{ソース・エスパニョ
ル}6 dlと\protect\hyperlink{fonds-de-gibier}{ジビエのフォン}3 dlを加
える。これを\troisquarts{}量弱まで煮詰める。布で漉し、松の実20 gをロー
ストしたものと、大きさが揃るよう選別したスミヌル干しぶどう\footnote{トルコ産の白い干しぶどう。}20
gお よびコリント干しぶとう\footnote{ギリシア産の黒い小粒の干しぶどう(\protect\hyperlink{sauce-moscovite}{モスクワ風ソー
  ス}参照)。}20 gを温湯でもどしたものを加えて仕上げる。

\hypertarget{ux539fux6ce8-1}{%
\subparagraph{【原注】}\label{ux539fux6ce8-1}}

上記のとおり作る場合、このソースは大型ジビエ料理用だが、ジビエのフォン
ではなく通常の\protect\hyperlink{fonds-brun}{茶色いフォン}を使えば、マリネした牛、羊肉
の料理に合わせることも可能。

\hypertarget{sauce-rouennaise}{%
\subsubsection{ルーアン風ソース}\label{sauce-rouennaise}}

\frsub{Sauce Rouennaise}\footnote{ルーアンは野生のcolvertコルヴェール、いわゆる青首鴨を家禽化した
  ルーアン鴨の産地として有名。}

\index{そーす@ソース!ふらうんはせい@ブラウン系の派生---!るーあんふう@ルーアン風---}
\index{そーす@ソース!るーあんふう@ルーアン風---}
\index{るーあんふう@ルーアン風!そーす@---ソース}
\index{sauce@sauce!petites brunes composees@Petites ---s Brunes Composées!rouannaise@--- Rouannaise}
\index{sauce@sauce!rouannaise@--- Rouannaise}
\index{rouannais@rouannais(e)!sauce@Sauce Rouannaise}

(仕上がり5 dl分)

\protect\hyperlink{sauce-bordelaise}{ボルドー風ソース}4 dl
を用意する。ただし、良質な赤
ワインを使って作ること。(\protect\hyperlink{sauce-bordelaise}{ボルドー風ソース}参照)。

中位の大きさの鴨のレバー3個を裏漉しする。こうして出来たレバーのピュレ
をソースに加え、沸騰させない程度の温度で火を通す\footnote{pocher
  ポシェする。}。絶対に沸騰させ
ないこと。沸騰させてしまうと途端にレバーのピュレが粒状になってしまう。

布で漉し、塩こしょうを効かせる。

このソースの特質\ldots{}\ldots{}エシャロットを加えた赤ワインを煮詰めたものに鴨の生
レバーのピュレを加えたもの。

\ldots{}\ldots{}ルーアン産鴨のローストには、いわば必須といってもいいソース。

\hypertarget{sauce-salmis}{%
\subsubsection{ソース・サルミ}\label{sauce-salmis}}

\frsub{Sauce Salmis}\footnote{語源は「ごった煮」を意味する salmigondis
  とするのが定説のようだ
  が、salmigondisがその意味で用いられるようになったのは19世紀以降と
  考えられ、それ以前はragoûtラグーと同義と見なされていた。ラグーはそ
  の語源的意味が「食欲をそそるもの」であり、17世紀に、それまでポター
  ジュと呼ばれていた煮込み料理についてラグーの名称をつけることが流行
  した。また、salmigondisの古い語形のひとつsalmigondinは16世紀の小説
  家フランソワ・ラブレー『ガルガンチュアとパンタグリュエル』の「第四
  の書」において用いられているが、日本語の「ごった煮」のニュアンスと
  はかなり違う意味で、美味な料理のひとつとして挙げられている。いずれ
  にしても、salmigondin, salmigondisというラグーの別称が、ある時期か
  ら鳥類を材料にしたものに限定されるようになったことは確かで、カレー
  ムの『19世紀フランス料理』ではsalmisの語で、野鳥などのラグーを呼ん
  でいる。例えば「ベカスのサルミ」「ペルドローのサルミ」など。カレー
  ムとエスコフィエを比較すると、しばしばカレームにおいてラグーとして
  ひとまとめにされていた料理とソースの組合せが、『料理の手引き』にお
  いては、例えば\protect\hyperlink{garniture-financiere}{ガルニチュール・フィナンシエール}と\protect\hyperlink{sauce-financiere}{ソース・フィ
  ナンシエール}のように、別々の項目に分離されてい るものが多くある。}

\index{そーす@ソース!ふらうんはせい@ブラウン系の派生---!さるみ@---・サルミ}
\index{そーす@ソース!さるみ@---・サルミ}
\index{さるみ@サルミ!そーす@ソース・---}
\index{sauce@sauce!petites brunes composees@Petites ---s Brunes Composées!salmis@--- Salmis}
\index{sauce@sauce!salmis@--- Salmis}
\index{salmis@salmis!sauce@Sauce ---}

ソースというよりはむしろクリ\footnote{coulis \textless{} couler
  クレ「流れる」から派生した語だが、料理用語とし
  ては、やや水分の多いピュレと理解するといい。日本では「クーリ」と呼
  ぶことも多い。ここでは二つの解釈が可能で、ひとつは\protect\hyperlink{}{ポタージュ・
  クリ}に近いという意味。もうひとつは「昔ながらのソース」の意。後
  者の場合、エスコフィエが「古典料理」と呼ぶ17、18世紀においてソース
  のことをクリと呼んでいたのを踏まえていると考えられる。}と呼んだほうがいいこのソースの作り方
はどんな場合も一点を除いて変わることがない。それは、このソースを合わせ
るジビエ(鳥)の種類によって、つまり普通に肉料理として扱えるジビエか、
肉断ち\footnote{小斉のこと。カトリックの習慣として(厳密な教義ではない)四旬節
  (復活祭までの46日間)や毎週金曜などに行なわれる、肉食を断つ行為の
  こと。}の際の食材として扱えるもの\footnote{ある種の水鳥はイルカと同様に魚と同等のものと見做され、小斉の場
  合にも食材として認められていた。具体的にはハシヒロ鴨、オナガ鴨、サ
  ルセル鴨など。もっとも、水鳥を肉断ちの際の食材として扱うというのは
  一種の詭弁ともいえなくないわけで、このソースを作る際に\protect\hyperlink{sauce-espagnole-maigre}{魚料理用ソー
  ス・エスパニョル}をベースとした\protect\hyperlink{sauce-demi-glace}{ソース・
  ドゥミグラス}を使うとは考え難く、本文にあるよう
  にフォンの代用としてマッシュルームの茹で汁を用いるという指示を守るだ
  けで、厳密に小斉の料理として成立するレシピと言えるかは疑問の残ると
  ころだ。}かで、どんな液体を用いるかと いうことだけだ。

細かく刻んだ\protect\hyperlink{mirepoix}{ミルポワ}150
gをバターでじっくり色付くまで炒め
る。そこに、その料理で用いているジビエの手羽と腿の皮、ガラを細かく刻ん
で加える。

白ワイン3
dlを注ぎ、\untiers{}量まで煮詰める。\protect\hyperlink{sauce-demi-glace}{ソース・ドゥミグラ
ス}8 dlを加えて、約45分間弱火で煮込む。漉し器で漉す
が、その際に香味野菜とガラのエキス\footnote{原文quintessence(カンテソンス)。本来の意味は錬金術でいう「第五元
  素」。16世紀の作家フランソワ・ラブレーは存命当時、自著を筆名「カン
  テサンス抽出をなし遂げたアルコフリバス師」で出版していた時期がある。
  もっとも、このカンテサンスという語自体は中世以来、料理において「エ
  キス」「美味しさの本質」程度の意味でよく用いられた。}が得られるよう、強く押し絞って
やること。こうして出来たクリを、このソースを合わせる鳥と同種のものでとっ
たフォン4 dlで薄める。

ジビエが肉断ちの食材と見做されるもので、なおかつそれを厳格に守って作ら
なければならない場合は、このときフォンの代わりにマッシュルームの茹で汁を
用いればいい。

約45分〜1時間、弱火にかけて浮いてくる不純物を丁寧に取り除いてやる\footnote{dépouiller
  デプイエ。現代ではécumerエキュメの語を用いる現場が多 い。}。
さらにソースを\deuxtiers{}以下の量になるまで煮詰める。これにマッシュルー
ムの茹で汁とトリュフエッセンスを適量加えて丁度いい濃度になるよう調製する。

布で漉し、軽くバターを加えて仕上げる\footnote{原文は légèrement
  beurrerでありそのまま訳したが、現代の調理現場 ではmonter au beurre
  バターでモンテする、という表現がよく使われる。}。

\hypertarget{ux539fux6ce8-2}{%
\subparagraph{【原注】}\label{ux539fux6ce8-2}}

仕上げの際に、ソース1 Lあたりバター約50 gを加えるが、これは任意。

\hypertarget{sauce-tortue}{%
\subsubsection{ソース・トルチュ}\label{sauce-tortue}}

\frsub{Sauce Tortue}\footnote{tortue
  (トルチュ)は海亀のこと。古くは海亀料理用のソースだった
  が、19世紀以降は仔牛の頭肉料理に合わせるのが一般的になった。なお、
  tortu(e)という形容詞があり「曲がりくねった、(性格が)ひねくれた」
  という同音異義語があるが、このソースの由来とは無関係。}

\index{そーす@ソース!ふらうんはせい@ブラウン系の派生---!とるちゆ@---・トルチュ}
\index{そーす@ソース!とるちゅ@---・トルチュ}
\index{とるちゅ@トルチュ!そーす@ソース・---}
\index{うみかめ@海亀 ⇒ トルチュ!そーすとるちゆ@ソース・トルチュ}
\index{sauce@sauce!petites brunes composees@Petites ---s Brunes Composées!tortue@--- Tortue}
\index{sauce@sauce!tortue@--- Tortue}
\index{tortue@tortue!sauce@Sauce ---}

2\undemi{}
Lの\protect\hyperlink{jus-de-veau-brun}{仔牛のフォン}を鍋で沸かし、セージ3
g、マジョラム1 g、ローズマリー1 g、バジル2 g、タイム1 g、ローリエの葉1
g、パセリの葉1つまみ、マッシュルームの切りくず25 gを投入する。蓋をして
25分間煎じる。こうして煎じた液体を漉す2分前に大粒のこしょう4個を加える。

布で漉し、\protect\hyperlink{sauce-demi-glace}{ソース・ドゥミグラス}7
dlに\protect\hyperlink{sauce-tomate}{トマトソー ス}3
dlを合わせたものに、上記で煎じた液体を、風味が際立
つ程度に適量加える。\troisquarts{}量まで煮詰め、布で漉す。仕上げにマデ
ラ酒1 dlとトリュフエッセンス少々を加え、さらにカイエンヌで風味を引き締
める。

\hypertarget{ux539fux6ce8-nota-sauce-tortue}{%
\subparagraph{【原注】
(\#nota-sauce-tortue)}\label{ux539fux6ce8-nota-sauce-tortue}}

このソースはある程度まとまった量で作る必要がある。カイエンヌを使う指示
があるからだ。それでも、カイエンヌはとても気をつけて量を加減する必要が
ある\footnote{フランス料理において(というよりも伝統的かつ一般的なフランス人
  にとって)、唐辛子の辛さは嫌われる傾向が非常に強い。}。

\hypertarget{sauce-venaison}{%
\subsubsection{ソース・ヴネゾン}\label{sauce-venaison}}

\frsub{Sauce Venaison}\footnote{Venaison(ヴネゾン)とはノロ鹿chevreuilや猪sanglierなどの大型ジ
  ビエのこと。なおニホンジカやエゾジカはcerf(セール)に分類され、フ
  ランス料理の食材としてはあまり高く評価されない傾向がある。}

\index{そーす@ソース!ふらうんはせい@ブラウン系の派生---!うねそん@---・ヴネゾン}
\index{そーす@ソース!うねそん@---・ヴネゾン}
\index{うねそん@ヴネゾン!そーす@ソース・---}
\index{おおかたしひえ@大型ジビエ ⇒ ヴネゾン!そーす@ソース・ヴネゾン}
\index{sauce@sauce!petites brunes composees@Petites ---s Brunes Composées!venaison@--- Venaison}
\index{sauce@sauce!venaison@--- Venaison}
\index{venaison@venaison!sauce@Sauce ---}

完全に仕上げた「\protect\hyperlink{sauce-poivrade-pour-gibier}{ジビエ用ソース・ポワヴラー
ド}」\troisquarts{}
Lに、\protect\hyperlink{gelee-de-groseilles-a}{グロゼイユのジュ
レ}大さじ3杯強を生クリーム1dlで溶いてから加える。

グロゼイユのジュレと生クリームを加えるのは、鍋を火から外して、提供直前
にすること。

\ldots{}\ldots{}大型ジビエ料理用。

\hypertarget{sauce-vin-rouge}{%
\subsubsection{赤ワインソース}\label{sauce-vin-rouge}}

\frsub{Sauce au Vin rouge}

\index{そーす@ソース!ふらうんはせい@ブラウン系の派生---!あかわいん@赤ワイン---}
\index{そーす@ソース!あかわいん@赤ワイン---}
\index{あかわいん@赤ワイン!そーす@---ソース}
\index{sauce@sauce!petites brunes composees@Petites ---s Brunes Composées!vin rouge@--- au Vin rouge}
\index{sauce@sauce!vin rouge@--- au Vin rouge}
\index{vin@vin!sauce rouge@Sauce au --- rouge}

「赤ワインソース」という場合、煮詰めてからブールマニエでとろみを付ける
ブルゴーニュ風の仕立てか、魚を煮るのに用いた赤ワインを使うことが特徴で
ある「ソース・マトロット」のいずれかから派生したものなのは言うまでもな
い。もっとも、後者の場合はワインの風味は失われてしまっていてソースの水
気と味付けの意味しか持っていないと言える。

両者どちらもまさしく「赤ワインソース」だが、\protect\hyperlink{sauce-bourguignonne}{ブルゴーニュ風ソー
ス}と\protect\hyperlink{sauce-matelote}{ソース・マトロット}はそれ
ぞれ作り方も用途も違うから別々の名称として、この「茶色い派生ソース」の
節で説明した。

筆者としては、本当の「赤ワインソース」は以下のように作るものと考えてい
る。

ごく細かく刻んだ標準的な\protect\hyperlink{mirepoix}{ミルポワ}125
gをバターで炒める。良 質の赤ワイン\undemi{}
Lを注ぐ。半量になるまで煮詰める。つぶしたにんに
く1片、\protect\hyperlink{sauce-espagnole}{ソース・エスパニョル}7\undemi{}
dlを加え、12〜
15分、火にかけて浮いてくる不純物を丁寧に取り除く\footnote{dépouiller
  デプイエ ≒ écumer エキュメ。}。

布で漉し、バター100 gとアンチョビエッセンス小さじ1杯、カイエンヌ1つま
みを加えて仕上げる。

\ldots{}\ldots{}魚料理用ソース。

\hypertarget{sauce-zingara-a}{%
\subsubsection{ソース・ザンガラ A}\label{sauce-zingara-a}}

\frsub{Sauce Zingara A}\footnote{もとの語形はzingaro
  ザンガロ、またはヂンガロ。ジプシー、ボヘミ
  アンの意。料理ではパプリカ粉末やカイエンヌを用いたものに命名される
  ことが多い。}

\index{そーす@ソース!ふらうんはせい@ブラウン系の派生---!さんからa@---・ザンガラA}
\index{そーす@ソース!さんからa@---・ザンガラ A}
\index{さんから@ザンガラ!そーすa@ソース・--- A}
\index{しふしーふう@ジプシー風!そーすa@ソース・ザンガラ A}
\index{sauce@sauce!petites brunes composees@Petites ---s Brunes Composées!zingara a@--- Zingara A}
\index{sauce@sauce!zingara a@--- Zingara A}
\index{zingara@Zingara!sauce a@Sauce --- A}

このソースは古典料理の\protect\hyperlink{garniture-zingara}{ガルニチュール・ザンガラ}とはまったく関係が
ない。むしろイギリス料理に由来し、本書でもイギリス風ソースの節において
似たようなものをいくつか採り上げている。

ヴィネガー2\undemi{} dlにエシャロットのみじん切り大さじ1杯を加えて半量
になるまで煮詰める。\protect\hyperlink{jus-de-veau-lie}{茶色いジュ}7
dlを注ぎ、バターで
揚げたパンの身160gを加える。弱火で5〜6分間煮る。パセリのみじん切り大さ
じ1杯とレモン\undemi{}個分の搾り汁を加えて仕上げる。

\hypertarget{sauce-zingara-b}{%
\subsubsection{ソース・ザンガラ B}\label{sauce-zingara-b}}

\frsub{Sauce Zingara B}

\index{そーす@ソース!ふらうんはせい@ブラウン系の派生---!さんからb@---・ザンガラB}
\index{そーす@ソース!さんからb@---・ザンガラ B}
\index{さんから@ザンガラ!そーすb@ソース・--- B}
\index{しふしーふう@ジプシー風!そーすb@ソース・ザンガラ B}
\index{sauce@sauce!petites brunes composees@Petites ---s Brunes Composées!zingara b@--- Zingara B}
\index{sauce@sauce!zingara b@--- Zingara B}
\index{zingara@Zingara!sauce b@Sauce --- B}

白ワイン3 dlとマッシュルームの茹で汁3 dlを合わせて\untiers{}量になるまで
煮詰める。

\protect\hyperlink{sauce-demi-glace}{ソース・ドゥミグラス}4
dlと\protect\hyperlink{sauce-tomate}{トマトソー ス}2\undemi{}
dl、\protect\hyperlink{fonds-blanc}{白いフォン}1 dlを注ぐ。
浮いてくる不純物を徹底的に取り除きながら5〜6分火にかける。

仕上げに、カイエンヌ1つまみで風味を引き締め、太さ1〜2 mmの千切りにした
\footnote{julienne
  (ジュリエーヌ)。日本語では「ジュリエンヌ」と言うこと
  が多いが、「ジュリヤン」のように言う調理現場もある。}ハム(脂身のないところ)と赤く漬けた舌肉70
gおよびマッシュルーム 50 g、トリュフ 30 gを加える。

\ldots{}\ldots{}仔牛料理、鶏料理用。
\end{recette}
\hypertarget{ux30dbux30efux30a4ux30c8ux7cfbux306eux6d3eux751fux30bdux30fcux30b9}{%
\section{ホワイト系の派生ソース}\label{ux30dbux30efux30a4ux30c8ux7cfbux306eux6d3eux751fux30bdux30fcux30b9}}

\hypertarget{petites-sauces-blanches-composuxe9es-et-de-ruxe9ductions}{%
\subsection{Petites Sauces Blanches Composées et de
Réductions}\label{petites-sauces-blanches-composuxe9es-et-de-ruxe9ductions}}

\maeaki
\begin{recette}
\hypertarget{ux30bdux30fcux30b9ux30a2ux30ebux30d3ux30e5ux30d5ux30a7ux30e91}{%
\subsubsection[ソース・アルビュフェラ]{\texorpdfstring{ソース・アルビュフェラ\footnote{ナポレオン軍の元帥、ルイ・ガブリエル・スーシェ
  Louis-Gabriel Suchet, duc d'Albufera
  (1770〜1826)のこと。スペイン戦役の際にそれ
  までの軍功を称えられ、ナポレオンが1812年にアルビュフェラ公爵位を新
  設して授けた。帝政期の英雄のひとりであり、アルビュフェラおよびスー
  シェの名を冠した料理がいくつかある。1814年に帝政が崩壊した後も軍務、
  政務に携わり、最終的にフランス貴族院議員の地位を得た。アルビュフェ
  ラ公爵位については、1815年7月24日の勅令においてに正式に抹消されて
  いる。このソースの特徴は赤ピーマン(パプリカ)を加熱してなめらかに
  すり潰し、バターに練り込んだものを使う点にあるが、どのような経緯で
  このソースに赤ピーマンを用いるようになったのかは不明。ただし、この
  ソースを合わせる「肥鶏 アルビュフェラ」は詰め物(ファルス)に米を
  用いるが、アルビュフェラは湖の周辺の湿地帯で米の生産がおこなわれて
  いるという点では一応の関連性が認められよう。なお、アルビュフェラは
  バレンシアの湖とそこに形成された潟であり、現在はバレンシア州のアル
  ブフェーラ自然公園となっている。}}{ソース・アルビュフェラ}}\label{ux30bdux30fcux30b9ux30a2ux30ebux30d3ux30e5ux30d5ux30a7ux30e91}}

\hypertarget{sauce-albufera}{%
\paragraph{Sauce Albuféra}\label{sauce-albufera}}

\index{そーす@ソース!あるひゆふえら@---・アルビュフェラ}
\index{あるひゆふえら@アルビュフェラ!そーす@ソース・---}
\index{sauce@sauce!albufera@--- Albuféra}
\index{albufera@Albuféra!sauce@Sauce ---}

\protect\hyperlink{sauce-supreme}{ソース・シュプレーム}1
Lあたりに、溶かしたブロンド色
の\protect\hyperlink{glace-de-viande}{グラスドヴィアンド}2
dlと、標準的な分量比率で作っ た\href{}{赤ピーマンバター}50 gを加える。

\maeaki

\hypertarget{ux30bdux30fcux30b9ux30a2ux30e1ux30eaux30b1ux30fcux30cc3}{%
\subsubsection[ソース・アメリケーヌ]{\texorpdfstring{ソース・アメリケーヌ\footnote{アメリケーヌという名称の由来は諸説あるが、19世紀フランスの料理人
  ピエール・フレス Pierre Fraysse がアメリカで働いた後にパリで1853年
  に開いたレストラン「シェ・ピーターズ」でこの料理名で提供したという
  のが定説。ただし、1853年以前にレストラン「ボヌフォワ」に「ラングドッ
  ク産オマール ソース・アメリケーヌ添え」というメニューあり、フレス
  はその料理に改変を加えたか、名前だけをシンプルに「アメリケーヌ」と
  した程度という説もある。かつては、オマールの主産地のひとつブルター
  ニュ地方を意味する古い形容詞 armoricain(e) アルモリカン、アルモリ
  ケーヌの音が変化した料理名だと主張されることもあったが、19世紀には
  南仏産が中心であったトマトを用いる点で矛盾が生じてしまう。いずれに
  しても、この料理名がフレスの店シェ・ピーターズを基点として広く知ら
  れるようになったことは事実と考えていい。}}{ソース・アメリケーヌ}}\label{ux30bdux30fcux30b9ux30a2ux30e1ux30eaux30b1ux30fcux30cc3}}

\hypertarget{sauce-americaine}{%
\paragraph{Sauce Américaine}\label{sauce-americaine}}

\index{そーす@ソース!あめりけーぬ@---・アメリケーヌ}
\index{あめりふう@アメリカ風!そーす@ソース・アメリケーヌ}
\index{sauce@sauce!americaine@--- Américaine}
\index{americain@américain!sauce americaine@Sauce Américaine}

このソースは\protect\hyperlink{homard-a-l-americaine}{オマール・アメリケーヌ}という料理
そのものと言っていい(「魚料理」の章、甲殻類、\protect\hyperlink{homard-a-l-americaine}{オマール・アメリケー
ヌ}参照)。

このソースは通常、オマールの身をガルニチュールとした魚料理に添えられる。
オマールの身をやや斜めになるよう厚さ1 cm程度の輪切りにし\footnote{escalopper
  エスカロペ。エスカロップに切る。ここで用いられるオマー
  ルは900g〜1kg程度の大きさのものを想定していることに注意。}、魚料理の
ガルニチュールとして供するわけだ。

\maeaki

\hypertarget{ux30a2ux30f3ux30c1ux30e7ux30d3ux30bdux30fcux30b9}{%
\subsubsection{アンチョビソース}\label{ux30a2ux30f3ux30c1ux30e7ux30d3ux30bdux30fcux30b9}}

\hypertarget{sauce-anchois}{%
\paragraph{Sauce Anchois}\label{sauce-anchois}}

\index{そーす@ソース!あんちょうい@アンチョビ---}
\index{あんちょひ@アンチョビ!そーす@---ソース}
\index{sauce@sauce!anchois@--- Anchois}
\index{anchois@anchois!sauce anchois@Sauce ---}

\href{}{ノルマンディー風ソース}8
dlを、バターを加える前のところまで作る。こ
れに\href{}{アンチョビバター}125 gを混ぜ込む。アンチョビのフィレ50
gを洗い、 よく水気を絞ってから小さなさいの目に切ったのを加えて仕上げる。

\ldots{}\ldots{}魚料理用。

\maeaki

\hypertarget{ux30bdux30fcux30b9ux30aaux30fcux30edux30fcux30eb4}{%
\subsubsection[ソース・オーロール]{\texorpdfstring{ソース・オーロール\footnote{夜明けの光、曙光のこと。オーロラの意味もあるため、日本では「オーロラソース」と呼ばれることもあるが、マヨネーズとトマトケチャップを同量で混ぜ合わせたものもそう呼ばれることが多いので注意。}}{ソース・オーロール}}\label{ux30bdux30fcux30b9ux30aaux30fcux30edux30fcux30eb4}}

\hypertarget{sauce-aurore}{%
\paragraph{Sauce Aurore}\label{sauce-aurore}}

\index{そーす@ソース!おーろーる@---・オーロール}
\index{おーろーる@オーロール!そーす@ソース・---}
\index{sauce@sauce!aurore@--- Aurore}
\index{aurore@aurore!sauce@Sauce ---}

\protect\hyperlink{veloute}{ヴルテ}に真っ赤なトマトピュレを加えたもの。分量は、ヴルテが\troisquarts{}に対し、トマトピュレ\unquart{}とする。仕上げに、ソース1
Lあたり100 gのバターを加える。

\ldots{}\ldots{}卵料理、仔牛、仔羊肉の料理、鶏料理用。

\maeaki

\hypertarget{ux9b5aux6599ux7406ux7528ux30bdux30fcux30b9ux30aaux30fcux30edux30fcux30eb}{%
\subsubsection{魚料理用ソース・オーロール}\label{ux9b5aux6599ux7406ux7528ux30bdux30fcux30b9ux30aaux30fcux30edux30fcux30eb}}

\hypertarget{sauce-aurore-maigre}{%
\paragraph{Sauce Aurore maigre}\label{sauce-aurore-maigre}}

\index{そーす@ソース!おーろーるさかなよう@魚料理用---・オーロール}
\index{おーろーる@オーロール!そーすさかな@魚料理用ソース・---}
\index{sauce@sauce!aurore maigre@--- Aurore maigre}
\index{aurore@aurore!sauce maigre@Sauce --- maigre}

\protect\hyperlink{veloute-de-poisson}{魚料理用ヴルテ}に、上記と同じ割合でトマトピュレ
を加える。ソース1 Lあたりバター125 gを加えて仕上げる。

\ldots{}\ldots{}魚料理用

\maeaki

\hypertarget{ux30d0ux30a4ux30a8ux30ebux30f3ux98a8ux30bdux30fcux30b9}{%
\subsubsection{バイエルン風ソース}\label{ux30d0ux30a4ux30a8ux30ebux30f3ux98a8ux30bdux30fcux30b9}}

\hypertarget{sauce-bavaroise}{%
\paragraph{Sauce Bavaroise}\label{sauce-bavaroise}}

\index{そーす@ソース!はいえるんふう@バイエルン風---}
\index{はいえるんふう@バイエルン風!そーす@---ソース}
\index{sauce@sauce!bavarois@--- Bavaroise}
\index{bavarois@bavarois!sauce bavaroise@Sauce Bavaroise}

ヴィネガー5
dlにタイムとローリエの葉少々とパセリの枝4本、大粒のこしょう7〜8個と、おろした\footnote{原文
  râpé \textless{} râpe
  ラープと呼ばれる器具を用いておろすが、日本のおろし金と目の大きさが違うので注意。多くの場合、マンドリーヌ
  mandrine と呼ばれる野菜用スライサーにこの機能が付属している。}レフォール\footnote{raifort
  西洋わさび、ホースラディッシュ。}大さじ2杯を加え、半量になるまで煮詰める。

この煮詰めた汁に卵黄6個を加え\footnote{卵黄を加える前に一度漉しておいたほうがいいだろう。}、\protect\hyperlink{sauce-hollandaise}{オランデーズソース}を作る要領で、バター400
gと大さじ1\undemi{}杯の水を少しずつ加えながら、ソースがしっかり乳化するまで混ぜていく。布で漉す。

\protect\hyperlink{beurre-d-ecrevisse}{エクルヴィスバター}100
gと泡立てた生クリーム大さじ2杯、さいの目に切ったエクルヴィスの尾の身を加えて仕上げる。

\ldots{}\ldots{}魚料理用のこのソースは、ムースのような仕上りにすること。

\end{recette}
\hypertarget{ux30a4ux30aeux30eaux30b9ux6599ux7406ux306eux30bdux30fcux30b9ux6e29ux88fd}{%
\section{イギリス料理のソース(温製)}\label{ux30a4ux30aeux30eaux30b9ux6599ux7406ux306eux30bdux30fcux30b9ux6e29ux88fd}}

\hypertarget{sauces-anglaises-chaudes}{%
\subsection{Sauces Anglaises Chaudes}\label{sauces-anglaises-chaudes}}
\begin{recette}
\hypertarget{ux30afux30e9ux30f3ux30d9ux30eaux30fcux30bdux30fcux30b9}{%
\subsubsection{クランベリーソース}\label{ux30afux30e9ux30f3ux30d9ux30eaux30fcux30bdux30fcux30b9}}

\hypertarget{cranberries-sauce}{%
\paragraph{\texorpdfstring{Sauce aux Airelles
(\emph{Cranberries-Sauce})}{Sauce aux Airelles (Cranberries-Sauce)}}\label{cranberries-sauce}}

\index{そーす@ソース!くらんへりー@クランベリー---}
\index{くらんへりー@クランベリー!そーす@---ソース}
\index{sauce@sauce!airelles@--- aux Airelles}
\index{airelle@airelle!sauce airelles@Sauce aux Airelles}
\index{sauce@sauce!cranberries@Cranberries-Sauce}
\end{recette}
\hypertarget{ux51b7ux88fdux30bdux30fcux30b9}{%
\section{冷製ソース}\label{ux51b7ux88fdux30bdux30fcux30b9}}

\hypertarget{sauces-froides}{%
\subsection{Sauces Froides}\label{sauces-froides}}

\index{sauce@sauce!sauces froides@sauces froides}
\index{そーす@ソース!れいせいそーす@冷製ソース}
\begin{recette}
\hypertarget{ux30a2ux30a4ux30e8ux30ea2-ux30d7ux30edux30f4ux30a1ux30f3ux30b9ux30d0ux30bfux30fc}{%
\subsubsection[アイヨリ /
プロヴァンスバター]{\texorpdfstring{アイヨリ\footnote{ailloliとも綴るが、
  ail(にんにく)+
  oil(油)の合成語。19世前半紀には既にアカデミーフランセージの辞書に収録されており、広く知られていたようだ。ブイヤベースに添えるルイユとよく似ているが、ルイユがカイエンヌを加えるのに対して、こちらはにんにくと油、塩、レモン汁と少々の水だけで作る。用途も、茹でた塩鱈やじゃがいも、茹で卵、アーティチョーク、さやいんげん、などに合わせることが多い。}
/
プロヴァンスバター}{アイヨリ / プロヴァンスバター}}\label{ux30a2ux30a4ux30e8ux30ea2-ux30d7ux30edux30f4ux30a1ux30f3ux30b9ux30d0ux30bfux30fc}}

\hypertarget{sauce-aioli}{%
\paragraph{Sauce Aïoli, ou Beurre de Provence}\label{sauce-aioli}}

\index{そーす@ソース!れいせい@冷製---!あいより@アイヨリ}
\index{そーす@ソース!れいせい@冷製---!ふろふあんすはたー@プロヴァンスバター}
\index{あいより@アイヨリ}
\index{ふろふあんす@プロヴァンス!ふろふあんすはたー@プロヴァンスバター}
\index{はたー@バター!ふろふあんすはたー@プロヴァンスバター}
\index{sauce@sauce!sauce froide@sauce froide!aioli@--- Aïoli}
\index{sauce@sauce!sauce froide@sauce froide!beurre de provence@Beurre de Provence}
\index{aioli@Aïoli!sauce@Sauce ---}
\index{provence@Provence!Beurre de Provence (Aïoli)}
\index{beurre@beurre!beurre de provence@Beurre de Provence (Aïoli)}

にんにく4片(30 g)を鉢\footnote{この種の作業には、大理石製のものが伝統的によく用いられる。。}に入れて細かくすり潰す。ここに生の卵黄1個、塩1つまみを加える。混ぜながら、2\undemi{}
dlの油\footnote{原書ではとくに言及されていないが、プロヴァンス地方ではオリーブオイルを用いることが一般的。}を初めは1滴ずつ加えていき、ソースがまとまりはじめたら糸を垂らすようにして加える。この作業は鉢に入れたままで、棒をはげしく動かして行なう。

攪拌する作業の途中、レモン1個分の搾り汁と冷水大さじ\undemi{}杯を少しずつ加えて、ソースが固くなり過ぎないようにしてやること。

\hypertarget{ux539fux6ce8}{%
\subparagraph{【原注】}\label{ux539fux6ce8}}

このアイヨリソースが分離してしまいそうな時は、卵黄をさらに1個足して、
マヨネーズと場合と同様に修正すること。

\maeaki

\hypertarget{ux30a2ux30f3ux30c0ux30ebux30b7ux30a25ux98a8ux30bdux30fcux30b9}{%
\subsubsection[アンダルシア風ソース]{\texorpdfstring{アンダルシア\footnote{いうまでもなくスペインのアンダルシア地方のことだが、トマトやオリーブオイル、チョリソなどこの地方を「想起」させる食材が使われている料理などがこの名称になっている傾向がある。ところが、トマトにしろオリーブオイルにしろアンダルシア地方特有というわけではなく、アンダルシアが産地として有名なチョリソくらいしか、料理名の根拠となり得るものはない。逆に言えば、アンダルシア地方の食文化との関係は、そこに用いられている食材以外にはないものと考えてもいい。料理名に付けられた地方名がとりたてて根拠や由来のないものであることを示す一例。}風ソース}{アンダルシア風ソース}}\label{ux30a2ux30f3ux30c0ux30ebux30b7ux30a25ux98a8ux30bdux30fcux30b9}}

\hypertarget{sauce-andalouse}{%
\paragraph{Sauce Andalouse}\label{sauce-andalouse}}

\index{そーす@ソース!れいせい@冷製---!あんたるしあふう@アンダルシア風---}
\index{あんたるしあ@アンダルシア!そーす@---風ソース}
\index{そーす@ソース!あんたるしあふう@アンダルシア風---}
\index{sauce@sauce!sauce froide@sauce froide!Andalouse@--- Andalouse}
\index{sauce@sauce!andalouse@--- Andalouse}
\index{andalous@Andalous(e)!sauce@Sauce Andalouse}

ごく固く仕上げた\protect\hyperlink{mayonnaise}{ソース・マヨネーズ}\troisquarts{}
Lに、上等な赤いトマトピュレ2\undemi{}dlを加える。小さなさいの目に切ったポワヴロン\footnote{Poivron
  いわゆる日本で青果として輸入されているパプリカ(肉厚の辛くないピーマン)とほぼ同じものだが、香辛料として用いられる粉末のパプリカと混同を避けるため、あえてフランス語をそのままカタカナに訳した。}75
gを仕上げに加える。

\maeaki

\hypertarget{ux30bdux30fcux30b9ux30dcux30d8ux30dfux30a2ux306eux5a18}{%
\subsubsection{ソース・ボヘミアの娘}\label{ux30bdux30fcux30b9ux30dcux30d8ux30dfux30a2ux306eux5a18}}

\hypertarget{sauce-bohemienne}{%
\paragraph[Sauce Bohémienne]{\texorpdfstring{Sauce Bohémienne\footnote{アイルランド出身の作曲家マイケル・ウィリアム・バルフェMichael
  William Balfe (1808〜1870)のオペラ\emph{The Bohemien
  Girl}『ボヘミアの少女』のフランス語版タイトル\href{https://archive.org/details/labohmiennegrand00balf}{\emph{La
  Bohémienne}}『ラボエミエーヌ』にちなんだものと言われている。この作品はロンドンで1843年初演、1862年に四幕形式のフランス語版がパリのオペラ=コミック劇場で上演され、大ヒットしたという。この名を冠した料理はいくつかあるが、いずれもチェコのボヘミア地方とは何の関連性も認められないため、オペラの人気作品にあやかった料理名と考えるのが妥当だろう。}}{Sauce Bohémienne}}\label{sauce-bohemienne}}

\index{そーす@ソース!れいせい@冷製---!ほへみあのむすめ@---ボヘミアの娘}
\index{ほへみあ@ボヘミア!そーす@ソース・---の娘}
\index{そーす@ソース!ほへみあ@---・ボヘミアの娘}
\index{sauce@sauce!sauce froide@sauce froide!bohemienne@--- Bohémienne}
\index{sauce@sauce!bohemienne@--- Bohémienne}
\index{bohemien@bohémien(ne)!sauce@Sauce Bohémienne}

陶製の容器に、濃厚でよく冷やした\protect\hyperlink{sauce-bechamel}{ベシャメルソース}1\undemi{}
dlと卵黄4個、塩10 g、こしょう少々、ヴィネガー数滴を入れる。

泡立て器で全体をよく混ぜ、標準的なマヨネーズを作るのとまったく同じ要領で、油1
Lとエストラゴンヴィネガー大さじ2杯程を加える。

\ldots{}\ldots{}仕上げに、マスタード大さじ1杯を加える。

\maeaki

\hypertarget{ux30bdux30fcux30b9ux30b7ux30e3ux30f3ux30c6ux30a3ux30a47}{%
\subsubsection[ソース・シャンティイ]{\texorpdfstring{ソース・シャンティイ\footnote{パリ近郊の地名。詳しくはホワイト系派生ソースの\protect\hyperlink{sauce-chantilly}{ソース・シャンティイ}訳注参照。}}{ソース・シャンティイ}}\label{ux30bdux30fcux30b9ux30b7ux30e3ux30f3ux30c6ux30a3ux30a47}}

\hypertarget{sauce-chantilly-froide}{%
\paragraph{Sauce Chantilly}\label{sauce-chantilly-froide}}

\index{そーす@ソース!れいせい@冷製---!しやんていい@---・シャンティイ}
\index{しやんていい@シャンティイ!そーす@ソース・---(冷製)}
\index{そーす@ソース!しやんていい@---・シャンティイ}
\index{sauce@sauce!sauce froide@sauce froide!chantilly@--- Chantilly}
\index{sauce@sauce!chantilly@--- Chantilly (froide)}
\index{chantilly@Chantilly!sauce@Sauce --- (froide)}

酸味付けにレモンを用いて、固く仕上げた\protect\hyperlink{mayonnaise}{ソース・マヨネーズ}\troisquarts{}
Lを用意しておく。提供直前に、ごく固く泡立てた生クリーム大さじ4杯\footnote{大さじ1杯=15ccという概念にとらわれないよう注意。原文は、大きなスプーンで泡立てた生クリームをざっくりと4回加えるイメージで書かれている。本書における通常のソースの仕上り量が約1
  Lであることを考慮すると、最低でも100ml以上は加えることになるだろう。}を加える。その後、味を\ruby{調}{ととの}える。

\ldots{}\ldots{}もっぱら、アスパラガスの冷製、温製に添える。

\hypertarget{ux539fux6ce8-1}{%
\subparagraph{【原注】}\label{ux539fux6ce8-1}}

生クリームを加えるのは、このソースを使うまさにその時にすること。前もっ
て加えておくと、ソースが分離してしまう恐れがあるので注意。

\maeaki

\hypertarget{ux30b8ux30a7ux30ceux30f4ux30a1ux98a812ux30bdux30fcux30b9}{%
\subsubsection[ジェノヴァ風ソース]{\texorpdfstring{ジェノヴァ風\footnote{あまり明確な由来はないが、ジェノヴァが地中海に面した港町であり、このソースが魚料理用であるという点で一応の説明はつくだろう。}ソース}{ジェノヴァ風ソース}}\label{ux30b8ux30a7ux30ceux30f4ux30a1ux98a812ux30bdux30fcux30b9}}

\hypertarget{sauce-genoise-froids}{%
\paragraph{Sauce Génoise}\label{sauce-genoise-froids}}

\index{そーす@ソース!れいせい@冷製---!しえのうあふう@ジェノヴァ風---}
\index{しえのうあふう@ジェノヴァ風!そーす@ソース・---(冷製)}
\index{そーす@ソース!しえのうあふう@ジェノヴァ風---}
\index{sauce@sauce!sauce froide@sauce froide!genoise@--- Génoise}
\index{sauce@sauce!genoise@--- Génoise (froide)}
\index{genois@Génois(e)!sauce@Sauce ---e (froide)}

殻と皮を剥いたばかりのピスタチオ40 gと、松の実25
g、松の実がない場合はスイートアーモンド20
gを鉢に入れてよくすり潰し、冷めた\protect\hyperlink{sauce-bechamel}{ベシャメルソース}小さじ1杯程度を加えて練ってペースト状にする。これを目の細かい網で裏漉しする。陶製の容器に卵黄6個、塩1つまみ、こしょう少々を入れる。泡立て器でよく混ぜる。油1
Lと中位の大きさのレモン2個の搾り汁を少しずつ加えてよく混ぜて乳化させていく\footnote{明記されていないが、ソースをしっかりと乳化させるためには\protect\hyperlink{mayonnaise}{マヨネーズ}と同様に作業すること。}。仕上げにハーブのピュレ大さじ3杯を加える。これは、パセリの葉とセルフイユ\footnote{cerfeuil
  チャービル。}、エストラゴン\footnote{estragon フレンチタラゴン。}、時季が合えばサラダバーネットを同量ずつ用意し、強火で2分間下茹でしてから湯をきり、冷水にさらしてから水気を強く絞り、裏漉しして作っておく。

\ldots{}\ldots{}冷製の魚料理全般に合わせられる。

\maeaki

\hypertarget{ux30bdux30fcux30b9ux30b0ux30eaux30d3ux30c3ux30b7ux30e5}{%
\subsubsection{ソース・グリビッシュ}\label{ux30bdux30fcux30b9ux30b0ux30eaux30d3ux30c3ux30b7ux30e5}}

\hypertarget{sauce-gribiche13}{%
\paragraph[Sauce Gribiche]{\texorpdfstring{Sauce Gribiche\footnote{由来不明の語。ノルマンディ方言で「子どもを怖がらせるおっかない
  おばさん」という意味で用いられるということがわかっているのみ。19世
  紀後半もしくは20世紀初頭に創案されたソースと思われる。本書初版には
  当然のように既に収録されており、その後の大きな異同もない。ただ、本
  書初版以前に出版された料理書においてこのソースのレシピはまだ見つかっ
  ていない。ファーヴルは1905年刊の『料理および食品衛生事典』第二版で
  「ある種のレムラードにレストランで付けられた名称」と定義し、掲載し
  ているレシピは本書初版のものと大差ないが、「ウスターシャソース少々
  も加える」となっているところが目を引く。また、1913年初版のプルース
  トの長編小説『失なわれた時を求めて』の「スワン家の方へ」冒頭におい
  て「彼(=スワン)を招いていない夕食会のために、ソース・グリビッシュ
  やパイナップルのサラダのレシピが必要になるや、ためらいもなく探しに
  行かせたりするのだった」(p.18)。もしこの語り手の記述が正確であるな
  ら、19世紀末には広く知られたものであったと考えるべきだが、小説の場
  合は必ずしも歴史的事実と符号するわけではないので注意が必要。}}{Sauce Gribiche}}\label{sauce-gribiche13}}

\index{そーす@ソース!れいせい@冷製---!くりひつしゆ@---・グリビッシュ}
\index{くりひつしゆ@グリビッシュ!そーす@ソース・---(冷製)}
\index{そーす@ソース!くりひつしゆ@グリビッシュ---}
\index{sauce@sauce!sauce froide@sauce froide!gribiche@--- Gribiche}
\index{sauce@sauce!gribiche@--- Gribiche (froide)}
\index{gribiche@!gribiche!sauce@Sauce --- (froide)}

茹であがったばかりの固茹で卵の黄身6個を陶製のボウルに入れ、マスタード
小さじ1杯、塩1つまみ強、こしょう適量を加えてよく練り、滑らかなペースト
状にする。植物油\undemi{} Lとヴィネガー大さじ1\undemi{}杯を加えながら
よく混ぜて乳化させる。仕上げに、コルニション\footnote{長さ3〜4cm程度の小さいうちに収穫してヴィネガー漬けにしたきゅうり。専用品種が用いられる。}とケイパーのみじん切
り計100 gと、パセリとセルフイユ、エストラゴンのみじん切りのミックスを
大さじ1杯、短かめの千切り\footnote{julienne
  ジュリエンヌ。1〜2mm角の千切り。}にした固茹で卵の白身3個分を加える。

\ldots{}\ldots{}冷製の魚料理に添えるのが一般的。
\end{recette}
\hypertarget{ux5408ux308fux305bux30d0ux30bfux30fc}{%
\section{合わせバター}\label{ux5408ux308fux305bux30d0ux30bfux30fc}}

\vspace{0\zw}

\hypertarget{ux30b0ux30eaux30ebux30bdux30fcux30b9ux306eux88dcux52a9ux6750ux6599ux30aaux30fcux30c9ux30d6ux30ebux7528}{%
\subsection{グリル、ソースの補助材料、オードブル用}\label{ux30b0ux30eaux30ebux30bdux30fcux30b9ux306eux88dcux52a9ux6750ux6599ux30aaux30fcux30c9ux30d6ux30ebux7528}}

\vspace*{-1.5\zw}

\hypertarget{beurres-composuxe9s-pour-adjuvants-de-sauces-et-hors-doeuvre}{%
\subsection{Beurres Composés pour Adjuvants de Sauces et
Hors-d'oeuvre}\label{beurres-composuxe9s-pour-adjuvants-de-sauces-et-hors-doeuvre}}

\index{あわせはたー@合わせバター} \index{はたー@バター ⇒ 合わせバター}
\index{ふーるこんほせ@ブール・コンポゼ ⇒ 合わせバター}
\index{みつくすはたー@ミックスバター ⇒ 合わせバター}
\index{beurre@beurre!beurres composes@Beurres Composés}

\hypertarget{observation-sur-les-beurres-composes}{%
\subsection{概説}\label{observation-sur-les-beurres-composes}}

本書においてレシピを掲載している合わせバター\footnote{beurre composé
  ブール・コンポゼ。ミックスバターとも。少なくとも
  バターは中世以来用長く用いられてきた食材だが、中世〜ルネサンスにお
  いては獣脂(もっぱらラード)のほうが多く用いられる傾向にあった。17
  世紀以降はたとえばラ・ヴァレーヌ『フランス料理の本』におけるアスパ
  ラガスの白いソース添え(\protect\hyperlink{sauce-hollandaise}{ソース・オランデーズ}
  訳注参照)のように、バターを料理に用いることが中世の料理書と比較す
  ると圧倒的に増えたのは事実である。ムノンの1741年刊『ブルジョワ屋敷
  に勤める女性料理人のための本』のバターの項には「良質のバターを用い
  ることは料理でとても重要なことであり、バターが匂いを放っているよう
  ではどんな素晴しい皿も台無しだ。料理担当の女中であればこのことをよ
  く理解しておくことと、良質なバターの価格を手に入れるのに金を惜しん
  ではならないことを肝に銘じておくこと。最良のバターは自然な黄色をし
  ており、白いものは大抵の場合、さして美味しくない。バルボットという
  植物から採った黄色で着色されたバターもある。こういうバターの色は、
  自然なバターの黄色よりもくすんだもので、慣れれば簡単に見分けること
  が出来る(p.320)」}のうちのほとんどは、甲
殻類の合わせバターを除いて、料理に直接用いられることがとても少ない。だ
が、合わせバターはさまざまなシチュエーションで役に立つ。ポタージュでは
野菜の合わせバターが、その他の合わせバターはソース作りにおいて有用だ。
ソースの風味と性格を明確に伝える決め手になるからだ。

だから、読者である料理人諸君には、ここに書いてあることを真剣に読みとっ
ていただきたい。\href{原文における内容矛盾。この後のパラグラフは甲殻類の\%20バターについての注意点ばかりが目立つ}{}

甲殻類のバターについては、経験上、湯煎にかけながら煮出して\footnote{infuser
  アンフュゼ。}から、氷
水で冷やした陶製の容器に布で漉し入れるのがいい。そうすれば、冷たい状態
で作るよりも赤みがきれいに出る。逆に、熱によって風味の繊細さは失なわれ
てしまい、雑味さえも出てしまう。

この問題点を解決するために、我々は二種類の違うバターを作るという方式を
採ることにした。ひとつは甲殻類の胴のクリーム状の部分と切りくずあるいは
身そのものを生のバターとともに鉢ですり潰して、目の細かい網で裏漉しする
か、布で漉すというもの。このバターはソースに完璧ともいうべき風味を添え
てくれる。とりわけベシャメルソースをベースとしたソースの場合はそうだ。

もうひとつは、甲殻類の殻だけを用いて、熱して作るものだ。これは「色付け」
の役割しか持たない。この方式はまことに素晴しい結果を得られるので、ぜひ
とも実行していただきたい。

場合によっては、我々はバターを同様の上等な生クリームに代えることがある。
生クリームのほうがバターよりも、素材の持つ風味や香気をよく吸収する。こ
うすればソースやポタージュの仕上げに加えるのに文句ないクリ\footnote{Coulis
  水分のやや多いピュレをイメージするといい。}を作ることが 出来るわけだ。

色付け用のバターを使うと、ソースがきれいに色付き、個性的なソースとなる。
どんな場合でも、カルミン色素\footnote{コチニール色素ともいう。ラックカイガラムシなどを原料として抽出し
  た色素。ヨーロッパでは古代から中世にかけてケルメスカイガラムシから
  抽出され利用されてきた、非常に歴史の古い色素。とりわけルネサン期に
  は高級毛織物の染料として需要が高まった。また絵の具にも使用された。
  その後、ウチワサボテンでエンジムシを大量に養殖していた中南米を支配
  下に置いたスペインが、これを新大陸産のカルミンとしてヨーロッパ各国
  に売ることで巨万の富を得たという。かつて食品工業において多用された。
  1838年の『ラルース・ガストロノミック』初版では、「コチニールから抽
  出される鮮かな赤色色素で毒性はない。多くの食品に着色料として用いら
  れている」とある。現在は食物アレルギーの原因物質すなわちアレルゲン
  となり得ることがわかり、使用は減りつつある。現在は代替品としてビー
  ツから抽出したビートレッドなどの使用が増えてきている。また、この本
  文でカルミン色素の使用を「くすんだ、情けない色合いを与える」として
  否定的に扱っているのは、この色素がpHによって色調が変化し、なおかつ
  蛋白質を多く含む料理に加えると紫色に変化する(ソースやポタージュ全
  体が濁ったような色になる)ことがあるためだろう。}よりもずっといい。カルミン色素はソース
やポタージュにくすんだ、なさけない色合いしか与えてはくれないのだ。

合わせバターは一般的に、使う際にその都度作る\footnote{原文 au moment
  (オモモン)その都度、の意。à la minute (アラミニュッ
  ト)と呼ぶ調理現場もある。}ものだが、作り置き
しておかなければならない場合は、白い紙で円筒形に包んで冷蔵保管すること。

\vspace*{1.7\zw}
\begin{recette}
\hypertarget{ux306bux3093ux306bux304fux30d0ux30bfux30fc}{%
\subsubsection{にんにくバター}\label{ux306bux3093ux306bux304fux30d0ux30bfux30fc}}

\hypertarget{beurre-d-ail}{%
\paragraph{Beurre d'Ail}\label{beurre-d-ail}}

\index{はたー@バター!あわせはたー@合わせバター!にんにくはたー@にんにくバター}
\index{あわせはたー@合わせバター!にんにくはたー@にんにくバター}
\index{にんにく@にんにく!はたー@---バター}
\index{beurre@beurre!beurres composes@Beurres Composés!beurre d'ail@Beurre d'Ail}
\index{ail@ail!beurre@Beurre d'---}

皮を剥いたにんにく200 gを強火でしっかり茹でる\footnote{生のにんにくには胃腸を刺激する酵素が含まれているが、熱により不活
  性化するので、よく火を通す必要がある。}。よく湯をきってから、鉢
に入れてすり潰し、バター250 gと合わせ、布で漉す。

\maeaki

\hypertarget{ux30a2ux30f3ux30c1ux30e7ux30d3ux30d0ux30bfux30fc}{%
\subsubsection{アンチョビバター}\label{ux30a2ux30f3ux30c1ux30e7ux30d3ux30d0ux30bfux30fc}}

\hypertarget{beurre-d-anchois}{%
\paragraph{Beurre d'Anchois}\label{beurre-d-anchois}}

\index{はたー@バター!あわせはたー@合わせバター!あんちよひはたー@アンチョビバター}
\index{あわせはたー@合わせバター!あんちよひはたー@アンチョビバター}
\index{あんちよひ@アンチョビ!はたー@---バター}
\index{beurre@beurre!beurres composes@Beurres Composés!beurre d'anchois@Beurre d'Anchois}
\index{anchois@anchois!beurre@Beurre d'---}

アンチョビのフィレ200 gをよく洗い、しっかり水気を絞る。これを鉢に入れ
て細かくすり潰す。バター250 gを加えて布で漉す。

\maeaki

\hypertarget{ux30a2ux30fcux30e2ux30f3ux30c9ux30d0ux30bfux30fc}{%
\subsubsection{アーモンドバター}\label{ux30a2ux30fcux30e2ux30f3ux30c9ux30d0ux30bfux30fc}}

\hypertarget{beurre-d-amande}{%
\paragraph{Beurre d'Amande}\label{beurre-d-amande}}

\index{はたー@バター!あわせはたー@合わせバター!あーもんとはたー@アーモンドバター}
\index{あわせはたー@合わせバター!あーもんとはたー@アーモンドバター}
\index{あーもんと@アーモンド!はたー@---バター}
\index{beurre@beurre!beurres composes@Beurres Composés!beurre d'amande@Beurre d'Amande}
\index{amande@amande!beurre@Beurre d'---}

アーモンド\footnote{アーモンドには一般的なスイートアーモンド amandes
  doucesと、苦味 のあるビターアーモンドamande
  amèresの二種がある。後者はあまり多く
  使われることはないが、香りがいいためリキュールなどの香り付けにごく
  少量が用いられることがある。}150
gを湯むきしてよく洗い、すぐに水数滴を加えてすり潰し
てペースト状にする。これをバター250 gと混ぜ合わせ、布で漉す。

\maeaki

\hypertarget{ux30d6ux30fcux30ebux30c0ux30f4ux30eaux30fcux30cc8}{%
\subsubsection[ブール・ダヴリーヌ]{\texorpdfstring{ブール・ダヴリーヌ\footnote{アヴリーヌはヘーゼルナッツの仲間でセイヨウハシバミの大粒な変種。
  イタリア、ピエモンテ産やシチリア産が有名。}}{ブール・ダヴリーヌ}}\label{ux30d6ux30fcux30ebux30c0ux30f4ux30eaux30fcux30cc8}}

\hypertarget{beurre-d-aveline}{%
\paragraph{Beurre d'Aveline}\label{beurre-d-aveline}}

\index{はたー@バター!あわせはたー@合わせバター!ふーるたうりーぬ@ブール・ダヴリーヌ}
\index{あわせはたー@合わせバター!ふーるたうりーぬ@ブール・ダヴリーヌ}
\index{あうりーぬ@アヴリーヌ!ふーる@ブール・---}
\index{へーせるなつつ@ヘーゼルナッツ!ふーるたうりーぬ@ブール・ダヴリーヌ}
\index{beurre@beurre!beurres composes@Beurres Composés!beurre d'aveline@Beurre d'Aveline}
\index{aveline@aveline!beurre@Beurre d'---}

アヴリーヌ150 gを焙煎して丁寧に皮を剥く。油が浮いてこないよう水を数滴
加えてペースト状にすり潰す。これとバター250 gを混ぜ合わせる。目の細か
い網で裏漉しするか、布で漉す。

\maeaki

\hypertarget{ux30d6ux30fcux30ebux30d9ux30ebux30b7ux30fc9}{%
\subsubsection[ブール・ベルシー]{\texorpdfstring{ブール・ベルシー\footnote{\protect\hyperlink{sauce-bercy}{ソース・ベルシー}訳注参照。}}{ブール・ベルシー}}\label{ux30d6ux30fcux30ebux30d9ux30ebux30b7ux30fc9}}

\hypertarget{beurre-bercy}{%
\paragraph{Beurre Bercy}\label{beurre-bercy}}

\index{はたー@バター!あわせはたー@合わせバター!ふーるへるしー@ブール・ベルシー}
\index{あわせはたー@合わせバター!ふーるたへるしー@ブール・ベルシー}
\index{へるしー@ベルシー!ふーる@ブール・---}
\index{beurre@beurre!beurres composes@Beurres Composés!beurre bercy@Beurre Bercy}
\index{bercy@Bercy!beurre@Beurre ---}

白ワイン2 dlに細かく刻んだエシャロット大さじ1杯を加えて半量になるまで
煮詰める。生温い程度まで冷ましてから、ポマード状に柔らかくしたバター 200
gを混ぜ込む。牛骨髄500 gをさいの目に切って\footnote{原文 couper en
  dés。フランス語のまま「デにする(切る)」と表現することもある。}、沸騰しない程度の
湯で火を通し、よく湯ぎりをして加える。パセリのみじん切り大さじ1杯と塩8
g、挽きたてのこしょう1つまみ強とレモン\undemi{}個分の果汁を加えて仕上
げる。

\maeaki

\hypertarget{ux30adux30e3ux30d3ux30a2ux30d0ux30bfux30fc}{%
\subsubsection{キャビアバター}\label{ux30adux30e3ux30d3ux30a2ux30d0ux30bfux30fc}}

\hypertarget{beurre-de-caviar}{%
\paragraph{Beurre de Caviar}\label{beurre-de-caviar}}

\index{はたー@バター!あわせはたー@合わせバター!きやひあはたー@キャビアバター}
\index{あわせはたー@合わせバター!きやひあはたー@キャビアバター}
\index{きやひあ@キャビア!はたー@---バター}
\index{beurre@beurre!beurres composes@Beurres Composés!beurre caviar@Beurre de Caviar}
\index{caviar@caviar!beurre@Beurre de ---}

圧縮キャビア\footnote{もとはロシアで雪の中の樽で保存するために圧縮したもの。キャビア
  のグレードはベルガ、オセトラ、セヴルガが混ざっているのが多いという。}75
gを細かくすり潰す。パター250 gを加えて、布で漉す。

\maeaki

\hypertarget{ux30d6ux30fcux30ebux30b7ux30f4ux30ea12-ux30d6ux30fcux30ebux30e9ux30f4ux30a3ux30b4ux30c3ux30c813}{%
\subsubsection[ブール・シヴリ /
ブール・ラヴィゴット]{\texorpdfstring{ブール・シヴリ\footnote{\protect\hyperlink{sacue-chivry}{ソース・シヴリ}訳注参照。}
/ ブール・ラヴィゴット\footnote{\protect\hyperlink{sauce-ravigote}{ソース・ラヴィゴット}訳注参照。}}{ブール・シヴリ / ブール・ラヴィゴット}}\label{ux30d6ux30fcux30ebux30b7ux30f4ux30ea12-ux30d6ux30fcux30ebux30e9ux30f4ux30a3ux30b4ux30c3ux30c813}}

\hypertarget{beurre-chivry}{%
\paragraph{Beurre Chivry}\label{beurre-chivry}}

\index{はたー@バター!あわせはたー@合わせバター!しうり@ブール・シヴリ}
\index{あわせはたー@合わせバター!しうり@ブール・シヴリ}
\index{はたー@バター!あわせはたー@合わせバター!らういこつと@ブール・ラヴィゴット}
\index{あわせはたー@合わせバター!らういこつと@ブール・ラヴィゴット}
\index{しうり@シヴリ!ふーる@ブール・---}
\index{らういこつと@ラヴィゴット!ふーる@ブール・---}
\index{beurre@beurre!beurres composes@Beurres Composés!beurre chivry@Beurre Chivrya}
\index{beurre@beurre!beurres composes@Beurres Composés!beurre ravigote@Beurre Ravigote}
\index{chivry@Chivry!beurre@Beurre ---}
\index{ravigote@ravitote!beurre@Beurre ---}

パセリの葉とセルフイユ、エストラゴン、シヴレット、若摘みのサラダバーネッ
ト100 gを数分間下茹でし、水にさらしてから圧して余分な水気を絞る。エシャ
ロットのみじん切り25 gも下茹でする。これらを鉢に入れてすり潰す。

バター125 gを加え、布で漉す。

\maeaki

\hypertarget{ux30d6ux30fcux30ebux30b3ux30ebux30d9ux30fcux30eb14}{%
\subsubsection[ブール・コルベール]{\texorpdfstring{ブール・コルベール\footnote{\protect\hyperlink{sauce-colbert}{ソース・コルベール}本文および訳注参照。}}{ブール・コルベール}}\label{ux30d6ux30fcux30ebux30b3ux30ebux30d9ux30fcux30eb14}}

\hypertarget{beurre-colbert}{%
\paragraph{Beurre Colbert}\label{beurre-colbert}}

\index{はたー@バター!あわせはたー@合わせバター!ふーるこるへーる@ブール・コルベール}
\index{あわせはたー@合わせバター!ふーるこるへーる@ブール・コルベール}
\index{こるへーる@コルベール!ふーる@ブール・---}
\index{beurre@beurre!beurres composes@Beurres Composés!beurre colbert@Beurre Colbert}
\index{colbert@Colbert!beurre@Beurre ---}

\protect\hyperlink{beurre-maitre-d-hotel}{メートルドテルバター}200lgに、溶かした\protect\hyperlink{glace-de-viande}{グラス
ドヴィアンド}大さじ2杯と細かく刻んだエストラゴン小さ じ2杯を加える。

\maeaki

\hypertarget{ux8272ux4ed8ux3051ux7528ux306eux8d64ux3044ux30d0ux30bfux30fc}{%
\subsubsection{色付け用の赤いバター}\label{ux8272ux4ed8ux3051ux7528ux306eux8d64ux3044ux30d0ux30bfux30fc}}

\hypertarget{beurre-colorant-rouge}{%
\paragraph{Beurre Colorant rouge}\label{beurre-colorant-rouge}}

\index{はたー@バター!あわせはたー@合わせバター!いろつけようのあかいはたー@色付け用の赤いバター}
\index{あわせはたー@合わせバター!いろつけようのあかいはたー@色付け用の赤いバター}
\index{ちやくしよくそざい@着色素材!いろつけようのあかいはたー@色付け用の赤いバター}
\index{beurre@beurre!beurres composes@Beurres Composés!beurre colorant rouge@Beurre Colorant rouge}
\index{colorant@colorant!beurre rouge@Beurre --- rouge}

出来るだけ沢山の甲殻類の殻などの残りをまとめて用意する。殻の内側、外側に張り付いている膜などをきれいに取り除く。よく乾燥させてから、鉢\footnote{伝統的には大理石製の鉢が用いられることが多かった。}に入れて細かく粉砕して、同じ重さのバターを加える。これを湯煎にかけてよく混ぜながら溶かす。氷水を入れた陶製の器に、布で漉し入れる。固まったバターをトーション\footnote{\protect\hyperlink{sauce-verte}{ソース・ヴェルト}訳注参照。}で包み、余計な水を絞り出す。

\hypertarget{ux539fux6ce8}{%
\subparagraph{【原注】}\label{ux539fux6ce8}}

この色付け用のバターを作るのに用いる甲殻類の殻がどうしてもない場合は、\protect\hyperlink{beurre-de-paprika}{パプリカバター}を用いてもいいだろう。だがいずれにせよ、どんなソースであっても、仕上りの色合いを決めるには、出来るだけ、他の植物由来の赤色着色料の使用は避けることを勧める\footnote{この原注は第三版から。「植物由来の赤色着色料」といっているのはおそらくカルミン色素(コチニール色素)のことと思われる(\protect\hyperlink{observation-sur-les-beurres-composes}{合わせバター「概説」参照})。他に赤系着色料として、ベニバナ色素、紅麹などもあるが、いずれも中国や日本において発達しことを考慮すると、両大戦間である1920年頃に「避けるべき」というほど普及していたのは、実際には昆虫由来であるコチニール色素と思われる。なお、ベニバナ色素も化学的にはカルミン酸色素。}。
\end{recette}
\href{未、原文対照チェック}{} \href{未、日本語表現校正}{}
\href{未、その他修正}{} \href{未、原稿最終校正}{}

\hypertarget{marinades-et-saumures}{%
\section[マリナードとソミュール]{\texorpdfstring{マリナードとソミュール\footnote{マリナードはマリネ液とも言う。marinade
  \textless{} mariner
  (マリネ)語源はラテン語のmare(海)。古フランス語では「海で泳ぐ、海に潜る」の意で使われていたが、16世紀には既に、料理用語として用いられていたようだ。ラブレー『ガルガンチュアとパンタグリュエル』第四の書(1548年)において、lancerons
  marinez (マリネしたブロシェの幼魚)という表現が見られる。なおブロシェ
  brochet
  はノーザンパイク、和名キタカワカマス。川カマス属の淡水、汽水魚。この場面はパンタグリュエルに「小斉」のご馳走として捧げられた料理のリストの一部であり、「塩漬けのメルルーサ、卵料理各種、モリュ(塩漬けにした鱈)、アドック(塩漬け後に燻製にした鱈)」などとともに列挙されており、いずれも塩辛いから、食後の消化をよくするために飲むワインの量が倍になった(p.681)とある。したがって、lancerons
  marinezのマリネとは「海水あるいは塩水に漬けた」の意に解釈されよう。一方、ソミュールについては、11世紀末頃に、「保存のため漬け込む塩水」の意味で
  salmuire
  という語形が使用され、16世紀には「塩水およびその他の液体からなるもの」としてsaumureという現在とおなじ語形が記録されている。マリナードとソミュールが明確に分化したのはおそらく17世紀頃。1651年刊ラ・ヴァレーヌ『フランス料理の本』に見られるマリナードの語には曖昧さが残っているが、例えば
  \emph{Poulets
  marinez}(鶏のマリネ)というレシピは「鶏を開いて叩き、しっかり味付けしたヴィネガーに漬ける。小麦粉をまぶすか、卵と小麦粉で作った衣を付けてラードで揚げる。マリナードに戻し入れて軽く弱火で煮てから供する(p.36)」。また、\emph{Longe
  de
  mouton}(仔羊の腰肉のロースト)は、「棒状に切った豚背脂をラルデ針を使って刺し込み、串を刺してローストする。玉ねぎ、塩、こしょう、ごく少量のオレンジまたはレモンの外皮(ゼスト)とブイヨンとヴィネガーでマリナードを作る。肉に火が通ったら、ソース(マリナード)とともに弱火で煮込む。とろみ付けには小麦粉をラードで茶色くなるまで炒めたもの、すなわち後代のルーの原型といえるものを少々加える(p.80)」とあり、別の項目では「(串を刺した肉の下の受け皿にある)マリナードを小まめにかけながらローストする(p.106)」とある。全体的な印象としてはラ・ヴァレーヌのマリナードとは中世のドディーヌにヴィネガーを効かせたもののようにも受け取れるが、最初に見たように、「漬け込む」ものとしてもヴィネガーを用いている点に注目すべきだろう。これは18、19世紀に引継がれ、1756年マラン『コモス神の贈り物』第1巻において、\emph{Cervelle
  de veau en
  marinade}(仔牛の脳のマリナード仕立て)では、血抜きした仔牛の脳を豚背脂のシートで包みブイヨン少々で茹で、「冷ましてからヴィネガーもしくはレモン果汁に漬け込む。その後、水気をきって溶き卵に浸し、パン粉をつけて揚げる。小麦粉を溶いた揚げ衣に浸して揚げてもいい(p.206)」とある。19世紀初頭のヴィアールも同様で、『帝国料理の本』初版(1806年)において、\emph{Pieds
  d'agneau en marinade} (仔羊の足のマリナード仕立て)などいくつかの
  marinadeを冠するレシピが掲載されている。肝心のマリナードについての記述は欠落しているが、この版においてはよく見られる現象。なお、仔羊の足のマリナード仕立ては、マリナードがない場合は「塩、こしょう、ビネガーに、茹でた仔羊の足を漬けてから、揚げ衣を付けて揚げる(p.214)」となっている。1814年ボヴィリエ『調理技法』では「加熱マリナード」のレシピが掲載されている。これは、卵くらいの大きさのバターを鍋に入れ、輪切りにしたにんじん1、2本、同様にした玉ねぎ、ローリエの葉1枚、にんにく1片、タイム、バジル、枝ごとのパセリ、シブール(≒葱)2〜3本を加えて強火で炒める。野菜が色付きはじめたら、約250
  mLの白ワインヴィネガーと約0.5
  Lの水を注ぎ、塩、こしょうする。そのまま沸かしてから漉し、必要に応じて使う(pp.60-61)、というもの。もっとも、仔牛の脳のマリナード仕立てなどマランのレシピと大差ない揚げものも同書に掲載されている。また、1834年版のオドにおいても鶏のマリナードはラ・ヴァレーヌのものと同工異曲のものに留まっている。1837年版でロースト用マリナードの項が追加され、豚背脂とにんにく1片を細かく刻み、パセリ1つまり、塩、こしょう、ヴィネガー大さじ1杯、油大さじ4杯を合わせてよく混ぜる
  (p.419)。1853年版ではマリネしたうなぎのグリル焼き、というレシピが掲載される。これは、皮を剥いてぶつ切りにし、バターでソテーしたうなぎを深皿に並べ、塩、こしょうハーブ、マッシュルーム、細かく刻んだエシャロットとシブールを被せ、油大さじ1杯をかける。2〜3時間マリネしたら、パン粉をまぶしてグリル焼きする(p.310)というもの。また、
  mariner(マリネ)という動詞は、オドの1834年版でもに、ノロ鹿の腿肉のローストにおいて、「オリーブオイルと塩で5〜6時間マリネする」
  (p.155)という記述が見られる。1867年刊グフェ『料理の本』では、ヴィネガーをベースとしたソースとしてのマリナード(p.404)と仕立てとしてのマリナードがある。後者の例としては
  \emph{Tête de veau en marinade}
  (仔牛の頭 マリナード仕立て)が好例だろう。仔牛の頭肉半分を3
  cm角に切り、下茹でしてから水にさらし、牛脂と小麦粉、香草類を加えた湯で茹でる。これを、塩、こしょう、油、ヴィネガーに1時間漬け込む。水気をきって揚げ衣を付けて油で揚げる(p.156)。ここでは肉を漬け込む液体としてmarinadeの語が用いられている。このように、marinadeという名詞とmariner「漬け込む」という動詞の用法に若干の不統一が見られるため、『料理の手引き』におけるマリナードすなわちマリネ液、という概念は比較的新しいものと思われる。}}{マリナードとソミュール}}\label{marinades-et-saumures}}

\frsec{Marinades et Saumures}

\index{marinade saumures@marinade et saumures} \index{marinade@marinade}
\index{まりなーととそみゆーる@マリナードとソミュール}
\index{まりなーと@マリナード}

\vspace{1\zw}

マリナードとソミュールにはいろいろな種類があるが、最終的な目的は同じで、

\begin{enumerate}
\def\labelenumi{\arabic{enumi}.}
\item
  素材に料理で使う香辛料やハーブの香りを浸み込ませる
\item
  ある種の肉を柔らかくさせる
\item
  場合によっては保存のために用いる。とりわけ温度と湿度で素材が駄目になってしまうような場合。さらに、目指す料理の仕上がりに合わせて素材の状態を調節する
\end{enumerate}
\begin{recette}
\hypertarget{marinade-instantanee}{%
\subsubsection{即席マリナード}\label{marinade-instantanee}}

\frsub{Marinade instantanée}

\index{marinade@marinade!marinade instantanee@marinade instantanée}
\index{まりなーと@マリナード!そくせき@即席---}

このマリナードはすぐに素材を使う場合、例えば赤身肉のグリル焼きや、ガランティーヌ、テリーヌ、パテのような冷製料理の補助材料\footnote{具体的には\protect\hyperlink{farces}{ファルス}のこと。}にする肉に用いる。

\begin{enumerate}
\def\labelenumi{\arabic{enumi}.}
\item
  グリル焼きにする肉の場合\ldots{}\ldots{}ごく薄くスライスしたエシャロットとパセリの枝、タイムの枝、ローリエの葉を肉の上に散らす。量は適宜加減すること。レモン果汁
  \(\frac{1}{2}\)個分に対して油大さじ1杯の割合で、上からかけてやる。
\item
  仔牛、ジビエのフィレ肉、ハム、豚背脂などを細かく切ったもの\footnote{原文
    lardon
    (ラルドン)、通常は拍子木状に切ったものを言うが、ここではファルスとして後で細かく挽くことになるので、形状はあまり問題にならない。}の場合\ldots{}\ldots{}塩こしょうしてから、白ワイン3、コニャック3、油1の割合のマリナードを上からかけてやる。
\end{enumerate}

ここで用いた風味付けの材料は、後でファルスにする際に加えることになる。

いずれの場合でも、マリナードに浸した肉を小まめに裏返してやり、マリナードがよく浸み込むようにしてやること。

\hypertarget{marinade-crue-pour-viandes-de-boucherie-ou-venaison}{%
\subsubsection{牛、羊肉および大型ジビエ用の非加熱マリナード}\label{marinade-crue-pour-viandes-de-boucherie-ou-venaison}}

\frsub{Marinade crue pour viandes de boucherie ou venaison}

\index{marinade@marinade!marinade crue viande boucherie venaison@marinade crue pour viande de boucherie ou venaison}
\index{まりなーと@マリナード!うしひつしおおかたしひえようひかねつ@牛、羊肉および大型ジビエ用非加熱---}

(仕上がり2 L分)

\begin{itemize}
\item
  香味素材\ldots{}\ldots{}にんじん100 g、玉ねぎ100 g、エシャロット40
  g、セロリ30 g、にんにく2片、パセリの枝3本、タイム1枝、ローリエの葉
  \(\frac{1}{2}\)枚、大粒のこしょう6個、クローブ2本。
\item
  使用する液体\ldots{}\ldots{}白ワイン1 \(\frac{1}{4}\) L、ヴィネガー5
  dL、油2 \(\frac{1}{2}\) dL。
\item
  作業手順\ldots{}\ldots{}マリネする素材に塩とこしょうを振る。にんじん、玉ねぎ、エシャロットを薄切り\footnote{émincer
    (エマンセ)薄切りにする、スライスする。}にし、半量を容器の底に敷く。容器の大きさは素材とマリナードがぴったり入る程度のものを用いること。素材を入れて、残りの香味野菜で蓋をするようにして、白ワインとヴィネガー、油を注ぎ入れる。
\end{itemize}

冷蔵し、マリネ液に漬かった素材を小まめに裏返してやること。

\hypertarget{marinade-cuite-pour-viandes-de-boucherie-ou-venaison}{%
\subsubsection{牛、羊肉および大型ジビエ用の加熱マリナード}\label{marinade-cuite-pour-viandes-de-boucherie-ou-venaison}}

\frsub{Marinade cuite pour viandes de boucherie ou venaison}

\index{marinade@marinade!marinade cuite viande boucherie venaison@marinade cuite pour viande de boucherie ou venaison}
\index{まりなーと@マリナード!うしひつしおおかたしひえようかねつ@牛、羊肉および大型ジビエ用加熱---}

(仕上がり2 L分)

\begin{itemize}
\item
  香味素材\ldots{}\ldots{}非加熱マリナードと同じ材料で同じ分量
\item
  使用する液体\ldots{}\ldots{}白ワイン1 \(\frac{1}{2}\) L、ヴィネガー3
  dL、油2 \(\frac{1}{2}\) dL。
\item
  作業手順\ldots{}\ldots{}鍋に油を熱し、ごく薄くスライスしたにんじん、玉ねぎ、エシャロットおよびその他の香味素材を軽く色付くまで炒める。

  白ワインとヴィネガーを注ぎ、弱火で約30分間火を通す。

  必ず、マリナードが完全に冷めてからマリネする素材にかけること。
\end{itemize}

\hypertarget{marinade-crue-ou-cuite-pour-grosse-venaison}{%
\subsubsection[とりわけ大型のジビエ用、非加熱および加熱マリナード]{\texorpdfstring{とりわけ大型のジビエ\footnote{具体的には赤鹿
  cerf(セール)
  や猪、トナカイの成獣など。ニホンジカやエゾジカはcerfに分類されるので、これを参考にするといいだろう。}用、非加熱および加熱マリナード}{とりわけ大型のジビエ用、非加熱および加熱マリナード}}\label{marinade-crue-ou-cuite-pour-grosse-venaison}}

\frsub{Marinade crue ou cuite pour grosse venaison}

\index{marinade@marinade!marinade crue cuite grosse venaison@marinade crue ou cuite pour grosse venaison}
\index{まりなーと@マリナード!とりわけおおかたのしひえようひかねつおよひかねつ@とりわけ大型のジビエ用非加熱および加熱---}

(仕上がり2 L分)

\begin{itemize}
\item
  香味素材\ldots{}\ldots{}牛、羊肉および大型ジビエ用のマリナードと同じだが、ローズマリー12
  gを追加する。
\item
  使用する液体\ldots{}\ldots{}ヴィネガー16 dL、油4 dL。
\item
  作業手順\ldots{}\ldots{}非加熱、加熱ともに作業手順は上記のレシピのとおり。
\end{itemize}

\hypertarget{marinade-cuite-pour-le-mouton-en-chevreuil}{%
\subsubsection[羊のシュヴルイユ仕立て用の加熱マリナード]{\texorpdfstring{羊のシュヴルイユ仕立て用の加熱マリナード\footnote{\protect\hyperlink{sauce-chevreuil}{ソース・シュヴルイユ}参照。}}{羊のシュヴルイユ仕立て用の加熱マリナード}}\label{marinade-cuite-pour-le-mouton-en-chevreuil}}

\frsub{Marinade cuite pour le mouton en chevreuil}

\index{marinade@marinade!marinade cuite mouton en chevreuil@marinade cuite pour le mouton en chevreuil}
\index{まりなーと@マリナード!ひつしのしゆうるいゆしたてようのかねつまりなーと@羊のシュヴルイユ仕立て用加熱---}

(仕上がり2 L分)

\begin{itemize}
\item
  香味素材\ldots{}\ldots{}上記のとおりの分量の素材に、ジュニパーベリー\footnote{セイヨウネズの実。ジンの香り付けに用いられている。}10粒とバジル1つまみ、ローズマリー1つまみを足す。
\item
  使用する液体\ldots{}\ldots{}牛、羊および大型ジビエ用の加熱マリナードと同じ。
\item
  作業手順\ldots{}\ldots{}鍋に油を熱し、薄切りにしたにんじん、玉ねぎ、エシャロットおよびその他の香味素材を軽く色付くまで炒める。

  白ワインとヴィネガーを注ぎ、弱火で約30分間火を通す。
\end{itemize}

\hypertarget{marinade-cuite-pour-le-mouton-en-chamois}{%
\subsubsection[羊のシャモワ仕立て用の加熱マリナード]{\texorpdfstring{羊のシャモワ仕立て\footnote{オートザルプ県の山岳地帯およびピレネー山脈に生息する野生の山羊。ピレネー山脈のものは
  Isard
  (イザール)と呼ばれる。若い獣の肉は大型ジビエのなかでもとりわけ美味とされる。成獣の肉は固く、しっかりマリネする必要があると言われている。しばしばノロ鹿と比較される。ここでは、羊肉を白ワインベースのマリナードに漬け込む仕立て、すなわちシュヴルイユ仕立てとの対比として、赤ワインでより強い風味のマリナードに漬け込むことで、シャモワ仕立てとしている。なお、本書にシャモワ仕立てのレシピは掲載されていない。シュヴルイユ仕立てと同様と考えていい。}用の加熱マリナード}{羊のシャモワ仕立て用の加熱マリナード}}\label{marinade-cuite-pour-le-mouton-en-chamois}}

\frsub{Marinade cuite pour le mouton en chamois}

\index{marinade@marinade!marinade cuite mouton en chevreuil@marinade cuite pour le mouton en chevreuil}
\index{まりなーと@マリナード!ひつしのしやもわしたてようのかねつまりなーと@羊のシャモワ仕立て用加熱---}

(仕上がり2 L分)

\begin{itemize}
\item
  香味素材\ldots{}\ldots{}非加熱マリナードと同じ分量の素材に、ジュニパーベリー\footnote{セイヨウネズの実。ジンの香り付けに用いられている。}15粒とバジル15
  g、ローズマリー15 gを足す。
\item
  使用する液体\ldots{}\ldots{}良質な赤ワイン1 \(\frac{1}{2}\)
  L、ヴィネガー3 dL、油2 \(\frac{1}{2}\) dL。
\item
  作業手順\ldots{}\ldots{}上記と同じ。

  このマリナードに上等な赤ワインを使える場合には、素材の量を次のように調整すること。赤ワイン12
  dL、ワインヴィネガー6 dL、油は上記の分量とする。

  ワインの酸味の強さによっては、ヴィネガーの量をワインと同量にすることさえ可能。
\end{itemize}

\hypertarget{observation-sur-les-marinades}{%
\subparagraph{マリナードについての注意事項}\label{observation-sur-les-marinades}}

\ldots{}\ldots{} 1.
加熱マリナードを使用するのは、素材へのマリナードの浸透作用を促進するのが目的。素材をマリナードに漬け込む時間は、加熱、非加熱ともに、素材の種類と大きさ、気温、環境の変化を勘案して決めること。

\begin{enumerate}
\def\labelenumi{\arabic{enumi}.}
\setcounter{enumi}{1}
\tightlist
\item
  一般的な牛、羊肉と肉質の柔らかい大型ジビエに使うマリナードに純粋な酢酸を用いるのは絶対にやめておくこと。酢酸の腐食作用によって肉の風味が失なわれてしまうからだ\footnote{この注記は第二版から。内容が当時の知見にもとづいたものであることに注意。ただし、19世紀には木酢液を原料として工業用の氷酢酸が既に製造されていた。また、タンパク質はpHの変化によって分解されるので、マリナードにヴィネガーを加えるのは理にかなっている。なお、肉を柔らかくする効果のあるタンパク質分解酵素(プロテアーゼ)の代表的なひとつであるパパインの発見は1940年代になってからのこと。パイナップルに含まれているブロメラインの効果は経験的に知られていた可能性もあるが、この酵素が60℃で不活性化することが広く知られるようになったのは、少なくとも日本では比較的近年のことに過ぎない。}。猪、赤鹿\footnote{cerf
    (セール)、ニホンジカやエゾジカもフランス語で表現するとこれに含まれるので、これらの料理について
    chevreuil (しゅう゛るいゆ)ノロ鹿の名をつけるは、厳密には誤り。}、トナカイなどの固い肉についても、純粋な酢酸だけを使うのは不可。
\end{enumerate}

\hypertarget{conservation-des-marinades}{%
\subsubsection{マリナードの保存方法}\label{conservation-des-marinades}}

\frsub{Conservation des marinades}

\index{marinade@marinade!conservation marinades@conservation des marinades}
\index{まりなーと@マリナード!ほそんほうほう@---の保存方法}

マリナードを長期間保存しておく必要がある場合には、とりわけ夏場は、本書で示した分量に対して2〜3
gのホウ酸を加えるといい。

さらに、夏のあいだは2日に一度、冬季は4〜5日に一度、マリナードを沸騰させ、冷めたら毎回そのマリナードに使っているのと同じワインを
2 dLとヴィネガー1 dLを足してやること。
\end{recette}
\hypertarget{saumures}{%
\subsection[ソミュール]{\texorpdfstring{ソミュール\footnote{この見出しは第四版のみ。初版〜第三版にかけては、マリナードとソミュールのレシピの間に区切りをつけるものは何も挿入されていない。}}{ソミュール}}\label{saumures}}

\frsecb{Saumures}

\index{saumure@saumure} \index{そみゆーる@ソミュール}
\begin{recette}
\hypertarget{saumure-au-sel}{%
\subsubsection{塩漬け用ソミュール}\label{saumure-au-sel}}

\frsub{Saumure au sel}

\index{saumure@saumure!sel@--- au sel}
\index{そみゆーる@ソミュール!しおつけよう@塩漬け用---}

このソミュールは、グレーソルト\footnote{フランス語は sel gris
  (セルグリ)または gros gris (グログリ)。灰色がかった粗塩。}1
kgに対して硝石\footnote{原文 salpêtre
  (サルペートル)硝酸カリウム。殺菌作用と、肉類を赤く発色させる効果を持つ。現代の日本では亜硝酸カリウム、亜硝酸ナトリウムが使われることが多い。いずれも日本では劇物指定されているが、シャルキュトリ(豚肉加工品の製造)においては不可欠とも言われるな薬品であり、とりわけボツリヌス菌対策の効果が大きい。そのため劇物ではあるが、食品添加物として認められており、使用限界量が厳密に定められている(食品添加物は国あるいは地域によって扱いが異なるので注意)。硝酸塩あるいは亜硝酸塩による肉の赤い発色を「着色料によるもの」と誤認する消費者は少なくない。これはかつて「魚肉ソーセージ」がコチニール色素でピンク色に染められていたことから連想される誤認と思われる。また、食品添加物イコール毒という安直な考えから忌避する消費者も少なくないのは事実だろう。こうしたことから、現代日本のレストランでは、製造後すぐに提供可能であるために、これら硝酸塩、亜硝酸塩の類を用いないところもある。}40
gの割合で作る。この硝石入りの塩の総量は、塩漬けにする肉の数と大きさで決まる。素材が完全に覆えて、重しが出来る分量とすること。

\begin{itemize}
\tightlist
\item
  作業手順\ldots{}\ldots{}肉を塩漬けにする前にまず、太い針を充分深く刺して穴を何箇所も空ける。次に硝石の粉末を肉の表面にすり付ける。塩1
  kgあたりタイム1枝、ローリエの葉
  \(\frac{1}{2}\)枚を加えて肉と塩を容器に詰める。
\end{itemize}

\hypertarget{saumure-liquide-pour-langues}{%
\subsubsection[舌肉用の液体ソミュール]{\texorpdfstring{舌肉用の液体ソミュール\footnote{このソミュールに舌肉を漬け込むと、硝石の作用で舌肉が赤く発色する。それを拍子木状などに切って鶏やフィレ肉の表面に、同様に切ったトリュフや豚背脂などとともに刺して装飾することが19世紀〜20世紀初頭までよく行なわれた。現代ではほとんど行なわれなくなった装飾方法。この場合はあくまでも料理の装飾を目的としたものであり、牛や豚の舌肉を保存食として利用する場合には塩漬けや燻製などの方法も用いられる。}}{舌肉用の液体ソミュール}}\label{saumure-liquide-pour-langues}}

\frsub{Saumure liquide pour langues}

\index{saumure@saumure!liquide langues@--- liquide pour langues}
\index{そみゆーる@ソミュール!したにくようのえきたい@舌肉用の液体---}

\begin{itemize}
\item
  材料\ldots{}\ldots{}水5 L、グレーソルト2.25 kg、硝石150
  g、茶色いカソナード\footnote{砂糖きびを原料とした粗糖。通常は茶褐色のものが多く「赤糖」とも呼ばれるが、精白したものもある。精製が不完全であるため独特の風味があり、料理および製菓でしばしば用いられる。}
  300 g、こしょう12 g、ジュニパーベリー12粒、タイム1枝、ローリエの葉1
  枚。
\item
  作業手順\ldots{}\ldots{}充分な大きさの鍋に材料を全て入れ、強火で沸騰させる。その後、完全に冷めてから、針で穴を複数空けて硝石をしっかりすり込んだ舌肉を入れた容器に注ぎ込む。平均的な重さの舌肉を漬け込む期間は冬季で8日間、夏季は6日間。
\end{itemize}

\hypertarget{grande-saumure}{%
\subsubsection[グランドソミュール]{\texorpdfstring{グランドソミュール\footnote{この項は第二版で追加された。通常はシャルキュティエすなわちシャルキュトリ専門の職人が行なう規模のものであり、料理人の仕事の範疇をやや越えるとも考えられる。}}{グランドソミュール}}\label{grande-saumure}}

\frsub{Grande saumure}

\index{saumure@saumure!grande@grande ---}
\index{そみゆーる@ソミュール!くらんと@グランド---}

(仕上がり50 L分)

\begin{itemize}
\tightlist
\item
  水\ldots{}\ldots{}50 L
\item
  塩\ldots{}\ldots{}25 kg
\item
  硝石\ldots{}\ldots{}2.7 kg
\item
  カソナード\ldots{}\ldots{}1.6 kg
\item
  作業手順\ldots{}\ldots{}メッキされた銅の鍋に材料を全て入れ、強火にかける。沸騰したら、皮を剥いたじゃがいも1個を投入する。じゃがいもが浮いてくるようであれば、じゃがいもが沈みはじめる寸前まで水を足す。逆に、じゃがいもが完全に底まで沈んでしまうようなら、じゃがいもが水面に見えてくるまで煮詰める必要がある。
\end{itemize}

ソミュールがちょうどいい具合になったら、鍋を火から外して、このソミュールで漬け込み槽に注ぎ込む。漬け込み槽の素材は、スレート製、岩製、セメント製、あるいはレンガ製でしっかりエナメル引きしたものを用いること。

漬け込み槽の底に、木製の網を敷き、その上に漬け込む肉を置くといい。肉が槽の底面に直接当たっていると、肉の下側にソミュール液が浸透しない可能性がある。

漬け込む肉は、たとえ小さなものであっても、専用の携行可能な注入器具を使ってソミュールを内部に注入してから、漬け込み槽に入れてやること。この準備作業を怠ると、肉全体が均等に塩漬けにならない可能性がある。肉の中心部がちょうどいい塩加減になる頃には外側は塩が強すぎるということになってしまうのだ。牛のランプ、イチボなどの塊肉で、4〜5
kgの大きさの場合は、ソミュール液を注入してやる方法を使えば8日間で漬かる。

牛舌肉をこの方法で漬ける場合は、出来るだけ新鮮なものを用いる必要がある。軟骨部分をきれいに取り除いてやり、肉叩きか麺棒で丁寧に叩いてやる。ブリデ針\footnote{主として鶏などの手羽や腿をまとめて整形し、その形状を保つよう糸で縫う際に用いる縫い針。}を使って、表面全体に刺し穴をつけてやる。それからソミュールに漬け込むが、何らかの重しをして浮き上がらないようにしてやること。

\hypertarget{observation-grande-saumure}{%
\subparagraph{【原注】}\label{observation-grande-saumure}}

ソミュールはマリナードほどは腐敗しにくいとはいえ、天候が悪い時季などはとりわけ、よく様子を見て、時々は沸騰させてやるのがいい。沸騰させれば多少は濃縮されてしまうから、本文記載の方法でじゃがいもを用いて、毎回少量の水を加える必要がある。
\end{recette}
\hypertarget{gelees-diverses}{%
\section{ジュレ}\label{gelees-diverses}}

\frsec{Gelées diverses}

\index{gelee@gelée} \index{しゆれ@ジュレ}

どんなジュレも、ベースとなっているのはほぼ全てフォンだ。だから、フォンのメインとなっている素材によってジュレの風味が決まるわけだ。その結果としてジュレの用途も\ruby{自}{おの}ずと決まってくる。

人工的な凝固剤を使わずにジュレを確実に固めるためには、フォンのメインとなる素材に、仔牛の足や豚皮のようなゼラチン質の量を計算して加えることになる。仔牛の足や豚の皮を使えば、ジュレを確実に凝固させられるし、しかも柔らかな口あたりに仕上げられる。

そうはいっても、とりわけ夏季には、クラリフィエ\footnote{clarifier
  \textgreater{} clarification 次項参照。}の作業を行なう前に必ず、フォンを氷の上に垂らしてみて、固さと濃度を確認し、必要があれば板ゼラチンを何枚か加えてやること。

追加する板ゼラチンの量は、どんな場合でも、フォン1 Lあたり9
g(6枚)を越えないこと。板ゼラチンは、透き通っていてぱりぱりと割れやすく、
\ruby{膠}{にかわ}っぽい味のしないものを選ぶこと。必ず冷水でもどしてから使うか、せめてよく洗ってから用いること。

標準的なジュレを作る際に人工着色料を使うことはお勧め出来ない。標準的なジュレは充分に色よく仕上がるものだ。さらに、最後にマデイラ酒を加えてやれば充分に、標準的なジュレの特徴ともいえる淡い琥珀色に仕上がる。
\begin{recette}
\hypertarget{fonds-pour-gelee-ordinaire}{%
\subsubsection[標準的なジュレ用のフォン]{\texorpdfstring{標準的なジュレ用のフォン\footnote{この項および次の「白いジュレ用のフォン」は初版と第二版以降の異同が大きい。この「標準的なジュレ用のフォン」は初版では使用する液体が
  8 litres et demi de remouillage
  いわゆる「二番のフォン」であり、加熱時間も6時間と短かい。第二版は「水8.5
  L」になるが、加熱時間は6時間のままで、作業手順が「ソース用の白いフォンと同じ」となっている。第三版で現在の記述となった。}}{標準的なジュレ用のフォン}}\label{fonds-pour-gelee-ordinaire}}

\frsub{Fonds pour gelée ordinaire}

\index{gelee@gelée!fonds ordinaire@Fonds pour gelée ordinaire}
\index{しゆれ@ジュレ!ひようひゆんてきなしゆれようのふおん@標準的なジュレ用のフォン}

(仕上がり5 L分)

\begin{itemize}
\item
  主素材\ldots{}\ldots{}仔牛のすね肉とバモルソー\footnote{原文 bas
    morceaux 煮込みなどに用いる部位の総称。bas
    は「低い」が原義であり、食材として低級な部位というニュアンス。}2
  kg、細かく砕いた仔牛の骨1.5 kg、牛の脚肉1.5
  kg\ldots{}\ldots{}これらの肉と骨はオーブンで軽く色付けておくこと。
\item
  ゼラチン質\ldots{}\ldots{}骨を取り除いて\footnote{原文 désosser
    (デゾセ)骨を取り除く。}下茹でした\footnote{原文 blanchir
    (ブランシール)。下茹ですることがだ、原義は「白くする」。もとは中世において肉を調理する際にはローストであれ煮込みであれ、ほぼ必ず下茹でしていた。赤い肉を茹でると表面が白くなることからこの用語が定着することになったが、現在ではもっぱら野菜の下茹でなどについて言うことがほとんど。「ブランシェ」と言う現場もあるようだが、もとのフランス語からやや離れているので「ブランシール」で覚えるといいだろう。}仔牛の足3本、背脂を付けたままの生の豚皮\footnote{塩漬けなどの加工をしていない、ということ。}250
  g。
\item
  香味素材\ldots{}\ldots{}にんじん200 g、玉ねぎ200 g、ポワロー\footnote{poireau(x)
    ポロねぎ。日本の長葱とは異なり、植物としてはむしろ、にんにくに近いが、風味はかなり異なる。古代ローマ時代からヨーロッパで広く親しまれてきた野菜のひとつ。ローマ皇帝ネロが演説で大きな声を出すために、ポワローの蜂蜜漬けを好んだという逸話がある。伝統的な栽培方法の場合、旬は秋〜冬。播種から収穫まで10ヶ月以上かかる品種も多い。太さ3〜5
    cm、軟白部が20〜40
    cmくらいのものが多い。フランスの標準的な規格では軟白部20
    cm以上。かつては日本の長葱と同様に成長に応じて「土寄せ」して栽培していたが、その方法では内部に土砂が入りやすい。また、太さ1
    cm程のミニ・ポワローも付け合わせ用の高級野菜として人気がある。元来ミニ・ポワローは苗の「間引き」を利用したものだったが、現在ではミニ・ポワローむけの品種も開発されている。いずれもヨーロッパでは大型機械を用いた大量生産が一般的。日本にも秋〜冬季はヨーロッパ産が、春〜夏季はオーストラリア産が安定的に輸入されている。日本国内での生産も明治以降、試みられてはいるが、需給バランスとコスト的に見合わないために断念せざるを得ないケースも少なくないようだ。なお、第二次大戦前は八丈島などでこうした西洋野菜の栽培が行なわれ、船便で東京まで運ばれていたという(cf.~大木健二『大木健二の洋菜ものがたり』日本デシマル、1997年)。なお、現代フランス語でブレット(ふだんそう)のことを
    poirée (ポワレ)とも呼ぶが、これは ポワロー poireau
    と同語源。中世の料理書にはしばしば、野菜をペースト状になるまで煮込んだポタージュとして
    porée
    (ポレ)というものが出てくるが、どちらを材料として用いているか判別できないケースもある。}50
  g、セロリ 50 g、充分な香りと量のブーケガルニ。
\item
  使用する液体\ldots{}\ldots{}水 8.5 L。
\item
  加熱時間\ldots{}\ldots{}6時間。
\item
  作業手順\ldots{}\ldots{}ソース用の\protect\hyperlink{fonds-brun}{茶色いフォン}とまったく同じ。ただし、ジュレ用のフォンの色合いはソース用のフォンよりも薄くしておくこと。
\end{itemize}

\hypertarget{fonds-pour-gelee-blanche}{%
\subsubsection[白いジュレ用のフォン]{\texorpdfstring{白いジュレ用のフォン\footnote{初版全文は「主素材、ゼラチン質、香味素材は上記のとおり。注ぐ液体は水(原文
  mouillage à
  blanc)、作業手順は基本の白いフォンと同様」。第二版で現在の記述となっている。この文脈からすると、白いフォンの二番を使うとも解釈され得るが、前項の「標準的なジュレ用のフォン」が最終的に水を用いて作ることになっているのと比較すると、加熱時間および作業手順が何と同様なのか曖昧になってしまうため、ここでは液体、加熱時間、作業手順を\protect\hyperlink{fonds-blanc}{標準的な白いフォン}と同じと解釈した。なお、英訳第5版では、but
  use very white stock instead of
  water「水ではなく白いフォン」を注ぐとなっている。}}{白いジュレ用のフォン}}\label{fonds-pour-gelee-blanche}}

\frsub{Fonds pour gelée blanche}

\index{gelee@gelée!fonds ordinaire@Fonds pour gelée ordinaire}
\index{しゆれ@ジュレ!しろいしゆれようのふおん@白いジュレ用のフォン}

主素材、ゼラチン質、香味素材の種類と分量は前記の\protect\hyperlink{fonds-pour-gelee-ordinaire}{標準的なジュレ用のフォン}を参照。

使用する液体の量は\protect\hyperlink{fonds-blanc}{標準的な白いフォン}とまったく同じにすること。

加熱時間も作業手順も同様。

\hypertarget{fonds-pour-gelee-de-volaille}{%
\subsubsection{鶏のジュレ用のフォン}\label{fonds-pour-gelee-de-volaille}}

\frsub{Fonds pour gelée de volaille}

\index{gelee@gelée!fonds volaille@Fonds pour gelée de volaille}
\index{しゆれ@ジュレ!とりのしゆれようのふおん@鶏のジュレ用のフォン}
\index{しゆれ@ジュレ!うおらいゆのしゆれようのふおん@ヴォライユのジュレ用のフォン ⇒ 鶏のジュレ用のフォン}
\index{とり@鶏!しゆれ@ジュレ!ふおん@---のジュレ用のフォン}
\index{うおらいゆ@ヴォライユ!しゆれ@ジュレ!うおらいゆのふおん@ヴォライユのジュレ用のフォン ⇒ 鶏のジュレ用のフォン}

(仕上がり5 L分)

\begin{itemize}
\item
  主素材\ldots{}\ldots{}仔牛のすね肉1.5 kg、牛の脚肉1.5
  kg、細かく砕いた仔牛の骨1.5
  kg、鶏ガラ、とさか、手羽先、足など(とりわけ湯通しした手羽と足)、1.5
  kg。
\item
  ゼラチン質\ldots{}\ldots{}骨を取り除いて下茹でした仔牛の足(小)3本。
\item
  香味素材\ldots{}\ldots{}材料の種類は標準的なジュレ用のフォンと同じだが、量はやや少なめにすること。
\item
  使用する液体\ldots{}\ldots{}軽く仕上げた\protect\hyperlink{fonds-blanc}{白いフォン}
  8 L。
\item
  加熱時間\ldots{}\ldots{}4時間半。
\item
  作業手順\ldots{}\ldots{}ソース用の\protect\hyperlink{fonds-de-volaille}{鶏のフォン}とまったく同じ。
\end{itemize}

\hypertarget{fonds-pour-gelee-de-gibier}{%
\subsubsection{ジビエのジュレ用のフォン}\label{fonds-pour-gelee-de-gibier}}

\frsub{Fonds pour gelée de gibier}

\index{gelee@gelée!fonds gibier@Fonds pour gelée de gibier}
\index{gibierl@gibier!gelee@gelée!fonds@Fonds pour gelée de ---}
\index{しゆれ@ジュレ!しひえのしゆれようのふおん@ジビエのジュレ用のフォン}
\index{しひえ@ジビエ!しゆれ@ジュレ!ふおん@---のジュレ用のフォン}

(仕上がり5 L分)

\begin{itemize}
\item
  主素材\ldots{}\ldots{}仔牛のすね肉1 kg、牛の脚肉2 kg、仔牛の骨750
  g、ジビエのガラやバモルソー\footnote{\protect\hyperlink{fonds-pour-gelee-ordinaire}{標準的なジュレ用のフォン}訳注参照。}1.75
  kg。これらはすべてオーブンで焼いて色付けておくこと。
\item
  ゼラチン質\ldots{}\ldots{}\protect\hyperlink{fonds-pour-gelee-de-volaille}{鶏のジュレ用のフォン}と同じ。
\item
  香味素材\ldots{}\ldots{}材料の種類は標準的なジュレ用のフォンと同じだが、セロリとタイムを
  \(\frac{1}{3}\)量多くすること。ジュニパーベリー\footnote{セイヨウネズの実。ジンの香りを特徴付けているもの。}7〜8粒を追加すること。
\item
  使用する液体\ldots{}\ldots{}水8 L。
\item
  加熱時間\ldots{}\ldots{}4時間。
\item
  作業手順\ldots{}\ldots{}ソース用の\protect\hyperlink{fonds-de-gibier}{ジビエのフォン}とまったく同じ。
\end{itemize}

\hypertarget{fonds-de-poisson-pour-gelee-ordinaire}{%
\subsubsection{標準的なジュレ用の魚のフォン}\label{fonds-de-poisson-pour-gelee-ordinaire}}

\frsub{Fonds de poisson pour gelée ordinaire}

\index{gelee@gelée!fonds poisson@Fonds de poisson pour gelée ordinaire}
\index{poisson@poisson!gelee@gelée!fonds@Fonds de --- pour gelée ordinaire}
\index{しゆれ@ジュレ!ひようしゆんてきなしゆれようのさかなのふおん@標準的なジュレ用の魚のフォン}
\index{さかな@魚!しゆれ@ジュレ!ふおん@標準的なのジュレ用の---のフォン}

(仕上がり5 L分)

\begin{itemize}
\item
  主素材\ldots{}\ldots{}グロンダン\footnote{ホウボウ科の魚。和名カナガシラ。}、ヴィーヴ\footnote{ハチミシカ科の海水魚の総称。}、メルラン\footnote{鱈の近縁種。}などの安い魚750
  g、舌びらめのアラと端肉750 g。
\item
  香味素材\ldots{}\ldots{}薄切りにした\footnote{émincer エマンセ。}玉ねぎ200
  g、パセリの根2本、フレッシュなマッシュルームの切りくず100 g。
\item
  使用する液体\ldots{}\ldots{}やや薄めで透き通った仕上がりの魚のフュメ6
  L。
\item
  加熱時間\ldots{}\ldots{}45分間。
\item
  作業手順\ldots{}\ldots{}\protect\hyperlink{fumet-de-poisson}{魚のフュメ}と同じ。
\end{itemize}

\hypertarget{fonds-pour-gelee-de-poisson-au-vin-rouge}{%
\subsubsection{赤ワインを用いた魚のジュレ用のフォン}\label{fonds-pour-gelee-de-poisson-au-vin-rouge}}

\frsub{Fonds pour gelée de poisson au vin rouge}

\index{gelee@gelée!fonds poisson rouge@Fonds pour gelée de poisson au vin rouge}
\index{poisson@poisson!gelee@gelée!fonds rouge@Fonds pour gelée de poisson au vin rouge}
\index{しゆれ@ジュレ!あかわいんをもちいたさかなのしゆれようのふおん@赤ワインを用いた魚のジュレ用のフォン}
\index{さかな@魚!しゆれ@ジュレ!ふおんあかわいん@赤ワインを用いた---のジュレ用のフォン}

このフォンは通常、鯉やトラウトなどの魚料理に用いられる。

このフォンに使用する液体は、良質なブルゴーニュ産赤ワインと\protect\hyperlink{fumet-de-poisson}{魚のフュメ}を同量ずつにする。魚のフュメは、ジュレが確実に固まるよう、ゼラチン質が多めのものを用いること。

風味付けは、魚に火を通すのに使った香味野菜によるもので充分だ。

\hypertarget{observation-sur-l-emplois-des-fonds-destines-aux-gelees}{%
\paragraph{ジュレ用のフォンについての注意}\label{observation-sur-l-emplois-des-fonds-destines-aux-gelees}}

\ldots{}\ldots{}ジュレ用のフォンは出来るだけ、使用する前日に仕込んでおくこと。いい具合に煮込んだら、浮き脂を取り除き\footnote{dégraisser
  デグレセ。}、漉してから陶製の容器に入れて冷ます。

冷めるとフォンは凝固する。取り除ききれなかったごくわずかな脂が表面に浮いてくるが、板状に固まるので容易に取り除くことが出来る。布あるいは漉し器でフォンを漉した際にすり抜けてしまった堆積物も自重で容器の底に沈むので、フォンを完全に澄ませることが出来る。
\end{recette}
\newpage

\hypertarget{clariication-des-gelees}{%
\subsection[ジュレのクラリフィエ]{\texorpdfstring{ジュレのクラリフィエ\footnote{clarification
  (クラリフィカスィオン)澄ませること、透明にさせること、の意の名詞だが、(1)本文にあるように、ただ単に「澄ませる」だけではなく、風味を補ったり強化し、色合いを調節する作業も兼ねていること、(2)現代日本の調理現場ではフランス語の動詞
  clarifier
  をカタカナにして「クラリフィエ」と呼ぶケースが多いことなどを考慮して、カタカナで動詞形のクラリフィエとした。なお、「クラリフェ」と呼ぶ現場もあるようだが、もとのフランス語がclarif\textbf{i}erとiの音があるのでこれは許容しがたい。}}{ジュレのクラリフィエ}}\label{clariication-des-gelees}}

\frsecb{Clarification des Gelées}

\index{gelee@geléé!clarification@Clarification des ---s}
\index{clarification@clarification!gelee@--- des gelées}
\index{しゆれ@ジュレ!くらりふいえ@---のクラリフィエ}
\index{くらりふいえ@クラリフィエ!しゆれ@ジュレの---}
\begin{recette}
\hypertarget{gelees-ordinaires}{%
\subsubsection[標準的なジュレ]{\texorpdfstring{標準的なジュレ\footnote{このgrasse
  \textless{} gras
  は「脂気のある、太った」の意ではなく、カトリックにおける「小斉」の食事を
  maigre
  と表現することと対になっているもの。すなわち「小斉ではない通常の」の意であることに注意。小斉については\protect\hyperlink{sauce-espagnole-maigre}{魚料理用ソース・エスパニョル}訳注および\protect\hyperlink{sauce-laguipiere}{ソース・ラギピエール}訳注参照。}}{標準的なジュレ}}\label{gelees-ordinaires}}

\frsub{Gelées grasses ordinaires}

\index{gelee@geléé!clarification@Clarification!grasses ordinaire@---s grasses ordinaires}
\index{clarification@clarification!gelee@gelée!grasses ordinaire@---s grasses ordinaires}
\index{しゆれ@ジュレ!くらりふいえ@---のクラリフィエ!ひようしゆんてきな@標準的な---}
\index{くらりふいえ@クラリフィエ!しゆれ@ジュレの---!ひようしゆんてきな@標準的な---}

(仕上がり5 L分)

\begin{enumerate}
\def\labelenumi{\arabic{enumi}.}
\item
  まずフォンの濃度を確認する。必要に応じて追加すべきゼラチンの量を調整する。
\item
  ジュレ用のフォンは充分に浮き脂を取り除き\footnote{dégraisser
    デグレセ。}、沈殿物も取り除い\footnote{décanter デカンテ。}てあること。
\item
  厚手で適切な大きさの片手鍋\footnote{casserole カスロール。}に、細挽き\footnote{ミートチョッパーやフードプロセッサが一般化する以前はアショワールhachoirという、両側に柄の付いた刃が湾曲した専用の包丁で細かく刻んでいた。}にした脂身のない\footnote{ここで原文はmaigreを用いているが、これはもちろん「脂気のない」の意。}赤身の牛肉
  500 gとセルフイユとエストラゴン計10 g、卵白3個分を入れる。
\item
  冷たい、あるいは生温い状態のジュレ用のフォンを挽肉の上から入れ、泡立て器かヘラで混ぜる。\\
  ゆっくり混ぜながら、強過ぎない程度の火加減で沸騰させる。卵白に含まれるアルブミンの分子が澄ませる作用を持っているので\footnote{やや大雑把な説明になるが、液体中に浮遊している不純物を抱き込むかたちで卵白が熱変性により凝固する、その結果として液体を「澄ませる」ことになる。ただし、これだけだと液体の味そのものや風味が薄くなってしまうために、それを補うあるいは強化する意味で挽肉や香草、香り付けの酒類を加える、ということ。}、混ぜることで卵白がまんべんなく広がるようにするわけだ。\\
  15分程、微沸騰の状態を保ち、目の詰まった布で漉す。
\end{enumerate}

\hypertarget{nota-gelees-grasses-ordinaires}{%
\subparagraph{【原注】}\label{nota-gelees-grasses-ordinaires}}

ジュレに酒類を添加するのは、ほぼ冷めた状態になってからにするのがいい。クラリフィエの作業中に酒類を加えるのは、沸騰しているために味が悪くなってしまうので、致命的な誤りでさえある。

そうではなく、ほぼ冷めた状態のジュレに酒類を添加すれば、その香気はそのまま保たれることになる。

作業の最後に酒類をジュレに添加すればジュレを薄めてしまう結果になるわけだからそれを考慮して、添加する酒類の量によっては、あらかじめジュレを充分に固めに作っておくのがいい。そうすれば、ジュレが固まるのに充分なゼラチンの濃度を保てるわけだ。

マデイラ酒、マルサラ酒、シェリー酒を加える場合の分量はジュレ1 Lあたり1
dLとすること。

ライン産のワインやシャンパーニュ、銘醸白ワインを加える場合は、ジュレ1
Lあたり2
dLとすること。加える酒類がどんなものであっても、文句ない程に良質のものを用いるべきだ。質の悪い酒類を加えてジュレの仕上がりを台無しにしてしまうくらいなら、加えないほうがまだましと言える。

\hypertarget{gelee-de-volaille}{%
\subsubsection{鶏のジュレ}\label{gelee-de-volaille}}

\frsub{Gelée de volaille}

\index{gelee@geléé!clarification@Clarification!volaille@--- de volaille}
\index{clarification@clarification!gelee@gelée!volaille@--- de volaille}
\index{しゆれ@ジュレ!くらりふいえ@---のクラリフィエ!とりの@鶏の---}
\index{くらりふいえ@クラリフィエ!しゆれ@ジュレの---!とりの@鶏の---}
\index{しゆれ@ジュレ!くらりふいえ@---のクラリフィエ!うおらいゆの@ヴォライユの---}
\index{くらりふいえ@クラリフィエ!しゆれ@ジュレの---!うおらいゆの@ヴォライユの---}

鶏のジュレのクラリフィエは標準的なジュレの場合とまったく同じに行なう。香味素材(セルフイユとエストラゴン)、澄ませるための材料(卵白)も同様にする。

ただし、味の補強に用いる肉については変更すること。すなわち牛の赤身肉を半量にして、残り半量は鶏の首肉にする。つまり、牛肉250
gと鶏の首肉250 g の挽肉を用いる。

\hypertarget{nota-gelee-de-volaile}{%
\subparagraph{【原注】}\label{nota-gelee-de-volaile}}

鶏のローストのガラを粗く砕いてエチューヴ\footnote{食品の乾燥などに主に用いられる低温のオーブンの一種。}でよく乾燥させて脂気を抜いたものを、このクラリフィエの際に加えると、素晴しい結果が得られる。

\hypertarget{gelee-de-gibier}{%
\subsubsection{ジビエのジュレ}\label{gelee-de-gibier}}

\frsub{Gelée de gibier}

\index{gelee@geléé!clarification@Clarification!gibier@--- de gibier}
\index{clarification@clarification!gelee@gelée!gibier@--- de gibier}
\index{しゆれ@ジュレ!くらりふいえ@---のクラリフィエ!しひえ@ジビエの---}
\index{くらりふいえ@クラリフィエ!しゆれ@ジュレの---!しひえ@ジビエの---}

クラリフィエの作業のやり方はまったく同様。ただし、このジュレを作る際には、いくつか留意すべきポイントがある。

標準的なジビエのジュレ、つまり特有の風味を持たせないものの場合は、味の補強には牛の挽肉250
gとジビエの赤身の挽肉250 gを用いること。

ジュレに独特の香りを持たせる必要がある場合には、必ず、肉それ自体に香気のあるジビエの肉、すなわち、ペルドロー、雉、ジェリノット\footnote{gélinotte
  雷鳥の一種。}などをクラリフィエの際に用いること。

どんなジビエのジュレでも仕上げに、ジュレ1
Lあたり大さじ2杯の上等なコニャックを加える。ただし、コニャックは絶対に良質のものでなければいけない。平凡なコニャックしか使えないのなら、これは省いたほうがいい。

この香り付けをしなくても、ジュレは不完全なものとはいえ、一応使えるものになる。いっぽうで、ありきたりのコニャックで香り付けすると、美味しくは仕上がらない。

\hypertarget{gelee-de-poisson-blanche}{%
\subsubsection{魚の白いジュレ}\label{gelee-de-poisson-blanche}}

\frsub{Gelée de poisson blanche}

\index{gelee@geléé!clarification@Clarification!poisson blanche@--- de poisson blanche}
\index{clarification@clarification!gelee@gelée!poisson blanche@--- de poisson blanche}
\index{しゆれ@ジュレ!くらりふいえ@---のクラリフィエ!さかなのしろい@魚の白い---}
\index{くらりふいえ@クラリフィエ!しゆれ@ジュレの---!さかなのしろい@魚の白い---}

魚のジュレのクラリフィエは以下のとおり\footnote{(1)または(2)の方法をとる、と解釈していいだろう。}。

\begin{enumerate}
\def\labelenumi{\arabic{enumi}.}
\item
  卵白を使う場合、ジュレ5
  Lあたり卵白3個分に、クラリフィエによって薄まってしまうのを補うためにメルランの身を細かく刻んだもの250
  gを加える。
\item
  もし可能なら新鮮なキャビア、なければ圧縮キャビア\footnote{\protect\hyperlink{beurre-de-caviar}{キャビアバター}訳注参照。}をジュレ1
  Lあたり50
  g用いる。方法は魚のコンソメのクラリフィエで説明している\footnote{概要は、キャビアをピュレ状にすり潰し、冷たい魚のコンソメでのばして加える。火にかけて絶えず混ぜながら沸かし、微沸騰の状態を20分保った後、布で漉す、という方法。}
  (ポタージュの章を参照)。
\end{enumerate}

魚のジュレの香り付けには、辛口のシャンパーニュもしくはブルゴーニュの銘醸白ワインを用いるといいが、\protect\hyperlink{gelees-ordinaires}{標準的なジュレ}の注において説明した酒類を加える場合の注意事項を勘案すること。

\hypertarget{nota-gelee-de-poisson-blanche}{%
\subparagraph{【原注】}\label{nota-gelee-de-poisson-blanche}}

場合によっては、ジュレ1
Lあたり4尾のエクルヴィスを用いることで、魚のジュレに独特の風味付けをすることも出来る。エクルヴィスをソテーしてビスクを作る要領で煮てから、鉢に入れて細かくすり潰し、最後に漉す作業の10分前に魚のフォンに加える。

\hypertarget{gelee-de-poisson-au-vin-rouge}{%
\subsubsection{赤ワインを用いた魚のジュレ}\label{gelee-de-poisson-au-vin-rouge}}

\frsub{Gelée de poisson au vin rouge}

\index{gelee@geléé!clarification@Clarification!poisson vin rouge@--- de poisson au vin rouge}
\index{clarification@clarification!gelee@gelée!poisson vin rouge@--- de poisson au vin rouge}
\index{しゆれ@ジュレ!くらりふいえ@---のクラリフィエ!あかわいんをもちいたさかなの@赤ワインを用いた魚の---}
\index{くらりふいえ@クラリフィエ!しゆれ@ジュレの---!あかわいんをもちいたさかなの@赤ワインを用いた魚の---}

このジュレのクラリフィエには、ジュレ5 Lあたり卵白4個分を用いる。

赤ワインで魚を煮ている途中や、ジュレのクラリフィエ作業の際に、タンニン由来の色素にすぐ変化してしまうことがしばしば、というかほぼ必ず起こる。ワインが分解してしまうのは魚のフュメに含まれているゼラチン質と接触して反応するためのようだ。こんにちに至るまで、これを避ける方法は見つかっていない。

そのため、色合いの不足を補うには人工色素(液体のカルミン\footnote{コチニール色素。\protect\hyperlink{beurres-composes}{合わせバター}本文および訳注参照。}か別の植物由来の色素)を加える必要がある。ただし、使用量にはごく細心の注意を払い、ジュレがやや深みをおびたバラ色を越えてしまわないようにすること。
\end{recette}

%%%%I. Sauces 
\hypertarget{sauces}{%
\chapter{I. ソース Sauces}\label{sauces}}

\hypertarget{les-fonds-de-cuisine}{%
\section{フォン、その他のストック}\label{les-fonds-de-cuisine}}

\frsec{Les Fonds de Cuisine}

\index{fonds@fonds} \index{ふおん@フォン}

\normalsize
\setstretch{1.0}

本書は実際に厨房で働く料理人を対象としたものだが、まず最初に料理のベースとして仕込んでストックしておくもの\footnote{本書での
  fonds の語は fond (基礎、土台)、fonds
  (資産、資本)、そして料理用語として一般に用いられているフォン、のトリプルミーニングになっている。そのまま「フォン」と訳したいところだが、日本語の場合「出汁」としての意味合いが強いため、本文中では分りやすさを重視してやや冗長に「料理のベースとして仕込んでストックしておくもの」のように訳している。}について少々述べておきたい\footnote{この部分は経営者に向けて書かれているようにも読めるが、エスコフィエの時代以降、料理人がオーナーシェフとして経営に携わるケースが激増したことを考えると、その先見の明に驚かざるを得ない。}。我々料理人にとって重要なものだからだ。

ここで述べる料理のベースとして仕込んでストックしておくものは、実際、料理の土台そのものであり、それなしでは美味しい料理を作ることの出来ない、まず最初に必要なものだ。だからこそ、料理のベースとして仕込んでおくストックはとても重要であり、いい仕事をしたいと努めている料理人ほどこれらを重視している。

これらは、料理において常に立ち戻るべき出発点となるものだが、料理人がいい仕事をしたいと望んでも、才能があっても、それだけでいいものを作ることは出来ない。料理のベースを作るにも材料が必要なのだ。だから、必要な材料は良質のものを自由に使えるようにしなければならない。

筆者としては、むやみな贅沢には反対だが、それと同じくらい、食材コストを抑え過ぎるのも良くないと考えている。そんなことをしていては、伸びる筈の才能の芽を摘んでしまうばかりか、意識の高い料理人ならモチベーションの維持すら出来ないだろう。

どんなに優秀な料理人だって、無から何かを作り出すことは不可能だ。期待される結果に対して、素材の質が劣っていたり量が足りないことがあれば、それでも料理人にいい仕事をしろと要求するなど言語道断である。

料理のベースとして仕込んでおくストックに関するの重要ポイントは、必要な材料は質、量ともに充分に、惜しげもなく使えるようにすることだ。

ある調理現場で可能なことが、別の調理現場では不可能な場合があるのは言うまでもない。料理人の仕事内容は顧客層によっても変わる。到達すべき目標によって手段も変わるということだ。

そういう意味で、何事も相対的なものであるとはいえ、こと料理のベースとして仕込んでストックすべきものに関しては絶対に外してはならないポイントがあるわけだ。組織のトップがこの点で出費を惜しんだり、コスト面で過度に目くじらを立てるようでは、美味しい料理なんて出来るわけがないのだから、現実に厨房を仕切っている料理長を批判する資格もない。そんなのが根拠のない言い掛かりなのは明らかだ。素材の質が悪かったり、量が足りないのであれば、料理長が素晴しい料理を出せないのは言うまでもあるまい。ぶどうの搾りかすに水を加えて醗酵させた安ワインを立派な瓶に詰めてしまえば高級ワインになると思う程に馬鹿げたことはないのだ。

料理人は、必要なものを何でも使っていいなら、料理のベースとして仕込んでおくストックにとりわけ力を入れるべきであり、文句のつけようのない出来になるよう気を使うべきだ。そこに手間隙かけていればそれだけ厨房全体の仕事がきちんと進むのだから、注文を受けた料理をきちんと作れるかどうかは、結局のところ、料理のベースとなる仕込み類にどれだけ手間\ruby{隙}{ひま}をかけるかということなのだ。

\newpage

\hypertarget{principaux-fonds-de-cuisine}{%
\section{主要なフォンとストック}\label{principaux-fonds-de-cuisine}}

\frsec{Principaux Fonds de Cuisine}

料理のベースとして仕込んでおくべきものは主として\ldots{}\ldots{}

\begin{itemize}
\tightlist
\item
  \textbf{コンソメ・サンプルとコンソメ・ドゥーブル}
\item
  \textbf{茶色いフォン、白いフォン、鶏のフォン、ジビエのフォン、魚のフォン}\ldots{}\ldots{}これらはとろみを付けたジュ、基本ソースのベースになる
\item
  \textbf{フュメ、エッセンス}\ldots{}\ldots{}派生ソースに用いる
\item
  \textbf{グラスドヴィアンド、鶏のグラス、ジビエのグラス}
\item
  \textbf{茶色いルー、ブロンドのルー、白いルー}
\item
  \textbf{基本ソース}\ldots{}\ldots{}エスパニョル、ヴルテ、ベシャメル、トマト
\item
  \textbf{肉料理用ジュレ、魚料理用ジュレ}
\end{itemize}

\vspace{1\zw}

以下も日常的に使う料理のベースとして仕込んでおくものとして扱う。

\begin{itemize}
\tightlist
\item
  \textbf{ミルポワ、マティニョン}
\item
  \textbf{クールブイヨン、肉および野菜用のブラン}
\item
  \textbf{マリナード、ソミュール}
\item
  \textbf{肉料理用ファルス、魚料理用ファルス}
\item
  \textbf{ガルニチュールに用いるアパレイユ}、など\ldots{}\ldots{}
\end{itemize}

\vspace{1\zw}

本書は上記を順に説明していく構成にはなっていない。グリル、ロースト、グラタン等の調理技法についても順を追っていくわけではない。料理の種類ごとに一定の位置、つまりは関連の深い料理の章の冒頭において説明していくことになる。

\vspace{1\zw}

そのようなわけで、本書においては以下のようになる\ldots{}\ldots{}

\begin{itemize}
\tightlist
\item
  フォン、フュメ、エッセンス、グラス、マリナード、ジュレの説明\ldots{}\ldots{}
  \textbf{ 第1章 ソース}
\item
  コンソメおよびそのクラリフィエ、ポタージュの浮き実についての説明\ldots{}\ldots{}\textbf{第3章
  ポタージュ}
\item
  ファルスとガルニチュール用アパレイユの作り方\ldots{}\ldots{}\textbf{第2章
  ガルニチュール}
\item
  クールブイヨン、魚料理用ファルス等\ldots{}\ldots{}\textbf{第6章
  魚料理}
\item
  グリル、ブレゼ、ポワレの調理理論\ldots{}\ldots{}\textbf{第7章 肉料理}
\end{itemize}

\newpage

\hypertarget{section-grandes-sauces-de-base}{%
\section{基本ソース}\label{section-grandes-sauces-de-base}}

\frsec{Grandes Sauces de Base}

\index{そーす@ソース!きほん@基本---}
\index{sauce@sauce!00grandes@*Grandes ---s de Base}

\begin{itemize}
\item
  \textbf{およびそれらを組み合せたり煮詰めるなどの方法で作る派生ソース}
\item
  \textbf{イギリス風ソース(温製および冷製)}
\item
  \textbf{いろいろな冷製ソース}
\item
  \textbf{ブール・コンポゼ(ミックスバター)}
\item
  \textbf{マリナード}
\item
  \textbf{ジュレ}
\end{itemize}

\hypertarget{osbservation-sur-la-sauce}{%
\section{概説}\label{osbservation-sur-la-sauce}}

ソースは料理においてもっとも主要な位置にある。フランス料理が世界に冠たるものであるのもひとえにソースの存在によるのだ。だから、ソースは出来るかぎり手間をかけ、細心の注意を払って作るようにしなければならない。

ソースを作るうえでその基礎となるのが何らかの「ジュ」である\footnote{ここではジュといわゆるフォンが同じ意味で使われている。}。すなわち、茶色いソースは「茶色いジュ」(エストゥファード)から作る。ヴルテには「澄んだジュ(白いフォン\footnote{日本の調理現場で「白いフォン」を意味する「フォン・ブラン」は主として鶏のフォンを指すことが多いが、本書で扱われている白いフォンのうち標準的なものは仔牛肉、家禽類をベースとしており、鶏のフォンは別途説明されている。})を使う。ソースを担当する料理人はまず第一に、完璧なジュを作るところから始めなければならない。キュシー侯爵
\footnote{1767-1841。19世紀の著名な美食家。
  著書に『食卓の古典』(1843)がある。料理名にキュシーの名を冠したものも多い。}が言うように、ソース担当の料理人は「頭脳明晰な化学者\footnote{原文
  chimiste。現代は分子ガストロノミーが盛んだが、料理を作る過程で起きる現象や結果を「化学」で説明しようとする試みは少なくともカレームまで遡ることが出来る。\protect\hyperlink{fonds-brun}{茶色いフォン}のレシピにおいて言及されるオスマゾームという想像上の物質もその範疇に含まれるだろう。また、化学の前身たる「錬金術」的概念は中世以来いくつかの料理書において散見される。}でありかつ天才的なクリエイターで、卓越した料理という建造物のいわば大黒柱たる存在」なのだ。

昔のフランス料理\footnote{本書において「昔の料理」と表現される場合は概ね17〜18世紀末と考えていい。}では、素材に串を刺してあぶり焼きするローストを別にすれば、どんな料理も「ブレゼ」か「エチュヴェ」のようなものばかりだった。だが、その時代には既に、フォンが料理という大建築の丸天井の\ruby{要}{か
なめ}だったし、材料コストが重視されるこんにちの我々と比べたら想像も出来ないくらい贅沢に材料を使ってフォンをとっていたのだ。実際、アンヌ・ドートリッシュ\footnote{17世紀に絶対王政を確立したルイ14世の母。}がスペインからルイ13世に嫁いだ際に随行してきたスペインの料理人たちによってフランス料理にルーを用いる方法が伝えられたが\footnote{ルーがスペインからもたらされたというのは逸話、伝承の域を出ない。}、当時はほとんど看過された。ジュそれ自体で充分だったからだ。ところが時代が下り、料理におけるコストの問題が重視されるようになった。ジュはその結果、貧相なものになってしまった。その美味しさを補うものとして、ルーを用いて作るソース・エスパニョルが欠くべからざる存在となった。

ソース・エスパニョルはその完成度の高さゆえに成功をおさめたわけだ。だが、すぐに当初の目的を越えた使い方をされるようになった。19世紀末には本当にこのソースが必要な場合以外にも使われたわけだ。ソース・エスパニョルの濫用によって、どんな料理も固有の香りのない、全部の風味の混ざりあったのっぺりとした調子のものばかりになってしまった。

ようやく近年になって、料理の風味がどれも同じようなものであることに批判が集まってきて、その結果として激しい揺り戻しが起きたのだった。グランドキュイジーヌでは、透き通ったような薄い色合いでしかも風味のしっかりした仔牛のフォンが見直されつつある。そのようなわけで、ソース・エスパニョルそれ自体の重要性はだんだん減っていくだろうと思われる。

ソース・エスパニョルが基本ソースとして扱われるべき理由は何か? ソース・エスパニョルそれ自体に固有の色合いや風味というものはなく、これらはどんなフォンを用いて作るかで決まる。まさにこの点にソース・エスパニョルの長所が存するのだ。補助材料としてルーを加えるが、ルーにはとろみを付けるという意味しかなく、風味にはまったく寄与しない。そもそも、ソースを完璧に仕上げるためには、とろみ以外のルーに含まれる成分はソースからほぼ完全に取り除いてしまっても差し支えはない。不純物を丁寧に取り除いたソースにはルーに含まれていたでんぷん質だけが残っているわけだ。だから、ソースの口あたりを滑らかなものにするために必要なのがでんぷん質だけなら、純粋なでんぷんだけを用いる方がずっと簡単で、作業時間も大幅に短縮されるし、その結果として、ソースを火にかけ過ぎてしまうようなミスも防げる。将来的には、小麦粉ではなく純粋なでんぷんでルーを作るようになるかも知れない。

料理界の現状を\ruby{鑑}{かんが}みるに、\ul{ソース・エスパニョル}と
\ul{とろみを付けたジュ}をそれぞれ使い分けざるを得ない。これにはさまざまな理由があるが、大きな仕立てのブレゼや、羊や仔羊以外を材料にしたラグーでは、肉汁が煮汁に染み出してきて美味しくなるわけだから、トマトを加えたソース・エスパニョルを用いるのがいい。なお、ソース・エスパニョルをさらに丁寧に仕上げるとソース・ドゥミグラスとなる。これはいろいろなソテーに不可欠なもので、今後も変わることはないだろう。

一方、牛や羊、家禽を使った繊細で軽い仕立ての料理にはとろみを付けたジュの方が好まれる。デグラセの際に少量だけ、料理の主素材と同じものからとったジュを用いる。

こんにちのフランス料理においては、肉とソースの調和がとれているべきという、まことに理に適った厳守すべき決まりがある。

だから、ジビエ料理にはジビエのフォンを用いるか、とりたてて際立った個性を持たないフォンを用いて作ったソースを添える。牛や羊のフォンは用いない。ジビエのフォンというのは、さほど濃厚なものを作ることは出来ないが、素材の個性的な風味を表現するには最適だ。こういった事情は魚料理にも当て
\ruby{嵌}{はま}る。ソースそれ自体が際だった風味を持たないものの場合には必ず魚のフュメを加えてやるのだ。このようにしてそれぞれの料理に個性的な風味を実現させることになる。

もちろん、ここまで述べた原則を実現しようにも、コストの問題がしばしば起こることは承知している。けれども、熱意のある、他者の評価を意識している料理人なら問題点を熟考して、完璧とは言わぬまでも満足のいく結果を得ることが出来るだろう。\newpage

\normalsize
\setstretch{1.0}

\hypertarget{traitement-des-elements-de-base}{%
\section{ソースのベース作り}\label{traitement-des-elements-de-base}}

\frsec{Traitement des Éléments de Base dans le Travail des Sauces}

\index{そーす@ソース!そーすつくりのべーす@---のベース作り}
\index{sauce@sauce!Traitement des elements de base dans le travail des sauces@Traitement des Éléments de Base dans le Travail des ---s}
\begin{recette}
\hypertarget{fonds-brun}{%
\subsubsection{茶色いフォン(エストゥファード)}\label{fonds-brun}}

\frsub{Fonds brun ou Estouffade}

\index{ふおん@フォン!ちやいろいふおん@茶色い---}
\index{えすとうふあーと@エストゥファード}
\index{fonds@fonds!brun@--- brun}
\index{fonds@fonds!estouffade@estouffade (fonds brun)}
\index{estouffade@estouffade!fonds brun@ --- (fonds brun)}

(仕上がり10 L分)

\begin{itemize}
\item
  主素材\ldots{}\ldots{}牛すね6
  kg、仔牛のすね6kgまたは仔牛の端肉で脂身を含まないもの6
  kg、骨付きハムのすねの部分1本(前もって下茹でしておくこと)、塩漬けしていない豚皮を下茹でしたもの650
  g。
\item
  香味素材\ldots{}\ldots{}にんじん650 g、玉ねぎ650
  g、ブーケガルニ(パセリの枝100 g、タイム10 g、ローリエ5
  g、にんにく1片)。
\item
  作業手順\ldots{}\ldots{}肉を骨から外す。
\end{itemize}

骨は細かく砕き、オーブンに入れて軽く焼き色を付ける。野菜は焼き色が付くまで炒める。これらを鍋に入れて14
Lの水を注ぎ、ゆっくりと、最低12時間煮込む。水位が下がらぬように、適宜沸騰した湯を足すこと。

大きめのさいの目に切った牛すね肉を別鍋で焼き色が付くまで炒める。先に煮込んでいたフォンを少量加えて煮詰める。この作業を2〜3回行ない、フォンの残りを注ぐ。

鍋を沸騰させて、浮いてくる泡を取り除く。浮き脂も丁寧に取り除く。蓋をして弱火で完全に火が通るまで煮込んだら、布で漉してストックしておく。

\hypertarget{nota-fonds-brun}{%
\subparagraph{【原注】}\label{nota-fonds-brun}}

フォンの材料に牛の骨などが含まれている場合には、事前にその骨だけで12〜
15時間かけてとろ火でフォンをとるといい。

フォンの材料を鍋に焦げ付くくらいまで強く焼き色を付ける\footnote{パンセ
  pincer
  と呼ばれる手法。原義は「抓む」。材料が鍋底に張り付いて、トングなどでしっかり「抓ま」ないと取れないくらい強く焼き付けることからそう呼ばれるようになった。古い料理書では推奨するものも多かった。}のはよろしくない。経験からいって、丁度いい色合いのフォンに仕上げるには、肉に含まれているオスマゾーム\footnote{19世紀頃、赤身肉の美味しさの本質であると考えられていた想像上の物質。赤褐色をした窒素化合物の一種で水に溶ける性質があるとされた。なお、当時のヨーロッパではグルタミン酸はもとよりイノシン酸が「うま味」の要素であるという概念すらなく、「コクがある」corsé
  とか「肉汁たっぷり」onctueux (オンクチュー)や succulent
  (スュキロン)などの表現で肉料理やソースの美味しさが表現された。}の働きだけで充分だ。

\hypertarget{fonds-blanc}{%
\subsubsection{白いフォン}\label{fonds-blanc}}

\frsub{Fonds blanc ordinaire}

\index{ふおん@フォン!しろい@白い---}
\index{fonds@fonds!blanc ordinaire@--- blanc ordinaire}

(仕上がり10 L分)

\begin{itemize}
\item
  主素材\ldots{}\ldots{}仔牛のすね、および端肉10k
  g、鶏の手羽やとさか、足など、または鶏がら4羽分、
\item
  香味素材\ldots{}\ldots{}にんじん800 g、玉ねぎ400 g、ポワロー300
  g、セロリ100 g、ブーケガルニ(パセリの枝100
  g、タイム1枝、ローリエの葉1枚、クローブ4本)。
\item
  使用する液体と味付け\ldots{}\ldots{}水12 L、塩60 g。
\item
  作業手順\ldots{}\ldots{}肉は骨を外し、紐で縛る。骨は細かく砕く。鍋に肉と骨を入れ、水を注ぎ塩を加える。火にかけ、浮いてくるアクを取り除き香味素材を加える。
\item
  加熱時間\ldots{}\ldots{}弱火で3時間。
\end{itemize}

\hypertarget{nota-fonds-blanc}{%
\subparagraph{【原注】}\label{nota-fonds-blanc}}

このフォンは火加減を抑えて、出来るだけ澄んだ仕上がりにすること。アクや浮き脂は丁寧に取り除くこと。

茶色いフォンの場合と同様に、始めに細かく砕いた骨だけを煮てから指定量の水を注ぎ、弱火で5時間煮る方法もある。

この骨を煮た汁で肉を煮るわけだ。その作業内容は上記茶色いフォンの場合と同様。この方法は、骨からゼラチン質を完全に抽出出来るという利点がある。当然のことだが、煮ている間に蒸発して失なわれてしまった分は湯を足してやり、全体量を12
Lにしてから肉を煮ること。

\hypertarget{fonds-de-volaille}{%
\subsubsection{鶏のフォン(フォンドヴォライユ)}\label{fonds-de-volaille}}

\frsub{Fonds de volaille}

\index{ふおん@フォン!とりのふおん@鶏の---}
\index{fonds@fonds!volaille@--- de volaille}
\index{かきん@家禽!とりのふおん@鶏のフォン}
\index{うおらいゆ@ヴォライユ!ふおんとうおらいゆ@フォンドヴォライユ}

白いフォンと同じ主素材、香味素材、水の量で、さらに鶏のとさかや手羽、ガラを適宜増量し、廃鶏3羽を加えて作る。

\hypertarget{jus-de-veau-brun}{%
\subsubsection{仔牛の茶色いフォン(仔牛の茶色いジュ)}\label{jus-de-veau-brun}}

\frsub{Fonds, ou Jus de veau brun}

\index{ふおん@フォン!こうしのちやいろい@仔牛の茶色い---}
\index{しゆ@ジュ!こうしのちやいろいしゆ@仔牛の茶色い---}
\index{fonds@fonds!fonds de veau brun@--- de veau brun}
\index{jus@jus!jus de veau brun@--- de veau brun}
\index{こうし@仔牛!こうしのちやいろいふおん@---の茶色いフォン(ジュ)}
\index{veau@veau!fonds brun@fonds ou jus de --- brun}

(仕上がり10 L分)

\begin{itemize}
\item
  主素材\ldots{}\ldots{}骨を取り除いた仔牛のすね肉と肩肉(紐で縛っておく)6kg、細かく砕いた仔牛の骨5
  kg。
\item
  香味素材\ldots{}\ldots{}にんじん600 g、玉ねぎ400 g、パセリの枝100
  g、ローリエの葉 2枚、タイム2枝。
\item
  使用する液体\ldots{}\ldots{}白いフォンまたは水12 L。水を用いる場合は1
  Lあたり3 gの塩を加える。
\item
  作業手順\ldots{}\ldots{}厚手の片手鍋または寸胴鍋の底に輪切りにしたにんじんと玉ねぎを敷きつめる。その他の香味素材と、あらかじめオーブンで焼き色を付けておいた骨と肉を鍋に加える。
\end{itemize}

蓋をして約10分間、蓋をして弱火にかけた野菜から水分が汗をかくように出るイメージで蒸し焼き状態にし、素材の味を引き出す\footnote{suer
  (スュエ)シュエ。}。フォンまたは水少量を加え、煮詰める。この作業をさらに1〜2回行なう。残りのフォンまたは水を注ぎ、蓋をし、沸騰させる。アクを丁寧に取る。微沸騰の状態で6時間煮る。

布で漉し、ストックしておく。使用目的や必要に応じて、さらに煮詰めてからストックしてもいい。

\hypertarget{fonds-de-gibier}{%
\subsubsection{ジビエのフォン}\label{fonds-de-gibier}}

\frsub{Fonds de gibier}

\index{ふおん@フォン!しひえ@ジビエの---}
\index{fonds@fonds!fonds de gibier@--- de gibier}
\index{しひえ@ジビエ!ふおん@---のフォン}
\index{gibier@gibier!fonds@fonds de ---}

(仕上がり5 L分)

\begin{itemize}
\item
  主素材\ldots{}\ldots{}ノロ鹿の頸、胸肉および端肉3
  kg(老いたノロ鹿がいいが、新鮮なものを使うこと)、野うさぎ\footnote{lièvre
    (リエーヴル)。}の端肉1 kg、老うさぎ2羽、山うずら2羽、老きじ1羽。
\item
  香味素材\ldots{}\ldots{}にんじん250 g、玉ねぎ250
  g、セージ1枝、ジュニパーベリー \footnote{セイヨウネズの樹の実。}15粒、標準的なブーケガルニ。
\end{itemize}

\begin{itemize}
\item
  使用する液体\ldots{}\ldots{}水6 Lおよび白ワイン1瓶。
\item
  加熱時間\ldots{}\ldots{}3時間。
\item
  作業手順\ldots{}\ldots{}ジビエは事前にオーブンで焼き色を付けておき、野菜と香草を敷き詰めた鍋に入れる。野菜類も事前に焼き色を付けておくこと。ジビエを焼くのに用いた天板を白ワインでデグラセし、これを鍋に注ぐ。同量の水も加え、ほぼ水分がなくなるまで煮詰める。
\end{itemize}

この作業の後で、残りの水全量を注ぎ、沸騰させる。丁寧にアクを引きながらごく弱火で煮る\footnote{最後に布で漉す必要があるが、当然のこととして明記されていないので注意。}。

\hypertarget{fumet-de-poisson}{%
\subsubsection[魚のフュメ(フュメドポワソン)]{\texorpdfstring{魚のフュメ(フュメドポワソン)\footnote{本質的には前出の「フォン」と同様のものだが、魚(およびジビエ)を素材としたフォンは香りがポイントとなるため、フュメ
  fumet (香気、良い香りの意)の名称のほうが一般的に使われている。}}{魚のフュメ(フュメドポワソン)}}\label{fumet-de-poisson}}

\frsub{Fonds, ou Fumet de poisson}

\index{ふおん@フォン!さかな@魚の---}
\index{ふゆめ@フュメ!さかな@魚の---}
\index{ふゆめ@フュメ!ほわそん@フュメドポワソン}
\index{fumet@fumet!fumet de poisson@--- de poisson}
\index{fonds@fonds!fumet de poisson@fumet de poisson}

(仕上がり10L分)

\begin{itemize}
\item
  主素材\ldots{}\ldots{}舌びらめ、メルラン\footnote{タラの近縁種。}やバルビュ\footnote{ヒラメの近縁種。}のあら10
  kg。
\item
  香味素材\ldots{}\ldots{}薄切りにした玉ねぎ500 g、パセリの根\footnote{パセリには根がにんじん形に肥大する品種もある(persil
    tubéreux 根パセリ。葉は平らでイタリアンパセリのように使う)。}と茎100
  g、マッシュルームの切りくず250 g、レモンの搾り汁1個分、粒こしょう15
  g(これはフュメを漉す10分前に投入する)。
\item
  使用する液体と調味料\ldots{}\ldots{}水10 L、白ワイン1瓶。液体1
  Lあたり3〜4 gの塩。
\item
  加熱時間\ldots{}\ldots{}30分。
\item
  作業手順\ldots{}\ldots{}鍋底に香味野菜を敷き詰め、魚のあらを入れる。水と白ワインを注ぎ、強火にかける。丁寧にアクを引き、微沸騰の状態を保つようにする。
  30分煮たら目の細かい網で漉す。
\end{itemize}

\hypertarget{nota-fumet-de-poisson}{%
\subparagraph{【原注】}\label{nota-fumet-de-poisson}}

質の悪い白ワインを使うと灰色がかったフュメになってしまう。品質の疑わしいワインは使わないほうがいい。

このフュメはソースを作る際に加える液体として用いる。魚料理用ソース・エスパニョルを作ることを想定する場合には、魚のあらをバターでエチュベしてから水と白ワインを注いで煮るといい。

\hypertarget{fonds-de-poisson-au-vin-rouge}{%
\subsubsection{赤ワインを用いた魚のフォン}\label{fonds-de-poisson-au-vin-rouge}}

\frsub{Fonds de poisson au vin rouge}

\index{ふおん@フォン!あかわいんをもちいたさかなのふおん@赤ワインを用いた魚の---}
\index{fonds@fonds!fonds de poisson au vin rouge@--- de poisson au vin rouge}

このフォンそれ自体を用意することは滅多にない。というのも、例えばマトロットのような料理の魚の煮汁そのものだからだ。

とはいえ、こんにちでは魚のアラをすっかり取り除いた状態で料理を提供する必要がますます高まってきているので、ここでそのレシピを記しておくべきだろう。このフォンの必要性と有用さはどんどん高まっていくと思われる。

原則として、このフォンの仕込みには、料理として提供するのと同じ種類の魚のアラを用いて、その香りの特徴を生かす必要がある。だが、どんな種類の魚を使う場合でも作り方は同じだ。

(仕上がり5 L分)

\begin{itemize}
\item
  主素材\ldots{}\ldots{}料理に用いるのと同じ魚種の頭とアラ2.5 kg。
\item
  香味素材\ldots{}\ldots{}薄切りにして下茹でした玉ねぎ300
  g、パセリの枝100
  g、タイムの小枝1本、小さめのローリエの葉2枚、にんにく5片、マッシュルームの切りくず100
  g。
\item
  使用する液体と調味料\ldots{}\ldots{}水3.5 L、良質の赤ワイン2 L、塩15
  g。
\item
  加熱時間\ldots{}\ldots{}30分。
\item
  作業手順\ldots{}\ldots{}「魚の白いフォン\footnote{前項のフュメドポワソンのこと。}」と同様にする。
\end{itemize}

\hypertarget{nota-fonds-de-poisson-au-vin-rouge}{%
\subparagraph{【原注】}\label{nota-fonds-de-poisson-au-vin-rouge}}

このフォンは魚の白いフォンよりも濃く煮詰めることが可能。とはいえ、保存のために煮詰めないでいいように、その都度、必要な量だけ仕込むことを勧める。

\hypertarget{essence-de-poisson}{%
\subsubsection{魚のエッセンス}\label{essence-de-poisson}}

\frsub{Essence de poisson}

\index{えつせんす@エッセンス!さかな@魚の---}
\index{essence@essence!poisson@--- de poisson}

\begin{itemize}
\item
  主素材\ldots{}\ldots{}メルラン\footnote{タラの近縁種。}および舌びらめの頭、アラ2
  kg。
\item
  香味素材\ldots{}\ldots{}薄切りにした玉ねぎ125
  g、マッシュルームの切りくず300 g、パセリの枝50
  g、レモンの搾り汁1個分。
\item
  使用する液体\ldots{}\ldots{}煮詰めていないフュメドポワソン1
  \(\frac{1}{2}\) L、良質の白ワイン3 dL。
\item
  所要時間\ldots{}\ldots{}45分。
\item
  作業手順\ldots{}\ldots{}鍋にバター100
  gと玉ねぎ、パセリの枝、マッシュルームの切りくずを入れ、強火で色づかないようさっと炒める。アラと端肉を加える。蓋をして約15分弱火で蒸し煮する\footnote{素材を入れた鍋に蓋をして弱火にかけ、少量の水分で蒸し煮状態にすることを
    étuver
    エチュベという。このフランス語をそのまま用いている調理現場も少なくない。}。その間、小まめに混ぜてやること。白ワインを注ぎ、半量になるまで煮詰める。最後にフュメドポワソンを注ぎ、レモン汁と塩2
  gを加える。
\end{itemize}

再び火にかけて、とろ火で15分程煮込んだら、布で漉す。

\hypertarget{nota-essence-de-poisson}{%
\subparagraph{【原注】}\label{nota-essence-de-poisson}}

魚のエッセンスは、舌びらめやチュルボ、チュルボタン、バルビュ\footnote{いずれも鰈、ひらめの近縁種。チュルボタンはチュルボの小さいものを言う。}
などのフィレ\footnote{3枚おろし、または5枚おろしにして、頭とアラを取り除いた状態。}をポシェする際に用いる。

さらに、このエッセンスを煮詰めて、上記でポシェした魚のソースに加えて風味を強くするのに使う。

\hypertarget{essences-diverses}{%
\subsubsection{エッセンスについて}\label{essences-diverses}}

\frsub{Essences diverses}

\index{えつせんす@エッセンス!01えつせんすについて@---について(フォン)}
\index{essence@essence!01 diverses@---s diverses (fonds)}

その名のとおり、エッセンスとはごく少量になるまで煮詰めて非常に強い風味を持たせたフォンのこと。

エッセンスは普通のフォンと本質的には同じものだが、素材の風味をしっかり出すために、使用する液体の量はずっと少ない。したがって、仕上げにエッセンスを加える指示がある料理の場合でも、そもそも充分に風味ゆたかなフォンを用いていれば、エッセンスは必要ないことが分かるだろう。

まず最初に、美味しく風味ゆたかなフォンを用いるほうが、あまり出来のよくないフォンで調理し、後からエッセンスで欠点を補うよりもずっと簡単なのだ。その方がいい結果が得られるし、時間と材料の節約にもなる。

セロリ、マッシュルーム、モリーユ\footnote{morille
  キノコの一種。和名アミガサタケ。}、トリュフなど、とりわけ明確な風味の素材のエッセンスを、必要に応じて用いるにとどめるのがいい。

また、十中八九、フォンを仕込む際に素材そのものを加えた方が、エッセンスを仕込むよりもいい結果が得られることは頭に入れておくこと。

そのようなわけで、エッセンスについてこれ以上長々と述べる必要もないと思われる。ベースとなるフォンがコクと風味がゆたかなものならであるなら、エッセンスはまったく無用の長物と言える。

\hypertarget{glaces-diverses}{%
\subsubsection{グラスについて}\label{glaces-diverses}}

\frsub{Glaces diverses}

\index{くらす@グラス!01くらすについて@---について}
\index{glace@glace!01 diverses@---s diverses}

グラスドヴィアンド、鶏のグラス(グラスドヴォライユ)、ジビエのグラス、魚のグラスの用途は多岐にわたる。これらは、上記いずれかの素材でとったフォンをシロップ状になるまで煮詰めたもののことだ。

これらの使い途は、料理の仕上げに表面に塗ってしっとりとした艶を出させるのに用いる場合もあれば、ソースの味や色合いを濃くするために用いたり、あるいは、あまりに出来のよくないフォンで作った料理の場合にはコクを与えるために使うこともある。また、料理によっては適量のバターやクリームを加えてグラスそのものをソースとして用いることもある。

グラスとエッセンスの違いだが、エッセンスが料理の風味そのものを強くすることだけが目的であるのに対して、グラスは素材の持つコクと風味をごく少量にまで濃縮したものだ。

だからほとんどの場合、エッセンスよりもグラスを使うほうがいい。

とはいえ昔の料理長たちの中には、グラスの使用を絶対に認めない者もいた。その理由は、料理を作る度に毎回その料理のためのフォンをとるべきであり、それだけで料理として充分なものにすべき、ということだった。


確かに時間と費用の点で制限がなければその理屈は正しい。だが、こんにちでは、そのようなことの出来る調理現場はほとんどない。そもそもグラスは、正しく適量を用いるのであれば、そのグラスが丁寧に作られたものであるならな、素晴しい結果が得られる。
だから多くの場合、グラスはまことに有用なものと言える。

\hypertarget{glace-de-viande}{%
\subsubsection{グラスドヴィアンド}\label{glace-de-viande}}


\frsub{Glace de viande}

\index{くらす@グラス!くらすとういあんと@---ドヴィアンド}
\index{glace@glace!viande@--- de viande}

茶色いフォン(エストゥファード)を煮詰めて作る。

煮詰めて濃くなっていく途中、何度か布で漉して、より小さな鍋に移しかえていく。煮詰めている際に、丁寧にアクを引くことが、澄んだグラスを作るポイント。

煮詰めている際には、フォンの濃縮具合に応じて、火加減を弱めていくこと。最初は強火でいいが、作業の最後の方は弱火にしてゆっくり煮詰めてやること。

スプーンを入れてみて、引き上げた際に、艶のあるグラスの層でスプーンが覆われ、しっかり張り付いているくらいが丁度いい。要するに、スプーンがグラスでコーティングされた状態になればいいということだ。

\hypertarget{nota-glace-de-viande}{%
\subparagraph{【原注】}\label{nota-glace-de-viande}}

色が薄くて軽い仕上がりのグラスが必要な場合には、茶色いフォンではなく、標準的な仔牛のフォンを用いる。

\hypertarget{glace-de-volaille}{%
\subsubsection{鶏のグラス(グラスドヴォライユ)}\label{glace-de-volaille}}

\frsub{Glace de volaille}

\index{くらす@グラス!とり@鶏の---(---ドヴォライユ)}
\index{くらす@グラス!うおらいゆ@---ドヴォライユ}
\index{glace@glace!volaille@--- de volaille}

鶏のフォン(フォンドヴォライユ)を用いて、グラスドヴィアンドと同様にして作る。

\hypertarget{glace-de-gibier}{%
\subsubsection{ジビエのグラス}\label{glace-de-gibier}}

\frsub{Glace de gibier}

\index{くらす@グラス!しひえ@ジビエの---}
\index{glace@glace!glace de gibier@--- de gibier}
\index{しひえ@ジビエ!くらす@---のグラス}
\index{gibier@gibier!gibier@glace de ---}

ジビエのフォンを煮詰めて作る。ある特定のジビエの風味を生かしたグラスを作るには、そのジビエだけでとったフォンを用いること。

\hypertarget{glace-de-poisson}{%
\subsubsection{魚のグラス}\label{glace-de-poisson}}

\frsub{Glace de poisson}

\index{くらす@グラス!さかな@魚の---}
\index{glace@glace!poisson@--- de poisson}

このグラスを用いることはあまり多くない。日常的な業務においては「魚のエッセンス」を用いることが好まれる。そのほうが魚の風味も繊細になる。魚のエッセンスで魚をポシェした後に煮詰めてソースに加える。
\end{recette}
\hypertarget{roux}{%
\section{ルー}\label{roux}}

\frsec{Roux}

\index{るー@ルー} \index{roux@roux}

ルーはいろいろな派生ソースのベースとなる基本ソースにとろみを付ける役目を持つ。ルーの仕込みは、一見したところさほど重要に思われぬだろうが、実際には正反対だ。丁寧に注意深く作業すること。

茶色いルーは加熱に時間がかかるので、大規模な調理現場では前もって仕込んでおく。ブロンドのルーと白いルーはその都度用意すればいい。
\begin{recette}
\hypertarget{roux-brun}{%
\subsubsection{茶色いルー}\label{roux-brun}}

\frsub{Roux brun}

\index{るー@ルー!ちやいろ@茶色い---} \index{roux@roux!brun@--- brun}

(仕上がり1 kg分)

\begin{enumerate}
\def\labelenumi{\arabic{enumi}.}
\tightlist
\item
  澄ましバター\ldots{}\ldots{}500 g
\item
  ふるった小麦粉\ldots{}\ldots{}600 g
\end{enumerate}

\hypertarget{cuisson-des-roux}{%
\subsubsection{ルーの火入れについて}\label{cuisson-des-roux}}

\index{るー@ルー!ひいれについて@---の火入れについて}
\index{roux@roux!cuisson@cuisson du ---}

加熱時間は使用する熱源の強さで変わってくる。だから数字で何分とは言えない。ただし、火力が強過ぎるよりは弱いくらいの方がいい。というのも、温度が高すぎると小麦粉の細胞が硬化して中身を閉じ込めてしまい、そうなると後でフォンなどの液体を加えた際に上手く混ざらず、滑らかなとろみの付いたソースにならない。乾燥豆をいきなり熱湯で茹でるのと同じようなことが起きるわけだ。低い温度から始めてだんだんと熱くしていけば、小麦粉の細胞壁がゆるんで細胞中のでんぷんが膨張し、熱によって発酵状態の初期のようになる。このようにして、でんぷんをデキストリンに変化させる\footnote{現代の科学的見地からすると必ずしも正確な記述ではないので注意。}。デキストリンは水溶性の物質で、これが「とろみ」の主な要素なのだ。茶色いルーは淡褐色の美しい色合いで滑らかな仕上がりにする。だまがあってはいけない。

ルーを作る際には必ず、澄ましバターを使うこと\footnote{初版〜第三版では「澄ましバターまたは充分に澄ましたグレスドマルミット」となっている。グレスドマルミットとは、コンソメなどを作る際に、浮いてくる油脂を取り除く必要があるが、それを捨てずにまとめてから漉して澄ませたもののこと。基本的に獣脂と考えていい。なお、同時代の料理書
  --- 例えばペラプラ『近代料理技術』(1935年)---
  には、ルーを作るのにバターを使う必要はなく、グレスドマルミットで充分、としているものもある。}。
生のバターには相当量のカゼインが含まれている。カゼインがあると火を均質に通すことが出来なくなってしまう。とはいえ、以下を覚えておくといい。ソースとして仕上げた段階で、ルーで使ったバターは風味という点ではほとんど意味が失なわれている。そもそもソースの仕上げに不純物を取り除く\footnote{dépouiller
  デプイエ。ソースや煮込み料理を仕上げる際に、浮き上がってくる不純物を徹底的に取り除き、目の細かい布などで漉すこと。原義は動物などの皮を剥ぐ、剥くことの意で、野うさぎの皮を剥ぐ、うなぎの皮を剥く、という意味で現代の厨房でも用いられているる。ソースの場合は表面に凝固した蛋白質や油脂の膜が出来、それを「剥ぐように」取り除くことから、あるいは表面に浮いてくる不純物を徹底的に取り除いてきれいなソースに仕上げることを、動物の皮を剥いてきれいな身だけにすることになぞらえて、この用語が用いられるようになったようだ。なお、本書においてécumer(エキュメ)が単に浮いてくる泡やアクを取る、という作業であるのに対して、dépouiller(デプイエ)は「徹底的に不純物を取り除いて美しく仕上げる」という意味合いが込められている。現代では品種改良や農法の変化によって野菜のアクも少なくなり、小麦粉も精製度の高いものを利用出来るなど、食材および調味料の多くで純度の高いものを使用する場合がほとんどであり、このデプイエという作業は20世紀後半にはほとんど行なわれなくなり、écumer(エキュメ)という用語だけで済ませることがほとんど(cf.辻静雄監訳『オリヴェ
  ソースの本』柴田書店、 1970年、27〜28頁)。}段階でバターも完全に取り除かれてしまうわけだ。だからルーに用いるバターは小麦粉に熱を通すためだけのものと考えていい。

ルーはソース作りの出発点だ。だから次の点も記憶に\ruby{留}{とど}めること。小麦粉にでんぷんが含まれているからこそソースに「とろみ」が付く。だから純粋なでんぷん(特性が小麦のでんぷんと同じでも異なったものでも)でルーを作っても、小麦粉の場合と同様の結果が得られるだろう。ただしその場合は小麦粉でルーを作る場合より注意して作業する必要がある。また、小麦粉と違って余計な物質が含まれていないために、全体の分量比率を考え直すことになる。

\hypertarget{nota-roux}{%
\subparagraph{【原注】}\label{nota-roux}}

本文で述べたように、茶色いルーを作る際には澄ましバターを用いる。他の動物性油脂はよほど経済的事情が逼迫していない限り使わないこと。材料コストが問題になる場合でも、ソースの仕上げに不純物を取り除く際に多少の注意を払えば、ルーに用いたバターを回収するのはさして難しいことではない\footnote{既に述べたように初版〜第三版まではバターまたはグレスドマルミットという指示であったことに留意する必要はあるだろう。実際のところ、良質のバターを用いてルーを作ったほうが、軽やかな仕上りのソースになる傾向があることは言うまでもない。}。それを後で他の用途で使えばいいだろう。

\hypertarget{roux-blond}{%
\subsubsection{ブロンドのルー}\label{roux-blond}}

\frsub{Roux blond}

\index{るー@ルー!ふろんと@ブロンドの---}
\index{roux@roux!blond@--- blond}

(仕上がり1 kg分)

材料の比率は茶色いルーと同じ。すなわちバター500 gと、ふるった小麦粉600
g。

火入れは、ルーがほんのりブロンド色になるまで、ごく弱火で行なう。

\hypertarget{roux-blanc}{%
\subsubsection{白いルー}\label{roux-blanc}}

\frsub{Roux blanc}

\index{るー@ルー!しろい@白い---} \index{roux@roux!blanc@--- blanc}

500 gのバターと、ふるった小麦粉600 g。

このルーの火入れは数分、つまり粉っぽさがなくなるまでの時間でいい。
\end{recette}
\PassOptionsToPackage{unicode=true}{hyperref} % options for packages loaded elsewhere
\PassOptionsToPackage{hyphens}{url}
%
\documentclass[14Q,a4paperpaper,]{ltjsbook}
\usepackage{lmodern}
\usepackage{amssymb,amsmath}
\usepackage{ifxetex,ifluatex}
\usepackage{fixltx2e} % provides \textsubscript
\ifnum 0\ifxetex 1\fi\ifluatex 1\fi=0 % if pdftex
  \usepackage[T1]{fontenc}
  \usepackage[utf8]{inputenc}
  \usepackage{textcomp} % provides euro and other symbols
\else % if luatex or xelatex
  \usepackage{unicode-math}
  \defaultfontfeatures{Ligatures=TeX,Scale=MatchLowercase}
\fi
% use upquote if available, for straight quotes in verbatim environments
\IfFileExists{upquote.sty}{\usepackage{upquote}}{}
% use microtype if available
\IfFileExists{microtype.sty}{%
\usepackage[]{microtype}
\UseMicrotypeSet[protrusion]{basicmath} % disable protrusion for tt fonts
}{}
\IfFileExists{parskip.sty}{%
\usepackage{parskip}
}{% else
\setlength{\parindent}{0pt}
\setlength{\parskip}{6pt plus 2pt minus 1pt}
}
\usepackage{hyperref}
\hypersetup{
            pdftitle={エスコフィエ『料理の手引き』全注解},
            pdfauthor={五 島 学},
            pdfborder={0 0 0},
            breaklinks=true}
\urlstyle{same}  % don't use monospace font for urls
\setlength{\emergencystretch}{3em}  % prevent overfull lines
\providecommand{\tightlist}{%
  \setlength{\itemsep}{0pt}\setlength{\parskip}{0pt}}
\setcounter{secnumdepth}{0}
% Redefines (sub)paragraphs to behave more like sections
\ifx\paragraph\undefined\else
\let\oldparagraph\paragraph
\renewcommand{\paragraph}[1]{\oldparagraph{#1}\mbox{}}
\fi
\ifx\subparagraph\undefined\else
\let\oldsubparagraph\subparagraph
\renewcommand{\subparagraph}[1]{\oldsubparagraph{#1}\mbox{}}
\fi

% set default figure placement to htbp
\makeatletter
\def\fps@figure{htbp}
\makeatother


\title{エスコフィエ『料理の手引き』全注解}
\author{五 島 学}
\date{}
%%%%%%%%%%%%%%%%%%%%%%% added by mgoto
\usepackage{setspace}


% %%%%%%%%% hyperref %%%%%%%%%%%%%
% \usepackage{refcount}
% \usepackage[unicode=true,hyperfootnotes=false,pageanchor]{hyperref}
% \hypersetup{hyperindex=false,%
%              breaklinks=true,%
%              bookmarks=true,%
%              pdfauthor={五島 学},%
%              pdftitle={エスコフィエ『料理の手引き』全注解},%
%              colorlinks=false%true,%
%              %colorlinks=true,%
%              citecolor=blue,%
%              urlcolor=cyan,%
%              linkcolor=magenta,%
%              bookmarksdepth=subsubsection,%
%              pdfborder={0 0 0},%
%              hyperfootnotes=false,%
%              plainpages=false,
%              }
% \urlstyle{same}





%% 欧文フォント設定
% Libertine/Biolinum
\setmainfont[Ligatures=Historic,Scale=1.0]{Linux Libertine O}
\setsansfont[Ligatures=TeX, Scale=MatchLowercase]{Linux Biolinum O} 
%\usepackage{libertine}
\usepackage{unicode-math}
\setmathfont[Scale=1.2]{libertinusmath-regular.otf}
\usepackage{luatexja}
\usepackage{luatexja-fontspec}
%\ltjdefcharrange{8}{"2000-"2013, "2015-"2025, "2027-"203A, "203C-"206F}
%\ltjsetparameter{jacharrange={-2, +8}}
\usepackage{luatexja-ruby}

\newopentypefeature{PKana}{On}{pkna} % "PKana" and "On" can be arbitrary string
%%%%明朝にIPAexMincho、ゴチ(太字)にMoboGoBを使う設定。和文カナプロプーショナル使用可能だが読みづらくなる。
\setmainjfont[%
     %YokoFeatures={JFM=prop,PKana=On},%
     %CharacterWidth=AlternateProportional,%
%    CharacterWidth=Proportional,%Mogo, IPAExMinchoには不可
     %Kerning=On,%
     BoldFont={ MoboGoB },%
     ItalicFont={ MoboGoB },%
     BoldItalicFont={ MoboGoExB }%
     % ]{ MogaHMin }
     ]{ IPAExMincho }
     % ]{ IPAmjMincho }
\setsansjfont[%
     %YokoFeatures={JFM=prop,PKana=On},%
     %CharacterWidth=AlternateProportional,%
     % CharacterWidth=Proportional,%Mobo, IPAExGOthicには不可
     %Kerning=On,
     BoldFont={ MoboGoB },%
     ItalicFont={ MoboGoB },%
     BoldItalicFont={ MoboGoExB }%
     % ]{ MoboGo}
     ]{ IPAExGothic }
     % %  %%%% 和文仮名プロプーショナルここまで

     
\renewcommand{\bfdefault}{bx}%和文ボールドを有効にする
\renewcommand{\headfont}{\gtfamily\sffamily\bfseries}%和文ボールドを有効にする

%リスト環境
\def\tightlist{\itemsep1pt\parskip0pt\parsep0pt}%pandoc対策

\makeatletter
  \parsep   = 0pt
  \labelsep = .5\zw
  \def\@listi{%
     \leftmargin = 0pt \rightmargin = 0pt
     \labelwidth\leftmargin \advance\labelwidth-\labelsep
     \topsep     = 0pt%\baselineskip
     %\topsep -0.1\baselineskip \@plus 0\baselineskip \@minus 0.1 \baselineskip
     \partopsep  = 0pt \itemsep       = 0pt
     \itemindent = -.5\zw \listparindent = 0\zw}
  \let\@listI\@listi
  \@listi
  \def\@listii{%
     \leftmargin = 1.8\zw \rightmargin = 0pt
     \labelwidth\leftmargin \advance\labelwidth-\labelsep
     \topsep     = 0pt \partopsep     = 0pt \itemsep   = 0pt
     \itemindent = 0pt \listparindent = 1\zw}
  \let\@listiii\@listii
  \let\@listiv\@listii
  \let\@listv\@listii
  \let\@listvi\@listii
\makeatother

%%%%インデックス準備
%\usepackage{makeidx}
\usepackage{index}
%\usepackage[useindex]{splitidx}
\newindex{src}{idx1}{ind1}{ソース名から料理を探す}
\makeindex
 
 \makeatletter
\renewenvironment{theindex}{% 索引を3段組で出力する環境
    \if@twocolumn
      \onecolumn\@restonecolfalse
    \else
      \clearpage\@restonecoltrue
    \fi
    \columnseprule.4pt \columnsep 2\zw
    \ifx\multicols\@undefined
      \twocolumn[\@makeschapterhead{\indexname}%
      \addcontentsline{toc}{chapter}{\indexname}]%
    \else
      \ifdim\textwidth=\fullwidth
        \setlength{\evensidemargin}{\oddsidemargin}
        \setlength{\textwidth}{\fullwidth}
        \setlength{\linewidth}{\fullwidth}
        \begin{multicols}{3}[\chapter*{\indexname}%
        \addcontentsline{toc}{chapter}{\indexname}]%
      \else
        \begin{multicols}{2}[\chapter*{\indexname}%
        \addcontentsline{toc}{chapter}{\indexname}]%
      \fi
    \fi
    \@mkboth{\indexname}{}%
    \plainifnotempty % \thispagestyle{plain}
    \parindent\z@
    \parskip\z@ \@plus .3\jsc@mpt\relax
    \let\item\@idxitem
    \raggedright
    \footnotesize\narrowbaselines
  }{
    \ifx\multicols\@undefined
      \if@restonecol\onecolumn\fi
    \else
      \end{multicols}
    \fi
    \clearpage
  }
 \makeatother


%%%% 本文中の参照ページ番号表示 %%%%%%%

\makeatletter

%\AtBeginDocument{%
%  \DeclareRobustCommand\ref{\@ifstar\@refstar\@refstar}%
%  \DeclareRobustCommand\pageref{\@ifstar\@pagerefstar\@pagerefstar}}
\let\orig@Hy@EveryPageAnchor\Hy@EveryPageAnchor
\def\Hy@EveryPageAnchor{%
    \begingroup
    \hypersetup{pdfview=Fit}%
    \orig@Hy@EveryPageAnchor
    \endgroup
  }
  \usepackage{etoolbox}
\if@mainmatter{\let\myhyperlink\hyperlink%
\renewcommand{\hyperlink}[2]{\myhyperlink{#1}{#2} [p.\getpagerefnumber{#1}{}] }}
  \AtBeginEnvironment{recette}{%
\let\myhyperlink\hyperlink%
\renewcommand{\hyperlink}[2]{\myhyperlink{#1}{#2} [p.\getpagerefnumber{#1}{}] }}
  \AtBeginEnvironment{Main}{%
\let\myhyperlink\hyperlink%
\renewcommand{\hyperlink}[2]{\myhyperlink{#1}{#2} [p.\getpagerefnumber{#1}{}] }}


%%%% pandoc が三点リーダーを勝手に変える対策
\renewcommand{\ldots}{\noindent…}

%%% 脚注番号のページ毎のリセットと脚注位置の調整
%\renewcommand{\footnotesize}{\small}

\makeatletter

\usepackage[bottom,perpage,stable]{footmisc}%
%\setlength{\skip\footins}{4mm plus 4mm}
%\usepackage{footnpag}
\renewcommand\@makefntext[1]{%
  \advance\leftskip 0\zw
  \parindent 1\zw
  \noindent
  \llap{\@thefnmark\hskip0.5\zw}#1}


\let\footnotes@ve=\footnote
\def\footnote{\inhibitglue\footnotes@ve}
\let\footnotemarks@ve=\footnotemark
%\def\footnotemark{\inhibitglue\footnotemarks@ve}
\renewcommand{\footnotemark}{\footnotemarks@ve}%変更
% %\def\thefootnote{\ifnum\c@footnote>\z@\leavevmode\lower.5ex\hbox{(}\@arabic\c@footnote\hbox{)}\fi}
\renewcommand{\thefootnote}{\ifnum\c@footnote>\z@\leavevmode\hbox{}\@arabic\c@footnote\hbox{)}\fi}

%%%%%%%%%レシピと本文%%%%%%%%%%%%
\usepackage{multicol}
\setlength{\columnsep}{3\zw}

%%% 本文
\newenvironment{Main}{}{}
%%% レシピ
% \setlength{\columnwidth}{24\zw}
%本文ヨリ小%\small
%\newenvironment{recette}{\setlength{\parindent}{0pt}\begin{small}\begin{spaceing}{0.8}\begin{multicols}{2}}{\end{multicols}\end{spacing}\end{small}}
%本文やや小%\medsmall
%\newenvironment{recette}{\setlength{\parindent}{0pt}\begin{medsmall}\begin{spacing}{0.75}\begin{multicols}{2}}{\end{multicols}\end{spacing}\end{medsmall}}
%本文ナミ(無指定)
\newenvironment{recette}{\setlength{\parindent}{0pt}\begin{spacing}{0.8}\begin{multicols}{2}\setlength\topskip{.8\baselineskip}}{\end{multicols}\end{spacing}}

%文字サイズ、見出しなどの再定義
\makeatletter
%\renewcommand{\large}{\jsc@setfontsize\large\@xipt{14}}
%\renewcommand{\Large}{\jsc@setfontsize\Large{13}{15}}

\newcommand{\medlarge}{\fontsize{11}{13}\selectfont}
\newcommand{\medsmall}{\fontsize{9.23}{9.5}\selectfont}
\newcommand{\twelveq}{\jsc@setfontsize\twelveq{9.230769}{9.75}\selectfont}
\newcommand{\thirteenq}{\jsc@setfontsize\fourteenq{10}{11}\selectfont}
\newcommand{\fourteenq}{\jsc@setfontsize\fourteenq{10.7692}{13}\selectfont}
\newcommand{\fifteenq}{\jsc@setfontsize\fifteenq{11.53846}{14}\selectfont}
\makeatletter
\renewcommand{\chapter}{%
  \if@openleft\cleardoublepage\else
  \if@openright\cleardoublepage\else\clearpage\fi\fi
  \plainifnotempty % 元: \thispagestyle{plain}
  \global\@topnum\z@
  \if@english \@afterindentfalse \else \@afterindenttrue \fi
  \secdef
    {\@omit@numberfalse\@chapter}%
    {\@omit@numbertrue\@schapter}}
\def\@chapter[#1]#2{%
  \ifnum \c@secnumdepth >\m@ne
    \if@mainmatter
      \refstepcounter{chapter}%
      \typeout{\@chapapp\thechapter\@chappos}%
      \addcontentsline{toc}{chapter}%
        {\protect\numberline
        % {\if@english\thechapter\else\@chapapp\thechapter\@chappos\fi}%
        {\@chapapp\thechapter\@chappos}%
        #1}%
    \else\addcontentsline{toc}{chapter}{#1}\fi
  \else
    \addcontentsline{toc}{chapter}{#1}%
  \fi
  \chaptermark{#1}%
  \addtocontents{lof}{\protect\addvspace{10\jsc@mpt}}%
  \addtocontents{lot}{\protect\addvspace{10\jsc@mpt}}%
  \if@twocolumn
    \@topnewpage[\@makechapterhead{#2}]%
  \else
    \@makechapterhead{#2}%
    \@afterheading
  \fi}
\def\@makechapterhead#1{%
  \vspace*{0\Cvs}% 欧文は50pt
  {\parindent \z@ \centering \normalfont
    \ifnum \c@secnumdepth >\m@ne
      \if@mainmatter
        \huge\headfont \@chapapp\thechapter\@chappos%変更
        \par\nobreak
        \vskip \Cvs % 欧文は20pt
      \fi
    \fi
    \interlinepenalty\@M
    \huge \headfont #1\par\nobreak
    \vskip 1\Cvs}} % 欧文は40pt%変更

\renewcommand{\section}{%
    \if@slide\clearpage\fi
    \@startsection{section}{1}{\z@}%
    {\Cvs \@plus.5\Cdp \@minus.2\Cdp}% 前アキ
    % {.5\Cvs \@plus.3\Cdp}% 後アキ
    {.5\Cvs}
    {\normalfont\Large\headfont\bfseries\centering}}%変更

\renewcommand{\subsection}{\@startsection{subsection}{2}{\z@}%
    {\Cvs \@plus.5\Cdp \@minus.2\Cdp}% 前アキ
    % {.5\Cvs \@plus.3\Cdp}% 後アキ
    {.5\Cvs}
  %  {\normalfont\large\headfont\bfseries\centering}} %変更
    {\normalfont\large\headfont\centering}} %変更

\renewcommand{\subsubsection}{\@startsection{subsubsection}{3}{\z@}%
  % {0\Cvs \@plus.8\Cdp \@minus.6\Cdp}%変更
    {1sp \@plus.5\Cdp \@minus.5\Cdp}%変更
    {\if@slide .5\Cvs \@plus.3\Cdp \else \z@ \fi}%
    % {\normalfont\medlarge\headfont\leftskip -1\zw}}
    {\normalfont\medlarge\headfont\leftskip -1\zw}}

\renewcommand{\paragraph}{\@startsection{paragraph}{4}{\z@}%
    {0.5\Cvs \@plus.5\Cdp \@minus.2\Cdp}%
    % {\if@slide .5\Cvs \@plus.3\Cdp \else -1\zw\fi}% 改行せず 1\zw のアキ
    {1sp}%後アキ
    {\normalfont\normalsize\headfont}}
\renewcommand{\subparagraph}{\@startsection{subparagraph}{5}{\z@}%
    {\z@}{\if@slide .5\Cvs \@plus.3\Cdp \else -.5\zw\fi}%
    {\normalfont\normalsize\headfont\hskip-.5\zw\noindent}}  

\newcommand{\frchap}[1]{\vspace*{-2ex}%
 \begin{center}\normalfont\headfont\LARGE\setstretch{0.8}
 \scshape#1\normalfont\normalsize
\end{center}\vspace{0.5\zw}\setstretch{1.0}}

\newcommand{\frsec}[1]{\vspace*{-2ex}%
 \begin{center}\normalfont\headfont\large\setstretch{0.8}
 \scshape#1\normalfont\normalsize
\end{center}\vspace{0.5\zw}\setstretch{1.0}}
  
\newcommand{\frsecb}[1]{\vspace*{-2ex}%
\begin{center}\normalfont\headfont\medlarge\setstretch{0.8}%
  \hspace{1em}\scshape#1\normalfont\normalsize%
\end{center}\vspace{0.5\zw}\setstretch{1.0}}
\makeatother

\newenvironment{frchapenv}{\vspace*{-2ex}\begin{center}\normalfont\headfont%
    \LARGE\setstretch{0.8}\normalfont\normalsize\scshape%
    }{\end{center}\vspace{0.5\zw}\setstretch{1.0}}

\newenvironment{frsecenv}{\vspace*{-2ex}%
\begin{center}\normalfont\headfont\medlarge\setstretch{0.8}%
  \hspace{1em}\normalfont\normalsize\scshape%
}{\end{center}\vspace{0.5\zw}\setstretch{1.0}}

\newenvironment{frsecbenv}{\vspace*{-2ex}%
\begin{center}\normalfont\headfont\medlarge\setstretch{0.8}%
  \hspace{1em}\normalfont\normalsize\scshape%
}{\end{center}\vspace{0.5\zw}\setstretch{1.0}}

\newenvironment{frsecenv}{\vspace*{-2ex}%
  \begin{center}\normalfont\headfont\large\setstretch{0.8}\normalfont\normalsize\scshape}%
  {\end{center}\vspace{0.5\zw}\setstretch{1.0}}

\newenvironment{frsubenv}{\begin{spacing}{0.2}\setlength{\leftskip}{-1\zw}\bfseries}{\end{spacing}\normalfont\normalsize\setlength{\leftskip}{0pt}\par\vspace{1.1\zw}}

\newcommand{\frsub}[1]{\begin{frsubenv}#1\end{frsubenv}\par\vspace{1.1\zw}}


\renewcommand{\thechapter}{}
\renewcommand{\thesection}{\hskip-1\zw}
\renewcommand{\thesubsection}{}
\renewcommand{\thesubsubsection}{}
\renewcommand{\theparagraph}{}


% \makeatletter
% \@removefromreset{subsubsection}{subsection}
% \def\thesubsubsection{\arabic{subsubsection}.}
% \newcounter{rnumber}
% \renewcommand{\thernumber}{\refstepcounter{rnumber} }
\makeatother
\renewcommand{\prepartname}{\if@english Part~\else {}\fi}
\renewcommand{\postpartname}{\if@english\else {}\fi}
\renewcommand{\prechaptername}{\if@english Chapter~\else {}\fi}
\renewcommand{\postchaptername}{\if@english\else {}\fi}
\renewcommand{\presectionname}{}%  第
\renewcommand{\postsectionname}{}% 節

%\makeatletter
\def\ps@headings{%
  \let\@oddfoot\@empty
  \let\@evenfoot\@empty
  \def\@evenhead{%
    \if@mparswitch \hss \fi
    \underline{\hbox to \fullwidth{\ltjsetparameter{autoxspacing={true}}
%      \textbf{\thepage}\hfil\leftmark}}%
       \normalfont\thepage\hfill\scshape\small\leftmark\normalfont}}%
    \if@mparswitch\else \hss \fi}%
  \def\@oddhead{\underline{\hbox to \fullwidth{\ltjsetparameter{autoxspacing={true}}
        {\if@twoside\scshape\small\rightmark\else\scshape\small\leftmark\fi}\hfil\thepage\normalfont}}\hss}%
  \let\@mkboth\markboth
  \def\chaptermark##1{\markboth{%
    \ifnum \c@secnumdepth >\m@ne
      \if@mainmatter
        \if@omit@number\else
          \@chapapp\thechapter\@chappos\hskip1\zw
        \fi
      \fi
    \fi
    ##1}{}}%
  \def\sectionmark##1{\markright{%
%    \ifnum \c@secnumdepth >\z@ \thesection \hskip1\zw\fi
    \ifnum \c@secnumdepth >\z@ \thesection \hskip-1\zw\fi
    ##1}}}%
\makeatother

\makeatletter
%%%%%%%% Lua GC
\patchcmd\@outputpage{\stepcounter{page}}{%
  \directlua{%
	if jit then
      local k = collectgarbage("count")
      if k>900000 then 
        collectgarbage("collect")
        texio.write_nl("term and log", "GC: ", math.floor(k), math.floor(collectgarbage("count")))
      end
	end
  }%
  \stepcounter{page}%
}{}{}
\makeatother

%\input{preamble/sources-index}%ソース名からの逆引きインデックス用コマンド

\makeatletter

%%%%%%基本ソース

\def\srcEspagnole#1#2#3#4{%
  \index[src]{espagnole@Espagnole!{#1}@{#2}}
  \index[src]{えすはによる@エスパニョル!{#3}@{#4}}}
 
\def\srcEspagnoleMaigre#1#2#3#4{%
  \index[src]{espagnole maigre@Espagnole maigre!{#1}@{#2}}%
  \index[src]{えすはによるさかな@エスパニョル(魚料理用)!{#3}@{#4}}}

\def\srcDemiGlace#1#2#3#4{%
  \index[src]{demi-glace@Demi-glace!{#1}@{#2}}%
  \index[src]{とうみくらす@ドゥミグラス!{#3}@{#4}}}

\def\srcJusDeVeauLie#1#2#3#4{%
  \index[src]{jus veau lie@Jus de veau lié!{#1}@{#2}}%
  \index[src]{とろみをつけたこうしのしゆ@とろみを付けた仔牛のジュ!{#3}@{#4}}}

\def\srcVeoute#1#2#3#4{%
  \index[src]{veloute@Velouté!{#1}@{#2}}%
  \index[src]{うるて@ヴルテ!{#3}@{#4}}}

\def\srcVeouteDeVolaille#1#2#3#4{%
  \index[src]{veloute de volaille@Velouté de volaille!{#1}@{#2}}%
  \index[src]{とりのうるて@鶏のヴルテ!{#3}@{#4}}}

\def\srcVeouteDePoisson#1#2#3#4{%
  \index[src]{veloute de poisson@Velouté de poisson!{#1}@{#2}}%
  \index[src]{さかなのうるて@魚のヴルテ!{#3}@{#4}}}

\def\srcAllemande#1#2#3#4{%
  \index[src]{allemande@Allemande!{#1}@{#2}}%
  \index[src]{あるまんと@アルマンド!{#3}@{#4}}}

\def\srcSupreme#1#2#3#4{%
  \index[src]{supreme@Suprême!{#1}@{#2}}%
  \index[src]{しゆふれーむ@シュプレーム!{#3}@{#4}}}

\def\srcBechamel#1#2#3#4{%
  \index[src]{bechamel@Bechamel!{#1}@{#2}}%
  \index[src]{へしやめる@ベシャメル!{#3}@{#4}}}

\def\srcTomate#1#2#3#4{%
  \index[src]{tomate@Tomate!{#1}@{#2}}%
  \index[src]{とまと@トマト!{#3}@{#4}}}

%%%%%%%%ブラウン系の派生ソース

\def\srcBigarade#1#2#3#4{%
  \index[src]{bigarade@Bigarade!{#1}@{#2}}%
  \index[src]{ひからーと@ビガラード!{#3}@{#4}}}

\def\srcBordelaise#1#2#3#4{%
  \index[src]{bordelaise@Bordelaise!{#1}@{#2}}%
  \index[src]{ほるとーふう@ボルドー風!{#1}@{#2}}}

\def\srcBourguignonne#1#2#3#4{%
  \index[src]{bourguignonne@Bourguignonne!{#1}@{#2}}%
  \index[src]{ふるこーにゆふう@ブルゴーニュ風!{#3}@{#4}}}

\def\srcBretonne#1#2#3#4{%
  \index[src]{bretonne@Bretonne!{#1}@{#2}}%
  \index[src]{ふるたーにゆふう@ブルターニュ風!{#3}@{#4}}}

\def\srcCerises#1#2#3#4{%
  \index[src]{cerises@Cerises (aux)!{#1}@{#2}}%
  \index[src]{すりーす@スリーズ!{#3}@{#4}}}

\def\srcChampignons#1#2#3#4{%
  \index[src]{champignons@Champignons (aux)!{#1}@{#2}}%
  \index[src]{しやんひによん@シャンピニョン!{#3}@{#4}}}

\def\srcChampignons#1#2#3#4{%
  \index[src]{charcutiere@Charcutière!{#1}@{#2}}%
  \index[src]{しやるきゆていえーる@シャルキュティエール!{#3}@{#4}}}

\def\srcChasseur#1#2#3#4{%
  \index[src]{chasseur@Chasseur!{#1}@{#2}}%
  \index[src]{しやすーる@シャスール!{#3}@{#4}}}

\def\srcChasseurEscoffier#1#2#3#4{%
  \index[src]{しやすーるえすこふいえ@シャスール(エスコフィエ)!{#1}@{#2}}%
  \index[src]{chasseur escoffier@Chasseur (Escoffier)!{#3}@{#4}}}

\def\srcChaudFroidBrune#1#2#3#4{%
  \index[src]{chaud-froid brune@Chaud-froid brune!{#1}@{#2}}%
  \index[src]{しよふろわしやいろ@ショフロワ(茶色)!{#3}@{#4}}}

\def\srcChaudFroidBruneCanard#1#2#3#4{%
  \index[src]{chaud-froid brune canards@Chaud-froid brune pour Canards!{#1}@{#2}}%
  \index[src]{しよふろわちやいろかも@ショフロワ(茶色、鴨用)!{#3}@{#4}}}

\def\srcChaudFroidBruneGibier#1#2#3#4{%
  \index[src]{chaud-froid brune gibier@Chaud-froid brune pour Gibier!{#1}@{#2}}%
  \index[src]{しよふろわちやいろしひえ@ショフロワ(茶色、ジビエ用)!{#3}@{#4}}}

\def\srcChaudFroidTomatee#1#2#3#4{%
  \index[src]{chaud-froid tometee@Chaud-froid tomatée!{#1}@{#2}}%
  \index[src]{しよふろわとまて@トマト入りショフロワ!{#3}@{#4}}}

\def\srcChevreuil#1#2#3#4{%
  \index[src]{chevreuil@Chevreuil!{#1}@{#2}}%
  \index[src]{しゆうるいゆ@シュヴルイユ!{#3}@{#4}}}

\def\srcColbert#1#2#3#4{%
  \index[src]{colbert@Colbert!{#1}@{#2}}%
  \index[src]{こるへーる@コルベール!{#3}@{#4}}}

\def\srcDiable#1#2#3#4{%
  \index[src]{diable@Diable!{#1}@{#2}}%
  \index[src]{ていあーふる@ディアーブル!{#3}@{#4}}}

\def\srcDiableEscoffier#1#2#3#4{%
  \index[src]{diable escoffier@Diable Escoffier!{#1}@{#2}}%
  \index[src]{ていあーふるえすこふいえ@ディアーブル・エスコフィエ!{#3}@{#4}}}

\def\srcDiane#1#2#3#4{%
  \index[src]{diane@Diane!{#1}@{#2}}%
  \index[src]{ていあーぬ@ディアーヌ!{#3}@{#4}}}

\def\srcDuxelles#1#2#3#4{%
  \index[src]{duxelles@Duxelles!{#1}@{#2}}%
  \index[src]{てゆくせる@デュクセル!{#3}@{#4}}}

\def\srcEstragon#1#2#3#4{%
  \index[src]{estragon@Estragon!{#1}@{#2}}%
  \index[src]{えすとらこん@エストラゴン!{#3}@{#4}}}

\def\srcFinanciere#1#2#3#4{%
  \index[src]{financierel@Financière!{#1}@{#2}}%
  \index[src]{ふいなんしえーる@フィナンシエール!{#3}@{#4}}}

\def\srcFinesHerbes#1#2#3#4{%
  \index[src]{fines herbes@Fines herbes (aux)!{#1}@{#2}}%
  \index[src]{こうそう@香草!{#3}@{#4}}}

\def\srcGenevoise#1#2#3#4{%
  \index[src]{genevoise@Genevoise!{#1}@{#2}}%
  \index[src]{しゆねーうふう@ジュネーヴ風!{#3}@{#4}}}

\def\srcGodard#1#2#3#4{%
  \index[src]{godard@Godard!{#1}@{#2}}%
  \index[src]{こたーる@ゴダール!{#3}@{#4}}}

\def\srcGrandVeneur#1#2#3#4{%
  \index[src]{grand-veneur@Grand-Veneur!{#1}@{#2}}%
  \index[src]{くらんうぬーる@グランヴヌール!{#3}@{#4}}}

\def\srcGrandVeneurEscoffier#1#2#3#4{%
  \index[src]{grand-veneur escoffier@Grand-Veneur Escoffier!{#1}@{#2}}%
  \index[src]{くらんうぬーるえすこふぃえ@グランヴヌール(エスコフィエ)!{#3}@{#4}}}

\def\srcGratin#1#2#3#4{%
  \index[src]{gratin@Gratin!{#1}@{#2}}%
  \index[src]{くらたん@グラタン!{#3}@{#4}}}

\def\srcHachee#1#2#3#4{%
  \index[src]{hachee@Hachée!{#1}@{#2}}%
  \index[src]{あしえ@アシェ!{#3}@{#4}}}

\def\srcHacheeMaigre#1#2#3#4{%
  \index[src]{hachee maigrel@Hachée maigre!{#1}@{#2}}%
  \index[src]{あしえさかな@アシェ(魚料理用)!{#3}@{#4}}}

\def\srcHussarde#1#2#3#4{%
  \index[src]{hussarde@Hussarde!{#1}@{#2}}%
  \index[src]{ゆさると@ユサルド!{#3}@{#4}}}

\def\srcItalienne#1#2#3#4{%
  \index[src]{italienne@Italienne!{#1}@{#2}}%
  \index[src]{いたりあふう@イタリア風!{#3}@{#4}}}

\def\srcJusLieEstragon#1#2#3#4{%
  \index[src]{jus lie a l'estragon@Jus lié à l'Estragon!{#1}@{#2}}%
  \index[src]{とろみをつけたしゆえすとらこん@とろみを付けたジュ・エストラゴン風味!{#3}@{#4}}}

\def\srcJusLieTomate#1#2#3#4{%
  \index[src]{jus lie tomate@Jus lié tomaté!{#1}@{#2}}%
  \index[src]{とろみをつけたしゆとまと@とろみを付けたジュ・トマト風味!{#3}@{#4}}}

\def\srcLyonnaise#1#2#3#4{%
  \index[src]{lyonnaise@Lyonnaise!{#1}@{#2}}%
  \index[src]{りよんふう@リヨン風!{#3}@{#4}}}

\def\srcMadere#1#2#3#4{%
  \index[src]{madere@Madère!{#1}@{#2}}%
  \index[src]{まてーる@マデール!{#3}@{#4}}}

\def\srcMatelote#1#2#3#4{%
  \index[src]{matelote@Matelote!{#1}@{#2}}%
  \index[src]{まとろつと@マトロット!{#3}@{#4}}}

\def\srcMoelle#1#2#3#4{%
  \index[src]{moelle@Moelle!{#1}@{#2}}%
  \index[src]{もわる@モワル!{#3}@{#4}}}

\def\srcMoscovite#1#2#3#4{%
  \index[src]{moscovite@Moscovite!{#1}@{#2}}%
  \index[src]{もすくわふう@モスクワ風!{#3}@{#4}}}

\def\srcPerigueux#1#2#3#4{%
  \index[src]{perigueux@Périgueux!{#1}@{#2}}%
  \index[src]{へりくー@ペリグー!{#3}@{#4}}}

\def\srcPerigourdine#1#2#3#4{%
  \index[src]{perigourdine@Périgourdine!{#1}@{#2}}%
  \index[src]{へりくるていーぬ@ペリグルディーヌ!{#3}@{#4}}}

\def\srcPiquante#1#2#3#4{%
  \index[src]{piquante@Piquante!{#1}@{#2}}%
  \index[src]{ひかんと@ピカント!{#3}@{#4}}}

\def\srcPoivradeOrdinaire#1#2#3#4{%
  \index[src]{poivrade ordinaire@Poivrade ordinaire!{#1}@{#2}}%
  \index[src]{ほわふらーとひようしゆん@ポワヴラード(標準)!{#3}@{#4}}}

\def\srcPoivradeGibier#1#2#3#4{%
  \index[src]{poivrade gibier@Poivrade pour Gibier!{#1}@{#2}}%
  \index[src]{ほわふらーとしひえ@ポワヴラード(ジビエ用)!{#3}@{#4}}}

\def\srcPorto#1#2#3#4{%
  \index[src]{porto@Porto!{#1}@{#2}}%
  \index[src]{ほると@ポルト!{#3}@{#4}}}

\def\srcPortugaise#1#2#3#4{%
  \index[src]{portugaise@Portugaise!{#1}@{#2}}
  \index[src]{ほるとかるふう@ポルトガル風!{#3}@{#4}}}

\def\srcProvencale#1#2#3#4{%
  \index[src]{provencale@Provençale!{#1}@{#2}}
  \index[src]{ふろうあんすふう@プロヴァンス風!{#3}@{#4}}}

\def\srcRegence#1#2#3#4{%
  \index[src]{regence@Régence!{#1}@{#2}}
  \index[src]{れしやんす@レジャンス!{#3}@{#4}}}

\def\srcRobert#1#2#3#4{%
  \index[src]{robert@Robert!{#1}@{#2}}
  \index[src]{ろへーる@ロベール!{#3}@{#4}}}

\def\srcRobertEscoffier#1#2#3#4{%
  \index[src]{robert escoffier@Robert Escoffier!{#1}@{#2}}
  \index[src]{ろへーるえすこふいえ@ロベール・エスコフィエ!{#3}@{#4}}}

\def\srcRomaine#1#2#3#4{%
  \index[src]{romaine@Romaine!{#1}@{#2}}
  \index[src]{ろーまふう@ローマ風!{#3}@{#4}}}

\def\srcRouennaise#1#2#3#4{%
  \index[src]{rouennaise@Rouennaise!{#1}@{#2}}
  \index[src]{るーあんふう@ルーアン風!{#3}@{#4}}}

\def\srcSalmis#1#2#3#4{%
  \index[src]{salmis@Salmis!{#1}@{#2}}
  \index[src]{さるみ@サルミ!{#3}@{#4}}}

\def\srcTortue#1#2#3#4{%
  \index[src]{tortue@Tortue!{#1}@{#2}}
  \index[src]{とるちゆ@トルチュ!{#3}@{#4}}}

\def\srcVenaison#1#2#3#4{%
  \index[src]{venaison@Venaison!{#1}@{#2}}
  \index[src]{うねそん@ヴネゾン!{#3}@{#4}}}

\def\srcVinRouge#1#2#3#4{%
  \index[src]{vin rouge@Vin rouge (au)!{#1}@{#2}}
  \index[src]{あかわいん@赤ワイン!{#3}@{#4}}}


\def\srcZingaraA#1#2#3#4{%
  \index[src]{zingara a@Zingara A!{#1}@{#2}}
  \index[src]{さんからa@ザンガラ A!{#3}@{#4}}}

\def\srcZingaraB#1#2#3#4{%
  \index[src]{zingara b@Zingara B!{#1}@{#2}}
  \index[src]{さんからb@ザンガラ B!{#3}@{#4}}}


%%%%%%% ホワイト系の派生ソース

\def\srcAlbufera#1#2#3#4{%
  \index[src]{albufera@Albuféra!{#1}@{#2}}
  \index[src]{あるひゆふえら@アルビュフェラ!{#3}@{#4}}}

\def\srcAmericaine#1#2#3#4{%
  \index[src]{americaine@Américaien!{#1}@{#2}}
  \index[src]{あめりけーぬ@アメリケーヌ!{#3}@{#4}}}

\def\srcAnchois#1#2#3#4{%
  \index[src]{anchois@Anchois!{#1}@{#2}}%
  \index[src]{あんちよひ@アンチョビ!{#3}@{#4}}}

\def\srcAurore#1#2#3#4{%
  \index[src]{aurore@Aurore!{#1}@{#2}}%
  \index[src]{おーろーる@オーロール!{#3}@{#4}}}

\def\srcAuroreMaigre#1#2#3#4{%
  \index[src]{aurore maigre@Aurore maigre!{#1}@{#2}}%
  \index[src]{おーろーるさかな@オーロール(魚料理用)!{#3}@{#4}}}

\def\srcBavaroise#1#2#3#4{%
  \index[src]{bavaroise@Bavaroise!{#1}@{#2}}%
  \index[src]{はいえるんふう@バイエルン風!{#3}@{#4}}}

\def\srcBearnaise#1#2#3#4{%
  \index[src]{bearnaise@Béarnaise!{#1}@{#2}}%
  \index[src]{へあるねーす@ベアルネーズ!{#3}@{#4}}}

\def\srcBearnaiseTomatee#1#2#3#4{%
  \index[src]{bearnaise tomatee@Béarnaise tomatée!{#1}@{#2}}%
  \index[src]{へあるねーすとまと@ベアルネース(トマト入り)!{#3}@{#4}}}

\def\srcChoron#1#2#3#4{%
  \index[src]{choron@Choron!{#1}@{#2}}%
  \index[src]{しよろん@ショロン!{#3}@{#4}}}

\def\srcBearnaiseGlaceDeViande#1#2#3#4{%
  \index[src]{bearnaise glace de viande@Bearnaise à la glace de viande!{#1}@{#2}}%
  \index[src]{へあるねーすくらすとういあんと@ベアルネーズ(グラスドヴィアンド入り)!{#3}@{#4}}}

\def\srcFoyot#1#2#3#4{%
  \index[src]{foyot@Foyot!{#1}@{#2}}%
  \index[src]{ふおいよ@フォイヨ!{#3}@{#4}}}

\def\srcValois#1#2#3#4{%
  \index[src]{valois@Valois!{#1}@{#2}}%
  \index[src]{うあろわ@ヴァロワ!{#3}@{#4}}}

\def\srcBercy#1#2#3#4{%
  \index[src]{bercy@Bercy!{#1}@{#2}}%
  \index[src]{へるしー@ベルシー!{#3}@{#4}}}

\def\srcBeurre#1#2#3#4{%
  \index[src]{beurre@Beurre (au)!{#1}@{#2}}%
  \index[src]{ふーる@オ・ブール!{#3}@{#4}}}

\def\srcBatarde#1#2#3#4{%
  \index[src]{batarde@Batarde!{#1}@{#2}}%
  \index[src]{はたると@バタルド!{#3}@{#4}}}

\def\srcBonnefoy#1#2#3#4{%
  \index[src]{bonnefoy@Bonnefoy!{#1}@{#2}}%
  \index[src]{ほぬふおわ@ボヌフォワ!{#3}@{#4}}}

\def\srcBordelaiseVinBlanc#1#2#3#4{%
  \index[src]{bordelaise vin blanc@Bordelaise au vin blanc!{#1}@{#2}}%
  \index[src]{ほるとーふうしろわいん@ボルドー風(白ワイン)!{#3}@{#4}}}

\def\srcBretonneBlanche#1#2#3#4{%
  \index[src]{bretonne blanche@Bretonne (blanche)!{#1}@{#2}}%
  \index[src]{ふるたーにゆふうしろ@ブルターニュ風(ホワイト系)!{#3}@{#4}}}

\def\srcCanotiere#1#2#3#4{%
  \index[src]{canotiere@Canotière!{#1}@{#2}}%
  \index[src]{かのていえーる@カノティエール!{#3}@{#4}}}

\def\srcCapres#1#2#3#4{%
  \index[src]{capres@Câpres (aux)!{#1}@{#2}}%
  \index[src]{けいはー@ケイパー!{#3}@{#4}}}

\def\srcCardinal#1#2#3#4{%
  \index[src]{cardinl@Cardinal!{#1}@{#2}}%
  \index[src]{かるていなる@カルディナル!{#3}@{#4}}}

\def\src#1#2#3#4{%
  \index[src]{champignons blanche@Champignons (aux)(blanche)!{#1}@{#2}}%
  \index[src]{まつしゆるーむしろ@マッシュルーム(ホワイト系)!{#3}@{#4}}}

\def\srcChantilly#1#2#3#4{%
  \index[src]{chantilly@Chantilly!{#1}@{#2}}%
  \index[src]{しやんていい@シャンティイ!{#3}@{#4}}}

\def\srcChateaubriand#1#2#3#4{%
  \index[src]{chateaubriand@Chateaubriand!{#1}@{#2}}
  \index[src]{しやとーふりやん@シャトーブリヤン!{#3}@{#4}}}

\def\srcChaudFroidBlancheOrdinaire#1#2#3#4{%
  \index[src]{choud-froid blanche ordinaire@Chaud-froid blanche ordinaire!{#1}@{#2}}
  \index[src]{しよふろわしろひようしゆん@ショフロワ(白)(標準)!{#3}@{#4}}}

\def\srcChaudFroidBlonde#1#2#3#4{%
  \index[src]{choud-froid blonde@Chaud-froid blonde!{#1}@{#2}}%
  \index[src]{しよふろわふろんと@ショフロワ(ブロンド)!{#3}@{#4}}}

\def\srcChaudFroidAurore#1#2#3#4{%
  \index[src]{chaud-froid aurore@Chaud-froid Aurore!{#1}@{#2}}%
  \index[src]{しよふろわおーろーる@ショフロワ・オーロール!{#3}@{#4}}}

\def\srcChaudFroidVertPre#1#2#3#4{%
  \index{chaud-froid vert-pre@Chaud-froid Vert-pré!{#1}@{#2}}%
  \index[src]{しよふろわうえーるふれ@ショフロワ・ヴェールプレ!{#3}@{#4}}}

\def\srcChaudFroidMaigre#1#2#3#4{%
  \index[src]{chaud-froid maigre@Chaud-froid maigre!{#1}@{#2}}%
  \index[src]{しよふろわさかな@ショフロワ(魚料理用)!{#3}@{#4}}}

\def\srcChivry#1#2#3#4{%
  \index[src]{chivry@Chivry!{#1}@{#2}}%
  \index[src]{しうり@シヴリ!{#3}@{#4}}}

\def\srcCreme#1#2#3#4{%
  \index[src]{creme@Crème (à la)!{#1}@{#2}}%
  \index[src]{くれーむ@クレーム!{#3}@{#4}}}

\def\srcCrevettes#1#2#3#4{%
  \index[src]{crevettes@Crevettes (aux)!{#1}@{#2}}%
  \index[src]{くるうえつと@クルヴェット!{#3}@{#4}}}

\def\srcCurrie#1#2#3#4{%
  \index[src]{currie@Currie!{#1}@{#2}}%
  \index[src]{かれー@カレー!{#3}@{#4}}}

\def\srcCurrieIndienne#1#2#3#4{%
  \index[src]{currie indienne@Currie à l'Indienne!{#1}@{#2}}%
  \index[src]{いんとふうかれー@インド風カレー!{#3}@{#4}}}

\def\srcDiplomate#1#2#3#4{%
  \index[src]{diplomate@Diplomate!{#1}@{#2}}%
  \index[src]{ていふろまつと@ディプロマット!{#3}@{#4}}}

\def\srcEcossaise#1#2#3#4{%
  \index[src]{ecossaise@Ecossaise!{#1}@{#2}}%
  \index[src]{すこつとらんとふう@スコットランド風!{#3}@{#4}}}

\def\srcEstragon#1#2#3#4{%
  \index[src]{estragon@Estragon!{#1}@{#2}}%
  \index[src]{えすとらこん@エストラゴン!{#3}@{#4}}}

\def\srcFinesHerbes#1#2#3#4{%
  \index[src]{fines herbes blanche@Fines herbes blanche (aux)!{#1}@{#2}}%
  \index[src]{こうそうしろ@香草(ホワイト系)!{#3}@{#4}}}

\def\srcGroseilles#1#2#3#4{%
  \index[src]{groseilles@Groseilles!{#1}@{#2}}%
  \index[src]{くろせいゆ@グロゼイユ!{#3}@{#4}}}

\def\srcHollandaise#1#2#3#4{%
  \index[src]{hollandaise@Hollandaise!{#1}@{#2}}%
  \index[src]{おらんてーす@オランデーズ!{#3}@{#4}}}

\def\srcHomard#1#2#3#4{%
  \index[src]{homard@Homard!{#1}@{#2}}%
  \index[src]{おまーる@オマール!{#3}@{#4}}}

\def\srcHongroise#1#2#3#4{%
  \index[src]{hongroise@Hongroise!{#1}@{#2}}%
  \index[src]{はんかりーふう@ハンガリー風!{#3}@{#4}}}

\def\srcHuitres#1#2#3#4{%
  \index[src]{huitres@Huîtres (aux)!{#1}@{#2}}%
  \index[src]{かきいり@牡蠣入り!{#3}@{#4}}}

\def\srcIndienne#1#2#3#4{%
  \index[src]{indienne@Indienne!{#1}@{#2}}%
  \index[src]{いんとふう@インド風!{#3}@{#4}}}

\def\srcIvoire#1#2#3#4{%
  \index[src]{ivoire@Ivoire!{#1}@{#2}}%
  \index[src]{いうおわーる@イヴォワール!{#3}@{#4}}}

\def\srcJoinville#1#2#3#4{%
  \index[src]{joinville@Joinville!{#1}@{#2}}%
  \index[src]{しよわんういる@ジョワンヴィル!{#3}@{#4}}}

\def\srcLaguipiere#1#2#3#4{%
  \index[src]{laguipiere@Laguipière!{#1}@{#2}}%
  \index[src]{らきひえーる@ラギピエール!{#3}@{#4}}}

\def\srcLivonienne#1#2#3#4{%
  \index[src]{livonienne@Livonienne!{#1}@{#2}}%
  \index[src]{りうおにあふう@リヴォニア風!{#3}@{#4}}}

\def\srcMaltaise#1#2#3#4{%
  \index[src]{maltaise@maltaise!{#1}@{#2}}%
  \index[src]{まるたふう@マルタ風!{#3}@{#4}}}

\def\srcMariniere#1#2#3#4{%
  \index[src]{mariniere@Marinière!{#1}@{#2}}%
  \index[src]{まりにえーる@マリニエール!{#3}@{#4}}}

\def\srcMateloteBlanche#1#2#3#4{%
  \index[src]{matelote blanche@Matelote blanche!{#1}@{#2}}%
  \index[src]{まとろつとしろ@マトロット(白)!{#3}@{#4}}}

\def\srcMornay#1#2#3#4{%
  \index[src]{mornay@Mornay!{#1}@{#2}}%
  \index[src]{もるねー@モルネー!{#3}@{#4}}}

\def\srcMousseline#1#2#3#4{%
  \index[src]{mousseline@Mousseline!{#1}@{#2}}%
  \index[src]{むすりーぬ@ムスリーヌ!{#3}@{#4}}}

\def\srcMousseuse#1#2#3#4{%
  \index[src]{mousseuse@Mousseuse!{#1}@{#2}}%
  \index[src]{むすーす@ムスーズ!{#3}@{#4}}}

\def\srcMoutarde#1#2#3#4{%
  \index[src]{moutarde@Moutarde!{#1}@{#2}}%
  \index[src]{むたると@ムタルド!{#3}@{#4}}}

\def\srcNantua#1#2#3#4{%
  \index[src]{nantua@Nantua!{#1}@{#2}}%
  \index[src]{なんちゆあ@ナンチュア!{#3}@{#4}}}

\def\srcNewBurgCru#1#2#3#4{%
  \index[src]{new-burg cru@New-burg avec le homard cru!{#1}@{#2}}%
  \index[src]{にゆーはーくいけ@ニューバーグ(活けオマール)!{#3}@{#4}}}

\def\srcNewBurgCuit#1#2#3#4{%
  \index[src]{new-burg cuit@New-burg avec le homard cuit!{#1}@{#2}}%
  \index[src]{にゆーはーくゆて@ニューバーグ(茹でオマール)!{#3}@{#4}}}

\def\srcNoisette#1#2#3#4{%
  \index[src]{noisette@Noisette!{#1}@{#2}}%
  \index[src]{のわせつと@ノワゼット!{#3}@{#4}}}

\def\srcNormande#1#2#3#4{%
  \index[src]{normande@Normande!{#1}@{#2}}%
  \index[src]{のるまんていふう@ノルマンディ風!{#3}@{#4}}}

\def\srcOrientale#1#2#3#4{%
  \index[src]{orientale@Orientale!{#1}@{#2}}%
  \index[src]{おりえんとふう@オリエント風!{#3}@{#4}}}

\def\srcPaloise#1#2#3#4{%
  \index[src]{paloise@Paloise!{#1}@{#2}}%
  \index[src]{ほーふう@ポー風!{#3}@{#4}}}  

\def\srcPoulette#1#2#3#4{%
  \index[src]{poulette@Poulette!{#1}@{#2}}%
  \index[src]{ふれつと@プレット!{#3}@{#4}}}

\def\srcRavigote#1#2#3#4{%
  \index[src]{ravigote@Ravigote!{#1}@{#2}}%
  \index[src]{らういこつと@ラヴィゴット!{#3}@{#4}}}

\def\srcRegencePoisson#1#2#3#4{%
  \index[src]{regence poisson@Régence pour Poissons!{#1}@{#2}}%
  \index[src]{れしやんすさかなよう@レジャンス(魚料理用)!{#3}@{#4}}}

\def\srcRegenceGarnituresVolaille#1#2#3#4{%
  \index[src]{regence garnitures volaille@Régence pour garnitures de Volaille!{#1}@{#2}}%
  \index[src]{れしやんすとりりようりのかるにちゆーるよう@レジャンス(鶏料理のガルニチュール用)!{#3}@{#4}}}

\def\srcRiche#1#2#3#4{%
  \index[src]{riche@Riche!{#1}@{#2}}%
  \index[src]{りつしゆ@リッシュ!{#3}@{#4}}}

\def\srcRubens#1#2#3#4{%
  \index[src]{rubens@Rubens!{#1}@{#2}}%
  \index[src]{るーへんす@ルーベンス!{#3}@{#4}}}

\def\srcSaintMalo#1#2#3#4{%
  \index[src]{saint-malo@Saint-Malo!{#1}@{#2}}%
  \index[src]{さんまろふう@サンマロ風!{#3}@{#4}}}

\def\srcSmitane#1#2#3#4{%
  \index[src]{smitane@Smitane!{#1}@{#2}}%
  \index[src]{すみたーぬ@スミターヌ!{#3}@{#4}}}

\def\srcSolferino#1#2#3#4{%
  \index[src]{solferino@Solférino!{#1}@{#2}}%
  \index[src]{そるふえりの@ソルフェリノ!{#3}@{#4}}}

\def\srcSoubise#1#2#3#4{%
  \index[src]{soubise@Soubise!{#1}@{#2}}
  \index[src]{すひーす@スビーズ!{#3}@{#4}}}

\def\srcSoubiseTomatee#1#2#3#4{%
  \index[src]{soubise tomatee@Soubise tomatée!{#1}@{#2}}%
  \index[src]{すひーすとまといり@スビース(トマト入り)!{#3}@{#4}}}

\def\srcSouchet#1#2#3#4{%
  \index[src]{souchet@Souchet!{#1}@{#2}}%
  \index[src]{すーしえ@スーシェ!{#3}@{#4}}}

\def\srcTyrolienne#1#2#3#4{%
  \index[src]{tyrolienne@Tyrolienne!{#1}@{#2}}%
  \index[src]{ちろるふう@チロル風!{#3}@{#4}}}

\def\srcTyrolienneAncienne#1#2#3#4{%
  \index[src]{tyroienne ancienne@!Tyrolienne à l'ancienne!{#1}@{#2}}%
  \index[src]{ちろるふうくらしつく@チロル風・クラシック!{#3}@{#4}}}

\def\srcValois#1#2#3#4{%
  \index[src]{valois@Valois!{#1}@{#2}}%
  \index[src]{うあろわ@ヴァロワ!{#3}@{#4}}}

\def\srcVenitienne#1#2#3#4{%
  \index[src]{venitienne@Vénitienne!{#1}@{#2}}%
  \index[src]{うえねついあふう@ヴェネツィア風!{#3}@{#4}}}

\def\srcVeron#1#2#3#4{%
  \index[src]{veron@Véron!{#1}@{#2}}%
  \index[src]{うえろん@ヴェロン!{#3}@{#4}}}

\def\srcVillageoise#1#2#3#4{%
  \index[src]{villageoise@Villageoise!{#1}@{#2}}%
  \index[src]{むらひとふう@村人風!{#3}@{#4}}}

\def\srcVilleroy#1#2#3#4{%
  \index[src]{villeroy@Villeroy!{#1}@{#2}}%
  \index[src]{ういるろわ@ヴィルロワ!{#3}@{#4}}}

\def\srcVilleroySoubisee#1#2#3#4{%
  \index[src]{villeroy soubisee@Villeroy soubisée!{#1}@{#2}}%
  \index[src]{ういるろわすひーすいり@ヴィルロワ(スビーズ入り)!{#3}@{#4}}}

\def\srcVilleroyTomatee#1#2#3#4{%
  \index[src]{villeroy tomatee@Villeroy tomatée!{#1}@{#2}}%
  \index[src]{ういるろわとまといり@ヴィルロワ(トマト入り)!{#3}@{#4}}}

\def\srcVinBlanc#1#2#3#4{%
  \index[src]{vin blanc@Vin blanc!{#1}@{#2}}%
  \index[src]{しろわいん@白ワイン!{#3}@{#4}}}

%%%%イギリス風ソース(温製)



\makeatother


%% レイアウト調整(A4Paper,14Q,twoside,ltjsbook.cls) 
%%
\setlength{\hoffset}{0\zw}
\setlength{\oddsidemargin}{0\zw}%タブレット前提の中央配置
\setlength{\evensidemargin}{\oddsidemargin}
% \setlength{\oddsidemargin}{1\zw}%製本時に右ページのみをオフセット
%\setlength{\evensidemargin}{0pt}%
\setlength{\fullwidth}{45\zw}
\setlength{\textwidth}{45\zw}%%ltjsclassesのみ有効
%\setlength{\fullwidth}{159mm}
%\setlength{\textwidth}{159mm}
\setlength{\marginparsep}{0pt}
\setlength{\marginparwidth}{0pt}
\setlength{\footskip}{0pt}
\setlength{\voffset}{-17mm}
\setlength{\textheight}{265mm}
\setlength{\parskip}{0pt}
\setlength{\parindent}{0pt}

\newcommand{\atoaki}{\vspace{1.25mm}}

%%分数の表記Obsolete
\usepackage{xfrac}
\let\frac\sfrac





\begin{document}
\maketitle

\begin{Main}

\hypertarget{grandes-sauces-de-base}{%
\section{基本ソース}\label{grandes-sauces-de-base}}

\begin{frsecenv}

Grandes Sauces de Base

\end{frsecenv}

\index{そーす@ソース!きほん@\textbf{基本---}|(}
\index{sauce@sauce!grandes@\textbf{Grandes ---s de Base}|(}

\end{Main}

\begin{recette}

\hypertarget{sauce-espagnole}{%
\subsubsection[ソース・エスパニョル]{\texorpdfstring{ソース・エスパニョル\footnote{本節冒頭では、ルーがスペインの料理人によってもたらされ、その結果としてソース・エスパニョルが作られるようになったと読める記述があるが、これはむしろ誤りと考えるべき。エスパニョル
  espagnol(e)は「スペイン(風)の」意だが、スペイン料理起源というわけでもない。スペインを想起させるトマトを使うから、あるいは、ソースが茶褐色なのがムーア系スペイン人を想起させるから、など定説はない。カレーム『19世紀フランス料理』第3巻に収められたソース・エスパニョルの作り方は、フォンをとるところから始まり4ページにわたって詳細なものとなっている(pp.8-11)。その中で、肉を入れた鍋に少量のブイヨンを注いで煮詰めることを繰り返す。ここまでは18世紀の料理書で一般的な手法であるが、その後に大量のブイヨンを注いだ後、いきなり強火にかけるのではなく、弱火で加熱していくやり方を「スペイン式の方法」と述べている。カレームにおいては、これがソースの名称の根拠のひとつになっていると考えていいだろう。もちろん、ソース・エスパニョルという名称のソースはカレーム以前からあり、1806年刊のヴィアール『帝国料理の本』にもカレームのレシピより簡単だが、ほぼ同様のものが基本ソースとして収録されている。また、それ以前にもソース・エスパニョルに類する名称のソースはあったが、たとえば1739年刊ムノン『新料理研究』第2巻にある「スペイン風ソース」はかなり趣きが異なる(コリアンダーひと把みを加えるのが特徴的)。同じ料理名でも時代や料理書の著者によってまったく違う料理になっていることは、食文化史において珍しいことではない。また、とりわけ料理名に地名、国名が冠されているものの中には根拠や由来のはっきりしないものも多い。いずれにしても、本書のソース・エスパニョルの源流は19世紀初頭のヴィアールあたりからと考えられる。ソース・エスパニョルは19世紀を通して普及し、茶色いソースの代表的な存在となった。こんにちでもフォンドヴォーをベースとしたソースは、ルーでとろみ付けこそしないが、仔牛の骨などから出るコラーゲンによるとろみを利用したもので、仕上がりの色合いや、ごく標準的ともいえる風味付けの方法などが引継がれ続けている調理現場も少なくない。もっとも、上述のように本書では「茶色いルー」を使うところに「エスパニョル」であることの理由を見い出そうとしていると解釈される。}}{ソース・エスパニョル}}\label{sauce-espagnole}}

\begin{frsubenv}

Sauce espagnole

\end{frsubenv}

\index{そーす@ソース!えすはによる@---・エスパニョル}
\index{えすはによる@エスパニョル!そーす@ソース・---}
\index{すへいんふう@スペイン風(エスパニョル)!そーすえすはによる@ソース・エスパニョル}
\index{sauce@sauce!espagnole@--- Espagnole}
\index{espagnol@espagnol!sauce@Sauce ---e}
\index[src]{えすはによる@エスパニョル} \index[src]{espagnole@Espagnole}

(仕上がり5 L分)

\begin{itemize}
\item
  とろみ付けのための\protect\hyperlink{roux-brun}{ルー}\ldots{}\ldots{}625
  g。
\item
  \protect\hyperlink{fonds-brun}{茶色いフォン}(ソースを仕上げるのに必要な全量)\ldots{}\ldots{}12
  L。
\item
  \protect\hyperlink{mirepoix}{ミルポワ}\footnote{mirepoix
    (ミルポワ)。ソースやフォンにコクを与える目的で、細かいさいの目に切った香味野菜や塩漬け豚ばら肉を合わせたもの。18世紀にミルポワ公爵の料理人が考案したといわれているが真偽は不明。同様のものにmatignon(マティニョン)がある。ミルポワより大きめのさいの目に切るのが一般的とされるが、調理現場によってはあまり区別せずミルポワとのみ呼称するケースも多い。第2章ガルニチュール、\protect\hyperlink{mirepoix}{ミルポワ}訳注参照。}(香味素材)\ldots{}\ldots{}小さなさいの目に切った塩漬け豚ばら肉150
  g、2 mm程度のさいの目\footnote{brunoise (ブリュノワーズ)。1〜2 mm
    のさいの目に切ること。 couper en
    mirepoix(クゥペオンミルポワ)ミルポワに切るとも言う。}に切ったにんじん250
  gと玉ねぎ 150
  g、タイム2枝、ローリエの葉2枚。\index{みるほわ@ミルポワ}\index{mirepoix}
\item
  作業手順
\end{itemize}

\begin{enumerate}
\def\labelenumi{\arabic{enumi}.}
\item
  フォン8
  Lを鍋で沸かす。あらかじめ柔らかくしておいたルーを加え、木杓子か泡立て器で混ぜながら沸騰させる。

  弱火にして\footnote{原文から直訳すると「鍋を火の脇に置く」だが、現代の調理環境では単純に「弱火にする」と解釈していい。}微沸騰の状態を保つ。
\item
  以下のようにしてあらかじめ用意しておいたミルポワを投入する。ソテー鍋に塩漬け豚ばら肉を入れて火にかけて脂を溶かす。そこに、細かく刻んだにんじんと玉ねぎ、タイム、ローリエの葉を加える。野菜が軽く色づくまで強火で炒める。丁寧に、余分な脂を捨てる。これをソースに加える。野菜を炒めたソテー鍋に白ワイン約100
  mL\footnote{原文 un verre de vin blanc
    (アンヴェールドヴァンブロン)。直訳すると「グラス1杯の白ワイン」だが、本書において
    un verre de 〜は「約1 dL=100 mL」と覚えておくといいだろう。}を加えてデグラセ
  \footnote{dégrasser
    鍋に粘液状になって付着している肉汁を酒類あるいは水で溶かし出してソースなどに利用すること。}し、それを半量まで煮詰める。これも同様にソースの鍋に加える。こまめに浮いてくる夾雑物を徹底的に取り除き\footnote{dépouiller
    デプイエ。前節「ルーの火入れについて」訳注参照。}ながら弱火で約1
  時間煮込む。
\item
  ソースをシノワ\footnote{小さな穴が多く空けられた円錐形で、取っ手の付いた漉し器の一種。金属製のものが主流。}で、ミルポワ野菜を軽く押しながら漉し、別の片手鍋に移す。フォン2
  Lを注ぎ足す。さらに2時間、微沸騰の状態を保ちながら煮込む。その後、陶製の鍋に移し、ゆっくり混ぜながら冷ます。
\item
  翌日、再び厚手の片手鍋に移してから、フォン2 Lとトマトピュレ1
  Lまたは同等の生のトマトつまり2
  kgを加える。トマトピュレを用いる場合は、あらかじめオーブンでほとんど茶色になるまで焼いておくといい。そうするとトマトピュレの酸味を抜くことが出来る。そうすればソースを澄ませる作業が楽になるし、ソースの色合いも温かそうで美しいものになる。ソースをヘラか泡立て器で混ぜながら強火で沸騰させる。弱火にして1時間微沸騰の状態を保つ。最後に、表面に浮いている不純物を、細心の注意を払いながら徹底的に取り除く。布で漉し、完全に冷めるまで、ゆっくり混ぜ続けること\footnote{この、ヘラなどでゆっくり混ぜながら冷ます作業を
    vanner
    (ヴァネ)すると呼ぶが、日本の調理現場ではあまり用いられていない。}。
\end{enumerate}

\hypertarget{nota-sauce-espagnole}{%
\subparagraph{【原注】}\label{nota-sauce-espagnole}}

ソース・エスパニョルで仕上げに不純物を取り除くのにかかる時間はいちがいには言えない。これは、ソースに用いるフォンの質次第で変わるからだ。

ソースにするフォンが上質なものであればある程、仕上げに不純物を取り除く作業は早く済む。そういう場合には、ソース・エスパニョルを5時間で作ることも無理ではない。

\atoaki{}

\hypertarget{sauce-espagnole-maigre}{%
\subsubsection[魚料理用ソース・エスパニョル]{\texorpdfstring{魚料理用ソース・エスパニョル\footnote{フランス語のソース名にあるmaigre(メーグル)はこの場合、一般的に「魚用、魚料理用」と訳すが、厳密には「小斉の際の料理用」となろう。小斉とは、カトリックで古くから特定の期間、曜日に肉類を断つ食事をする宗教的食習慣。日本の「お精進」とニュアンスは近いが、小斉においては忌避されるのは鳥獣肉のみであり、魚介や乳製品はいいとされた。こじつけのように、水鳥は水のものだから魚介扱いであり、またイルカも魚類として扱われていた。小斉が行なわれるのは復活祭の前46日間(四旬節、逆に言えばカーニバルの最終日マルディグラの翌日から46日)と、週に一度(多くの場合は金曜)であった。合計すると小斉が行なわれるのは年間100日近くもあり、中世から18世紀の料理人たちは小斉の宴席に供する料理に工夫を凝らしていた。この習慣は19世紀になるとだんだん廃れていき、エスコフィエの時代には、料理人に対して小斉のための料理を要求することは少なくなっていった。}}{魚料理用ソース・エスパニョル}}\label{sauce-espagnole-maigre}}

\begin{frsubenv}

Sauce espagnole maigre

\end{frsubenv}

\index{そーす@ソース!えすはによるるさかな@---・エスパニョル (魚料理用)}
\index{えすはによる@エスパニョル!そーすさかなよう@ソース・--- (魚料理用)}
\index{すへいんふう@スペイン風(エスパニョル)!そーすえすはによるさかな@ソース・エスパニョル(魚料理用)}
\index{sauce@sauce!espagnole maigre@--- Espagnole maigre}
\index{espagnol@espagnol!sauce maigre@Sauce Espagnole maigre}
\index[src]{さかなりようりようえすはによる@魚料理用エスパニョル}
\index[src]{espagnole maigre@Espagnole maigre}

(仕上がり5 L分)

\begin{itemize}
\item
  バターを用いて\footnote{初版〜第三版にかけては、茶色いルーを作るのに「バターまたは、きれいなグレスドマルミット(コンソメなどを作る際に表面に浮いてくる脂をすくい取って、不純物を漉し取ったものであり、基本的に獣脂)」を用いる、とある。上述のように、カトリックにおける「小斉」の場合、獣脂は忌避されたがバターなどの乳製品は許容された。そのため特に「バターを用いて作ったルー」という指定がなされ、第四版では茶色いルーに澄ましバターのみを使う旨が強調されたが、ここでは初版以来の記述がそのまま残っているために、やや冗長に思われる表現となっている。}作った\protect\hyperlink{roux-brun}{ルー}\ldots{}\ldots{}500
  g。
\item
  \protect\hyperlink{fumet-de-poisson}{魚のフュメ(フュメドポワソン)}(ソースを仕上げるために必要な全量)\ldots{}\ldots{}10
  L。
\item
  ミルポワ\ldots{}\ldots{}標準的なソース・エスパニョルと同じ\protect\hyperlink{mirepoix}{ミルポワ}野菜を同量と、塩漬け豚ばら肉の代わりにバターを用い、マッシュルームまたはマッシュルームの切りくず\footnote{champignons
    de Paris
    (シャンピニョンドパリ)いわゆるマッシュルームは、ガルニチュールなど料理の一部として提供する際に、トゥルネ
    tourner
    といって螺旋(らせん)状の切れ込みを入れて装飾したものを使う。その際に少なくない量、具体的には重量で15〜20%程度が「切りくず」として発生するのでこれを利用する。なお、tourner(トゥルネ)の原義は「回す」であり、包丁を持った側の手は動かさずに、材料のほうを回すようにして切れ目を入れたり、アーティチョークや果物などの皮を剥くことを意味する。}250
  gを加える。
\item
  作業手順\ldots{}\ldots{}標準的なソース・エスパニョルとまったく同様に作る。
\item
  加熱時間と不純物を取り除くのに必要な時間\ldots{}\ldots{}5時間。
\end{itemize}

仕上げに漉してから、標準的なソース・エスパニョルとまったく同様に、完全に冷めるまでゆっくり混ぜ続けること。

\atoaki{}

\hypertarget{observation-sauce-espagnole-maigre}{%
\subsubsection{魚料理用ソース・エスパニョル補足}\label{observation-sauce-espagnole-maigre}}

このソースを日常的な料理のベースとなる仕込みに含めるかどうかについては意見が分れるところだ。

普通のソース・エスパニョルは、つまるところ風味の点ではほとんどニュートラルなものだから、それに魚のフュメを加えれば、魚料理用ソース・エスパニョルとして充分に通用するだろう。どうしても上で挙げた魚料理用ソース・エスパニョルが必要になるのは、宗教的に厳格に小斉の決まりを守って料理を作る場合のみで、さすがにその場合は代用品などない。

\atoaki{}

\hypertarget{sauce-demi-glace}{%
\subsubsection[ソース・ドゥミグラス]{\texorpdfstring{ソース・ドゥミグラス\footnote{日本の洋食などで一般的な「デミグラス」あるいは「ドミグラス」」とはかなり異なった仕上りのソースであることに注意。ソース・エスパニョルの仕上げにあたって、徹底的に不純物を取り除くことを何度も強調しているのは、透き通った茶色がかった色合いの、なめらかなソースを目指すからであり、それをさらに徹底させるということは、透明度、なめらかさの面でさらに上を目指すということを意味するからだ。ちなみに、アメリカに本社のあるメーカーの「デミグラスソース」の缶詰はもっぱら日本で販売されている製品であり、ヨーロッパおよびアメリカでは同一ブランドに該当する商品は存在しないようだ。}}{ソース・ドゥミグラス}}\label{sauce-demi-glace}}

\begin{frsubenv}

Sauce demi-glace

\end{frsubenv}

\index{そーす@ソース!とうみくらす@---・ドゥミグラス}
\index{とうみくらす@ドゥミグラス!そーす@ソース・---}
\index{sauce@sauce!demi-glace@--- Demi-glace}
\index{demi-glace@demi-glace!sauce@sauce ---}
\index[src]{demi-glace@Demi-glace}
\index[src]{とうみくらす@ドゥミグラス}

一般に「ドゥミグラス」と呼ばれているものは、いったん仕上がった\protect\hyperlink{sauce-espagnole}{ソース・エスパニョル}をさらに、もうこれ以上は無理という位に徹底的に不純物を取り除いたもののことだ。

最後の仕上げに\protect\hyperlink{glace-de-viande}{グラスドヴィアンド}などを加える。風味付けに何らかの酒類\footnote{本書ではマデイラ酒(マデイラワイン、ポルトガルの酒精強化ワイン、すなわちブドウ果汁が酵母により醗酵している途中で蒸留酒を加えて醗酵を止める製法のもので、甘口のものが多い)が用いられることが多い。}を加えれば、当然ながらソースの性格も変わるので、最終的な使い途に応じて決めること。

\hypertarget{nota-sauce-demi-glace}{%
\subsubsection{【原注】}\label{nota-sauce-demi-glace}}

ソースの色合いを決めるワインを仕上げに加える際には、「火から外して」行なうこと。沸騰しているとワインの香りがとんでしまうからだ。

\atoaki{}

\hypertarget{jus-de-veau-lie}{%
\subsubsection{とろみを付けた仔牛のジュ}\label{jus-de-veau-lie}}

\begin{frsubenv}

Jus de veau lié

\end{frsubenv}

\index{しゆ@ジュ!こうしのしゆ@仔牛の---(とろみを付けた)}
\index{そーす@ソース!とろみをつけたこうしのしゆ@とろみを付けた仔牛のジュ}
\index{こうし@仔牛!とろみをつけたこうしのしゆ@とろみを付けた---のジュ}
\index{jus@jus!jus veau lie@--- de veau lié}
\index{veau@veau!jus lie@jus de --- lié}
\index{jus de veau lie@Jus de veau lié}
\index[src]{とろみをつけたこうしのしゆ@とろみを付けた仔牛のジュ}

(仕上がり1 L分)

\begin{itemize}
\item
  仔牛のフォン\ldots{}\ldots{}\protect\hyperlink{jus-de-veau-brun}{仔牛の茶色いフォン}
  4 L。
\item
  とろみ付け材料\ldots{}\ldots{}アロールート\footnote{allow-root
    南米産のクズウコンを原料とした良質のでんぷん。日本では入手が難しいこともあり、コーンスターチが用いられることがほとんど。}30
  g。
\item
  作業手順\ldots{}\ldots{}よく澄んだ仔牛のフォン4
  Lを強火にかけ、\(\frac{1}{4}\) 量つまり1 Lになるまで煮詰める。
\end{itemize}

大さじ数杯分の冷たいフォンでアロールートを溶く。これを沸騰している鍋に加える。1分程度だけ火にかけ続けたら、布で漉す。

\hypertarget{nota-jus-de-veau-lie}{%
\subparagraph{【原注】}\label{nota-jus-de-veau-lie}}

この、とろみを付けた仔牛のジュは、本書では頻繁に使う指示をしているが、必ず、しっかりした味で透き通った、きれいな薄茶色に仕上げること。

\atoaki{}

\hypertarget{veloute}{%
\subsubsection[ヴルテ(標準的な白いソース)]{\texorpdfstring{ヴルテ\footnote{velouté
  (ヴルテ)原義は「ビロードのように柔らかな、なめらかな」。日本ではベシャメルソースと混同されやすいが、内容がまったく異なるソースなので注意。}(標準的な白いソース)}{ヴルテ(標準的な白いソース)}}\label{veloute}}

\begin{frsubenv}

Velouté, ou sauce blanche graisse

\end{frsubenv}

\index{うるて@ヴルテ!ひようひゆんてきなそーすうるて@標準的なソース・---}
\index{そーす@ソース!うるてひようひゆん@ヴルテ(標準的な)}
\index{ふるーて@ブルーテ ⇒ ヴルテ} \index{veloute@velouté}
\index{veloute@velouté!sauce blanche grasse@sauce blanche grasse}
\index[src]{veloute@Velouté} \index[src]{うるて@ヴルテ}

(仕上がり5 L分)

\begin{itemize}
\item
  とろみ付けの材料\ldots{}\ldots{}バターを用いて作った\footnote{\protect\hyperlink{sauce-espagnole-maigre}{魚料理用ソース・エスパニョル}、訳注参照。}\protect\hyperlink{roux-blond}{ブロンドのルー}
  625 g。
\item
  よく澄んだ\protect\hyperlink{fonds-blanc-ordinaire}{仔牛の白いフォン}\ldots{}\ldots{}5
  L。
\item
  作業手順\ldots{}\ldots{}ルーをフォンに溶かし込む。フォンは冷たくても熱くてもいいが、フォンが熱い場合にはソースが充分なめらかになるよう注意して溶かすこと。混ぜながら沸騰させる。微沸騰の状態を保ちながら、浮いてくる不純物を完全に取り除いていく\footnote{dépouiller
    (デプイエ)。\protect\hyperlink{sauce-espagnole}{ソース・エスパニョル}、訳注参照。}。この作業はとりわけ細心の注意を払って行なうこと。
\item
  加熱時間と不純物を取り除く作業に必要な時間\ldots{}\ldots{}1時間半。
\end{itemize}

その後、ヴルテを布で漉す\footnote{ある程度濃度のある液体やピュレを布で漉す場合、昔は「二人がかりで行なう必要があり、それぞれが巻いた布の端を左手に持ち、右手に持った木杓子を使って圧し搾る」(『ラルース・ガストロノミーク』初版、
  1938年)という方法が一般的だった。}。陶製の鍋に移してゆっくり混ぜながら完全に冷ます。

\atoaki{}

\hypertarget{veloute-de-volaille}{%
\subsubsection{鶏のヴルテ}\label{veloute-de-volaille}}

\begin{frsubenv}

Velouté de volaille

\end{frsubenv}

\index{うるて@ヴルテ!とりのうるて@鶏の---(ヴルテドヴォライユ)}
\index{そーす@ソース!うるてとり@ヴルテ(鶏)}
\index{ふるーて@ブルーテ ⇒ ヴルテ}
\index{うおらいゆ@ヴォライユ!うるてとうおらいゆ@ヴルテドヴォライユ(鶏のヴルテ)}
\index{かきん@家禽!とりのうるて@鶏のヴルテ}
\index{veloute@velouté!volaille@--- de Volaille}
\index{sauce@sauce!veloute volaille@Velout\'e de Volaille}
\index[src]{veloute de volaille@Velouté de volaille}
\index[src]{とりのうるて@鶏のヴルテ}

このヴルテの作り方だが、上述の\protect\hyperlink{veloute}{標準的なヴルテ}と、材料比率と作業はまったく同じ。使用する液体として\protect\hyperlink{fonds-de-volaille}{鶏の白いフォン(フォンドヴォライユ)}を使う。

\atoaki{}

\hypertarget{veloute-de-poisson}{%
\subsubsection{魚料理用ヴルテ}\label{veloute-de-poisson}}

\begin{frsubenv}

Velouté de poisson

\end{frsubenv}

\index{うるて@ヴルテ!さかなうるて@魚料理用---}
\index{そーす@ソース!うるてさかな@ヴルテ(魚料理用)}
\index{veloute@velouté!poisson@--- de Poisson}
\index{sauce@sauce!veloute poisson@Velouté de Poisson}
\index[src]{veloute de poisson@Velouté de poisson}
\index[src]{さかなのうるて@魚のヴルテ}

ルーと液体の分量は標準的なヴルテとまったく同じだが、仔牛のフォンではなく\protect\hyperlink{fumet-de-poisson}{魚のフュメ}を用いて作る。

ただし、魚を素材として用いるストックはどれもそうだが、手早く作業すること。不純物を取り除く作業も20分程度にとどめること。その後、布で漉し、陶製の鍋に移してゆっくり混ぜながら完全に冷ます。

\atoaki{}

\hypertarget{sauce-allemande}{%
\subsubsection[ソース・アルマンド(パリ風ソース)]{\texorpdfstring{ソース・アルマンド(パリ風ソース\footnote{原書では「パリ風ソース(元ソース・アルマンド)」となっているが、後述のように、こんにちでもソース・アルマンドの名称のほうが一般的であるため、ここではSauce
  Parisienneの「訳語」としてソース・アルマンドをあてることとした。})}{ソース・アルマンド(パリ風ソース)}}\label{sauce-allemande}}

\begin{frsubenv}

Sauce parisienne (ex-Allemande)

\end{frsubenv}

\index{そーす@ソース!ぱりふう@パリ風--- ⇒ ---・アルマンド}
\index{はりふう@パリ風!そーす@---ソース ⇒ ---・アルマンド}
\index{といつふう@ドイツ風!そーす@ソース・アルマンド}
\index{あるまん@アルマン(ド)!そーす@ソース・アルマンド}
\index{sauce@sauce!parisienne@--- parisienne (ex-Allemande)}
\index{parisien@parisien!sauce@Sauce Parisienne = Sauce Allemande}
\index{allemand@allemand!sauce@Sauce allemande (--- Parisienne)}
\index[src]{allemande@Allemande} \index[src]{あるまんと@アルマンド}

(仕上がり1 L分)

これは、\protect\hyperlink{veloute}{標準的なヴルテ}に卵黄でとろみを付けたソース。

\begin{itemize}
\item
  標準的なヴルテ\ldots{}\ldots{}1 L。
\item
  追加素材\ldots{}\ldots{}卵黄5個、白いフォン(冷たいもの)
  \(\frac{1}{2}\)
  L、粗く砕いたこしょう1ひとつまみ、すりおろしたナツメグ少々、マッシュルームの煮汁2
  dL、レモン汁少々。
\item
  作業手順\ldots{}\ldots{}厚手のソテー鍋にマッシュルームの茹で汁と白いフォン、卵黄、粗く砕いたこしょう、ナツメグ、レモン汁を入れる。泡立て器でよく混ぜ、そこにヴルテを加える。火にかけて沸騰させ、強火で
  \$\frac{2}{3} 量になるまで、ヘラで混ぜながら煮詰める。
\end{itemize}

ヘラの表面がソースでコーティングされる状態になるまで煮詰めたら、布で漉す。

膜が張らないよう、表面にバターのかけらをいくつか載せてやり、湯煎にかけておく。

\begin{itemize}
\tightlist
\item
  仕上げ\ldots{}\ldots{}提供直前に、バター100 gを加えて仕上げる。
\end{itemize}

\hypertarget{nota-sauce-allemande}{%
\subparagraph{【原注】}\label{nota-sauce-allemande}}

ソース・アルマンド(ドイツ風)とも呼ばれるが、本書では「パリ風」の名称を採用した。そもそも「アルマンド」というの名称に正当性がないからだ。習慣としてそう呼ばれてきただけであって、明らかに理屈に合わない名称だ
\footnote{エスコフィエは普仏戦争に従軍した経歴があり、ドイツ嫌いとして知られていた。}。1883年に雑誌「料理技術」に某タヴェルネ氏が寄せた記事には、当時ある優秀な料理人がアルマンドなどという理屈に合わない名称を使うのはやめたという話が出ている。

こんにち既に「パリ風ソース」の名称を採用している料理長もいる。そう呼んだほうが好ましいわけだが、残念なことにまだ一般的にはなっていない
\footnote{エスコフィエの願いもむなしく、現代においてもソース・アルマンドの名称で定着している。この「全注解」においても以後は「ソース・アルマンド」と訳しているので注意されたい。なお、「ドイツ風」というソース名の由来について、ソースの淡い黄色がドイツ人に多い金髪を想起させるからだとカレームは述べている。}。

\atoaki{}

\hypertarget{sauce-supreme}{%
\subsubsection[ソース・シュプレーム]{\texorpdfstring{ソース・シュプレーム\footnote{suprême
  原義は「至高の」だが、料理においてはしばしば鶏や鴨の胸肉、白身魚のフィレなどを意味する。また、このソースのように、とくに意味もなくこの名を料理につけられているケースも多い。}}{ソース・シュプレーム}}\label{sauce-supreme}}

\begin{frsubenv}

Sauce supême

\end{frsubenv}

\index{そーす@ソース!そーすしゆふれーむ@---・シュプレーム}
\index{しゆふれーむ@シュプレーム!そーす@ソース・---}
\index{sauce@sauce!supreme@--- Suprême}
\index{supreme@suprême!sauce@Sauce ---} \index[src]{supreme@Suprême}
\index[src]{しゆふれーむ@シュプレーム}

\protect\hyperlink{veloute-de-volaille}{鶏のヴルテ}に生クリーム\footnote{フランスの生クリームのうち、料理でよく使われるのは、日本の生クリームにやや近い「クレーム・フレッシュ・パストゥリゼ」(低温殺菌した生クリームで乳脂肪分30〜38%)のほか、「クレーム・フレッシュ・エペス」(低温殺菌後に乳酸醗酵させたもので日本で一般的な生クリームより濃度がある)、「クレーム・ドゥーブル」(殺菌後に乳酸醗酵させたもので乳脂肪分40%程度でかなり濃度がある)などがある。}を加えてなめらかに仕上げ\footnote{monter
  モンテ。原義は「上げる、ホイップする」だが、ソースの仕上げの際などに、バターや生クリームを加えて、なめらかに仕上げることも「モンテ」の語を使用する場合が多い。}たもの。ソース・シュプレームは、正しく作った場合「白さの\ruby{際}{きわ}だったとても繊細な」仕上がりのものでなくてはいけない。

(仕上がり1 L分)

\begin{itemize}
\item
  鶏のヴルテ\ldots{}\ldots{}1 L。
\item
  追加素材\ldots{}\ldots{}\protect\hyperlink{fonds-de-volaille}{鶏の白いフォン}
  1 L、マッシュルームの茹で汁1 dL、良質な生クリーム2 \(\frac{1}{2}\)
  dL。
\item
  作業手順\ldots{}\ldots{}鍋に鶏のフォンとマッシュルームの茹で汁、鶏のヴルテを入れて強火にかけ、ヘラで混ぜながら、生クリームを少しずつ加え、煮詰めていく。このヴルテと生クリームを煮詰めたものの分量は、上で示した仕上がり
  1 Lのソース・シュプレームを作るには、 \(\frac{1}{3}\)
  量まで煮詰まっていなくてはならない。
\end{itemize}

布で漉し、仕上げに1 dLの生クリームとバター80
gを加えてゆっくり混ぜながら冷ますと、丁度最初のヴルテと同量になる。

\atoaki{}

\hypertarget{sauce-bechamel}{%
\subsubsection[ベシャメルソース]{\texorpdfstring{ベシャメルソース\footnote{17世紀にルイ14世のメートルドテルを務めたこともあるルイ・ベシャメイユLouis
  Béchameil(1630〜1703)の名が冠されているこのソースは、彼自身の創案あるいは彼に仕えていた料理人によるものという説もあったが真偽は疑わしい。17世紀頃の成立であることは確かだが、おそらくは古くからあったソースを改良したものに過ぎず、また、19世紀前半のカレームのレシピはヴルテを煮詰め、卵黄と煮詰めた生クリームでとろみを付けるというものだった。同様に1867年刊グフェ『料理の本』のレシピも、炒めた仔牛肉と野菜に小麦粉を振りかけてからブイヨン注ぎ、これを煮詰め、漉してから生クリームを加えるというものだった。}}{ベシャメルソース}}\label{sauce-bechamel}}

\begin{frsubenv}

Sauce Béchamel

\end{frsubenv}

\index{そーす@ソース!へしやめる@ベシャメル---}
\index{へしやめる@ベシャメル!そーす@---ソース}
\index{sauce@sauce!bechamel@--- Béchamel}
\index{bechamel@Béchamel (sauce)} \index[src]{bechamel@Bechamel}
\index[src]{へしやめる@ベシャメル}

(仕上がり 5 L分)

\begin{itemize}
\item
  \protect\hyperlink{roux-blanc}{白いルー}\ldots{}\ldots{}650 g。
\item
  使用する液体\ldots{}\ldots{}沸かした牛乳5 L。
\item
  追加素材\ldots{}\ldots{}白身で脂肪のない仔牛肉300
  gをさいの目に切り、みじん切りにした玉ねぎ(小)2個分とタイム1枝、粗く砕いたこしょう1つまみ、塩25
  g とバターを鍋に入れて蓋をし、色付かないように弱火で蒸し煮したもの。
\item
  作業手順\ldots{}\ldots{}沸かした牛乳でルーを溶く。混ぜながら沸騰させる。ここに、先に蒸し煮しておいた野菜と調味料、仔牛肉を加える。弱火で1時間煮込む。布で漉し\footnote{\protect\hyperlink{veloute}{ヴルテ}訳注参照。}、表面にバターのかけらをいくつか載せて膜が張らないようにする。肉類を絶対に使わない\footnote{小斉のこと。\protect\hyperlink{sauce-espagnole-maigre}{魚料理用ソース・エスパニョル}訳注参照。}で調理する必要がある場合は、仔牛肉を省き、香味野菜などは上記のとおりに作ること。
\end{itemize}

このソースは次のようなやり方をすると手早く作ることも出来る。沸かした牛乳に塩、薄切りにした玉ねぎ、タイム、粗く砕いたこしょう、ナツメグを加える。蓋をして弱火で10分煮る。これを漉してルーを入れた鍋の中に入れ、強火にかけて沸騰させる。その後15〜20分だけ煮込めばいい。

\atoaki{}

\hypertarget{sauce-tomate}{%
\subsubsection{トマトソース}\label{sauce-tomate}}

\begin{frsubenv}

Sauce tomate

\end{frsubenv}

\index{そーす@ソース!とまとそーす@トマト---}
\index{とまと@トマト!ソース@---ソース}
\index{sauce@sauce!tomate@--- tomate}
\index{tomate@tomate!sauce@Sauce ---} \index[src]{tomate@Tomate}
\index[src]{とまと@トマト}

(仕上がり5 L分)

\begin{itemize}
\item
  主素材\ldots{}\ldots{}トマトピュレ4 L、または生のトマト6 kg。
\item
  \protect\hyperlink{mirepoix}{ミルポワ}\ldots{}\ldots{}さいの目に切って下茹でしておいた塩漬け豚ばら肉140
  g 、1〜2 mm 角のさいの目に刻んだにんじん200 gと玉ねぎ150
  g、ローリエの葉 1枚、タイム1枝、バター100 g。
\item
  追加素材\ldots{}\ldots{}小麦粉150 g、白いフォン2 L、にんにく2片。
\item
  調味料\ldots{}\ldots{}塩20 g、砂糖30 g、こしょう1つまみ。
\item
  作業手順\ldots{}\ldots{}厚手の片手鍋で、塩漬け豚ばら肉を軽く色付くまで炒める。ミルポワの野菜を加え、野菜も色よく炒める。小麦粉を振りかける。ブロンド色になるまで炒めてから、トマトピュレまたは潰した生トマトと白いフォン、砕いたにんにく、塩、砂糖、こしょうを加える。
\end{itemize}

火にかけて混ぜながら沸騰させる。鍋に蓋をして弱火のオーブンに入れ1時間半〜2時間加熱する。

目の細かい漉し器または布で漉す。再度、火にかけて数分間沸騰させる。保存用の器に移し、ソースが空気に触れて表面に膜が張らないよう、バターのかけらを載せてやる。

\hypertarget{nota-sauce-tomate}{%
\subparagraph{【原注】}\label{nota-sauce-tomate}}

トマトピュレを使い、小麦粉は使わず、その他は上記のとおりに作ってもいい。漉し器か布で漉してから、充分な濃度になるまでしっかり煮詰めてやること。

\index{そーす@ソース!きほん@\textbf{基本---}|)}
\index{sauce@sauce!grandes@\textbf{Grandes ---s de Base}|)}

\end{recette}

\end{document}

\hypertarget{petites-sauces-brunes-composees}{%
\section{ブラウン系の派生ソース}\label{petites-sauces-brunes-composees}}

\frsec{Petites Sauces Brunes Composées}

\index{そーす@ソース!ふらうんはせい@ブラウン系の派生---}
\index{sauce@sauce!petites brunes composees@Petites ---s Brunes Composées}
\begin{recette}
\hypertarget{sauce-bigarade}{%
\subsubsection{ソース・ビガラード}\label{sauce-bigarade}}

\frsub{Sauce Bigarade}\footnote{ビガラードは本来、南フランスで栽培されるビターオレンジの一種。}

\index{そーす@ソース!ひからーと@---・ビガラード}
\index{そーす@ソース!ふらうんはせい@ブラウン系の派生---!ひからーと@---・ビガラード}
\index{ひからーと@ビガラード!そーす@ソース・---}
\index{sauce@sauce!petites brunes composees@Petites ---s Brunes Composées!bigarade@--- Bigarade}
\index{sauce@sauce!bigarade@--- Bigarade}
\index{bigarade@bigarade!sauce@Sauce ---}

\hypertarget{sauce-bigarade-pour-caneton-braise}{%
\subparagraph{仔鴨のブレゼ用}\label{sauce-bigarade-pour-caneton-braise}}

\ldots{}\ldots{}仔鴨をブレゼ\footnote{料理の仕立てとしてのブレゼはたんに「蒸し煮」することではない。原
  則的な手順をごく簡単に述べておく。厚めに輪切りにしたにんじんと玉ね
  ぎをバターまたはラードで炒め、ブーケガルニとともに鍋に入れる。表面
  を色よく焼き固めた肉を、脂身の少ない肉の場合には豚背脂のシートで包
  んで素材がぴったり入る大きさ鍋に入れ、\protect\hyperlink{fonds-brun}{茶色いフォン}
  を注ぎ、蓋をしてオーブンに入れ、微沸騰の状態を保つようにして煮込む。
  火が通ったら肉を取り出し、鍋に残った煮汁でソースを作る。詳細につい
  ては\protect\hyperlink{}{第7章 肉料理}参照。}した際の煮汁を漉してから浮き脂を取り除き\footnote{dégraisser
  デグレセ。}、煮詰める。 煮詰まったらさらに目の細かい布で漉し、ソース1
Lあたりオレンジ4個とレモ ン1個の搾り汁でのばす。

\hypertarget{sauce-bigarade-pour-caneton-poele}{%
\subparagraph{仔鴨のポワレ用}\label{sauce-bigarade-pour-caneton-poele}}

\ldots{}\ldots{}仔鴨をポワレ\footnote{ポワレについても簡単に述べておく。本書においてポワレは「フライパ
  ンで焼く」という意味で用いられることは決してない(フライパンで魚な
  どを焼くことをポワレと呼ぶようになったのは20世紀後半のこと)。本書
  では「ローストの一種」と定義されており(この点がカレームとはまった
  く異なる)、3〜4mm角に切った香味野菜(マティニョン)を生のまま鍋の
  底に入れ、その上に味付けをした肉を置く。溶かしバターをかけてから、
  蓋をして中火のオーブンに入れて蒸し焼きにする。時折様子を見て溶かし
  バターをかけてやること。肉に火が通ったら鍋から取り出し、\protect\hyperlink{jus-de-veau-brun}{茶色い仔
  牛のフォン}を注いで弱火にかけて10分程煮込み、
  マティニョンとして用いた野菜から風味を引き出してソースにする。これ
  がレシピにある「ポワレのフォン」となる。}のフォンから浮き脂を取り除き、でんぷんで軽くとろみ付け
する。砂糖20gに大さじ\undemi{}杯のヴィネガーを加えて火にかけカラメル状
にしたものを加える。ブレゼ用と同様に、オレンジとレモンの搾り汁でのばす。

仔鴨のブレゼ用、ポワレ用いずれの場合も、細かい千切りにしてよく下茹でし
ておいたオレンジの皮大さじ2とレモンの皮\footnote{柑橘類の表皮を薄く剥いてごく細い千切りにしたり、器具を用いてお
  ろしたものをzeste(ゼスト)と呼ぶ。千切りにしたものは苦味を取り除く
  ために下茹ですることが多い。}大さじ1を加えて仕上げる。

\hypertarget{sauce-bordelaise}{%
\subsubsection{ボルドー風ソース}\label{sauce-bordelaise}}

\frsub{Sauce Bordelaise}

\index{そーす@ソース!ほるとーふう@ボルドー風---}
\index{そーす@ソース!ふらうんはせい@ブラウン系の派生---!ほるとーふう@ボルドー風---}
\index{ほるとーふう@ボルドー風!そーす@---ソース}
\index{sauce@sauce!petites brunes composees@Petites ---s Brunes Composées!bordelaise@--- Bordelaise}
\index{sauce@sauce!bordelaise@--- Bordelaise}
\index{bordelais@bordelais(e)!sauce@Sauce Bordelaise}

赤ワイン3 dlにエシャロットのみじん切り大さじ2、粗く砕いたこしょう、タ
イム、ローリエの葉\undemi{}枚を加えて火にかけ、\unquart{}量になるまで
煮詰める。ソース・エスパニョル1 dlを加えて火にかけ、浮いてくる夾雑物を
丁寧に取り除きながら弱火で15分間煮る。目の細かい布で漉す。

溶かした\protect\hyperlink{glace-de-viande}{グラスドヴィアンド}大さじ1杯とレモン汁\unquart{}個分、細かいさ
いの目か輪切りにしてポシェしておいた牛骨髄を加えて仕上げる。

\ldots{}\ldots{}牛、羊の赤身肉のグリル用

【原注】こんにちではボルドー風ソースをこのように赤ワインを用いて作るが、
本来的には誤りである。元来は白ワインが用いられていた。これは\protect\hyperlink{sauce-bonnefoy}{ボルドー
風ソース・ボヌフォワ}として後述。

\hypertarget{sauce-bourguignonne}{%
\subsubsection{ブルゴーニュ風ソース}\label{sauce-bourguignonne}}

\frsub{Sauce Bourguignonne}

\index{そーす@ソース!ふるこーにゆふう@ブルゴーニュ風---}
\index{そーす@ソース!ふらうんはせい@ブラウン系の派生---!ふるこーにゆふう@ブルゴーニュ風---}
\index{ふるこーにゆふう@ブルゴーニュ風!そーす@---ソース}
\index{sauce@sauce!petites brunes composees@Petites ---s Brunes Composées!bourguignonne@--- bourguignonne}
\index{sauce@sauce!bourguignonne@--- Bourguignonne}
\index{bourguignon@bourguignon(ne)!sauce@Sauce Bourguignonne}

上質の赤ワイン1\undemi{} L に、エシャロット5個の薄切りとパセリの枝、タ
イム、ローリエの葉\undemi{}枚、マッシュルームの切りくず\footnote{料理に使うマッシュルームは通常、トゥルネ(包丁を持った側の手は動
  かさずに材料を回して切ることからついた用語)すなわち螺旋状に切って
  供するが、その際に少なくない量の切りくずが出るのでこれを使う。}25gを加えて、
半量になるまで煮詰める。布で漉し、ブールマニエ80g(バター45gと小麦粉
35g)を加えてとろみを付ける。提供直前にバター150gを溶かし込み、カイエ
ンヌ\footnote{赤唐辛子の粉末だがカイエンヌは本来、品種名。日本のタカノツメと
  比べると辛さもややマイルドで、風味も異なる。}ごく少量で加えて風味よく仕上げる。

\ldots{}\ldots{}いろいろな卵料理や、家庭料理に好適なソース。

\hypertarget{sauce-bretonne}{%
\subsubsection{ブルターニュ風ソース}\label{sauce-bretonne}}

\frsub{Sauce Bretonne}

\index{そーす@ソース!ふるたーにゆふうふらうんけい@ブルターニュ風---(ブラウン系)}
\index{そーす@ソース!ふらうんはせい@ブラウン系の派生---!ふるたーにゆふう@ブルターニュ風---}
\index{ふるたーにゆふう@ブルターニュ風!そーすふらうんけい@---ソース(ブラウン系)}
\index{sauce@sauce!petites brunes composees@Petites ---s Brunes Composées!bretonne@--- Bretonne}
\index{sauce@sauce!bretonne brune@--- Bretonne (brune)}
\index{breton@breton(ne)!sauce brune@Sauce Bretonne (brune)}

中位の玉ねぎ2個をみじん切りにして、バターでブロンド色になるまで炒める。
白ワイン2\undemi{} dlを注ぎ、半量になるまで煮詰める。ここにソース・エス
パニョル3\undemi{}
dlおよびトマトソース同量を加える。7〜8分間煮立ててから、
刻んだパセリを加えて仕上げる。

【原注】このソースは\protect\hyperlink{haricots-blancs-bretonne}{白いんげん豆のブルターニュ風}以外にはほとんど使わ
れない。

\hypertarget{sauce-aux-cerises}{%
\subsubsection{ソース・スリーズ}\label{sauce-aux-cerises}}

\frsub{Sauce aux Cerises}\footnote{さくらんぼのこと。このレシピでグロゼイユ(すぐり)のジュレを用い
  るが、古くはさくらんぼを用いていたことからこの名称となったと言われ
  ている。}

\index{そーす@ソース!すりーす@---・スリーズ}
\index{そーす@ソース!ふらうんはせい@ブラウン系の派生---!すりーす@---・スリーズ}
\index{くろせいゆ@グロゼイユ!そーす@ソース!すりーす@---・スリーズ}
\index{さくらんほ@サクランボ!そーす@ソース!すりーす@---・スリーズ}
\index{sauce@sauce!petites brunes composees@Petites ---s Brunes Composées!cerises@--- aux Cerises}
\index{sauce@sauce!cerise@--- aux Cerises}

ポルト酒2 dlにイギリス風ミックススパイス\footnote{Mixed
  spiceのこと。Pudding spiceとも呼ばれる。シナモン、ナツメ
  グ、オールスパイスの組み合わせが典型的。これにクローブ、生姜、コリ
  アンダーシード、キャラウェイシードなどが加わっていることも多い。}1つまみと、すりおろしたオレ
ンジの皮を大さじ\undemi{}杯加えて\deuxtiers{}量になるまで煮詰める。
\protect\hyperlink{gelee-de-groseilles-a}{グロゼイユのジュレ} 2\undemi{}
dlを加え、仕上げにオレンジ果汁を加える。

\ldots{}\ldots{}大型ジビエの料理用だが、鴨のポワレやブレゼにも用いられる。

\hypertarget{sauce-aux-champignons}{%
\subsubsection{ソース・シャンピニョン}\label{sauce-aux-champignons}}

\frsub{Sauce aux Champignons}\footnote{champignons
  キノコ全般を意味する語だが、単独で用いられる場合はい
  わゆるマッシュルームを指す。}

\index{そーす@ソース!まつしゆるーむ@マッシュルーム ⇒ ---・シャンピニョン}
\index{そーす@ソース!ふらうんはせい@ブラウン系の派生---!しやんひによん@---・シャンピニョン}
\index{そーす@ソース!ふらうんはせい@ブラウン系の派生---!まつしゆるーむ@マッシュルーム ⇒ ---・シャンピニョン}
\index{まっしゅるーむ@マッシュルーム ⇒ シャンピニョン}
\index{しやんひによん@シャンピニョン!そーすふらうんけいはせい@ソース・---(ブラウン系)}
\index{sauce@sauce!petites brunes composees@Petites ---s Brunes Composées!champignons@--- aux Champignons}
\index{sauce@sauce!champignons brune@--- aux Champignons (brune)}
\index{champignon@champignon!sauce brune@Sauce aux Champignons (brune)}

マッシュルームの茹で汁2\undemi{} dl を半量になるまで煮詰める。
\protect\hyperlink{sauce-demi-glace}{ソース・ ドゥミグラス}8
dlを加えて数分間煮立てる。布で漉し、 バター50
gを投入して味を調え、あらかじめ下茹でしておいた小さめのマッシュ
ルームの笠100 gを加えて仕上げる。

\hypertarget{sauce-charcutiere}{%
\subsubsection{ソース・シャルキュティエール}\label{sauce-charcutiere}}

\frsub{Sauce Charcutière}\footnote{シャルキュトリ(豚肉加工業)風、の意。Charcutrieの語源はchar(肉)
  +cuite(調理された)+rie(業)。ハムやソーセージなどと定番の組合せ
  であるマスタードを使う\protect\hyperlink{sauce-robert}{ソース・ロベール}と、おなじ
  く定番のつけ合わせであるコルニション(小さいうちに収穫してヴィネガー
  漬けにしたきゅうり。専用品種がある)を使うことに由来。}

\index{そーす@ソース!しやるききゆとりふう@シャルキュトリ風 ⇒ ---・シャルキュティエール}
\index{そーす@ソース!ふらうんはせい@ブラウン系の派生---!しやるきゆていえーる@---・シャルキュティエール}
\index{しやるきゆとりふう@シャルキュトリ風!そーす@---ソース ⇒ ソース・シャルキュティエール}
\index{sauce@sauce!petites brunes composees@Petites ---s Brunes Composées!charcutiere@--- Charcutière}
\index{sauce@sauce!charcutiere@--- Charcutière}
\index{charcutier@charcutier(ère)!sauce@Sauce Charcutière}

提供直前に、\protect\hyperlink{sauce-robert}{ソース・ロベール}1 L
に細さ2 mm程度で短かめの千切り\footnote{1〜2mm程度の細さの千切りにした野菜などをジュリエンヌjulienneと呼
  ぶ。} にしたもの100
gを加える(\protect\hyperlink{sauce-robert}{ソース・ロベール}参照)。

\hypertarget{sauce-chasseur}{%
\subsubsection{ソース・シャスール}\label{sauce-chasseur}}

\frsub{Sauce Chasseur}\footnote{狩人風、の意。古くは猟獣肉をすり潰したものを使った料理を指した
  という説もある。マッシュルームとエシャロット、白ワインを使うのが特
  徴であり、このソースを使った料理にも「シャスール」の名が付けられる。}

\index{そーす@ソース!しやすーる@---・シャスール}
\index{そーす@ソース!ふらうんはせい@ブラウン系の派生---!しやすーる@---・シャスール}
\index{しやすーる@シャスール!そーす@ソース・---}
\index{かりうとふう@狩人風 ⇒ ソース・シャスール}
\index{sauce@sauce!petites brunes composees@Petites ---s Brunes Composées!chasseur@--- Chasseur}
\index{sauce@sauce!chasseur@--- Chasseur}
\index{chasseur@chasseur!sauce@Sauce ---}

生のマッシュルームを薄切りにしたもの150gをバターで炒める。エシャロット
\footnote{échalote
  玉ねぎによく似ているが、小ぶりで水分が少なく、香味野菜
  としてよく用いられる。伝統的な品種は種子ではなく種球を植えて栽培す
  る。なお、日本でしばしば「エシャレット」の名称で流通しているものは
  ラッキョウの若どりであり、フランス料理で用いるエシャロットとはまっ
  たく異なる。}のみじん切り大さじ2\undemi{}杯を加えてさらに軽く炒め、白ワイン3
dl
を注ぎ、半量になるまで煮詰める。\protect\hyperlink{sauce-tomate}{トマトソース}3
dl と\protect\hyperlink{sauce-demi-glace}{ソース・ドゥミグラス}2
dlを加える。数分間沸騰さ せたら、バター150 gと、セルフイユ\footnote{cerfeuil
  日本ではチャービルとも呼ばれるセリ科のハーブ。}とエストラゴン\footnote{estragon
  日本ではタラゴンとも呼ばれるヨモギ科のハーブ。フランス
  料理ではとても好まれる重要なハーブのひとつ。フレンチタラゴンとロシ
  アンタラゴンの2種がある。料理に用いるのはフレンチタラゴンであり、
  この品種は種子ではなく株分けや挿し芽で殖やして栽培される。寒さには
  比較的強いが、日本の梅雨の湿度や夏の暑さには弱い。}をみじん切り
にしたもの大さじ1\undemi{}杯を加えて仕上げる。

\hypertarget{sauce-chasseur-procede-escoffier}{%
\subsubsection{ソース・シャスール(エスコフィエ流)}\label{sauce-chasseur-procede-escoffier}}

\frsub{Sauce Chasseur (Procédé Escoffier)}

\index{そーす@ソース!しやすーるえすこふいえ@---・シャスール(エスコフィエ流)}
\index{そーす@ソース!ふらうんはせい@ブラウン系の派生---!しやすーるえすこふいえ@---・シャスール(エスコフィエ流)}
\index{しやすーる@シャスール!そーすしゃすーるえすこふぃえ@ソース・---(エスコフィエ流)}
\index{かりうとふう@狩人風 ⇒ ソース・シャスール(エスコフィエ流)}
\index{sauce@sauce!petites brunes composees@Petites ---s Brunes Composées!chasseur escoffier@--- Chasseur (Procédé Escoffier)}
\index{sauce@sauce!chasseur escoffier@--- Chasseur (Procédé Escoffier)}
\index{chasseur@chasseur!sauce escoffier@Sauce --- (Procédé Escoffier)}

生のマッシュルームを薄切りにしたもの150gを、バターと植物油で軽く色付く
まで炒める。みじん切りにしたエシャロット大さじ1杯を加え、なるべくすぐ
に余分な油をきる。白ワイン2 dl とコニャック約50 ml を注ぎ、半量になるま
で煮詰める。\protect\hyperlink{sauce-demi-glace}{ソース・ドゥミグラス}4
dlと\protect\hyperlink{sauce-tomate}{トマトソー ス}2
dl、\protect\hyperlink{glace-de-viande}{グラスドヴィアンド}大さじ\undemi{}杯を加え
る。

5分間沸騰させたら、仕上げにパセリのみじん切り少々を加える。

\hypertarget{sauce-chaud-froid-brune}{%
\subsubsection{茶色いソース・ショフロワ}\label{sauce-chaud-froid-brune}}

\frsub{Sauce Chaud-froid brune}\footnote{chaudショ「熱い、温かい」とfroidフロワ「冷たい」の合成語で、火
  を通した肉や魚を冷まし、表面にこのソース・ショフロワを覆うように塗
  り付け、さらにジュレを覆いかけた料理。料理の発祥については諸説あり、
  なかでもルイ15世に仕えていた料理長ショフロワChaufroixが考案したと
  いう説を支持してなのか、英語ではこの料理をChaufroixと綴ることも多
  い。Chaud-froidの表記は19世紀後半には文献に見られる。なお、複数形
  はchauds-froidsと綴る。トリュフの薄切りやエストラゴンなどのハーブ
  その他で表面に華麗な装飾を施すことが19世紀には盛んに行なわれていた。
  現代でも装飾に凝った仕立てにするケースは多い。}

\index{そーす@ソース!ふらうんはせい@ブラウン系の派生---!しよふろわちやいろ@茶色い---・ショフロワ(エスコフィエ流)}
\index{そーす@ソース!しよふろわちやいろ@茶色い---・ショフロワ}
\index{そーす@ソース!ふらうんはせい@ブラウン系の派生---!しよふろわふらうんけい@茶色い---・ショフロワ}
\index{しよふろわ@ショフロワ!そーすふらうんけい@茶色いソース・---}
\index{sauce@sauce!petites brunes composees@Petites ---s Brunes Composées!chaud-froid brune@--- Chaud-froid brune}
\index{sauce@sauce!chaud-froid brune@--- Chaud-froid brune}
\index{chaud-froid@chaud-froid!sauce brune@Sauce --- brune}

(仕上がり1 L分)

\protect\hyperlink{sauce-demi-glace}{ソース・ドゥミグラス}\troisquarts{}
Lとトリュフエッ センス1 dl、ジュレ6〜7 dlを用意する。

ソース・ドゥミグラスにトリュフエッセンスを加えて、強火で煮詰めるが、こ
の時に鍋から離れないこと。煮詰めながらジュレを少量ずつ加えていく。最終
的に\deuxtiers{}量程度まで煮詰める。

味見をして、ソースがショフロワに使うのに丁度いい濃さになっているか確認
すること。

マデラ酒またはポルト酒\undemi{}dlを加える。布で漉し、ショフロワの主素
材の表面に塗り付けるのに丁度いい固さになるまで、丁寧にゆっくり混ぜなが
ら冷ます。

\hypertarget{sauce-chaud-froid-brune-pour-canards}{%
\subsubsection{茶色いソース・ショフロワ(鴨用)}\label{sauce-chaud-froid-brune-pour-canards}}

\frsub{Sauce Chaud-froid brune pour Canards}

\index{そーす@ソース!ふらうんはせい@ブラウン系の派生---!しよふろわちやいろかもよう@茶色い---・ショフロワ(鴨用)}
\index{そーす@ソース!しよふろわちやいと@茶色い---・ショフロワ(鴨用)}
\index{しよふろわ@ショフロワ!ちやいろいそーすかもよう@茶色いソース・---(鴨用)}
\index{sauce@sauce!petites brunes composees@Petites ---s Brunes Composées!chaud-frois canard@--- Chaud-froid pouir Canards}
\index{sauce@sauce!chaud-froid brune pour canards@--- Chaud-froid brune pour Canards}
\index{chaud-froid@chaud-froid!sauce brune pour Canards@Sauce --- brune pour Canards}

作り方は上記、\protect\hyperlink{sauce-chaud-froid-brune}{茶色いソース・ショフロワ}と同
様だが、トリュフエッセンスではなく、鴨のガラでとったフュメ1\undemi{}
dlを用いること。また、上記のレシピよりややしっかり煮詰めること。

ソースを布で漉したら、オレンジ3個分の搾り汁、とオレンジの皮をごく薄く
剥いて細かい千切りにしたもの\footnote{zeste
  ゼスト。オレンジやレモンの皮の表面を器具を用いてすりおろ
  すか、ナイフでごく薄く表皮を向き、細かい千切りにしたもの。ここでは
  後者を使う指定になっている。}大さじ2杯を加える。オレンジの皮の千切
りはしっかりと下茹でしてよく水気をきっておくこと。

\hypertarget{sauce-chaud-froid-brune-pour-gibier}{%
\subsubsection{茶色いソース・ショフロワ(ジビエ用)}\label{sauce-chaud-froid-brune-pour-gibier}}

\frsub{Sauce Chaud-froid brune pour Gibier}

\index{そーす@ソース!ふらうんはせい@ブラウン系の派生---!しよふろわちやいろしひえよう@茶色い---・ショフロワ(ジビエ用)}
\index{そーす@ソース!しよふろわちやいろじびえよう@茶色い---・ショフロワ(ジビエ用)}
\index{しよふろわ@ショフロワ!そーすしよふろわちやいろじびえよう@茶色いソース・---(ジビエ用)}
\index{sauce@sauce!petites brunes composees@Petites ---s Brunes Composées!chaud-frois gibier@--- Chaud-froid pour Gibier}
\index{sauce@sauce!chaud-froid brune pour Gibier@--- Chaud-froid brune pour Gibier}
\index{chaud-froid@chaud-froid!sauce brune pour Gibier@Sauce --- brune
pour Gibier}

作り方は上記\protect\hyperlink{sauce-chaud-froid-brune}{標準的なソース・ショフロワ}と同
じだが、トリュフエッセンスではなく、ショフロワとして供するジビエのガラ
でとったフュメ\footnote{\protect\hyperlink{fonds-de-gibier}{ジビエのフォン}参照。}2dlを用いること。

\hypertarget{sauce-chaud-froid-tomatee}{%
\subsubsection{トマト入りソース・ショフロワ}\label{sauce-chaud-froid-tomatee}}

\frsub{Sauce Chaud-froid tomatée}

\index{そーす@ソース!ふらうんはせい@ブラウン系の派生---!とまといりしよふろわ@トマト入り---・ショフロワ}
\index{そーす@ソース!しよふろわとまといり@トマト入り---・ショフロワ}
\index{しよふろわ@ショフロワ!そーすちやいろとまといり@トマト入りソース・---}
\index{sauce@sauce!petites brunes composees@Petites ---s Brunes Composées!chaud-froid tomatee@--- Chaud-froid tomatée}
\index{sauce@sauce!chaud-froid tomatée@--- Chaud-froid tomatée}
\index{chaud-froid@chaud-froid!sauce tomatée@Sauce --- tomatée}

良質で、既によく煮詰めてあるトマトピュレ1 Lを、さらに煮詰めながら7〜8
dlのジュレを少しずつ加えていく。全体量が1L以下になるまで煮詰めること。

布で漉し、使いやすい固さになるまで、ゆっくり混ぜながら冷ます。

\hypertarget{sauce-chevreuil}{%
\subsubsection{ソース・シュヴルイユ}\label{sauce-chevreuil}}

\frsub{Sauce Chevreuil}

\index{そーす@ソース!ふらうんはせい@ブラウン系の派生---!しゆうるいゆ@---・シュヴルイユ}
\index{しゆうるいゆ@シュヴルイユ!そーす@ソース・---}
\index{そーす@ソース!しゅうるいゆ@---・シュヴルイユ}
\index{のろしか@ノロ鹿 ⇒ シュヴルイユ!そーす@ソース!しゆうるいゆ@ソース・シュヴルイユ}
\index{sauce@sauce!petites brunes composees@Petites ---s Brunes Composées!chevreuil@--- Chevreuil}
\index{sauce@sauce!chevreuil@--- Chevreuil}
\index{chevreuil@chevreuil!sauce@Sauce ---}

\protect\hyperlink{sauce-poivrade}{標準的なソース・ポワヴラード}と同様に作るが、

\begin{enumerate}
\def\labelenumi{\arabic{enumi}.}
\item
  マリネした牛・羊肉の料理に添える場合\footnote{chevreuil
    シュヴルイユはノロ鹿のことだが、このように事前にマリ
    ネした牛・羊肉を用いた料理にもこのソースを使い「シュヴルイユ(風)(仕立て)」
    と\ruby{謳}{うた}う。1806年刊ヴィアール『帝国料理の本』においてノ
    ロ鹿のフィレは香辛料を加えたワインヴィネガーで48時間マリネしてから
    調理すると書かれている。オド『女性料理人のための本』では、確認出来
    た1834年の第4版から1900年の第78版に至るまで、ノロ鹿の項において
    「一週間もヴィネガーたっぷりの漬け汁でマリネするのはやりすぎだが、
    強い味が好みなら1〜4日間」香辛料と赤ワインあるいはヴィネガーでマリ
    ネするといい、と説明されている。つまり、ノロ鹿とは必ずマリネしてか
    ら調理するものという一種のコンセンサスがあったために、マリネした牛・
    羊肉の料理にも「シュヴルイユ(風)」の名称が謳われるようになったと考
    えられる。}は、ハム入りの\protect\hyperlink{mirepoix}{ミルポ
  ワ}を加える。
\item
  ジビエ料理に添える場合は、そのジビエの端肉を加える。
\end{enumerate}

素材をヘラなどで強く押し付けるようにして漉す\footnote{シノワ(\protect\hyperlink{sauce-espagnole}{ソース・エスパニョル}訳注参照)などを用いる。}。良質の赤ワイン
1\undemi{}dlをスプーン1杯ずつ加えながら煮て、浮き上がってくる不純物を
丁寧に取り除いていく\footnote{dépouiller デプイエ ≒ écumer エキュメ。}。

最後に、カイエンヌごく少量と砂糖1つまみを加えて味を\ruby{調}{とと
の}え、布で漉す。

\hypertarget{sauce-colbert}{%
\subsubsection{ソース・コルベール}\label{sauce-colbert}}

\frsub{Sauce Colbert}\footnote{17世紀の政治家、ジャン・バティスト・コルベール(1619〜1683)の
  名を冠したもの。}

\index{そーす@ソース!ふらうんはせい@ブラウン系の派生---!こるへーる@---・コルベール}
\index{そーす@ソース!こるへーる@---・コルベール}
\index{こるへーる@コルベール!そーす@ソース・---}
\index{sauce@sauce!petites brunes composees@Petites ---s Brunes Composées!colbert@--- Colbert}
\index{sauce@sauce!colbert@--- Colbert}
\index{colbert@Colbert!sauce@Sauce ---}

\protect\hyperlink{beurre-maitre-d-hotel}{メートルドテルバター}に\protect\hyperlink{glace-de-viande}{グラスドヴィアンド}を加
えたもののことだが、正しくは「\protect\hyperlink{beurre-colbert}{ブール・コルベール}」と
呼ぶべきものだ\footnote{具体的なレシピは\protect\hyperlink{beurre-colbert}{ブール・コルベール}参照のこと。}。

また、ブール・コルベールと\protect\hyperlink{sauce-chateaubriand}{ソース・シャトーブリアン}との違いを明確にさ
せようとして、メートルドテルバターにエストラゴンを加える者もいる。だが、
必ずそうすべきということではない。実際、ブール・コルベールとソース・シャ
トーブリアンは明らかに違うものだからだ。ソース・シャトーブリアンは軽く
仕上げたグラスドヴィアントにバターとパセリのみじん切りを加えたものであ
る。一方、ブール・コルベールあるいはソース・コルベールと呼ばれているもの
はあくまでもバターが主であって、グラスドヴィアンドは補助的なものに過ぎ
ない。

\hypertarget{sauce-diable}{%
\subsubsection{ソース・ディアーブル}\label{sauce-diable}}

\frsub{Sauce Diable}\footnote{悪魔の意。}

\index{そーす@ソース!ふらうんはせい@ブラウン系の派生---!ていあーふる@---・ディアーブル}
\index{そーす@ソース!ていあーふる@---・ディアーブル}
\index{ていあーふる@ディアーブル!そーす@ソース・---}
\index{あくま@悪魔 ⇒ ディアーブル!そーす@ソース!そーすていあーふる@ソース・ディアーブル}
\index{sauce@sauce!petites brunes composees@Petites ---s Brunes Composées!diable@--- Diable}
\index{sauce@sauce!diable@--- Diable}
\index{diable@diable!sauce@Sauce ---}

このソースはごく少量ずつ作るのが一般的だが、ここではそれを守らずに、仕
上り2\undemi{} dlとして説明する

白ワイン3 dlにエシャロット3個分のみじん切りを加え、\untiers{}量以下にな
るまで煮詰める。

\protect\hyperlink{sauce-demi-glace}{ソース・ドゥミグラス}2
dlを加えて数分間煮立たせ、
仕上げにカイエンヌの粉末をたっぷり効かせる\footnote{「たっぷり」という表現に惑わされないよう注意。}。

\ldots{}\ldots{}鶏と鳩のグリルに合わせる。

\hypertarget{nota-sauce-diable}{%
\subparagraph{【原注】}\label{nota-sauce-diable}}

白ワインではなくヴィネガーを煮詰め、仕上げにハーブを加えて作る調理現場
もあるが、著者としては本書で示しているの作り方がいいと思う。

\hypertarget{sauce-diable-escoffier}{%
\subsubsection{ソース・ディアーブル・エスコフィエ}\label{sauce-diable-escoffier}}

\frsub{Sauce Diable Escoffier}

\index{そーす@ソース!ふらうんはせい@ブラウン系の派生---!ていあーふるえすこふいえ@---・ディアーブル・エスコフィエ}
\index{そーす@ソース!ていあーふるえすこふいえ@---・ディアーブル・エスコフィエ}
\index{ていあーふる@ディアーブル!そーす@ソース!えすこふいえ@ソース・---・エスコフィエ}
\index{あくま@悪魔 ⇒ ディアーブル!そーす@ソース!エスコフイエ@ソース・ディアーブル・エスコフィエ}
\index{sauce@sauce!petites brunes composees@Petites ---s Brunes Composées!diable escoffier@--- Diable Escoffier}
\index{sauce@sauce!diable escoffier@--- Diable Escoffier}
\index{diable@diable!sauce escoffier@Sauce --- Escoffier}

このソースは完成品が市販\footnote{現在は市販されていないと思われる。フランスにおいては未確認だが、
  1980年代までアメリカ合衆国ではナビスコがソース・ロベール・エスコフィ
  エとともに瓶詰めを生産、販売していた。初版ではこれら2つの製品への
  言及がなく、第二版で追加されたことから、1903年〜1907年の間に製品化
  された可能性もある。また、第二版(1907年)と同年の英訳版、第三版
  (1912年)にはソース・スリーズ・エスコフィエの記述が見られるが、こ
  れは第四版で削除されており、生産中止になったと思われる。エスコフィ
  エ・ブランドの既製品ソースはさらに他にもあったようだが詳細は不明。なお、
  エスコフィエは1922年頃、ジュリユス・マジがブイヨンキューブ(日本で
  は「マギーブイヨン」の商品名)を開発する際にも協力した。}されている。同量の柔くしたバターを混ぜ合
わせるだけでいい。

\hypertarget{sauce-diane}{%
\subsubsection{ソース・ディアーヌ}\label{sauce-diane}}

\frsub{Sauce Diane}\footnote{ローマ神話の女神ディアーナのこと。ギリシア神話のアルテミスに相
  当し、狩猟、貞潔の女神。また月の女神ルーナ(セレーネー)と同一視さ
  れた。ここでは大型ジビエ料理用のソースであるから、狩猟の女神という
  意味合いが強い。}

\index{そーす@ソース!ふらうんはせい@ブラウン系の派生---!ていあーぬ@---・ディアーヌ}
\index{そーす@ソース!ていあーぬ@---・ディアーヌ}
\index{ていあーぬ@ディアーヌ!そーす@ソース・---}
\index{sauce@sauce!petites brunes composees@Petites ---s Brunes Composées!diane@--- Diane}
\index{sauce@sauce!diane@--- Diane} \index{diane@Diane!sauce@Sauce ---}

不純物を充分に取り除き、コクと風味ゆたかな\protect\hyperlink{sauce-poivrade}{ソース・ポワヴラー
ド}5 dlを用意する。提供直前に、泡立てた生クリーム4 dl
(生クリーム2dlを泡立てて倍量にする)と、小さな三日月の形にしたトリュ
フのスライスと固茹で卵の白身を加える。

\ldots{}\ldots{}大型ジビエの骨付き背肉および、その中心部を円筒形に切り出したもの
\footnote{noisette ノワゼット。}、フィレ料理用。

\hypertarget{sauce-duxelles}{%
\subsubsection{ソース・デュクセル}\label{sauce-duxelles}}

\frsub{Sauce Duxelles}{[}\^{}

\index{そーす@ソース!ふらうんはせい@ブラウン系の派生---!てゆくせる@---・デュクセル}
\index{そーす@ソース!てゆくせる@---・デュクセル}
\index{てゆくせる@デュクセル!そーす@ソース・---}
\index{sauce@sauce!petites brunes composees@Petites ---s Brunes Composées!duxelles@--- Duxelles}
\index{sauce@sauce!duxelles@--- Duxelles}
\index{duxelles@duxelles!sauce@Sauce ---}

白ワイン2dlとマッシュルームの茹で汁2 dlにエシャロットのみじん切り大さじ2
杯を加えて、\untiers{}量まで煮詰める。\protect\hyperlink{sauce-demi-glace}{ソース・ドゥミグラ
ス}\undemi{} Lとトマトピュレ1\undemi{}
dl、\protect\hyperlink{duxelles-seche}{デュク
セル・セッシュ}大さじ4杯を加える。5分間煮立たせ、パセリのみじん切り
大さじ\undemi{}を加える。

\ldots{}\ldots{}グラタンの他、いろいろな料理に用いられる。

\hypertarget{ux539fux6ce8}{%
\subparagraph{【原注】}\label{ux539fux6ce8}}

ソース・デュクセルはイタリア風ソースと混同されることが多いが、ソース・
デュクセルにはハムも、赤く漬けた舌肉も入れないので、まったく別のものだ。

\hypertarget{sauce-estragon}{%
\subsubsection{ソース・エストラゴン}\label{sauce-estragon}}

\frsub{Sauce Estragon}\footnote{ヨモギ科のハーブ。\protect\hyperlink{sauce-chasseur}{ソース・シャスール}訳注参照。}

\index{そーす@ソース!ふらうんはせい@ブラウン系の派生---!えすとらこん@---・エストラゴン}
\index{そーす@ソース!えすとらこんちゃいろ@---・エストラゴン(ブラウン系)}
\index{えすとらこん@エストラゴン!そーすふらうんけい@ソース・---(ブラウン系)}
\index{sauce@sauce!petites brunes composees@Petites ---s Brunes Composées!estragon@--- Estragon}
\index{sauce@sauce!estragon brune@--- Estragon (brune)}
\index{estragon@estragon!sauce brune@Sauce --- (brune)}

(仕上がり2\undemi{} dl分)

白ワイン2dlを沸かし、エストラゴンの枝20 gを投入する。蓋をして10分間、煎
じる\footnote{infuser(アンフュゼ)。}。2\undemi{}
dlの\protect\hyperlink{sauce-demi-glace}{ソース・ドゥミグラス}また
は、\protect\hyperlink{jus-de-veau-lie}{とろみを付けた仔牛のジュ}を加え、約\deuxtiers{}
量になるまで煮詰める。布で漉し、みじん切りにしたエストラゴン小さじ1杯
を加えて仕上げる。

\ldots{}\ldots{}仔牛や仔羊の背肉の中心を円筒形に切り出した料理や家禽料理用。

\hypertarget{sauce-financiere}{%
\subsubsection{ソース・フィナンシエール}\label{sauce-financiere}}

\frsub{Sauce Financière}\footnote{Financier徴税官(財務官)風の意。フランス革命以前の徴税官は、王
  に代わって徴税を行なう大貴族が就く役職であり、膨大な利権によりきわめて
  裕福であったという。このソースと組み合わせる\protect\hyperlink{garniture-financiere}{ガルニチュール・フィナン
  シエール}が、雄鶏のとさかと睾丸、仔羊の胸腺肉、トリュフなどの比較的
  入手困難あるいは高級とされる食材で構成されていることが名称の由来と思わ
  れる。ブリヤ=サヴァランは『美味礼讃』(味覚の生理学)において、徴税官
  たちは旬のはしりの食材を真っ先に食べられる、いわば特権階級だと述べてい
  る。なお、カレーム『19世紀フランス料理』においては、ソースとガルニチュー
  ルを分離せず、「ラグー・アラ・フィナンシエール」として採りあげられてい
  るが、全ての素材を別々に加熱調理してソースと合わせるものであり、いわゆ
  る「煮込み」とは呼びがたいものとなっている。フランス料理の影響が比較的
  強かった北イタリアにこの原型に近いと思われるラグー「ピエモンテ風フィナ
  ンツィエラ」がある。鶏のとさか、肉垂、睾丸、鶏レバーおよび仔牛の胸腺肉
  などを煮込んだものだが、レシピを読む限りにおいては比較的庶民的あるいは
  農民的料理に変化したものと思われる (cf.~Anna Gosetti della Salda,
  \emph{Le Ricette Regionali Italiane}, Milano, Solares, 1967,
  p.57.)。ちなみに焼
  き菓子のフィナンシエfinancierも同語源だが、何故その名称になったかは不
  明。}

\index{そーす@ソース!ふらうんはせい@ブラウン系の派生---!ふいなんしえーる@---・フィナンシエール}
\index{そーす@ソース!ふいなんしえーる@---・フィナンシエール}
\index{ふいなんしえーる@フィナンシエール!そーす@ソース・---}
\index{ちょうせいかんふう@徴税官風 ⇒ フィナンシエール!そーすふぃなんしえーる@ソース・フィナンシエール}
\index{sauce@sauce!petites brunes composees@Petites ---s Brunes Composées!financiere@--- Financière}
\index{sauce@sauce!financiere@--- Financière}
\index{financier@financier(ère)!sauce@Sauce Financière}

\protect\hyperlink{sauce-madere}{ソース・マデール}1\unquart{}Lを\troisquarts{}量以下に
なるまで煮詰め、火から外してトリュフエッセンス1 dlを加える。布で漉して
仕上げる。

\ldots{}\ldots{}\protect\hyperlink{garniture-financiere}{ガルニチュール・フィナンシエール}用だが、その他の肉料理にも用い
られる。

\hypertarget{sauce-aux-fines-herbes}{%
\subsubsection{香草ソース}\label{sauce-aux-fines-herbes}}

\frsub{Sauce aux Fines Herbes}\footnote{料理名では、いわゆる「ハーブ」についてかつてfines
  herbesの表
  現が多く用いられた。だが、こんにちでは特定のハーブ名をソースや料理名
  に添えて言うことが多い。例えばCôtelette de veau au thymコトレット
  ドヴォオタン(仔牛の骨付き背肉、タイム風味)、やFilet de bar poêlé,
  compote de tomate au basilicフィレドバールポワレ コンポットートド
  トマトバジリック(スズキのフィレとトマトのコンポート、バジル風味)
  など。また、栽培レベルで「香草、ハーブ」の総称としては herbes
  aromatiques
  (エルブザロマティック)、あるいはたんにaromatiques(アロマ
  ティック)が一般的。}

\index{そーす@ソース!ふらうんはせい@ブラウン系の派生---!こうそう@香草---}
\index{そーす@ソース!こうそうふらうんけい@香草---(ブラウン系)}
\index{こうそう@香草!そーすふらうんけい@---ソース(ブラウン系)}
\index{はーぶ@ハーブ ⇒ 香草!こうそうそーすふらうんけい@香草ソース(ブラウン系)}
\index{sauce@sauce!petites brunes composees@Petites ---s Brunes Composées!fines herbes@--- aux Fines Herbes}
\index{sauce@sauce!fines herbes@--- aux Fines Herbes} \index{fines
herbes@fines herbes!sauce@Sauce aux ---}

白ワイン3 dlを沸かし、パセリの葉、セルフイユ、エストラゴン、シブレット
を各1つまみ強、投入する。約20分間煎じる。布で漉し、\protect\hyperlink{sauce-demi-glace}{ソース・ドゥミグラ
ス}または\protect\hyperlink{jus-de-veau-lie}{とろみを付けた仔牛の ジュ}6
dlを加える。仕上げに、煎じるのに使ったのと同
じ香草を細かく刻んだもの計、大さじ2\undemi{}杯とレモンの搾り汁少々を加
える。

\hypertarget{ux539fux6ce8nota-sauce-aux-fines-herbes}{%
\subparagraph{【原注】\{nota-sauce-aux-fines-herbes\}}\label{ux539fux6ce8nota-sauce-aux-fines-herbes}}

古典料理ではこの「香草ソース」と\protect\hyperlink{sauce-duxelles}{ソース・デュクセル}
が混同されることもあったが、こんにちではまったく違うものとして扱われて
いる。

\hypertarget{sauce-genevoise}{%
\subsubsection{ジュネーヴ風ソース}\label{sauce-genevoise}}

\frsub{Sauce Genevoise}

\index{そーす@ソース!ふらうんはせい@ブラウン系の派生---!しゆねーうふう@ジュネーヴ風---}
\index{そーす@ソース!しゆねーうふう@ジュネーヴ風---}
\index{しゆねーうふう@ジュネーヴ風!そーす@---ソース}
\index{sauce@sauce!petites brunes composees@Petites ---s Brunes Composées!genevoise@--- Genevoise}
\index{sauce@sauce!genevoise@--- Genevoise}
\index{genevois@genevois(e)!sauce@Sauce Genevoise}

鍋にバターを熱し、細かく刻んだミルポワを色付かないよう強火でさっと炒め
る。ミルポワの材料は、にんじん100 g、玉ねぎ80 g、タイムとローリエ少々、
パセリの枝20 g。そこにサーモンの頭1kgと粗く砕いたこしょう1つまみを入れ、
蓋をして弱火で15分程蒸し煮する。

鍋に残ったバターを捨て、赤ワイン1Lを注ぐ。半量になるまで煮詰める。そこ
に\protect\hyperlink{sauce-espagnole-maigre}{魚料理用ソース・エスパニョル}\undemi{}
Lを
加える。弱火で1時間煮込む。漉し器を使い、材料を押しつけながら漉す。し
ばらく休ませてから、表面に浮いた油脂を取り除く\footnote{dégraisser
  デグレセ。レードルなどを用いて浮いてきた余計な油脂を取り除く作業。}

さらに赤ワイン\undemi{} Lと、魚のフュメ\undemi{} Lを加える。ソースの表
面に浮いてくる不純物を徹底的に取り除き\footnote{dépouiller デプイエ ≒
  écumer エキュメ。}、丁度いい濃さになるまで煮 詰める。

これを布で漉し、静かに混ぜながら、アンチョヴィのエッセンス大さじ1杯と
バター150 gを加えて仕上げる。

\ldots{}\ldots{}サーモン、鱒料理用。

\hypertarget{nota-sauce-genevoise}{%
\subparagraph{【原注】}\label{nota-sauce-genevoise}}

このソースはもともとカレームが「ジェノヴァ風」\footnote{Sauce à la
  génoise au vin de Bordeaux ボルドー産ワインを用いた
  ジェノヴァ風ソース(『19世紀フランス料理』第3巻、80頁)。本書のこのレシ
  ピと同様に魚料理用ソースだ。ボルドーの赤ワインにみじん切りにして下茹で
  したマッシュルーム、トリュフ、エシャロットを加えてオールスパイスとこしょ
  う少々を入れ、適度に煮詰める。ソース・エスパニョルと赤ワインを加え、湯
  煎にかけておく。提供直前にバター少量を加えて仕上げる、というもの。本書
  においてこのソースを「原型」とするのには疑問が残るところだろう。}と名付けたものだが、その
後ルキュレ、グフェ\footnote{グフェ『料理の本』(1867年)の420ページにあるジュネーヴ風ソー
  スは、薄切りにした玉ねぎ、エシャロット、粗挽きこしょう、にんにく、
  バターを鍋に入れて色付くまで炒め、そこにブルゴーニュ産赤ワインを注
  ぐ。弱火で玉ねぎに火が通るまで煮る。ソース・エスパニョルと仔牛のブ
  ロンドのジュを加えて煮詰め、布で漉す。提供直前にマデラ酒の風味を加
  えて茹でたトリュフのみじん切りとアンチョビバターを加える、というも
  の。赤ワインと玉ねぎ、仕上げにアンチョビを加える点は共通しているが、
  グフェのが肉料理用であるのに対して、本書のこのソースは明らかに魚料
  理用であり、まったく同じソースと呼べるとは言い難い。}が立て続けに「ジュネーヴ風」の名称を用いた。だが、ジュ
ネーヴは赤ワインの産地ではないから理屈としてはおかしい\footnote{料理名に冠された地名は、由来が明確にあるものがある一方で、まっ
  たく意味不明か、あるいはいい加減な思い付きで付けられたのではないか
  とさえ思われるものも少なくない。(à la) russe「ロシア風」や (à la)
  moscovite「モスクワ風」などはロシア料理起源か、あるいは18世紀末〜
  19世紀前半にかけてロシア帝国の宮廷や貴族がこぞってフランスから料理
  人を招聘し、帰国した彼らが創案した料理などはある程度しっかりとした
  由来がわかるものも多い。一方で、(à l')espagnole「スペイン風」(à
  l')italienne「イタリア風」(à la) romaine「ローマ風」(à la grecque)
  「ギリシア風」(à l')allemande「ドイツ風」(à l')hollandaise「オラン
  ダ風」などは由来の不明なケースが非常に多い。\protect\hyperlink{sauce-espagnole}{ソース・エスパニョ
  ル}などはその典型例とも言うべきものだろう。\\
  この原注では由来に非常にこだわっているが、そもそもカレームのレシピ
  は上述のように「ボルドー産ワインを用いたジェノバ風ソース」であるか
  ら、赤ワインの産地かどうかということは実はさしたる問題にはならない。
  重要なのは後半の、赤ワインを用いることがこのソースのポイントという
  こと。}。

間違っているとはいえ、ジュネーヴ風という名称で定着してしまっているので、
本書でもそのままにしている。だが、ジュネーヴ風であれジェノヴァ風であれ、
カレーム、ルキュレ、デュボワ、グフェはいずれもこのソースに赤ワインを用
いるよう指示している。つまり赤ワインを用いることがこのソースのポイント。

\hypertarget{sauce-godard}{%
\subsubsection{ソース・ゴダール}\label{sauce-godard}}

\frsub{Sauce Godard}\footnote{ガルニチュール・ゴダールの構成要素がガルニチュール・フィナンシ
  エールとよく似ている点などから、おそらくは18世紀の徴税官(つまりフィ
  ナンシエ)であり作家としても活動したクロード・ゴダール・ドクール
  Claude Godard d'Aucour(1716〜1795)の名を冠したものと考えられる。
  なお、底本とした現行版(第四版)では最後がdではなくtとなっているが、
  初版から第三版にいたるまでdとなっており、現行版は明らかな誤植。}

\index{そーす@ソース!ふらうんはせい@ブラウン系の派生---!こたーる@---・ゴダール}
\index{そーす@ソース!こたーる@---・ゴダール}
\index{こたーる@ゴダール!そーす@ソース・---}
\index{sauce@sauce!petites brunes composees@Petites ---s Brunes Composées!godard@--- Godard}
\index{sauce@sauce!godard@--- Godard}
\index{godard@Godard!sauce@Sauce ---}

シャンパーニュまたは辛口の白ワイン4
dlにハム入りの細かく刻んだ\protect\hyperlink{mirepoix}{ミルポ
ワ}、\protect\hyperlink{sauce-demi-glace}{ソース・ドゥミグラス}1
Lとマッシュルームのエッセンス2dlを加える。
弱火に10分かけ、シノワ\footnote{\protect\hyperlink{sauce-espagnole}{ソース・エスパニョル}訳注参照。}で漉す。

\deuxtiers{}量になるまで煮詰め、布で漉す。

\ldots{}\ldots{}\protect\hyperlink{garniture-godard}{ガルニチュール ゴタール}用。

\hypertarget{sauce-grand-veneur}{%
\subsubsection{ソース・グランヴヌール}\label{sauce-grand-veneur}}

\frsub{Sauce Grand-Veneur}\footnote{王家や貴族に仕える狩猟長のことをグランヴヌールと呼ぶ。}

\index{そーす@ソース!ふらうんはせい@ブラウン系の派生---!くらんうぬーる@---・グランヴヌール}
\index{そーす@ソース!くらんうぬーる@---・グランヴヌール}
\index{くらんうぬーる@グランヴヌール!そーす@ソース・---}
\index{sauce@sauce!petites brunes composees@Petites ---s Brunes Composées!grand-veneur@--- Grand-Veneur}
\index{sauce@sauce!grand-veneur@--- Grand-Veneur}
\index{grand-veneur@grand-veneur!sauce@Sauce ---}

\protect\hyperlink{fonds-de-gibier}{大型ジビエのフュメ}で澄んだ色合いに作った\protect\hyperlink{sauce-poivrade}{ソース・
ポワヴラード}に、ソース1Lあたり野うさぎの血1dlをマリ
ネ液1dlで薄めたものを加える。

火をごく弱くして、血が沸騰しないよう気をつけながら数分間煮る。布で漉す。

\hypertarget{sauce-grand-veneur-procede-escoffier}{%
\subsubsection{ソース・グランヴヌール(エスコフィエ流)}\label{sauce-grand-veneur-procede-escoffier}}

\frsub{Sauce Grand-Veneur (Procédé Escoffier)}

\index{そーす@ソース!ふらうんはせい@ブラウン系の派生---!くらんうぬーるえすこふいえ@---・グランヴヌール(エスコフィエ流)}
\index{そーす@ソース!くらんうぬーるえすこふいえ@---・グランヴヌール(エスコフィエ流)}
\index{くらんうぬーる@グランヴヌール!そーすえすこふいえ@ソース・---(エスコフィエ流)}
\index{sauce@sauce!petites brunes composees@Petites ---s Brunes Composées!grand-veneur escoffier@--- Grand-Veneur (Procédé Escoffier)}
\index{sauce@sauce!grand-veneur escoffier@--- Grand-Veneur (Procédé Escoffier)}
\index{grand-veneur@grand-veneur!sauce escoffier@Sauce --- (Procédé Escoffier)}

軽く仕上げた\protect\hyperlink{sauce-poivrade}{ソース・ポワヴラード}1
Lあたり{[}グロゼイ ユのジュレ{]}大さじ2杯と生クリーム2\undemi{}
dlを加える。

\ldots{}\ldots{}上記2つのソースは鹿、猪などの大きな塊肉の料理に用いる。

\hypertarget{sauce-gratin}{%
\subsubsection{ソース・グラタン}\label{sauce-gratin}}

\frsub{Sauce Gratin}\footnote{魚のグラタン用ソースだが、グラタンの技術的ポイントについては\protect\hyperlink{gratins}{第
  7章「肉料理」のグタランの項目}参照。}

\index{そーす@ソース!ふらうんはせい@ブラウン系の派生---!くらたん@---・グラタン}
\index{そーす@ソース!くらたん@---・グラタン}
\index{くらたん@グラタン!そーす@ソース・---}
\index{sauce@sauce!petites brunes composees@Petites ---s Brunes Composées!gratin@--- Gratin}
\index{sauce@sauce!gratin@--- Gratin}
\index{gratin@gratin!sauce@Sauce ---}

白ワインと、このソースを合わせる魚のアラなどでとった\protect\hyperlink{fumet-de-poisson}{魚のフュ
メ}各3 dlにエシャロットのみじん切り大さじ1\undemi{}
杯を加え、半量以下になるまで煮詰める。

\protect\hyperlink{duxelles-seche}{デュクセル・セッシュ}大さじ3杯と、\protect\hyperlink{sauce-espagnole-maigre}{魚料理用ソース・エスパニョ
ル}または\protect\hyperlink{sauce-demi-glace}{ソース・ドゥミグラ ス}5
dlを加える。5〜6分間煮立たせる。提供直前に、パ
セリのみじん切り大さじ\undemi{}を加えて仕上げる。

\ldots{}\ldots{}舌びらめ、メルラン\footnote{タラの近縁種。}、バルビュ\footnote{鰈の近縁種。この場合のフィレはいわゆる「五枚おろし」にしたもの。}のフィレなどのグラタン用。

\hypertarget{sauce-hachee}{%
\subsubsection{ソース・アシェ}\label{sauce-hachee}}

\frsub{Sauce Hachée}\footnote{細かく刻んだもの、の意。}

\index{そーす@ソース!ふらうんはせい@ブラウン系の派生---!あしえ@---・アシェ}
\index{そーす@ソース!あしえ@---・アシェ}
\index{sauce@sauce!petites brunes composees@Petites ---s Brunes Composées!hachee@--- Hachée}
\index{sauce@sauce!hachee@--- Hachée}
\index{hache@haché(e)!sauce@Sauce Hachée}

玉ねぎの細かいみじん切り100gと、エシャロットの細かいみじん切り大さじ
1\undemi{}杯をバターで色付かないよう炒める。ヴィネガー3 dlを注ぎ、半量
まで煮詰める。\protect\hyperlink{sauce-espagnole}{ソース・エスパニョル}4
dlと\protect\hyperlink{sauce-tomate}{トマトソース}1\undemi{} dl
を加える。5〜6分煮立たせる。

ハムの脂身のない部分を細かく刻んだもの大さじ1\undemi{}杯と小ぶりのケイ
パー大さじ1\undemi{}杯、{[}デュクセル・セッシュ{]}大さじ1\undemi{}杯、パセ
リのみじん切り大さじ\undemi{}杯を加えて仕上げる

\ldots{}\ldots{}このソースは\protect\hyperlink{sauce-piquante}{ソース・ピカント}と等価のものと考えていい。用途も同じ。

\hypertarget{sauce-hachee-maigre}{%
\subsubsection{魚料理用ソース・アシェ}\label{sauce-hachee-maigre}}

\frsub{Sauce Hachée maigre}

\index{そーす@ソース!ふらうんはせい@ブラウン系の派生---!あしえさかな@魚料理用---・アシェ}
\index{そーす@ソース!あしえ@魚料理用---・アシェ}
\index{sauce@sauce!petites brunes composees@Petites ---s Brunes Composées!hachee maigre@--- Hachée maigre}
\index{sauce@sauce!hachee maigre@--- Hachée maigre}
\index{hache@haché(e)!sauce maigre@Sauce Hachée maigre}

上記と同様に、玉ねぎとエシャロットを色付かないようバターで炒め、ヴィネ
ガーを注いで煮詰める。

魚の\protect\hyperlink{courts-bouillons-de-poisson}{クールブイヨン}5
dlを注ぎ、\protect\hyperlink{roux-brun}{茶色いルー}45 gまたはブー
ルマニエ50 gでとろみを付ける。弱火で8〜10分間煮込む。

提供直前に、細かく刻んだハーブミックス大さじ1杯と\protect\hyperlink{duxelles-seche}{デュクセル・セッシュ}大
さじ1\undemi{}杯、小粒のケイパー大さじ1\undemi{}杯、アンチョヴィソース
大さじ\undemi{}杯とバター60 g、または80〜100 gのアンチョヴィバターを加
えて仕上げる。

\ldots{}\ldots{}エイのような、あまり高級ではない茹でた魚\footnote{原文
  poissons bouillis。このフランス語の表現だと加熱する際に沸
  騰させているニュアンスがあるが、本書の「魚料理」の章において、魚を
  塩を加えて茹でる、あるいはクールブイヨンで煮る際に、沸騰しない程度
  の温度で加熱(ポシェ pocher)すべきと強調されている。この表現は初
  版からのものであり、恐らくはこのソースの部分を実際に執筆した者と、
  魚料理の説明部分を執筆した者が異なることによるわかりにくさ、という
  可能性も排除出来ない。いずれにしても、このソースの場合は、合わせる
  魚をクールブイヨンで沸騰しない程度の温度で加熱(ポシェ)し、そのクー
  ルブイヨンの一部をソースに加えていることから、単に「茹でた魚」と言っ
  ても、本書における魚の加熱方法に則った調理をすべきと解されよう。}用。

\hypertarget{sauce-hussarde}{%
\subsubsection{ソース・ユサルド}\label{sauce-hussarde}}

\frsub{Sauce Hussarde}\footnote{もとはハンガリーで農家20戸につき1人の割合で招集された騎兵
  hussard を指す。この語は16世紀まで遡ることが出来るが、のちに「乱暴
  者」といったニュアンスでも使われるようになった。à la hussarde は
  「乱暴に、粗野に」の意味でも用いられるが、料理においてはレフォール
  を使ったものに名付けられることが多い。}

\index{そーす@ソース!ふらうんはせい@ブラウン系の派生---!ゆさると@---・ユサルド}
\index{そーす@ソース!ゆさると@---・ユサルド}
\index{ゆさると@ユサルド!そーす@ソース・---}
\index{sauce@sauce!petites brunes composees@Petites ---s Brunes Composées!hussarde@--- Hussarde}
\index{sauce@sauce!hussarde@--- Hussarde}
\index{hussard@Hussard(e)!sauce@Sauce Hussarde}

玉ねぎ2個とエシャロット2個を細かくみじん切りにして、バターで色よく炒め
る。白ワイン4
dlを注ぎ、半量になるまで煮詰める。\protect\hyperlink{sauce-demi-glace}{ソース・ドゥミグラ
ス}4 dlとトマトピュレ大さじ2杯、\protect\hyperlink{fonds-blanc}{白いフォ
ン}2 dl、生ハムの脂身のないところ80 g、潰した
にんにく1片、ブーケガルニを加える。弱火で25〜30分煮込む。

ハムを取り出して、ソースをスプーンで押すようにして布で漉す。

火にかけて温め、小さなさいの目\footnote{brunoise ブリュノワーズ。}に刻んだハムと、おろしたレフォール
\footnote{raifort (レフォール)いわゆる西洋わさび、ホースラディッシュ。}少々、パセリのみじん切りをたっぷり1つまみ加えて仕上げる。

\ldots{}\ldots{}牛、羊肉のグリルまたは串を刺してローストしてアントレ\footnote{通常、ローストは料理区分としてアントレに含められることはないが、
  牛フィレは牛の部位のなかでも比較的小さいものとして、まるごと1本の
  ローストであっても原則的にはアントレに分類される。このソースを用い
  る「牛フィレ ユサルド」は牛フィレの塊に串を刺してローストし、ポム・
  デュシェスとマッシュルームを合わせる。}として供 する際に用いる。

\hypertarget{sauce-italienne}{%
\subsubsection{イタリア風ソース}\label{sauce-italienne}}

\frsub{Sauce Italienne}\footnote{この「イタリア風」には根拠も由来も見出すことが出来ない。地名、
  国名を料理名に冠した代表例のひとつ。}

\index{そーす@ソース!ふらうんはせい@ブラウン系の派生---!イタリア風@---}
\index{そーす@ソース!いたりあふう@イタリア風---}
\index{いたりあふう@イタリア風!そーす@---ソース}
\index{sauce@sauce!petites brunes composees@Petites ---s Brunes Composées!italienne@--- Italienne}
\index{sauce@sauce!Italienne@--- Italienne}
\index{italien@italien(ne)!sauce@Sauce Italienne}

トマトの風味の効いた\protect\hyperlink{sauce-demi-glace}{ソース・ドゥミグラ
ス}\troisquarts{}
Lに、\protect\hyperlink{duxelles-seche}{デュクセル・セッシュ}大さ
じ4杯と、加熱ハムの脂身のないところを小さなさいの目に切ったもの125 gを
加える。5〜6分間煮る。提供直前に、パセリとセルフイユ、エスゴラゴンのみ
じん切り大さじ1杯を加えて仕上げる。

\ldots{}\ldots{}いろいろな肉料理に合わせる。

\hypertarget{nota-sauce-italienne}{%
\subparagraph{【原注】}\label{nota-sauce-italienne}}

このソースを魚料理に合わせる場合、ハムは使わずに\protect\hyperlink{fumet-de-poisson}{魚のフュ
メ}を煮詰めて加える。

\hypertarget{jus-lie-a-lestragon}{%
\subsubsection{とろみを付けたジュ エストラゴン風味}\label{jus-lie-a-lestragon}}

\frsub{Jus lié à l'Estragon}

\index{そーす@ソース!ふらうんはせい@ブラウン系の派生---!とろみをつけたしゆえすとらこん@とろみを付けたジュ エストラゴン風味}
\index{そーす@ソース!とろみをつけたしゆえすとらこん@とろみを付けたジュエストラゴン風味}
\index{しゆ@ジュ!とろみをつけたえすとらごん@とろみを付けた--- エストラゴン風味}
\index{sauce@sauce!petites brunes composees@Petites ---s Brunes Composées!jus lie estragon@Jus lié à l'estragon}
\index{sauce@sauce!jus lie estragon@Jus lié à l'Estragon}
\index{estragon@estragon!jus lie estragon@Jus lié à l'Estragon}
\index{jus@jus!lie estragon@--- lié à l'Estragon}

\protect\hyperlink{jus-de-veau-brun}{仔牛のフォン}または\protect\hyperlink{fonds-de-volaille}{鶏のフォ
ン}に、エストラゴン50gを加えて香りを煮出し\footnote{imfuser アンフュゼ。}た
もの。

布で漉してから、アロールート\footnote{コーンスターチで代用する。}または、でんぷん30
gでとろみを付ける。

\ldots{}\ldots{}白身肉のノワゼットや家禽のフィレなどに添える。

\hypertarget{jus-lie-tomate}{%
\subsubsection{とろみを付けたジュ トマト風味}\label{jus-lie-tomate}}

\frsub{Jus lié tomaté}

\index{そーす@ソース!ふらうんはせい@ブラウン系の派生---!とろみをつけたしゆとまとふうみ@とろみを付けたジュ トマト風味}
\index{そーす@ソース!とろみをつけたしゆとまと@とろみを付けたジュ トマト風味}
\index{しゆ@ジュ!とろみをつけたとまとふうみ@とろみを付けた--- トマト風味}
\index{sauce@sauce!petites brunes composees@Petites ---s Brunes Composées!jus lie tomatee@Jus lié tomatée}
\index{sauce@sauce!jus lie tomatee@Jus lié tomaté}
\index{tomate@tomate!jus lie tomate@Jus lié tomaté}
\index{jus@jus!lie tomate@--- lié tomaté}

\protect\hyperlink{jus-de-veau-brun}{仔牛のフォン}1
Lあたりトマトエッセンス3 dlを加え、 \quatrecinquiemes{}量まで煮詰める。

\ldots{}\ldots{}牛、羊肉料理用。

\hypertarget{sauce-lyonnaise}{%
\subsubsection{リヨン風ソース}\label{sauce-lyonnaise}}

\frsub{Sauce Lyonnaise}

\index{そーす@ソース!ふらうんはせい@ブラウン系の派生---!りよんふう@リヨン風---}
\index{そーす@ソース!りよんふう@リヨン風---}
\index{りよんふう@リヨン風!そーす@---ソース}
\index{sauce@sauce!petites brunes composees@Petites ---s Brunes Composées!lyonnaise@--- Lyonnaise}
\index{sauce@sauce!lyonnaise@--- Lyonnaise}
\index{liyonnais@lyonnais(e)!sauce lyonnaise@Sauce Lyonnaise}

中位の大きさの玉ねぎ3個をみじん切りにし、バターでじっくり、ごく弱火で
ブロンド色になるまで炒める。白ワイン2 dlとヴィネガー2 dlを注ぐ。
\untiers{}量まで煮詰め、\protect\hyperlink{sauce-demi-glace}{ソース・ドゥミグラ
ス}\troisquarts{} Lを加える。5〜6分かけて表面に浮い
てくる不純物を丁寧に取り除き\footnote{dépouiller
  デプイエ。現代ではエキュメと呼ぶ現場が多い。}、布で漉す。

\hypertarget{ux539fux6ce8-nota-sauce-lyonnaise}{%
\subparagraph{【原注】
\{nota-sauce-lyonnaise\}}\label{ux539fux6ce8-nota-sauce-lyonnaise}}

このソースを合わせる料理によっては、ソースを布で漉さずに玉ねぎを残して
もいい。

\hypertarget{sauce-madere}{%
\subsubsection{ソース・マデール}\label{sauce-madere}}

\frsub{Sauce Madère}

\index{そーす@ソース!ふらうんはせい@ブラウン系の派生---!まてーる@---・マデール}
\index{そーす@ソース!までーる@---・マデール}
\index{までらしゆ@マデラ酒 ⇒ マデール!そーす@ソース・マデール}
\index{sauce@sauce!petites brunes composees@Petites ---s Brunes Composées!madere@--- Madère}
\index{sauce@sauce!madere@--- Madère}
\index{madere@madère!sauce@Sauce ---}

\protect\hyperlink{sauce-demi-glace}{ソース・ドゥミグラス}を煮詰め\footnote{ソース・ドゥミグラスは既に煮詰めて仕上がった状態のものなので、9
  割程度にまでしか煮詰めないことに注意。}、火から外して、 ソース1
Lあたりマデラ酒1 dlの割合で加え、普通の濃度にする。

\hypertarget{sauce-matelote}{%
\subsubsection{ソース・マトロット}\label{sauce-matelote}}

\frsub{Sauce Matelote}\footnote{水夫風、船員風、の意。トゥーレーヌ地方の郷土料理Matelote
  d'anguille(マトロットダンギーユ)うなぎの赤ワイン煮込み、が有名。
  とはいえ本書にも数種のレシピが収録されているように、赤ワイン煮込み
  にとどまらず、マトロットの名称を持つ料理は他にも複数存在する。}

\index{そーす@ソース!ふらうんはせい@ブラウン系の派生---!まとろつと@---・マトロット}
\index{そーす@ソース!まとろつと@---・マトロット}
\index{まとろつと@マトロット!そーす@ソース・---}
\index{sauce@sauce!petites brunes composees@Petites ---s Brunes Composées!matelote@--- Matelote}
\index{sauce@sauce!matelote@--- Matelote}
\index{matelote@matelote!sauce@Sauce ---}

魚をポシェするのに使った\protect\hyperlink{court-bouillon-c}{赤ワイン入りの魚用クールブイヨン}3
dlにマッ シュルームの切りくず25
gを加え、\untiers{}量になるまで煮詰める。

煮詰めたら\protect\hyperlink{sauce-espagnole-maigre}{魚料理用ソース・エスパニョル}8
dl を加えてひと煮立ちさせる。布で漉し、バター150 gとごく少量のカイエンヌ
の粉末を加えて仕上げる。

\hypertarget{sauce-moelle}{%
\subsubsection{ソース・モワル}\label{sauce-moelle}}

\frsub{Sauce Moelle}\footnote{moelle 骨髄のこと。}

\index{そーす@ソース!ふらうんはせい@ブラウン系の派生---!もわる@---・モワル}
\index{そーす@ソース!もわる@---・モワル}
\index{こつずい@骨髄 ⇒ モワル!そーすもわる@ソース・モワル}
\index{sauce@sauce!petites brunes composees@Petites ---s Brunes Composées!moelle@--- Moelle}
\index{sauce@sauce!moelle@--- Moelle}
\index{moelle@moelle!sauce@Sauce ---}

ソースの作り方は\protect\hyperlink{sauce-bordelaise}{ボルドー風ソース}とまったく同じだ
が、バターを加えるのは何らかの野菜料理に添える場合のみであり、その場合
のバターの量は通常どおりとするこ。

どんな場合にせよ、仕上げに、小さなさいの目に切ってポシェしておいた骨髄
をソース1 Lあたり150〜180 gおよび刻んで下茹でしたパセリの葉小さじ1杯を
加える。

\hypertarget{sauce-moscovite}{%
\subsubsection{モスクワ風ソース}\label{sauce-moscovite}}

\frsub{Sauce Moscovite}\footnote{moscovite(モスコヴィット)すなわちモスクワ風の名称を持つ料理や
  菓子は多い。 18世紀後半から19世紀前半にかけて、ロシアの宮廷や貴族
  らの間でフランスの食文化が流行し、多くのフランス人料理人が招聘され、
  彼らはロシア料理のレシピをフランスに持ち帰った。クーリビヤックなど
  が代表的な例だろう。また、19世紀後半になると、とりわけフランス料理
  においてもロシア料理からの影響が多く見られるようになる。キャビアと
  ウォトカを食前に愉しむのが流行したのもその時代からである。フランス
  とロシアの食文化は相互に影響関係にあったと言えよう。}

\index{そーす@ソース!ふらうんはせい@ブラウン系の派生---!もすくわふう@モスクワ---}
\index{そーす@ソース!もすくわふう@モスクワ風---}
\index{もすくわふう@モスクワ風!そーす@---ソース}
\index{sauce@sauce!petites brunes composees@Petites ---s Brunes Composées!moscovite@--- Moscovite}
\index{sauce@sauce!moscovite@--- Moscovite}
\index{moscovite@moscovite!sauce@Sauce ---}

\protect\hyperlink{fonds-de-gibier}{大型ジビエのフュメ}で作った\protect\hyperlink{sauce-poivrade}{ソース・ポワヴラー
ド}を\troisquarts{} L用意する。提供直前にマラガ酒1 dl
とジェニパーベリーを煎じた汁7 cl\footnote{1 cl = 10
  ml、つまりこの場合は70 ml。}、焼いた松の実かスライスして焼いたアー
モンド40 g、大きさを揃えてぬるま湯でもどしておいたコリント産干しぶどう
\footnote{小粒で黒いギリシア産干しぶどう。}40 gを加えて仕上げる。

\ldots{}\ldots{}大型ジビエ\footnote{venaison
  ヴネゾン。ジビエのうち大型のものを指す。実際は
  ノロ鹿や猪を指すことがほとんど。}の塊肉の料理用。

\hypertarget{sauce-perigueux}{%
\subsubsection{ソース・ペリグー}\label{sauce-perigueux}}

\frsub{Sauce Périgueux}\footnote{トリュフの産地として有名なペリゴール地方の町の名。}

\index{そーす@ソース!ふらうんはせい@ブラウン系の派生---!へりくー@---・ペリグー}
\index{そーす@ソース!へりくー@---・ペリグー}
\index{へりくー@ペリグー!そーす@ソース・---}
\index{sauce@sauce!petites brunes composees@Petites ---s Brunes Composées!perigueux@--- Périgueux}
\index{sauce@sauce!perigueux@--- Périgueux}
\index{perigueux@Périgueux!sauce@Sauce ---}

やや濃いめに煮詰めた\protect\hyperlink{sauce-demi-glace}{ソース・ドゥミグラ
ス}\troisquarts{} Lに、トリュフエッセンス1 \undemi{}
dlと細かく刻んだトリュフ100 gを加える。

\ldots{}\ldots{}いろいろな肉料理、\protect\hyperlink{}{タンバル}、\protect\hyperlink{}{温製パテ}に合わせる。

\hypertarget{sauce-perigourdine}{%
\subsubsection{ソース・ペリグルディーヌ}\label{sauce-perigourdine}}

\frsub{Sauce Périgourdine}\footnote{ペリゴール地方風の意。}

\index{そーす@ソース!ふらうんはせい@ブラウン系の派生---!へりくるていーぬ@---・ペリグルディーヌ}
\index{そーす@ソース!へりくるていーぬ@---・ペリグゥルディーヌ}
\index{へりこーるふう@ペリゴール風 ⇒ ペリグルダン/ペリグルディーヌ!そーす@ソース・ペリグルディーヌ}
\index{sauce@sauce!petites brunes composees@Petites ---s Brunes Composées!perigourdine@--- Périgourdine}
\index{sauce@sauce!perigourdine@--- Périgourdine}
\index{perigourdin@périgourdin(e)!sauce@Sauce Périgourdine}

ソース・ペリグーのバリエーション。トリュフを細かく刻むのではなく、オリー
ブ形か小さな真珠のような形状にナイフで成形\footnote{tourner
  トゥルネ。包丁を持っている側の手は動かさずに材料を回す
  ようにして形を整えること。}したものを加える。トリュ
フを厚めにスライスして加える場合もある。

\hypertarget{sauce-piquante}{%
\subsubsection{ソース・ピカント}\label{sauce-piquante}}

\frsub{Sauce Piquante}\footnote{piquant(e) (ピカン、ピカント)
  一般的には唐辛子などが「辛い」
  の意だが、このソースでは唐辛子の類は使われておらず、むしろ酸味の効
  いたソースと言えよう。古くからのソース名。}

\index{そーす@ソース!ふらうんはせい@ブラウン系の派生---!ひかんと@---・ピカント}
\index{そーす@ソース!ひかんと@---・ピカント}
\index{sauce@sauce!petites brunes composees@Petites ---s Brunes Composées!piquante@--- piquante}
\index{sauce@sauce!piquante@--- Piquante}
\index{piquant@piquant(e)!sauce@Sauce Piquante}

白ワイン3 dlと良質のヴィネガー3 dlにエシャロットのみじん切り大さじ2
\undemi{}杯を合わせて半量に煮詰める。

\protect\hyperlink{sauce-espagnole}{ソース・エスパニョル}6
dlを加え、浮いてくる不純物を 取り除きながら\footnote{dépouiller
  デプイエ。エキュメécumerと呼ぶ現場も多い。}10分間煮る。

火から外し、コルニション\footnote{専用品種のきゅうりを小さなうちに収穫して酢漬けにしたもの。同様
  のピクルス用きゅうりとしてガーキンスという品種系統があるがもっぱら
  アメリカのハンバーガーに挟まれるようなサイズで収穫して漬けたもので
  あり、フランス料理では用いない。}、パセリ、セルフイユ、エストラゴンを細か
く刻んだもの大さじ2杯を加えて仕上げる。

\ldots{}\ldots{}豚肉のグリル焼き、ブイイ\footnote{bouilli
  茹で肉。もとはブイヨンをとった後の茹で肉のことを指した。
  単純に「茹でた肉」としてもいいのだが、17世紀にはこの食べ方が流行し
  たという歴史もあり、野菜などと共に、あるいは他の素材なしに茹でた肉
  はたんに「ブイイ」bouilli と呼ばれる。}、ローストによく合わせるソース。牛肉
のブイイや牛や羊の\protect\hyperlink{}{エマンセ}にも合わせることが出来る。

\hypertarget{sauce-poivrade}{%
\subsubsection{ソース・ポワヴラード (標準)}\label{sauce-poivrade}}

\frsub{Sauce Poivrade ordinaire}\footnote{このソースは遅くとも16世紀まで遡ることが出来る。1505年に出版さ
  れた\href{http://gallica.bnf.fr/ark:/12148/bpt6k792720}{『フランス語版プラティ
  ナ』}がpoivradeとい
  うフランス語の初出。この本において「ジビエ用こしょうのソース、ポワ
  ヴラード」Saulce de poyvre ou poyvrade pour saulvagieとしてレシピ
  が見られる。パンをよく焼いてヴィネガーに浸してすり潰す。水でもどし
  た干しぶどうと獣の血を加えて混ぜ、玉ねぎと未熟ぶどう果汁、パンを浸
  した残りのヴィネガーを加えて漉し器か布で漉す。これを鍋に入れ、こしょ
  う、生姜、シナモンを入れて炭火の上で30分程煮込む。獣の肉を獣脂を熱
  したフライパンで焼き、皿に盛る。上からポワヴラードをかけて供する、
  という内容(f.LXII)。またこの本には、魚料理用のポワヴラードも掲載
  されている。ただし、これが現代まで続くソース・ポワヴラードの原型と
  捉えるのは早計に過ぎる。ここで注目すべきは、最終的に肉あるいは魚の
  ような主素材とソースが一体化したものは中世〜ルネサンス期にはポター
  ジュと呼ばれていたのに対し、ここではソースを別のものと捉えている点
  である。ポワヴラードという語そのものは「こしょうを効かせたもの」と
  いう意味に過ぎず、1660年刊ピエール・ド・リュヌPierre de Lune『新フ
  ランス料理』におけるPoivrade de pigeonneaux 若鳩のポワヴラードは、
  背開きにした若鳩を平たくのばし、塩、こしょう をして弱火でグリルす
  る。薔薇の香りもしくはにんにく風味のヴィネガーを添えて供する、とい
  うもの(p.190)。ピエール・ド・リュヌのレシピにおいてソースに相当す
  るものはヴィネガーであり、むしろ味付けでこしょうを効かせているとい
  うことが料理名の根拠となっているに過ぎない。ちなみに、生食可能な小
  さなサイズのアーティチョークも古くからポワヴラードと呼ばれている。}

\index{そーす@ソース!ふらうんはせい@ブラウン系の派生---!ほわうらーと@---・ポワヴラード(標準)}
\index{そーす@ソース!ほわうらーと@---・ポワヴラード(標準)}
\index{ほわうらーと@ポワヴラード!そーす@ソース・---(標準)}
\index{sauce@sauce!petites brunes composees@Petites ---s Brunes Composées!poivrade ordinaire@--- Poivrade}
\index{sauce@sauce!poivrade ordinaire@--- Poivrade ordinaire}
\index{poivrade@poivrade!sauce ordinaire@Sauce --- ordinaire}

細かいさいの目に切ったにんじん100 gと玉ねぎ80 g、刻んだパセリの茎、タ
イム少々、ローリエの葉少々からなる\protect\hyperlink{mirepoix}{ミルポワ}を油で色付くま
で炒める。

ヴィネガー1 dlとマリナード2 dlを注ぎ、\untiers{}量になるまで煮詰める。
\protect\hyperlink{sauce-espagnole}{ソース・エスパニョル}1
Lを注ぎ、約45分間煮込む。

ソースを漉す10分前に、大粒のこしょう8個を叩きつぶして加える。ソースに
こしょうを入れてからの時間がこれ以上少しでも長いと、こしょうの風味が支
配的になり過ぎることになるので注意。

漉し器で香味素材を軽く押すようにして漉す。\protect\hyperlink{marinades-et-saumuresux5cux257D}{マリナード}\footnote{ヴィネガーやワイン、香味素材、塩などを合わせて肉を漬け込む液体。
  マリネ液と呼ぶこともある。}2 dlでソー
スをのばす。火にかけて35分間、所定の量\footnote{明記されていないが、ここでは約1
  L。}になるまで煮詰めながら、表
面に浮いてくる不純物を徹底的に取り除く\footnote{dépouiller
  デプイエ。現代ではécumerエキュメの語を使う現場が多い。}。

さらに布で漉し、バター50 gを加えて仕上げる\footnote{現代では、バターでモンテするmonter
  au beurreという表現を用いる 現場も多い。}。

\hypertarget{sauce-poivrade-pour-gibier}{%
\subsubsection{ソース・ポワヴラード(ジビエ用)}\label{sauce-poivrade-pour-gibier}}

\frsub{Sauce Poivrade pour Gibier}

\index{そーす@ソース!ふらうんはせい@ブラウン系の派生---!ほわうらーとしひえ@---・ポワヴラード(ジビエ用)}
\index{そーす@ソース!ほわうらーとしひえ@---・ポワヴラード(ジビエ用)}
\index{ほわうらーと@ポワヴラード!そーすしひえ@ソース・---(ジビエ用)}
\index{sauce@sauce!petites brunes composees@Petites ---s Brunes Composées!poivrade gibier@--- Poivrade pour Gibier}
\index{sauce@sauce!poivrade gibier@--- Poivrade pour Gibier}
\index{poivrade@poivrade!sauce  gibier@Sauce --- pour Gibier}

細かいさいの目に切ったにんじん125 gと玉ねぎ125 g、タイムの枝と鳥類では
ないジビエ\footnote{gibier à poil
  逐語訳すると「毛の生えているジビエ」すなわち」鹿、
  猪、野うさぎなどを指す。}の端肉1
kgからなる\protect\hyperlink{mirepoix}{ミルポワ}を油で色よく炒め る。

ミルポワが色付いてきたら、鍋の油を捨てる。ヴィネガー3 dlと白ワイン2 dl
を注ぎ、完全に煮詰める。

\protect\hyperlink{sauce-espagnole}{ソース・エスパニョル}1
Lと\protect\hyperlink{fonds-de-gibier}{ジビエの茶色いフォン}2 L、
\protect\hyperlink{marinades-et-saumures}{マリナード}1 Lを加える。

鍋に蓋をして弱火にかける。可能ならオーブンがいい。3時間半〜4時間加熱す
る。

ソースを漉す8分前に、大粒のこしょう12個を叩きつぶして加える。

漉し器で材料を押すようにして漉す。

これをジビエのフォン\unquart{} Lとマリナード\unquart{} Lでのばし、再び
火にかけて40分間、表面に浮いてくる不純物を丁寧に取り除きながら、1 Lに
なるまで煮詰める。

これを布で漉し、バター75gを加えて仕上げる。

\hypertarget{sauce-poivrade-pour-gibier}{%
\subparagraph{【原注】}\label{sauce-poivrade-pour-gibier}}

一般的にはジビエ料理のソースにはバターを加えないことになっているが、本
書では軽くバターを加えることを推奨する。そうすると、ソースの色の赤みは
薄まるが、繊細で滑らかな口あたりに仕上がる。

\hypertarget{sauce-au-porto}{%
\subsubsection{ソース・ポルト}\label{sauce-au-porto}}

\frsub{Sauce au Porto}

\index{そーす@ソース!ふらうんはせい@ブラウン系の派生---!ほると@---・ポルト}
\index{そーす@ソース!ほると@---・ポルト}
\index{ほるとしゆ@ポルト酒 ⇒ ポルト!そーす@ソース・---}
\index{sauce@sauce!petites brunes composees@Petites ---s Brunes Composées!porto@--- Porto}
\index{sauce@sauce!porto@--- au Porto}
\index{porto@Porto!sauce@Sauce au ---}

マデラ酒ではなくポルト酒を用いて、\protect\hyperlink{sauce-madere}{ソース・マデール}と
同様に作る。

\hypertarget{sauce-portugaise}{%
\subsubsection{ポルトガル風ソース}\label{sauce-portugaise}}

\frsub{Sauce Portugaise}\footnote{日本でもフランス語のままソース・ポルチュゲーズと呼ばれることは
  多い。フランス料理においてポルトガル風の名称を付けた料理はトマトをベー
  スとしたものがほとんど。ただし、トマトを使うからといってポルトガル風の
  名が必ず付くというわけではない。}

\index{そーす@ソース!ふらうんはせい@ブラウン系の派生---!ほるとかるふう@ポルトガル風---}
\index{そーす@ソース!ほるとかるふう@ポルトガル風---}
\index{ほるとかるふう@ポルトガル風!そーす@---ソース}
\index{sauce@sauce!petites brunes composees@Petites ---s Brunes Composées!portugaise@--- Portugaise}
\index{sauce@sauce!porugaise@--- Portugaise}
\index{portugais@portugais(e)!sauce@Sauce Portugaise}

(仕上がり1 L分)

大きめの玉ねぎ1個を細かくみじん切りにする。鍋に油を熱し、強火で玉ねぎ
を炒める。玉ねぎがブロンド色になったら、皮を剥いて種子を取り除き、粗み
じん切りにしたトマト750 gと、つぶしたにんにく1片、塩、こしょうを加える。
トマトの酸味が強い場合は砂糖少々も加える。鍋に蓋をして、弱火で煮る。
\protect\hyperlink{essences-diverses}{トマトエッセンス}少々と、薄めに作ったトマトソース
を適量\footnote{仕上がりの全体量が1
  Lなので、トマトソースを加える量は、グラスドヴィアンド
  を加える前の段階で0.9 L程度になるよう調整する。}、温めて溶かした\protect\hyperlink{glace-de-viande}{グラスドヴィアンド}1
dl、 新鮮なパセリの葉のみじん切り大さじ1杯を加えて仕上げる。

\hypertarget{sauce-provencal}{%
\subsubsection{プロヴァンス風ソース}\label{sauce-provencal}}

\frsub{Sauce Provençale}

\index{そーす@ソース!ふらうんはせい@ブラウン系の派生---!ふろうあんすふう@プロヴァンス風---}
\index{そーす@ソース!ふろうあんすふう@プロヴァンス風---}
\index{ふろうあんすふう@プロヴァンス風!そーす@---ソース}
\index{sauce@sauce!petites brunes composees@Petites ---s Brunes Composées!provencale@--- Provençale}
\index{sauce@sauce!provencale@--- Provençale}
\index{provencal@provençal(e)!sauce@Sauce Provençale}

大ぶりのトマト12個の皮を剥き、つぶして種子は取り除いて、粗く刻む\footnote{concasser
  コンカセ。}。 ソテー鍋に2\undemi{}
dlの油を熱し、そこにトマトを入れる。塩、こしょう、
粉砂糖1つまみで味を調える。しっかりつぶしたにんにく(小)1片と細かく刻
んだパセリ小さじ1杯を加える。

蓋をして弱火で30分間程、煮溶かす。

\hypertarget{nota-sauce-provencale}{%
\subparagraph{【原注】}\label{nota-sauce-provencale}}

このソースについてはさまざまな解釈があるが、本書ではブルジョワ料理にお
ける本物の「プロヴァンス風ソース」のレシピ、つまりはトマトの「フォン
デュ」\footnote{加熱によって溶かしたもの、の意。トマトフォンデュと呼ぶ調理現場も多い。}、を収録した。

\hypertarget{sauce-regence}{%
\subsubsection{ソース・レジャンス}\label{sauce-regence}}

\frsub{Sauce Régence}\footnote{Régence(レジョンス)とは「摂政時代」、すなわちオルレアン公フィリップが幼少だったルイ15世の
  摂政を務めた時代(1715〜1723年)のこと。オルレアン公は美食家として
  有名で、とりわけシャンパーニュを好んだという。この時代はフランス宮
  廷料理の絶頂期でもあった。}

\index{そーす@ソース!ふらうんはせい@ブラウン系の派生---!れしやんす@---・レジャンス}
\index{そーす@ソース!れしやんす@---・レジャンス}
\index{れしやんす@レジャンス!そーす@ソース・---}
\index{sauce@sauce!petites brunes composees@Petites ---s Brunes Composées!regence@--- Régence}
\index{sauce@sauce!regence@--- Régence}
\index{regence@Régence!sauce@Sauce ---}

ライン産ワイン3
dlに、細かく刻んであらかじめ火を通しておいた\protect\hyperlink{mirepoix}{ミルポ
ワ}1 dlと生トリュフの切りくず25gを加え、半量になるまで煮詰
める。トリュフのシーズンでない時季はトリュフエッセンスを使う。\protect\hyperlink{sauce-demi-glace}{ソース・
ドゥミグラス}8 dlを加え、数分間弱火にかけて浮いてく
る不純物を丁寧に取り除き\footnote{dépouiller デプイエ ≒ écumer
  エキュメ。}、布で漉す。

\ldots{}\ldots{}牛、羊の大きな塊肉の料理用。

\hypertarget{sauce-robert}{%
\subsubsection{ソース・ロベール}\label{sauce-robert}}

\frsub{Sauce Robert}\footnote{この名称のソースは古くからある。文献で初めて出てくるのは16世紀
  フランソワ・ラブレーの小説『ガルガンチュアとパンタグリュエル』。そ
  の「第四の書」で料理人の名が大量に列挙される章がある。そのうちの多
  くは架空の人名だが、その中のロベールという料理人がこのソースを考案
  したと書いている。ただし、具体的にどのようなソースかまでは描写され
  ておらず「うさぎのロースト、鴨、加工していない豚肉、卵のポシェ、塩
  漬けのメルラン{[}鱈の近縁種{]}、その他まことに多くの料理に欠かせない
  ソース」と書いてあるのみ(第40章)。どんな料理にも合うと書かれてし
  まうとむしろ特徴を捉え難くなってしまう。いずれにせよ、遅くとも16世
  紀には「ソース」として成立していたと考えられる。また、17世紀のシャ
  ルル・ペロー著『物語集』の「眠れる森の美女」においても、このソース
  名が登場する一節がある。このように16世紀以降多くの文学作品をはじめ
  とする文献にこのソース名は見られる。レシピとしては、1651年刊ラ・ヴァ
  レーヌ『フランス料理の本』における「豚腰肉 ソース・ロベール添え」
  がもっとも古いもののひとつだろう。概略は、豚腰肉を、ヴェルジュ{[}未
  熟ぶどう果汁、中世料理においてよく用いられた{]}とヴィネガー、セージ
  を振り掛けながらローストする。下に置いた脂受け皿に焼いた豚肉から流
  れ落ちた脂がたまるので、これを使って玉ねぎをこんがり炒める。炒めた
  玉ねぎの上に豚後ろ身を載せ、豚腰肉をローストする際にかけたのと同じ
  ソースをかける。このソースはソースロベールと呼ばれている(p.51)。ま
  た、干鱈のソース・ロベール添えの場合は、バターとヴェルジュ少々、マ
  スタードで作るが、ケイパーやシブール{[}葱{]}を加えてもいい(p.202)と
  あり、同じ名称のソースとは見做しがたい。18世紀以降のソース・ロベー
  ルは多かれ少なかれいずれもマスタードを加える点が共通しているので、
  名称が先にあり、内容が時代とともにはっきりしたものになっていたのだ
  ろう。}

\index{そーす@ソース!ふらうんはせい@ブラウン系の派生---!ろへーる@---・ロベール}
\index{そーす@ソース!ろへーる@---・ロベール}
\index{ろへーる@ロベール!そーす@ソース・---}
\index{sauce@sauce!petites brunes composees@Petites ---s Brunes Composées!Robert@--- Robert}
\index{sauce@sauce!robert@--- Robert}
\index{robert@Robert!sauce@Sauce ---}

(仕上がり5 dl分)

大きめの玉ねぎを細かくみじん切りにし、バターで色付かないよう強火でさっ
と炒める。

白ワイン2
dlを注ぎ、\untiers{}量になるまで煮詰める。\protect\hyperlink{sauce-demi-glace}{ソース・ドゥミグ
ラス}3 dlを加え、弱火で10分間煮る。

シノワ\footnote{主として金属製で円錐形に取っ手の付いた漉し器。清朝の高級役人が
  かぶっていた帽子の形状から「中国の」を意味するchinoisの名称となっ
  たと言われている。}で漉し(これは任意。漉さなくてもいい)、火から外して、粉砂
糖1つまみとマスタード大さじ1杯を加えて仕上げる。

\hypertarget{sauce-robert-escoffier}{%
\subsubsection{ソース・ロベール・エスコフィエ}\label{sauce-robert-escoffier}}

\frsub{Sauce Robert Escoffier}

\index{そーす@ソース!ふらうんはせい@ブラウン系の派生---!ろへーるえすこふいえ@---・ロベール・エスコフィエ}
\index{そーす@ソース!ろへーるえすこふいえ@---・ロベール・エスコフィエ}
\index{ろへーる@ロベール!そーすえすこふいえ@ソース・---・エスコフィエ}
\index{sauce@sauce!petites brunes composees@Petites ---s Brunes Composées!robert escoffier@--- Robert Escoffier}
\index{sauce@sauce!robert escoffier@--- Robert Escoffier}
\index{robert@Robert!sauce escoffier@Sauce --- Escoffier}

このソースは完成品が市販されている\footnote{\protect\hyperlink{sauce-diable-escoffier}{ソース・ディアーブル・エスコフィエ}訳注参照。}。

温かい料理にも冷たい料理にもよく合う。温かい料理に合わせる場合は、同量
の\protect\hyperlink{jus-de-veau-brun}{仔牛の茶色いフォン}と混ぜること。

\ldots{}\ldots{}豚、仔牛、鶏、魚のグリル焼きに特によく合う。

\hypertarget{sauce-romaine}{%
\subsubsection{ローマ風ソース}\label{sauce-romaine}}

\frsub{Sauce Romaine}\footnote{フランス料理における「ローマ風」の名称は「イタリア風」と同様に
  とくに根拠や由来が見出せないものが多い。このソースの場合は松の実を
  使うところから、20世紀前半に活躍したイタリアの作曲家レスピーギのロー
  マ三部作のうちの「ローマの松」を想起させるが、残念ながらこの曲が作
  曲されたのは1924年、つまり本書より後なので関係はない。だが、松の実
  を採るイタリアカサマツは、アッピア街道の並木などで有名なように、イ
  タリアとりわけローマ近辺において多く見られる(だからこそレスピーギ
  が曲の題材にしたわけだが)。その意味においては、松の実を使っている
  ということがこのソース名の根拠と見ることも不可能ではないだろう。し
  かしながら、それを証明する文献、史料があるかは不明。}

\index{そーす@ソース!ふらうんはせい@ブラウン系の派生---!ろーまふう@ローマ風---}
\index{そーす@ソース!ろーまふう@ローマ風---}
\index{ろーまふう@ローマ風!そーす@---ソース}
\index{sauce@sauce!petites brunes composees@Petites ---s Brunes Composées!romaine@--- Romaine}
\index{sauce@sauce!romain@--- Romaine}
\index{romain@romain(e)!Sauce Romaine}

砂糖50 gを火にかけてブロンド色にカラメリゼ\footnote{焦がさないように弱火で混ぜながら熱で砂糖を溶かしていく。}する。これをヴィネガー
1\undemi{}
dlでのばす。砂糖を完全に溶かし込めたら、\protect\hyperlink{sauce-espagnole}{ソース・エスパニョ
ル}6 dlと\protect\hyperlink{fonds-de-gibier}{ジビエのフォン}3 dlを加
える。これを\troisquarts{}量弱まで煮詰める。布で漉し、松の実20 gをロー
ストしたものと、大きさが揃るよう選別したスミヌル干しぶどう\footnote{トルコ産の白い干しぶどう。}20
gお よびコリント干しぶとう\footnote{ギリシア産の黒い小粒の干しぶどう(\protect\hyperlink{sauce-moscovite}{モスクワ風ソー
  ス}参照)。}20 gを温湯でもどしたものを加えて仕上げる。

\hypertarget{ux539fux6ce8-1}{%
\subparagraph{【原注】}\label{ux539fux6ce8-1}}

上記のとおり作る場合、このソースは大型ジビエ料理用だが、ジビエのフォン
ではなく通常の\protect\hyperlink{fonds-brun}{茶色いフォン}を使えば、マリネした牛、羊肉
の料理に合わせることも可能。

\hypertarget{sauce-rouennaise}{%
\subsubsection{ルーアン風ソース}\label{sauce-rouennaise}}

\frsub{Sauce Rouennaise}\footnote{ルーアンは野生のcolvertコルヴェール、いわゆる青首鴨を家禽化した
  ルーアン鴨の産地として有名。}

\index{そーす@ソース!ふらうんはせい@ブラウン系の派生---!るーあんふう@ルーアン風---}
\index{そーす@ソース!るーあんふう@ルーアン風---}
\index{るーあんふう@ルーアン風!そーす@---ソース}
\index{sauce@sauce!petites brunes composees@Petites ---s Brunes Composées!rouannaise@--- Rouannaise}
\index{sauce@sauce!rouannaise@--- Rouannaise}
\index{rouannais@rouannais(e)!sauce@Sauce Rouannaise}

(仕上がり5 dl分)

\protect\hyperlink{sauce-bordelaise}{ボルドー風ソース}4 dl
を用意する。ただし、良質な赤
ワインを使って作ること。(\protect\hyperlink{sauce-bordelaise}{ボルドー風ソース}参照)。

中位の大きさの鴨のレバー3個を裏漉しする。こうして出来たレバーのピュレ
をソースに加え、沸騰させない程度の温度で火を通す\footnote{pocher
  ポシェする。}。絶対に沸騰させ
ないこと。沸騰させてしまうと途端にレバーのピュレが粒状になってしまう。

布で漉し、塩こしょうを効かせる。

このソースの特質\ldots{}\ldots{}エシャロットを加えた赤ワインを煮詰めたものに鴨の生
レバーのピュレを加えたもの。

\ldots{}\ldots{}ルーアン産鴨のローストには、いわば必須といってもいいソース。

\hypertarget{sauce-salmis}{%
\subsubsection{ソース・サルミ}\label{sauce-salmis}}

\frsub{Sauce Salmis}\footnote{語源は「ごった煮」を意味する salmigondis
  とするのが定説のようだ
  が、salmigondisがその意味で用いられるようになったのは19世紀以降と
  考えられ、それ以前はragoûtラグーと同義と見なされていた。ラグーはそ
  の語源的意味が「食欲をそそるもの」であり、17世紀に、それまでポター
  ジュと呼ばれていた煮込み料理についてラグーの名称をつけることが流行
  した。また、salmigondisの古い語形のひとつsalmigondinは16世紀の小説
  家フランソワ・ラブレー『ガルガンチュアとパンタグリュエル』の「第四
  の書」において用いられているが、日本語の「ごった煮」のニュアンスと
  はかなり違う意味で、美味な料理のひとつとして挙げられている。いずれ
  にしても、salmigondin, salmigondisというラグーの別称が、ある時期か
  ら鳥類を材料にしたものに限定されるようになったことは確かで、カレー
  ムの『19世紀フランス料理』ではsalmisの語で、野鳥などのラグーを呼ん
  でいる。例えば「ベカスのサルミ」「ペルドローのサルミ」など。カレー
  ムとエスコフィエを比較すると、しばしばカレームにおいてラグーとして
  ひとまとめにされていた料理とソースの組合せが、『料理の手引き』にお
  いては、例えば\protect\hyperlink{garniture-financiere}{ガルニチュール・フィナンシエール}と\protect\hyperlink{sauce-financiere}{ソース・フィ
  ナンシエール}のように、別々の項目に分離されてい るものが多くある。}

\index{そーす@ソース!ふらうんはせい@ブラウン系の派生---!さるみ@---・サルミ}
\index{そーす@ソース!さるみ@---・サルミ}
\index{さるみ@サルミ!そーす@ソース・---}
\index{sauce@sauce!petites brunes composees@Petites ---s Brunes Composées!salmis@--- Salmis}
\index{sauce@sauce!salmis@--- Salmis}
\index{salmis@salmis!sauce@Sauce ---}

ソースというよりはむしろクリ\footnote{coulis \textless{} couler
  クレ「流れる」から派生した語だが、料理用語とし
  ては、やや水分の多いピュレと理解するといい。日本では「クーリ」と呼
  ぶことも多い。ここでは二つの解釈が可能で、ひとつは\protect\hyperlink{}{ポタージュ・
  クリ}に近いという意味。もうひとつは「昔ながらのソース」の意。後
  者の場合、エスコフィエが「古典料理」と呼ぶ17、18世紀においてソース
  のことをクリと呼んでいたのを踏まえていると考えられる。}と呼んだほうがいいこのソースの作り方
はどんな場合も一点を除いて変わることがない。それは、このソースを合わせ
るジビエ(鳥)の種類によって、つまり普通に肉料理として扱えるジビエか、
肉断ち\footnote{小斉のこと。カトリックの習慣として(厳密な教義ではない)四旬節
  (復活祭までの46日間)や毎週金曜などに行なわれる、肉食を断つ行為の
  こと。}の際の食材として扱えるもの\footnote{ある種の水鳥はイルカと同様に魚と同等のものと見做され、小斉の場
  合にも食材として認められていた。具体的にはハシヒロ鴨、オナガ鴨、サ
  ルセル鴨など。もっとも、水鳥を肉断ちの際の食材として扱うというのは
  一種の詭弁ともいえなくないわけで、このソースを作る際に\protect\hyperlink{sauce-espagnole-maigre}{魚料理用ソー
  ス・エスパニョル}をベースとした\protect\hyperlink{sauce-demi-glace}{ソース・
  ドゥミグラス}を使うとは考え難く、本文にあるよう
  にフォンの代用としてマッシュルームの茹で汁を用いるという指示を守るだ
  けで、厳密に小斉の料理として成立するレシピと言えるかは疑問の残ると
  ころだ。}かで、どんな液体を用いるかと いうことだけだ。

細かく刻んだ\protect\hyperlink{mirepoix}{ミルポワ}150
gをバターでじっくり色付くまで炒め
る。そこに、その料理で用いているジビエの手羽と腿の皮、ガラを細かく刻ん
で加える。

白ワイン3
dlを注ぎ、\untiers{}量まで煮詰める。\protect\hyperlink{sauce-demi-glace}{ソース・ドゥミグラ
ス}8 dlを加えて、約45分間弱火で煮込む。漉し器で漉す
が、その際に香味野菜とガラのエキス\footnote{原文quintessence(カンテソンス)。本来の意味は錬金術でいう「第五元
  素」。16世紀の作家フランソワ・ラブレーは存命当時、自著を筆名「カン
  テサンス抽出をなし遂げたアルコフリバス師」で出版していた時期がある。
  もっとも、このカンテサンスという語自体は中世以来、料理において「エ
  キス」「美味しさの本質」程度の意味でよく用いられた。}が得られるよう、強く押し絞って
やること。こうして出来たクリを、このソースを合わせる鳥と同種のものでとっ
たフォン4 dlで薄める。

ジビエが肉断ちの食材と見做されるもので、なおかつそれを厳格に守って作ら
なければならない場合は、このときフォンの代わりにマッシュルームの茹で汁を
用いればいい。

約45分〜1時間、弱火にかけて浮いてくる不純物を丁寧に取り除いてやる\footnote{dépouiller
  デプイエ。現代ではécumerエキュメの語を用いる現場が多 い。}。
さらにソースを\deuxtiers{}以下の量になるまで煮詰める。これにマッシュルー
ムの茹で汁とトリュフエッセンスを適量加えて丁度いい濃度になるよう調製する。

布で漉し、軽くバターを加えて仕上げる\footnote{原文は légèrement
  beurrerでありそのまま訳したが、現代の調理現場 ではmonter au beurre
  バターでモンテする、という表現がよく使われる。}。

\hypertarget{ux539fux6ce8-2}{%
\subparagraph{【原注】}\label{ux539fux6ce8-2}}

仕上げの際に、ソース1 Lあたりバター約50 gを加えるが、これは任意。

\hypertarget{sauce-tortue}{%
\subsubsection{ソース・トルチュ}\label{sauce-tortue}}

\frsub{Sauce Tortue}\footnote{tortue
  (トルチュ)は海亀のこと。古くは海亀料理用のソースだった
  が、19世紀以降は仔牛の頭肉料理に合わせるのが一般的になった。なお、
  tortu(e)という形容詞があり「曲がりくねった、(性格が)ひねくれた」
  という同音異義語があるが、このソースの由来とは無関係。}

\index{そーす@ソース!ふらうんはせい@ブラウン系の派生---!とるちゆ@---・トルチュ}
\index{そーす@ソース!とるちゅ@---・トルチュ}
\index{とるちゅ@トルチュ!そーす@ソース・---}
\index{うみかめ@海亀 ⇒ トルチュ!そーすとるちゆ@ソース・トルチュ}
\index{sauce@sauce!petites brunes composees@Petites ---s Brunes Composées!tortue@--- Tortue}
\index{sauce@sauce!tortue@--- Tortue}
\index{tortue@tortue!sauce@Sauce ---}

2\undemi{}
Lの\protect\hyperlink{jus-de-veau-brun}{仔牛のフォン}を鍋で沸かし、セージ3
g、マジョラム1 g、ローズマリー1 g、バジル2 g、タイム1 g、ローリエの葉1
g、パセリの葉1つまみ、マッシュルームの切りくず25 gを投入する。蓋をして
25分間煎じる。こうして煎じた液体を漉す2分前に大粒のこしょう4個を加える。

布で漉し、\protect\hyperlink{sauce-demi-glace}{ソース・ドゥミグラス}7
dlに\protect\hyperlink{sauce-tomate}{トマトソー ス}3
dlを合わせたものに、上記で煎じた液体を、風味が際立
つ程度に適量加える。\troisquarts{}量まで煮詰め、布で漉す。仕上げにマデ
ラ酒1 dlとトリュフエッセンス少々を加え、さらにカイエンヌで風味を引き締
める。

\hypertarget{ux539fux6ce8-nota-sauce-tortue}{%
\subparagraph{【原注】
(\#nota-sauce-tortue)}\label{ux539fux6ce8-nota-sauce-tortue}}

このソースはある程度まとまった量で作る必要がある。カイエンヌを使う指示
があるからだ。それでも、カイエンヌはとても気をつけて量を加減する必要が
ある\footnote{フランス料理において(というよりも伝統的かつ一般的なフランス人
  にとって)、唐辛子の辛さは嫌われる傾向が非常に強い。}。

\hypertarget{sauce-venaison}{%
\subsubsection{ソース・ヴネゾン}\label{sauce-venaison}}

\frsub{Sauce Venaison}\footnote{Venaison(ヴネゾン)とはノロ鹿chevreuilや猪sanglierなどの大型ジ
  ビエのこと。なおニホンジカやエゾジカはcerf(セール)に分類され、フ
  ランス料理の食材としてはあまり高く評価されない傾向がある。}

\index{そーす@ソース!ふらうんはせい@ブラウン系の派生---!うねそん@---・ヴネゾン}
\index{そーす@ソース!うねそん@---・ヴネゾン}
\index{うねそん@ヴネゾン!そーす@ソース・---}
\index{おおかたしひえ@大型ジビエ ⇒ ヴネゾン!そーす@ソース・ヴネゾン}
\index{sauce@sauce!petites brunes composees@Petites ---s Brunes Composées!venaison@--- Venaison}
\index{sauce@sauce!venaison@--- Venaison}
\index{venaison@venaison!sauce@Sauce ---}

完全に仕上げた「\protect\hyperlink{sauce-poivrade-pour-gibier}{ジビエ用ソース・ポワヴラー
ド}」\troisquarts{}
Lに、\protect\hyperlink{gelee-de-groseilles-a}{グロゼイユのジュ
レ}大さじ3杯強を生クリーム1dlで溶いてから加える。

グロゼイユのジュレと生クリームを加えるのは、鍋を火から外して、提供直前
にすること。

\ldots{}\ldots{}大型ジビエ料理用。

\hypertarget{sauce-vin-rouge}{%
\subsubsection{赤ワインソース}\label{sauce-vin-rouge}}

\frsub{Sauce au Vin rouge}

\index{そーす@ソース!ふらうんはせい@ブラウン系の派生---!あかわいん@赤ワイン---}
\index{そーす@ソース!あかわいん@赤ワイン---}
\index{あかわいん@赤ワイン!そーす@---ソース}
\index{sauce@sauce!petites brunes composees@Petites ---s Brunes Composées!vin rouge@--- au Vin rouge}
\index{sauce@sauce!vin rouge@--- au Vin rouge}
\index{vin@vin!sauce rouge@Sauce au --- rouge}

「赤ワインソース」という場合、煮詰めてからブールマニエでとろみを付ける
ブルゴーニュ風の仕立てか、魚を煮るのに用いた赤ワインを使うことが特徴で
ある「ソース・マトロット」のいずれかから派生したものなのは言うまでもな
い。もっとも、後者の場合はワインの風味は失われてしまっていてソースの水
気と味付けの意味しか持っていないと言える。

両者どちらもまさしく「赤ワインソース」だが、\protect\hyperlink{sauce-bourguignonne}{ブルゴーニュ風ソー
ス}と\protect\hyperlink{sauce-matelote}{ソース・マトロット}はそれ
ぞれ作り方も用途も違うから別々の名称として、この「茶色い派生ソース」の
節で説明した。

筆者としては、本当の「赤ワインソース」は以下のように作るものと考えてい
る。

ごく細かく刻んだ標準的な\protect\hyperlink{mirepoix}{ミルポワ}125
gをバターで炒める。良 質の赤ワイン\undemi{}
Lを注ぐ。半量になるまで煮詰める。つぶしたにんに
く1片、\protect\hyperlink{sauce-espagnole}{ソース・エスパニョル}7\undemi{}
dlを加え、12〜
15分、火にかけて浮いてくる不純物を丁寧に取り除く\footnote{dépouiller
  デプイエ ≒ écumer エキュメ。}。

布で漉し、バター100 gとアンチョビエッセンス小さじ1杯、カイエンヌ1つま
みを加えて仕上げる。

\ldots{}\ldots{}魚料理用ソース。

\hypertarget{sauce-zingara-a}{%
\subsubsection{ソース・ザンガラ A}\label{sauce-zingara-a}}

\frsub{Sauce Zingara A}\footnote{もとの語形はzingaro
  ザンガロ、またはヂンガロ。ジプシー、ボヘミ
  アンの意。料理ではパプリカ粉末やカイエンヌを用いたものに命名される
  ことが多い。}

\index{そーす@ソース!ふらうんはせい@ブラウン系の派生---!さんからa@---・ザンガラA}
\index{そーす@ソース!さんからa@---・ザンガラ A}
\index{さんから@ザンガラ!そーすa@ソース・--- A}
\index{しふしーふう@ジプシー風!そーすa@ソース・ザンガラ A}
\index{sauce@sauce!petites brunes composees@Petites ---s Brunes Composées!zingara a@--- Zingara A}
\index{sauce@sauce!zingara a@--- Zingara A}
\index{zingara@Zingara!sauce a@Sauce --- A}

このソースは古典料理の\protect\hyperlink{garniture-zingara}{ガルニチュール・ザンガラ}とはまったく関係が
ない。むしろイギリス料理に由来し、本書でもイギリス風ソースの節において
似たようなものをいくつか採り上げている。

ヴィネガー2\undemi{} dlにエシャロットのみじん切り大さじ1杯を加えて半量
になるまで煮詰める。\protect\hyperlink{jus-de-veau-lie}{茶色いジュ}7
dlを注ぎ、バターで
揚げたパンの身160gを加える。弱火で5〜6分間煮る。パセリのみじん切り大さ
じ1杯とレモン\undemi{}個分の搾り汁を加えて仕上げる。

\hypertarget{sauce-zingara-b}{%
\subsubsection{ソース・ザンガラ B}\label{sauce-zingara-b}}

\frsub{Sauce Zingara B}

\index{そーす@ソース!ふらうんはせい@ブラウン系の派生---!さんからb@---・ザンガラB}
\index{そーす@ソース!さんからb@---・ザンガラ B}
\index{さんから@ザンガラ!そーすb@ソース・--- B}
\index{しふしーふう@ジプシー風!そーすb@ソース・ザンガラ B}
\index{sauce@sauce!petites brunes composees@Petites ---s Brunes Composées!zingara b@--- Zingara B}
\index{sauce@sauce!zingara b@--- Zingara B}
\index{zingara@Zingara!sauce b@Sauce --- B}

白ワイン3 dlとマッシュルームの茹で汁3 dlを合わせて\untiers{}量になるまで
煮詰める。

\protect\hyperlink{sauce-demi-glace}{ソース・ドゥミグラス}4
dlと\protect\hyperlink{sauce-tomate}{トマトソー ス}2\undemi{}
dl、\protect\hyperlink{fonds-blanc}{白いフォン}1 dlを注ぐ。
浮いてくる不純物を徹底的に取り除きながら5〜6分火にかける。

仕上げに、カイエンヌ1つまみで風味を引き締め、太さ1〜2 mmの千切りにした
\footnote{julienne
  (ジュリエーヌ)。日本語では「ジュリエンヌ」と言うこと
  が多いが、「ジュリヤン」のように言う調理現場もある。}ハム(脂身のないところ)と赤く漬けた舌肉70
gおよびマッシュルーム 50 g、トリュフ 30 gを加える。

\ldots{}\ldots{}仔牛料理、鶏料理用。
\end{recette}
\hypertarget{ux30dbux30efux30a4ux30c8ux7cfbux306eux6d3eux751fux30bdux30fcux30b9}{%
\section{ホワイト系の派生ソース}\label{ux30dbux30efux30a4ux30c8ux7cfbux306eux6d3eux751fux30bdux30fcux30b9}}

\hypertarget{petites-sauces-blanches-composuxe9es-et-de-ruxe9ductions}{%
\subsection{Petites Sauces Blanches Composées et de
Réductions}\label{petites-sauces-blanches-composuxe9es-et-de-ruxe9ductions}}

\maeaki
\begin{recette}
\hypertarget{ux30bdux30fcux30b9ux30a2ux30ebux30d3ux30e5ux30d5ux30a7ux30e91}{%
\subsubsection[ソース・アルビュフェラ]{\texorpdfstring{ソース・アルビュフェラ\footnote{ナポレオン軍の元帥、ルイ・ガブリエル・スーシェ
  Louis-Gabriel Suchet, duc d'Albufera
  (1770〜1826)のこと。スペイン戦役の際にそれ
  までの軍功を称えられ、ナポレオンが1812年にアルビュフェラ公爵位を新
  設して授けた。帝政期の英雄のひとりであり、アルビュフェラおよびスー
  シェの名を冠した料理がいくつかある。1814年に帝政が崩壊した後も軍務、
  政務に携わり、最終的にフランス貴族院議員の地位を得た。アルビュフェ
  ラ公爵位については、1815年7月24日の勅令においてに正式に抹消されて
  いる。このソースの特徴は赤ピーマン(パプリカ)を加熱してなめらかに
  すり潰し、バターに練り込んだものを使う点にあるが、どのような経緯で
  このソースに赤ピーマンを用いるようになったのかは不明。ただし、この
  ソースを合わせる「肥鶏 アルビュフェラ」は詰め物(ファルス)に米を
  用いるが、アルビュフェラは湖の周辺の湿地帯で米の生産がおこなわれて
  いるという点では一応の関連性が認められよう。なお、アルビュフェラは
  バレンシアの湖とそこに形成された潟であり、現在はバレンシア州のアル
  ブフェーラ自然公園となっている。}}{ソース・アルビュフェラ}}\label{ux30bdux30fcux30b9ux30a2ux30ebux30d3ux30e5ux30d5ux30a7ux30e91}}

\hypertarget{sauce-albufera}{%
\paragraph{Sauce Albuféra}\label{sauce-albufera}}

\index{そーす@ソース!あるひゆふえら@---・アルビュフェラ}
\index{あるひゆふえら@アルビュフェラ!そーす@ソース・---}
\index{sauce@sauce!albufera@--- Albuféra}
\index{albufera@Albuféra!sauce@Sauce ---}

\protect\hyperlink{sauce-supreme}{ソース・シュプレーム}1
Lあたりに、溶かしたブロンド色
の\protect\hyperlink{glace-de-viande}{グラスドヴィアンド}2
dlと、標準的な分量比率で作っ た\href{}{赤ピーマンバター}50 gを加える。

\maeaki

\hypertarget{ux30bdux30fcux30b9ux30a2ux30e1ux30eaux30b1ux30fcux30cc3}{%
\subsubsection[ソース・アメリケーヌ]{\texorpdfstring{ソース・アメリケーヌ\footnote{アメリケーヌという名称の由来は諸説あるが、19世紀フランスの料理人
  ピエール・フレス Pierre Fraysse がアメリカで働いた後にパリで1853年
  に開いたレストラン「シェ・ピーターズ」でこの料理名で提供したという
  のが定説。ただし、1853年以前にレストラン「ボヌフォワ」に「ラングドッ
  ク産オマール ソース・アメリケーヌ添え」というメニューあり、フレス
  はその料理に改変を加えたか、名前だけをシンプルに「アメリケーヌ」と
  した程度という説もある。かつては、オマールの主産地のひとつブルター
  ニュ地方を意味する古い形容詞 armoricain(e) アルモリカン、アルモリ
  ケーヌの音が変化した料理名だと主張されることもあったが、19世紀には
  南仏産が中心であったトマトを用いる点で矛盾が生じてしまう。いずれに
  しても、この料理名がフレスの店シェ・ピーターズを基点として広く知ら
  れるようになったことは事実と考えていい。}}{ソース・アメリケーヌ}}\label{ux30bdux30fcux30b9ux30a2ux30e1ux30eaux30b1ux30fcux30cc3}}

\hypertarget{sauce-americaine}{%
\paragraph{Sauce Américaine}\label{sauce-americaine}}

\index{そーす@ソース!あめりけーぬ@---・アメリケーヌ}
\index{あめりふう@アメリカ風!そーす@ソース・アメリケーヌ}
\index{sauce@sauce!americaine@--- Américaine}
\index{americain@américain!sauce americaine@Sauce Américaine}

このソースは\protect\hyperlink{homard-a-l-americaine}{オマール・アメリケーヌ}という料理
そのものと言っていい(「魚料理」の章、甲殻類、\protect\hyperlink{homard-a-l-americaine}{オマール・アメリケー
ヌ}参照)。

このソースは通常、オマールの身をガルニチュールとした魚料理に添えられる。
オマールの身をやや斜めになるよう厚さ1 cm程度の輪切りにし\footnote{escalopper
  エスカロペ。エスカロップに切る。ここで用いられるオマー
  ルは900g〜1kg程度の大きさのものを想定していることに注意。}、魚料理の
ガルニチュールとして供するわけだ。

\maeaki

\hypertarget{ux30a2ux30f3ux30c1ux30e7ux30d3ux30bdux30fcux30b9}{%
\subsubsection{アンチョビソース}\label{ux30a2ux30f3ux30c1ux30e7ux30d3ux30bdux30fcux30b9}}

\hypertarget{sauce-anchois}{%
\paragraph{Sauce Anchois}\label{sauce-anchois}}

\index{そーす@ソース!あんちょうい@アンチョビ---}
\index{あんちょひ@アンチョビ!そーす@---ソース}
\index{sauce@sauce!anchois@--- Anchois}
\index{anchois@anchois!sauce anchois@Sauce ---}

\href{}{ノルマンディー風ソース}8
dlを、バターを加える前のところまで作る。こ
れに\href{}{アンチョビバター}125 gを混ぜ込む。アンチョビのフィレ50
gを洗い、 よく水気を絞ってから小さなさいの目に切ったのを加えて仕上げる。

\ldots{}\ldots{}魚料理用。

\maeaki

\hypertarget{ux30bdux30fcux30b9ux30aaux30fcux30edux30fcux30eb4}{%
\subsubsection[ソース・オーロール]{\texorpdfstring{ソース・オーロール\footnote{夜明けの光、曙光のこと。オーロラの意味もあるため、日本では「オーロラソース」と呼ばれることもあるが、マヨネーズとトマトケチャップを同量で混ぜ合わせたものもそう呼ばれることが多いので注意。}}{ソース・オーロール}}\label{ux30bdux30fcux30b9ux30aaux30fcux30edux30fcux30eb4}}

\hypertarget{sauce-aurore}{%
\paragraph{Sauce Aurore}\label{sauce-aurore}}

\index{そーす@ソース!おーろーる@---・オーロール}
\index{おーろーる@オーロール!そーす@ソース・---}
\index{sauce@sauce!aurore@--- Aurore}
\index{aurore@aurore!sauce@Sauce ---}

\protect\hyperlink{veloute}{ヴルテ}に真っ赤なトマトピュレを加えたもの。分量は、ヴルテが\troisquarts{}に対し、トマトピュレ\unquart{}とする。仕上げに、ソース1
Lあたり100 gのバターを加える。

\ldots{}\ldots{}卵料理、仔牛、仔羊肉の料理、鶏料理用。

\maeaki

\hypertarget{ux9b5aux6599ux7406ux7528ux30bdux30fcux30b9ux30aaux30fcux30edux30fcux30eb}{%
\subsubsection{魚料理用ソース・オーロール}\label{ux9b5aux6599ux7406ux7528ux30bdux30fcux30b9ux30aaux30fcux30edux30fcux30eb}}

\hypertarget{sauce-aurore-maigre}{%
\paragraph{Sauce Aurore maigre}\label{sauce-aurore-maigre}}

\index{そーす@ソース!おーろーるさかなよう@魚料理用---・オーロール}
\index{おーろーる@オーロール!そーすさかな@魚料理用ソース・---}
\index{sauce@sauce!aurore maigre@--- Aurore maigre}
\index{aurore@aurore!sauce maigre@Sauce --- maigre}

\protect\hyperlink{veloute-de-poisson}{魚料理用ヴルテ}に、上記と同じ割合でトマトピュレ
を加える。ソース1 Lあたりバター125 gを加えて仕上げる。

\ldots{}\ldots{}魚料理用

\maeaki

\hypertarget{ux30d0ux30a4ux30a8ux30ebux30f3ux98a8ux30bdux30fcux30b9}{%
\subsubsection{バイエルン風ソース}\label{ux30d0ux30a4ux30a8ux30ebux30f3ux98a8ux30bdux30fcux30b9}}

\hypertarget{sauce-bavaroise}{%
\paragraph{Sauce Bavaroise}\label{sauce-bavaroise}}

\index{そーす@ソース!はいえるんふう@バイエルン風---}
\index{はいえるんふう@バイエルン風!そーす@---ソース}
\index{sauce@sauce!bavarois@--- Bavaroise}
\index{bavarois@bavarois!sauce bavaroise@Sauce Bavaroise}

ヴィネガー5
dlにタイムとローリエの葉少々とパセリの枝4本、大粒のこしょう7〜8個と、おろした\footnote{原文
  râpé \textless{} râpe
  ラープと呼ばれる器具を用いておろすが、日本のおろし金と目の大きさが違うので注意。多くの場合、マンドリーヌ
  mandrine と呼ばれる野菜用スライサーにこの機能が付属している。}レフォール\footnote{raifort
  西洋わさび、ホースラディッシュ。}大さじ2杯を加え、半量になるまで煮詰める。

この煮詰めた汁に卵黄6個を加え\footnote{卵黄を加える前に一度漉しておいたほうがいいだろう。}、\protect\hyperlink{sauce-hollandaise}{オランデーズソース}を作る要領で、バター400
gと大さじ1\undemi{}杯の水を少しずつ加えながら、ソースがしっかり乳化するまで混ぜていく。布で漉す。

\protect\hyperlink{beurre-d-ecrevisse}{エクルヴィスバター}100
gと泡立てた生クリーム大さじ2杯、さいの目に切ったエクルヴィスの尾の身を加えて仕上げる。

\ldots{}\ldots{}魚料理用のこのソースは、ムースのような仕上りにすること。

\end{recette}
\hypertarget{ux30a4ux30aeux30eaux30b9ux6599ux7406ux306eux30bdux30fcux30b9ux6e29ux88fd}{%
\section{イギリス料理のソース(温製)}\label{ux30a4ux30aeux30eaux30b9ux6599ux7406ux306eux30bdux30fcux30b9ux6e29ux88fd}}

\hypertarget{sauces-anglaises-chaudes}{%
\subsection{Sauces Anglaises Chaudes}\label{sauces-anglaises-chaudes}}
\begin{recette}
\hypertarget{ux30afux30e9ux30f3ux30d9ux30eaux30fcux30bdux30fcux30b9}{%
\subsubsection{クランベリーソース}\label{ux30afux30e9ux30f3ux30d9ux30eaux30fcux30bdux30fcux30b9}}

\hypertarget{cranberries-sauce}{%
\paragraph{\texorpdfstring{Sauce aux Airelles
(\emph{Cranberries-Sauce})}{Sauce aux Airelles (Cranberries-Sauce)}}\label{cranberries-sauce}}

\index{そーす@ソース!くらんへりー@クランベリー---}
\index{くらんへりー@クランベリー!そーす@---ソース}
\index{sauce@sauce!airelles@--- aux Airelles}
\index{airelle@airelle!sauce airelles@Sauce aux Airelles}
\index{sauce@sauce!cranberries@Cranberries-Sauce}
\end{recette}
\hypertarget{ux51b7ux88fdux30bdux30fcux30b9}{%
\section{冷製ソース}\label{ux51b7ux88fdux30bdux30fcux30b9}}

\hypertarget{sauces-froides}{%
\subsection{Sauces Froides}\label{sauces-froides}}

\index{sauce@sauce!sauces froides@sauces froides}
\index{そーす@ソース!れいせいそーす@冷製ソース}
\begin{recette}
\hypertarget{ux30a2ux30a4ux30e8ux30ea2-ux30d7ux30edux30f4ux30a1ux30f3ux30b9ux30d0ux30bfux30fc}{%
\subsubsection[アイヨリ /
プロヴァンスバター]{\texorpdfstring{アイヨリ\footnote{ailloliとも綴るが、
  ail(にんにく)+
  oil(油)の合成語。19世前半紀には既にアカデミーフランセージの辞書に収録されており、広く知られていたようだ。ブイヤベースに添えるルイユとよく似ているが、ルイユがカイエンヌを加えるのに対して、こちらはにんにくと油、塩、レモン汁と少々の水だけで作る。用途も、茹でた塩鱈やじゃがいも、茹で卵、アーティチョーク、さやいんげん、などに合わせることが多い。}
/
プロヴァンスバター}{アイヨリ / プロヴァンスバター}}\label{ux30a2ux30a4ux30e8ux30ea2-ux30d7ux30edux30f4ux30a1ux30f3ux30b9ux30d0ux30bfux30fc}}

\hypertarget{sauce-aioli}{%
\paragraph{Sauce Aïoli, ou Beurre de Provence}\label{sauce-aioli}}

\index{そーす@ソース!れいせい@冷製---!あいより@アイヨリ}
\index{そーす@ソース!れいせい@冷製---!ふろふあんすはたー@プロヴァンスバター}
\index{あいより@アイヨリ}
\index{ふろふあんす@プロヴァンス!ふろふあんすはたー@プロヴァンスバター}
\index{はたー@バター!ふろふあんすはたー@プロヴァンスバター}
\index{sauce@sauce!sauce froide@sauce froide!aioli@--- Aïoli}
\index{sauce@sauce!sauce froide@sauce froide!beurre de provence@Beurre de Provence}
\index{aioli@Aïoli!sauce@Sauce ---}
\index{provence@Provence!Beurre de Provence (Aïoli)}
\index{beurre@beurre!beurre de provence@Beurre de Provence (Aïoli)}

にんにく4片(30 g)を鉢\footnote{この種の作業には、大理石製のものが伝統的によく用いられる。。}に入れて細かくすり潰す。ここに生の卵黄1個、塩1つまみを加える。混ぜながら、2\undemi{}
dlの油\footnote{原書ではとくに言及されていないが、プロヴァンス地方ではオリーブオイルを用いることが一般的。}を初めは1滴ずつ加えていき、ソースがまとまりはじめたら糸を垂らすようにして加える。この作業は鉢に入れたままで、棒をはげしく動かして行なう。

攪拌する作業の途中、レモン1個分の搾り汁と冷水大さじ\undemi{}杯を少しずつ加えて、ソースが固くなり過ぎないようにしてやること。

\hypertarget{ux539fux6ce8}{%
\subparagraph{【原注】}\label{ux539fux6ce8}}

このアイヨリソースが分離してしまいそうな時は、卵黄をさらに1個足して、
マヨネーズと場合と同様に修正すること。

\maeaki

\hypertarget{ux30a2ux30f3ux30c0ux30ebux30b7ux30a25ux98a8ux30bdux30fcux30b9}{%
\subsubsection[アンダルシア風ソース]{\texorpdfstring{アンダルシア\footnote{いうまでもなくスペインのアンダルシア地方のことだが、トマトやオリーブオイル、チョリソなどこの地方を「想起」させる食材が使われている料理などがこの名称になっている傾向がある。ところが、トマトにしろオリーブオイルにしろアンダルシア地方特有というわけではなく、アンダルシアが産地として有名なチョリソくらいしか、料理名の根拠となり得るものはない。逆に言えば、アンダルシア地方の食文化との関係は、そこに用いられている食材以外にはないものと考えてもいい。料理名に付けられた地方名がとりたてて根拠や由来のないものであることを示す一例。}風ソース}{アンダルシア風ソース}}\label{ux30a2ux30f3ux30c0ux30ebux30b7ux30a25ux98a8ux30bdux30fcux30b9}}

\hypertarget{sauce-andalouse}{%
\paragraph{Sauce Andalouse}\label{sauce-andalouse}}

\index{そーす@ソース!れいせい@冷製---!あんたるしあふう@アンダルシア風---}
\index{あんたるしあ@アンダルシア!そーす@---風ソース}
\index{そーす@ソース!あんたるしあふう@アンダルシア風---}
\index{sauce@sauce!sauce froide@sauce froide!Andalouse@--- Andalouse}
\index{sauce@sauce!andalouse@--- Andalouse}
\index{andalous@Andalous(e)!sauce@Sauce Andalouse}

ごく固く仕上げた\protect\hyperlink{mayonnaise}{ソース・マヨネーズ}\troisquarts{}
Lに、上等な赤いトマトピュレ2\undemi{}dlを加える。小さなさいの目に切ったポワヴロン\footnote{Poivron
  いわゆる日本で青果として輸入されているパプリカ(肉厚の辛くないピーマン)とほぼ同じものだが、香辛料として用いられる粉末のパプリカと混同を避けるため、あえてフランス語をそのままカタカナに訳した。}75
gを仕上げに加える。

\maeaki

\hypertarget{ux30bdux30fcux30b9ux30dcux30d8ux30dfux30a2ux306eux5a18}{%
\subsubsection{ソース・ボヘミアの娘}\label{ux30bdux30fcux30b9ux30dcux30d8ux30dfux30a2ux306eux5a18}}

\hypertarget{sauce-bohemienne}{%
\paragraph[Sauce Bohémienne]{\texorpdfstring{Sauce Bohémienne\footnote{アイルランド出身の作曲家マイケル・ウィリアム・バルフェMichael
  William Balfe (1808〜1870)のオペラ\emph{The Bohemien
  Girl}『ボヘミアの少女』のフランス語版タイトル\href{https://archive.org/details/labohmiennegrand00balf}{\emph{La
  Bohémienne}}『ラボエミエーヌ』にちなんだものと言われている。この作品はロンドンで1843年初演、1862年に四幕形式のフランス語版がパリのオペラ=コミック劇場で上演され、大ヒットしたという。この名を冠した料理はいくつかあるが、いずれもチェコのボヘミア地方とは何の関連性も認められないため、オペラの人気作品にあやかった料理名と考えるのが妥当だろう。}}{Sauce Bohémienne}}\label{sauce-bohemienne}}

\index{そーす@ソース!れいせい@冷製---!ほへみあのむすめ@---ボヘミアの娘}
\index{ほへみあ@ボヘミア!そーす@ソース・---の娘}
\index{そーす@ソース!ほへみあ@---・ボヘミアの娘}
\index{sauce@sauce!sauce froide@sauce froide!bohemienne@--- Bohémienne}
\index{sauce@sauce!bohemienne@--- Bohémienne}
\index{bohemien@bohémien(ne)!sauce@Sauce Bohémienne}

陶製の容器に、濃厚でよく冷やした\protect\hyperlink{sauce-bechamel}{ベシャメルソース}1\undemi{}
dlと卵黄4個、塩10 g、こしょう少々、ヴィネガー数滴を入れる。

泡立て器で全体をよく混ぜ、標準的なマヨネーズを作るのとまったく同じ要領で、油1
Lとエストラゴンヴィネガー大さじ2杯程を加える。

\ldots{}\ldots{}仕上げに、マスタード大さじ1杯を加える。

\maeaki

\hypertarget{ux30bdux30fcux30b9ux30b7ux30e3ux30f3ux30c6ux30a3ux30a47}{%
\subsubsection[ソース・シャンティイ]{\texorpdfstring{ソース・シャンティイ\footnote{パリ近郊の地名。詳しくはホワイト系派生ソースの\protect\hyperlink{sauce-chantilly}{ソース・シャンティイ}訳注参照。}}{ソース・シャンティイ}}\label{ux30bdux30fcux30b9ux30b7ux30e3ux30f3ux30c6ux30a3ux30a47}}

\hypertarget{sauce-chantilly-froide}{%
\paragraph{Sauce Chantilly}\label{sauce-chantilly-froide}}

\index{そーす@ソース!れいせい@冷製---!しやんていい@---・シャンティイ}
\index{しやんていい@シャンティイ!そーす@ソース・---(冷製)}
\index{そーす@ソース!しやんていい@---・シャンティイ}
\index{sauce@sauce!sauce froide@sauce froide!chantilly@--- Chantilly}
\index{sauce@sauce!chantilly@--- Chantilly (froide)}
\index{chantilly@Chantilly!sauce@Sauce --- (froide)}

酸味付けにレモンを用いて、固く仕上げた\protect\hyperlink{mayonnaise}{ソース・マヨネーズ}\troisquarts{}
Lを用意しておく。提供直前に、ごく固く泡立てた生クリーム大さじ4杯\footnote{大さじ1杯=15ccという概念にとらわれないよう注意。原文は、大きなスプーンで泡立てた生クリームをざっくりと4回加えるイメージで書かれている。本書における通常のソースの仕上り量が約1
  Lであることを考慮すると、最低でも100ml以上は加えることになるだろう。}を加える。その後、味を\ruby{調}{ととの}える。

\ldots{}\ldots{}もっぱら、アスパラガスの冷製、温製に添える。

\hypertarget{ux539fux6ce8-1}{%
\subparagraph{【原注】}\label{ux539fux6ce8-1}}

生クリームを加えるのは、このソースを使うまさにその時にすること。前もっ
て加えておくと、ソースが分離してしまう恐れがあるので注意。

\maeaki

\hypertarget{ux30b8ux30a7ux30ceux30f4ux30a1ux98a812ux30bdux30fcux30b9}{%
\subsubsection[ジェノヴァ風ソース]{\texorpdfstring{ジェノヴァ風\footnote{あまり明確な由来はないが、ジェノヴァが地中海に面した港町であり、このソースが魚料理用であるという点で一応の説明はつくだろう。}ソース}{ジェノヴァ風ソース}}\label{ux30b8ux30a7ux30ceux30f4ux30a1ux98a812ux30bdux30fcux30b9}}

\hypertarget{sauce-genoise-froids}{%
\paragraph{Sauce Génoise}\label{sauce-genoise-froids}}

\index{そーす@ソース!れいせい@冷製---!しえのうあふう@ジェノヴァ風---}
\index{しえのうあふう@ジェノヴァ風!そーす@ソース・---(冷製)}
\index{そーす@ソース!しえのうあふう@ジェノヴァ風---}
\index{sauce@sauce!sauce froide@sauce froide!genoise@--- Génoise}
\index{sauce@sauce!genoise@--- Génoise (froide)}
\index{genois@Génois(e)!sauce@Sauce ---e (froide)}

殻と皮を剥いたばかりのピスタチオ40 gと、松の実25
g、松の実がない場合はスイートアーモンド20
gを鉢に入れてよくすり潰し、冷めた\protect\hyperlink{sauce-bechamel}{ベシャメルソース}小さじ1杯程度を加えて練ってペースト状にする。これを目の細かい網で裏漉しする。陶製の容器に卵黄6個、塩1つまみ、こしょう少々を入れる。泡立て器でよく混ぜる。油1
Lと中位の大きさのレモン2個の搾り汁を少しずつ加えてよく混ぜて乳化させていく\footnote{明記されていないが、ソースをしっかりと乳化させるためには\protect\hyperlink{mayonnaise}{マヨネーズ}と同様に作業すること。}。仕上げにハーブのピュレ大さじ3杯を加える。これは、パセリの葉とセルフイユ\footnote{cerfeuil
  チャービル。}、エストラゴン\footnote{estragon フレンチタラゴン。}、時季が合えばサラダバーネットを同量ずつ用意し、強火で2分間下茹でしてから湯をきり、冷水にさらしてから水気を強く絞り、裏漉しして作っておく。

\ldots{}\ldots{}冷製の魚料理全般に合わせられる。

\maeaki

\hypertarget{ux30bdux30fcux30b9ux30b0ux30eaux30d3ux30c3ux30b7ux30e5}{%
\subsubsection{ソース・グリビッシュ}\label{ux30bdux30fcux30b9ux30b0ux30eaux30d3ux30c3ux30b7ux30e5}}

\hypertarget{sauce-gribiche13}{%
\paragraph[Sauce Gribiche]{\texorpdfstring{Sauce Gribiche\footnote{由来不明の語。ノルマンディ方言で「子どもを怖がらせるおっかない
  おばさん」という意味で用いられるということがわかっているのみ。19世
  紀後半もしくは20世紀初頭に創案されたソースと思われる。本書初版には
  当然のように既に収録されており、その後の大きな異同もない。ただ、本
  書初版以前に出版された料理書においてこのソースのレシピはまだ見つかっ
  ていない。ファーヴルは1905年刊の『料理および食品衛生事典』第二版で
  「ある種のレムラードにレストランで付けられた名称」と定義し、掲載し
  ているレシピは本書初版のものと大差ないが、「ウスターシャソース少々
  も加える」となっているところが目を引く。また、1913年初版のプルース
  トの長編小説『失なわれた時を求めて』の「スワン家の方へ」冒頭におい
  て「彼(=スワン)を招いていない夕食会のために、ソース・グリビッシュ
  やパイナップルのサラダのレシピが必要になるや、ためらいもなく探しに
  行かせたりするのだった」(p.18)。もしこの語り手の記述が正確であるな
  ら、19世紀末には広く知られたものであったと考えるべきだが、小説の場
  合は必ずしも歴史的事実と符号するわけではないので注意が必要。}}{Sauce Gribiche}}\label{sauce-gribiche13}}

\index{そーす@ソース!れいせい@冷製---!くりひつしゆ@---・グリビッシュ}
\index{くりひつしゆ@グリビッシュ!そーす@ソース・---(冷製)}
\index{そーす@ソース!くりひつしゆ@グリビッシュ---}
\index{sauce@sauce!sauce froide@sauce froide!gribiche@--- Gribiche}
\index{sauce@sauce!gribiche@--- Gribiche (froide)}
\index{gribiche@!gribiche!sauce@Sauce --- (froide)}

茹であがったばかりの固茹で卵の黄身6個を陶製のボウルに入れ、マスタード
小さじ1杯、塩1つまみ強、こしょう適量を加えてよく練り、滑らかなペースト
状にする。植物油\undemi{} Lとヴィネガー大さじ1\undemi{}杯を加えながら
よく混ぜて乳化させる。仕上げに、コルニション\footnote{長さ3〜4cm程度の小さいうちに収穫してヴィネガー漬けにしたきゅうり。専用品種が用いられる。}とケイパーのみじん切
り計100 gと、パセリとセルフイユ、エストラゴンのみじん切りのミックスを
大さじ1杯、短かめの千切り\footnote{julienne
  ジュリエンヌ。1〜2mm角の千切り。}にした固茹で卵の白身3個分を加える。

\ldots{}\ldots{}冷製の魚料理に添えるのが一般的。
\end{recette}
\hypertarget{ux5408ux308fux305bux30d0ux30bfux30fc}{%
\section{合わせバター}\label{ux5408ux308fux305bux30d0ux30bfux30fc}}

\vspace{0\zw}

\hypertarget{ux30b0ux30eaux30ebux30bdux30fcux30b9ux306eux88dcux52a9ux6750ux6599ux30aaux30fcux30c9ux30d6ux30ebux7528}{%
\subsection{グリル、ソースの補助材料、オードブル用}\label{ux30b0ux30eaux30ebux30bdux30fcux30b9ux306eux88dcux52a9ux6750ux6599ux30aaux30fcux30c9ux30d6ux30ebux7528}}

\vspace*{-1.5\zw}

\hypertarget{beurres-composuxe9s-pour-adjuvants-de-sauces-et-hors-doeuvre}{%
\subsection{Beurres Composés pour Adjuvants de Sauces et
Hors-d'oeuvre}\label{beurres-composuxe9s-pour-adjuvants-de-sauces-et-hors-doeuvre}}

\index{あわせはたー@合わせバター} \index{はたー@バター ⇒ 合わせバター}
\index{ふーるこんほせ@ブール・コンポゼ ⇒ 合わせバター}
\index{みつくすはたー@ミックスバター ⇒ 合わせバター}
\index{beurre@beurre!beurres composes@Beurres Composés}

\hypertarget{observation-sur-les-beurres-composes}{%
\subsection{概説}\label{observation-sur-les-beurres-composes}}

本書においてレシピを掲載している合わせバター\footnote{beurre composé
  ブール・コンポゼ。ミックスバターとも。少なくとも
  バターは中世以来用長く用いられてきた食材だが、中世〜ルネサンスにお
  いては獣脂(もっぱらラード)のほうが多く用いられる傾向にあった。17
  世紀以降はたとえばラ・ヴァレーヌ『フランス料理の本』におけるアスパ
  ラガスの白いソース添え(\protect\hyperlink{sauce-hollandaise}{ソース・オランデーズ}
  訳注参照)のように、バターを料理に用いることが中世の料理書と比較す
  ると圧倒的に増えたのは事実である。ムノンの1741年刊『ブルジョワ屋敷
  に勤める女性料理人のための本』のバターの項には「良質のバターを用い
  ることは料理でとても重要なことであり、バターが匂いを放っているよう
  ではどんな素晴しい皿も台無しだ。料理担当の女中であればこのことをよ
  く理解しておくことと、良質なバターの価格を手に入れるのに金を惜しん
  ではならないことを肝に銘じておくこと。最良のバターは自然な黄色をし
  ており、白いものは大抵の場合、さして美味しくない。バルボットという
  植物から採った黄色で着色されたバターもある。こういうバターの色は、
  自然なバターの黄色よりもくすんだもので、慣れれば簡単に見分けること
  が出来る(p.320)」}のうちのほとんどは、甲
殻類の合わせバターを除いて、料理に直接用いられることがとても少ない。だ
が、合わせバターはさまざまなシチュエーションで役に立つ。ポタージュでは
野菜の合わせバターが、その他の合わせバターはソース作りにおいて有用だ。
ソースの風味と性格を明確に伝える決め手になるからだ。

だから、読者である料理人諸君には、ここに書いてあることを真剣に読みとっ
ていただきたい。\href{原文における内容矛盾。この後のパラグラフは甲殻類の\%20バターについての注意点ばかりが目立つ}{}

甲殻類のバターについては、経験上、湯煎にかけながら煮出して\footnote{infuser
  アンフュゼ。}から、氷
水で冷やした陶製の容器に布で漉し入れるのがいい。そうすれば、冷たい状態
で作るよりも赤みがきれいに出る。逆に、熱によって風味の繊細さは失なわれ
てしまい、雑味さえも出てしまう。

この問題点を解決するために、我々は二種類の違うバターを作るという方式を
採ることにした。ひとつは甲殻類の胴のクリーム状の部分と切りくずあるいは
身そのものを生のバターとともに鉢ですり潰して、目の細かい網で裏漉しする
か、布で漉すというもの。このバターはソースに完璧ともいうべき風味を添え
てくれる。とりわけベシャメルソースをベースとしたソースの場合はそうだ。

もうひとつは、甲殻類の殻だけを用いて、熱して作るものだ。これは「色付け」
の役割しか持たない。この方式はまことに素晴しい結果を得られるので、ぜひ
とも実行していただきたい。

場合によっては、我々はバターを同様の上等な生クリームに代えることがある。
生クリームのほうがバターよりも、素材の持つ風味や香気をよく吸収する。こ
うすればソースやポタージュの仕上げに加えるのに文句ないクリ\footnote{Coulis
  水分のやや多いピュレをイメージするといい。}を作ることが 出来るわけだ。

色付け用のバターを使うと、ソースがきれいに色付き、個性的なソースとなる。
どんな場合でも、カルミン色素\footnote{コチニール色素ともいう。ラックカイガラムシなどを原料として抽出し
  た色素。ヨーロッパでは古代から中世にかけてケルメスカイガラムシから
  抽出され利用されてきた、非常に歴史の古い色素。とりわけルネサン期に
  は高級毛織物の染料として需要が高まった。また絵の具にも使用された。
  その後、ウチワサボテンでエンジムシを大量に養殖していた中南米を支配
  下に置いたスペインが、これを新大陸産のカルミンとしてヨーロッパ各国
  に売ることで巨万の富を得たという。かつて食品工業において多用された。
  1838年の『ラルース・ガストロノミック』初版では、「コチニールから抽
  出される鮮かな赤色色素で毒性はない。多くの食品に着色料として用いら
  れている」とある。現在は食物アレルギーの原因物質すなわちアレルゲン
  となり得ることがわかり、使用は減りつつある。現在は代替品としてビー
  ツから抽出したビートレッドなどの使用が増えてきている。また、この本
  文でカルミン色素の使用を「くすんだ、情けない色合いを与える」として
  否定的に扱っているのは、この色素がpHによって色調が変化し、なおかつ
  蛋白質を多く含む料理に加えると紫色に変化する(ソースやポタージュ全
  体が濁ったような色になる)ことがあるためだろう。}よりもずっといい。カルミン色素はソース
やポタージュにくすんだ、なさけない色合いしか与えてはくれないのだ。

合わせバターは一般的に、使う際にその都度作る\footnote{原文 au moment
  (オモモン)その都度、の意。à la minute (アラミニュッ
  ト)と呼ぶ調理現場もある。}ものだが、作り置き
しておかなければならない場合は、白い紙で円筒形に包んで冷蔵保管すること。

\vspace*{1.7\zw}
\begin{recette}
\hypertarget{ux306bux3093ux306bux304fux30d0ux30bfux30fc}{%
\subsubsection{にんにくバター}\label{ux306bux3093ux306bux304fux30d0ux30bfux30fc}}

\hypertarget{beurre-d-ail}{%
\paragraph{Beurre d'Ail}\label{beurre-d-ail}}

\index{はたー@バター!あわせはたー@合わせバター!にんにくはたー@にんにくバター}
\index{あわせはたー@合わせバター!にんにくはたー@にんにくバター}
\index{にんにく@にんにく!はたー@---バター}
\index{beurre@beurre!beurres composes@Beurres Composés!beurre d'ail@Beurre d'Ail}
\index{ail@ail!beurre@Beurre d'---}

皮を剥いたにんにく200 gを強火でしっかり茹でる\footnote{生のにんにくには胃腸を刺激する酵素が含まれているが、熱により不活
  性化するので、よく火を通す必要がある。}。よく湯をきってから、鉢
に入れてすり潰し、バター250 gと合わせ、布で漉す。

\maeaki

\hypertarget{ux30a2ux30f3ux30c1ux30e7ux30d3ux30d0ux30bfux30fc}{%
\subsubsection{アンチョビバター}\label{ux30a2ux30f3ux30c1ux30e7ux30d3ux30d0ux30bfux30fc}}

\hypertarget{beurre-d-anchois}{%
\paragraph{Beurre d'Anchois}\label{beurre-d-anchois}}

\index{はたー@バター!あわせはたー@合わせバター!あんちよひはたー@アンチョビバター}
\index{あわせはたー@合わせバター!あんちよひはたー@アンチョビバター}
\index{あんちよひ@アンチョビ!はたー@---バター}
\index{beurre@beurre!beurres composes@Beurres Composés!beurre d'anchois@Beurre d'Anchois}
\index{anchois@anchois!beurre@Beurre d'---}

アンチョビのフィレ200 gをよく洗い、しっかり水気を絞る。これを鉢に入れ
て細かくすり潰す。バター250 gを加えて布で漉す。

\maeaki

\hypertarget{ux30a2ux30fcux30e2ux30f3ux30c9ux30d0ux30bfux30fc}{%
\subsubsection{アーモンドバター}\label{ux30a2ux30fcux30e2ux30f3ux30c9ux30d0ux30bfux30fc}}

\hypertarget{beurre-d-amande}{%
\paragraph{Beurre d'Amande}\label{beurre-d-amande}}

\index{はたー@バター!あわせはたー@合わせバター!あーもんとはたー@アーモンドバター}
\index{あわせはたー@合わせバター!あーもんとはたー@アーモンドバター}
\index{あーもんと@アーモンド!はたー@---バター}
\index{beurre@beurre!beurres composes@Beurres Composés!beurre d'amande@Beurre d'Amande}
\index{amande@amande!beurre@Beurre d'---}

アーモンド\footnote{アーモンドには一般的なスイートアーモンド amandes
  doucesと、苦味 のあるビターアーモンドamande
  amèresの二種がある。後者はあまり多く
  使われることはないが、香りがいいためリキュールなどの香り付けにごく
  少量が用いられることがある。}150
gを湯むきしてよく洗い、すぐに水数滴を加えてすり潰し
てペースト状にする。これをバター250 gと混ぜ合わせ、布で漉す。

\maeaki

\hypertarget{ux30d6ux30fcux30ebux30c0ux30f4ux30eaux30fcux30cc8}{%
\subsubsection[ブール・ダヴリーヌ]{\texorpdfstring{ブール・ダヴリーヌ\footnote{アヴリーヌはヘーゼルナッツの仲間でセイヨウハシバミの大粒な変種。
  イタリア、ピエモンテ産やシチリア産が有名。}}{ブール・ダヴリーヌ}}\label{ux30d6ux30fcux30ebux30c0ux30f4ux30eaux30fcux30cc8}}

\hypertarget{beurre-d-aveline}{%
\paragraph{Beurre d'Aveline}\label{beurre-d-aveline}}

\index{はたー@バター!あわせはたー@合わせバター!ふーるたうりーぬ@ブール・ダヴリーヌ}
\index{あわせはたー@合わせバター!ふーるたうりーぬ@ブール・ダヴリーヌ}
\index{あうりーぬ@アヴリーヌ!ふーる@ブール・---}
\index{へーせるなつつ@ヘーゼルナッツ!ふーるたうりーぬ@ブール・ダヴリーヌ}
\index{beurre@beurre!beurres composes@Beurres Composés!beurre d'aveline@Beurre d'Aveline}
\index{aveline@aveline!beurre@Beurre d'---}

アヴリーヌ150 gを焙煎して丁寧に皮を剥く。油が浮いてこないよう水を数滴
加えてペースト状にすり潰す。これとバター250 gを混ぜ合わせる。目の細か
い網で裏漉しするか、布で漉す。

\maeaki

\hypertarget{ux30d6ux30fcux30ebux30d9ux30ebux30b7ux30fc9}{%
\subsubsection[ブール・ベルシー]{\texorpdfstring{ブール・ベルシー\footnote{\protect\hyperlink{sauce-bercy}{ソース・ベルシー}訳注参照。}}{ブール・ベルシー}}\label{ux30d6ux30fcux30ebux30d9ux30ebux30b7ux30fc9}}

\hypertarget{beurre-bercy}{%
\paragraph{Beurre Bercy}\label{beurre-bercy}}

\index{はたー@バター!あわせはたー@合わせバター!ふーるへるしー@ブール・ベルシー}
\index{あわせはたー@合わせバター!ふーるたへるしー@ブール・ベルシー}
\index{へるしー@ベルシー!ふーる@ブール・---}
\index{beurre@beurre!beurres composes@Beurres Composés!beurre bercy@Beurre Bercy}
\index{bercy@Bercy!beurre@Beurre ---}

白ワイン2 dlに細かく刻んだエシャロット大さじ1杯を加えて半量になるまで
煮詰める。生温い程度まで冷ましてから、ポマード状に柔らかくしたバター 200
gを混ぜ込む。牛骨髄500 gをさいの目に切って\footnote{原文 couper en
  dés。フランス語のまま「デにする(切る)」と表現することもある。}、沸騰しない程度の
湯で火を通し、よく湯ぎりをして加える。パセリのみじん切り大さじ1杯と塩8
g、挽きたてのこしょう1つまみ強とレモン\undemi{}個分の果汁を加えて仕上
げる。

\maeaki

\hypertarget{ux30adux30e3ux30d3ux30a2ux30d0ux30bfux30fc}{%
\subsubsection{キャビアバター}\label{ux30adux30e3ux30d3ux30a2ux30d0ux30bfux30fc}}

\hypertarget{beurre-de-caviar}{%
\paragraph{Beurre de Caviar}\label{beurre-de-caviar}}

\index{はたー@バター!あわせはたー@合わせバター!きやひあはたー@キャビアバター}
\index{あわせはたー@合わせバター!きやひあはたー@キャビアバター}
\index{きやひあ@キャビア!はたー@---バター}
\index{beurre@beurre!beurres composes@Beurres Composés!beurre caviar@Beurre de Caviar}
\index{caviar@caviar!beurre@Beurre de ---}

圧縮キャビア\footnote{もとはロシアで雪の中の樽で保存するために圧縮したもの。キャビア
  のグレードはベルガ、オセトラ、セヴルガが混ざっているのが多いという。}75
gを細かくすり潰す。パター250 gを加えて、布で漉す。

\maeaki

\hypertarget{ux30d6ux30fcux30ebux30b7ux30f4ux30ea12-ux30d6ux30fcux30ebux30e9ux30f4ux30a3ux30b4ux30c3ux30c813}{%
\subsubsection[ブール・シヴリ /
ブール・ラヴィゴット]{\texorpdfstring{ブール・シヴリ\footnote{\protect\hyperlink{sacue-chivry}{ソース・シヴリ}訳注参照。}
/ ブール・ラヴィゴット\footnote{\protect\hyperlink{sauce-ravigote}{ソース・ラヴィゴット}訳注参照。}}{ブール・シヴリ / ブール・ラヴィゴット}}\label{ux30d6ux30fcux30ebux30b7ux30f4ux30ea12-ux30d6ux30fcux30ebux30e9ux30f4ux30a3ux30b4ux30c3ux30c813}}

\hypertarget{beurre-chivry}{%
\paragraph{Beurre Chivry}\label{beurre-chivry}}

\index{はたー@バター!あわせはたー@合わせバター!しうり@ブール・シヴリ}
\index{あわせはたー@合わせバター!しうり@ブール・シヴリ}
\index{はたー@バター!あわせはたー@合わせバター!らういこつと@ブール・ラヴィゴット}
\index{あわせはたー@合わせバター!らういこつと@ブール・ラヴィゴット}
\index{しうり@シヴリ!ふーる@ブール・---}
\index{らういこつと@ラヴィゴット!ふーる@ブール・---}
\index{beurre@beurre!beurres composes@Beurres Composés!beurre chivry@Beurre Chivrya}
\index{beurre@beurre!beurres composes@Beurres Composés!beurre ravigote@Beurre Ravigote}
\index{chivry@Chivry!beurre@Beurre ---}
\index{ravigote@ravitote!beurre@Beurre ---}

パセリの葉とセルフイユ、エストラゴン、シヴレット、若摘みのサラダバーネッ
ト100 gを数分間下茹でし、水にさらしてから圧して余分な水気を絞る。エシャ
ロットのみじん切り25 gも下茹でする。これらを鉢に入れてすり潰す。

バター125 gを加え、布で漉す。

\maeaki

\hypertarget{ux30d6ux30fcux30ebux30b3ux30ebux30d9ux30fcux30eb14}{%
\subsubsection[ブール・コルベール]{\texorpdfstring{ブール・コルベール\footnote{\protect\hyperlink{sauce-colbert}{ソース・コルベール}本文および訳注参照。}}{ブール・コルベール}}\label{ux30d6ux30fcux30ebux30b3ux30ebux30d9ux30fcux30eb14}}

\hypertarget{beurre-colbert}{%
\paragraph{Beurre Colbert}\label{beurre-colbert}}

\index{はたー@バター!あわせはたー@合わせバター!ふーるこるへーる@ブール・コルベール}
\index{あわせはたー@合わせバター!ふーるこるへーる@ブール・コルベール}
\index{こるへーる@コルベール!ふーる@ブール・---}
\index{beurre@beurre!beurres composes@Beurres Composés!beurre colbert@Beurre Colbert}
\index{colbert@Colbert!beurre@Beurre ---}

\protect\hyperlink{beurre-maitre-d-hotel}{メートルドテルバター}200lgに、溶かした\protect\hyperlink{glace-de-viande}{グラス
ドヴィアンド}大さじ2杯と細かく刻んだエストラゴン小さ じ2杯を加える。

\maeaki

\hypertarget{ux8272ux4ed8ux3051ux7528ux306eux8d64ux3044ux30d0ux30bfux30fc}{%
\subsubsection{色付け用の赤いバター}\label{ux8272ux4ed8ux3051ux7528ux306eux8d64ux3044ux30d0ux30bfux30fc}}

\hypertarget{beurre-colorant-rouge}{%
\paragraph{Beurre Colorant rouge}\label{beurre-colorant-rouge}}

\index{はたー@バター!あわせはたー@合わせバター!いろつけようのあかいはたー@色付け用の赤いバター}
\index{あわせはたー@合わせバター!いろつけようのあかいはたー@色付け用の赤いバター}
\index{ちやくしよくそざい@着色素材!いろつけようのあかいはたー@色付け用の赤いバター}
\index{beurre@beurre!beurres composes@Beurres Composés!beurre colorant rouge@Beurre Colorant rouge}
\index{colorant@colorant!beurre rouge@Beurre --- rouge}

出来るだけ沢山の甲殻類の殻などの残りをまとめて用意する。殻の内側、外側に張り付いている膜などをきれいに取り除く。よく乾燥させてから、鉢\footnote{伝統的には大理石製の鉢が用いられることが多かった。}に入れて細かく粉砕して、同じ重さのバターを加える。これを湯煎にかけてよく混ぜながら溶かす。氷水を入れた陶製の器に、布で漉し入れる。固まったバターをトーション\footnote{\protect\hyperlink{sauce-verte}{ソース・ヴェルト}訳注参照。}で包み、余計な水を絞り出す。

\hypertarget{ux539fux6ce8}{%
\subparagraph{【原注】}\label{ux539fux6ce8}}

この色付け用のバターを作るのに用いる甲殻類の殻がどうしてもない場合は、\protect\hyperlink{beurre-de-paprika}{パプリカバター}を用いてもいいだろう。だがいずれにせよ、どんなソースであっても、仕上りの色合いを決めるには、出来るだけ、他の植物由来の赤色着色料の使用は避けることを勧める\footnote{この原注は第三版から。「植物由来の赤色着色料」といっているのはおそらくカルミン色素(コチニール色素)のことと思われる(\protect\hyperlink{observation-sur-les-beurres-composes}{合わせバター「概説」参照})。他に赤系着色料として、ベニバナ色素、紅麹などもあるが、いずれも中国や日本において発達しことを考慮すると、両大戦間である1920年頃に「避けるべき」というほど普及していたのは、実際には昆虫由来であるコチニール色素と思われる。なお、ベニバナ色素も化学的にはカルミン酸色素。}。
\end{recette}
\href{未、原文対照チェック}{} \href{未、日本語表現校正}{}
\href{未、その他修正}{} \href{未、原稿最終校正}{}

\hypertarget{marinades-et-saumures}{%
\section[マリナードとソミュール]{\texorpdfstring{マリナードとソミュール\footnote{マリナードはマリネ液とも言う。marinade
  \textless{} mariner
  (マリネ)語源はラテン語のmare(海)。古フランス語では「海で泳ぐ、海に潜る」の意で使われていたが、16世紀には既に、料理用語として用いられていたようだ。ラブレー『ガルガンチュアとパンタグリュエル』第四の書(1548年)において、lancerons
  marinez (マリネしたブロシェの幼魚)という表現が見られる。なおブロシェ
  brochet
  はノーザンパイク、和名キタカワカマス。川カマス属の淡水、汽水魚。この場面はパンタグリュエルに「小斉」のご馳走として捧げられた料理のリストの一部であり、「塩漬けのメルルーサ、卵料理各種、モリュ(塩漬けにした鱈)、アドック(塩漬け後に燻製にした鱈)」などとともに列挙されており、いずれも塩辛いから、食後の消化をよくするために飲むワインの量が倍になった(p.681)とある。したがって、lancerons
  marinezのマリネとは「海水あるいは塩水に漬けた」の意に解釈されよう。一方、ソミュールについては、11世紀末頃に、「保存のため漬け込む塩水」の意味で
  salmuire
  という語形が使用され、16世紀には「塩水およびその他の液体からなるもの」としてsaumureという現在とおなじ語形が記録されている。マリナードとソミュールが明確に分化したのはおそらく17世紀頃。1651年刊ラ・ヴァレーヌ『フランス料理の本』に見られるマリナードの語には曖昧さが残っているが、例えば
  \emph{Poulets
  marinez}(鶏のマリネ)というレシピは「鶏を開いて叩き、しっかり味付けしたヴィネガーに漬ける。小麦粉をまぶすか、卵と小麦粉で作った衣を付けてラードで揚げる。マリナードに戻し入れて軽く弱火で煮てから供する(p.36)」。また、\emph{Longe
  de
  mouton}(仔羊の腰肉のロースト)は、「棒状に切った豚背脂をラルデ針を使って刺し込み、串を刺してローストする。玉ねぎ、塩、こしょう、ごく少量のオレンジまたはレモンの外皮(ゼスト)とブイヨンとヴィネガーでマリナードを作る。肉に火が通ったら、ソース(マリナード)とともに弱火で煮込む。とろみ付けには小麦粉をラードで茶色くなるまで炒めたもの、すなわち後代のルーの原型といえるものを少々加える(p.80)」とあり、別の項目では「(串を刺した肉の下の受け皿にある)マリナードを小まめにかけながらローストする(p.106)」とある。全体的な印象としてはラ・ヴァレーヌのマリナードとは中世のドディーヌにヴィネガーを効かせたもののようにも受け取れるが、最初に見たように、「漬け込む」ものとしてもヴィネガーを用いている点に注目すべきだろう。これは18、19世紀に引継がれ、1756年マラン『コモス神の贈り物』第1巻において、\emph{Cervelle
  de veau en
  marinade}(仔牛の脳のマリナード仕立て)では、血抜きした仔牛の脳を豚背脂のシートで包みブイヨン少々で茹で、「冷ましてからヴィネガーもしくはレモン果汁に漬け込む。その後、水気をきって溶き卵に浸し、パン粉をつけて揚げる。小麦粉を溶いた揚げ衣に浸して揚げてもいい(p.206)」とある。19世紀初頭のヴィアールも同様で、『帝国料理の本』初版(1806年)において、\emph{Pieds
  d'agneau en marinade} (仔羊の足のマリナード仕立て)などいくつかの
  marinadeを冠するレシピが掲載されている。肝心のマリナードについての記述は欠落しているが、この版においてはよく見られる現象。なお、仔羊の足のマリナード仕立ては、マリナードがない場合は「塩、こしょう、ビネガーに、茹でた仔羊の足を漬けてから、揚げ衣を付けて揚げる(p.214)」となっている。1814年ボヴィリエ『調理技法』では「加熱マリナード」のレシピが掲載されている。これは、卵くらいの大きさのバターを鍋に入れ、輪切りにしたにんじん1、2本、同様にした玉ねぎ、ローリエの葉1枚、にんにく1片、タイム、バジル、枝ごとのパセリ、シブール(≒葱)2〜3本を加えて強火で炒める。野菜が色付きはじめたら、約250
  mLの白ワインヴィネガーと約0.5
  Lの水を注ぎ、塩、こしょうする。そのまま沸かしてから漉し、必要に応じて使う(pp.60-61)、というもの。もっとも、仔牛の脳のマリナード仕立てなどマランのレシピと大差ない揚げものも同書に掲載されている。また、1834年版のオドにおいても鶏のマリナードはラ・ヴァレーヌのものと同工異曲のものに留まっている。1837年版でロースト用マリナードの項が追加され、豚背脂とにんにく1片を細かく刻み、パセリ1つまり、塩、こしょう、ヴィネガー大さじ1杯、油大さじ4杯を合わせてよく混ぜる
  (p.419)。1853年版ではマリネしたうなぎのグリル焼き、というレシピが掲載される。これは、皮を剥いてぶつ切りにし、バターでソテーしたうなぎを深皿に並べ、塩、こしょうハーブ、マッシュルーム、細かく刻んだエシャロットとシブールを被せ、油大さじ1杯をかける。2〜3時間マリネしたら、パン粉をまぶしてグリル焼きする(p.310)というもの。また、
  mariner(マリネ)という動詞は、オドの1834年版でもに、ノロ鹿の腿肉のローストにおいて、「オリーブオイルと塩で5〜6時間マリネする」
  (p.155)という記述が見られる。1867年刊グフェ『料理の本』では、ヴィネガーをベースとしたソースとしてのマリナード(p.404)と仕立てとしてのマリナードがある。後者の例としては
  \emph{Tête de veau en marinade}
  (仔牛の頭 マリナード仕立て)が好例だろう。仔牛の頭肉半分を3
  cm角に切り、下茹でしてから水にさらし、牛脂と小麦粉、香草類を加えた湯で茹でる。これを、塩、こしょう、油、ヴィネガーに1時間漬け込む。水気をきって揚げ衣を付けて油で揚げる(p.156)。ここでは肉を漬け込む液体としてmarinadeの語が用いられている。このように、marinadeという名詞とmariner「漬け込む」という動詞の用法に若干の不統一が見られるため、『料理の手引き』におけるマリナードすなわちマリネ液、という概念は比較的新しいものと思われる。}}{マリナードとソミュール}}\label{marinades-et-saumures}}

\frsec{Marinades et Saumures}

\index{marinade saumures@marinade et saumures} \index{marinade@marinade}
\index{まりなーととそみゆーる@マリナードとソミュール}
\index{まりなーと@マリナード}

\vspace{1\zw}

マリナードとソミュールにはいろいろな種類があるが、最終的な目的は同じで、

\begin{enumerate}
\def\labelenumi{\arabic{enumi}.}
\item
  素材に料理で使う香辛料やハーブの香りを浸み込ませる
\item
  ある種の肉を柔らかくさせる
\item
  場合によっては保存のために用いる。とりわけ温度と湿度で素材が駄目になってしまうような場合。さらに、目指す料理の仕上がりに合わせて素材の状態を調節する
\end{enumerate}
\begin{recette}
\hypertarget{marinade-instantanee}{%
\subsubsection{即席マリナード}\label{marinade-instantanee}}

\frsub{Marinade instantanée}

\index{marinade@marinade!marinade instantanee@marinade instantanée}
\index{まりなーと@マリナード!そくせき@即席---}

このマリナードはすぐに素材を使う場合、例えば赤身肉のグリル焼きや、ガランティーヌ、テリーヌ、パテのような冷製料理の補助材料\footnote{具体的には\protect\hyperlink{farces}{ファルス}のこと。}にする肉に用いる。

\begin{enumerate}
\def\labelenumi{\arabic{enumi}.}
\item
  グリル焼きにする肉の場合\ldots{}\ldots{}ごく薄くスライスしたエシャロットとパセリの枝、タイムの枝、ローリエの葉を肉の上に散らす。量は適宜加減すること。レモン果汁
  \(\frac{1}{2}\)個分に対して油大さじ1杯の割合で、上からかけてやる。
\item
  仔牛、ジビエのフィレ肉、ハム、豚背脂などを細かく切ったもの\footnote{原文
    lardon
    (ラルドン)、通常は拍子木状に切ったものを言うが、ここではファルスとして後で細かく挽くことになるので、形状はあまり問題にならない。}の場合\ldots{}\ldots{}塩こしょうしてから、白ワイン3、コニャック3、油1の割合のマリナードを上からかけてやる。
\end{enumerate}

ここで用いた風味付けの材料は、後でファルスにする際に加えることになる。

いずれの場合でも、マリナードに浸した肉を小まめに裏返してやり、マリナードがよく浸み込むようにしてやること。

\hypertarget{marinade-crue-pour-viandes-de-boucherie-ou-venaison}{%
\subsubsection{牛、羊肉および大型ジビエ用の非加熱マリナード}\label{marinade-crue-pour-viandes-de-boucherie-ou-venaison}}

\frsub{Marinade crue pour viandes de boucherie ou venaison}

\index{marinade@marinade!marinade crue viande boucherie venaison@marinade crue pour viande de boucherie ou venaison}
\index{まりなーと@マリナード!うしひつしおおかたしひえようひかねつ@牛、羊肉および大型ジビエ用非加熱---}

(仕上がり2 L分)

\begin{itemize}
\item
  香味素材\ldots{}\ldots{}にんじん100 g、玉ねぎ100 g、エシャロット40
  g、セロリ30 g、にんにく2片、パセリの枝3本、タイム1枝、ローリエの葉
  \(\frac{1}{2}\)枚、大粒のこしょう6個、クローブ2本。
\item
  使用する液体\ldots{}\ldots{}白ワイン1 \(\frac{1}{4}\) L、ヴィネガー5
  dL、油2 \(\frac{1}{2}\) dL。
\item
  作業手順\ldots{}\ldots{}マリネする素材に塩とこしょうを振る。にんじん、玉ねぎ、エシャロットを薄切り\footnote{émincer
    (エマンセ)薄切りにする、スライスする。}にし、半量を容器の底に敷く。容器の大きさは素材とマリナードがぴったり入る程度のものを用いること。素材を入れて、残りの香味野菜で蓋をするようにして、白ワインとヴィネガー、油を注ぎ入れる。
\end{itemize}

冷蔵し、マリネ液に漬かった素材を小まめに裏返してやること。

\hypertarget{marinade-cuite-pour-viandes-de-boucherie-ou-venaison}{%
\subsubsection{牛、羊肉および大型ジビエ用の加熱マリナード}\label{marinade-cuite-pour-viandes-de-boucherie-ou-venaison}}

\frsub{Marinade cuite pour viandes de boucherie ou venaison}

\index{marinade@marinade!marinade cuite viande boucherie venaison@marinade cuite pour viande de boucherie ou venaison}
\index{まりなーと@マリナード!うしひつしおおかたしひえようかねつ@牛、羊肉および大型ジビエ用加熱---}

(仕上がり2 L分)

\begin{itemize}
\item
  香味素材\ldots{}\ldots{}非加熱マリナードと同じ材料で同じ分量
\item
  使用する液体\ldots{}\ldots{}白ワイン1 \(\frac{1}{2}\) L、ヴィネガー3
  dL、油2 \(\frac{1}{2}\) dL。
\item
  作業手順\ldots{}\ldots{}鍋に油を熱し、ごく薄くスライスしたにんじん、玉ねぎ、エシャロットおよびその他の香味素材を軽く色付くまで炒める。

  白ワインとヴィネガーを注ぎ、弱火で約30分間火を通す。

  必ず、マリナードが完全に冷めてからマリネする素材にかけること。
\end{itemize}

\hypertarget{marinade-crue-ou-cuite-pour-grosse-venaison}{%
\subsubsection[とりわけ大型のジビエ用、非加熱および加熱マリナード]{\texorpdfstring{とりわけ大型のジビエ\footnote{具体的には赤鹿
  cerf(セール)
  や猪、トナカイの成獣など。ニホンジカやエゾジカはcerfに分類されるので、これを参考にするといいだろう。}用、非加熱および加熱マリナード}{とりわけ大型のジビエ用、非加熱および加熱マリナード}}\label{marinade-crue-ou-cuite-pour-grosse-venaison}}

\frsub{Marinade crue ou cuite pour grosse venaison}

\index{marinade@marinade!marinade crue cuite grosse venaison@marinade crue ou cuite pour grosse venaison}
\index{まりなーと@マリナード!とりわけおおかたのしひえようひかねつおよひかねつ@とりわけ大型のジビエ用非加熱および加熱---}

(仕上がり2 L分)

\begin{itemize}
\item
  香味素材\ldots{}\ldots{}牛、羊肉および大型ジビエ用のマリナードと同じだが、ローズマリー12
  gを追加する。
\item
  使用する液体\ldots{}\ldots{}ヴィネガー16 dL、油4 dL。
\item
  作業手順\ldots{}\ldots{}非加熱、加熱ともに作業手順は上記のレシピのとおり。
\end{itemize}

\hypertarget{marinade-cuite-pour-le-mouton-en-chevreuil}{%
\subsubsection[羊のシュヴルイユ仕立て用の加熱マリナード]{\texorpdfstring{羊のシュヴルイユ仕立て用の加熱マリナード\footnote{\protect\hyperlink{sauce-chevreuil}{ソース・シュヴルイユ}参照。}}{羊のシュヴルイユ仕立て用の加熱マリナード}}\label{marinade-cuite-pour-le-mouton-en-chevreuil}}

\frsub{Marinade cuite pour le mouton en chevreuil}

\index{marinade@marinade!marinade cuite mouton en chevreuil@marinade cuite pour le mouton en chevreuil}
\index{まりなーと@マリナード!ひつしのしゆうるいゆしたてようのかねつまりなーと@羊のシュヴルイユ仕立て用加熱---}

(仕上がり2 L分)

\begin{itemize}
\item
  香味素材\ldots{}\ldots{}上記のとおりの分量の素材に、ジュニパーベリー\footnote{セイヨウネズの実。ジンの香り付けに用いられている。}10粒とバジル1つまみ、ローズマリー1つまみを足す。
\item
  使用する液体\ldots{}\ldots{}牛、羊および大型ジビエ用の加熱マリナードと同じ。
\item
  作業手順\ldots{}\ldots{}鍋に油を熱し、薄切りにしたにんじん、玉ねぎ、エシャロットおよびその他の香味素材を軽く色付くまで炒める。

  白ワインとヴィネガーを注ぎ、弱火で約30分間火を通す。
\end{itemize}

\hypertarget{marinade-cuite-pour-le-mouton-en-chamois}{%
\subsubsection[羊のシャモワ仕立て用の加熱マリナード]{\texorpdfstring{羊のシャモワ仕立て\footnote{オートザルプ県の山岳地帯およびピレネー山脈に生息する野生の山羊。ピレネー山脈のものは
  Isard
  (イザール)と呼ばれる。若い獣の肉は大型ジビエのなかでもとりわけ美味とされる。成獣の肉は固く、しっかりマリネする必要があると言われている。しばしばノロ鹿と比較される。ここでは、羊肉を白ワインベースのマリナードに漬け込む仕立て、すなわちシュヴルイユ仕立てとの対比として、赤ワインでより強い風味のマリナードに漬け込むことで、シャモワ仕立てとしている。なお、本書にシャモワ仕立てのレシピは掲載されていない。シュヴルイユ仕立てと同様と考えていい。}用の加熱マリナード}{羊のシャモワ仕立て用の加熱マリナード}}\label{marinade-cuite-pour-le-mouton-en-chamois}}

\frsub{Marinade cuite pour le mouton en chamois}

\index{marinade@marinade!marinade cuite mouton en chevreuil@marinade cuite pour le mouton en chevreuil}
\index{まりなーと@マリナード!ひつしのしやもわしたてようのかねつまりなーと@羊のシャモワ仕立て用加熱---}

(仕上がり2 L分)

\begin{itemize}
\item
  香味素材\ldots{}\ldots{}非加熱マリナードと同じ分量の素材に、ジュニパーベリー\footnote{セイヨウネズの実。ジンの香り付けに用いられている。}15粒とバジル15
  g、ローズマリー15 gを足す。
\item
  使用する液体\ldots{}\ldots{}良質な赤ワイン1 \(\frac{1}{2}\)
  L、ヴィネガー3 dL、油2 \(\frac{1}{2}\) dL。
\item
  作業手順\ldots{}\ldots{}上記と同じ。

  このマリナードに上等な赤ワインを使える場合には、素材の量を次のように調整すること。赤ワイン12
  dL、ワインヴィネガー6 dL、油は上記の分量とする。

  ワインの酸味の強さによっては、ヴィネガーの量をワインと同量にすることさえ可能。
\end{itemize}

\hypertarget{observation-sur-les-marinades}{%
\subparagraph{マリナードについての注意事項}\label{observation-sur-les-marinades}}

\ldots{}\ldots{} 1.
加熱マリナードを使用するのは、素材へのマリナードの浸透作用を促進するのが目的。素材をマリナードに漬け込む時間は、加熱、非加熱ともに、素材の種類と大きさ、気温、環境の変化を勘案して決めること。

\begin{enumerate}
\def\labelenumi{\arabic{enumi}.}
\setcounter{enumi}{1}
\tightlist
\item
  一般的な牛、羊肉と肉質の柔らかい大型ジビエに使うマリナードに純粋な酢酸を用いるのは絶対にやめておくこと。酢酸の腐食作用によって肉の風味が失なわれてしまうからだ\footnote{この注記は第二版から。内容が当時の知見にもとづいたものであることに注意。ただし、19世紀には木酢液を原料として工業用の氷酢酸が既に製造されていた。また、タンパク質はpHの変化によって分解されるので、マリナードにヴィネガーを加えるのは理にかなっている。なお、肉を柔らかくする効果のあるタンパク質分解酵素(プロテアーゼ)の代表的なひとつであるパパインの発見は1940年代になってからのこと。パイナップルに含まれているブロメラインの効果は経験的に知られていた可能性もあるが、この酵素が60℃で不活性化することが広く知られるようになったのは、少なくとも日本では比較的近年のことに過ぎない。}。猪、赤鹿\footnote{cerf
    (セール)、ニホンジカやエゾジカもフランス語で表現するとこれに含まれるので、これらの料理について
    chevreuil (しゅう゛るいゆ)ノロ鹿の名をつけるは、厳密には誤り。}、トナカイなどの固い肉についても、純粋な酢酸だけを使うのは不可。
\end{enumerate}

\hypertarget{conservation-des-marinades}{%
\subsubsection{マリナードの保存方法}\label{conservation-des-marinades}}

\frsub{Conservation des marinades}

\index{marinade@marinade!conservation marinades@conservation des marinades}
\index{まりなーと@マリナード!ほそんほうほう@---の保存方法}

マリナードを長期間保存しておく必要がある場合には、とりわけ夏場は、本書で示した分量に対して2〜3
gのホウ酸を加えるといい。

さらに、夏のあいだは2日に一度、冬季は4〜5日に一度、マリナードを沸騰させ、冷めたら毎回そのマリナードに使っているのと同じワインを
2 dLとヴィネガー1 dLを足してやること。
\end{recette}
\hypertarget{saumures}{%
\subsection[ソミュール]{\texorpdfstring{ソミュール\footnote{この見出しは第四版のみ。初版〜第三版にかけては、マリナードとソミュールのレシピの間に区切りをつけるものは何も挿入されていない。}}{ソミュール}}\label{saumures}}

\frsecb{Saumures}

\index{saumure@saumure} \index{そみゆーる@ソミュール}
\begin{recette}
\hypertarget{saumure-au-sel}{%
\subsubsection{塩漬け用ソミュール}\label{saumure-au-sel}}

\frsub{Saumure au sel}

\index{saumure@saumure!sel@--- au sel}
\index{そみゆーる@ソミュール!しおつけよう@塩漬け用---}

このソミュールは、グレーソルト\footnote{フランス語は sel gris
  (セルグリ)または gros gris (グログリ)。灰色がかった粗塩。}1
kgに対して硝石\footnote{原文 salpêtre
  (サルペートル)硝酸カリウム。殺菌作用と、肉類を赤く発色させる効果を持つ。現代の日本では亜硝酸カリウム、亜硝酸ナトリウムが使われることが多い。いずれも日本では劇物指定されているが、シャルキュトリ(豚肉加工品の製造)においては不可欠とも言われるな薬品であり、とりわけボツリヌス菌対策の効果が大きい。そのため劇物ではあるが、食品添加物として認められており、使用限界量が厳密に定められている(食品添加物は国あるいは地域によって扱いが異なるので注意)。硝酸塩あるいは亜硝酸塩による肉の赤い発色を「着色料によるもの」と誤認する消費者は少なくない。これはかつて「魚肉ソーセージ」がコチニール色素でピンク色に染められていたことから連想される誤認と思われる。また、食品添加物イコール毒という安直な考えから忌避する消費者も少なくないのは事実だろう。こうしたことから、現代日本のレストランでは、製造後すぐに提供可能であるために、これら硝酸塩、亜硝酸塩の類を用いないところもある。}40
gの割合で作る。この硝石入りの塩の総量は、塩漬けにする肉の数と大きさで決まる。素材が完全に覆えて、重しが出来る分量とすること。

\begin{itemize}
\tightlist
\item
  作業手順\ldots{}\ldots{}肉を塩漬けにする前にまず、太い針を充分深く刺して穴を何箇所も空ける。次に硝石の粉末を肉の表面にすり付ける。塩1
  kgあたりタイム1枝、ローリエの葉
  \(\frac{1}{2}\)枚を加えて肉と塩を容器に詰める。
\end{itemize}

\hypertarget{saumure-liquide-pour-langues}{%
\subsubsection[舌肉用の液体ソミュール]{\texorpdfstring{舌肉用の液体ソミュール\footnote{このソミュールに舌肉を漬け込むと、硝石の作用で舌肉が赤く発色する。それを拍子木状などに切って鶏やフィレ肉の表面に、同様に切ったトリュフや豚背脂などとともに刺して装飾することが19世紀〜20世紀初頭までよく行なわれた。現代ではほとんど行なわれなくなった装飾方法。この場合はあくまでも料理の装飾を目的としたものであり、牛や豚の舌肉を保存食として利用する場合には塩漬けや燻製などの方法も用いられる。}}{舌肉用の液体ソミュール}}\label{saumure-liquide-pour-langues}}

\frsub{Saumure liquide pour langues}

\index{saumure@saumure!liquide langues@--- liquide pour langues}
\index{そみゆーる@ソミュール!したにくようのえきたい@舌肉用の液体---}

\begin{itemize}
\item
  材料\ldots{}\ldots{}水5 L、グレーソルト2.25 kg、硝石150
  g、茶色いカソナード\footnote{砂糖きびを原料とした粗糖。通常は茶褐色のものが多く「赤糖」とも呼ばれるが、精白したものもある。精製が不完全であるため独特の風味があり、料理および製菓でしばしば用いられる。}
  300 g、こしょう12 g、ジュニパーベリー12粒、タイム1枝、ローリエの葉1
  枚。
\item
  作業手順\ldots{}\ldots{}充分な大きさの鍋に材料を全て入れ、強火で沸騰させる。その後、完全に冷めてから、針で穴を複数空けて硝石をしっかりすり込んだ舌肉を入れた容器に注ぎ込む。平均的な重さの舌肉を漬け込む期間は冬季で8日間、夏季は6日間。
\end{itemize}

\hypertarget{grande-saumure}{%
\subsubsection[グランドソミュール]{\texorpdfstring{グランドソミュール\footnote{この項は第二版で追加された。通常はシャルキュティエすなわちシャルキュトリ専門の職人が行なう規模のものであり、料理人の仕事の範疇をやや越えるとも考えられる。}}{グランドソミュール}}\label{grande-saumure}}

\frsub{Grande saumure}

\index{saumure@saumure!grande@grande ---}
\index{そみゆーる@ソミュール!くらんと@グランド---}

(仕上がり50 L分)

\begin{itemize}
\tightlist
\item
  水\ldots{}\ldots{}50 L
\item
  塩\ldots{}\ldots{}25 kg
\item
  硝石\ldots{}\ldots{}2.7 kg
\item
  カソナード\ldots{}\ldots{}1.6 kg
\item
  作業手順\ldots{}\ldots{}メッキされた銅の鍋に材料を全て入れ、強火にかける。沸騰したら、皮を剥いたじゃがいも1個を投入する。じゃがいもが浮いてくるようであれば、じゃがいもが沈みはじめる寸前まで水を足す。逆に、じゃがいもが完全に底まで沈んでしまうようなら、じゃがいもが水面に見えてくるまで煮詰める必要がある。
\end{itemize}

ソミュールがちょうどいい具合になったら、鍋を火から外して、このソミュールで漬け込み槽に注ぎ込む。漬け込み槽の素材は、スレート製、岩製、セメント製、あるいはレンガ製でしっかりエナメル引きしたものを用いること。

漬け込み槽の底に、木製の網を敷き、その上に漬け込む肉を置くといい。肉が槽の底面に直接当たっていると、肉の下側にソミュール液が浸透しない可能性がある。

漬け込む肉は、たとえ小さなものであっても、専用の携行可能な注入器具を使ってソミュールを内部に注入してから、漬け込み槽に入れてやること。この準備作業を怠ると、肉全体が均等に塩漬けにならない可能性がある。肉の中心部がちょうどいい塩加減になる頃には外側は塩が強すぎるということになってしまうのだ。牛のランプ、イチボなどの塊肉で、4〜5
kgの大きさの場合は、ソミュール液を注入してやる方法を使えば8日間で漬かる。

牛舌肉をこの方法で漬ける場合は、出来るだけ新鮮なものを用いる必要がある。軟骨部分をきれいに取り除いてやり、肉叩きか麺棒で丁寧に叩いてやる。ブリデ針\footnote{主として鶏などの手羽や腿をまとめて整形し、その形状を保つよう糸で縫う際に用いる縫い針。}を使って、表面全体に刺し穴をつけてやる。それからソミュールに漬け込むが、何らかの重しをして浮き上がらないようにしてやること。

\hypertarget{observation-grande-saumure}{%
\subparagraph{【原注】}\label{observation-grande-saumure}}

ソミュールはマリナードほどは腐敗しにくいとはいえ、天候が悪い時季などはとりわけ、よく様子を見て、時々は沸騰させてやるのがいい。沸騰させれば多少は濃縮されてしまうから、本文記載の方法でじゃがいもを用いて、毎回少量の水を加える必要がある。
\end{recette}
\hypertarget{gelees-diverses}{%
\section{ジュレ}\label{gelees-diverses}}

\frsec{Gelées diverses}

\index{gelee@gelée} \index{しゆれ@ジュレ}

どんなジュレも、ベースとなっているのはほぼ全てフォンだ。だから、フォンのメインとなっている素材によってジュレの風味が決まるわけだ。その結果としてジュレの用途も\ruby{自}{おの}ずと決まってくる。

人工的な凝固剤を使わずにジュレを確実に固めるためには、フォンのメインとなる素材に、仔牛の足や豚皮のようなゼラチン質の量を計算して加えることになる。仔牛の足や豚の皮を使えば、ジュレを確実に凝固させられるし、しかも柔らかな口あたりに仕上げられる。

そうはいっても、とりわけ夏季には、クラリフィエ\footnote{clarifier
  \textgreater{} clarification 次項参照。}の作業を行なう前に必ず、フォンを氷の上に垂らしてみて、固さと濃度を確認し、必要があれば板ゼラチンを何枚か加えてやること。

追加する板ゼラチンの量は、どんな場合でも、フォン1 Lあたり9
g(6枚)を越えないこと。板ゼラチンは、透き通っていてぱりぱりと割れやすく、
\ruby{膠}{にかわ}っぽい味のしないものを選ぶこと。必ず冷水でもどしてから使うか、せめてよく洗ってから用いること。

標準的なジュレを作る際に人工着色料を使うことはお勧め出来ない。標準的なジュレは充分に色よく仕上がるものだ。さらに、最後にマデイラ酒を加えてやれば充分に、標準的なジュレの特徴ともいえる淡い琥珀色に仕上がる。
\begin{recette}
\hypertarget{fonds-pour-gelee-ordinaire}{%
\subsubsection[標準的なジュレ用のフォン]{\texorpdfstring{標準的なジュレ用のフォン\footnote{この項および次の「白いジュレ用のフォン」は初版と第二版以降の異同が大きい。この「標準的なジュレ用のフォン」は初版では使用する液体が
  8 litres et demi de remouillage
  いわゆる「二番のフォン」であり、加熱時間も6時間と短かい。第二版は「水8.5
  L」になるが、加熱時間は6時間のままで、作業手順が「ソース用の白いフォンと同じ」となっている。第三版で現在の記述となった。}}{標準的なジュレ用のフォン}}\label{fonds-pour-gelee-ordinaire}}

\frsub{Fonds pour gelée ordinaire}

\index{gelee@gelée!fonds ordinaire@Fonds pour gelée ordinaire}
\index{しゆれ@ジュレ!ひようひゆんてきなしゆれようのふおん@標準的なジュレ用のフォン}

(仕上がり5 L分)

\begin{itemize}
\item
  主素材\ldots{}\ldots{}仔牛のすね肉とバモルソー\footnote{原文 bas
    morceaux 煮込みなどに用いる部位の総称。bas
    は「低い」が原義であり、食材として低級な部位というニュアンス。}2
  kg、細かく砕いた仔牛の骨1.5 kg、牛の脚肉1.5
  kg\ldots{}\ldots{}これらの肉と骨はオーブンで軽く色付けておくこと。
\item
  ゼラチン質\ldots{}\ldots{}骨を取り除いて\footnote{原文 désosser
    (デゾセ)骨を取り除く。}下茹でした\footnote{原文 blanchir
    (ブランシール)。下茹ですることがだ、原義は「白くする」。もとは中世において肉を調理する際にはローストであれ煮込みであれ、ほぼ必ず下茹でしていた。赤い肉を茹でると表面が白くなることからこの用語が定着することになったが、現在ではもっぱら野菜の下茹でなどについて言うことがほとんど。「ブランシェ」と言う現場もあるようだが、もとのフランス語からやや離れているので「ブランシール」で覚えるといいだろう。}仔牛の足3本、背脂を付けたままの生の豚皮\footnote{塩漬けなどの加工をしていない、ということ。}250
  g。
\item
  香味素材\ldots{}\ldots{}にんじん200 g、玉ねぎ200 g、ポワロー\footnote{poireau(x)
    ポロねぎ。日本の長葱とは異なり、植物としてはむしろ、にんにくに近いが、風味はかなり異なる。古代ローマ時代からヨーロッパで広く親しまれてきた野菜のひとつ。ローマ皇帝ネロが演説で大きな声を出すために、ポワローの蜂蜜漬けを好んだという逸話がある。伝統的な栽培方法の場合、旬は秋〜冬。播種から収穫まで10ヶ月以上かかる品種も多い。太さ3〜5
    cm、軟白部が20〜40
    cmくらいのものが多い。フランスの標準的な規格では軟白部20
    cm以上。かつては日本の長葱と同様に成長に応じて「土寄せ」して栽培していたが、その方法では内部に土砂が入りやすい。また、太さ1
    cm程のミニ・ポワローも付け合わせ用の高級野菜として人気がある。元来ミニ・ポワローは苗の「間引き」を利用したものだったが、現在ではミニ・ポワローむけの品種も開発されている。いずれもヨーロッパでは大型機械を用いた大量生産が一般的。日本にも秋〜冬季はヨーロッパ産が、春〜夏季はオーストラリア産が安定的に輸入されている。日本国内での生産も明治以降、試みられてはいるが、需給バランスとコスト的に見合わないために断念せざるを得ないケースも少なくないようだ。なお、第二次大戦前は八丈島などでこうした西洋野菜の栽培が行なわれ、船便で東京まで運ばれていたという(cf.~大木健二『大木健二の洋菜ものがたり』日本デシマル、1997年)。なお、現代フランス語でブレット(ふだんそう)のことを
    poirée (ポワレ)とも呼ぶが、これは ポワロー poireau
    と同語源。中世の料理書にはしばしば、野菜をペースト状になるまで煮込んだポタージュとして
    porée
    (ポレ)というものが出てくるが、どちらを材料として用いているか判別できないケースもある。}50
  g、セロリ 50 g、充分な香りと量のブーケガルニ。
\item
  使用する液体\ldots{}\ldots{}水 8.5 L。
\item
  加熱時間\ldots{}\ldots{}6時間。
\item
  作業手順\ldots{}\ldots{}ソース用の\protect\hyperlink{fonds-brun}{茶色いフォン}とまったく同じ。ただし、ジュレ用のフォンの色合いはソース用のフォンよりも薄くしておくこと。
\end{itemize}

\hypertarget{fonds-pour-gelee-blanche}{%
\subsubsection[白いジュレ用のフォン]{\texorpdfstring{白いジュレ用のフォン\footnote{初版全文は「主素材、ゼラチン質、香味素材は上記のとおり。注ぐ液体は水(原文
  mouillage à
  blanc)、作業手順は基本の白いフォンと同様」。第二版で現在の記述となっている。この文脈からすると、白いフォンの二番を使うとも解釈され得るが、前項の「標準的なジュレ用のフォン」が最終的に水を用いて作ることになっているのと比較すると、加熱時間および作業手順が何と同様なのか曖昧になってしまうため、ここでは液体、加熱時間、作業手順を\protect\hyperlink{fonds-blanc}{標準的な白いフォン}と同じと解釈した。なお、英訳第5版では、but
  use very white stock instead of
  water「水ではなく白いフォン」を注ぐとなっている。}}{白いジュレ用のフォン}}\label{fonds-pour-gelee-blanche}}

\frsub{Fonds pour gelée blanche}

\index{gelee@gelée!fonds ordinaire@Fonds pour gelée ordinaire}
\index{しゆれ@ジュレ!しろいしゆれようのふおん@白いジュレ用のフォン}

主素材、ゼラチン質、香味素材の種類と分量は前記の\protect\hyperlink{fonds-pour-gelee-ordinaire}{標準的なジュレ用のフォン}を参照。

使用する液体の量は\protect\hyperlink{fonds-blanc}{標準的な白いフォン}とまったく同じにすること。

加熱時間も作業手順も同様。

\hypertarget{fonds-pour-gelee-de-volaille}{%
\subsubsection{鶏のジュレ用のフォン}\label{fonds-pour-gelee-de-volaille}}

\frsub{Fonds pour gelée de volaille}

\index{gelee@gelée!fonds volaille@Fonds pour gelée de volaille}
\index{しゆれ@ジュレ!とりのしゆれようのふおん@鶏のジュレ用のフォン}
\index{しゆれ@ジュレ!うおらいゆのしゆれようのふおん@ヴォライユのジュレ用のフォン ⇒ 鶏のジュレ用のフォン}
\index{とり@鶏!しゆれ@ジュレ!ふおん@---のジュレ用のフォン}
\index{うおらいゆ@ヴォライユ!しゆれ@ジュレ!うおらいゆのふおん@ヴォライユのジュレ用のフォン ⇒ 鶏のジュレ用のフォン}

(仕上がり5 L分)

\begin{itemize}
\item
  主素材\ldots{}\ldots{}仔牛のすね肉1.5 kg、牛の脚肉1.5
  kg、細かく砕いた仔牛の骨1.5
  kg、鶏ガラ、とさか、手羽先、足など(とりわけ湯通しした手羽と足)、1.5
  kg。
\item
  ゼラチン質\ldots{}\ldots{}骨を取り除いて下茹でした仔牛の足(小)3本。
\item
  香味素材\ldots{}\ldots{}材料の種類は標準的なジュレ用のフォンと同じだが、量はやや少なめにすること。
\item
  使用する液体\ldots{}\ldots{}軽く仕上げた\protect\hyperlink{fonds-blanc}{白いフォン}
  8 L。
\item
  加熱時間\ldots{}\ldots{}4時間半。
\item
  作業手順\ldots{}\ldots{}ソース用の\protect\hyperlink{fonds-de-volaille}{鶏のフォン}とまったく同じ。
\end{itemize}

\hypertarget{fonds-pour-gelee-de-gibier}{%
\subsubsection{ジビエのジュレ用のフォン}\label{fonds-pour-gelee-de-gibier}}

\frsub{Fonds pour gelée de gibier}

\index{gelee@gelée!fonds gibier@Fonds pour gelée de gibier}
\index{gibierl@gibier!gelee@gelée!fonds@Fonds pour gelée de ---}
\index{しゆれ@ジュレ!しひえのしゆれようのふおん@ジビエのジュレ用のフォン}
\index{しひえ@ジビエ!しゆれ@ジュレ!ふおん@---のジュレ用のフォン}

(仕上がり5 L分)

\begin{itemize}
\item
  主素材\ldots{}\ldots{}仔牛のすね肉1 kg、牛の脚肉2 kg、仔牛の骨750
  g、ジビエのガラやバモルソー\footnote{\protect\hyperlink{fonds-pour-gelee-ordinaire}{標準的なジュレ用のフォン}訳注参照。}1.75
  kg。これらはすべてオーブンで焼いて色付けておくこと。
\item
  ゼラチン質\ldots{}\ldots{}\protect\hyperlink{fonds-pour-gelee-de-volaille}{鶏のジュレ用のフォン}と同じ。
\item
  香味素材\ldots{}\ldots{}材料の種類は標準的なジュレ用のフォンと同じだが、セロリとタイムを
  \(\frac{1}{3}\)量多くすること。ジュニパーベリー\footnote{セイヨウネズの実。ジンの香りを特徴付けているもの。}7〜8粒を追加すること。
\item
  使用する液体\ldots{}\ldots{}水8 L。
\item
  加熱時間\ldots{}\ldots{}4時間。
\item
  作業手順\ldots{}\ldots{}ソース用の\protect\hyperlink{fonds-de-gibier}{ジビエのフォン}とまったく同じ。
\end{itemize}

\hypertarget{fonds-de-poisson-pour-gelee-ordinaire}{%
\subsubsection{標準的なジュレ用の魚のフォン}\label{fonds-de-poisson-pour-gelee-ordinaire}}

\frsub{Fonds de poisson pour gelée ordinaire}

\index{gelee@gelée!fonds poisson@Fonds de poisson pour gelée ordinaire}
\index{poisson@poisson!gelee@gelée!fonds@Fonds de --- pour gelée ordinaire}
\index{しゆれ@ジュレ!ひようしゆんてきなしゆれようのさかなのふおん@標準的なジュレ用の魚のフォン}
\index{さかな@魚!しゆれ@ジュレ!ふおん@標準的なのジュレ用の---のフォン}

(仕上がり5 L分)

\begin{itemize}
\item
  主素材\ldots{}\ldots{}グロンダン\footnote{ホウボウ科の魚。和名カナガシラ。}、ヴィーヴ\footnote{ハチミシカ科の海水魚の総称。}、メルラン\footnote{鱈の近縁種。}などの安い魚750
  g、舌びらめのアラと端肉750 g。
\item
  香味素材\ldots{}\ldots{}薄切りにした\footnote{émincer エマンセ。}玉ねぎ200
  g、パセリの根2本、フレッシュなマッシュルームの切りくず100 g。
\item
  使用する液体\ldots{}\ldots{}やや薄めで透き通った仕上がりの魚のフュメ6
  L。
\item
  加熱時間\ldots{}\ldots{}45分間。
\item
  作業手順\ldots{}\ldots{}\protect\hyperlink{fumet-de-poisson}{魚のフュメ}と同じ。
\end{itemize}

\hypertarget{fonds-pour-gelee-de-poisson-au-vin-rouge}{%
\subsubsection{赤ワインを用いた魚のジュレ用のフォン}\label{fonds-pour-gelee-de-poisson-au-vin-rouge}}

\frsub{Fonds pour gelée de poisson au vin rouge}

\index{gelee@gelée!fonds poisson rouge@Fonds pour gelée de poisson au vin rouge}
\index{poisson@poisson!gelee@gelée!fonds rouge@Fonds pour gelée de poisson au vin rouge}
\index{しゆれ@ジュレ!あかわいんをもちいたさかなのしゆれようのふおん@赤ワインを用いた魚のジュレ用のフォン}
\index{さかな@魚!しゆれ@ジュレ!ふおんあかわいん@赤ワインを用いた---のジュレ用のフォン}

このフォンは通常、鯉やトラウトなどの魚料理に用いられる。

このフォンに使用する液体は、良質なブルゴーニュ産赤ワインと\protect\hyperlink{fumet-de-poisson}{魚のフュメ}を同量ずつにする。魚のフュメは、ジュレが確実に固まるよう、ゼラチン質が多めのものを用いること。

風味付けは、魚に火を通すのに使った香味野菜によるもので充分だ。

\hypertarget{observation-sur-l-emplois-des-fonds-destines-aux-gelees}{%
\paragraph{ジュレ用のフォンについての注意}\label{observation-sur-l-emplois-des-fonds-destines-aux-gelees}}

\ldots{}\ldots{}ジュレ用のフォンは出来るだけ、使用する前日に仕込んでおくこと。いい具合に煮込んだら、浮き脂を取り除き\footnote{dégraisser
  デグレセ。}、漉してから陶製の容器に入れて冷ます。

冷めるとフォンは凝固する。取り除ききれなかったごくわずかな脂が表面に浮いてくるが、板状に固まるので容易に取り除くことが出来る。布あるいは漉し器でフォンを漉した際にすり抜けてしまった堆積物も自重で容器の底に沈むので、フォンを完全に澄ませることが出来る。
\end{recette}
\newpage

\hypertarget{clariication-des-gelees}{%
\subsection[ジュレのクラリフィエ]{\texorpdfstring{ジュレのクラリフィエ\footnote{clarification
  (クラリフィカスィオン)澄ませること、透明にさせること、の意の名詞だが、(1)本文にあるように、ただ単に「澄ませる」だけではなく、風味を補ったり強化し、色合いを調節する作業も兼ねていること、(2)現代日本の調理現場ではフランス語の動詞
  clarifier
  をカタカナにして「クラリフィエ」と呼ぶケースが多いことなどを考慮して、カタカナで動詞形のクラリフィエとした。なお、「クラリフェ」と呼ぶ現場もあるようだが、もとのフランス語がclarif\textbf{i}erとiの音があるのでこれは許容しがたい。}}{ジュレのクラリフィエ}}\label{clariication-des-gelees}}

\frsecb{Clarification des Gelées}

\index{gelee@geléé!clarification@Clarification des ---s}
\index{clarification@clarification!gelee@--- des gelées}
\index{しゆれ@ジュレ!くらりふいえ@---のクラリフィエ}
\index{くらりふいえ@クラリフィエ!しゆれ@ジュレの---}
\begin{recette}
\hypertarget{gelees-ordinaires}{%
\subsubsection[標準的なジュレ]{\texorpdfstring{標準的なジュレ\footnote{このgrasse
  \textless{} gras
  は「脂気のある、太った」の意ではなく、カトリックにおける「小斉」の食事を
  maigre
  と表現することと対になっているもの。すなわち「小斉ではない通常の」の意であることに注意。小斉については\protect\hyperlink{sauce-espagnole-maigre}{魚料理用ソース・エスパニョル}訳注および\protect\hyperlink{sauce-laguipiere}{ソース・ラギピエール}訳注参照。}}{標準的なジュレ}}\label{gelees-ordinaires}}

\frsub{Gelées grasses ordinaires}

\index{gelee@geléé!clarification@Clarification!grasses ordinaire@---s grasses ordinaires}
\index{clarification@clarification!gelee@gelée!grasses ordinaire@---s grasses ordinaires}
\index{しゆれ@ジュレ!くらりふいえ@---のクラリフィエ!ひようしゆんてきな@標準的な---}
\index{くらりふいえ@クラリフィエ!しゆれ@ジュレの---!ひようしゆんてきな@標準的な---}

(仕上がり5 L分)

\begin{enumerate}
\def\labelenumi{\arabic{enumi}.}
\item
  まずフォンの濃度を確認する。必要に応じて追加すべきゼラチンの量を調整する。
\item
  ジュレ用のフォンは充分に浮き脂を取り除き\footnote{dégraisser
    デグレセ。}、沈殿物も取り除い\footnote{décanter デカンテ。}てあること。
\item
  厚手で適切な大きさの片手鍋\footnote{casserole カスロール。}に、細挽き\footnote{ミートチョッパーやフードプロセッサが一般化する以前はアショワールhachoirという、両側に柄の付いた刃が湾曲した専用の包丁で細かく刻んでいた。}にした脂身のない\footnote{ここで原文はmaigreを用いているが、これはもちろん「脂気のない」の意。}赤身の牛肉
  500 gとセルフイユとエストラゴン計10 g、卵白3個分を入れる。
\item
  冷たい、あるいは生温い状態のジュレ用のフォンを挽肉の上から入れ、泡立て器かヘラで混ぜる。\\
  ゆっくり混ぜながら、強過ぎない程度の火加減で沸騰させる。卵白に含まれるアルブミンの分子が澄ませる作用を持っているので\footnote{やや大雑把な説明になるが、液体中に浮遊している不純物を抱き込むかたちで卵白が熱変性により凝固する、その結果として液体を「澄ませる」ことになる。ただし、これだけだと液体の味そのものや風味が薄くなってしまうために、それを補うあるいは強化する意味で挽肉や香草、香り付けの酒類を加える、ということ。}、混ぜることで卵白がまんべんなく広がるようにするわけだ。\\
  15分程、微沸騰の状態を保ち、目の詰まった布で漉す。
\end{enumerate}

\hypertarget{nota-gelees-grasses-ordinaires}{%
\subparagraph{【原注】}\label{nota-gelees-grasses-ordinaires}}

ジュレに酒類を添加するのは、ほぼ冷めた状態になってからにするのがいい。クラリフィエの作業中に酒類を加えるのは、沸騰しているために味が悪くなってしまうので、致命的な誤りでさえある。

そうではなく、ほぼ冷めた状態のジュレに酒類を添加すれば、その香気はそのまま保たれることになる。

作業の最後に酒類をジュレに添加すればジュレを薄めてしまう結果になるわけだからそれを考慮して、添加する酒類の量によっては、あらかじめジュレを充分に固めに作っておくのがいい。そうすれば、ジュレが固まるのに充分なゼラチンの濃度を保てるわけだ。

マデイラ酒、マルサラ酒、シェリー酒を加える場合の分量はジュレ1 Lあたり1
dLとすること。

ライン産のワインやシャンパーニュ、銘醸白ワインを加える場合は、ジュレ1
Lあたり2
dLとすること。加える酒類がどんなものであっても、文句ない程に良質のものを用いるべきだ。質の悪い酒類を加えてジュレの仕上がりを台無しにしてしまうくらいなら、加えないほうがまだましと言える。

\hypertarget{gelee-de-volaille}{%
\subsubsection{鶏のジュレ}\label{gelee-de-volaille}}

\frsub{Gelée de volaille}

\index{gelee@geléé!clarification@Clarification!volaille@--- de volaille}
\index{clarification@clarification!gelee@gelée!volaille@--- de volaille}
\index{しゆれ@ジュレ!くらりふいえ@---のクラリフィエ!とりの@鶏の---}
\index{くらりふいえ@クラリフィエ!しゆれ@ジュレの---!とりの@鶏の---}
\index{しゆれ@ジュレ!くらりふいえ@---のクラリフィエ!うおらいゆの@ヴォライユの---}
\index{くらりふいえ@クラリフィエ!しゆれ@ジュレの---!うおらいゆの@ヴォライユの---}

鶏のジュレのクラリフィエは標準的なジュレの場合とまったく同じに行なう。香味素材(セルフイユとエストラゴン)、澄ませるための材料(卵白)も同様にする。

ただし、味の補強に用いる肉については変更すること。すなわち牛の赤身肉を半量にして、残り半量は鶏の首肉にする。つまり、牛肉250
gと鶏の首肉250 g の挽肉を用いる。

\hypertarget{nota-gelee-de-volaile}{%
\subparagraph{【原注】}\label{nota-gelee-de-volaile}}

鶏のローストのガラを粗く砕いてエチューヴ\footnote{食品の乾燥などに主に用いられる低温のオーブンの一種。}でよく乾燥させて脂気を抜いたものを、このクラリフィエの際に加えると、素晴しい結果が得られる。

\hypertarget{gelee-de-gibier}{%
\subsubsection{ジビエのジュレ}\label{gelee-de-gibier}}

\frsub{Gelée de gibier}

\index{gelee@geléé!clarification@Clarification!gibier@--- de gibier}
\index{clarification@clarification!gelee@gelée!gibier@--- de gibier}
\index{しゆれ@ジュレ!くらりふいえ@---のクラリフィエ!しひえ@ジビエの---}
\index{くらりふいえ@クラリフィエ!しゆれ@ジュレの---!しひえ@ジビエの---}

クラリフィエの作業のやり方はまったく同様。ただし、このジュレを作る際には、いくつか留意すべきポイントがある。

標準的なジビエのジュレ、つまり特有の風味を持たせないものの場合は、味の補強には牛の挽肉250
gとジビエの赤身の挽肉250 gを用いること。

ジュレに独特の香りを持たせる必要がある場合には、必ず、肉それ自体に香気のあるジビエの肉、すなわち、ペルドロー、雉、ジェリノット\footnote{gélinotte
  雷鳥の一種。}などをクラリフィエの際に用いること。

どんなジビエのジュレでも仕上げに、ジュレ1
Lあたり大さじ2杯の上等なコニャックを加える。ただし、コニャックは絶対に良質のものでなければいけない。平凡なコニャックしか使えないのなら、これは省いたほうがいい。

この香り付けをしなくても、ジュレは不完全なものとはいえ、一応使えるものになる。いっぽうで、ありきたりのコニャックで香り付けすると、美味しくは仕上がらない。

\hypertarget{gelee-de-poisson-blanche}{%
\subsubsection{魚の白いジュレ}\label{gelee-de-poisson-blanche}}

\frsub{Gelée de poisson blanche}

\index{gelee@geléé!clarification@Clarification!poisson blanche@--- de poisson blanche}
\index{clarification@clarification!gelee@gelée!poisson blanche@--- de poisson blanche}
\index{しゆれ@ジュレ!くらりふいえ@---のクラリフィエ!さかなのしろい@魚の白い---}
\index{くらりふいえ@クラリフィエ!しゆれ@ジュレの---!さかなのしろい@魚の白い---}

魚のジュレのクラリフィエは以下のとおり\footnote{(1)または(2)の方法をとる、と解釈していいだろう。}。

\begin{enumerate}
\def\labelenumi{\arabic{enumi}.}
\item
  卵白を使う場合、ジュレ5
  Lあたり卵白3個分に、クラリフィエによって薄まってしまうのを補うためにメルランの身を細かく刻んだもの250
  gを加える。
\item
  もし可能なら新鮮なキャビア、なければ圧縮キャビア\footnote{\protect\hyperlink{beurre-de-caviar}{キャビアバター}訳注参照。}をジュレ1
  Lあたり50
  g用いる。方法は魚のコンソメのクラリフィエで説明している\footnote{概要は、キャビアをピュレ状にすり潰し、冷たい魚のコンソメでのばして加える。火にかけて絶えず混ぜながら沸かし、微沸騰の状態を20分保った後、布で漉す、という方法。}
  (ポタージュの章を参照)。
\end{enumerate}

魚のジュレの香り付けには、辛口のシャンパーニュもしくはブルゴーニュの銘醸白ワインを用いるといいが、\protect\hyperlink{gelees-ordinaires}{標準的なジュレ}の注において説明した酒類を加える場合の注意事項を勘案すること。

\hypertarget{nota-gelee-de-poisson-blanche}{%
\subparagraph{【原注】}\label{nota-gelee-de-poisson-blanche}}

場合によっては、ジュレ1
Lあたり4尾のエクルヴィスを用いることで、魚のジュレに独特の風味付けをすることも出来る。エクルヴィスをソテーしてビスクを作る要領で煮てから、鉢に入れて細かくすり潰し、最後に漉す作業の10分前に魚のフォンに加える。

\hypertarget{gelee-de-poisson-au-vin-rouge}{%
\subsubsection{赤ワインを用いた魚のジュレ}\label{gelee-de-poisson-au-vin-rouge}}

\frsub{Gelée de poisson au vin rouge}

\index{gelee@geléé!clarification@Clarification!poisson vin rouge@--- de poisson au vin rouge}
\index{clarification@clarification!gelee@gelée!poisson vin rouge@--- de poisson au vin rouge}
\index{しゆれ@ジュレ!くらりふいえ@---のクラリフィエ!あかわいんをもちいたさかなの@赤ワインを用いた魚の---}
\index{くらりふいえ@クラリフィエ!しゆれ@ジュレの---!あかわいんをもちいたさかなの@赤ワインを用いた魚の---}

このジュレのクラリフィエには、ジュレ5 Lあたり卵白4個分を用いる。

赤ワインで魚を煮ている途中や、ジュレのクラリフィエ作業の際に、タンニン由来の色素にすぐ変化してしまうことがしばしば、というかほぼ必ず起こる。ワインが分解してしまうのは魚のフュメに含まれているゼラチン質と接触して反応するためのようだ。こんにちに至るまで、これを避ける方法は見つかっていない。

そのため、色合いの不足を補うには人工色素(液体のカルミン\footnote{コチニール色素。\protect\hyperlink{beurres-composes}{合わせバター}本文および訳注参照。}か別の植物由来の色素)を加える必要がある。ただし、使用量にはごく細心の注意を払い、ジュレがやや深みをおびたバラ色を越えてしまわないようにすること。
\end{recette}

%%% chapitre ii. garnitures
%%% II. garnitures
\hypertarget{garniture}{%
\chapter{II. ガルニチュール Garnitures}\label{garniture}}

\index{garnitures@garnitures} \index{かるにちゆーる@ガルニチュール}

\vspace*{1.7\zw}

料理においてガルニチュール\footnote{garniture
  一般的には「付け合せ」と訳すが、本書におけるガルニチュー
  ルはたんなる料理の「付け合わせ」にとどまらず、こんにちではそれ自体
  がひとつの料理として成立し得るものも多い。そのため、あえて片仮名で
  ガルニチュールとした。なお、「付け合わせ」の意味で「ガルニ」または
  「ガロニ」などというスラングを用いる調理現場もある。}は重要なものだから、料理人は決してガルニ
チュールの役割を軽視してはいけない。ガルニチュールの構成をどうするかは、
添える料理の主素材との関係性で決まる。気まぐれ的なものや不自然なもの
は絶対にいけない。

ガルニチュールの構成要素は、場合によりけりだが、もっぱらどんな種類の料
理に添えるかで決まる。具体的には、野菜料理やパスタ、ファルスでさ
まざまな形状に作ったクネル\footnote{quenelle
  仔牛肉や鶏肉、豚肉などと獣脂をすり潰して、しばしば「つ
  なぎ」として後述のパナードを加えて練り、スプーンなどを用いて整形し、
  沸騰しない程度の温度で茹でる{[}ポシェ{]}またはオーブンで焼いたもの。
  スプーンを2つ使ってラグビーボールに似た形状にしたものが代表的だが、
  他にもいろいろな形状、大きさにする。}、あるいは雄鶏のとさかとロニョン\footnote{\protect\hyperlink{garniture-financiere}{ガルニチュール・フィナンシエール}やその
  バリエーションともいえる\protect\hyperlink{garniture-godard}{ガルニチュール・ゴダー
  ル}で必須の素材。ロニョンrognonは通常なら腎臓を
  意味するが、この場合のロニョンは rognon blanc ロニョンブラン(白い
  ロニョン)とも呼ばれるもので、雄鶏の精巣のこと。}、さ
まざまな種類の茸、オリーブとトリュフ、イカや貝および甲殻類、場合によっ
ては卵、小魚、牛や羊の副生物\footnote{正肉以外の部分。例えば内臓や骨髄など。Ris
  de vea(リドヴォー)仔牛胸腺肉などはこれに含まれる。}など。

その昔、ガルニチュールというのは、マトロットやコンポート、ブルゴーニュ
風料理などのように風味付けのために用いた素材がそのまま添えられたもので
あった。

ガルニチュールにする野菜は、どういう仕立ての皿にするかで役割が決まり、
それに合うように切って形状を整え、調理する。ただし、野菜の調理法は「野
菜料理」として調理する場合と同じだ。

パスタやイカ、貝類、甲殻類についても同様のことが言える。

この章では、それぞれのガルニチュールを構成する素材とその分量を示すに留
めるので、各素材の調理法ついてはその素材に対応する章を参照すること。

\hypertarget{ux30d5ux30a1ux30ebux30b9-5}{%
\section[ファルス ]{\texorpdfstring{ファルス \footnote{本来は「詰め物」の意で、鶏のローストの内臓を抜いた空洞部分に詰めたり、ガランティーヌやパテアンクルートの内部の詰め物などの用途に用いられる。この意味はこんにちでも変化がないが、本文にあるように、クネルにしてガルニチュールの一部にするなど、用途は多岐にわたる。本書ではファルスとして用いられるもののうち、肉および魚肉をベースにしたものをこの節にまとめて分類、説明している。したがって、ここでファルスとして挙げられていないファルスも料理によっては多い(例えば丸鶏の空洞部分に米などを詰めるのもファルス)ことに注意。}}{ファルス }}\label{ux30d5ux30a1ux30ebux30b9-5}}

\hypertarget{serie-des-farces-diverses}{%
\subsection{Série des farces diverses}\label{serie-des-farces-diverses}}

\index{garnitures@garnitures!farces@farces} \index{farce@farce}
\index{かるにちゆーる@ガルニチュール!ふあるす@ファルス}
\index{ふあるす@ファルス}

ガルニチュールの多くは、その構成要素にファルスあるいはファルスで作った
「クネル」が含まれている。ファルスはまた、多くの大きな仕立ての料理にも
使われる。ここではまずファルスの材料および作り方を示し、使い途について
は後で述べることにする。

ファルスは大きく5種に分類される。

\begin{enumerate}
\def\labelenumi{\arabic{enumi}.}
\item
  仔牛肉と脂で作るもの。すなわち古典料理における\textbf{ゴディヴォ}。
\item
  基本となる材料はさまざまだが、「つなぎ」に主としてパナードを使うもの。
\item
  近代的な手法で、生クリームを用いてふんわり泡立てたファルス。ムース、ムスリーヌに用いる。
\item
  レバーをベースとした「ファルス・\textbf{グラタン}」。種類はいろいろだが作り方は常に同じ。
\item
  \ruby{主}{おも}に\protect\hyperlink{}{ガランティーヌ}、\protect\hyperlink{}{パテアンクルート}、\protect\hyperlink{}{テリーヌ}などの冷製料理に用いるシンプルなファルス。
\end{enumerate}

\hypertarget{ux30d5ux30a1ux30ebux30b9ux7528ux306eux30d1ux30caux30fcux30c9ux306bux3064ux3044ux30666}{%
\subsection[ファルス用のパナードについて]{\texorpdfstring{ファルス用のパナードについて\footnote{パナードは本来、パンと水、バターを弱火で時間をかけて煮た粥のようなものを意味した。本書ではその意味を拡大して肉や魚肉をベースとしたファルスを加熱する際に崩れないようにする「つなぎ」として、この語を用いている。そのため、必ずしもパンを材料としていないものが含まれている。}}{ファルス用のパナードについて}}\label{ux30d5ux30a1ux30ebux30b9ux7528ux306eux30d1ux30caux30fcux30c9ux306bux3064ux3044ux30666}}

\vspace*{-1.7\zw}

\hypertarget{les-panades-pour-farces}{%
\subsection{Les Panades pour Farces}\label{les-panades-pour-farces}}

\index{garnitures@garnitures!farces@farces}
\index{farce@farce!panade@les panades pour farces}
\index{かるにちゆーる@ガルニチュール!ふあるす@ファルス!はなーと@パナード}
\index{ふあるす@ファルス!はなーと@---用パナード}

ファルスに用いるパナードにはいくつもの種類がある。ファルスの種類や、そ
のファルスを添える料理の性質によって使い分けることとなる。

原則として、パナードの分量は、ファルスのベースとする素材が何であれ、そ
の半量を越えないようにすること。

卵とバターを用いるパナードの場合はレシピの分量どおりに作らなければなら
ないから、それを合わせて作るファルスの全体量のほうを調節してやること。

パナードE以外のパナードは使用する際には必ず完全に冷めた状態になってい
ること。パナードが出来上がったら、バターを塗った平皿か天板に流し広げ、
早く冷めるようにする。このとき、バターを塗った紙で蓋をするか、表面にバ
ターのかけらをいくつか置いてやり、パナードが直接空気に触れないようにし
てやること。

以下のパナードのレシピは仕上がり重量が正味500
gになるように調整してある。

したがって、必要な量のパナードを作るのに材料を増やしたり減らしたりする
のも難しくはないだろう\footnote{原文では、Rien de plus simple, donc, que
  \ldots{}
  となっており、直訳すると「これ以上に簡単なことはない」と言いきっているが、都度計算しなければならないことに変わりはないので、多少ニュアンスを柔らげて訳した。}。

\hypertarget{ux30d1ux30caux30fcux30c9}{%
\subsection{パナード}\label{ux30d1ux30caux30fcux30c9}}

\vspace*{-1.7\zw}

\hypertarget{panades}{%
\subsection{Panades}\label{panades}}

\index{panade} \index{garniture!panade} \index{garniture!farce!panade}
\index{かるにちゆーる@ガルニチュール!ふあるす@ファルス!はなーと@パナード}
\index{はなーと@パナード}
\begin{recette}
\hypertarget{a.-ux30d1ux30f3ux306eux30d1ux30caux30fcux30c9}{%
\subsubsection{A.
パンのパナード}\label{a.-ux30d1ux30f3ux306eux30d1ux30caux30fcux30c9}}

\hypertarget{panade-a}{%
\paragraph{Panade au pain}\label{panade-a}}

\index{garnitures@garnitures!farces@farces!panade a@panade A}
\index{farce@farce!panade@les panades pour farces!panade a@panade A}
\index{panade!a pain@A. --- au pain}
\index{かるにちゆーる@ガルニチュール!ふあるす@ファルス!はなーとa@パナードA. パンの---}
\index{ふあるす@ファルス!はなーと@---用パナード!a@A. パンのパナード}
\index{はなーと@パナード!a@A. パンの---}

\ldots{}\ldots{}\textbf{魚を素材にした固めのファルス用}

\begin{itemize}
\item
  \textbf{材料}\ldots{}\ldots{}沸かした牛乳3
  dl、固くなった白パン\footnote{ここでいわゆるバゲットのようなパンの外側を削り落した白い部分、あ
    るいは食パンの「耳」を切り落した白い部分を使う、ということ。なお、
    パンは使う小麦粉の精白度や種類によって、pain complet (パンコンプ
    レ)全粒粉パン、pain de sègle(パンドセーグル)ライ麦パン、一般的
    な小麦粉と食塩、塩、パン種だけで作るバゲットなどの pain と、バター
    や砂糖を加えて作るヴィエノワズリ(クロワッサンやパンオショコラ、ブ
    リオシュなど)に分けられる。イギリスやアメリカのいわゆる食パン(フ
    ランス語 pain de mie パンドミ)は小麦粉、バター、塩、イースト菌、
    牛乳などで作られている。また、現代フランスでバゲットなどのパンに用
    いられている小麦粉の精白度は、T-55と呼ばれる灰分(小麦粉を燃やした
    際に残る炭水化物およびタンパク質以外の要素)0.5〜0.6%のものが主流
    であり、いわゆる食パンpain de mie(パンドミ)やヴィエノワズリには
    T-45(灰分0.5%以下)が多く用いられている。このほかT-65(灰分0.62〜
    0.75%)およびT-80(灰分0.75〜0.9%)、T-110(灰分1.0〜1.2%)、
    T-150(灰分1.4%前後、いわゆる全粒粉)のように種類がある。このうち
    T-45およびT-55はfarine blanche(ファリーヌブロンシュ)と呼ばれ、
    T-150はfarine complète(ファリーヌコンプレット)と通称されている。
    灰分が高くなればそれだけ不純物が多いわけだから、粉は薄い茶色あるい
    はグレーがかった色合いになり、パンを焼く場合などはグルテン形成が難
    しくなりやすい。ただし、香りゆたかなパンを実現しやすいという面もあ
    る。結果として、例えば全粒粉パンは香りはいいが固い仕上がりになる。か
    つては精白度の低い(すなわち灰分の多い)粉ほど重量あたりの価格が安
    く、パンの価格もそれに比例していた。また、本書では基本的に小麦粉を
    使う場合にその精白度についての指示はないが、概ねT-55またはT-45相当
    のもの考えていいだろう。なお、日本に輸入されている小麦は北米産のも
    のがほとんどで、硬質小麦を粉にしたものが「強力粉」、軟質小麦の場合
    は「薄力粉」と呼ばれ、精白度合いによる分類は通常なされていないが、
    製品としては概ねT-45相当あるいはそれ以上の精白度のものが多い。}の身250
  g、塩5 g。
\item
  \textbf{作業手順}\ldots{}\ldots{}パンの身を牛乳に浸して完全にもどす。強火にかけて、ペー
  スト状になったパンがヘラから簡単に取れるくらいまで水気をとばす。バター
  を塗った平皿か天板に広げ、冷ます。
\end{itemize}

\maeaki

\hypertarget{b.-ux5c0fux9ea6ux7c89ux306eux30d1ux30caux30fcux30c9}{%
\subsubsection{B.
小麦粉のパナード}\label{b.-ux5c0fux9ea6ux7c89ux306eux30d1ux30caux30fcux30c9}}

\hypertarget{panade-b}{%
\paragraph{Panade à la farine}\label{panade-b}}

\index{garnitures@garnitures!farces@farces!panade b@panade B}
\index{farce@farce!panade@les panades pour farces!panade b@panade B}
\index{panade!b farine@B. --- à la farine}
\index{かるにちゆーる@ガルニチュール!ふあるす@ファルス!はなーとb@パナードB. 小麦粉の---}
\index{ふあるす@ファルス!はなーと@---用パナード!b@パナードB. 小麦粉の---}
\index{はなーと@パナード!b@B. 小麦粉の---}

\ldots{}\ldots{}\textbf{肉、魚などあらゆるファルスに用いられる}

\begin{itemize}
\item
  \textbf{材料}\ldots{}\ldots{}水3 dl、塩2 g、バター50
  g、篩にかけた小麦粉150 g。
\item
  \textbf{作業手順}\ldots{}\ldots{}片手鍋に水、塩、バターを入れて火にかけ、沸騰させる。
  火から外して小麦粉を加えて混ぜる。再度火にかけて、\protect\hyperlink{}{シュー生地}を
  作る要領で余計な水分をとばす。上記パナードAと同様にして冷ます。
\end{itemize}

\maeaki

\hypertarget{c.-ux30d1ux30caux30fcux30c9ux30d5ux30e9ux30f3ux30b8ux30d1ux30fcux30cc12}{%
\subsubsection[C. パナード・フランジパーヌ]{\texorpdfstring{C.
パナード・フランジパーヌ\footnote{フランジパーヌとは製菓で用いられる、小麦粉、砂糖、卵を混ぜて牛
  乳とバニラを加えて煮、砕いたマカロンmacaronを加えたクリーム。なお、
  これに用いられるマカロンは、現代日本でよく知られているタイプとは違
  い、すり潰したアーモンドと卵白、砂糖を混ぜた生地を紙の上にクルミ大
  に絞り出してオーブンで焼いただけもの。macaron craquelé(マカロンク
  ラクレ)はこのタイプの代表的なもので、焼く際に膨らんで割れ目が出来
  ることからクラクレの名称が付けられた。ところで、日本にマカロンが伝
  わった時期は判然としないが、このタイプのものが太平洋戦争前には、アー
  モンドを落花生に代え、「まころん」の名称でいくつかの製菓会社で製造
  されるようになり、現在も生産されている。}}{C. パナード・フランジパーヌ}}\label{c.-ux30d1ux30caux30fcux30c9ux30d5ux30e9ux30f3ux30b8ux30d1ux30fcux30cc12}}

\hypertarget{panade-c}{%
\paragraph{Panade à la Frangipane}\label{panade-c}}

\index{garnitures@garnitures!farces@farces!panade c@panade C}
\index{farce@farce!panade@les panades pour farces!panade c@panade C}
\index{panade!c frangipane@C. --- à la Frangipane}
\index{かるにちゆーる@ガルニチュール!ふあるす@ファルス!はなーとc@パナードC}
\index{ふあるす@ファルス!はなーと@---用パナード!はなーとC@パナードC}
\index{はなーと@パナード!c@C. ---・フランジパーヌ}

\ldots{}\ldots{}\textbf{鶏のファルス、魚のファルス用}

\begin{itemize}
\item
  \textbf{材料}\ldots{}\ldots{}小麦粉125 g、卵黄4個、溶かしバター90
  g、塩2 g、こしょう1 g、おろしたナツメグの粉ごく少量、牛乳2\undemi{}
  dl。
\item
  \textbf{作業手順}\ldots{}\ldots{}片手鍋に小麦粉と卵黄を入れてよく練る。溶かしバター、
  塩、こしょう、ナツメグを加える。沸かした牛乳で少しずつ溶きのばしてい
  く。
\end{itemize}

\protect\hyperlink{}{標準的なフランジパーヌ}と同様に、火にかけて5〜6分間、泡立て器で混
ぜながら煮る。ちょうどいい漉さになったら、バットに移して\footnote{débarasser
  (デバラセ)バットなどに移す、片付ける、の意。とりわけ前者の意味に注意。}冷ます。

\maeaki

\hypertarget{d.-ux7c73ux306eux30d1ux30caux30fcux30c9}{%
\subsubsection{D.
米のパナード}\label{d.-ux7c73ux306eux30d1ux30caux30fcux30c9}}

\hypertarget{panade-d}{%
\paragraph{Panade au Riz}\label{panade-d}}

\index{garnitures@garnitures!farces@farces!panade d@panade D}
\index{farce@farce!panade@les panades pour farces!panade d@panade D}
\index{panade!d riz@D. --- au Riz}
\index{かるにちゆーる@ガルニチュール!ふあるす@ファルス!はなーとd@パナードD. 米の---}
\index{ふあるす@ファルス!はなーと@---用パナード!はなーとd@D. 米のパナード}
\index{はなーと@パナード!d@D. 米の---}

\ldots{}\ldots{}\textbf{いろいろなファルスに用いられる}

\begin{itemize}
\item
  \textbf{材料}\ldots{}\ldots{}米200 gすなわち2
  dlあるいは大さじ8杯。\protect\hyperlink{}{白いコンソメ}6 dl、バター20
  g。
\item
  \textbf{作業手順}\ldots{}\ldots{}米を入れた鍋にコンソメを注ぎ、バターを加える。火にかけて沸騰させたら、オーブンに入れて40〜45分間加熱する。この間、米に触れないようにすること。
\end{itemize}

オーブンから出したら、米粒がよく潰れるようにヘラでしっかりと混ぜる。その後、冷ます。

\maeaki

\hypertarget{e.-ux3058ux3083ux304cux3044ux3082ux306eux30d1ux30caux30fcux30c9}{%
\subsubsection{E.
じゃがいものパナード}\label{e.-ux3058ux3083ux304cux3044ux3082ux306eux30d1ux30caux30fcux30c9}}

\hypertarget{panade-e}{%
\paragraph{Panade à la pomme de terre}\label{panade-e}}

\index{garnitures@garnitures!farces@farces!panade e@panade E}
\index{farce@farce!panade@les panades pour farces!panade e@panade E}
\index{panade!e riz@E. --- à la pomme de terre}
\index{かるにちゆーる@ガルニチュール!ふあるす@ファルス!はなーとe@パナードE}
\index{ふあるす@ファルス!はなーと@---用パナード!はなーとe@パナードE}
\index{はなーと@パナード!e@E. じゃがいもの---}

\ldots{}\ldots{}\textbf{仔牛および他の白身肉の、詰め物\footnote{fourrré
  (フレ)詰め物をした。farci (ファルシ)も同様に「詰め
  物をした」の意だが、後者はより一般的で、前者はオムレツやクレープに
  中身を詰めて「包む」のが本来の意味。すなわち、このパナードを加えた
  ファルスで、何らかの素材を「包む」と解釈してもいい。とりわけこの
  fourréには日本料理の用語「射込む」をあてる場合もある。}をする大きなクネルに用いられる}

\begin{itemize}
\item
  \textbf{材料}\ldots{}\ldots{}茹でて皮を剥いたばかりの中位のサイズのじゃがいも2個、牛
  乳3 dl、塩 2g、白こしょう\undemi{} g、ナツメグ少々、バター20 g。
\item
  \textbf{作業手順}\ldots{}\ldots{}牛乳を2.5
  dlになるまで煮詰める\footnote{原文は réduire le lait d'un sixième
    直訳すると「牛乳を
    \unsixieme{}量だけ煮詰める」すなわち「\cinqsixiemes{}量まで煮詰め
    る」のだが、かえって分かりにくいだろうから、ここでは具体的な数字に
    直して訳した。分量を代えて作る場合には85%まで煮詰めるくらいと考え
    てもいいだろう。そもそも、じゃがいもの重さが曖昧なのだから、あまり
    細かい数字にこだわらず臨機応変に考えること。}。バター、調味料、
  薄く輪切りにしたじゃがいもを加え、15分間程加熱する。
\end{itemize}

このパナードはまだ少し\ruby{微温}{ぬる}いくらいで使用すること。完全に
冷めてからではいけない。完全に冷めてから練ると粘りが出てしまうからだ。
\end{recette}
\hypertarget{ux30d5ux30a1ux30ebux30b9}{%
\subsection{ファルス}\label{ux30d5ux30a1ux30ebux30b9}}

\vspace*{-1.7\zw}

\hypertarget{farces}{%
\subsection{Farces}\label{farces}}

\index{farce} \index{garniture!farce}
\index{かるにちゆーる@ガルニチュール!ふあるす@ファルス}
\index{ふあるす@ファルス}

ベースとなる素材が\textbf{仔牛}、\textbf{鶏}、\textbf{ジビエ}あるいは\textbf{甲殻類}であっても、分量と作
業手順はどんなファルスでも同じだ。そのベースにする素材を代えればいいの
だから、ここでは各種ファルスの典型的なレシピを示せば充分だろう。料理で
用いられるファルスひとつひとつを説明するのに一章をあてる必要はないと思
われる。
\begin{recette}
\hypertarget{a.-ux30d1ux30caux30fcux30c9ux3068ux30d0ux30bfux30fcux3092ux7528ux3044ux308bux30d5ux30a1ux30ebux30b9}{%
\subsubsection{A.
パナードとバターを用いるファルス}\label{a.-ux30d1ux30caux30fcux30c9ux3068ux30d0ux30bfux30fcux3092ux7528ux3044ux308bux30d5ux30a1ux30ebux30b9}}

\hypertarget{farce-a}{%
\paragraph{Farce à la Panade et au beurre}\label{farce-a}}

\index{farce!a@A. --- à la Panade et au beurre}
\index{garniture!farce!a@A. Farce à la Panade et au beurre}
\index{かるにちゆーる@ガルニチュール!ふあるす@ファルス!a@A. パナードとバターを用いるファルス}
\index{ふあるす@ファルス!a@A. パナードとバターを用いる---}

(標準的なクネル、肉料理\footnote{原文 entrée
  (アントレ)、現代では「前菜」の意味で用いられるが、
  本書では概ね10人前を一皿に盛ったものを指し、現代では立派にメインの
  料理として通用するものが多くある。}の縁飾り etc.)

\begin{itemize}
\item
  \textbf{材料}\ldots{}\ldots{}ていねいに筋取りをした肉1
  kg、\protect\hyperlink{panade-b}{パナードB} 500 g、塩12 g、こしょう2
  g、全卵4個、卵黄8個。
\item
  \textbf{作業手順}\ldots{}\ldots{}肉をさいの目に切って鉢に入れ、調味料を加えてすり潰す。
  いったん肉を取り出して、パナードをよくすり潰しながらバターを加える。
  肉を戻し入れ、すりこ木\footnote{pilon
    (ピロン)形状は日本のすりこ木をやや異なるのが多い。ここ
    では大理石の鉢もしくは陶製のボウルを用いて作業していることに注意。
    現代ではフードプロセッサを用いるところだろうが、かつては人力で、力
    を込めて丁寧に作業していたということは頭に留めておきたい。}で力強く練って全体をまとめる。
\end{itemize}

次に全卵と卵黄を加えて混ぜ合わせる。これは2回に分けても1回でやってもい
い。裏漉しして陶製の容器に入れる。さらに泡立て器で滑かになるまで混ぜる。

\hypertarget{ux539fux6ce8}{%
\subparagraph{【原注】}\label{ux539fux6ce8}}

どんな種類のファルスを作る場合でも、必ず少量を沸騰しない程度の温度で茹
でて\footnote{pocher (ポシェ)。}テストしてから、クネルの整形に取りかかること。

\maeaki

\hypertarget{b.-ux30d1ux30caux30fcux30c9ux3068ux751fux30afux30eaux30fcux30e0ux3092ux7528ux3044ux308bux30d5ux30a1ux30ebux30b9}{%
\subsubsection{B.
パナードと生クリームを用いるファルス}\label{b.-ux30d1ux30caux30fcux30c9ux3068ux751fux30afux30eaux30fcux30e0ux3092ux7528ux3044ux308bux30d5ux30a1ux30ebux30b9}}

\hypertarget{farce-b}{%
\paragraph{Farce à la Panade et à la Crème}\label{farce-b}}

\index{farce!b@B. --- à la Panade et à la crème}
\index{garniture!farce!b@B. Farce à la Panade et à la crème}
\index{かるにちゆーる@ガルニチュール!ふあるす@ファルス!b@B. パナードと生クリームを用いるファルス}
\index{ふあるす@ファルス!b@B. パナードと生クリームを用いる---}

(滑らかな仕上がりのクネル用)

\begin{itemize}
\item
  \textbf{材料}\ldots{}\ldots{}筋取りをした肉1
  kg、\protect\hyperlink{panade-c}{パナードC} 400 g、卵白5 個分、塩15
  g、白こしょう2 g、ナツメグ1 g、クレーム・ドゥーブル \footnote{乳酸発酵させた濃い生クリーム。フランスの生クリームについては\protect\hyperlink{sauce-supreme}{ソー
    ス・シュプレーム}訳注参照。}1\undemi{} L。
\item
  \textbf{作業手順}\ldots{}\ldots{}どんな肉を使う場合でも、卵白を少しずつ加えながらしっ
  かりとすり潰すこと。
\end{itemize}

パナードを加え、すりこ木でしっかり練り、二つの素材がよくよく混ざ
り合うようにする。

目の細かい網で裏漉しし、鍋にファルスを入れる。ヘラで滑らかになるよう混
ぜ、鍋を氷の上に置いて一時間ほど休ませる。

生クリームの\untiers{}量を少しずつ加えながら、のばしていく。最終的に残
りの\deuxtiers{}の生クリームも加えるが、これは先に泡立て器で軽く立てておくこと。

生クリームを全部加えた時点で、ファルスは真っ白で滑らかでしかも、ふんわりとし
た仕上がりにならなくてはいけない。

\hypertarget{ux539fux6ce8-1}{%
\subparagraph{【原注】}\label{ux539fux6ce8-1}}

手に入った生クリームが必ずしも最上級のものでない場合には、パナードC を
用いて\protect\hyperlink{farce-a}{バターを用いたファルス}を作った方がまだいい。

\maeaki

\hypertarget{c.-ux751fux30afux30eaux30fcux30e0ux3092ux7528ux3044ux308bux6ed1ux3089ux304bux306aux30d5ux30a1ux30ebux30b9-ux30d5ux30a1ux30ebux30b9ux30e0ux30b9ux30eaux30fcux30cc}{%
\subsubsection{C. 生クリームを用いる滑らかなファルス /
ファルス・ムスリーヌ}\label{c.-ux751fux30afux30eaux30fcux30e0ux3092ux7528ux3044ux308bux6ed1ux3089ux304bux306aux30d5ux30a1ux30ebux30b9-ux30d5ux30a1ux30ebux30b9ux30e0ux30b9ux30eaux30fcux30cc}}

\hypertarget{farce-c}{%
\paragraph{Farce à la Crème, ou Mousseline}\label{farce-c}}

\index{farce!c@B. --- fine à la crème, ou Mousseline}
\index{garniture!farce!c@C. Farce fine à la crème, ou Mousseline}
\index{mousseline!farce mousseline}
\index{かるにちゆーる@ガルニチュール!ふあるす@ファルス!c@C. 生クリームを用いる滑らかなファルス / ファルス・ムスリーヌ}
\index{ふあるす@ファルス!c@C. 生クリームを用いる滑らかな--- / ---・ムスリーヌ}
\index{むすりーぬ@ムスリーヌ!ふあるす@ファルス・---}

(ムース、ムスリーヌ、ポタージュ用クネルなど)

\begin{itemize}
\item
  \textbf{材料}\ldots{}\ldots{}丁寧に掃除をして筋取りをした肉1
  kg、卵白4個分、クレーム・ エペス\footnote{crème épaisse fraîche
    低温殺菌の後、乳酸醗酵させたとても濃い生クリーム。}1\undemi{}
  L、塩18 g、白こしょう3 g。
\item
  \textbf{作業手順}\ldots{}\ldots{}肉と調味料を鉢に入れて細かくすり潰す。卵白を少量ずつ
  加えていく。目の細かい網で裏漉しする。
\end{itemize}

これをソテー鍋に入れ、ヘラで滑らかになるまで混ぜたら、たっぷりの氷で鍋
を囲むようにして2時間冷やす。

次に、生クリームを少しずつ加えながらファルスをのばしていく。丁寧に練っ
ていくこと。またこの作業は鍋底を常に氷にあてた状態で行なうこと。

\hypertarget{ux539fux6ce8-2}{%
\subparagraph{【原注】}\label{ux539fux6ce8-2}}

\ldots{}\ldots{}

\begin{enumerate}
\def\labelenumi{\arabic{enumi}.}
\item
  上で示した生クリームの分量は平均的な数字だ。ファルスのベースとなっ
  ている素材つまり肉、魚、甲殻類によってそれぞれタンパク質の特性が違
  うのだから、素材に吸収される生クリームの量には多少の違いがでてくる
  わけだ。
\item
  ここで示したファルスの作り方は、滑らかな仕上がりのファルスの典型で
  あって、これを越える繊細さを出せるものはないから、ファルスに出来る
  材料すべて、つまり各種の肉、ジビエ、鶏、魚、甲殻類などに適用してい
  い。
\item
  卵白の量は、ファルスのベースと素材によって調整する必要がある。鶏や
  仔牛肉のようにアルブミンが多く含まれていて新鮮な肉であれば、成獣の
  固くなった肉を使う場合よりも量は少なくて済む。つまり、捌いたばかり
  でまだ温かい若鳥の胸肉を使ってこのファルス・ムスーズを作るのであれ
  ば、卵白は省略してもいい。
\item
  良質の生クリームが入手できる環境にあるなら、他のファルスを作るより
  もこのファルスの方がいいだろう。とりわけ、甲殻類をベースとしたファ
  ルスについては重要なことだ。
\end{enumerate}
\end{recette}

\hypertarget{ux725bux8102ux3092ux52a0ux3048ux3066ux4f5cux308bux4ed4ux725bux8089ux306eux30d5ux30a1ux30ebux30b9-ux30b4ux30c7ux30a3ux30f4ux30a920}{%
\subsection[牛脂を加えて作る仔牛肉のファルス /
ゴディヴォ]{\texorpdfstring{牛脂を加えて作る仔牛肉のファルス /
ゴディヴォ\footnote{ゴディヴォgodiveau
  は16世紀フランソワ・ラブレーの小説『ガルガン
  チュアとパンタグリュエル』の「第三の書」(1546年)が初出。原書の綴
  りはguodiveaulx。これは「アンドゥイエット(のようなもの)」とする
  のが一般的な解釈となっている。また、ラブレーはこれに先立つ1534年
  「ガルガンチュア」(=第一の書)において gaudebillaux という表現を
  用いている。これについては Gaudebillaux: sont grasses tripes de
  coiraux 「ゴドビヨとは、たっぷり肥育した牛のトリップ(胃と腸)のこ
  と」と本文で説明している。これらを敷衍すると、ゴディヴォはもともと
  牛などの胃や腸を刻んで詰めた腸詰すなわちアンドゥイエットのことだっ
  た、と考えたくなっても不思議はない。しかし、たとえ16世紀のラブレー
  における guodiveaulx = godiveau が当時アンドゥイエットと呼ばれるも
  のとほぼ同じだったとしても、アンドゥイエット andouilette がアンドゥ
  イユ andouille に縮小辞を付したものであることから、中世のアンドゥ
  イユを確認する必要が出てくる。14世紀末に成立したとされる『ル・メナ
  ジエ・ド・パリ』においてアンドゥイユは確かに「細かく刻んだ胃や腸を、
  腸詰にする」という説明がまず出てくるが、その他に、牛の第1胃だけを
  詰めるもの、豚のコトレットを切り出した端肉を材料にするもの、胸腺肉
  やレバーを掃除した残りの肉を材料にするもの、が挙げられている
  (t.2,p.127)。これに従うなら、中世におけるアンドゥイユとは素材の定
  義があまりはっきりしていなかったもの、言えるだろう。ところが167世
  紀、ピエール・ド・リュヌ『新料理の本』(1660年)において「スペイン
  風アンドゥイエット」というレシピが掲載されている。概要を記すと、仔
  牛肉を細かく刻む。豚背脂少々、香草、卵黄、塩、こしょう、ナツメグ、
  粉にしたシナモンを加える。豚背脂のシートで巻いてアンドゥイエットの
  形状にする。串を刺してローストする。ローストする際に滴り落ちてくる
  肉汁は受け皿で受ける。火が通ったらその肉汁をかける。茹で卵の黄身8〜
  10個分と細かくおろしたパン粉を順につけて、しっかりした衣を作る。提
  供時にレモン汁と羊のジュをかけ、揚げたパセリを添える、というものだ。
  1693年刊マシアロ『宮廷および大ブルジョワ料理の本』では豚のアンドゥ
  イユ、仔牛のアンドゥイユとともに、仔牛のアンドゥイエットというレシ
  ピが掲載されている。最後のものには材料として「細かく刻んだ仔牛肉、
  豚背脂、香草、卵黄、塩、こしょう、ナツメグ、シナモンを加えて作る」
  とある(pp.108-109)。また、1750年に出版された『食品、ワイン、リキュー
  ル事典』において、アンドゥイエットは「細かく刻んだ仔牛肉を楕円形に
  巻いたもの」と定義されている。実際、17、18世紀の料理書に出てくるア
  ンドゥイエットは腸詰であるかどうかは別にしても、仔牛肉を主材料にし
  たものが多い。18世紀ヴァンサン・ラ・シャペル『近代料理』第1巻のア
  ンドゥイエットも細かく刻んだ仔牛肉を豚の腸に詰めて作る。さて、ゴディ
  ヴォに戻ると、17世紀、1653年刊の『フランスのパティスリの本』(ラ・
  ヴァレーヌが著者だと言われている)にはFaire un pasté de gaudiueau
  「ゴディヴォのパテの作り方」という節があり、仔牛腿肉あるいは他の肉
  と脂身を細かく刻んだもの、をパテ(≒パイ包み焼き)に入れると書いて
  ある。つまりここでは「仔牛腿肉」の使用が前提となっている。仮にラブ
  レーの guodiveaulx がこんにち我々のよく知る、牛などの胃や腸を刻ん
  で詰めたアンドゥイエットと同様のものだったとしたら、わずか百年で大
  きな変化を遂げてしまい、逆にこんにちの我々が知るものと大きく違って
  しまったことになってしまう。したがって、これら勘案すれば、ラブレー
  のguodiveaulxもまた仔牛肉を材料にしていたものだった可能性は充分に
  考えられるだろう。 もちろんgaudebillaux という別の巻の名詞との関連
  性は無視出来ないものだが、中世〜ルネサンス期において、食にかかわる
  名詞、概念がしばしば曖昧だったことを考えると、多少のわかりにくさは
  許容せざるを得ない。したがって、本書において仔羊腿肉とケンネ脂を使
  うゴディヴォを「古典的」なファルスとして扱っているのはまことに正鵠
  を射ていると言えよう。}}{牛脂を加えて作る仔牛肉のファルス / ゴディヴォ}}\label{ux725bux8102ux3092ux52a0ux3048ux3066ux4f5cux308bux4ed4ux725bux8089ux306eux30d5ux30a1ux30ebux30b9-ux30b4ux30c7ux30a3ux30f4ux30a920}}

\vspace*{-1.7\zw}

\hypertarget{farce-de-veau-uxe0-la-graisse-de-boeuf-ou-godiveau}{%
\subsection{Farce de Veau à la Graisse de boeuf, ou
Godiveau}\label{farce-de-veau-uxe0-la-graisse-de-boeuf-ou-godiveau}}

\index{farce!veau graisse de boeuf@--- de veau à la graisse de boeuf}
\index{garniture!farce!veau graisse de boeuf@Farce de veau à la graisse de boeuf}
\index{farce!veau glodiveau@Godiveau}
\index{garniture!farce!godiveau@Godiveau}
\index{かるにちゆーる@ガルニチュール!ふあるす@ファルス!きゆうしをくわえてつくるこうしにくのふあるす@牛脂を加えて作る仔牛肉のファルス / ゴディヴォ}
\index{ふあるす@ファルス!きゆうしをくわえてつくるこうしにくのふあるす@牛脂を加えて作る仔牛肉の--- / ゴディヴォ}
\index{かるにちゆーる@ガルニチュール!ふあるす@ファルス!こていうお@ゴディヴォ}
\index{ふあるす@ファルス!こていうお@ゴディヴォ} \index{godiveau}
\index{こていうお@ゴディヴォ}

\begin{recette}
\hypertarget{a.-ux6c37ux3092ux5165ux308cux3066ux4f5cux308bux30b4ux30c7ux30a3ux30f4ux30a9}{%
\subsubsection{A.
氷を入れて作るゴディヴォ}\label{a.-ux6c37ux3092ux5165ux308cux3066ux4f5cux308bux30b4ux30c7ux30a3ux30f4ux30a9}}

\hypertarget{godiveau-mouilluxe9-uxe0-la-glace}{%
\paragraph{Godiveau mouillé à la
glace}\label{godiveau-mouilluxe9-uxe0-la-glace}}

\index{farce@farce!godiveau a@Godiveau A. Godeiveau mouillé à la glace}
\index{ふあるす@ファルス!こていうお@ゴディヴォ!a@A. 氷を入れて作るゴディヴォ}
\index{godiveau@godiveau!a@A. --- mouillé à la glace}
\index{こていうお@ゴディヴォ!a@A. 氷を入れて作る---}

\begin{itemize}
\item
  \textbf{材料}\ldots{}\ldots{}筋をきれいに取り除いた仔牛腿肉1
  kg、\textbf{水気を含んでいない}牛ケンネ脂\footnote{腎臓の周囲を厚く覆っている脂肪。融解温度が低く、上質の牛脂(ヘッ
    ト)の原料にされる。}1.5 kg、全卵8個、塩25 g、白こしょう5
  g、ナツメグ1 g、 透明な氷7〜800 gまたは氷水7〜8 dl。
\item
  \textbf{作業手順}\ldots{}\ldots{}はじめに、仔牛肉とケンネ脂を別々に、細かく刻む。仔牛
  肉はさいの目に切り、調味料と合わせておく。牛脂は細かくして、薄皮は筋
  はきれいに取り除いておく。
\end{itemize}

仔牛肉と牛脂を別々の鉢に入れて、それぞれすり潰す。次にこれらを合わせて
から、完全に混ざり合って一体化するまでよくすり潰し、卵を一個ずつ、すり
潰す作業を止めずに加えていく。

裏漉しして、平皿に\footnote{大きなバット。}広げ、氷の上に置いて翌日まで休ませる。

翌日になったら、再度ファルスをすり潰す。この時、小さく割った氷を少しず
つ加えていき、よく混ぜ合わせる。

ゴディヴォに氷を加え終わったら、必ずテスト\footnote{少量を、沸騰しない程度の温度で火を通し(ポシェ)て様子を見ること。}を行ない、必要に応じて
修正する。固すぎるようなら水を少々加え、柔らかすぎるようなら卵白を少し
加えること。
\end{recette}\newpage
\hypertarget{serie-des-appareiles-et-preparations-diverses-pour-garnitures-chaudes}{%
\section{温製ガルニチュールのためのアパレイユなど}\label{serie-des-appareiles-et-preparations-diverses-pour-garnitures-chaudes}}

\frsec{Série des Appareils et Préparations diverses pour Garnitures chaudes}

\index{garniture@garniture!appareils garnitures chaudes@appareils et préparations diverses pour garnitures chaudes}
\index{appareil@appareil!garnitures chaudes@--- et préparations diverses pour garnitures chaudes}
\index{かるにちゆーる@ガルニチュール!あはれいゆおんせい@温製ガルニチュールのためのアパレイユなど}
\index{あはれいゆ@アパレイユ!おんせいかるにちゆーる@温製ガルニチュールのための---など}
\begin{recette}
\hypertarget{appareils-a-cromesquis-et-a-croquettes}{%
\subsubsection{クロメスキとクロケットのアパレイユ}\label{appareils-a-cromesquis-et-a-croquettes}}

\frsub{Appareils à Cromesquis et à Croquettes}

\index{garniture@garniture!appareil@appareil!cromesquis croquettes@appareils à cromesquis et à croqeuttes}
\index{appareil@appareil!cromesquis croquettes@---s à cromesquis et à croquettes}
\index{cromesqui@cromesqui!appareil@appareils à --- et à croquettes}
\index{croquette@croquette!appareil@appareils à cromesquis et à ---}
\index{かるにちゆーる@ガルニチュール!あはれいゆ@アパレイユ!くろめすきとくろけつと@クロケットとクロメスキのアパレイユ}
\index{あはれいゆ@アパレイユ!くろめすきとくろけつと@クロケットとクロメスキの---}
\index{くろめすき@クロメスキ!あはれいゆ@---とクロケットのアパレイユ}
\index{くろけつと@クロケット!あはれいゆ@クロメスキと---のアパレイユ}

⇒ \protect\hyperlink{hors-d-oeuvres-chauds}{温製オードブル}の章を参照。

\hypertarget{appareils-a-pomme-dauphine-duchesse-marquise}{%
\subsubsection{じゃがいものドフィーヌ、デュシェス、マルキーズのアパレイユ}\label{appareils-a-pomme-dauphine-duchesse-marquise}}

\frsub{Appareils à pomme Dauphine, Duchesse et Marquise}

\index{garniture@garniture!appareil@appareil!pomme dauphine@appareils à pomme Dauphine, Duchesse et Marquise}
\index{appareil@appareil!pomme dauphine@---s à pomme Dauphine, Duchesse et Marquise}
\index{dauphin@dauphin(e)!appareil@appareils à pomme ---e}
\index{duc@duc / duchesse!appareil@appareils à pomme duchesse}
\index{marquis@marquis(s)!appareil@appareils à pomme ---e}
\index{かるにちゆーる@ガルニチュール!あはれいゆ@アパレイユ!しやかいものとふいーぬ@じゃがいものドフィーヌ、デュシェス、マルキーズのアパレイユ}
\index{あはれいゆ@アパレイユ!しやかいものとふいーぬと@じゃがいものドフィーヌ、デュシェス、マルキーズの---}
\index{とふいーぬ@ドフィーヌ!あはれいゆ@アパレイユ!しやかいものとふいーぬ@じゃがいもの---、デュシェス、マルキーズのアパレイユ}
\index{とゆしえす@デュシェス!あはれいゆ@アパレイユ!しやかいものてゆしえす@じゃがいものドフィーヌ、---、マルキーズのアパレイユ}
\index{まるきーす@マルキーズ!あはれいゆ@アパレイユ!しやかいものまるきーす@じゃがいものドフィーヌ、デュシェス、---のアパレイユ}

⇒ \protect\hyperlink{legumes}{野菜料理}の章、\protect\hyperlink{pommes-de-terre}{じゃがいも}の項を参照。
\end{recette}\newpage
\hypertarget{serie-des-appareiles-et-preparations-diverses-pour-garnitures-froides}{%
\section[冷製ガルニチュール用アパレイユなど]{\texorpdfstring{冷製ガルニチュール用アパレイユなど\footnote{この節は、初版で「冷製料理」の章の冒頭に概説としてまとめられて
  いたものを、第二版の改訂時に、ほぼそのままの内容で現在の位置に移動
  させられている。もちろん順序および内容の加筆も行なわれており、異同
  は少なくない。}}{冷製ガルニチュール用アパレイユなど}}\label{serie-des-appareiles-et-preparations-diverses-pour-garnitures-froides}}

\frsec{Série des Appareils et Préparations diverses pour Garnitures froides}

\index{garniture@garniture!appareils garnitures froides@appareils et préparations diverses pour garnitures froides}
\index{appareil@appareil!garnitures froides@--- et préparations diverses pour garnitures froides}
\index{かるにちゆーる@ガルニチュール!あはれいゆれいせい@冷製ガルニチュールのためのアパレイユなど}
\index{あはれいゆ@アパレイユ!れいせいかるにちゆーる@冷製ガルニチュールのための---など}

\hypertarget{mousses-mousselines-et-souffles-froids}{%
\subsection{冷製のムース、ムスリーヌ、スフレ}\label{mousses-mousselines-et-souffles-froids}}

\frsecb{Mousse, Moussseline, et Soufflé froids}

\index{mousse@mousse!froide@--- froide}
\index{mousseline@mousseline!froide@--- froide}
\index{souffle@soufflé!froid@--- froid}
\index{むーす@ムース!れいせい@冷製の---}
\index{むすりーぬ@ムスリーヌ!れいせい@冷製の---}
\index{すふれ@スフレ!れいせい@冷製の---}

温製の場合でも冷製の場合でも、\ul{ムースとムスリーヌはどちらも同じ材料から作られる}。

ムースとムスリーヌの違いは、温製でも冷製でも、通常は10人分が入る大きな
型に詰めて作るのが\ul{ムース}と呼ばれ、いっぽう、\ul{ムスリーヌ}はスプー
ンで整形したり絞り袋を使ったり、あるいは大きなクネルの形をした専用の型
に入れたりして作るが、基本的に\ul{1つ}で1人分と決まっている。スフレは
小さなスフレ型に詰める。
\begin{recette}
\hypertarget{composition-de-l-appareil-pour-mousses-et-mousseline-froides}{%
\subsubsection{冷製のムースとムスリーヌのアパレイユ}\label{composition-de-l-appareil-pour-mousses-et-mousseline-froides}}

\frsub{Composition de l'Appareil pour Mousses et Mousseline froides}

\index{garniture@garniture!appareils garnitures froides@appareils et préparations diverses pour garnitures froides!appareil mousses mousselines froides@composition de l'appareil pour mousses et mousselines froides}
\index{appareil@appareil!garnitures froides@--- et préparations diverses pour garnitures froides!appareil mousses mousselines froides@composition de l'appareil pour mousses et mousselines froides}
\index{mousse@mousse!froide@froide!composition appareil@Composition de l'appareil pour mousses et mousseline froides}
\index{mousseline@mousseline!froide@froide!composition appareil@Composition de l'appareil pour mousses et mousseline froides}
\index{かるにちゆーる@ガルニチュール!あはれいゆれいせい@冷製ガルニチュールのためのアパレイユなど!れいせいのむーすとむすりーぬのあぱれいゆ@冷製のムースとムスリーヌのアパレイユ}
\index{あはれいゆ@アパレイユ!れいせいかるにちゆーる@冷製ガルニチュールのための---など!れいせいのむーすとむすりーぬのあぱれいゆ@冷製のムースとムスリーヌのアパレイユ}
\index{むーす@ムース!れいせい@冷製!むーすとむすりーぬのあぱれいゆ@ムースとムスリーヌのアパレイユ}
\index{むすりーぬ@ムスリーヌ!れいせい@冷製!むーすとむすりーぬのあぱれいゆ@ムースとムスリーヌのアパレイユ}

\begin{itemize}
\tightlist
\item
  \textbf{材料}\ldots{}\ldots{}主素材のピュレ\footnote{本書では加熱した肉や魚、甲殻類のピュレを作る方法への言及はないが、
    \textbf{本章冒頭にある\protect\hyperlink{farce-mousseline}{ファルス・ムスリーヌ}をそのま
    ま使おうなどと考えてはいけない。ここで説明されている冷製のムース、
    ムスリーヌ、スフレの作り方に加熱の工程がまったく含まれていないのは、
    主素材のピュレが既に加熱済みであることを当然の前提としている}から
    だ。つまりここで材料として示されているピュレは\textbf{すべて加熱済みのも
    のをピュレにしたものだ}と考えなければならない。『料理の手引き』の
    当時はローストするか茹でるなどの加熱後に、鉢に入れてすり潰し、裏漉
    ししてから何らかのソース(ここではヴルテ)を加えて漉さ(固さ)を調
    節するなどしていた。現代ではフードプロセッサーや冷凍粉砕調理機など
    を利用すればより容易に滑らかなピュレを作ることが可能だろう。また、
    第3章ポタージュに\protect\hyperlink{les-purees}{ポタージュ・ピュレ}についての概説が
    あるが、そこではポタージュにすることを前提として「つなぎ」の使用が
    作業のプロセスに組込まれて説明されているために、あくまで参考程度に
    読むのがいいだろう。}1 Lすなわち鶏のピュレ、ジビエ、フォワグラ
  や魚、甲殻類のピュレ。溶かした\protect\hyperlink{gelees-ordinaires}{ジュレ}2\undemi{}
  dl、\protect\hyperlink{veloute}{ヴルテ}4 dl、生クリーム4
  dlはちょうどいい固さに立てて6 dl相当にしておく。
\end{itemize}

素材の特性によって、これらの分量比率は多少変更してもいい。同様に、ある
種のムースを作る際にはジュレまたはヴルテのどちらかしか用いなくてもいい。

\begin{itemize}
\tightlist
\item
  \textbf{作業手順}\ldots{}\ldots{}まずベースとなるピュレを入れたボウルを氷の上に置いて、軽
  く混ぜながら、ジュレとヴルテを加える(どちらかしか使わない場合は使う
  もののみ)。次に泡立てた生クリームを加える。
\end{itemize}

味付けを確認する。これは冷製料理ではとても重要なことだ。いつも気をつけ
て確認し、修正を加えるようにすること。

\hypertarget{nota-composition-de-l-appareil-pour-mousses-et-mousseline-froides}{%
\subparagraph{【原注】}\label{nota-composition-de-l-appareil-pour-mousses-et-mousseline-froides}}

生クリームは五分立てにしておくこと。完全に立ててしまっていたら、ムース
は滑らかさが失なわれてパサついた仕上りになってしまう。

\hypertarget{moulage-des-mousses-froides}{%
\subsubsection{冷製ムースの型詰め}\label{moulage-des-mousses-froides}}

\frsub{Moulage des Mousses froides}

\index{garniture@garniture!appareils garnitures froides@appareils et préparations diverses pour garnitures froides!moulage mousses froides@moulage des mousses froides}
\index{appareil@appareil!garnitures froides@--- et préparations diverses pour garnitures froides!moulage mousses froides@moulage des mousses froides}
\index{mousse@mousse!froide@froide!moulage@moulage des mousses froides}
\index{かるにちゆーる@ガルニチュール!あはれいゆれいせい@冷製ガルニチュールのためのアパレイユなど!れいせいむーすのかたつめ@冷製ムースの型詰め}
\index{あはれいゆ@アパレイユ!れいせいかるにちゆーる@冷製ガルニチュールのための---など!れいせいむーすのかたつめ@冷製ムースの型詰め}
\index{むーす@ムース!れいせい@冷製!むーすのかたつめ@ムースの型詰め}

いまもそうしている料理人は少なくないようだが、かつては、プレーンな型あ
るいは浮き彫り模様の付いた型の中に透明なジュレを流して層をつくってやり\footnote{chemiser
  (シュミゼ)ジュレなどを型の内側に流して薄い層を作ること。}、
ムースの主素材と関連あるものを装飾要素として貼り付けていた。

こんにちでは次の方法がむしろ好ましい。銀製のタンバル型\footnote{timbale
  (タンバル)円筒形の比較的浅い型。野菜料理用の深皿もこの語で呼ぶので注意。}の底面だけに
透明なジュレの薄い層をつくる。型の側面の外側に紙の帯を冷たいバターで貼
り付ける。型の\ruby{縁}{ふち}から2〜3 cmくらい高くなるようにすること。
そうするとスフレのような見た目のムースになる。紙の帯は型の内側に貼り付
けてもいい。この紙の帯は提供直前に、ぬるま湯で濡らしてナイフの刃を使っ
てムースからそっと引き剥してやる。

タンバル型の用意が整ったら、ムースを詰めて冷やす。アイスクリーム用の冷
凍庫に入れるほうがいいだろう。この方法は、小さな銀製のスフレ型に詰めて
やってもいいが、それは冷製のスフレにとっておいたほうがいいだろう。アパ
レイユの構成が同じであるにもかかわらず、冷製ムースと冷製スフレの違いを
はっきりさせることが出来るからだ。

とりわけジビエのムースやフォワグラのムースについては、近代的な料理の提
供方法に合わせて作られた銀製かガラス製の容器を用いてもいい。その場合は、
型の底面だけジュレの層をつくってやり、アパレイユをそのまま流し込めばい
い。表面はパレットナイフなどで丁寧に滑らかにならしてやってから、ムース
を冷やす。その後\footnote{型から出して、ということだろう。}、ムースに直接装飾を施し、ジュレをかけて艶を出させる。

ジビエのムースの場合には、そのジビエの胸肉を冷やして、ムースの周囲に飾
るようにする。

\hypertarget{moulage-des-mousselines-froides}{%
\subsubsection{冷製ムスリーヌの整形}\label{moulage-des-mousselines-froides}}

\frsub{Moulage des Mousselines froides}

\index{garniture@garniture!appareils garnitures froides@appareils et préparations diverses pour garnitures froides!moulage mousselines froides@moulage des mousselines froides}
\index{appareil@appareil!garnitures froides@--- et préparations diverses pour garnitures froides!moulage mousselines froides@moulage des mousselines froides}
\index{mousseline@mousseline!froide@froide!moulage@moulage des mousselines froides}
\index{かるにちゆーる@ガルニチュール!あはれいゆれいせい@冷製ガルニチュールのためのアパレイユなど!れいせいむすりーぬのかたつめ@冷製ムスリーヌの型詰め}
\index{あはれいゆ@アパレイユ!れいせいかるにちゆーる@冷製ガルニチュールのための---など!れいせいむすりーぬのかたつめ@冷製ムスリーヌの型詰め}
\index{むすりーぬ@ムスリーヌ!れいせい@冷製!むすりーぬのかたつめ@ムスリーヌの型詰め}

冷製ムスリーヌの型詰めには2つの方法がある。たんに、型にジュレの層を作っ
てやるか、ソース・ショフロワの層を作ってやるかの違いでしかない。どちら
の場合でも、卵形の型に詰めるか、大きなクネルの形状のものにするか、とい
うことになる。

\hypertarget{procede-un-moulage-des-mousselines-froides}{%
\subparagraph{方法1\ldots{}\ldots{}}\label{procede-un-moulage-des-mousselines-froides}}

型の内側に透明なジュレを流して薄い層を作ってやる\footnote{chemiser
  (シュミゼ)。}。その上にアパレイ
ユを張るように塗り、アパレイユのベースとなっている素材とおなじもの---
鶏、ジビエ、甲殻類の身など、とトリュフ---で構成された\protect\hyperlink{salpicons-divers}{サルピコ
ン}を盛り込む。その上からアパレイユを塗って覆い、パ
レットナイフなどを使ってドーム形に滑らかにならす。冷蔵庫に入れて冷し固
める。

\hypertarget{procede-deux-moulage-des-mousselines-froides}{%
\subparagraph{方法2\ldots{}\ldots{}}\label{procede-deux-moulage-des-mousselines-froides}}

型の内側にアパレイユを詰め、さらにサルピコンをその内側に射込む。アパレ
イユで覆って、冷し固める。

型から外す。ムスリーヌのアパレイユの素材と関連性のある\protect\hyperlink{sauce-chaud-froid-ordinaire}{ソース・ショフ
ロワ}を表面を覆うように塗る\footnote{napper
  (ナペ)。覆いかける(ように塗る)こと。}。トリュフおよびそ
の他の素材(これもムスリーヌと関連性があること)を装飾用に細工したもの
を飾り付ける。装飾が剥れないように、上からジュレを塗って艶を出させる。

銀製またはガラス製の深皿の底に透明なジュレの層を作り、その上にムスリー
ヌを並べる。再度ジュレを上からかけてやり、冷蔵庫に入れて提供するまで保
管しておく。

\hypertarget{souffles-froids}{%
\subsubsection{冷製スフレ}\label{souffles-froids}}

\frsub{Soufflés froids}

\index{garniture@garniture!appareils garnitures froides@appareils et préparations diverses pour garnitures froides!souffles froids@soufflés froids}
\index{appareil@appareil!garnitures froides@--- et préparations diverses pour garnitures froides!souffles froids@soufflés froids}
\index{souffle@soufflé!froid@---s froids}
\index{かるにちゆーる@ガルニチュール!あはれいゆれいせい@冷製ガルニチュールのためのアパレイユなど!れいせいすふれ@冷製スフレ}
\index{あはれいゆ@アパレイユ!れいせいかるにちゆーる@冷製ガルニチュールのための---など!れいせいすふれ@冷製スフレ}
\index{すふれ@スフレ!れいせい@冷製---}

冷製スフレはムースそのものに他ならない。だから構成はまったく同じだ。た
だ、先に見たようにスフレが10人分\footnote{1 service
  (アンセルヴィス)、格式のある宴席料理などを作る際の単位。基本は10人分。}を確保できるだけの大きな型に詰める
のに対して、スフレはそもそも、小さなスフレ型に入れてひとり1つ宛で作る
ものだ。

アパレイユを型に詰める方法はムースの場合と同様、つまり、スフレ型の底に
ジュレの層を敷いてその上にアパレイユを盛り、型の縁より高くなるように周
囲に巻いた紙の帯を利用して縁より高くアパレイユを盛る。そうすると、冷や
し固めた後で紙の帯を取り除けば、まるで温製のスフレのように見えることに
なる。

\hypertarget{nota-souffles-froids}{%
\subparagraph{【原注】}\label{nota-souffles-froids}}

ここまで述べた3種の作り方の基礎はおなじだから、ポイントは次のようにまとめられる。

\begin{enumerate}
\def\labelenumi{\arabic{enumi}.}
\item
  ムースは「スフレ」の名称で供してもいいものだが、混同されるのを避けるために「ムース」の名称で約10人分をひとつの型に入れて作る。
\item
  ムスリーヌはサルピコンを射込んだものであってもそうでなくても、大きなクネルであって、ひとりあたり1つにする。
\item
  スフレは小さなムースであって、スフレ型あるいは似たような型に詰めて、これもひとりあたり1つとする。
\end{enumerate}
\end{recette}
\hypertarget{aspics}{%
\subsection{アスピック}\label{aspics}}

\frsecb{Aspics}

\index{garniture@garniture!appareils garnitures froides@appareils et préparations diverses pour garnitures froides!aspics@aspics}
\index{appareil@appareil!garnitures froides@--- et préparations diverses pour garnitures froides!aspics@aspics}
\index{aspics@aspics (généralité)}
\index{かるにちゆーる@ガルニチュール!あはれいゆれいせい@冷製ガルニチュールのためのアパレイユなど!あすぴつく@アスピック(概説)}
\index{あはれいゆ@アパレイユ!れいせいかるにちゆーる@冷製ガルニチュールのための---など!あすぴつく@アスピック(概説)}
\index{あすぴつく@アスピック(概説)}

アスピックを作る際に、肝に銘じておくべき第一のポイントは、どんなアスピッ
クでも、ジュレがジューシー\footnote{原文succulent(スュキュロン)はsuc(スュック=肉汁)から派生した
  形容詞で、もともとは「汁気の多い」の意味だったが、そこから転じて
  「美味な、滋味に富んだ」の意味で一般的に用いられている。ここでは、
  両方のニュアンスで表現されていると解釈できる。}で美味しく、完全に透き通ったもので、ちょうど
いい加減に固まっていなければならないことである。

アスピックを作る際には、昔もそうだったが現代でも、中央に穴の空いたアス
ピック型\footnote{moule à douille
  (ムーラドゥイユ)サヴァラン型のような中央に穴 が空いた型。
  現代では「アスピック型」というと楕円形で中央に穴のな
  いものを指すことが多いが、それとは異なる。クグロフ型のようなものを
  イメージするとわかりやすいだろう。19世紀、アスピックには高さのある
  型が多く用いられたようである。なお、現代では一般にサヴァラン型とい
  うと、型の高さや穴の大きさ等さまざまなタイプのものをまとめて指すこ
  とになるので注意。高さのない(低い)、中央の穴が大きな型について、
  エスコフィエはボルデュール型 moule à bordure (ムーラボルデュール)
  と呼んで区別している}でプレーンなもの、波模様等の装飾のあるものが用いられている。

ボルデュール型\footnote{moule à bordure
  (ムーラボルデュール)蛇の目形に料理の縁り飾り
  を作るための、やや丈が低く中央の穴が大きいリング型。}も使われることがあるが、一般的に、アスピックの中心にガル
ニチュールを盛り込む場合のみである。

アスピックを型に入れる時には、まず、型の底と周囲に装飾をする。

そのために、型は砕いた氷の中に入れてよく冷やしておく。やや固まりかけた
ジュレ少量を流し入れ、型を氷の上で転がしながらジュレを周囲に貼り付かせ
る(シュミゼ)。次に、装飾するパーツを、固まらない程度に冷たいジュレに浸
してからすぐに貼り付ける。装飾については料理人のセンスとアイデア次第な
ので、ここで明確に述べておくべきことはほとんどない。ひとつだけ言えるの
は、常に正確な作業を期して、型からアスピックを出したときに装飾がはっき
りと見えるようにすべき、ということだけだ。

装飾に用いる素材はアスピックの主素材と関連性のあるものでなくてはならな
い。一般的には、トリュフ、ポシェした卵白、コルニション\footnote{cornichon
  主としてピクルスにする小型のきゅうり、およびそのピク
  ルスのこと。日本では、ハンバーガーによく用いられているドイツ系のピ
  クルス用品種であるガーキンス(英 gherkins 独 Einlegegurken)と混同さ
  れることがあるが、コルニションはより小さなサイズで収穫し、フレッシュ
  な状態では「いぼ」が尖っているのが特徴。}、ケイパー、
いろいろな香草の葉先、ラディッシュの薄い輪切り、オマールのコライユ\footnote{胴の背側にあるオレンジ色がかった「内子」。}、
\protect\hyperlink{saumure-liquide-pour-langues}{赤く漬けた舌肉}、等。

アスピックのガルニチュールが種々のエスカロップ\footnote{escalope
  (エスカロップ)筋線維とは垂直方向に、厚さ1〜2 cmに薄
  切りにした仔牛などの肉や魚の薄い切り身。}や長方形に切ったフォワグ
ラ等で、型の大きさから何度も並べなければならない場合、ジュレの層と交互
に重ねて型に入れていく。新しい層を並べる際には先に入れたジュレがある程
度固まってからにする。

アスピックの型入れでは常に、最後のジュレの層を充分な厚みにする。できる
だけ、型を氷に埋めるようにしながらジュレを流し込んでいくが、早く冷やす
ために氷に塩を加えてはいけない。塩を使うとジュレの透明さが損なわれるか
らである。

\noindent\textbf{型から外す方法}\ldots{}\ldots{}型を湯につけてただちに水気を拭い、折り
ただんだナフキンや彫刻した氷のブロック等に、アスピックを裏返してあける。

菱形や正方形に切ったジュレのクルトン\footnote{パンで作るクルトンと同様に、菱形やさいの目に切った冷製料理装飾用
  のジュレもクルトンと呼ぶ。}、またはアシェしたジュレで周囲を飾 る。

\hypertarget{nota-aspics}{%
\subparagraph{【原注】}\label{nota-aspics}}

アスピックを型に入れて作るには、必然的に、ジュレが相当に固いものでなけ
ればならないが、これはまことに具合がよろしくない。というのも、固いジュ
レは口あたりがよくないからだ。だから現代の調理現場では、以下のような方
法を採っている。タンバル型か、氷に嵌め込むようにした銀やガラスあるいは
陶製の深皿の底に予めジュレの層を作って固めておき、その上にアスピックの
素材を並べる。次に、固まりかけのジュレをたっぷり覆いかける。この方法で
は、装飾をしなければならない場合は、アスピックの調理をおこなう前に、主
素材にじかに装飾することになる。

\hypertarget{chauds-froids}{%
\subsection[ショフロワ]{\texorpdfstring{ショフロワ\footnote{chaud-froid
  このchaud(熱い)とfroid(冷たい)の合成語の複数形は、それぞれ
  にsを付ける、chauds-froidsとなる。合成語の複数形はいろいろなパターンがあるので、必要が
  出たらその都度覚えるようにしたほうがいい。}}{ショフロワ}}\label{chauds-froids}}

\frsecb{Chauds-froids}

\index{garniture@garniture!appareils garnitures froides@appareils et préparations diverses pour garnitures froides!chauds-froids@chauds-froids}
\index{appareil@appareil!garnitures froides@--- et préparations diverses pour garnitures froides!chauds-froids@chauds-froids}
\index{chauds-froids@chauds-froids (généralité)}
\index{かるにちゆーる@ガルニチュール!あはれいゆれいせい@冷製ガルニチュールのためのアパレイユなど!しよふろわ@ショフロワ(概説)}
\index{あはれいゆ@アパレイユ!れいせいかるにちゆーる@冷製ガルニチュールのための---など!しよふろわ@ショフロワ(概説)}
\index{しよふろわ@ショフロワ(概説)}

\protect\hyperlink{sauce-chaud-froid-ordinaire}{ソース・ショフロワ}には大抵の場合、切り
分けた素材を浸す。が、時として大きな塊肉全体をソース・ショフロワで覆わ
なくてはならない場合もある。ただ、そういう仕立てにする場合には、別の料
理名となっている。

ショフロワが複数のばらばらのパーツからなる場合には、それらをソース・ショ
フロワに漬けたら網の上に並べておく。ソースが冷えたら、それぞれのパーツ
に装飾をし、ジュレを覆いかけて艶を出してやる。さらに盛り付けの際にはみ
出す余分なソースについてはきれいに取り除いておくこと。

大きな塊肉の場合は、よく冷えてはいるけれどまだ流動性のある状態のソース・
ショフロワを一気に塗りつけて、その後に装飾をし、ジュレを塗って艶出しす
ること。

切り分けた素材からなるショフロワの盛り付けは、\protect\hyperlink{fonds-de-plats}{皿の上の
台}の上に盛り付けてもいいし、縁飾りの内側に、パンまた
は米、セモリナ粉で作った台を置いてその上に盛り付けてもいい。あるいは、
銀製か陶製、ガラス製の深皿に盛り付けてもいい。

大きな塊肉のショフロワの場合、皿の上の台にのせてもいいし、あるいは、氷
のブロックに料理が嵌まるようにブロックを削ってからそこに盛り付けるのも
いい。

ショフロワ仕立ての鶏やジビエについては、正確に切り分けて\footnote{基本的に鶏および鳥類のジビエの可食部は胸肉のみとされていたこと
  に留意。}皮は剥いでおく
こと。手羽や下腿肉は使わないので、別の用途に取り置いておくといい。

細かく切った素材のショフロワ仕立ての場合、添えてやるマッシュルームや雄
鶏のとさかとロニョン\footnote{rognon
  (ロニョン)牛、羊などの場合は腎臓だが、雄鶏の場合は精巣
  のこと。高級食材として珍重された。}にもソース・ショフロワを塗ってやること。トリュ
フはただジュレをかけて艶を出すだけでいい。
\newpage


%% II. garnitures
\hypertarget{garniture}{%
\chapter{II. ガルニチュール Garnitures}\label{garniture}}

\index{garnitures@garnitures} \index{かるにちゆーる@ガルニチュール}

\vspace*{1.7\zw}

料理においてガルニチュール\footnote{garniture
  一般的には「付け合せ」と訳すが、本書におけるガルニチュー
  ルはたんなる料理の「付け合わせ」にとどまらず、こんにちではそれ自体
  がひとつの料理として成立し得るものも多い。そのため、あえて片仮名で
  ガルニチュールとした。なお、「付け合わせ」の意味で「ガルニ」または
  「ガロニ」などというスラングを用いる調理現場もある。}は重要なものだから、料理人は決してガルニ
チュールの役割を軽視してはいけない。ガルニチュールの構成をどうするかは、
添える料理の主素材との関係性で決まる。気まぐれ的なものや不自然なもの
は絶対にいけない。

ガルニチュールの構成要素は、場合によりけりだが、もっぱらどんな種類の料
理に添えるかで決まる。具体的には、野菜料理やパスタ、ファルスでさ
まざまな形状に作ったクネル\footnote{quenelle
  仔牛肉や鶏肉、豚肉などと獣脂をすり潰して、しばしば「つ
  なぎ」として後述のパナードを加えて練り、スプーンなどを用いて整形し、
  沸騰しない程度の温度で茹でる{[}ポシェ{]}またはオーブンで焼いたもの。
  スプーンを2つ使ってラグビーボールに似た形状にしたものが代表的だが、
  他にもいろいろな形状、大きさにする。}、あるいは雄鶏のとさかとロニョン\footnote{\protect\hyperlink{garniture-financiere}{ガルニチュール・フィナンシエール}やその
  バリエーションともいえる\protect\hyperlink{garniture-godard}{ガルニチュール・ゴダー
  ル}で必須の素材。ロニョンrognonは通常なら腎臓を
  意味するが、この場合のロニョンは rognon blanc ロニョンブラン(白い
  ロニョン)とも呼ばれるもので、雄鶏の精巣のこと。}、さ
まざまな種類の茸、オリーブとトリュフ、イカや貝および甲殻類、場合によっ
ては卵、小魚、牛や羊の副生物\footnote{正肉以外の部分。例えば内臓や骨髄など。Ris
  de vea(リドヴォー)仔牛胸腺肉などはこれに含まれる。}など。

その昔、ガルニチュールというのは、マトロットやコンポート、ブルゴーニュ
風料理などのように風味付けのために用いた素材がそのまま添えられたもので
あった。

ガルニチュールにする野菜は、どういう仕立ての皿にするかで役割が決まり、
それに合うように切って形状を整え、調理する。ただし、野菜の調理法は「野
菜料理」として調理する場合と同じだ。

パスタやイカ、貝類、甲殻類についても同様のことが言える。

この章では、それぞれのガルニチュールを構成する素材とその分量を示すに留
めるので、各素材の調理法ついてはその素材に対応する章を参照すること。

\hypertarget{ux30d5ux30a1ux30ebux30b9-5}{%
\section[ファルス ]{\texorpdfstring{ファルス \footnote{本来は「詰め物」の意で、鶏のローストの内臓を抜いた空洞部分に詰めたり、ガランティーヌやパテアンクルートの内部の詰め物などの用途に用いられる。この意味はこんにちでも変化がないが、本文にあるように、クネルにしてガルニチュールの一部にするなど、用途は多岐にわたる。本書ではファルスとして用いられるもののうち、肉および魚肉をベースにしたものをこの節にまとめて分類、説明している。したがって、ここでファルスとして挙げられていないファルスも料理によっては多い(例えば丸鶏の空洞部分に米などを詰めるのもファルス)ことに注意。}}{ファルス }}\label{ux30d5ux30a1ux30ebux30b9-5}}

\hypertarget{serie-des-farces-diverses}{%
\subsection{Série des farces diverses}\label{serie-des-farces-diverses}}

\index{garnitures@garnitures!farces@farces} \index{farce@farce}
\index{かるにちゆーる@ガルニチュール!ふあるす@ファルス}
\index{ふあるす@ファルス}

ガルニチュールの多くは、その構成要素にファルスあるいはファルスで作った
「クネル」が含まれている。ファルスはまた、多くの大きな仕立ての料理にも
使われる。ここではまずファルスの材料および作り方を示し、使い途について
は後で述べることにする。

ファルスは大きく5種に分類される。

\begin{enumerate}
\def\labelenumi{\arabic{enumi}.}
\item
  仔牛肉と脂で作るもの。すなわち古典料理における\textbf{ゴディヴォ}。
\item
  基本となる材料はさまざまだが、「つなぎ」に主としてパナードを使うもの。
\item
  近代的な手法で、生クリームを用いてふんわり泡立てたファルス。ムース、ムスリーヌに用いる。
\item
  レバーをベースとした「ファルス・\textbf{グラタン}」。種類はいろいろだが作り方は常に同じ。
\item
  \ruby{主}{おも}に\protect\hyperlink{}{ガランティーヌ}、\protect\hyperlink{}{パテアンクルート}、\protect\hyperlink{}{テリーヌ}などの冷製料理に用いるシンプルなファルス。
\end{enumerate}

\hypertarget{ux30d5ux30a1ux30ebux30b9ux7528ux306eux30d1ux30caux30fcux30c9ux306bux3064ux3044ux30666}{%
\subsection[ファルス用のパナードについて]{\texorpdfstring{ファルス用のパナードについて\footnote{パナードは本来、パンと水、バターを弱火で時間をかけて煮た粥のようなものを意味した。本書ではその意味を拡大して肉や魚肉をベースとしたファルスを加熱する際に崩れないようにする「つなぎ」として、この語を用いている。そのため、必ずしもパンを材料としていないものが含まれている。}}{ファルス用のパナードについて}}\label{ux30d5ux30a1ux30ebux30b9ux7528ux306eux30d1ux30caux30fcux30c9ux306bux3064ux3044ux30666}}

\vspace*{-1.7\zw}

\hypertarget{les-panades-pour-farces}{%
\subsection{Les Panades pour Farces}\label{les-panades-pour-farces}}

\index{garnitures@garnitures!farces@farces}
\index{farce@farce!panade@les panades pour farces}
\index{かるにちゆーる@ガルニチュール!ふあるす@ファルス!はなーと@パナード}
\index{ふあるす@ファルス!はなーと@---用パナード}

ファルスに用いるパナードにはいくつもの種類がある。ファルスの種類や、そ
のファルスを添える料理の性質によって使い分けることとなる。

原則として、パナードの分量は、ファルスのベースとする素材が何であれ、そ
の半量を越えないようにすること。

卵とバターを用いるパナードの場合はレシピの分量どおりに作らなければなら
ないから、それを合わせて作るファルスの全体量のほうを調節してやること。

パナードE以外のパナードは使用する際には必ず完全に冷めた状態になってい
ること。パナードが出来上がったら、バターを塗った平皿か天板に流し広げ、
早く冷めるようにする。このとき、バターを塗った紙で蓋をするか、表面にバ
ターのかけらをいくつか置いてやり、パナードが直接空気に触れないようにし
てやること。

以下のパナードのレシピは仕上がり重量が正味500
gになるように調整してある。

したがって、必要な量のパナードを作るのに材料を増やしたり減らしたりする
のも難しくはないだろう\footnote{原文では、Rien de plus simple, donc, que
  \ldots{}
  となっており、直訳すると「これ以上に簡単なことはない」と言いきっているが、都度計算しなければならないことに変わりはないので、多少ニュアンスを柔らげて訳した。}。

\hypertarget{ux30d1ux30caux30fcux30c9}{%
\subsection{パナード}\label{ux30d1ux30caux30fcux30c9}}

\vspace*{-1.7\zw}

\hypertarget{panades}{%
\subsection{Panades}\label{panades}}

\index{panade} \index{garniture!panade} \index{garniture!farce!panade}
\index{かるにちゆーる@ガルニチュール!ふあるす@ファルス!はなーと@パナード}
\index{はなーと@パナード}
\begin{recette}
\hypertarget{a.-ux30d1ux30f3ux306eux30d1ux30caux30fcux30c9}{%
\subsubsection{A.
パンのパナード}\label{a.-ux30d1ux30f3ux306eux30d1ux30caux30fcux30c9}}

\hypertarget{panade-a}{%
\paragraph{Panade au pain}\label{panade-a}}

\index{garnitures@garnitures!farces@farces!panade a@panade A}
\index{farce@farce!panade@les panades pour farces!panade a@panade A}
\index{panade!a pain@A. --- au pain}
\index{かるにちゆーる@ガルニチュール!ふあるす@ファルス!はなーとa@パナードA. パンの---}
\index{ふあるす@ファルス!はなーと@---用パナード!a@A. パンのパナード}
\index{はなーと@パナード!a@A. パンの---}

\ldots{}\ldots{}\textbf{魚を素材にした固めのファルス用}

\begin{itemize}
\item
  \textbf{材料}\ldots{}\ldots{}沸かした牛乳3
  dl、固くなった白パン\footnote{ここでいわゆるバゲットのようなパンの外側を削り落した白い部分、あ
    るいは食パンの「耳」を切り落した白い部分を使う、ということ。なお、
    パンは使う小麦粉の精白度や種類によって、pain complet (パンコンプ
    レ)全粒粉パン、pain de sègle(パンドセーグル)ライ麦パン、一般的
    な小麦粉と食塩、塩、パン種だけで作るバゲットなどの pain と、バター
    や砂糖を加えて作るヴィエノワズリ(クロワッサンやパンオショコラ、ブ
    リオシュなど)に分けられる。イギリスやアメリカのいわゆる食パン(フ
    ランス語 pain de mie パンドミ)は小麦粉、バター、塩、イースト菌、
    牛乳などで作られている。また、現代フランスでバゲットなどのパンに用
    いられている小麦粉の精白度は、T-55と呼ばれる灰分(小麦粉を燃やした
    際に残る炭水化物およびタンパク質以外の要素)0.5〜0.6%のものが主流
    であり、いわゆる食パンpain de mie(パンドミ)やヴィエノワズリには
    T-45(灰分0.5%以下)が多く用いられている。このほかT-65(灰分0.62〜
    0.75%)およびT-80(灰分0.75〜0.9%)、T-110(灰分1.0〜1.2%)、
    T-150(灰分1.4%前後、いわゆる全粒粉)のように種類がある。このうち
    T-45およびT-55はfarine blanche(ファリーヌブロンシュ)と呼ばれ、
    T-150はfarine complète(ファリーヌコンプレット)と通称されている。
    灰分が高くなればそれだけ不純物が多いわけだから、粉は薄い茶色あるい
    はグレーがかった色合いになり、パンを焼く場合などはグルテン形成が難
    しくなりやすい。ただし、香りゆたかなパンを実現しやすいという面もあ
    る。結果として、例えば全粒粉パンは香りはいいが固い仕上がりになる。か
    つては精白度の低い(すなわち灰分の多い)粉ほど重量あたりの価格が安
    く、パンの価格もそれに比例していた。また、本書では基本的に小麦粉を
    使う場合にその精白度についての指示はないが、概ねT-55またはT-45相当
    のもの考えていいだろう。なお、日本に輸入されている小麦は北米産のも
    のがほとんどで、硬質小麦を粉にしたものが「強力粉」、軟質小麦の場合
    は「薄力粉」と呼ばれ、精白度合いによる分類は通常なされていないが、
    製品としては概ねT-45相当あるいはそれ以上の精白度のものが多い。}の身250
  g、塩5 g。
\item
  \textbf{作業手順}\ldots{}\ldots{}パンの身を牛乳に浸して完全にもどす。強火にかけて、ペー
  スト状になったパンがヘラから簡単に取れるくらいまで水気をとばす。バター
  を塗った平皿か天板に広げ、冷ます。
\end{itemize}

\maeaki

\hypertarget{b.-ux5c0fux9ea6ux7c89ux306eux30d1ux30caux30fcux30c9}{%
\subsubsection{B.
小麦粉のパナード}\label{b.-ux5c0fux9ea6ux7c89ux306eux30d1ux30caux30fcux30c9}}

\hypertarget{panade-b}{%
\paragraph{Panade à la farine}\label{panade-b}}

\index{garnitures@garnitures!farces@farces!panade b@panade B}
\index{farce@farce!panade@les panades pour farces!panade b@panade B}
\index{panade!b farine@B. --- à la farine}
\index{かるにちゆーる@ガルニチュール!ふあるす@ファルス!はなーとb@パナードB. 小麦粉の---}
\index{ふあるす@ファルス!はなーと@---用パナード!b@パナードB. 小麦粉の---}
\index{はなーと@パナード!b@B. 小麦粉の---}

\ldots{}\ldots{}\textbf{肉、魚などあらゆるファルスに用いられる}

\begin{itemize}
\item
  \textbf{材料}\ldots{}\ldots{}水3 dl、塩2 g、バター50
  g、篩にかけた小麦粉150 g。
\item
  \textbf{作業手順}\ldots{}\ldots{}片手鍋に水、塩、バターを入れて火にかけ、沸騰させる。
  火から外して小麦粉を加えて混ぜる。再度火にかけて、\protect\hyperlink{}{シュー生地}を
  作る要領で余計な水分をとばす。上記パナードAと同様にして冷ます。
\end{itemize}

\maeaki

\hypertarget{c.-ux30d1ux30caux30fcux30c9ux30d5ux30e9ux30f3ux30b8ux30d1ux30fcux30cc12}{%
\subsubsection[C. パナード・フランジパーヌ]{\texorpdfstring{C.
パナード・フランジパーヌ\footnote{フランジパーヌとは製菓で用いられる、小麦粉、砂糖、卵を混ぜて牛
  乳とバニラを加えて煮、砕いたマカロンmacaronを加えたクリーム。なお、
  これに用いられるマカロンは、現代日本でよく知られているタイプとは違
  い、すり潰したアーモンドと卵白、砂糖を混ぜた生地を紙の上にクルミ大
  に絞り出してオーブンで焼いただけもの。macaron craquelé(マカロンク
  ラクレ)はこのタイプの代表的なもので、焼く際に膨らんで割れ目が出来
  ることからクラクレの名称が付けられた。ところで、日本にマカロンが伝
  わった時期は判然としないが、このタイプのものが太平洋戦争前には、アー
  モンドを落花生に代え、「まころん」の名称でいくつかの製菓会社で製造
  されるようになり、現在も生産されている。}}{C. パナード・フランジパーヌ}}\label{c.-ux30d1ux30caux30fcux30c9ux30d5ux30e9ux30f3ux30b8ux30d1ux30fcux30cc12}}

\hypertarget{panade-c}{%
\paragraph{Panade à la Frangipane}\label{panade-c}}

\index{garnitures@garnitures!farces@farces!panade c@panade C}
\index{farce@farce!panade@les panades pour farces!panade c@panade C}
\index{panade!c frangipane@C. --- à la Frangipane}
\index{かるにちゆーる@ガルニチュール!ふあるす@ファルス!はなーとc@パナードC}
\index{ふあるす@ファルス!はなーと@---用パナード!はなーとC@パナードC}
\index{はなーと@パナード!c@C. ---・フランジパーヌ}

\ldots{}\ldots{}\textbf{鶏のファルス、魚のファルス用}

\begin{itemize}
\item
  \textbf{材料}\ldots{}\ldots{}小麦粉125 g、卵黄4個、溶かしバター90
  g、塩2 g、こしょう1 g、おろしたナツメグの粉ごく少量、牛乳2\undemi{}
  dl。
\item
  \textbf{作業手順}\ldots{}\ldots{}片手鍋に小麦粉と卵黄を入れてよく練る。溶かしバター、
  塩、こしょう、ナツメグを加える。沸かした牛乳で少しずつ溶きのばしてい
  く。
\end{itemize}

\protect\hyperlink{}{標準的なフランジパーヌ}と同様に、火にかけて5〜6分間、泡立て器で混
ぜながら煮る。ちょうどいい漉さになったら、バットに移して\footnote{débarasser
  (デバラセ)バットなどに移す、片付ける、の意。とりわけ前者の意味に注意。}冷ます。

\maeaki

\hypertarget{d.-ux7c73ux306eux30d1ux30caux30fcux30c9}{%
\subsubsection{D.
米のパナード}\label{d.-ux7c73ux306eux30d1ux30caux30fcux30c9}}

\hypertarget{panade-d}{%
\paragraph{Panade au Riz}\label{panade-d}}

\index{garnitures@garnitures!farces@farces!panade d@panade D}
\index{farce@farce!panade@les panades pour farces!panade d@panade D}
\index{panade!d riz@D. --- au Riz}
\index{かるにちゆーる@ガルニチュール!ふあるす@ファルス!はなーとd@パナードD. 米の---}
\index{ふあるす@ファルス!はなーと@---用パナード!はなーとd@D. 米のパナード}
\index{はなーと@パナード!d@D. 米の---}

\ldots{}\ldots{}\textbf{いろいろなファルスに用いられる}

\begin{itemize}
\item
  \textbf{材料}\ldots{}\ldots{}米200 gすなわち2
  dlあるいは大さじ8杯。\protect\hyperlink{}{白いコンソメ}6 dl、バター20
  g。
\item
  \textbf{作業手順}\ldots{}\ldots{}米を入れた鍋にコンソメを注ぎ、バターを加える。火にかけて沸騰させたら、オーブンに入れて40〜45分間加熱する。この間、米に触れないようにすること。
\end{itemize}

オーブンから出したら、米粒がよく潰れるようにヘラでしっかりと混ぜる。その後、冷ます。

\maeaki

\hypertarget{e.-ux3058ux3083ux304cux3044ux3082ux306eux30d1ux30caux30fcux30c9}{%
\subsubsection{E.
じゃがいものパナード}\label{e.-ux3058ux3083ux304cux3044ux3082ux306eux30d1ux30caux30fcux30c9}}

\hypertarget{panade-e}{%
\paragraph{Panade à la pomme de terre}\label{panade-e}}

\index{garnitures@garnitures!farces@farces!panade e@panade E}
\index{farce@farce!panade@les panades pour farces!panade e@panade E}
\index{panade!e riz@E. --- à la pomme de terre}
\index{かるにちゆーる@ガルニチュール!ふあるす@ファルス!はなーとe@パナードE}
\index{ふあるす@ファルス!はなーと@---用パナード!はなーとe@パナードE}
\index{はなーと@パナード!e@E. じゃがいもの---}

\ldots{}\ldots{}\textbf{仔牛および他の白身肉の、詰め物\footnote{fourrré
  (フレ)詰め物をした。farci (ファルシ)も同様に「詰め
  物をした」の意だが、後者はより一般的で、前者はオムレツやクレープに
  中身を詰めて「包む」のが本来の意味。すなわち、このパナードを加えた
  ファルスで、何らかの素材を「包む」と解釈してもいい。とりわけこの
  fourréには日本料理の用語「射込む」をあてる場合もある。}をする大きなクネルに用いられる}

\begin{itemize}
\item
  \textbf{材料}\ldots{}\ldots{}茹でて皮を剥いたばかりの中位のサイズのじゃがいも2個、牛
  乳3 dl、塩 2g、白こしょう\undemi{} g、ナツメグ少々、バター20 g。
\item
  \textbf{作業手順}\ldots{}\ldots{}牛乳を2.5
  dlになるまで煮詰める\footnote{原文は réduire le lait d'un sixième
    直訳すると「牛乳を
    \unsixieme{}量だけ煮詰める」すなわち「\cinqsixiemes{}量まで煮詰め
    る」のだが、かえって分かりにくいだろうから、ここでは具体的な数字に
    直して訳した。分量を代えて作る場合には85%まで煮詰めるくらいと考え
    てもいいだろう。そもそも、じゃがいもの重さが曖昧なのだから、あまり
    細かい数字にこだわらず臨機応変に考えること。}。バター、調味料、
  薄く輪切りにしたじゃがいもを加え、15分間程加熱する。
\end{itemize}

このパナードはまだ少し\ruby{微温}{ぬる}いくらいで使用すること。完全に
冷めてからではいけない。完全に冷めてから練ると粘りが出てしまうからだ。
\end{recette}
\hypertarget{ux30d5ux30a1ux30ebux30b9}{%
\subsection{ファルス}\label{ux30d5ux30a1ux30ebux30b9}}

\vspace*{-1.7\zw}

\hypertarget{farces}{%
\subsection{Farces}\label{farces}}

\index{farce} \index{garniture!farce}
\index{かるにちゆーる@ガルニチュール!ふあるす@ファルス}
\index{ふあるす@ファルス}

ベースとなる素材が\textbf{仔牛}、\textbf{鶏}、\textbf{ジビエ}あるいは\textbf{甲殻類}であっても、分量と作
業手順はどんなファルスでも同じだ。そのベースにする素材を代えればいいの
だから、ここでは各種ファルスの典型的なレシピを示せば充分だろう。料理で
用いられるファルスひとつひとつを説明するのに一章をあてる必要はないと思
われる。
\begin{recette}
\hypertarget{a.-ux30d1ux30caux30fcux30c9ux3068ux30d0ux30bfux30fcux3092ux7528ux3044ux308bux30d5ux30a1ux30ebux30b9}{%
\subsubsection{A.
パナードとバターを用いるファルス}\label{a.-ux30d1ux30caux30fcux30c9ux3068ux30d0ux30bfux30fcux3092ux7528ux3044ux308bux30d5ux30a1ux30ebux30b9}}

\hypertarget{farce-a}{%
\paragraph{Farce à la Panade et au beurre}\label{farce-a}}

\index{farce!a@A. --- à la Panade et au beurre}
\index{garniture!farce!a@A. Farce à la Panade et au beurre}
\index{かるにちゆーる@ガルニチュール!ふあるす@ファルス!a@A. パナードとバターを用いるファルス}
\index{ふあるす@ファルス!a@A. パナードとバターを用いる---}

(標準的なクネル、肉料理\footnote{原文 entrée
  (アントレ)、現代では「前菜」の意味で用いられるが、
  本書では概ね10人前を一皿に盛ったものを指し、現代では立派にメインの
  料理として通用するものが多くある。}の縁飾り etc.)

\begin{itemize}
\item
  \textbf{材料}\ldots{}\ldots{}ていねいに筋取りをした肉1
  kg、\protect\hyperlink{panade-b}{パナードB} 500 g、塩12 g、こしょう2
  g、全卵4個、卵黄8個。
\item
  \textbf{作業手順}\ldots{}\ldots{}肉をさいの目に切って鉢に入れ、調味料を加えてすり潰す。
  いったん肉を取り出して、パナードをよくすり潰しながらバターを加える。
  肉を戻し入れ、すりこ木\footnote{pilon
    (ピロン)形状は日本のすりこ木をやや異なるのが多い。ここ
    では大理石の鉢もしくは陶製のボウルを用いて作業していることに注意。
    現代ではフードプロセッサを用いるところだろうが、かつては人力で、力
    を込めて丁寧に作業していたということは頭に留めておきたい。}で力強く練って全体をまとめる。
\end{itemize}

次に全卵と卵黄を加えて混ぜ合わせる。これは2回に分けても1回でやってもい
い。裏漉しして陶製の容器に入れる。さらに泡立て器で滑かになるまで混ぜる。

\hypertarget{ux539fux6ce8}{%
\subparagraph{【原注】}\label{ux539fux6ce8}}

どんな種類のファルスを作る場合でも、必ず少量を沸騰しない程度の温度で茹
でて\footnote{pocher (ポシェ)。}テストしてから、クネルの整形に取りかかること。

\maeaki

\hypertarget{b.-ux30d1ux30caux30fcux30c9ux3068ux751fux30afux30eaux30fcux30e0ux3092ux7528ux3044ux308bux30d5ux30a1ux30ebux30b9}{%
\subsubsection{B.
パナードと生クリームを用いるファルス}\label{b.-ux30d1ux30caux30fcux30c9ux3068ux751fux30afux30eaux30fcux30e0ux3092ux7528ux3044ux308bux30d5ux30a1ux30ebux30b9}}

\hypertarget{farce-b}{%
\paragraph{Farce à la Panade et à la Crème}\label{farce-b}}

\index{farce!b@B. --- à la Panade et à la crème}
\index{garniture!farce!b@B. Farce à la Panade et à la crème}
\index{かるにちゆーる@ガルニチュール!ふあるす@ファルス!b@B. パナードと生クリームを用いるファルス}
\index{ふあるす@ファルス!b@B. パナードと生クリームを用いる---}

(滑らかな仕上がりのクネル用)

\begin{itemize}
\item
  \textbf{材料}\ldots{}\ldots{}筋取りをした肉1
  kg、\protect\hyperlink{panade-c}{パナードC} 400 g、卵白5 個分、塩15
  g、白こしょう2 g、ナツメグ1 g、クレーム・ドゥーブル \footnote{乳酸発酵させた濃い生クリーム。フランスの生クリームについては\protect\hyperlink{sauce-supreme}{ソー
    ス・シュプレーム}訳注参照。}1\undemi{} L。
\item
  \textbf{作業手順}\ldots{}\ldots{}どんな肉を使う場合でも、卵白を少しずつ加えながらしっ
  かりとすり潰すこと。
\end{itemize}

パナードを加え、すりこ木でしっかり練り、二つの素材がよくよく混ざ
り合うようにする。

目の細かい網で裏漉しし、鍋にファルスを入れる。ヘラで滑らかになるよう混
ぜ、鍋を氷の上に置いて一時間ほど休ませる。

生クリームの\untiers{}量を少しずつ加えながら、のばしていく。最終的に残
りの\deuxtiers{}の生クリームも加えるが、これは先に泡立て器で軽く立てておくこと。

生クリームを全部加えた時点で、ファルスは真っ白で滑らかでしかも、ふんわりとし
た仕上がりにならなくてはいけない。

\hypertarget{ux539fux6ce8-1}{%
\subparagraph{【原注】}\label{ux539fux6ce8-1}}

手に入った生クリームが必ずしも最上級のものでない場合には、パナードC を
用いて\protect\hyperlink{farce-a}{バターを用いたファルス}を作った方がまだいい。

\maeaki

\hypertarget{c.-ux751fux30afux30eaux30fcux30e0ux3092ux7528ux3044ux308bux6ed1ux3089ux304bux306aux30d5ux30a1ux30ebux30b9-ux30d5ux30a1ux30ebux30b9ux30e0ux30b9ux30eaux30fcux30cc}{%
\subsubsection{C. 生クリームを用いる滑らかなファルス /
ファルス・ムスリーヌ}\label{c.-ux751fux30afux30eaux30fcux30e0ux3092ux7528ux3044ux308bux6ed1ux3089ux304bux306aux30d5ux30a1ux30ebux30b9-ux30d5ux30a1ux30ebux30b9ux30e0ux30b9ux30eaux30fcux30cc}}

\hypertarget{farce-c}{%
\paragraph{Farce à la Crème, ou Mousseline}\label{farce-c}}

\index{farce!c@B. --- fine à la crème, ou Mousseline}
\index{garniture!farce!c@C. Farce fine à la crème, ou Mousseline}
\index{mousseline!farce mousseline}
\index{かるにちゆーる@ガルニチュール!ふあるす@ファルス!c@C. 生クリームを用いる滑らかなファルス / ファルス・ムスリーヌ}
\index{ふあるす@ファルス!c@C. 生クリームを用いる滑らかな--- / ---・ムスリーヌ}
\index{むすりーぬ@ムスリーヌ!ふあるす@ファルス・---}

(ムース、ムスリーヌ、ポタージュ用クネルなど)

\begin{itemize}
\item
  \textbf{材料}\ldots{}\ldots{}丁寧に掃除をして筋取りをした肉1
  kg、卵白4個分、クレーム・ エペス\footnote{crème épaisse fraîche
    低温殺菌の後、乳酸醗酵させたとても濃い生クリーム。}1\undemi{}
  L、塩18 g、白こしょう3 g。
\item
  \textbf{作業手順}\ldots{}\ldots{}肉と調味料を鉢に入れて細かくすり潰す。卵白を少量ずつ
  加えていく。目の細かい網で裏漉しする。
\end{itemize}

これをソテー鍋に入れ、ヘラで滑らかになるまで混ぜたら、たっぷりの氷で鍋
を囲むようにして2時間冷やす。

次に、生クリームを少しずつ加えながらファルスをのばしていく。丁寧に練っ
ていくこと。またこの作業は鍋底を常に氷にあてた状態で行なうこと。

\hypertarget{ux539fux6ce8-2}{%
\subparagraph{【原注】}\label{ux539fux6ce8-2}}

\ldots{}\ldots{}

\begin{enumerate}
\def\labelenumi{\arabic{enumi}.}
\item
  上で示した生クリームの分量は平均的な数字だ。ファルスのベースとなっ
  ている素材つまり肉、魚、甲殻類によってそれぞれタンパク質の特性が違
  うのだから、素材に吸収される生クリームの量には多少の違いがでてくる
  わけだ。
\item
  ここで示したファルスの作り方は、滑らかな仕上がりのファルスの典型で
  あって、これを越える繊細さを出せるものはないから、ファルスに出来る
  材料すべて、つまり各種の肉、ジビエ、鶏、魚、甲殻類などに適用してい
  い。
\item
  卵白の量は、ファルスのベースと素材によって調整する必要がある。鶏や
  仔牛肉のようにアルブミンが多く含まれていて新鮮な肉であれば、成獣の
  固くなった肉を使う場合よりも量は少なくて済む。つまり、捌いたばかり
  でまだ温かい若鳥の胸肉を使ってこのファルス・ムスーズを作るのであれ
  ば、卵白は省略してもいい。
\item
  良質の生クリームが入手できる環境にあるなら、他のファルスを作るより
  もこのファルスの方がいいだろう。とりわけ、甲殻類をベースとしたファ
  ルスについては重要なことだ。
\end{enumerate}
\end{recette}

\hypertarget{ux725bux8102ux3092ux52a0ux3048ux3066ux4f5cux308bux4ed4ux725bux8089ux306eux30d5ux30a1ux30ebux30b9-ux30b4ux30c7ux30a3ux30f4ux30a920}{%
\subsection[牛脂を加えて作る仔牛肉のファルス /
ゴディヴォ]{\texorpdfstring{牛脂を加えて作る仔牛肉のファルス /
ゴディヴォ\footnote{ゴディヴォgodiveau
  は16世紀フランソワ・ラブレーの小説『ガルガン
  チュアとパンタグリュエル』の「第三の書」(1546年)が初出。原書の綴
  りはguodiveaulx。これは「アンドゥイエット(のようなもの)」とする
  のが一般的な解釈となっている。また、ラブレーはこれに先立つ1534年
  「ガルガンチュア」(=第一の書)において gaudebillaux という表現を
  用いている。これについては Gaudebillaux: sont grasses tripes de
  coiraux 「ゴドビヨとは、たっぷり肥育した牛のトリップ(胃と腸)のこ
  と」と本文で説明している。これらを敷衍すると、ゴディヴォはもともと
  牛などの胃や腸を刻んで詰めた腸詰すなわちアンドゥイエットのことだっ
  た、と考えたくなっても不思議はない。しかし、たとえ16世紀のラブレー
  における guodiveaulx = godiveau が当時アンドゥイエットと呼ばれるも
  のとほぼ同じだったとしても、アンドゥイエット andouilette がアンドゥ
  イユ andouille に縮小辞を付したものであることから、中世のアンドゥ
  イユを確認する必要が出てくる。14世紀末に成立したとされる『ル・メナ
  ジエ・ド・パリ』においてアンドゥイユは確かに「細かく刻んだ胃や腸を、
  腸詰にする」という説明がまず出てくるが、その他に、牛の第1胃だけを
  詰めるもの、豚のコトレットを切り出した端肉を材料にするもの、胸腺肉
  やレバーを掃除した残りの肉を材料にするもの、が挙げられている
  (t.2,p.127)。これに従うなら、中世におけるアンドゥイユとは素材の定
  義があまりはっきりしていなかったもの、言えるだろう。ところが167世
  紀、ピエール・ド・リュヌ『新料理の本』(1660年)において「スペイン
  風アンドゥイエット」というレシピが掲載されている。概要を記すと、仔
  牛肉を細かく刻む。豚背脂少々、香草、卵黄、塩、こしょう、ナツメグ、
  粉にしたシナモンを加える。豚背脂のシートで巻いてアンドゥイエットの
  形状にする。串を刺してローストする。ローストする際に滴り落ちてくる
  肉汁は受け皿で受ける。火が通ったらその肉汁をかける。茹で卵の黄身8〜
  10個分と細かくおろしたパン粉を順につけて、しっかりした衣を作る。提
  供時にレモン汁と羊のジュをかけ、揚げたパセリを添える、というものだ。
  1693年刊マシアロ『宮廷および大ブルジョワ料理の本』では豚のアンドゥ
  イユ、仔牛のアンドゥイユとともに、仔牛のアンドゥイエットというレシ
  ピが掲載されている。最後のものには材料として「細かく刻んだ仔牛肉、
  豚背脂、香草、卵黄、塩、こしょう、ナツメグ、シナモンを加えて作る」
  とある(pp.108-109)。また、1750年に出版された『食品、ワイン、リキュー
  ル事典』において、アンドゥイエットは「細かく刻んだ仔牛肉を楕円形に
  巻いたもの」と定義されている。実際、17、18世紀の料理書に出てくるア
  ンドゥイエットは腸詰であるかどうかは別にしても、仔牛肉を主材料にし
  たものが多い。18世紀ヴァンサン・ラ・シャペル『近代料理』第1巻のア
  ンドゥイエットも細かく刻んだ仔牛肉を豚の腸に詰めて作る。さて、ゴディ
  ヴォに戻ると、17世紀、1653年刊の『フランスのパティスリの本』(ラ・
  ヴァレーヌが著者だと言われている)にはFaire un pasté de gaudiueau
  「ゴディヴォのパテの作り方」という節があり、仔牛腿肉あるいは他の肉
  と脂身を細かく刻んだもの、をパテ(≒パイ包み焼き)に入れると書いて
  ある。つまりここでは「仔牛腿肉」の使用が前提となっている。仮にラブ
  レーの guodiveaulx がこんにち我々のよく知る、牛などの胃や腸を刻ん
  で詰めたアンドゥイエットと同様のものだったとしたら、わずか百年で大
  きな変化を遂げてしまい、逆にこんにちの我々が知るものと大きく違って
  しまったことになってしまう。したがって、これら勘案すれば、ラブレー
  のguodiveaulxもまた仔牛肉を材料にしていたものだった可能性は充分に
  考えられるだろう。 もちろんgaudebillaux という別の巻の名詞との関連
  性は無視出来ないものだが、中世〜ルネサンス期において、食にかかわる
  名詞、概念がしばしば曖昧だったことを考えると、多少のわかりにくさは
  許容せざるを得ない。したがって、本書において仔羊腿肉とケンネ脂を使
  うゴディヴォを「古典的」なファルスとして扱っているのはまことに正鵠
  を射ていると言えよう。}}{牛脂を加えて作る仔牛肉のファルス / ゴディヴォ}}\label{ux725bux8102ux3092ux52a0ux3048ux3066ux4f5cux308bux4ed4ux725bux8089ux306eux30d5ux30a1ux30ebux30b9-ux30b4ux30c7ux30a3ux30f4ux30a920}}

\vspace*{-1.7\zw}

\hypertarget{farce-de-veau-uxe0-la-graisse-de-boeuf-ou-godiveau}{%
\subsection{Farce de Veau à la Graisse de boeuf, ou
Godiveau}\label{farce-de-veau-uxe0-la-graisse-de-boeuf-ou-godiveau}}

\index{farce!veau graisse de boeuf@--- de veau à la graisse de boeuf}
\index{garniture!farce!veau graisse de boeuf@Farce de veau à la graisse de boeuf}
\index{farce!veau glodiveau@Godiveau}
\index{garniture!farce!godiveau@Godiveau}
\index{かるにちゆーる@ガルニチュール!ふあるす@ファルス!きゆうしをくわえてつくるこうしにくのふあるす@牛脂を加えて作る仔牛肉のファルス / ゴディヴォ}
\index{ふあるす@ファルス!きゆうしをくわえてつくるこうしにくのふあるす@牛脂を加えて作る仔牛肉の--- / ゴディヴォ}
\index{かるにちゆーる@ガルニチュール!ふあるす@ファルス!こていうお@ゴディヴォ}
\index{ふあるす@ファルス!こていうお@ゴディヴォ} \index{godiveau}
\index{こていうお@ゴディヴォ}

\begin{recette}
\hypertarget{a.-ux6c37ux3092ux5165ux308cux3066ux4f5cux308bux30b4ux30c7ux30a3ux30f4ux30a9}{%
\subsubsection{A.
氷を入れて作るゴディヴォ}\label{a.-ux6c37ux3092ux5165ux308cux3066ux4f5cux308bux30b4ux30c7ux30a3ux30f4ux30a9}}

\hypertarget{godiveau-mouilluxe9-uxe0-la-glace}{%
\paragraph{Godiveau mouillé à la
glace}\label{godiveau-mouilluxe9-uxe0-la-glace}}

\index{farce@farce!godiveau a@Godiveau A. Godeiveau mouillé à la glace}
\index{ふあるす@ファルス!こていうお@ゴディヴォ!a@A. 氷を入れて作るゴディヴォ}
\index{godiveau@godiveau!a@A. --- mouillé à la glace}
\index{こていうお@ゴディヴォ!a@A. 氷を入れて作る---}

\begin{itemize}
\item
  \textbf{材料}\ldots{}\ldots{}筋をきれいに取り除いた仔牛腿肉1
  kg、\textbf{水気を含んでいない}牛ケンネ脂\footnote{腎臓の周囲を厚く覆っている脂肪。融解温度が低く、上質の牛脂(ヘッ
    ト)の原料にされる。}1.5 kg、全卵8個、塩25 g、白こしょう5
  g、ナツメグ1 g、 透明な氷7〜800 gまたは氷水7〜8 dl。
\item
  \textbf{作業手順}\ldots{}\ldots{}はじめに、仔牛肉とケンネ脂を別々に、細かく刻む。仔牛
  肉はさいの目に切り、調味料と合わせておく。牛脂は細かくして、薄皮は筋
  はきれいに取り除いておく。
\end{itemize}

仔牛肉と牛脂を別々の鉢に入れて、それぞれすり潰す。次にこれらを合わせて
から、完全に混ざり合って一体化するまでよくすり潰し、卵を一個ずつ、すり
潰す作業を止めずに加えていく。

裏漉しして、平皿に\footnote{大きなバット。}広げ、氷の上に置いて翌日まで休ませる。

翌日になったら、再度ファルスをすり潰す。この時、小さく割った氷を少しず
つ加えていき、よく混ぜ合わせる。

ゴディヴォに氷を加え終わったら、必ずテスト\footnote{少量を、沸騰しない程度の温度で火を通し(ポシェ)て様子を見ること。}を行ない、必要に応じて
修正する。固すぎるようなら水を少々加え、柔らかすぎるようなら卵白を少し
加えること。
\end{recette}
\hypertarget{serie-des-appareiles-et-preparations-diverses-pour-garnitures-chaudes}{%
\section{温製ガルニチュールのためのアパレイユなど}\label{serie-des-appareiles-et-preparations-diverses-pour-garnitures-chaudes}}

\frsec{Série des Appareils et Préparations diverses pour Garnitures chaudes}

\index{garniture@garniture!appareils garnitures chaudes@appareils et préparations diverses pour garnitures chaudes}
\index{appareil@appareil!garnitures chaudes@--- et préparations diverses pour garnitures chaudes}
\index{かるにちゆーる@ガルニチュール!あはれいゆおんせい@温製ガルニチュールのためのアパレイユなど}
\index{あはれいゆ@アパレイユ!おんせいかるにちゆーる@温製ガルニチュールのための---など}
\begin{recette}
\hypertarget{appareils-a-cromesquis-et-a-croquettes}{%
\subsubsection{クロメスキとクロケットのアパレイユ}\label{appareils-a-cromesquis-et-a-croquettes}}

\frsub{Appareils à Cromesquis et à Croquettes}

\index{garniture@garniture!appareil@appareil!cromesquis croquettes@appareils à cromesquis et à croqeuttes}
\index{appareil@appareil!cromesquis croquettes@---s à cromesquis et à croquettes}
\index{cromesqui@cromesqui!appareil@appareils à --- et à croquettes}
\index{croquette@croquette!appareil@appareils à cromesquis et à ---}
\index{かるにちゆーる@ガルニチュール!あはれいゆ@アパレイユ!くろめすきとくろけつと@クロケットとクロメスキのアパレイユ}
\index{あはれいゆ@アパレイユ!くろめすきとくろけつと@クロケットとクロメスキの---}
\index{くろめすき@クロメスキ!あはれいゆ@---とクロケットのアパレイユ}
\index{くろけつと@クロケット!あはれいゆ@クロメスキと---のアパレイユ}

⇒ \protect\hyperlink{hors-d-oeuvres-chauds}{温製オードブル}の章を参照。

\hypertarget{appareils-a-pomme-dauphine-duchesse-marquise}{%
\subsubsection{じゃがいものドフィーヌ、デュシェス、マルキーズのアパレイユ}\label{appareils-a-pomme-dauphine-duchesse-marquise}}

\frsub{Appareils à pomme Dauphine, Duchesse et Marquise}

\index{garniture@garniture!appareil@appareil!pomme dauphine@appareils à pomme Dauphine, Duchesse et Marquise}
\index{appareil@appareil!pomme dauphine@---s à pomme Dauphine, Duchesse et Marquise}
\index{dauphin@dauphin(e)!appareil@appareils à pomme ---e}
\index{duc@duc / duchesse!appareil@appareils à pomme duchesse}
\index{marquis@marquis(s)!appareil@appareils à pomme ---e}
\index{かるにちゆーる@ガルニチュール!あはれいゆ@アパレイユ!しやかいものとふいーぬ@じゃがいものドフィーヌ、デュシェス、マルキーズのアパレイユ}
\index{あはれいゆ@アパレイユ!しやかいものとふいーぬと@じゃがいものドフィーヌ、デュシェス、マルキーズの---}
\index{とふいーぬ@ドフィーヌ!あはれいゆ@アパレイユ!しやかいものとふいーぬ@じゃがいもの---、デュシェス、マルキーズのアパレイユ}
\index{とゆしえす@デュシェス!あはれいゆ@アパレイユ!しやかいものてゆしえす@じゃがいものドフィーヌ、---、マルキーズのアパレイユ}
\index{まるきーす@マルキーズ!あはれいゆ@アパレイユ!しやかいものまるきーす@じゃがいものドフィーヌ、デュシェス、---のアパレイユ}

⇒ \protect\hyperlink{legumes}{野菜料理}の章、\protect\hyperlink{pommes-de-terre}{じゃがいも}の項を参照。
\end{recette}
\hypertarget{serie-des-appareiles-et-preparations-diverses-pour-garnitures-froides}{%
\section[冷製ガルニチュール用アパレイユなど]{\texorpdfstring{冷製ガルニチュール用アパレイユなど\footnote{この節は、初版で「冷製料理」の章の冒頭に概説としてまとめられて
  いたものを、第二版の改訂時に、ほぼそのままの内容で現在の位置に移動
  させられている。もちろん順序および内容の加筆も行なわれており、異同
  は少なくない。}}{冷製ガルニチュール用アパレイユなど}}\label{serie-des-appareiles-et-preparations-diverses-pour-garnitures-froides}}

\frsec{Série des Appareils et Préparations diverses pour Garnitures froides}

\index{garniture@garniture!appareils garnitures froides@appareils et préparations diverses pour garnitures froides}
\index{appareil@appareil!garnitures froides@--- et préparations diverses pour garnitures froides}
\index{かるにちゆーる@ガルニチュール!あはれいゆれいせい@冷製ガルニチュールのためのアパレイユなど}
\index{あはれいゆ@アパレイユ!れいせいかるにちゆーる@冷製ガルニチュールのための---など}

\hypertarget{mousses-mousselines-et-souffles-froids}{%
\subsection{冷製のムース、ムスリーヌ、スフレ}\label{mousses-mousselines-et-souffles-froids}}

\frsecb{Mousse, Moussseline, et Soufflé froids}

\index{mousse@mousse!froide@--- froide}
\index{mousseline@mousseline!froide@--- froide}
\index{souffle@soufflé!froid@--- froid}
\index{むーす@ムース!れいせい@冷製の---}
\index{むすりーぬ@ムスリーヌ!れいせい@冷製の---}
\index{すふれ@スフレ!れいせい@冷製の---}

温製の場合でも冷製の場合でも、\ul{ムースとムスリーヌはどちらも同じ材料から作られる}。

ムースとムスリーヌの違いは、温製でも冷製でも、通常は10人分が入る大きな
型に詰めて作るのが\ul{ムース}と呼ばれ、いっぽう、\ul{ムスリーヌ}はスプー
ンで整形したり絞り袋を使ったり、あるいは大きなクネルの形をした専用の型
に入れたりして作るが、基本的に\ul{1つ}で1人分と決まっている。スフレは
小さなスフレ型に詰める。
\begin{recette}
\hypertarget{composition-de-l-appareil-pour-mousses-et-mousseline-froides}{%
\subsubsection{冷製のムースとムスリーヌのアパレイユ}\label{composition-de-l-appareil-pour-mousses-et-mousseline-froides}}

\frsub{Composition de l'Appareil pour Mousses et Mousseline froides}

\index{garniture@garniture!appareils garnitures froides@appareils et préparations diverses pour garnitures froides!appareil mousses mousselines froides@composition de l'appareil pour mousses et mousselines froides}
\index{appareil@appareil!garnitures froides@--- et préparations diverses pour garnitures froides!appareil mousses mousselines froides@composition de l'appareil pour mousses et mousselines froides}
\index{mousse@mousse!froide@froide!composition appareil@Composition de l'appareil pour mousses et mousseline froides}
\index{mousseline@mousseline!froide@froide!composition appareil@Composition de l'appareil pour mousses et mousseline froides}
\index{かるにちゆーる@ガルニチュール!あはれいゆれいせい@冷製ガルニチュールのためのアパレイユなど!れいせいのむーすとむすりーぬのあぱれいゆ@冷製のムースとムスリーヌのアパレイユ}
\index{あはれいゆ@アパレイユ!れいせいかるにちゆーる@冷製ガルニチュールのための---など!れいせいのむーすとむすりーぬのあぱれいゆ@冷製のムースとムスリーヌのアパレイユ}
\index{むーす@ムース!れいせい@冷製!むーすとむすりーぬのあぱれいゆ@ムースとムスリーヌのアパレイユ}
\index{むすりーぬ@ムスリーヌ!れいせい@冷製!むーすとむすりーぬのあぱれいゆ@ムースとムスリーヌのアパレイユ}

\begin{itemize}
\tightlist
\item
  \textbf{材料}\ldots{}\ldots{}主素材のピュレ\footnote{本書では加熱した肉や魚、甲殻類のピュレを作る方法への言及はないが、
    \textbf{本章冒頭にある\protect\hyperlink{farce-mousseline}{ファルス・ムスリーヌ}をそのま
    ま使おうなどと考えてはいけない。ここで説明されている冷製のムース、
    ムスリーヌ、スフレの作り方に加熱の工程がまったく含まれていないのは、
    主素材のピュレが既に加熱済みであることを当然の前提としている}から
    だ。つまりここで材料として示されているピュレは\textbf{すべて加熱済みのも
    のをピュレにしたものだ}と考えなければならない。『料理の手引き』の
    当時はローストするか茹でるなどの加熱後に、鉢に入れてすり潰し、裏漉
    ししてから何らかのソース(ここではヴルテ)を加えて漉さ(固さ)を調
    節するなどしていた。現代ではフードプロセッサーや冷凍粉砕調理機など
    を利用すればより容易に滑らかなピュレを作ることが可能だろう。また、
    第3章ポタージュに\protect\hyperlink{les-purees}{ポタージュ・ピュレ}についての概説が
    あるが、そこではポタージュにすることを前提として「つなぎ」の使用が
    作業のプロセスに組込まれて説明されているために、あくまで参考程度に
    読むのがいいだろう。}1 Lすなわち鶏のピュレ、ジビエ、フォワグラ
  や魚、甲殻類のピュレ。溶かした\protect\hyperlink{gelees-ordinaires}{ジュレ}2\undemi{}
  dl、\protect\hyperlink{veloute}{ヴルテ}4 dl、生クリーム4
  dlはちょうどいい固さに立てて6 dl相当にしておく。
\end{itemize}

素材の特性によって、これらの分量比率は多少変更してもいい。同様に、ある
種のムースを作る際にはジュレまたはヴルテのどちらかしか用いなくてもいい。

\begin{itemize}
\tightlist
\item
  \textbf{作業手順}\ldots{}\ldots{}まずベースとなるピュレを入れたボウルを氷の上に置いて、軽
  く混ぜながら、ジュレとヴルテを加える(どちらかしか使わない場合は使う
  もののみ)。次に泡立てた生クリームを加える。
\end{itemize}

味付けを確認する。これは冷製料理ではとても重要なことだ。いつも気をつけ
て確認し、修正を加えるようにすること。

\hypertarget{nota-composition-de-l-appareil-pour-mousses-et-mousseline-froides}{%
\subparagraph{【原注】}\label{nota-composition-de-l-appareil-pour-mousses-et-mousseline-froides}}

生クリームは五分立てにしておくこと。完全に立ててしまっていたら、ムース
は滑らかさが失なわれてパサついた仕上りになってしまう。

\hypertarget{moulage-des-mousses-froides}{%
\subsubsection{冷製ムースの型詰め}\label{moulage-des-mousses-froides}}

\frsub{Moulage des Mousses froides}

\index{garniture@garniture!appareils garnitures froides@appareils et préparations diverses pour garnitures froides!moulage mousses froides@moulage des mousses froides}
\index{appareil@appareil!garnitures froides@--- et préparations diverses pour garnitures froides!moulage mousses froides@moulage des mousses froides}
\index{mousse@mousse!froide@froide!moulage@moulage des mousses froides}
\index{かるにちゆーる@ガルニチュール!あはれいゆれいせい@冷製ガルニチュールのためのアパレイユなど!れいせいむーすのかたつめ@冷製ムースの型詰め}
\index{あはれいゆ@アパレイユ!れいせいかるにちゆーる@冷製ガルニチュールのための---など!れいせいむーすのかたつめ@冷製ムースの型詰め}
\index{むーす@ムース!れいせい@冷製!むーすのかたつめ@ムースの型詰め}

いまもそうしている料理人は少なくないようだが、かつては、プレーンな型あ
るいは浮き彫り模様の付いた型の中に透明なジュレを流して層をつくってやり\footnote{chemiser
  (シュミゼ)ジュレなどを型の内側に流して薄い層を作ること。}、
ムースの主素材と関連あるものを装飾要素として貼り付けていた。

こんにちでは次の方法がむしろ好ましい。銀製のタンバル型\footnote{timbale
  (タンバル)円筒形の比較的浅い型。野菜料理用の深皿もこの語で呼ぶので注意。}の底面だけに
透明なジュレの薄い層をつくる。型の側面の外側に紙の帯を冷たいバターで貼
り付ける。型の\ruby{縁}{ふち}から2〜3 cmくらい高くなるようにすること。
そうするとスフレのような見た目のムースになる。紙の帯は型の内側に貼り付
けてもいい。この紙の帯は提供直前に、ぬるま湯で濡らしてナイフの刃を使っ
てムースからそっと引き剥してやる。

タンバル型の用意が整ったら、ムースを詰めて冷やす。アイスクリーム用の冷
凍庫に入れるほうがいいだろう。この方法は、小さな銀製のスフレ型に詰めて
やってもいいが、それは冷製のスフレにとっておいたほうがいいだろう。アパ
レイユの構成が同じであるにもかかわらず、冷製ムースと冷製スフレの違いを
はっきりさせることが出来るからだ。

とりわけジビエのムースやフォワグラのムースについては、近代的な料理の提
供方法に合わせて作られた銀製かガラス製の容器を用いてもいい。その場合は、
型の底面だけジュレの層をつくってやり、アパレイユをそのまま流し込めばい
い。表面はパレットナイフなどで丁寧に滑らかにならしてやってから、ムース
を冷やす。その後\footnote{型から出して、ということだろう。}、ムースに直接装飾を施し、ジュレをかけて艶を出させる。

ジビエのムースの場合には、そのジビエの胸肉を冷やして、ムースの周囲に飾
るようにする。

\hypertarget{moulage-des-mousselines-froides}{%
\subsubsection{冷製ムスリーヌの整形}\label{moulage-des-mousselines-froides}}

\frsub{Moulage des Mousselines froides}

\index{garniture@garniture!appareils garnitures froides@appareils et préparations diverses pour garnitures froides!moulage mousselines froides@moulage des mousselines froides}
\index{appareil@appareil!garnitures froides@--- et préparations diverses pour garnitures froides!moulage mousselines froides@moulage des mousselines froides}
\index{mousseline@mousseline!froide@froide!moulage@moulage des mousselines froides}
\index{かるにちゆーる@ガルニチュール!あはれいゆれいせい@冷製ガルニチュールのためのアパレイユなど!れいせいむすりーぬのかたつめ@冷製ムスリーヌの型詰め}
\index{あはれいゆ@アパレイユ!れいせいかるにちゆーる@冷製ガルニチュールのための---など!れいせいむすりーぬのかたつめ@冷製ムスリーヌの型詰め}
\index{むすりーぬ@ムスリーヌ!れいせい@冷製!むすりーぬのかたつめ@ムスリーヌの型詰め}

冷製ムスリーヌの型詰めには2つの方法がある。たんに、型にジュレの層を作っ
てやるか、ソース・ショフロワの層を作ってやるかの違いでしかない。どちら
の場合でも、卵形の型に詰めるか、大きなクネルの形状のものにするか、とい
うことになる。

\hypertarget{procede-un-moulage-des-mousselines-froides}{%
\subparagraph{方法1\ldots{}\ldots{}}\label{procede-un-moulage-des-mousselines-froides}}

型の内側に透明なジュレを流して薄い層を作ってやる\footnote{chemiser
  (シュミゼ)。}。その上にアパレイ
ユを張るように塗り、アパレイユのベースとなっている素材とおなじもの---
鶏、ジビエ、甲殻類の身など、とトリュフ---で構成された\protect\hyperlink{salpicons-divers}{サルピコ
ン}を盛り込む。その上からアパレイユを塗って覆い、パ
レットナイフなどを使ってドーム形に滑らかにならす。冷蔵庫に入れて冷し固
める。

\hypertarget{procede-deux-moulage-des-mousselines-froides}{%
\subparagraph{方法2\ldots{}\ldots{}}\label{procede-deux-moulage-des-mousselines-froides}}

型の内側にアパレイユを詰め、さらにサルピコンをその内側に射込む。アパレ
イユで覆って、冷し固める。

型から外す。ムスリーヌのアパレイユの素材と関連性のある\protect\hyperlink{sauce-chaud-froid-ordinaire}{ソース・ショフ
ロワ}を表面を覆うように塗る\footnote{napper
  (ナペ)。覆いかける(ように塗る)こと。}。トリュフおよびそ
の他の素材(これもムスリーヌと関連性があること)を装飾用に細工したもの
を飾り付ける。装飾が剥れないように、上からジュレを塗って艶を出させる。

銀製またはガラス製の深皿の底に透明なジュレの層を作り、その上にムスリー
ヌを並べる。再度ジュレを上からかけてやり、冷蔵庫に入れて提供するまで保
管しておく。

\hypertarget{souffles-froids}{%
\subsubsection{冷製スフレ}\label{souffles-froids}}

\frsub{Soufflés froids}

\index{garniture@garniture!appareils garnitures froides@appareils et préparations diverses pour garnitures froides!souffles froids@soufflés froids}
\index{appareil@appareil!garnitures froides@--- et préparations diverses pour garnitures froides!souffles froids@soufflés froids}
\index{souffle@soufflé!froid@---s froids}
\index{かるにちゆーる@ガルニチュール!あはれいゆれいせい@冷製ガルニチュールのためのアパレイユなど!れいせいすふれ@冷製スフレ}
\index{あはれいゆ@アパレイユ!れいせいかるにちゆーる@冷製ガルニチュールのための---など!れいせいすふれ@冷製スフレ}
\index{すふれ@スフレ!れいせい@冷製---}

冷製スフレはムースそのものに他ならない。だから構成はまったく同じだ。た
だ、先に見たようにスフレが10人分\footnote{1 service
  (アンセルヴィス)、格式のある宴席料理などを作る際の単位。基本は10人分。}を確保できるだけの大きな型に詰める
のに対して、スフレはそもそも、小さなスフレ型に入れてひとり1つ宛で作る
ものだ。

アパレイユを型に詰める方法はムースの場合と同様、つまり、スフレ型の底に
ジュレの層を敷いてその上にアパレイユを盛り、型の縁より高くなるように周
囲に巻いた紙の帯を利用して縁より高くアパレイユを盛る。そうすると、冷や
し固めた後で紙の帯を取り除けば、まるで温製のスフレのように見えることに
なる。

\hypertarget{nota-souffles-froids}{%
\subparagraph{【原注】}\label{nota-souffles-froids}}

ここまで述べた3種の作り方の基礎はおなじだから、ポイントは次のようにまとめられる。

\begin{enumerate}
\def\labelenumi{\arabic{enumi}.}
\item
  ムースは「スフレ」の名称で供してもいいものだが、混同されるのを避けるために「ムース」の名称で約10人分をひとつの型に入れて作る。
\item
  ムスリーヌはサルピコンを射込んだものであってもそうでなくても、大きなクネルであって、ひとりあたり1つにする。
\item
  スフレは小さなムースであって、スフレ型あるいは似たような型に詰めて、これもひとりあたり1つとする。
\end{enumerate}
\end{recette}
\hypertarget{aspics}{%
\subsection{アスピック}\label{aspics}}

\frsecb{Aspics}

\index{garniture@garniture!appareils garnitures froides@appareils et préparations diverses pour garnitures froides!aspics@aspics}
\index{appareil@appareil!garnitures froides@--- et préparations diverses pour garnitures froides!aspics@aspics}
\index{aspics@aspics (généralité)}
\index{かるにちゆーる@ガルニチュール!あはれいゆれいせい@冷製ガルニチュールのためのアパレイユなど!あすぴつく@アスピック(概説)}
\index{あはれいゆ@アパレイユ!れいせいかるにちゆーる@冷製ガルニチュールのための---など!あすぴつく@アスピック(概説)}
\index{あすぴつく@アスピック(概説)}

アスピックを作る際に、肝に銘じておくべき第一のポイントは、どんなアスピッ
クでも、ジュレがジューシー\footnote{原文succulent(スュキュロン)はsuc(スュック=肉汁)から派生した
  形容詞で、もともとは「汁気の多い」の意味だったが、そこから転じて
  「美味な、滋味に富んだ」の意味で一般的に用いられている。ここでは、
  両方のニュアンスで表現されていると解釈できる。}で美味しく、完全に透き通ったもので、ちょうど
いい加減に固まっていなければならないことである。

アスピックを作る際には、昔もそうだったが現代でも、中央に穴の空いたアス
ピック型\footnote{moule à douille
  (ムーラドゥイユ)サヴァラン型のような中央に穴 が空いた型。
  現代では「アスピック型」というと楕円形で中央に穴のな
  いものを指すことが多いが、それとは異なる。クグロフ型のようなものを
  イメージするとわかりやすいだろう。19世紀、アスピックには高さのある
  型が多く用いられたようである。なお、現代では一般にサヴァラン型とい
  うと、型の高さや穴の大きさ等さまざまなタイプのものをまとめて指すこ
  とになるので注意。高さのない(低い)、中央の穴が大きな型について、
  エスコフィエはボルデュール型 moule à bordure (ムーラボルデュール)
  と呼んで区別している}でプレーンなもの、波模様等の装飾のあるものが用いられている。

ボルデュール型\footnote{moule à bordure
  (ムーラボルデュール)蛇の目形に料理の縁り飾り
  を作るための、やや丈が低く中央の穴が大きいリング型。}も使われることがあるが、一般的に、アスピックの中心にガル
ニチュールを盛り込む場合のみである。

アスピックを型に入れる時には、まず、型の底と周囲に装飾をする。

そのために、型は砕いた氷の中に入れてよく冷やしておく。やや固まりかけた
ジュレ少量を流し入れ、型を氷の上で転がしながらジュレを周囲に貼り付かせ
る(シュミゼ)。次に、装飾するパーツを、固まらない程度に冷たいジュレに浸
してからすぐに貼り付ける。装飾については料理人のセンスとアイデア次第な
ので、ここで明確に述べておくべきことはほとんどない。ひとつだけ言えるの
は、常に正確な作業を期して、型からアスピックを出したときに装飾がはっき
りと見えるようにすべき、ということだけだ。

装飾に用いる素材はアスピックの主素材と関連性のあるものでなくてはならな
い。一般的には、トリュフ、ポシェした卵白、コルニション\footnote{cornichon
  主としてピクルスにする小型のきゅうり、およびそのピク
  ルスのこと。日本では、ハンバーガーによく用いられているドイツ系のピ
  クルス用品種であるガーキンス(英 gherkins 独 Einlegegurken)と混同さ
  れることがあるが、コルニションはより小さなサイズで収穫し、フレッシュ
  な状態では「いぼ」が尖っているのが特徴。}、ケイパー、
いろいろな香草の葉先、ラディッシュの薄い輪切り、オマールのコライユ\footnote{胴の背側にあるオレンジ色がかった「内子」。}、
\protect\hyperlink{saumure-liquide-pour-langues}{赤く漬けた舌肉}、等。

アスピックのガルニチュールが種々のエスカロップ\footnote{escalope
  (エスカロップ)筋線維とは垂直方向に、厚さ1〜2 cmに薄
  切りにした仔牛などの肉や魚の薄い切り身。}や長方形に切ったフォワグ
ラ等で、型の大きさから何度も並べなければならない場合、ジュレの層と交互
に重ねて型に入れていく。新しい層を並べる際には先に入れたジュレがある程
度固まってからにする。

アスピックの型入れでは常に、最後のジュレの層を充分な厚みにする。できる
だけ、型を氷に埋めるようにしながらジュレを流し込んでいくが、早く冷やす
ために氷に塩を加えてはいけない。塩を使うとジュレの透明さが損なわれるか
らである。

\noindent\textbf{型から外す方法}\ldots{}\ldots{}型を湯につけてただちに水気を拭い、折り
ただんだナフキンや彫刻した氷のブロック等に、アスピックを裏返してあける。

菱形や正方形に切ったジュレのクルトン\footnote{パンで作るクルトンと同様に、菱形やさいの目に切った冷製料理装飾用
  のジュレもクルトンと呼ぶ。}、またはアシェしたジュレで周囲を飾 る。

\hypertarget{nota-aspics}{%
\subparagraph{【原注】}\label{nota-aspics}}

アスピックを型に入れて作るには、必然的に、ジュレが相当に固いものでなけ
ればならないが、これはまことに具合がよろしくない。というのも、固いジュ
レは口あたりがよくないからだ。だから現代の調理現場では、以下のような方
法を採っている。タンバル型か、氷に嵌め込むようにした銀やガラスあるいは
陶製の深皿の底に予めジュレの層を作って固めておき、その上にアスピックの
素材を並べる。次に、固まりかけのジュレをたっぷり覆いかける。この方法で
は、装飾をしなければならない場合は、アスピックの調理をおこなう前に、主
素材にじかに装飾することになる。

\hypertarget{chauds-froids}{%
\subsection[ショフロワ]{\texorpdfstring{ショフロワ\footnote{chaud-froid
  このchaud(熱い)とfroid(冷たい)の合成語の複数形は、それぞれ
  にsを付ける、chauds-froidsとなる。合成語の複数形はいろいろなパターンがあるので、必要が
  出たらその都度覚えるようにしたほうがいい。}}{ショフロワ}}\label{chauds-froids}}

\frsecb{Chauds-froids}

\index{garniture@garniture!appareils garnitures froides@appareils et préparations diverses pour garnitures froides!chauds-froids@chauds-froids}
\index{appareil@appareil!garnitures froides@--- et préparations diverses pour garnitures froides!chauds-froids@chauds-froids}
\index{chauds-froids@chauds-froids (généralité)}
\index{かるにちゆーる@ガルニチュール!あはれいゆれいせい@冷製ガルニチュールのためのアパレイユなど!しよふろわ@ショフロワ(概説)}
\index{あはれいゆ@アパレイユ!れいせいかるにちゆーる@冷製ガルニチュールのための---など!しよふろわ@ショフロワ(概説)}
\index{しよふろわ@ショフロワ(概説)}

\protect\hyperlink{sauce-chaud-froid-ordinaire}{ソース・ショフロワ}には大抵の場合、切り
分けた素材を浸す。が、時として大きな塊肉全体をソース・ショフロワで覆わ
なくてはならない場合もある。ただ、そういう仕立てにする場合には、別の料
理名となっている。

ショフロワが複数のばらばらのパーツからなる場合には、それらをソース・ショ
フロワに漬けたら網の上に並べておく。ソースが冷えたら、それぞれのパーツ
に装飾をし、ジュレを覆いかけて艶を出してやる。さらに盛り付けの際にはみ
出す余分なソースについてはきれいに取り除いておくこと。

大きな塊肉の場合は、よく冷えてはいるけれどまだ流動性のある状態のソース・
ショフロワを一気に塗りつけて、その後に装飾をし、ジュレを塗って艶出しす
ること。

切り分けた素材からなるショフロワの盛り付けは、\protect\hyperlink{fonds-de-plats}{皿の上の
台}の上に盛り付けてもいいし、縁飾りの内側に、パンまた
は米、セモリナ粉で作った台を置いてその上に盛り付けてもいい。あるいは、
銀製か陶製、ガラス製の深皿に盛り付けてもいい。

大きな塊肉のショフロワの場合、皿の上の台にのせてもいいし、あるいは、氷
のブロックに料理が嵌まるようにブロックを削ってからそこに盛り付けるのも
いい。

ショフロワ仕立ての鶏やジビエについては、正確に切り分けて\footnote{基本的に鶏および鳥類のジビエの可食部は胸肉のみとされていたこと
  に留意。}皮は剥いでおく
こと。手羽や下腿肉は使わないので、別の用途に取り置いておくといい。

細かく切った素材のショフロワ仕立ての場合、添えてやるマッシュルームや雄
鶏のとさかとロニョン\footnote{rognon
  (ロニョン)牛、羊などの場合は腎臓だが、雄鶏の場合は精巣
  のこと。高級食材として珍重された。}にもソース・ショフロワを塗ってやること。トリュ
フはただジュレをかけて艶を出すだけでいい。
%\newpage
\hypertarget{garnitures-recettes}{%
\subsection{ガルニチュールのレシピ}\label{garnitures-recettes}}

\frsecb{Garnitures}

\begin{center}
\medlarge(ここで示す分量はすべて仕上がり10人分)
\end{center}
\normalsize
\begin{recette}
\hypertarget{garniture-algerienne}{%
\subsubsection{ガルニチュール・アルジェリア風}\label{garniture-algerienne}}

\frsub{Garniture à l'Algérienne}

\index{garniture@garniture!algerienne@--- à l'Algérienne}
\index{algerien@algérien(nne)!garuniture à l'---ne}
\index{かるにちゆーる@ガルニチュール!あるしえりあふう@---・アルジェリア風}
\index{あるしえりあふう@アルジェリア風!かるにちゆーる@ガルニチュール・---}

(牛、羊の塊肉\footnote{原文 Pour les pièce de boucherie
  より正確に訳すなら、「肉屋
  (boucherie)が伝統的に扱かってきた、白身肉を除く畜産精肉、具体的には牛、羊(馬も含まれる)の塊肉であり、牛の場合は基本的にランプ、イチボに相当する部位、羊の場合は鞍下肉から腿上部にかけての部位」を塊のまま調理したものを意味する。フランス語のpièceには「小さい細切れ」の意味もあるのだが、長い間の習慣として、pièce
  de boeuf は牛の大きな塊肉を意味する用語として一般化している。}の料理に添える)

\begin{itemize}
\item
  ワインの栓の形にしたさつまいもの\protect\hyperlink{croquettes}{クロケット}10個
\item
  小さなトマト10個は中をくり抜いて味付けをし、植物油少々で弱火で蒸し煮する
\item
  ソース\ldots{}\ldots{}薄く仕上げた\protect\hyperlink{sauce-tomate}{トマトソース}に、グリルして皮を剥き、細かい千切りにしたポワヴロン\footnote{いわゆる青果としてのパプリカ。}を加える
\end{itemize}

\hypertarget{garniture-alsacienne}{%
\subsubsection{ガルニチュール・アルザス風}\label{garniture-alsacienne}}

\frsub{Garniture à l'Alsacienne}

\index{garniture@garniture!alsacienne@--- à l'Alsacienne}
\index{alsacien@alsacien(ne)!garuniture à l'---ne}
\index{かるにちゆーる@ガルニチュール!あるさすふう@---・アルザス風}
\index{あるさすふう@アルザス風!かるにちゆーる@ガルニチュール・---}

(牛、羊の塊肉、牛フィレ、トゥルヌドに添える)

\begin{itemize}
\item
  ブレゼ\footnote{\protect\hyperlink{chou-braise}{キャベツのブレゼ}を参考にすること。}したシュークルート\footnote{生食出来ないくらい固くて大きな専用品種であるキャベツを千切りにして香辛料などとともに塩蔵、醗酵さたもの。ドイツのザワークラウトが原型だが、歴史的にフランスとドイツで領土の取り合いとなったアルザス地方で独自に発展した。温めたシュークルートにソーセージなどの豚肉加工品を添えたchoucoûte
    garnie(シュークルートガルニ)はアルザスの名物料理のひとつ。}を詰めてハムの脂身のないところを円く切ってのせたタルトレット10個
\item
  ソース\ldots{}\ldots{}\protect\hyperlink{jus-de-veau-lie}{とろみを付けた仔牛のジュ}
\end{itemize}

\hypertarget{garniture-americaine}{%
\subsubsection[ガルニチュール・アメリケーヌ]{\texorpdfstring{ガルニチュール・アメリケーヌ\footnote{\protect\hyperlink{sauce-americaine}{ソース・アメリケーヌ}も参照されたい。}}{ガルニチュール・アメリケーヌ}}\label{garniture-americaine}}

\frsub{Garniture à l'Américaine}

\index{garniture@garniture!americaine@--- à l'Américaine}
\index{americain@américain(e)!garuniture à l'---e}
\index{かるにちゆーる@ガルニチュール!あめりけーぬ@---・アメリケーヌ}
\index{あめりかん@アメリカン/アメリケーヌ!かるにちゆーる@ガルニチュール・アメリケーヌ}

(魚料理に添える)

\begin{itemize}
\item
  このガルニチュールは必ず、\protect\hyperlink{homard-americaine}{オマール・アメリケーヌ}の方法で調理した尾の身をやや斜めに1
  cm程度の薄切り\footnote{escalope
    (エスカロップ)肉などを筋線維と直角に、丸くスライスしたもの。}にして供する
\item
  ソース\ldots{}\ldots{}オマール・アメリケーヌのソース
\end{itemize}

\hypertarget{garniture-andalouse}{%
\subsubsection[ガルニチュール・アンダルシア風]{\texorpdfstring{ガルニチュール・アンダルシア風\footnote{アンダルシア風、つまりスペイン風といいながら、ギリシャ風ライスを使うという点からも、料理名に付けられた地名がしばしば不確かで大雑把な理由さえないことが多いことが理解されよう。}}{ガルニチュール・アンダルシア風}}\label{garniture-andalouse}}

\frsub{Garniture à l'Andalouse}

\index{garniture@garniture!andalouse@--- à l'Andalouse}
\index{andalou@andalou(se)!garuniture à l'---se}
\index{かるにちゆーる@ガルニチュール!あんたるしあふう@---・アンダルシア風}
\index{あんたるしあふう@アンダルシア風!かるにちゆーる@ガルニチュール・---}

(牛、羊の塊肉料理や鶏料理に添える)

\begin{itemize}
\item
  中位の大きさのポワヴロン10個をグリル焼きして中をくり抜き、\protect\hyperlink{riz-grecque}{ギリシャ風ライス}を詰める
\item
  なす\footnote{フランスで伝統的なタイプのなすはヘタが緑色で、風味や調理特性はいわゆる米なすに近いが、形状は比較的細長い。直径4〜6
    cm、長さ25 cmくらいのものが多い。}を4
  cmの厚さの輪切りにして面取りをし、中に窪みをつくって油で揚げ、提供直前に油で炒めたトマトをのせる
\item
  ソース\ldots{}\ldots{}\protect\hyperlink{jus-de-veau-lie}{とろみを付けたジュ}
\end{itemize}

\hypertarget{garniture-arlesienne}{%
\subsubsection[ガルニチュール・アルル風]{\texorpdfstring{ガルニチュール・アルル風\footnote{南フランスの都市
  Arles
  (アルル)の形容詞および名詞形。名詞の場合は「アルルの人」の意味になる。アルルはオランダ出身の印象派〜ポスト印象派の画家フィンセント・ファン・ゴッホ
  Vincent van Gogh
  (フランス語では昔からヴァンソンヴァンゴーグと呼ぶ習慣が付いてしまっており、現代フランス語の原語発音尊重の風潮にもかかわらず、そのように発音されることは多いようだ)が1888年から1889年までアトリエを構え、「ひまわり」など多くの傑作を描いた。有名な、自分の耳を切り落すという「事件」を起こしたのもアルルでのことだ。この時期の作品のひとつに、「アルルの女(ジヌー夫人)」と呼ばれる一連のものがある。モデルはアルルのカフェの経営者だといわれている。もっとも、フランスにおいて画家としてのゴッホおよび彼の作品は生前はほとんど評価されることがなく、生前に売れた絵は1枚だけだったとさえいわれている。このレシピは初版つまり1903年から収められているため、ゴッホの絵との関連はほぼないと考えていいだろう。むしろ、小説化アルフォンス・ドーデ原作を戯曲化してジョルジュ・ビゼーが劇音楽を付けた『アルルの女』(1872年初演、
  1878年再演)との関連があると見るのがいいだろう。この作品は初演時点ではあまり好評ではなかったが、再演で大ヒットとなった。\protect\hyperlink{sauce-bohemienne}{ソース・ボヘミアの娘}のように、人気のある劇やオペラのタイトルを料理名につけて、その人気にあやかろうという風潮が19世紀後半には比較的多かった。そのため、トマトとなすという南フランスを思わせる食材を使ってはいてもアルルという土地に何の関係もないと思われる、内容的にも凡庸なこのガルニチュールに、当時の人気作品の名をつけて、いかにも流行のものであるかのように供したのが定着した、と考えることも可能だろう。その場合は「\ul{ガルニチュール・アルルの\\女}」と訳すべきかも知れない。なお、ビゼーが最初に作曲したのは27曲からなる舞台音楽であって、独立した音楽作品でもなければ、オペラでもなかったが、そのなかから数曲を選んで編曲し(あるいは作曲しなおし)、『アルルの女 組曲』としてこんにち広く知られている。第1組曲と第2組曲があり、前者はビゼー自身によるオーケストラ用編曲。後者はビゼーの死後1879年に友人エルネスト・ギローが完成させた。第1組曲の「メヌエット」や第2
  組曲の「ファランドール」など、曲名は知らずとも、メロディーを聴いたことのある読者も少なくないとと思われる。}}{ガルニチュール・アルル風}}\label{garniture-arlesienne}}

\frsub{Garniture à l'Arlésienne}

\index{garniture@garniture!arlesienne@--- à l'Arlésienne}
\index{arlesien@arlésien(ne)!garuniture à l'---ne}
\index{かるにちゆーる@ガルニチュール!あるるふう@---・アルル風}
\index{あるるふう@アルル風!かるにちゆーる@ガルニチュール・---}

(トゥルヌドやノワゼットの料理に添える)

\begin{itemize}
\item
  なす\footnote{なす、トマト、玉ねぎの分量は記されていないので適宜判断すること。}は1
  cm程の厚さにスライスして塩こしょうをし、小麦粉をまぶして油で揚げる
\item
  トマト皮を剥いてスライスし、バターでソテーする
\item
  玉ねぎは輪切りにして指輪のようにばらばらにし、小麦粉をまぶして油で揚げ、花束のように盛る
\item
  ソース\ldots{}\ldots{}トマト風味の\protect\hyperlink{sauce-demi-glace}{ソース・ドゥミグラス}
\end{itemize}

\hypertarget{garniture-banquiere}{%
\subsubsection[ガルニチュール・銀行家夫人風]{\texorpdfstring{ガルニチュール・銀行家夫人風\footnote{原文の
  à la Banquière をここでは文字通り訳した。料理名において {[}à la +
  形容詞の女性形{]}は通常、à la manière/façon
  〜のmanièreもしくはfaçonが省力されたものと考えられている。これらmanière,
  façon
  いずれも女性名詞であるために、この後に付ける形容詞も女性形となる。ところが「〜風」」「〜を記念して/〜を称揚して」の意味で{[}à
  la + (固有)名詞{]}という用法もある。これは à la manière de + 名詞、の
  manière
  deが省略されたものと考える。Banquier(ボンキエ)は「銀行家」を意味する名詞であり、女性の場合はbanquièreとなり、女性銀行家あるいは銀行家夫人ということになる。そのため、従来は「銀行家風」と訳されていたが、あえて文法の原則に忠実に「銀行家夫人風」を訳した。さて、この料理名だが、日仏料理協会編『フランス 食の事典』(白水社、2000
  年)には「産業革命に伴う産業の隆盛を支えた銀行は、現代にいたるまで資本主義社会の根幹をなすもので、その経営者は19世紀において金持ちの代名詞ともなった。当時、「銀行家風」は王風、王妃風にかわる新しい表現だった(pp.162-163」と説明されている。ところが、料理書においてこのà
  la
  Banquièreという表現は1856年のデュボワ、ベルナール共著『古典料理』以前には見つからない。しかも、「冷製料理用ガルニチュール・銀行家夫人風」Garniture
  à la banquière, pour froid (t.1,
  p.259)および「若鶏のガランティーヌ・銀行家夫人風」Galantine de poulet
  à la banquière (t.2,
  p.40)の2つでのみ料理名に使われているのみ。ガルニチュールの概要は、オマール2尾の身をやや斜めの円形(エスカロップ)にスライスする。これをひとつずつ別々の陶製の器に入れ、小さなアーティチョークの基底部を茹でたもの、大きな黒または白トリュフのスライス、マッシュルームのスライス、コルニションのスライスを盛り込み、塩、こしょう、植物油、パセリとエストラゴンのみじん切りで味付けし、銘々に供する、というもの。本書のガルニチュールと温製、冷製の違いはあっても、同じ名称とは思い難いくらい異なった内容。その前後および以前については、毎年のように版を重ねながら増補されたために料理の流行、変遷を見るのに非常に便利なヴィアールにもオドにも収録されておらず、グフェ『料理の本』(1867年)にも見あたらない。本書よりやや時代が下って、
  1838年の『ラルース・ガストロノミック』初版の「ガルニチュール・銀行家夫人風」は「鶏、仔牛胸腺肉(リドヴォー)の料理、ヴォロヴァン用。クネル、マッシュルーム、トリュフのスライス、ソース・バンキエール
  (p.136)」と定義されている。ソース・バンキエールsauce
  banquièreについては「卵料理、鶏料理、牛や羊の副生物(リドヴォーなど)、ヴォロヴァン用。ソース・シュプレーム2
  dLにマデイラ酒 \(\frac{1}{2}\)
  dLを加え、布で漉す。トリュフのみじん切り大さじ2杯を加えて仕上げる(p.959)」とある。
  2007年版の『ラルース・ガストロノミック』でもほぼ同様の内容だが、ソース・バンキエールのレシピはこの版では欠落している。また、20世紀についても、1950年に刊行されたレシピ集『フランス料理技法』(Flammarion)
  にソース・バンキエールのレシピは見られるが(p.147)、これはモンタニェの『料理大全』(1929年)からの引用であり、ガルニチュール・バンキエールについては何も出ていない。1952年のペラプラ『近代料理技術』にも、
  1953年のキュルノンスキー編『フランスの料理とワイン』にもこれらへの言及なない。ところが2018年現在、インターネットで検索するとpoularde
  à la banquière
  「肥鶏 女銀行家風」のような、ここで見てきたものとはかなり内容の違うレシピが見つかる。「銀行家風」にしろ「女銀行家風」「銀行家夫人風」にしろ、銀行家という語には肯定的な「富の象徴」というイメージがあると同時に、「\ruby{吝嗇} {りんしょく}家」あるいは「カネ貸し」場合によっては「官僚主義的」のようなマイナスイメージが伴なわれ得ることもまた事実だろうし、銀行家が出席している宴席で「銀行家風」の料理を出す場合にはいろいろな誤解やトラブルの原因となる可能性さえあるかも知れない。このことから、『ラルース・ガストロノミック』が初版から2007年版までほぼ記述を変えなかった、つまり誰もこの名称のガルニチュールに手を加えなかった、ということの証左ともなろう。}}{ガルニチュール・銀行家夫人風}}\label{garniture-banquiere}}

\frsub{Garniture à la Banquière}

\index{garniture@garniture!banauiere@--- à la Banquière}
\index{banquier@banquier(ère)!garuniture à la Banquière}
\index{かるにちゆーる@ガルニチュール!きんこうかふしんふう@---・銀行家夫人風}
\index{きんこうかふしんふう@銀行家夫人風!かるにちゆーる@ガルニチュール・---}

(肥鶏の料理に添える)

\begin{itemize}
\item
  ひばり\footnote{mauviette
    (モヴィエット)、ひばりの食材としての名称。生物としてはalouette(アルエット)と呼ぶ。なお、オルレアネ地方の郷土料理に、
    pithiviers de mauviettes
    という、脳と鶏のファルスを詰めたひばりを折込みパイ生地で包んで焼いた料理があるが、pithiviers(ピティヴィエ)とだけ言う場合は、バターと砂糖、アーモンドパウダーなどを折込みパイ生地で包んで上部を渦巻模様に装飾したオルレアネ地方発祥の菓子を指すので注意。}10羽を背側から開いて骨をすべて取り除き\footnote{désosser
    (デゾセ)。日本の調理現場でも比較的よく使われる用語。この語に含まれるosは「骨」のこと、déは「反対、除去」などを意味する接頭辞、erは動詞であることを示す語尾。したがって、文字どおり「骨を取り除く」の意になる。}、\protect\hyperlink{farce-gratin-c}{ファルス・グラタン}を詰めて、表面を色よく焼き、カスロールで火を通す\footnote{en
    casserole
    (オンカスロール)カスロール仕立てと解釈も可能。\protect\hyperlink{sauce-smitane}{ソース・スミターヌ}訳注参照。}
\item
  \protect\hyperlink{farce-b}{鶏のファルス}で小さなクネル10個
\item
  トリュフのスライス10枚
\item
  ソース\ldots{}\ldots{}トリュフエッセンスを加えた\protect\hyperlink{sauce-demi-glace}{ソース・ドゥミグラス}
\end{itemize}

\hypertarget{garniture-berrichonne}{%
\subsubsection[ガルニチュール・ベリー風]{\texorpdfstring{ガルニチュール・ベリー風\footnote{berrichon(ne)(ベリション/ベリショーヌ)
  はフランス中央部にある地方名 Berry の形容詞。ここでは女性形
  berrichonne(ベリショーヌ)となる。山羊乳のチーズで有名。なおフランス史関連の書物ににおいてよく見かける、ベリー公
  duc de Berry
  (デュックドベリー)という公爵位はフランスの王族(つまりその時の王の近縁者)に与えられた爵位で、その後フランス王となった者も多い。このため、いわゆる「世襲」はされてこなかった。また、中世フランスでもっとも豪華で美しい写本のひとつ『ベリー公のいとも豪華なる時祷書』\href{http://gallica.bnf.fr/ark:/12148/btv1b520004510}{\emph{Les
  Très Riches Heures du Duc de
  Berry}}(14世紀)は当時のベリー公ジャン1世が作成させたもので、美術史的にも重要。}}{ガルニチュール・ベリー風}}\label{garniture-berrichonne}}

\frsub{Garniture à la Berrreichonne}

\index{garniture@garniture!berrichonne@--- à la Berrichonne}
\index{berrichon@berrichon(ne)!garuniture à la Berrichonne}
\index{かるにちゆーる@ガルニチュール!へりーふう@---・ベリー風}
\index{へりーふう@ベリー風!かるにちゆーる@ガルニチュール・---}

(牛、羊肉の大がかりな料理\footnote{ルルヴェ relevé
  のこと。\protect\hyperlink{releve}{第二版序文訳注}参照。}に添える)

\begin{itemize}
\item
  卵の大きさにした\protect\hyperlink{chou-braise}{サヴォイキャベツのブレゼ}20個
\item
  キャベツとともに火を通した塩漬け豚バラ肉の小さなスライス10枚
\item
  小玉ねぎ20個と大粒のマロン20個はこのガルニチュールを添える肉の煮汁で火を通す
\item
  ソース\ldots{}\ldots{}アロールート\footnote{Allow-root
    南米産クズウコンを原料とした良質のでんぷん。現代の日本ではコーンスターチで代用することがほとんど。}でとろみを付けた、ブレゼの煮汁
\end{itemize}

\hypertarget{garniture-berny}{%
\subsubsection[ガルニチュール・ベルニ]{\texorpdfstring{ガルニチュール・ベルニ\footnote{ピエール・ド・ベルニPierre
  de Bernis
  (1715〜1794)のこと。なぜか料理名としてはBernyの綴りが一般的だが、個人名なのでもちろん誤り。
  29才でアカデミーフランセーズに入った俊才。ポンパドゥール夫人の庇護のもとルイ15世からも重用された。駐ヴェネツィア大使として食卓外交を展開したが、フランス革命後、ローマで客死した。}}{ガルニチュール・ベルニ}}\label{garniture-berny}}

\frsub{Garniture à la Berny}

\index{garniture@garniture!berny@--- à la Berny}
\index{Berny@Berny (Bernis)!garuniture à la Berny}
\index{かるにちゆーる@ガルニチュール!へるに@---・ベルニ}
\index{へるに@ベルニ!かるにちゆーる@ガルニチュール・---}

(ジビエおよびマリネした牛、羊肉料理\footnote{シュヴルイユ仕立てのこと。\protect\hyperlink{sauce-porvrade}{ソース・ポワヴラード}および\protect\hyperlink{marinade-crue-pour-viandes-de-boucherie-ou-venaison}{マリナード}参照。}に添える)

\begin{itemize}
\item
  ワインの栓の形にしたじゃがいものクロケット・ベルニ\footnote{本書の温製オードブルの節に「クロケット・ベルニ」は掲載されていない。野菜料理の章にある「\protect\hyperlink{pommes-de-terre-berny}{じゃがいも・ベルニ}」をアパレイユとしてクロケットを作ることになる。}10個
\item
  空焼きしたタルトレット10個にバターを加えたマロンのピュレをドーム状に詰め、バターで軽くソテーして艶を出させたトリュフのスライスをタルトレットに1枚ずつのせる
\item
  ソース\ldots{}\ldots{}軽く仕上げた\protect\hyperlink{sauce-poivrade}{ソース・ポワヴラード}。
\end{itemize}

\hypertarget{garniture-bezontinne}{%
\subsubsection[ガルニチュール・ブザンソン風]{\texorpdfstring{ガルニチュール・ブザンソン風\footnote{Besonçon
  (ブゾンソン)フランス東部、ブルゴーニュ=フランシュ=コンテ圏の都市。形容詞は通常bisontin(e)(ビゾンタン/ビゾンティーヌ)だが、本書のようにbizontin(e)と綴ることもある。なお、1980年代に画期的といわれたフランス語教材\emph{C'est
  le
  printemps}の第1課においてはじめて出てくる地名がブザンソンだった。この教材は会話例のリアリティや題材としてdocuments
  authentiques(ドキュモンオトンティック=現実にあるドキュメントすなわち言語を用いたさまざまな書類、看板、広告など)を積極的に採用したこととともに、アプレ68(フランスの学生運動および現代思想における転換期のひとつとなった1968年の「五月革命」以後に多方面において展開された時代特有の雰囲気)が強く表われているのが特徴だった。同時期のフランス語教材の傑作とされる(やや保守的な傾向の)通称「カペル」\emph{Le
  français en direct}
  と並び、フランス語教育・教授法において現在のEUおよびフランスで定められ運用されている「外国語としての言語コミュニケーション能力」の概念形成の先駆けとなった。アプレ68的なものは食文化、料理の世界においても、ゴ\&ミヨの批評と店の格付けにおける、既存のミシュランのガイドブックのオルタナティヴとしての方向性、ヌーヴェルキュイジーヌ宣言などによく表われている。}}{ガルニチュール・ブザンソン風}}\label{garniture-bezontinne}}

\frsub{Garniture à la Bizontine}

\index{garniture@garniture!bizontinne@--- à la Bizontine}
\index{bizontin@bizontin(e) ⇒ bisontin(e)!garuniture à la ---e}
\index{かるにちゆーる@ガルニチュール!ふさんそんふう@---・ブザンソン風}
\index{ふさんそん@ブザンソン!かるにちゆーる@ガルニチュール・---風}

(牛、羊の塊肉料理およびトゥルヌドに添える)

\begin{itemize}
\item
  \protect\hyperlink{croustade-en-pomme-duchesse}{クルスタード・ポム・デュシェス}\footnote{\protect\hyperlink{pomme-de-terre-duchesse}{ポム・デュシェス}をバターを塗ったダリオル型(小さな円筒形の型)をに詰めて整形してからイギリス式パン粉衣を付けて油で揚げ、中をくり抜いてケースにする。詳細は温製オードブルの節参照。}10個は提供直前にドリュール\footnote{色艶よく焼き上げるために卵黄を溶いたもの、あるいは卵黄に水を加えて溶いたものをdorure(ドリュール)と呼び、それを塗ることをdorer
    (ドレ)という動詞で表現する。}を塗り、オーブンに入れて色よく焼く。生クリームを加えたカリフラワーのピュレを詰めてクルスタードの中に絞り袋を使って詰める
\item
  半割りにした\protect\hyperlink{laitues-farcies-pour-garniture}{ガルニチュール用レチュのファルシ}10個
\item
  ソース\ldots{}\ldots{}バターを加えて仕上げた\protect\hyperlink{jus-de-veau-lie}{とろみを付けたジュ}
\end{itemize}

\hypertarget{garniture-boulangere}{%
\subsubsection[ガルニチュール・ブランジェール]{\texorpdfstring{ガルニチュール・ブランジェール\footnote{boulanger/boulangère
  は「パン屋、パン職人」の意。}}{ガルニチュール・ブランジェール}}\label{garniture-boulangere}}

\frsub{Garniture à la Boulangère}

\index{garniture@garniture!boulangere@--- à la Boulangère}
\index{boulanger@boulanger/boulangère!garuniture à la ---ère}
\index{かるにちゆーる@ガルニチュール!ふらんしえーる@---・ブランジェール}
\index{ふらんしえ@ブランジェ/ブランジェール ⇒ パン屋!かるにちゆーる@ガルニチュール・ブランジェール}
\index{はんや@パン屋 ⇒ ブランジェ/ブランジェール!かるにちゆーる@ガルニチュール・ブランジェール}

(羊、乳呑み仔羊、鶏料理に添える)

\begin{enumerate}
\def\labelenumi{\arabic{enumi}.}
\item
  玉ねぎ250 gは薄切りにし\footnote{émincer (エマンセ)。}て、バターで色よく炒める
\item
  じゃがいも750 gは櫛切りか薄切りにする
\item
  塩15 gとこしょう5 g
\end{enumerate}

\begin{itemize}
\item
  1〜3を混ぜ合わせて、このガルニチュールを添える肉を油を熱したフライパンで表面を焼き固め\footnote{rissoler
    (リソレ)。}とともにオーヴンに入れて、一緒に火を通す
\item
  鶏の場合は、じゃがいもはオリーブ形に整形し\footnote{tourner
    (トゥルネ)}、小玉ねぎをあらかじめバターでこんがり焼き色を付けておく。
\item
  ソース\ldots{}\ldots{}美味しい肉汁(ジュ)少々
\end{itemize}

\hypertarget{garniture-bouquetiere}{%
\subsubsection[ガルニチュール・ブクティエール]{\texorpdfstring{ガルニチュール・ブクティエール\footnote{花売り娘、の意。}}{ガルニチュール・ブクティエール}}\label{garniture-bouquetiere}}

\frsub{Garniture à la Bouquetière}

\index{garniture@garniture!bouquetiere@--- à la Bouquetière}
\index{bouquetiere@bouquetière!garuniture à la ---}
\index{かるにちゆーる@ガルニチュール!ふくていえーる@---・ブクティエール}
\index{ふくていえーる@ブクティエール ⇒ 花売り娘!かるにちゆーる@ガルニチュール・ブクティエール}
\index{はなうりむすめ@花売り娘 ⇒ ブクティエール!かるにちゆーる@ガルニチュール・ブクティエール}

(牛、羊の大掛かりな仕立ての料理\footnote{ルルヴェ relevé
  のこと。\protect\hyperlink{releve}{第二版序文訳注}参照。}に添える)

\begin{itemize}
\item
  にんじん250 gと蕪250
  gはスプーンで中をくり抜いて下茹でし、バターで色艶よく炒める\footnote{glacer
    (グラセ)。}
\item
  小さなじゃがいも250 gはシャトー\footnote{長さ6
    cm程度の細長い樽の形状にすること。両端は切り落すので、ラグビーボール形ではない。}に整形する\footnote{いずれも適切に加熱調理するが、この節では細かく説明されていないので、対応する野菜のページを参照すること。}
\item
  プチポワ\footnote{petits pois
    (プティポワ)いわゆるグリンピースのことだが、日本でよく知られているものよりも若どりで小さく、風味も軽やかで甘みがある。}250
  gと、さいの目に切ったアリコヴェール\footnote{haricots verts
    さやいんげんのことだが、これも日本のものより若どりに適した品種が好まれる。}250
  g
\item
  カリフラワー250 gは花束の形状にバラしておく
\end{itemize}

以上の材料をそれぞれ加熱調理した後に、塊肉の周囲に、ブーケ状に、それぞれを離してニュアンスが明確になるように盛り付ける。カリフラワーのブーケには\protect\hyperlink{sauce-hollandaise}{オランデーズソース}を薄く塗ること。

\begin{itemize}
\tightlist
\item
  ソース\ldots{}\ldots{}塊肉を調理した際の肉汁の浮き脂を取り除き\footnote{dégraisser
    (デグレセ)。}、澄ませたもの
\end{itemize}

\hypertarget{garniture-bourgeoise}{%
\subsubsection[ガルニチュール・ブルジョワーズ]{\texorpdfstring{ガルニチュール・ブルジョワーズ\footnote{bourgeois(e)
  (ブルジョワ/ブルジョワーズ)。ブルジョワ風の意。中世においては都市に住む貴族ではないある種の特権階級を意味したが、
  19世紀以降は、肉体労働をせずに快適できわめて豊かな生活をおくれる社会階層、の意に変化した。社会が物質的に、経済的に豊かになるにともない
  petit bourgeois
  (プティブルジョワ)なる階層も出現したが、ブルジョワの本義はあくまでも「大金持ち」であり、現代日本語でいうところの「セレブ」に相当すると思っていい。}}{ガルニチュール・ブルジョワーズ}}\label{garniture-bourgeoise}}

\frsub{Garniture à la Bourgeoise}

\index{garniture@garniture!bourgeoise@--- à la Bourgeoise}
\index{bourgeois@bourgeois(e)!garuniture à la ---}
\index{かるにちゆーる@ガルニチュール!ふるしよわーす@---・ブルジョワーズ}
\index{ふるしよわーす@ブルジョワーズ!かるにちゆーる@ガルニチュール・ブルジョワーズ}
\index{ふるしよわふう@フルジョワ風 ⇒ ブルジョワーズ!かるにちゆーる@ガルニチュール・ブルジョワーズ}

(牛、羊の塊肉料理に添える)

\begin{itemize}
\item
  にんじん500 gは、にんにくのような形に整形して\footnote{tourner
    (トゥルネ)。}下茹でし、バターで色艶よく炒める\footnote{glacer
    (グラセ)。もともとは「鏡のようにする」ところから「艶を出す」の意となり、野菜の場合はもっぱら下茹でした後にバターで軽く炒めて艶を出すことをいうが、場合によっては茹でる段階で砂糖を煮含めたりもする。}
\item
  小玉ねぎ\footnote{日本のいわゆる「ペコロス」は黄色系品種が多いが、フランスの小さな玉ねぎはもっぱら白系品種であり、甘さや風味がまったく異なるので注意。}500
  gは下茹でした後にバターで色艶よく炒める
\item
  塩漬け豚バラ肉\footnote{原文 lard de poitrine
    (ラールドポワトリーヌ)は豚バラ肉のことだが、通常は塩蔵、熟成させたもの、およびそれを冷燻にかけたものを指す。しばしば「ベーコン」と誤訳されているが、日本語のいわゆるベーコンとは違うので注意。}125
  gはさいの目に切ってバターでこんがり炒める
\item
  このガルニチュールは、塊肉にほぼ火が通った段階で、鍋の中の肉の周囲に入れてやり、ブレゼの煮汁で火入れを完全にすること
\end{itemize}

\hypertarget{garniture-brabanconne}{%
\subsubsection[ガルニチュール・ブラバント風]{\texorpdfstring{ガルニチュール・ブラバント風\footnote{現在はベルギー中部の州ブラバントBrabantの、の意。なお、この名称のガルニチュールは『ラルース・ガストロノミック』初版にも掲載されているが、内容がまったく異なる。アンディーヴとじゃがいものピュレ、ホップの若芽を茹でてバターか生クリームであえたもので構成するという
  (p.239)。なおブラバントは中世においてブラバント公国として独立した国家であった。ベルギー王国成立後は、儀礼称号としてベルギー王家の法定推定相続人にブラバント公の称号が授けられるようになった。なお、エスコフィエによる\protect\hyperlink{peches-melba}{ピーチメルバ}創案のきっかけとなったといわれるワーグナーの楽劇『ローエングリン』においてネリー・メルバNellie
  Melba(1861〜1931)が演じていたエルザ・フォン・ブラバントはブラバント公国の公女という設定。}}{ガルニチュール・ブラバント風}}\label{garniture-brabanconne}}

\frsub{Garniture à la Brabançonne}

\index{garniture@garniture!brabanconne@--- à la Brabançonne}
\index{brabanconne@brabançon(ne)!garuniture à la ---ne}
\index{かるにちゆーる@ガルニチュール!ふらはんとふう@---・ブラバント風}
\index{ふらはんとふう@ブラバント風!かるにちゆーる@ガルニチュール・---}

(牛、羊の塊肉の料理に添える)

\begin{itemize}
\item
  空焼きしたタルトレット10個に、下茹でしてバターで蒸し煮した\footnote{étuver
    (エチュヴェ)。}芽キャベツ\footnote{芽キャベツはchoux de Bruxelles
    (シュドブリュクセル、ブリュッセルのキャベツの意)と呼ぶ。}をピュレにして詰め、\protect\hyperlink{sauce-mornay}{ソース・モルネー}を塗る
\item
  \protect\hyperlink{pommes-de-terre-duchesse}{ポムデュシェス}で作った小さな円盤形のクロケット10個
\item
  ソース\ldots{}\ldots{}\protect\hyperlink{jus-de-veau-lie}{とろみを付けたジュ}
\end{itemize}

\hypertarget{garniture-brehan}{%
\subsubsection[ガルニチュール・ブレオン]{\texorpdfstring{ガルニチュール・ブレオン\footnote{このガルニチュールについては、初版から掲載されているにもかかわらず、Bréhanがブルターニュ地方の町の名であることしかわかっていない。ファーヴルにもデュボワ、ベルナール『古典料理』にも言及は見られない。いささか疑問なのは、Bréhanの住人はbréhannaisという語で表わすことから、形容詞も同様であり、garniture
  à la bréhannaise
  (ガルニチュールアラブレアネーズ)の名称でもおかしくないのだが、第二版および第三版ではGarniture
  à la
  Bréhanとなっており、まるで人名のように扱われていることだろう。なお、ブルターニュ地方はアーティチョークの生産で有名だが旬は晩春から初夏にかけてであり、このガルニチュールの構成要素に初版はトリュフのスライスをそら豆のピュレを詰めたアーティチョークの上にのせる指示がある。カリフラワーも基本的には冬の野菜である。それに対してそら豆は乾物であれば1年中、フレッシュのものはやはり晩春から初夏が旬である。レシピには乾物を使うかフレッシュを使うかの指示がないが、「季節感」を演出するためには、フレッシュのそら豆を用いたいところだろう。}}{ガルニチュール・ブレオン}}\label{garniture-brehan}}

\frsub{Garniture Bréhan}

\index{garniture@garniture!brehan@--- Bréhan}
\index{brehan@Bréhan!garuniture ---}
\index{かるにちゆーる@ガルニチュール!ふれおん@---・ブレオン}
\index{ふれおん@ブレオン!かるにちゆーる@ガルニチュール・---}

(牛、仔牛の塊肉の料理に添える)

\begin{itemize}
\item
  小さなアーティチョークの基底部に、そら豆のピュレをドーム状に詰める。
\item
  カリフラワーの小房10個は\protect\hyperlink{sauce-hollandaise}{ソース・オランデーズ}を軽く塗っておく\footnote{茹でてよく水気をきっておくこと}
\item
  小さなじゃがいも10個はバターで火を通し、パセリのみじん切りを振る
\item
  ソース\ldots{}\ldots{}塊肉をブレゼした際の煮汁をソースに仕上げる
\end{itemize}

\hypertarget{garniture-bretonne}{%
\subsubsection{ガルニチュール・ブルターニュ風}\label{garniture-bretonne}}

\frsub{Garniture à la Bretonne}

\index{garniture@garniture!bretonne@--- à la Bretonne}
\index{breton@breton(ne)!garuniture à la ---ne}
\index{かるにちゆーる@ガルニチュール!ふるたーにゆふう@---・ブルターニュ風}
\index{ふるたーにゆふう@ブルターニュ風!かるにちゆーる@ガルニチュール・---}

(羊料理に添える)

\begin{itemize}
\item
  茹でた白いんげん豆またはフラジョレ\footnote{flageolet
    白いんげん豆の一種で、通常のものより小粒。}1
  Lを\protect\hyperlink{sauce-bretonne}{ブルターニュ風ソース}(ブラウン系の派生ソース参照)であえる、パセリのみじん切りを振りかける。
\item
  ソース\ldots{}\ldots{}塊肉の肉汁(ジュ)
\end{itemize}

\hypertarget{garniture-brillat-savarin}{%
\subsubsection[ガルニチュール・ブリヤサヴァラン]{\texorpdfstring{ガルニチュール・ブリヤサヴァラン\footnote{ジャン・アンテルム・ブリア=サヴァラン(Jean
  Anthelme
  Brillat-Savarin)(1755〜1826)。法律家であり、弁護士、一時はアメリカに亡命し、のちに裁判官として活躍したが、とりわけ、はじめ匿名で出版した『美味礼讃』\emph{Physiologie
  du
  Goût}(1825年、タイトルを直訳すれば「味覚の生理学」)で知られる。この著作は食をめぐる考察からなる随筆集だが、必ずしも生真面目な哲学的記述ばかりではない。むしろ「食をめぐる知的な面白読み物」ともいうすべき内容であり、のちに「生理学もの」というジャンルが流行する嚆矢となった。これにインスパイアされたバルザックが『結婚の生理学』(1829年)を出版し文筆家バルザックとして最初のヒット作となった。その後に続けとばかりに「○○の生理学」と題した書物が19世紀中頃まで数多く出版された。その多くはほとんど文学的にも省みられることのないもので、「丸わかり○○」あるいは「○○
  のすべて」的なものばかりだった。このため、「生理学もの」のうちで文学史において一般的に価値を認められている作品は『美味礼讃』および『結婚の生理学』くらいしかない。}}{ガルニチュール・ブリヤサヴァラン}}\label{garniture-brillat-savarin}}

\frsub{Garniture Bréhan}

\index{garniture@garniture!brillat-savarin@--- Brillat-Savarin}
\index{brillat-savarin@Brillat-Savarin!garuniture ---}
\index{かるにちゆーる@ガルニチュール!ふりやさうあらん@---・ブリヤサヴァラン}
\index{ふりやさうあらん@ブリヤサヴァラン!かるにちゆーる@ガルニチュール・---}

(鳥類のジビエ料理に添える)

\begin{itemize}
\item
  空焼きしたごく小さなタルトレットに、トリュフを加えた\protect\hyperlink{souffle-de-becasse}{ベカスのスフレ}\footnote{現行版の原書でベカスのスフレの項を見ると、\protect\hyperlink{becasse-favart}{ベカス・ファヴァール}と同じ、とある。なお、ファヴァールFavartというのは劇場の名称で、オペラコミック座が19世紀以来本拠地にしていたが、
    2度の火災に遭い、その度に再建された。19世紀にはイタリアオペラを主な演目とする「イタリア座」(テアトル・イタリアン)が間借りのようになかたちでファヴァール劇場を本拠にしていた時期もある。現在のファヴァール劇場は1898年に再建され、2005年以降国立となったオペラコミック座の本拠地となっている。}のアパレイユをピラミッド形に盛り、提供直前にやや低温のオーブンで焦がさないように火を通す。
\item
  大きなトリュフのスライス。
\item
  ソース\ldots{}\ldots{}このガルニチュールを添える\protect\hyperlink{fonds-de-gibier}{ジビエのフュメ}で作った上等な\protect\hyperlink{sauce-demi-glace}{ソース・ドゥミグラス}
\end{itemize}

\hypertarget{garniture-bristol}{%
\subsubsection[ガルニチュール・ブリストル]{\texorpdfstring{ガルニチュール・ブリストル\footnote{Bristol
  はイギリス西部の港湾都市。このガルニチュールの名称となった由来などは不明。}}{ガルニチュール・ブリストル}}\label{garniture-bristol}}

\frsub{Garniture Bristol}

\index{garniture@garniture!bristol@--- Bristol}
\index{bristol@Bristol!garniture@garuniture ---}
\index{かるにちゆーる@ガルニチュール!ふりすとる@---・ブリストル}
\index{ふりすとる@ブリストル!かるにちゆーる@ガルニチュール・---}

(牛、羊の塊肉料理に添える)

\begin{itemize}
\item
  アプリコットの形状、大きさの\protect\hyperlink{croquette-de-riz}{米のクロケット}10個。
\item
  茹でたフラジョレ\footnote{\protect\hyperlink{garniture-bretonne}{ガルニチュール・ブルターニュ風}訳注参照。}
  \(\frac{1}{2}\) Lを\protect\hyperlink{veloute}{ヴルテ}であえる。
\item
  くるみ大の丸い小さなじゃがいも20個はバターで火を通し、溶かした\protect\hyperlink{glace-de-viande}{グラスドヴィアンド}を塗る。
\item
  ソース\ldots{}\ldots{}塊肉をブレゼした煮汁をソースとして仕上げる
\end{itemize}

\hypertarget{garniture-bluxelloise}{%
\subsubsection[ガルニチュール・ブリュッセル風]{\texorpdfstring{ガルニチュール・ブリュッセル風\footnote{芽キャベツchoux
  de Bruxelles
  とアンディーヴendiveはいずれもベルギーで品種改良、開発された野菜であり、これらを組み合わせてブリュッセル風とするのはいささか安易なようにも思われる。}}{ガルニチュール・ブリュッセル風}}\label{garniture-bluxelloise}}

\frsub{Garniture à la Bruxelloise}

\index{garniture@garniture!bruxelloise@--- à la Bruxelloise}
\index{bruxellois@bruxellois(e)!garniture@garuniture à la ---e}
\index{かるにちゆーる@ガルニチュール!ふりゆつせるふう@---・フリュッセル風}
\index{ふりゆつせるふう@ブリュッセル風!かるにちゆーる@ガルニチュール・---}

(牛、羊の塊肉料理に添える)

\begin{itemize}
\item
  アンディーヴ10個は白さを保つようにしてブレゼする。
\item
  シャトー\footnote{\protect\hyperlink{garniture-bouquetiere}{ガルニチュール・ブクティエール}訳注参照。}に整形したじゃがいも10個。
\item
  芽キャベツ500gは下茹でした後バターで蒸し煮する\footnote{étuver
    (エチュヴェ)。下茹での段階で \(\frac{2}{3}\)〜
    \(\frac{3}{4}\)くらいまで火を通しておくこと。サヴォイキャベツもそうだが、下茹でにはアクを除去する意味もあり、エチュヴェの段階で変色してしまうことがあるため、アクを充分に取り除いてから比較的短時間でエチュヴェするのが望ましい。}。
\item
  ソース\ldots{}\ldots{}やや薄めのマデイラ酒風味の\protect\hyperlink{sauce-demi-glace}{ソース・ドゥミグラス}。
\end{itemize}

\hypertarget{garniture-cancalaise}{%
\subsubsection[ガルニチュール・カンカル風]{\texorpdfstring{ガルニチュール・カンカル風\footnote{ブルターニュ地方の地名Cancale(カンカール)の形容詞
  cancalais(e) (カンカレ/カンカレーズ)。牡蠣の産地として知られ、
  cancaleという牡蠣の品種もある。17世紀、ルイ14世は、ヴェルサイユ宮殿へカンカル産カキを取り寄せていたといわれている。なお、ブルターニュ地方とはいえノルマンディ地方に非常に近い位置にあるため、牡蠣を中心にしたこのガルニチュールにブルターニュの地名を冠し、ノルマンディ風ソースを合わせるのは、一種の洒落とも考えられなくもないが、ブルターニュが言語文化的にフランスにおいてやや異質な歴史を持っていることを考慮すると、無神経な命名ともとられかねない。}}{ガルニチュール・カンカル風}}\label{garniture-cancalaise}}

\frsub{Garniture à la Cancalaise}

\index{garniture@garniture!cancalaise@--- à la Cancalaise}
\index{cancalais@cancalais(e)!garniture@garuniture à la ---e}
\index{かるにちゆーる@ガルニチュール!かんかるふう@---・カンカル風}
\index{かんかるふう@カンカル風!かるにちゆーる@ガルニチュール・---}

(魚料理に添える)

\begin{itemize}
\item
  牡蠣20個の剥き身は、沸騰しない程度の温度の湯で火を通し、周囲をきれいに掃除する。殻を剥いたクルヴェットの尾125g
\item
  ノルマンディ風ソース
\end{itemize}

\hypertarget{garniture-cardinal}{%
\subsubsection[ガルニチュール・カルディナル]{\texorpdfstring{ガルニチュール・カルディナル\footnote{カトリック教会における枢機卿のこと。枢機卿の衣が真紅であることからオマールを用いた料理に付けられた名称とも、オマールが「海の枢機卿」と呼ばれるから、ともいわれている。なお、\ul{à la + 男性名詞}
  の形態は、固有名詞の場合および、対応する女性名詞がない場合にも成立する。これは
  \ul{à la manière de + 名詞} のmanière de
  が省略されたものと解釈される。さらに、料理名において à la
  も省略される傾向にあるため、garuniture Cardinal あるいは garniture
  cardinal という表現も\ul{料理名においては}正しいとされている。}}{ガルニチュール・カルディナル}}\label{garniture-cardinal}}

\frsub{Garniture à la Cardinal}

\index{garniture@garniture!cardinal@--- à la Cardinal}
\index{cardinal@cardinal!garniture@garuniture à la ---}
\index{かるにちゆーる@ガルニチュール!かるていなる@---・カルディナル}
\index{かるていなる@カルディナル!かるにちゆーる@ガルニチュール・---}
\index{すうききよう@枢機卿 ⇒ カルディナル!かるにちゆーる@ガルニチュール・カルディナル}

(魚料理に添える)

\begin{itemize}
\item
  立派なオマールの尾の身をやや斜めに厚さ1cm程度にスライスしたもの10枚。
\item
  真黒なトリュフのスライス10枚。
\item
  さいの目に切ったオマールの身60 gとトリュフ50 g。
\item
  \protect\hyperlink{sauce-cardinal}{ソース・カルディナル}
\end{itemize}

\hypertarget{garniture-castillane}{%
\subsubsection[ガルニチュール・カスティリア風]{\texorpdfstring{ガルニチュール・カスティリア風\footnote{Castilla
  (カスティーリャ、カスティージャ)はスペイン中部の地域で、中世はカスティリア王国だった。「カステラ」の語源ともいわれる。}}{ガルニチュール・カスティリア風}}\label{garniture-castillane}}

\frsub{Garniture à la Castillane}

\index{garniture@garniture!castillane@--- à la Castillane}
\index{castillan@castillan(e)!garniture@garuniture à la ---e}
\index{かるにちゆーる@ガルニチュール!かすていりあふう@---・カスティリア風}
\index{かすていりあふう@カスティリア風!かるにちゆーる@ガルニチュール・---}

(トゥルヌド、ノワゼットに添える)

\begin{itemize}
\item
  \protect\hyperlink{pommes-de-terre-duchesse}{ポム・デュシェス}で作ったた小さなケースにドリュールを塗ってオーブンで焼き色を付ける。そこに、軽くにんにく風味を効かせた\protect\hyperlink{portugaise}{トマトのフォンデュ}を詰める。
\item
  皿の周囲に、輪切りにして塩こしょうし、小麦粉をまぶして油で揚げた玉ねぎを飾る。
\item
  トマト風味を加えたデグラセした肉汁(ジュ)\footnote{トゥルヌド、ノワゼットをフライパンでソテーし、デグラセしてトマトピュレまたは本文にあるトマトのフォンデュを加えてソースにするということ。}
\end{itemize}

\hypertarget{garniture-chambord}{%
\subsubsection[ガルニチュール・シャンボール]{\texorpdfstring{ガルニチュール・シャンボール\footnote{シャンボールとは16世紀、ロワール河の近くに建てられた瀟洒な城のある地名。このガルニチュールを添えた場合、料理名にシャンボールが冠される。鯉、サーモンが代表的だが、とりわけ19世紀は鯉が好まれ、カレーム『19世紀フランス料理』第2巻では鯉のシャンボールだけで近代風、ロヤイヤル、レジャンスの3種の仕立てについて詳述されている
  (pp.181-189)。なお、このガルニチュールの構成も時代や料理人によって多少の変化があり、『ラルース・ガストロノミック』初版では、魚でつくった大小のクネル、マッシュルーム、舌びらめのフィレ、バターでソテーした白子、オリーヴ形に整形したトリュフ、クールブイヨンで火を通したエクルヴィス、揚げたクルトン、となっている(p.516)。}}{ガルニチュール・シャンボール}}\label{garniture-chambord}}

\frsub{Garniture Chambord}

\index{garniture@garniture!chambord@--- Chambord}
\index{chambord@Chambord!garniture@garuniture ---}
\index{かるにちゆーる@ガルニチュール!しやんほーる@---・シャンボール}
\index{しやんほーる@シャンボール!かるにちゆーる@ガルニチュール・---}

(魚のブレゼの大掛かりな仕立てに添える\footnote{ルルヴェのこと。\protect\hyperlink{releve}{第二版序文訳注}参照。19世紀前半くらいまではカトリックの習慣としての「小斉」が比較的厳格に守られており、料理人たちは四旬節やその他の小斉の日の献立としていかに豪華で美味な魚料理を提供するかに腐心していたのが、17〜18世紀の料理書を読むとよくわかる。カレームの著書にも魚の大掛かりな仕立てのレシピが数多く収められている。})

\begin{itemize}
\item
  トリュフを加えてスプーンで整形した魚のファルスで作ったクネル10個。
\item
  長卵形の大きな、表面に装飾を施したクネル4個。
\item
  渦巻模様を付けた\footnote{原文 canneler (カヌレ)。この場合はtourner
    (トゥルネ)とほぼ同義だが、凹凸の刻み模様を付けた、の意。}小さなマッシュルーム200
  g。
\item
  鯉の白子を1
  cm程度の厚さにスライスして塩こしょうし、小麦粉をまぶしてソテーしたもの10枚。
\item
  オリーブ形に整形した\footnote{tourner
    (トゥルネ)。原義は「回す」。野菜などを包丁ではなく材料を回すようにして皮を剥いたり整形するところから。}トリュフ200
  g。
\item
  エクルヴィス\footnote{ecrevisse ヨーロッパザリガニ。}6尾は\protect\hyperlink{courtbouillon-a}{クールブイヨン}\footnote{court-bouillon
    (クールブイヨン)。court
    は少量の意。つまり、原則としては少量の液体を煮汁として魚介類その他を加熱調理するのに用いる。また、とりわけ魚介類の場合は沸騰しない程度の温度で火を通す(pocher
    ポシェ)が原則。たんなる水、塩水だけでなく、ワインや香味野菜、香辛料などを加えて風味付け(および場合によっては臭みのマスキング効果)も兼ねて事前に用意しておくこともある。ただしこれらはあくまでも原則論にすぎない。詳細は\protect\hyperlink{poissons}{魚料理}の\protect\hyperlink{serie-de-courts-bouillons-de-poisson}{クールブイヨン}および\protect\hyperlink{ecrevisse-a-la-nage}{エクルヴィス・ナージュ}参照のこと。なお、エクルヴィスの場合は上記の「少量」にあまりこだわらず、後ではさみを背に回しやすくなるように鍋に入れて加熱すればいいだろう。エクルヴィスはジストマ(寄生虫)のリスクがあるためしっかり加熱すること。またエクルヴィスは腕が取れやすいが、その場合でも可食部である尾の身には問題がないので装飾以外の利用はもちろん可能であり、装飾用としてはロス分を見込んで用意しておくのがいいだろう。}で火を通すし、はさみを背に回すように整形する\footnote{trousser
    (トゥルセ)。}(しなくてもよい)。
\item
  食パンを鶏のとさかの形に切りバターで揚げたクルトン6枚。
\item
  魚をブレゼした際の煮汁をベースにしたソース。
\end{itemize}

\hypertarget{garniture-chatelaine}{%
\subsubsection[ガルニチュール・シャトレーヌ]{\texorpdfstring{ガルニチュール・シャトレーヌ\footnote{châtelain(e)
  (シャトラン/シャトレーヌ)。城館の主の意。城館に住む者を思わせる豪華な、の意で料理名として使われるようになったようだ。}}{ガルニチュール・シャトレーヌ}}\label{garniture-chatelaine}}

\frsub{Garniture Châtelaine}

\index{garniture@garniture!chatelaine@--- Châtelaine}
\index{chatelaine@Châtelaine!garniture@garuniture ---}
\index{かるにちゆーる@ガルニチュール!しやとれーぬ@---・シャトレーヌ}
\index{しやとれーぬ@シャトレーヌ!かるにちゆーる@ガルニチュール・---}

(牛、羊の塊肉や鶏料理に添える)

\begin{itemize}
\item
  アーティチョークの基底部10個に、固く作った\protect\hyperlink{sauce-soubise}{スビーズ}を詰める。
\item
  殻を剥いて塊肉をブレゼした煮汁で蒸し煮したマロン30個。
\item
  \protect\hyperlink{pommes-de-terre-noisette}{じゃがいものノワゼット}300
  g。
\item
  ブレゼした煮汁を加えた\protect\hyperlink{sauce-madere}{ソース・マデール}
\end{itemize}

\hypertarget{garniture-chipolata}{%
\subsubsection[ガルニチュール・シポラタ]{\texorpdfstring{ガルニチュール・シポラタ\footnote{もとはイタリアで玉ねぎとソーセージを煮込んだ料理(cipollata
  チポッラータ \textless{} cipolla
  チポッラ=玉ねぎ)を意味していたが、フランスに伝わった際に、語本来の意味に含まれていた玉ねぎが脱落して、羊腸に豚挽肉を詰めた小さなソーセージをこう呼ぶようになったといわれている。}}{ガルニチュール・シポラタ}}\label{garniture-chipolata}}

\frsub{Garniture à la Chipolata}

\index{garniture@garniture!chipolata@--- à la Chipolata}
\index{chipolata@chipolata!garniture@garuniture à la ---}
\index{かるにちゆーる@ガルニチュール!しほらた@---・シポラタ}
\index{しほらた@シポラタ!かるにちゆーる@ガルニチュール・---}

(牛、羊の塊肉および鶏料理に添える)

\begin{itemize}
\item
  小玉ねぎ20個は下茹でしてバターで色艶よく炒める\footnote{glacer
    (グラセ)。本文下のにんじんも同様の指示。}。
\item
  シポラタソーセージ10本。
\end{itemize}

コンソメで煮たマロン10個。

塩漬け豚バラ肉125 gはさいの目に切って、強火でこんがり炒める。

\begin{itemize}
\item
  オリーブ形に整形して下茹でし、バターで色艶よく炒めたにんじん20個(なくてもよい)。
\item
  ソース\ldots{}\ldots{}このガルニチュールを添える料理の煮汁を加えた\protect\hyperlink{sauce-demi-glace}{ソース・ドゥミグラス}
\end{itemize}

\hypertarget{garniture-choisy}{%
\subsubsection[ガルニチュール・ショワジー]{\texorpdfstring{ガルニチュール・ショワジー\footnote{パリのセーヌ川上流(=東側)約12
  kmのところにある Choisy-le-Roi
  の地名に由来。17世紀にショワジー城が建てられ、18世紀にこれを相続したルイ15世が狩りの際に使う邸宅として利用し、現在の名称ショワジールロワになった。その後、ポンパドゥール夫人がここに移り住み、豪華な夕食会がしばしば開かれたという。ショワジーの名称はレチュを用いた料理に付けられることが多い。}}{ガルニチュール・ショワジー}}\label{garniture-choisy}}

\frsub{Garniture Choisy}

\index{garniture@garniture!choisy@--- Choisy}
\index{choisy@Choisy!garniture@garuniture ---}
\index{かるにちゆーる@ガルニチュール!しよわしー@---・ショワジー}
\index{しよわしー@ショワジー!かるにちゆーる@ガルニチュール・---}

(トゥルヌドおよびノワゼットに添える)

\begin{itemize}
\item
  半割りにした\protect\hyperlink{laitue-braise}{レチュのブレゼ}10個。
\item
  シャトーに整形した小さなじゃがいも20個。
\item
  ソース\ldots{}\ldots{}バターを加えた\protect\hyperlink{glace-de-viande}{グラスドビアンド}
\end{itemize}

\hypertarget{garniture-choron}{%
\subsubsection[ガルニチュール・ショロン]{\texorpdfstring{ガルニチュール・ショロン\footnote{19世紀にあったパリの有名レストラン、ヴォワザンの料理長の名。\protect\hyperlink{sauce-bearnaise-tomatee}{ソース・ショロン}も参照。}}{ガルニチュール・ショロン}}\label{garniture-choron}}

\frsub{Garniture Choron}

\index{garniture@garniture!choron@--- Choron}
\index{choron@Choron!garniture@garuniture ---}
\index{かるにちゆーる@ガルニチュール!しよろん@---・ショロン}
\index{しよろん@ショロン!かるにちゆーる@ガルニチュール・---}

(トゥルヌドおよびノワゼットに添える)

\begin{itemize}
\item
  中位か小さいアーティチョークの基底部をにバターであえたアスパラガスの穂先を詰める。アスパラガスがなければ、バターであえた小粒のプチポワでもいい。
\item
  \protect\hyperlink{pommes-de-terre-noisette}{じゃがいものノワゼット}30個。
\item
  \protect\hyperlink{sauce-bearnaise-tomatee}{トマト入りソース・ベアルネーズ}。
\end{itemize}

\hypertarget{garniture-clamart}{%
\subsubsection[ガルニチュール・クラマール]{\texorpdfstring{ガルニチュール・クラマール\footnote{パリ郊外の町の名。プチポワを使った料理にこの名が冠されるものがいくつかある。}}{ガルニチュール・クラマール}}\label{garniture-clamart}}

\frsub{Garniture à la Clamart}

\index{garniture@garniture!clamart@--- à la Clamart}
\index{clamart@Clamart!garniture@garuniture à la ---}
\index{かるにちゆーる@ガルニチュール!くらまーる@---・クラマール}
\index{くらまーる@クラマール!かるにちゆーる@ガルニチュール・---}

(牛、羊の塊肉の料理に添える)

\begin{itemize}
\item
  \protect\hyperlink{petits-pois-francaise}{プチポワ・アラフランセーズ}に細かく刻んだレチュの葉を加えてバターであえ、空焼きしたタルトレット10個に詰める。
\item
  \protect\hyperlink{pommes-de-terre-macaire}{じゃがいものマケール}で作った円形の小さな台の上に、タルトレットをひとつずつのせる。
\item
  ソース\ldots{}\ldots{}\protect\hyperlink{jus-de-veau-lie}{とろみを付けたジュ}\footnote{このガルニチュールを添える料理がポワレ(\protect\hyperlink{sauce-bigarade}{ソース・ビガラード}訳注および\protect\hyperlink{releves-et-entrees}{肉料理}参照)の場合には、鍋に残った香味野菜(マティニョン)にフォン少量を注いで風味を引き出し、それにコーンスターチでとろみを付けることになるだろう。}
\end{itemize}

\hypertarget{garniture-de-compote}{%
\subsubsection[ガルニチュール・コンポート]{\texorpdfstring{ガルニチュール・コンポート\footnote{compote
  (コンポート)。果物のシロップ煮のイメージが強いが、肉や野菜をばらばらになるまで煮込んだ料理のことも指す。}}{ガルニチュール・コンポート}}\label{garniture-de-compote}}

\frsub{Garniture de Compote}

\index{garniture@garniture!compote@--- de Compote}
\index{compote@compote!garniture@garuniture de ---}
\index{かるにちゆーる@ガルニチュール!こんほーと@---・コンポート}
\index{こんほーと@コンポート!かるにちゆーる@ガルニチュール・---}

(鳩およびプレ・ド・グラン\footnote{poulet de grain
  鶏の大きさや飼育方法による区別については\protect\hyperlink{sauce-chaud-froid-vert-pre}{ソース・ショフロワ・ヴェールプレ}参照。}に添える)

\begin{itemize}
\item
  塩漬け豚バラ肉\footnote{lard de poitrine
    (ラールドポワトリーヌ)、lard maigre
    (ラールメーグル)あるいは原文のように合わせて lard de poitrine
    maigre
    (ラールドポワトリーヌメーグル)とも呼ぶが、塩漬けにして熟成させた豚バラ肉のこと。通常、lard
    だけの場合は lard gras
    (ラールグラ)すなわち豚背脂のことを意味するので注意。}250は拍子木\footnote{lardon
    (ラルドン)。たんに lardon
    というだけで、この豚バラ肉の塩漬けを拍子木状に切ったものを意味することはごく一般的で、塩漬け後に冷燻にかけた
    lard de poitrine fumé を拍子木に切ったものは lardon fumé
    (ラルドンフュメ)と呼ばれる。}に切り、下茹でしてからバターでこんがり炒める\footnote{rissoler
    (リソレ)油脂を熱して、素材の表面をこんがり焼くこと。語源は中世からある
    rissole
    (リソール)という円形または半円形、塩味または甘い焼き菓子(揚げ菓子)---
    つまり時代や地域とともに非常にバリエーションに富むものだが、こんがりとした色合いに仕上げるのは共通している。}。
\item
  小玉ねぎ300 gは下茹でしてバターで色艶よく炒める\footnote{glacer
    (グラセ)。}。
\item
  小さなマッシュルーム300 gは生のまま2つに切り、バターで炒める。
\item
  これらは鳩とともに火入れを仕上げ、供する際には鳩を覆うようにガルニチュールを盛る。
\end{itemize}

\hypertarget{garniture-conti}{%
\subsubsection[ガルニチュール・コンティ]{\texorpdfstring{ガルニチュール・コンティ\footnote{ブルボン王家のひとつCondé(コンデ)家の傍流(いわゆる分家筋)で、代々のコンティ大公
  le prince de Conti
  (ルプランスドコンティ)がいる。もとはピカルディ地方アミアンの近くにあるContyというところを領地にしていたのが家名の起源。コンティの名は本書でも、レンズ豆のポタージュ、\protect\hyperlink{puree-conti}{ピュレ・コンティ}が収められている。このガルニチュールは18世紀のコンティ大公ルイ・フランソワ・ド・ブルボン(1727〜
  1776)の料理人が考案したと伝えられているが、実際にそうだったとしても、調理法としてはあまりにシンプルなものなので、ガルニチュールとして供することを考え出した、というのがせいぜいのところだろう。}}{ガルニチュール・コンティ}}\label{garniture-conti}}

\frsub{Garniture Conti}

\index{garniture@garniture!conti@--- Conti}
\index{conti@Conti!garniture@garuniture de ---}
\index{かるにちゆーる@ガルニチュール!こんてい@---・コンティ}
\index{こんてい@コンティ!かるにちゆーる@ガルニチュール・---}

(牛、羊の塊肉のブレゼに添える)

\begin{itemize}
\item
  レンズ豆\footnote{lentille
    (ロンティーユ)。西アジア原産で旧約聖書の「創世記」にも出てくる。アブラハムの息子イサクの双子のうちのひとりであるエサウはすぐれた狩人に、もうひとりのヤコブは「穏かな人で天幕の周りで働くのを常として。(中略)ある日のこと、ヤコブが煮物をしていると、エサウが疲れきって野原から帰ってきた。エサウはヤコブに言った。『お願いだ、その赤いもの(アドム)、そこの赤いものを食べさせてほしい。わたしは疲れきっているんだ。』(中略)エサウは誓い、長子の権利をヤコブに譲ってしまった。ヤコブはエサウにパンとレンズ豆の煮物を与えた。(中略)こうしてエサウは長子の権利を軽んじた。(「創世記」25-27〜34、新共同訳『聖書』)」。レンズ豆はおそらく農耕がはじまったごく初期からの作物であり、エジプトや地中海沿岸で多く栽培されていた。このため温暖な気候に向いた作物であり、その意味ではフランス北部に縁があるコンティ大公の名はふさわしくないかも知れない。ただ、Condé
    すなわちコンデ大公の名を冠した料理、とりわけポタージュ\protect\hyperlink{puree-conde}{ピュレ・コンデ}が赤いんげん豆のポタージュであることとの釣り合いはとれている考えられよう。}のピュレ750
  g。
\item
  脂身のほとんどない豚バラ肉の塩漬け250
  gは長方形に切って、レンズ豆を煮る際に一緒に煮ておく。
\item
  ソース\ldots{}\ldots{}ブレゼの煮汁をソースとして仕上げたもの。
\end{itemize}

\hypertarget{garniture-a-la-commodore}{%
\subsubsection[ガルニチュール・コモドール]{\texorpdfstring{ガルニチュール・コモドール\footnote{もとは英語
  commodore であり、イギリスでは艦隊司令官、アメリカでは准将の意。}}{ガルニチュール・コモドール}}\label{garniture-a-la-commodore}}

\frsub{Garniture à la Commodore}

\index{garniture@garniture!commodore@--- à la Commodore}
\index{commodore@commodore!garniture@garuniture de ---}
\index{かるにちゆーる@ガルニチュール!こもとーる@---・コモドール}
\index{こもとーる@コモドール!かるにちゆーる@ガルニチュール・---}

(魚の大掛かりな仕立てに添える)

\begin{itemize}
\item
  エクルヴィスの尾の身を入れた小さなグラタン皿10個。
\item
  メルラン\footnote{merlan (メルロン)タラ科の海水魚。}の\protect\hyperlink{farce-a}{ファルス}に\protect\hyperlink{beurre-d-ecrevissse}{エクルヴィスバター}を加え、スプーンで整形したクネル10個。
\item
  大きな\protect\hyperlink{moules-a-la-villeroy}{ムール貝のヴィルロワ}10個。
\item
  仕上げにエクルヴィスバターを加えた\protect\hyperlink{sauce-normande}{ノルマンディ風ソース}。
\end{itemize}

\hypertarget{garniture-a-la-commodore}{%
\subsubsection[ガルニチュール・キュシー]{\texorpdfstring{ガルニチュール・キュシー\footnote{キュシー侯爵(1767〜1841)。\protect\hyperlink{osbservation-sur-la-sauce}{基本ソース 概説}訳注参照。}}{ガルニチュール・キュシー}}\label{garniture-a-la-commodore}}

\frsub{Garniture Cussy}

\index{garniture@garniture!cussy@--- Cussuy}
\index{cussy@Cussy (marquis de)!garniture@garuniture ---}
\index{かるにちゆーる@ガルニチュール!きゆしー@---・キュシー}
\index{きゆしー@キュシー!かるにちゆーる@ガルニチュール・---}

(トゥルヌド、ノワゼット、鶏料理に添える)

\begin{itemize}
\item
  マロンのピュレを詰めてグリル焼きした大きなマッシュルーム10個。
\item
  完全に球形に整形し、マデイラ酒風味で火を通した小さなトリュフ10個。
\item
  大きな雄鶏のロニョン\footnote{rognon
    仔牛などでは腎臓のこと。鶏の場合はrognon
    blanc(ロニョンブロン)とも呼び、精巣のこと。この場合は後者。もちろんきちんと加熱調理したものをガルニチュールの構成要素とする。}20個。
\item
  \protect\hyperlink{sauce-madere}{ソース・マデール}
\end{itemize}

\hypertarget{garniture-Daumont}{%
\subsubsection[ガルニチュール・ドモン]{\texorpdfstring{ガルニチュール・ドモン\footnote{ドモン公爵家duché
  d'Aumon(デュシェドモン)にちなんだ名称をいわれている。}}{ガルニチュール・ドモン}}\label{garniture-Daumont}}

\frsub{Garniture Daumont}

\index{garniture@garniture!daumont@--- Daumont}
\index{daumont@Daumont!garniture@garuniture ---}
\index{かるにちゆーる@ガルニチュール!ともん@---・ドモン}
\index{ともん@ドモン!かるにちゆーる@ガルニチュール・---}

(魚料理に添える)

\begin{itemize}
\item
  バターで鍋に蓋をして弱火で火を通した\footnote{étuver au beurre
    (エチュヴェオブール)}マッシュルーム10個に、それぞれエクルヴィスの尾の身を半分に切ったもの6枚ずつ添える。
\item
  \protect\hyperlink{farce-c}{生クリームを加えてつくった魚のファルス}で小さな球形にし、トリュフで装飾を施したクネル10個。
\item
  厚さ1cm程にスライスした\footnote{escalope (エスカロップ)肉や魚を1〜2
    cmの厚さで、筋腺維と直角にスライスした円形または楕円形にスライスしたもの。}白子10枚はイギリス式パン粉衣\footnote{paner
    à l'anglaise
    (パネアロングレーズ)。素材に小麦粉をまぶしてから、卵液に浸し、細かいパン粉で衣を付けること。日本でフライなどをつくる際に一般的な方法とよく似ているが、日本では粗いパン粉が好まれるのに対して、フランスやイギリスでは細かいパン粉を使うのが一般的。}を付けて油で揚げる。
\item
  \protect\hyperlink{sauce-nantua}{ソース・ナンチュア}
\end{itemize}

\hypertarget{garniture-a-la-dauphine}{%
\subsubsection[ガルニチュール・ドフィーヌ]{\texorpdfstring{ガルニチュール・ドフィーヌ\footnote{à
  la Dauphine
  (アラドフィーヌ)王太子妃風、の意。この料理名には由来や理由がないことがほとんど。あえていえば「豪華」であるという程度だが、存外、簡素な仕立ての料理にも付けられることがある。}}{ガルニチュール・ドフィーヌ}}\label{garniture-a-la-dauphine}}

\frsub{Garniture à la Dauphine}

\index{garniture@garniture!dauphine@--- à la Dauphine}
\index{dauphin@dauphin(e)!garniture@garuniture ---e}
\index{かるにちゆーる@ガルニチュール!とふいぬ@---・ドフィーヌ}
\index{とふあん@ドファン/ドフィーヌ!かるにちゆーる@ガルニチュール・ドフィーヌ}

(牛、羊の塊肉の料理に添える)

\begin{itemize}
\item
  \protect\hyperlink{pomme-de-terres-dauphine}{じゃがいものドフィーヌ}をアパレイユにした\protect\hyperlink{croquettes}{クロケット}20個。大きな塊肉料理に添える場合はコルクの栓の形状に、トゥルヌドやノワゼットに添えるときは平たい円盤の形にする。
\item
  マデイラ酒風味の\protect\hyperlink{sauce-demi-glace}{ソース・ドゥミグラス}。
\end{itemize}

\hypertarget{garniture-a-la-dieppoise}{%
\subsubsection[ガルニチュール・ディエープ風]{\texorpdfstring{ガルニチュール・ディエープ風\footnote{dieppois(e)
  (ディエポワ/ディエポワーズ) \textless{} Dieppe
  (ディエープ)ノルマンディ地方の港町の名。}}{ガルニチュール・ディエープ風}}\label{garniture-a-la-dieppoise}}

\frsub{Garniture à la Dieppoise}

\index{garniture@garniture!dieppoise@--- à la Dieppoise}
\index{dieppois@dieppois(e)!garniture@garuniture ---e}
\index{かるにちゆーる@ガルニチュール!ていえーふふう@---・ディエープ風}
\index{ていえーふふう@ディエープ風!かるにちゆーる@ガルニチュール・---}

(魚料理に添える)

\begin{itemize}
\item
  殻を剥いたクルヴェット\footnote{小海老。生の状態で灰色のcrevette
    grise(クルヴェットグリーズ)とやや大きめのcrevette
    rose(クルヴェットローズ)が代表的。}の尾の身100g。
\item
  \(\frac{3}{4}\)
  L(約30個)のムール貝は白ワインを加えた湯で沸騰させない程度の温度で火を通し\footnote{pocher
    (ポシェ)}、周囲をきれいに掃除する\footnote{ébarber
    (エバルベ)貝類の身の周囲をきれいにすること。帆立貝の場合は「ひもを取る」ともいう。}。
\item
  このガルニチュールを添える魚の煮汁を煮詰めて加えた\protect\hyperlink{sauce-vin-blanc}{白ワインソース}。
\end{itemize}

\hypertarget{garniture-doria}{%
\subsubsection[ガルニチュール・ドリア]{\texorpdfstring{ガルニチュール・ドリア\footnote{原書現行版ではDorlaとなっているが初版〜第三版はDoria。19世紀パリのカフェ・アングレの顧客として知られていた名家ドリアの名を冠したといわれている。このドリア家は12世紀ジェノヴァの
  de Auria (ラテン語の filiis
  Auriaeすなわちアウリアの子孫の意)から発する由緒ある家系として有名。なお、日本の洋食のドリアは1930年頃横浜ホテルニューグランド総料理長サリー・ワイルが発案したものといわれており、上記のドリア家とはまったく関係がない。また古代ギリシア時代の民族ドーリア人とも関係がない。ちなみに、バルザックの小説『幻滅』におなじ発音の名の
  Dauriatという登場人物がいる。}}{ガルニチュール・ドリア}}\label{garniture-doria}}

\frsub{Garniture Doria}

\index{garniture@garniture!doria@--- Doria}
\index{doria@Doria!garniture@garuniture ---e}
\index{かるにちゆーる@ガルニチュール!とりあ@---・ドリア}
\index{とりあ@ドリア!かるにちゆーる@ガルニチュール・---}

(魚料理に添える)

\begin{itemize}
\item
  オリーブ形に整形したきゅうり\footnote{concombre
    (コンコンブル)日本で一般的なきゅうりと品種系統も異なるものが多く、サイズも太さ4〜5
    cm、長さ30〜45
    cmで収穫する(品種によって異なる)。青臭さがなく、加熱調理することが多い。}30個をバターとともに鍋に入れて蓋をして弱火で蒸し煮する\footnote{étuver
    (エチュヴェ)。}。
\item
  表皮を剥いて種を取り除いたレモンのスライスを魚の上に並べる。魚は\protect\hyperlink{meuniere}{ムニエル}にしたもの。
\end{itemize}

\hypertarget{garniture-dubarry}{%
\subsubsection[ガルニチュール・デュバリー]{\texorpdfstring{ガルニチュール・デュバリー\footnote{Madame
  du Barry (マダムデュバリー)デュバリー夫人
  (1743〜1793)のこと。ルイ15世の公妾であり、フランス革命により断頭台に送られ命を落したことで知られる。もとはシャンパーニュ地方の貧しい家庭の生まれ。パリに出てのちデュ・バリー子爵に囲われ、いわゆるdemi-mondaine(ドゥミモンデーヌ)、courtisane
  (クルティザーヌ)すなわち高等娼婦として知られるようになる。その後、ポンパドゥール夫人を亡くしたルイ15世が彼女を妾にすることにし、形式上、デュ・バリー子爵の弟と結婚したことにして、正式な社交界デビューを果たした。}}{ガルニチュール・デュバリー}}\label{garniture-dubarry}}

\frsub{Garniture Dubarry}

\index{garniture@garniture!dubarry@--- Dubarry}
\index{dubarry@Dubarry!garniture@garuniture ---}
\index{かるにちゆーる@ガルニチュール!とゆはりー@---・デュバリー}
\index{とゆはりー@デュバリー!かるにちゆーる@ガルニチュール・---}

(牛、羊の塊肉の料理やノワゼット、トゥルヌドに添える)

\begin{itemize}
\item
  カリフラワーの花房を小さく分け、小さなボウルに詰め半球形になるようまとめて裏返し、\protect\hyperlink{sauce-mornay}{ソース・モルネー}で覆ったもの10個。おろした\footnote{râper
    (ラペ)}チーズを振りかけて高温のオーブンでこんがり焼く\footnote{gratiner
    (グラティネ)。また、原文moulés en
    boulesを文字通りに読むと「完全な球形」にするようにも解釈出来なくはないが、そのためには「つなぎ」が必要になる。ソース・モルネー以外に「つなぎ」の役割を果たすものの指定がないため、球形を維持する「つなぎ」としては熱に弱過ぎるだろう。ここは\protect\hyperlink{chou-fleur-au-gratin}{カリフラワーのグラタン}にあるように
    moulé dans un bol 「ボウルに詰める」と同様と解釈してもいいだろう。}。
\item
  \protect\hyperlink{pommes-de-terre-fondantes}{じゃがいものフォンダント}10個。
\item
  塊肉をブレゼあるいはポワレした際のフォン、もしくはノワゼットやトゥルヌドをソテした後にデグラセしてソースに仕上げる。
\end{itemize}

\hypertarget{garniture-a-la-duchesse}{%
\subsubsection[ガルニチュール・デュシェス]{\texorpdfstring{ガルニチュール・デュシェス\footnote{duc
  (デュック=公爵)、duchesse(デュシェス=公爵夫人)。ここではたんに、\protect\hyperlink{pommes-de-terre-duchesse}{じゃがいものデュシェス}を用いるからこの名称になっているが、デュシェスそれ自体にも料理名としての由来や根拠はほとんどない。}}{ガルニチュール・デュシェス}}\label{garniture-a-la-duchesse}}

\frsub{Garniture à la Duchesse}

\index{garniture@garniture!duchesse@--- à la Duchesse}
\index{duc@duc / duchesse!garniture@garuniture à la Duchesse}
\index{かるにちゆーる@ガルニチュール!てゆせす@---・デュシェス}
\index{てゆしえす@デュシェス!かるにちゆーる@ガルニチュール・---}

(牛、羊の塊肉の料理やノワゼット、トゥルヌドに添える)

\begin{itemize}
\item
  じゃがいものデュシェスを舟形または円盤状、あるいはブリオシュ型に詰めて整形し、溶き卵\footnote{dorer
    (ドレ)\textless{} dorure
    (ドリュール)焼いた際に艶を出すために塗る溶き卵。水や牛乳などを混ぜることもある。}を塗って、提供直前にオーブンでこんがり焼いたもの20個。
\item
  \protect\hyperlink{sauce-madere}{ソース・マデール}
\end{itemize}

\hypertarget{garniture-a-la-favorite}{%
\subsubsection[ガルニチュール・ラファヴォリータ]{\texorpdfstring{ガルニチュール・ラファヴォリータ\footnote{『愛の妙薬』や『ランメルモールのルチア』で知られる作曲家ガエタノ・ドニゼッティ(1797〜1848)のグランドオペラ
  \emph{La Favorite}
  (1840年初演)にあやかって付けられた名称。グランドオペラとは19世紀前半から中葉にかけて、パリのオペラ座において、豪華な舞台装置と派手な演出、大編成のオーケストラ、歴史的題材などをテーマとしたわかりやすい筋書きなどを特徴とした様式のこと。ジャコモ・マイアベーア『悪魔ロベール』(1831年)や『ユグノー教徒』(1836年)などはその代表。なお、ロッシーニはこの様式が流行る前のオペラ作曲家と位置付けられている。ドニゼッティのこの作品もマイアベーアの諸作品同様にフランス語の台本、歌詞であり、原題もフランス語で
  \emph{La
  Favorite}(ラファヴォリット)だが、どういうわけか、こんにちの日本ではイタリア語式に直した『ラファヴォリータ』と呼ばれることが多いためにそれに合わせた。なおこのオペラのリブレット(台本、歌詞)はアルフォンス・ロワイエとギュスターヴ・ヴァエズによるものだが、18世紀バキュラール・ダルノー(1718〜1805)の戯曲『不幸な恋人たち』を原作としている。さらにいえば、バキュラール・ダルノーの戯曲もまた、クロディーヌ・ゲラン・ド・トンサン(1682〜1749)の小説『コマンジュ伯爵の手記』を翻案したもの。『ロメオとジュリエット』の物語のバリエーションのひとつともいえるこの小説は18世紀に大きな反響を呼び、多くの小説、戯曲に影響を与えた。ダルノーの戯曲はその代表例。}}{ガルニチュール・ラファヴォリータ}}\label{garniture-a-la-favorite}}

\frsub{Garniture à la Favorite}

\index{garniture@garniture!favorite@--- à la Favorite}
\index{favorite@Favorige (La)!garniture@garuniture à la Favorite}
\index{かるにちゆーる@ガルニチュール!らふあうおりーた@---・ラファヴォリータ}
\index{らふあうおりーた@ラファヴォリータ!かるにちゆーる@ガルニチュール・---}

(ノワゼット、トゥルヌドに添える)

\begin{itemize}
\item
  小さめのフォワグラを厚さ1 cm程にスライス\footnote{éscalope
    (エスカロップ)。}し、塩こしょうしてから小麦粉をまぶしてバターでソテーしたもの10枚。
\item
  大きなトリュフのスライスをソテーしたフォラグラに1枚ずつのせる。
\item
  アスパラガスの穂先を束にしたもの。
\item
  ソース\ldots{}\ldots{}\protect\hyperlink{jus-de-veau-lie}{とろみを付けたジュ}。
\end{itemize}

\hypertarget{garniture-a-la-fermiere}{%
\subsubsection[ガルニチュール・フェルミエール]{\texorpdfstring{ガルニチュール・フェルミエール\footnote{日本語にすれば「農場主風」。野菜を厚さ1
  mmくらい、長さ1 cm程度の四角形に切ることを détailler en paysanne
  (デタイエオンペイザーヌ)というが、そのpaysanneとはpaysan(ペイゾン=農民)の女性形であり、このガルニチュールでは野菜をすべてそのように切るところにかけての名称。}}{ガルニチュール・フェルミエール}}\label{garniture-a-la-fermiere}}

\frsub{Garniture à la Fernière}

\index{garniture@garniture!fermiere@--- à la Fermière}
\index{fermier@fermier/fermière!garniture@garuniture à la Fermière}
\index{かるにちゆーる@ガルニチュール!ふえるみえーる@---・フェルミエール}
\index{ふえるみえ@フェルミエ/フェルミエール!かるにちゆーる@ガルニチュール・フェルミエール}

(鶏料理に添える)

\begin{itemize}
\item
  にんじん150 gと蕪150 gは厚さ1 mm程度、長さ1
  cm程度の四角形に切る\footnote{原文émincer en paysanne
    (エマンセオンペイザーヌ)。ペイザーヌに切る場合、動詞にはémincer
    薄くスライスする、も使われる。}。
\item
  玉ねぎ50 gとセロリ50 gも同様に切る。
\item
  これらを鍋に入れてバターと、塩3g、粉砂糖5
  gを加えて蓋をして弱火で軽く蒸し煮する\footnote{étuver (エチュヴェ)。}。
\item
  野菜を鶏の周囲に盛り、野菜の火入れを仕上げる\footnote{ここでは\protect\hyperlink{poulet-saute-a-la-fermiere}{鶏のソテー・フェルミエール}が念頭にあるので、表面に焼き色を付けた鶏をあらかじめ軽く蒸し煮しておいた、このガルニチュール・フェルミエールとともに陶製の鍋に入れて、さいの目に切ったハムを加え、蓋をしてオーブンに入れて鶏と野菜の火入れを完成させることになる。}。
\end{itemize}

\hypertarget{garniture-a-la-financiere}{%
\subsubsection[ガルニチュール・フィナンシエール]{\texorpdfstring{ガルニチュール・フィナンシエール\footnote{徴税官風の意。名称について詳しくは\protect\hyperlink{sauce-financiere}{ソース・フィナンシエール}訳注参照のこと。なお、カレームは『19世紀フランス料理』で、多少の違いはあるが、これらの具材とソースを合わせることで「ラグー・フィナンシエール」と呼んでいる(t.3,
  pp.146-148)。これはつまり、ソース・フィナンシエールの訳注でも述べたように、もともとはガルニチュールとソースが別々のものではなく、一体化したものとして調理されていた可能性を示唆している。現代でもイタリア北部でつくられている内臓や鶏のとさかなどの煮込み、Ragù
  a la financiera
  (ラグーアラフィナンツィエーラ)がこの原型に近いものと考えられよう。ただし、徴税官という役職にある貴族が裕福だから珍味佳肴を愉しんだ、という意味による命名と考えるか、徴税官からごっそり税をとられてしまうからさしたるものを食べられなかったというところから来たのかは議論の分かれるところだろう。ただ、フランスにおいて雄鶏のとさかとロニョンが高級食材として珍重されていたたこともまた事実。}}{ガルニチュール・フィナンシエール}}\label{garniture-a-la-financiere}}

\frsub{Garniture à la Financière}

\index{garniture@garniture!financiere@--- à la Financière}
\index{financier@financier/financère!garniture@garuniture à la Financière}
\index{かるにちゆーる@ガルニチュール!ふいなんしえーる@---・フィナンシエール}
\index{ふいなんしえ@フィナンシエ/フィナンシエール!かるにちゆーる@ガルニチュール・フィナンシエール}

(牛、羊の塊肉あるいは鶏料理に添える)

\begin{itemize}
\item
  仔牛か鶏のファルスでつくった標準的なクネル20個。ファルスに仔牛を使うか鶏を使うかは、このガルニチュールを添える料理に合わせて決めること。
\item
  渦巻状の刻み模様を入れた小さなマッシュルーム150 g。
\item
  雄鶏のとさかとロニョン\footnote{rognonは通常「腎臓」を指すが、雄鶏の場合はrognon
    blanc(ロニョンブロン)=
    testicule(テスティキュル)すなわち精巣のこと。}100 g。
\item
  トリュフのスライス50 g。
\item
  皮を剥いて下茹でしたオリーブ12個。
\item
  \protect\hyperlink{sauce-financiere}{ソース・フィナンシエール}
\end{itemize}
\end{recette}


%%% chapitre iii. potages
%% エスコフィエ『料理の手引き』全注解
% 五島 学
%% III. potages
\href{原稿下準備20180414五島、連載からコピー}{} \href{訳と注釈}{}
\href{未、原文対照チェック}{} \href{未、日本語表現校正}{}
\href{未、その他修正}{} \href{未、原稿最終校正}{}

\hypertarget{potages}{%
\chapter{III. ポタージュ}\label{potages}}

\hypertarget{considerations-generales-potages}{%
\section{概説}\label{considerations-generales-potages}}

\frsec{Considérations Générales}

我々がポタージュと呼んでいる料理は、今ある形態としては比較的新しいもの
であり、せいぜい19世紀初頭までしか起源を遡ることができない。

それ以前のポタージュは一皿にすべて揃った料理だった\footnote{potage
  の語源は pot「壺、鍋」。古くは、鍋に肉や野菜を入れて煮込
  んだ料理(シヴェ、ラグー、ブルゥエなど)の総称であった。こんにちのよ
  うにポタージュが専ら液体料理の意味で用いられるようになったのは、エ
  スコフィエがここで述べているように、19世紀以降である。なお、日本語
  では一般的に「ポタージュ」という語はとろみがあるスープを意味し、
  「コンソメ」が澄んだスープを指すが、これは英語由来。}。こんにちではポ
タージュという名称は液体を指すだけだが、古くは、その液体を作るのに用い
た畜肉または家禽、ジビエ、魚そのものと、ガルニテュールである野菜も常に
ポタージュに含まれていたのだ。

いくつか挙げるなら、フランドルの「オシュポ\footnote{hochepotフランドルの地方料理としては、牛の尾を主素材としたポトフ
  の一種を指す。他に、豚の耳と尾、煮込み用牛肉と野菜の煮込みを意味す
  る場合もある。料理名自体は非常に古く、14世紀のタイユヴァン『ル・ヴィ
  アンディエ』には「鶏のオシュポ」hochepot de poullaille がある。}」、スペインの「オヤ
\footnote{olla (olla
  podrida)豚肉と各種内臓肉、豆、野菜の煮込み料理。南西 フランスのウイヤ
  ouillat (オイユ oilleとも)の原型になったと言われ
  ている。また、日本語の「おじや」の語源とする説もある。}」、そして我が国の「プティート・マルミート\footnote{petite
  marmite小ぶりの陶製の鍋に具材と汁を入れてオーヴンで熱して
  供するポトフに似た料理。牛肉、牛の尾、骨髄、鶏などをブイヨンで煮込
  む。19世紀パリのレストランで大流行した。澄んだポタージュ(コンソメ)
  に分類されるが、クラリフィエ(澄ませる作業)は行なわない。また、必ず
  鶏を用いるのが特徴。}」もまた、古くから残
されてきたポタージュの代表例と言える。もっとも、現代ではこれらの料理を
作る際に、多かれ少なかれ単純な構成にしているから、こう書いただけでは明
確なイメージが得られないかも知れない。

これらの料理にはごった煮のようなイメージがあるが、昔の献立というのはつ
まるところそういうものだったのだ。食べ進むにつれ食欲がだんだん満たされ
ていくのに合わせて、順序よく料理を進行させるようなことはしなかった\footnote{現代のように、食べる順に料理を提供する「ロシア式サーヴィス」が行
  なわれるようになったのは19世紀のこと。それ以前は大きな皿に盛られた
  複数の料理をまとめて食卓に供する「フランス式サーヴィス」であった。}。
昔の献立に非常に多くの種類のポタージュが見られるのは、適切に料理を配し
た結果というよりは、まさにそのこと自体が献立の特徴なのだと言える\footnote{典拠不明。確かに、17世紀以前の料理書ではポタージュに多くのページ
  を割いているものも少なくないが、17世紀にL.S.R.という筆名で出版され
  た『饗応術』や、同じく17世紀マシアロの著作にある献立例では、必ずし
  もポタージュの数が突出して多いとは言えない。大規模な宴席を除けば、
  通常の献立におけるポタージュは2〜3種であり、同時に供されるアントレ、
  オル・ドゥーヴルの方が種類が多い。また、『ル・ヴィアンディエ』巻末
  の献立例では料理が4回に分けて供されるが、1回目はポタージュ、2回目
  はロ(ロティ)があてられており、料理の種類はほぼ同数ずつになっている。}。

他の多くの調理技術についても同じだが、ポタージュに関してカレームの功績
は大きい。文字通りの意味では、カレームを現代料理の礎を築いた先駆者とは
呼べないが、少なくとも、新たな料理理論の普及に大いに貢献したのだ。

けれど、カレームの後継者たちがポタージュをこんにちの姿に完成させるまで、
1世紀近い時間を必要とした。

彼らは風味豊かで軽やかな、理想的なまでに繊細で美味な料理を創案したのだ
が、それらの新しいポタージュに正しい料理名をつけることにはあまり頓着し
なかったのだろう。とりわけ、とろみをつけたポタージュに関しては、しばし
ば同じルセットについて「ビスク」や「ピュレ」「クーリ」「ヴルテ」「クレー
ム」の名称をかまわずつけていた。理屈から言って、これらの名称はそれぞれ
完全に異なった料理を意味しなくてはいけないにもかかわらずである。その結
果として、残念なことに用語の混乱が起きてしまったわけだが、本書では、そ
れぞれのポタージュの特徴を明確にし、様々なルセットを合理的に分類して、
この問題を正してある。

これは以下のような考察に基づいている。すなわち、「ヴルテ」と「クレーム」
の語がポタージュについて用いられるようになったのは比較的近年のことであ
る。その理由としては、「ビスク」「クーリ」はやや古めかしい印象だし、
「ピュレ」はあまりに俗っぽく、しかも不適切だから、ビスク、クーリ、ピュ
レの代りにヴルテとクレームの語が用いられるようになったと思われる。

だから、ポタージュの各種別を明確に定義し、調理技術体系の欠落を埋める必
要があるのだ。

それぞれの種類のポタージュの特徴について次にまとめておいたが、本書で企
図したこの改革の意義がお解りいただけることだろう。

\hypertarget{classification-des-potages}{%
\section{ポタージュの種類}\label{classification-des-potages}}

\frsec{Classification des Potages}

料理を提供するという観点からは、ポタージュは大きく2つに分類される。
「澄んだポタージュ」と「とろみのあるポタージュ」である。

格式ある大がかりな献立では常に、それぞれの分類から1つ以上のポタージュ
が供される。通常の献立では、ポタージュを1つだけにする場合、献立全体の
流れに応じて、2つの分類のうちどちらか一方とする。

\hypertarget{classification-potages-claires}{%
\subsection{澄んだポタージュ}\label{classification-potages-claires}}

\frsecb{Les Potages claires}

澄んだポタージュ

澄んだポタージュは、主素材として畜肉、家禽、ジビエ、魚、甲殻類、海亀な
どのどれを用いたものでも、分類としてはただひとつとなる。つまり、澄んだ
コンソメ\footnote{consommé 語源は動詞
  consommer「完遂する、完成させる」。つまり
  「完全に仕上げたもの」の意。もともとは必ずしも液体料理を指す語では
  なかった。例えば18世紀マラン『食の贈り物』には以下のような「コンソ
  メ」が出ている。まず「最初のブイヨン」をとり、それを用いて「ブイヨ
  ン・ミトナージュ」をとり、ブイヨン・ミトナージュを用いて「ブイヨン・
  コルディアル」をとり、さらにブイヨン・コルディアルを用いて「コンソ
  メ」を作る。最終的にはとろみがつく程度に煮詰める。このコンソメは単
  体で料理として供するものではなく、調味料的にポタージュにこくを与え
  たり、ソースを作るのに用いると記されている。}である。これはタピオカでんぷんでごく軽くとろみをつける場合
もあるが、いずれにせよ、それぞれのポタージュの性格に合わせて浮き実は少
量とする。

\hypertarget{classification-potages-lies}{%
\subsection{とろみを付けたポタージュ}\label{classification-potages-lies}}

\frsecb{Les Potages Liés}

とろみのあるポタージュ

とろみのあるポタージュは5つに分類される。すなわち

\begin{enumerate}
\def\labelenumi{\arabic{enumi})}
\item
  ピュレ、クーリ、ビスク
\item
  クレーム
\item
  ヴルテ
\item
  とろみをつけたコンソメ
\item
  特殊な複合ポタージュ。上記のうち複数のポタージュの性格を持ち、ヴァ
  リエーションを展開出来ないもの。
\end{enumerate}

より単純な分類にするため、本書では、ジェルミニ\footnote{刻んだオゼイユの葉をバターで炒め溶かして裏漉しし、白いコンソメを
  加えて火にかける。供する直前に溶きほぐした卵黄と生クリームを加えて
  火にかけ、クレーム・アングレーズのようにする。とろみがついたら、バ
  ターを加えて仕上げ、セルフイユの葉を飾る(原書p159)。}のようなとろみをつけ
たコンソメは特殊なポタージュに含めてある。その理由は「特殊なポタージュ」
の節の冒頭で述べてある\footnote{原書 p.157}。

上記のポタージュのうち最初の3種はベースとして何らかのピュレを用いるが、
主に何によってとろみをつけるかで違ってくる。

「ピュレ」「クーリ」「ビスク」でとろみをつけるのは、主素材に応じて、米、
油で揚げたパン、じゃがいも、いんげん豆、レンズ豆などのでんぷん質の野菜。
これらのつなぎと主素材との分量比率はきちんと守らなければならないので、
「とろみをつけたポタージュ」の章の冒頭を参照のこと\footnote{原書 p.135}。

「クレーム」と「ヴルテ」のとろみは白いルウがベースとなってはいるが、実
際に用いるつなぎは違う\footnote{ポタージュ・ヴルテは基本ソースのヴルテをつなぎとして用いるのに
  対し、ポタージュ・クレームはベシャメル(原書pp.136-138)。}。また、仕上げ方も異なる。

ヴルテの仕上げは必ず卵黄とバターでつなぐ。クレームはとろみをつける要素
は足さず、バターではなく良質の生クリーム適量を加えて仕上げる。

つまり、異なるつなぎの要素を用いているわけだから、クレームとヴルテは明
確に違うものとして区別すべきなのだ。

ピュレ、クーリ、ビスクは、いずれも作り方がほぼ同じわけだが、にもかかわ
らず、これらの語は同義ではない。むしろ明らかに違う意味を持っていること
に注意。

慣習として「ピュレ」の名称は野菜をベースにしたものに用いるが、この名称
には俗っぽい印象があるため、避けられる傾向にある。

「クーリ」の名称は家禽、ジビエ、魚および甲殻類のピュレについて用いる。

だが、甲殻類のピュレについては、「ビスク」の名称を用いる方が多い。むし
ろ、ビスクと言えば甲殻類のピュレを指すことになっている。ただし、この語
が生まれた当時から18世紀末までは、家禽と鳩を主素材にしたポタージュのこ
とであった。

とろみのあるポタージュの多くは、主素材そのままで単に調理法を変えれば、
ピュレ、クレームおよびヴルテとして展開出来るのだが、これについては後で
記すことにする\footnote{原書 pp.134-138}。

\hypertarget{consideration-sur-le-service-des-potages}{%
\subsection{ポタージュを供するにあたっての注意}\label{consideration-sur-le-service-des-potages}}

\vspace{-1\zw} \begin{center}
\hspace{1\zw}\large\textit{Considérations sur le Service des Potages}
\normalsize \end{center}
\newpage
\href{原稿下準備20180414五島、連載からコピー}{} \href{訳と注釈}{}
\href{未、原文対照チェック}{} \href{未、日本語表現校正}{}
\href{未、その他修正}{} \href{未、原稿最終校正}{}

\hypertarget{ux30ddux30bfux30fcux30b8ux30e5ux3064ux304fux308aux306eux57faux790e}{%
\section{ポタージュつくりの基礎}\label{ux30ddux30bfux30fcux30b8ux30e5ux3064ux304fux308aux306eux57faux790e}}

\begin{center}
プチットマルミット、グランドブイヨン、いろいろなコンソメのクラリフィエの方法
\end{center}

\frsec{Précis des éléments nurtirifs, aromatiques}

\begin{center}

et de l'assaisonnement pour la Petite Marmite, les Grands Bouillons,
et la clarification des Consommé divers

\end{center}

\begin{recette}
\hypertarget{ux767dux3044ux30b3ux30f3ux30bdux30e1ux30b5ux30f3ux30d7ux30eb1}{%
\subsubsection[白いコンソメ・サンプル]{\texorpdfstring{白いコンソメ・サンプル\footnote{consommé
  simple「単純な(簡素な)コンソメ」の意。肉や魚、野菜を煮
  て漉しただけのもの。ここでは具体的な作業手順は記されていないが、モ
  ンタニェの『ラルース・ガストロノミーク』初版(1937年)の記述は概ね以
  下のとおり(材料はエスコフィエとほぼ同じ)。(a) 牛肉を紐で縛り、大鍋
  (陶製が良い)に入れて水7 Lを注ぐ。火にかけて沸騰したら、表面にアルブ
  ミンの軽く固まった膜が張るので、丁寧にこの膜を取り除く。鍋に野菜を
  加える。かすかに沸騰する火加減で5時間煮る。浮き脂を丁寧に取り除き、
  布または目の細かい漉し器で漉す。5時間以上煮込んではいけない。だが、
  5時間では骨に含まれているおいしさを全て抽出出来ないので、砕いた骨
  を長時間煮て第1のブイヨンをとり、これで肉と野菜を煮るようにすると
  良い。(b) 鍋に砕いた骨を入れ、水をかぶる程度注ぐ。沸騰させ、あくを
  引き、塩を加える。弱火で2時間半煮る。この「沸騰したブイヨン」に、
  骨を外して紐で縛った肉を入れる。再び沸騰させ、あくを引いて味を調え
  る。野菜を加え、弱火で約4時間煮る。塩は最初に全量を入れないこと。
  必要なら作業の最終段階でも塩を加える。}}{白いコンソメ・サンプル}}\label{ux767dux3044ux30b3ux30f3ux30bdux30e1ux30b5ux30f3ux30d7ux30eb1}}

(仕上がり 10 L分)

\begin{itemize}
\item
  主素材\ldots{}\ldots{}牛赤身肉4 kgと牛骨付きすね肉3 kg。
\item
  香味素材\ldots{}\ldots{}にんじん1.1 kg (5〜6本)、かぶ900
  g(5〜6ヶ)、ポワロー200 g、パース ニップ\footnote{panaisパネ。和名アメリカボウフウ。セリ科の根菜で、香りが良い。
    白く、にんじんに似た円錐形のため、俗に「白にんじん」と呼ばれること
    もあるが、にんじんとは別種。でんぷん質が豊富で、ピュレ等の調理にも
    適している。}200 g、玉ねぎ(中)2ヶ(200
  g)、クローブ3本、にんにく3片(20 g)、 セロリ120 g。
\item
  加える液体\ldots{}\ldots{}水14 L。
\item
  調味料\ldots{}\ldots{}粗塩70 g。
\item
  加熱時間\ldots{}\ldots{}5時間。
\end{itemize}

\hypertarget{ux4f5cux308aux65b9ux306bux95a2ux3059ux308bux88dcux8db33}{%
\paragraph[作り方に関する補足]{\texorpdfstring{作り方に関する補足\footnote{この部分は第二版で加筆された。}}{作り方に関する補足}}\label{ux4f5cux308aux65b9ux306bux95a2ux3059ux308bux88dcux8db33}}

\ldots{}\ldots{}コンソメ・サンプルを作る際、一般的には5時間かけて煮ることになっている。
肉汁を抽出するには充分な時間である。

しかし、骨の組織を壊して可溶性物質を確実に抽出するには5時間では絶対に
足りない。骨から可溶性物質を抽出することはとても重要だが、そのためには
弱火で12〜15時間煮る必要がある。

だから、グランド・キュイジーヌでは、粗く砕いた骨を12時間以上煮て第1の
コンソメをとるようになってきている。

この第1のコンソメを第2のコンソメをとる鍋に注ぐ。この鍋で肉を約4時間、
すなわち肉を煮るのに最低限必要な時間、火にかける。

肉を野菜を塊のままではなく細かく刻めば、2つめの作業時間をさらに短かく
することも可能だ。その場合は、通常のクラリフィエと同様の作業となる。
(「クラリフィエ」の項参照)。
\end{recette}
\hypertarget{ux30afux30e9ux30eaux30d5ux30a3ux30a84}{%
\subsection[クラリフィエ]{\texorpdfstring{クラリフィエ\footnote{原文
  clarifications クラリフィカシオン (動詞 clarifier「澄ませ
  る」の名詞形)。字義通りには「澄ませる作業」だが、実際にはコンソメ・
  ドゥーブルconsommé double (コンソメ・リッシュ consommé richeコンソ
  メ・クラリフィエ consommé clarifiéとも呼ばれる)を作ることを意味す
  る。本来はその工程のひとつであった「澄ませる作業」が作業全体を指す
  語として定着したのだろう。}}{クラリフィエ}}\label{ux30afux30e9ux30eaux30d5ux30a3ux30a84}}

\frsecb{Clarifications}
\begin{recette}
\hypertarget{ux901aux5e38ux306eux30b3ux30f3ux30bdux30e1}{%
\subsubsection{通常のコンソメ}\label{ux901aux5e38ux306eux30b3ux30f3ux30bdux30e1}}

(仕上がり4 L分)

\begin{itemize}
\item
  白いコンソメ・サンプル\ldots{}\ldots{}5 L。
\item
  主素材\ldots{}\ldots{}牛赤身肉1.5
  kg。丁寧に筋を除き、挽いておく\footnote{原文 hacher
    アシェ(細かく刻む)。語源は hacheアーシュ(斧)。日本
    語の「刻む」は包丁を用い、「挽く」はミートチョッパーのような器具を
    用いる場合を指すが、フランス語では区別せずどちらもhacher と表現す
    る。}。
\item
  香味素材\ldots{}\ldots{}にんじん100 g、ポワロー200
  g。小さなさいの目\footnote{brunoise ブリュノワーズ}に刻んでおく。
\item
  澄ませるための素材\ldots{}\ldots{}卵白2ヶ分。
\item
  所要時間\ldots{}\ldots{}1時間半。
\item
  作業\ldots{}\ldots{}片手鍋\footnote{casserole カスロール}または小ぶりの寸胴鍋\footnote{marmiteマルミート。一般的には、大型で深さが直径以上ある両手鍋を
    指す。}に牛挽肉、小さなさいの目に刻
  んだ野菜、卵白を入れ、全体をよく混ぜる。白いコンソメ・サンプルを注ぎ入
  れ、時々混ぜながら\footnote{ここは原文に忠実に訳したが、実際には常に混ぜ続けないと卵白が鍋
    底にくっついて無駄になってしまう。『ラルース・ガストロノミック』初
    版では「絶えず混ぜる」ように指示されている。}沸騰させる。軽く沸騰させながら1時間半煮る。
\end{itemize}

布で漉して仕上げる。

\hypertarget{ux9d8fux306eux30b3ux30f3ux30bdux30e1}{%
\subsubsection{鶏のコンソメ}\label{ux9d8fux306eux30b3ux30f3ux30bdux30e1}}

(仕上がり4 L分)**

*白いコンソメ・サンプル\ldots{}\ldots{}同上。

\begin{itemize}
\item
  主素材と香味素材\ldots{}\ldots{}同上に、以下を加える。オーヴンで軽く色づけた鶏1羽。
  鶏の首づる、手羽先、足など\footnote{原文
    abatisアバティ。鶏肉として食べられる以外の部位の総称。
    鶏の「内臓」と訳されることが多いが、とさか、頭、首づる、手羽先、足
    なども含まれる。}を刻んだもの6羽分。ロティールした鶏
  のがら\footnote{鶏のロティ(ローストチキン)を提供した際に出る「がら」。}2羽分。
\item
  澄ませるための素材、方法、時間は通常のコンソメと同様にする。
\end{itemize}
\end{recette}\newpage
\href{原稿下準備20180414五島、連載からコピー}{} \href{訳と注釈}{}
\href{未、原文対照チェック}{} \href{未、日本語表現校正}{}
\href{未、その他修正}{} \href{未、原稿最終校正}{}

\hypertarget{ux30ddux30bfux30fcux30b8ux30e5ux306eux4e3bux306aux6d6eux304dux5b9fux30acux30ebux30cbux30c1ux30e5ux30fcux30eb}{%
\section{ポタージュの主な浮き実(ガルニチュール)}\label{ux30ddux30bfux30fcux30b8ux30e5ux306eux4e3bux306aux6d6eux304dux5b9fux30acux30ebux30cbux30c1ux30e5ux30fcux30eb}}

\frsec{Elémtent divers de Garnitures pour Potages}
\newpage
%\href{スミ、原稿下準備\%20amanojack0615-20180510}{}
\href{未、原文対照チェック}{} \href{未、日本語表現校正}{}
\href{未、その他修正}{} \href{未、原稿最終校正}{}

\hypertarget{potages-claires-et-consommes-garnis}{%
\section{澄んだポタージュ、ガルニチュール入りコンソメ}\label{potages-claires-et-consommes-garnis}}

\frsec{Série des Potages claires et Consommés garnis}

\index{potages@potages!claire@---s claires}
\index{consomme@consommé!garnis@---s garnis}
\index{ほたーしゆ@ポタージュ!すんた@澄んだ---}
\index{こんそめ@コンソメ!かるにちゆーるいり@ガルニチュール入り---}

\hypertarget{nota-potages-claires-et-consommes-garnis}{%
\paragraph{【原注】}\label{nota-potages-claires-et-consommes-garnis}}

\href{コメント\ldots{}\ldots{}この部分は無視してください。この下からスタートしてください}{}
\begin{recette}
\hypertarget{consomme-aux-ailerons}{%
\subsubsection{コンソメ・鶏手羽入り}\label{consomme-aux-ailerons}}

\frsub{Consommé aux ailerons}

\index{consomme@consommé!ailerons@--- ailerons}
\index{aileron@aileron!consomme@consommé ---}
\index{こんそめ@コンソメ!とりてはいり@---・鶏手羽入り}
\index{とりては@鶏手羽!こんそめ@コンソメ・---入り}

\hypertarget{consomme-alexandra}{%
\subsubsection{コンソメ・アレクサンドラ}\label{consomme-alexandra}}

\frsub{Consommé Alexandra}

\index{consomme@consommé!alexandra@--- Alexandra}
\index{alexandra@Alexandra!consomme@consommé ---}
\index{こんそめ@コンソメ!あれくさんとら@---・アレクサンドラ}
\index{あれくさんとら@アレクサンドラ!こんそめ@コンソメ・---}

\hypertarget{consomme-a-l-ancienne}{%
\subsubsection{コンソメ・クラシック}\label{consomme-a-l-ancienne}}

\frsub{Consommé à l'Ancienne}

\index{consomme@consommé!ancienne@--- Ancienne}
\index{ancienne@Ancienne!consomme@consommé ---}
\index{こんそめ@コンソメ!くらしつく@---・クラシック}
\index{くらしつく@クラシック!こんそめ@コンソメ・---}

\hypertarget{consomme-d-arenberg}{%
\subsubsection{コンソメ・アーレンベルク}\label{consomme-d-arenberg}}

\frsub{Consommé d'Arenberg}

\index{consomme@consommé!arenberg@--- Arenberg}
\index{arenberg@Arenberg!consomme@consommé ---}
\index{こんそめ@コンソメ!あーれんへるく@---・アーレンベルク}
\index{あーれんへるく@アーレンベルク!こんそめ@コンソメ・---}

\hypertarget{consomme-a-l-aurore}{%
\subsubsection{コンソメ・オーロール}\label{consomme-a-l-aurore}}

\frsub{Consommé à l'Aurore}

\index{consomme@consommé!aurore@--- Aurore}
\index{aurore@Aurore!consomme@consommé ---}
\index{こんそめ@コンソメ!おーろーる@---・オーロール}
\index{おーろーる@オーロール!こんそめ@コンソメ・---}

\hypertarget{consomme-belle-fermiere}{%
\subsubsection{コンソメ・ベル・フェルミエール}\label{consomme-belle-fermiere}}

\frsub{Consommé Belle Fermière}

\index{consomme@consommé!bellefermiere@--- Belle Fermière}
\index{bellefermiere@Bell Fermière!consomme@consommé ---}
\index{こんそめ@コンソメ!へるふえるみえーる@---・ベル・フェルミエール}
\index{へるふえるみえーる@ベル・フェルミエール!こんそめ@コンソメ・---}

\hypertarget{consomme-bellini}{%
\subsubsection{コンソメ・ベッリーニ}\label{consomme-bellini}}

\frsub{Consommé Bellini}

\index{consomme@consommé!bellini@--- Bellini}
\index{bellini@Bellini!consomme@consommé ---}
\index{こんそめ@コンソメ!へつりーに@---・ベッリーニ}
\index{へつりーに@ベッリーニ!こんそめ@コンソメ・---}

\hypertarget{consomme-a-la-bouchere}{%
\subsubsection{コンソメ・ブーシェール}\label{consomme-a-la-bouchere}}

\frsub{Consommé à la Bouchère}

\index{consomme@consommé!bouchere@--- Bouchère}
\index{bouchere@Bouchère!consomme@consommé ---}
\index{こんそめ@コンソメ!ふーしえーる@---・ブーシェール}
\index{ふーしえーる@ブーシェール!こんそめ@コンソメ・---}

\hypertarget{consomme-a-la-bouquetiere}{%
\subsubsection{コンソメ・ブクティエール}\label{consomme-a-la-bouquetiere}}

\frsub{Consommé à la Bouquetière}

\index{consomme@consommé!bouquetiere@--- Bouquetière}
\index{bouquetiere@Bouquetière!consomme@consommé ---}
\index{こんそめ@コンソメ!ふくていえーる@---・ブクティエール}
\index{ふくていえーる@ブクティエール!こんそめ@コンソメ・---}

\href{ガルニチュールで「ブクティエール」にしているので合わさせてください}{}

\hypertarget{consomme-a-la-brunoise}{%
\subsubsection{コンソメ・ブリュノワーズ}\label{consomme-a-la-brunoise}}

\frsub{Consommé à la Brunoise}

\index{consomme@consommé!brunoise@--- Brunoise}
\index{brunoise@Brunoise!consomme@consommé ---}
\index{こんそめ@コンソメ!ふりゆのわーす@---・ブリュノワーズ}
\index{ふりゆのわーす@ブリュノワーズ!こんそめ@コンソメ・---}

\hypertarget{consomme-carmen}{%
\subsubsection{コンソメ・カルメン}\label{consomme-carmen}}

\frsub{Consommé Carmen}

\index{consomme@consommé!carmen@--- Carmen}
\index{carmen@Carmen!consomme@consommé ---}
\index{こんそめ@コンソメ!かるめん@---・カルメン}
\index{かるめん@カルメン!こんそめ@コンソメ・---}

\href{ココからセレスティーヌまで項目のダブリがあったので削除しました。20180407五島}{}

\hypertarget{consomme-celestine}{%
\subsubsection{コンソメ・ケレスティヌス}\label{consomme-celestine}}

\frsub{Consommé Célestine}

\index{consomme@consommé!celestine@--- Célestine}
\index{celestine@Célestine!consomme@consommé ---}
\index{こんそめ@コンソメ!けれすていぬす@---・ケレスティヌス}
\index{けれすていぬす@ケレスティヌス!こんそめ@コンソメ・---}

\hypertarget{consomme-cendrillon}{%
\subsubsection{コンソメ・シンデレラ}\label{consomme-cendrillon}}

\frsub{Consommé Cendrillon}

\index{consomme@consommé!cendrillon@--- Cendrillon}
\index{cendrillon@Cendrillon!consomme@consommé ---}
\index{こんそめ@コンソメ!しんてれら@---・シンデレラ}
\index{しんてれら@シンデレラ!こんそめ@コンソメ・---}

\hypertarget{consomme-chanceliere}{%
\subsubsection{コンソメ・ションスリエール}\label{consomme-chanceliere}}

\frsub{Consommé Chancelière}

\index{consomme@consommé!chanceliere@--- Chancelière}
\index{chanceliere@Chancelière!consomme@consommé ---}
\index{こんそめ@コンソメ!しよんすりえーる@---・ションスリエール}
\index{しよんすりえーる@ションスリエール!こんそめ@コンソメ・---}

\hypertarget{consomme-au-chasseur}{%
\subsubsection{コンソメ・シャッスール}\label{consomme-au-chasseur}}

\frsub{Consommé au Chasseur}

\index{consomme@consommé!chasseur@--- Chasseur}
\index{chasseur@Chasseur!consomme@consommé ---}
\index{こんそめ@コンソメ!しやつすーる@---・シャッスール}
\index{しやつすーる@シャッスール!こんそめ@コンソメ・---}

\hypertarget{consomme-chatelaine}{%
\subsubsection{コンソメ・シャトレーヌ}\label{consomme-chatelaine}}

\frsub{Consommé Châtelaine }

\index{consomme@consommé!chatelaine@---Châtelaine}
\index{chatelaine@Châtelaine!consomme@consommé ---}
\index{こんそめ@コンソメ!しやとれーぬ@---・シャトレーヌ}
\index{しやとれーぬ@シャトレーヌ!こんそめ@コンソメ・---}

\href{以下項目のダブリがあったので削除しました}{}

\hypertarget{consomme-aux-cheveux-d-ange}{%
\subsubsection{コンソメ・ヴァミセリ入り}\label{consomme-aux-cheveux-d-ange}}

\frsub{Consommé aux Cheveux d'ange}

\index{consomme@consommé!cheveud'ange@--- Cheveux d'ange}
\index{cheveud'ange@Cheveux d'ange!consomme@consommé ---}
\index{こんそめ@コンソメ!うあみせり@---・ヴァミセリ}
\index{うあみせり@ヴァミセリ!こんそめ@コンソメ・---}

\hypertarget{consomme-colbert}{%
\subsubsection{コンソメ・コルべール}\label{consomme-colbert}}

\frsub{Consommé Colbert}

\index{consomme@consommé!colbert@--- Colbert}
\index{colbert@Colbert!consomme@consommé ---}
\index{こんそめ@コンソメ!こるへーる@---・コルベール}
\index{こるへーる@コルベール!こんそめ@コンソメ・---}

\hypertarget{consomme-colombine}{%
\subsubsection{コンソメ・コロンバン}\label{consomme-colombine}}

\frsub{Consommé Colombine}

\index{consomme@consommé!colombine@--- Colombine}
\index{colombine@Colombine!consomme@consommé ---}
\index{こんそめ@コンソメ!ころんはん@---・コロンバン}
\index{ころんはん@コロンバン!こんそめ@コンソメ・---}

\hypertarget{consomme-croute-au-pot}{%
\subsubsection{コンソメ・クルトン入り}\label{consomme-croute-au-pot}}

\frsub{Consommé Croûte au pot}

\index{consomme@consommé!crouteaupot@--- Croûte au pot}
\index{croûteaupot@Croûte au pot!consomme@consommé ---}
\index{こんそめ@コンソメ!くるとん@---・クルトン入り}
\index{くるとん@クルトン入り!こんそめ@コンソメ・---}

\hypertarget{consomme-cyrano}{%
\subsubsection{コンソメ・シラノ}\label{consomme-cyrano}}

\frsub{Consommé Cyrano}

\index{consomme@consommé!cyrano@--- Cyrano}
\index{cyrano@Cyrano!consomme@consommé ---}
\index{こんそめ@コンソメ!しらの@---・シラノ}
\index{しらの@シラノ!こんそめ@コンソメ・---}

\hypertarget{consomme-dame-blanche}{%
\subsubsection{コンソメ・ダム・ブランシュ}\label{consomme-dame-blanche}}

\frsub{Consommé Dame Blanche}

\index{consomme@consommé!dameblanche@--- Dame Blanche}
\index{dameblanche@Dame Blanche!consomme@consommé ---}
\index{こんそめ@コンソメ!たむふらんしゆ@---・ダム・ブランシュ}
\index{たむふらんしゆ@ダム・ブランシュ!こんそめ@コンソメ・---}

\hypertarget{consomme-demidoff}{%
\subsubsection{コンソメ・ドゥミドフ}\label{consomme-demidoff}}

\frsub{Consommé Demidoff}

\index{consomme@consommé!demidoff@--- Domidoff}
\index{demidoff@Demidoff!consomme@consommé ---}
\index{こんそめ@コンソメ!とうみとふ@---・ドゥミドフ}
\index{とうみとふ@ドゥミドフ!こんそめ@コンソメ・---}

\hypertarget{consomme-deslignac}{%
\subsubsection{コンソメ・デリニャック}\label{consomme-deslignac}}

\frsub{Consommé Deslignac}

\index{consomme@consommé!deslignac@--- Deslignac}
\index{deslignac@Deslignac!consomme@consommé ---}
\index{こんそめ@コンソメ!てりにやつく@---・デリニャック}
\index{てりにやつく@デリニャック!こんそめ@コンソメ・---}

\hypertarget{consomme-aux-diablotins}{%
\subsubsection{コンソメ・小悪魔風}\label{consomme-aux-diablotins}}

\frsub{Consommé aux Diablotins}

\index{consomme@consommé!diablotin@--- Diablotins}
\index{diablotin@Diablotins!consomme@consommé ---}
\index{こんそめ@コンソメ!こあくまふう@---・小悪魔風}
\index{こあくまふう@小悪魔風!こんそめ@コンソメ・---}

\hypertarget{consomme-diane}{%
\subsubsection{コンソメ・ディアーヌ}\label{consomme-diane}}

\frsub{Consommé Diane}

\index{consomme@consommé!diane@--- Diane}
\index{diane@Diane!consomme@consommé ---}
\index{こんそめ@コンソメ!ていあーぬ@---・ディアーヌ}
\index{ていあーぬ@ディアーヌ!こんそめ@コンソメ・---}

\href{ソース・ディアーヌとしているのに合わさせてもらいました20180407五島}{}

\hypertarget{consomme-diplomate}{%
\subsubsection{コンソメ・ディプロマ}\label{consomme-diplomate}}

\frsub{Consommé Diplomate}

\index{consomme@consommé!diplomate@--- Diplomate}
\index{diplomate@Diplomate!consomme@consommé ---}
\index{こんそめ@コンソメ!ていふろま@---・ディプロマ}
\index{ていふろま@ディプロマ!こんそめ@コンソメ・---}

\hypertarget{consomme-divette}{%
\subsubsection{コンソメ・ディヴェット}\label{consomme-divette}}

\frsub{Consommé Divette}

\index{consomme@consommé!divette@--- Divette}
\index{divette@Divette!consomme@consommé ---}
\index{こんそめ@コンソメ!ていうえつと@---・ディヴェット}
\index{ていうえつと@ディヴェット!こんそめ@コンソメ・---}

\hypertarget{consomme-dominicaine}{%
\subsubsection{コンソメ・ドミニコ修道会風}\label{consomme-dominicaine}}

\frsub{Consommé Dominicaine}

\index{consomme@consommé!dominicaine@--- Dominicaine}
\index{dominicaine@Dominicaine!consomme@consommé ---}
\index{こんそめ@コンソメ!とみにこしゆうとうかいふう@---・ドミニコ修道会風}
\index{とみにこしゆうとうかいふう@ドミニコ修道会風!こんそめ@コンソメ・---}

\hypertarget{consomme-doria}{%
\subsubsection{コンソメ・ドリア}\label{consomme-doria}}

\frsub{Consommé Doria}

\index{consomme@consommé!doria@--- Doria}
\index{doria@Doria!consomme@consommé ---}
\index{こんそめ@コンソメ!とりあ@---・ドリア}
\index{とりあ@ドリア!こんそめ@コンソメ・---}

\hypertarget{consomme-douglas}{%
\subsubsection{コンソメ・ドゥーグラ}\label{consomme-douglas}}

\frsub{Consommé Douglas}

\index{consomme@consommé!douglas@--- Douglas}
\index{douglas@Douglas!consomme@consommé ---}
\index{こんそめ@コンソメ!とうーくら@---・ドゥーグラ}
\index{とうーくら@ドゥーグラ!こんそめ@コンソメ・---}

\hypertarget{consomme-dubarry}{%
\subsubsection{コンソメ・デュバリー}\label{consomme-dubarry}}

\frsub{Consommé Dubarry}

\index{consomme@consommé!dubarry@--- Dubarry}
\index{dubarry@Dubarry!consomme@consommé ---}
\index{こんそめ@コンソメ!てゆはりー@---・デュバリー}
\index{てゆはりー@デュバリー!こんそめ@コンソメ・---}

\hypertarget{consomme-a-l-ecossaise}{%
\subsubsection{コンソメ・スコットランド風}\label{consomme-a-l-ecossaise}}

\frsub{Consommé à l'Ecossaise}

\index{consomme@consommé!ecossaise@--- Ecossaise}
\index{ecossaise@Ecossaise!consomme@consommé ---}
\index{こんそめ@コンソメ!すこつとらんとふう@---・スコットランド風}
\index{すこつとらんとふう@スコットランド風!こんそめ@コンソメ・---}

\hypertarget{consomme-edouard-vii}{%
\subsubsection{コンソメ・エドワード7世}\label{consomme-edouard-vii}}

\frsub{Consommé Edouard VII}

\index{consomme@consommé!edouard vii@--- Edouard VII}
\index{edouard vii@Edouard VII!consomme@consommé ---}
\index{こんそめ@コンソメ!えとわーとななせい@---・エドワード7世}
\index{えとわーとななせい@エドワード7世!こんそめ@コンソメ・---}

\hypertarget{consomme-flavigny}{%
\subsubsection{コンソメ・フラヴィニー}\label{consomme-flavigny}}

\frsub{Consommé Flavigny}

\index{consomme@consommé!flavigny@--- Flavigny}
\index{flavigny@Flavigny!consomme@consommé ---}
\index{こんそめ@コンソメ!ふらういにー@---・フラヴィニー}
\index{ふらういにー@フラヴィニー!こんそめ@コンソメ・---}

\hypertarget{consomme-floreal}{%
\subsubsection{コンソメ・フロレアル}\label{consomme-floreal}}

\frsub{Consommé Floréal}

\index{consomme@consommé!floreal@--- Floréal}
\index{floréal@Floréal!consomme@consommé ---}
\index{こんそめ@コンソメ!ふろれある@---・フロレアル}
\index{ふろれある@フロレアル!こんそめ@コンソメ・---}

\href{パッチのコメントに書いたとおりの理由からフロレアルとさせていただきました。20180407五島}{}

\hypertarget{consomme-florentine}{%
\subsubsection{コンソメ・フロランタン}\label{consomme-florentine}}

\frsub{Consommé Florentine}

\index{consomme@consommé!frolentine@--- Florentine}
\index{florentine@Florentine!consomme@consommé ---}
\index{こんそめ@コンソメ!ふろらんたん@---・フロランタン}
\index{ふろらんたん@フロランタン!こんそめ@コンソメ・---}

\hypertarget{consomme-florian}{%
\subsubsection{コンソメ・フロリアン}\label{consomme-florian}}

\frsub{Consommé Florian}

\index{consomme@consommé!florian@--- Florian}
\index{florian@Florian!consomme@consommé ---}
\index{こんそめ@コンソメ!ふろりあん@---・フロリアン}
\index{ふろりあん@フロリアン!こんそめ@コンソメ・---}

\hypertarget{consomme-a-la-gauloise}{%
\subsubsection{コンソメ・ガリア風}\label{consomme-a-la-gauloise}}

\frsub{Consommé à la Gauloise}

\index{consomme@consommé!gauloise@--- Gauloise}
\index{gauloise@Gauloise!consomme@consommé ---}
\index{こんそめ@コンソメ!かりあふう@---・ガリア風}
\index{かりあふう@ガリア風!こんそめ@コンソメ・---}

\hypertarget{consomme-georges-V}{%
\subsubsection{コンソメ・ジョルジュⅤ世風}\label{consomme-georges-V}}

\frsub{Consommé Georges V}

\index{consomme@consommé!georgesv@--- Georges V}
\index{georgesv@Georges V!consomme@consommé ---}
\index{こんそめ@コンソメ!しよるしゆこせいふう@---・ジョルジュⅤ世風}
\index{しよるしゆこせいふう@ジョルジュⅤ世風!こんそめ@コンソメ・---}

\hypertarget{consomme-germinal}{%
\subsubsection{コンソメ・ジェルミナル}\label{consomme-germinal}}

\frsub{Consommé Germinal}

\index{consomme@consommé!germinal@--- Germinal}
\index{germinal@Germinal!consomme@consommé ---}
\index{こんそめ@コンソメ!しえるみなる@---・ジェルミナル}
\index{しえるみなる@ジェルミナル!こんそめ@コンソメ・---}

\hypertarget{consomme-gladiateur}{%
\subsubsection{コンソメ・剣闘士風}\label{consomme-gladiateur}}

\frsub{Consommé Gladiateur}

\index{consomme@consommé!gladiateur@--- Gladiateur}
\index{gladiateur@Gladiateur!consomme@consommé ---}
\index{こんそめ@コンソメ!けんとうしふう@---・剣闘士風}
\index{けんとうしふう@剣闘士風!こんそめ@コンソメ・---}

\hypertarget{consomme-grimaldi}{%
\subsubsection{コンソメ・グリマルディ}\label{consomme-grimaldi}}

\frsub{Consommé Grimaldi}

\index{consomme@consommé!grimaldi@--- Grimaldi}
\index{grimaldi@Grimaldi!consomme@consommé ---}
\index{こんそめ@コンソメ!くりまるてい@---・くりまるてい}
\index{グリマルディ@くりまるてい!こんそめ@コンソメ・---}

\hypertarget{consomme-helene}{%
\subsubsection{コンソメ・エレーヌ}\label{consomme-helene}}

\frsub{Consommé Helène}

\index{consomme@consommé!helène@--- Helène}
\index{helène@Helène!consomme@consommé ---}
\index{こんそめ@コンソメ!えれーぬ@---・エレーヌ}
\index{えれーぬ@エレーヌ!こんそめ@コンソメ・---}

\hypertarget{consomme-henriette}{%
\subsubsection{コンソメ・アンリエッテ}\label{consomme-henriette}}

\frsub{Consommé Henriette}

\index{consomme@consommé!henriette@--- Henriette}
\index{henriette@Henriette!consomme@consommé ---}
\index{こんそめ@コンソメ!あんりえつて@---・アンリエッテ}
\index{アンリエッテ@あんりえつて!こんそめ@コンソメ・---}

\hypertarget{consomme-a-indienne}{%
\subsubsection{コンソメ・インド風}\label{consomme-a-indienne}}

\frsub{Consommé à l'Indienne}

\index{consomme@consommé!indien(ne)@--- à l'Indienne}
\index{indien(ne)@Indienne!consomme@consommé à l' ---}
\index{こんそめ@コンソメ!いんとふう@---・インド風}
\index{いんとふう@インド風!こんそめ@コンソメ・---}

\hypertarget{consomme-a-l-infante}{%
\subsubsection{コンソメ・親王風}\label{consomme-a-l-infante}}

\frsub{Consommé à l'Infante}

\index{consomme@consommé!infante@--- à l'Infante}
\index{infante@Infante!consomme@consommé à l' ---}
\index{こんそめ@コンソメ!しんのうふう@---・親王風}
\index{しんのうふう@親王風!こんそめ@コンソメ・---}

\hypertarget{consomme-isabelle-de-france}{%
\subsubsection{コンソメ・イザベル・ド・フランス風}\label{consomme-isabelle-de-france}}

\frsub{Consommé Isabelle de France}

\index{consomme@consommé!isabelledefrance@--- Isabelle de France}
\index{isabelledefrance@Isabelle de France!consomme@consommé ---}
\index{こんそめ@コンソメ!いさへるとふらんすふう@---・イザベル・ド・フランス風}
\index{いさへるとふらんすふう@イザベル・ド・フランス風!こんそめ@コンソメ・---}

\hypertarget{consomme-ivan}{%
\subsubsection{コンソメ・イヴァン}\label{consomme-ivan}}

\frsub{Consommé Ivan}

\index{consomme@consommé!ivan@--- Ivan}
\index{ivan@Ivan!consomme@consommé ---}
\index{こんそめ@コンソメ!いうあん@---・イヴァン}
\index{いうあん@イヴァン!こんそめ@コンソメ・---}

\hypertarget{consomme-jeanne-garnier}{%
\subsubsection{コンソメ・ジャン・ガルニエ}\label{consomme-jeanne-garnier}}

\frsub{Consommé Jeanne Garnier}

\index{consomme@consommé!jeanne garnier@--- Jeanne Garnier}
\index{jeanne garnier@Jeanne Garnier!consomme@consommé ---}
\index{こんそめ@コンソメ!しやんかるにえ@---・ジャン・ガルニエ}
\index{しやんかるにえ@ジャン・ガルニエ!こんそめ@コンソメ・---}

\hypertarget{consomme-judic}{%
\subsubsection{コンソメ・ジュディク}\label{consomme-judic}}

\frsub{Consommé Judic}

\index{consomme@consommé!judic@--- Judic}
\index{judic@Judic!consomme@consommé ---}
\index{こんそめ@コンソメ!しゆていく@---・ジュディク}
\index{しゆていく@ジュディク!こんそめ@コンソメ・---}

\hypertarget{potage-julienne}{%
\subsubsection{ポタージュ・ジュリエンヌ}\label{potage-julienne}}

\frsub{Potage Julienne}

\index{potage@potage!julienne@--- Julienne}
\index{julienne@Julienne!potage@potage ---}
\index{ほたーしゆ@ポタージュ!しゆりえんぬ@---・ジュリエンヌ}
\index{しゆりえんぬ@ジュリエンヌ!ほたーしゆ@ポタージュ・---}

\hypertarget{consomme-juliette}{%
\subsubsection{コンソメ・ジュリエット}\label{consomme-juliette}}

\frsub{Consommé Juliette}

\index{consomme@consommé!juliette@--- Juliette}
\index{juliette@Juliette!consomme@consommé ---}
\index{こんそめ@コンソメ!しゆりえつと@---・ジュリエット}
\index{しゆりえつと@ジュリエット!こんそめ@コンソメ・---}

\hypertarget{consomme-kleber}{%
\subsubsection{コンソメ・クレーベル}\label{consomme-kleber}}

\frsub{Consommé Kléber}

\index{consomme@consommé!kleber@--- Kléber}
\index{kleber@Kléber!consomme@consommé ---}
\index{こんそめ@コンソメ!くれーへる@---・クレーベル}
\index{くれーへる@クレーベル!こんそめ@コンソメ・---}

\hypertarget{consomme-la-perouse}{%
\subsubsection{コンソメ・ラ・ペルーズ風}\label{consomme-la-perouse}}

\frsub{Consommé La Pérouse}

\index{consomme@consommé!laperouse@--- La Pérouse}
\index{laperouse@La Pérouse!consomme@consommé ---}
\index{こんそめ@コンソメ!らへるーすふう@---・ラ・ペルーズ風}
\index{らへるーすふう@ラ・ペルーズ風!こんそめ@コンソメ・---}

\hypertarget{consomme-lorette}{%
\subsubsection{コンソメ・ロレット}\label{consomme-lorette}}

\frsub{Consommé Lorette}

\index{consomme@consommé!lorette@--- Lorette}
\index{lorette@Lorette!consomme@consommé ---}
\index{こんそめ@コンソメ!ろれつと@---・ロレット}
\index{ろれつと@ロレット!こんそめ@コンソメ・---}

\hypertarget{consomme-lucette}{%
\subsubsection{コンソメ・ルセット}\label{consomme-lucette}}

\frsub{Consommé Lucette}

\index{consomme@consommé!lucette@--- Lucette}
\index{lucette@Lucette!consomme@consommé ---}
\index{こんそめ@コンソメ!るせつと@---・ルセット}
\index{るせつと@ルセット!こんそめ@コンソメ・---}

\hypertarget{consomme-lucullus}{%
\subsubsection{コンソメ・ルクッルス}\label{consomme-lucullus}}

\frsub{Consommé Lucullus}

\index{consomme@consommé!lucullus@--- Lucullus}
\index{lucullus@Lucullus!consomme@consommé ---}
\index{こんそめ@コンソメ!るくつるす@---・ルクッルス}
\index{るくつるす@ルクッルス!こんそめ@コンソメ・---}

\hypertarget{consomme-a-la-madrilene}{%
\subsubsection{コンソメ・マドリッド風}\label{consomme-a-la-madrilene}}

\frsub{Consommé à la Madrilène}

\index{consomme@consommé!madrilene@--- Madrilène}
\index{madrilene@Madrilène!consomme@consommé à la ---}
\index{こんそめ@コンソメ!まとりつとふう@---・マドリッド風}
\index{まとりつとふう@マドリッド風!こんそめ@コンソメ・---}

\hypertarget{consomme-maintenon}{%
\subsubsection{コンソメ・マントノン}\label{consomme-maintenon}}

\frsub{Consommé Maintenon}

\index{consomme@consommé!maintenon@--- Maintenon}
\index{maintenon@Maintenon!consomme@consommé ---}
\index{こんそめ@コンソメ!まんとのん@---・マントノン}
\index{まんとのん@マントノン!こんそめ@コンソメ・---}

\hypertarget{consomme-messaline}{%
\subsubsection{コンソメ・メッザリーヌ}\label{consomme-messaline}}

\frsub{Consommé Messaline}

\index{consomme@consommé!messaline@--- Messaline}
\index{messaline@Messaline!consomme@consommé ---}
\index{こんそめ@コンソメ!めつさりーぬ@---・メッザリーヌ}
\index{めつさりーぬ@メッザリーヌ!こんそめ@コンソメ・---}

\hypertarget{consomme-midinette}{%
\subsubsection{コンソメ・ミディネット}\label{consomme-midinette}}

\frsub{Consommé Midinette}

\index{consomme@consommé!midinette@--- Midinette}
\index{midinette@Midinette!consomme@consommé ---}
\index{こんそめ@コンソメ!みていねつと@---・ミディネット}
\index{みていねつと@ミディネット!こんそめ@コンソメ・---}

\hypertarget{consomme-mikado}{%
\subsubsection{コンソメ・ミカド}\label{consomme-mikado}}

\frsub{Consommé Mikado}

\index{consomme@consommé!mikado@--- Mikado}
\index{mikado@Mikado!consomme@consommé ---}
\index{こんそめ@コンソメ!みかと@---・ミカド}
\index{みかと@ミカド!こんそめ@コンソメ・---}

\hypertarget{consomme-mireille}{%
\subsubsection{コンソメ・ミレイユ}\label{consomme-mireille}}

\frsub{Consommé Mireille}

\index{consomme@consommé!mireille@--- Mireille}
\index{mireille@Mireille!consomme@consommé ---}
\index{こんそめ@コンソメ!みれいゆ@---・ミレイユ}
\index{みれいゆ@ミレイユ!こんそめ@コンソメ・---}

\hypertarget{consomme-mirette}{%
\subsubsection{コンソメ・ミレット}\label{consomme-mirette}}

\frsub{Consommé Mirette}

\index{consomme@consommé!mirette@--- Mirette}
\index{mirette@Mirette!consomme@consommé ---}
\index{こんそめ@コンソメ!みれつと@---・ミレット}
\index{みれつと@ミレット!こんそめ@コンソメ・---}

\hypertarget{consomme-mistral}{%
\subsubsection{コンソメ・ミストラル}\label{consomme-mistral}}

\frsub{Consommé Mistral}

\index{consomme@consommé!mistral@--- Mistral}
\index{mistral@Mistral!consomme@consommé ---}
\index{こんそめ@コンソメ!みすとらる@---・ミストラル}
\index{みすとらる@ミストラル!こんそめ@コンソメ・---}

\hypertarget{consomme-monsigny}{%
\subsubsection{コンソメ・モンシニー}\label{consomme-monsigny}}

\frsub{Consommé Monsigny}

\index{consomme@consommé!monsigny@--- Monsigny}
\index{monsigny@Monsigny!consomme@consommé ---}
\index{こんそめ@コンソメ!もんしにー@---・モンシニー}
\index{もんしにー@モンシニー!こんそめ@コンソメ・---}

\hypertarget{consomme-montespan}{%
\subsubsection{コンソメ・モンテスパン}\label{consomme-montespan}}

\frsub{Consommé Montespan}

\index{consomme@consommé!montespan@--- Montespan}
\index{montespan@Montespan!consomme@consommé ---}
\index{こんそめ@コンソメ!もんてすはん@---・モンテスパン}
\index{もんてすはん@モンテスパン!こんそめ@コンソメ・---}

\hypertarget{consomme-montmorency}{%
\subsubsection{コンソメ・モンモランシー}\label{consomme-montmorency}}

\frsub{Consommé Montmorency}

\index{consomme@consommé!montmorency@--- Montmorency}
\index{montmorency@Montmorency!consomme@consommé ---}
\index{こんそめ@コンソメ!もんもらんしー@---・モンモランシー}
\index{もんもらんしー@モンモランシー!こんそめ@コンソメ・---}

\hypertarget{consomme-murat}{%
\subsubsection{コンソメ・ミュラ}\label{consomme-murat}}

\frsub{Consommé Murat}

\index{consomme@consommé!murat@--- Murat}
\index{murat@Murat!consomme@consommé ---}
\index{こんそめ@コンソメ!みゆら@---・ミュラ}
\index{みゆら@ミュラ!こんそめ@コンソメ・---}

\hypertarget{consomme-murillo}{%
\subsubsection{コンソメ・ムリーリョ}\label{consomme-murillo}}

\frsub{Consommé Murillo}

\index{consomme@consommé!murillo@--- Murillo}
\index{murillo@Murillo!consomme@consommé ---}
\index{こんそめ@コンソメ!むりーりよ@---・ムリーリョ}
\index{むりーりよ@ムリーリョ!こんそめ@コンソメ・---}

\hypertarget{consomme-nana}{%
\subsubsection{コンソメ・ナナ}\label{consomme-nana}}

\frsub{Consommé Nana}

\index{consomme@consommé!nana@--- Nana}
\index{nana@Nana!consomme@consommé ---}
\index{こんそめ@コンソメ!なな@---・ナナ}
\index{なな@ナナ!こんそめ@コンソメ・---}

\hypertarget{consomme-nantua}{%
\subsubsection{コンソメ・ナンチュア}\label{consomme-nantua}}

\frsub{Consommé Nantua}

\index{consomme@consommé!nantua@--- Nantua}
\index{nantua@Nantua!consomme@consommé ---}
\index{こんそめ@コンソメ!なんちゆあ@---・ナンチュア}
\index{なんちゆあ@ナンチュア!こんそめ@コンソメ・---}

\hypertarget{consomme-a-la-neige-de-florence}{%
\subsubsection{コンソメ・ネージュ・ド・フローレンス}\label{consomme-a-la-neige-de-florence}}

\frsub{Consommé à la Neige de Florence}

\index{consomme@consommé!neigedeflorence@--- à la Neige de Florence}
\index{neigedeflorence@Neige de Florence!consomme@consommé à la ---}
\index{こんそめ@コンソメ!ねーしゆとふろーれんす@---・ネージュ・ド・フローレンス}
\index{ねーしゆとふろーれんす@ネージュ・ド・フローレンス!こんそめ@コンソメ・---}

\hypertarget{consomme-nelson}{%
\subsubsection{コンソメ・ネルソン}\label{consomme-nelson}}

\frsub{Consommé Nelson}

\index{consomme@consommé!nelson@--- Nelson}
\index{nelson@Nelson!consomme@consommé ---}
\index{こんそめ@コンソメ!ねるそん@---・ネルソン}
\index{ねるそん@ネルソン!こんそめ@コンソメ・---}

\hypertarget{consomme-nesselrode}{%
\subsubsection{コンソメ・ネッセルローデ}\label{consomme-nesselrode}}

\frsub{Consommé Nesselrode}

\index{consomme@consommé!nesselrode@--- Nesselrode}
\index{nesselrode@Nesselrode!consomme@consommé ---}
\index{こんそめ@コンソメ!ねつせるろーて@---・ネッセルローデ}
\index{ねつせるろーて@ネッセルローデ!こんそめ@コンソメ・---}

\hypertarget{consomme-aux-nids-d-hirondelles}{%
\subsubsection{コンソメ・ツバメの巣入り}\label{consomme-aux-nids-d-hirondelles}}

\frsub{Consommé aux Nids d'hirondelles}

\index{consomme@consommé!niddhirondelle@--- aux-nids-d-hirondelles}
\index{niddhirondelle@Nids d'hirondelles!consomme@consommé aux ---}
\index{こんそめ@コンソメ!つはめのす@---・ツバメの巣入り}
\index{つはめのす@ツバメの巣入り!こんそめ@コンソメ・---}

\hypertarget{consomme-ninon}{%
\subsubsection{コンソメ・ニノン}\label{consomme-ninon}}

\frsub{Consommé Ninon}

\index{consomme@consommé!ninon@--- Ninon}
\index{ninon@Ninon!consomme@consommé ---}
\index{こんそめ@コンソメ!にのん@---・ニノン}
\index{にのん@ニノン!こんそめ@コンソメ・---}

\hypertarget{consomme-a-l-orge-perle}{%
\subsubsection{コンソメ・大麦入り}\label{consomme-a-l-orge-perle}}

\frsub{Consommé à l'Orge perlé}

\index{consomme@consommé!orgeperle@--- à l'Orge perlé}
\index{orgeperle@Orge perlé!consomme@consommé à l'---}
\index{こんそめ@コンソメ!おおむき@---・大麦入り}
\index{おおむき@大麦入り!こんそめ@コンソメ・---}

\hypertarget{consomme-a-l-oriental}{%
\subsubsection{コンソメ・オリエンタル}\label{consomme-a-l-oriental}}

\frsub{Consommé à l'Orientale}

\index{consomme@consommé!oriental@--- à l'Orientale}
\index{oriental@Orientale!consomme@consommé à l'---}
\index{こんそめ@コンソメ!おりえんたる@---・オリエンタル}
\index{おりえんたる@オリエンタル!こんそめ@コンソメ・---}

\hypertarget{consomme-olga}{%
\subsubsection{コンソメ・オルガ}\label{consomme-olga}}

\frsub{Consommé Olga}

\index{consomme@consommé!olga@--- Olga}
\index{olga@Olga!consomme@consommé ---}
\index{こんそめ@コンソメ!おるか@---・オルガ}
\index{おるか@オルガ!こんそめ@コンソメ・---}

\hypertarget{consomme-a-la-d-orleans}{%
\subsubsection{コンソメ・オルレアン風}\label{consomme-a-la-d-orleans}}

\frsub{Consommé à la d'Orléans}

\index{consomme@consommé!orlean@--- à la d'Orléans}
\index{orlean@Orléans!consomme@consommé à la d'---}
\index{こんそめ@コンソメ!おるれあんふう@---・オルレアン風}
\index{おるれあんふう@オルレアン風!こんそめ@コンソメ・---}

\hypertarget{consomme-orloff}{%
\subsubsection{コンソメ・オルロフ}\label{consomme-orloff}}

\frsub{Consommé Orloff}

\index{consomme@consommé!orloff@--- Orloff}
\index{rloff@Orloff!consomme@consommé ---}
\index{こんそめ@コンソメ!おるろふ@---・オルロフ}
\index{おるろふ@オルロフ!こんそめ@コンソメ・---}

\hypertarget{consomme-d-orsay}{%
\subsubsection{コンソメ・オルセイ}\label{consomme-d-orsay}}

\frsub{Consommé d'Orsay}

\index{consomme@consommé!orsay@--- d'Orsay}
\index{orsay@Orsay!consomme@consommé d'---}
\index{こんそめ@コンソメ!おるせい@---・オルセイ}
\index{おるせい@オルセイ!こんそめ@コンソメ・---}

\hypertarget{consomme-otello}{%
\subsubsection{コンソメ・オテッロ}\label{consomme-otello}}

\frsub{Consommé Otello}

\index{consomme@consommé!otello@--- Otello}
\index{otello@Otello!consomme@consommé ---}
\index{こんそめ@コンソメ!おてつろ@---・オテッロ}
\index{おてつろ@オテッロ!こんそめ@コンソメ・---}

\hypertarget{consomme-otero}{%
\subsubsection{コンソメ・オテロ}\label{consomme-otero}}

\frsub{Consommé Otero}

\index{consomme@consommé!otero@--- Otero}
\index{otero@Otero!consomme@consommé ---}
\index{こんそめ@コンソメ!おてろ@---・オテロ}
\index{おてろ@オテロ!こんそめ@コンソメ・---}

\hypertarget{consomme-palestro}{%
\subsubsection{コンソメ・パレストロ}\label{consomme-palestro}}

\frsub{Consommé Palestro}

\index{consomme@consommé!palestro@--- Palestro}
\index{palestro@Palestro!consomme@consommé ---}
\index{こんそめ@コンソメ!はれすとろ@---・パレストロ}
\index{はれすとろ@パレストロ!こんそめ@コンソメ・---}

\hypertarget{consomme-aux-pates-diverses}{%
\subsubsection{コンソメ・パスタ入り}\label{consomme-aux-pates-diverses}}

\frsub{Consommé aux Pâtes diverses}

\index{consomme@consommé!patediverse@--- aux Pâtes diverses}
\index{patediverse@Pâtes diverses!consomme@consommé aux ---}
\index{こんそめ@コンソメ!はすたいり@---・パスタ入り}
\index{はすたいり@パスタ入り!こんそめ@コンソメ・---}

\hypertarget{consomme-petite-mariee}{%
\subsubsection{コンソメ・新婦風}\label{consomme-petite-mariee}}

\frsub{Consommé Petite Mariée}

\index{consomme@consommé!petitemariee@--- Petite Mariée}
\index{petitemariee@Petite Mariée!consomme@consommé ---}
\index{こんそめ@コンソメ!しんふふう@---・新婦風}
\index{しんふふう@新婦風!こんそめ@コンソメ・---}

\hypertarget{petite-marmite}{%
\subsubsection{プティット・マルミット}\label{petite-marmite}}

\frsub{Petite Marmite}

\index{petite marmite@Petite Marmite}
\index{こんそめ@コンソメ!ふていつとまるみつと@プティットマルミット}
\index{ふていつとまるみつと@プティットマルミット}

\hypertarget{consomme-polaire}{%
\subsubsection{コンソメ・ポレール}\label{consomme-polaire}}

\frsub{Consommé Polaire}

\index{consomme@consommé!polaire@--- Polaire}
\index{polaire@Polaire!consomme@consommé ---}
\index{こんそめ@コンソメ!ほれーる@---・ポレール}
\index{ほれーる@ポレール!こんそめ@コンソメ・---}

\hypertarget{consomme-pompadour}{%
\subsubsection{コンソメ・ポンパドール夫人風}\label{consomme-pompadour}}

\frsub{Consommé Pompadour}

\index{consomme@consommé!pompadour@--- Pompadour}
\index{pompadour@Pompadour!consomme@consommé ---}
\index{こんそめ@コンソメ!ほんはとーるふしんふう@---・ポンパドール夫人風}
\index{ほんはとーるふしんふう@ポンパドール夫人風!こんそめ@コンソメ・---}

\hypertarget{consomme-portalis}{%
\subsubsection{コンソメ・ポルタリ風}\label{consomme-portalis}}

\frsub{Consommé Portalis}

\index{consomme@consommé!portalis@--- Portalis}
\index{portalis@Portalis!consomme@consommé ---}
\index{こんそめ@コンソメ!ほるたりふう@---・ポルタリ風}
\index{ほるたりふう@ポルタリ風!こんそめ@コンソメ・---}

\hypertarget{consomme-printanier}{%
\subsubsection{コンソメ・プランタニエ}\label{consomme-printanier}}

\frsub{Consommé Printanier}

\index{consomme@consommé!printanier@--- Printanier}
\index{printanier@Printanier!consomme@consommé ---}
\index{こんそめ@コンソメ!ふらんたにえ@---・プランタニエ}
\index{ふらんたにえ@プランタニエ!こんそめ@コンソメ・---}

\hypertarget{consomme-aux-quenelles-a-la-moelle}{%
\subsubsection{コンソメ・モワルのクネル入り}\label{consomme-aux-quenelles-a-la-moelle}}

\frsub{Consommé aux Quenelles à la moelle}

\index{consomme@consommé!aux quenelles a la moelle@--- aux Quenelles à la moelle}
\index{quenelles a la moelle@Quenelles à la moelle!consomme@consommé aux ---}
\index{こんそめ@コンソメ!もわるのくねるいり@---・モワルのクネル入り}
\index{もわるのくねるいり@モワルのクネル入り!こんそめ@コンソメ・---}

\hypertarget{potage-queue-de-boeuf-a-la-francaise}{%
\subsubsection{牛テールのコンソメ・アラフランセズ}\label{potage-queue-de-boeuf-a-la-francaise}}

\frsub{Potage Queue de Boeuf à la française}

\index{potage queue de boeuf a la francaise@Potage Queue de Boeuf à la française}
\index{こんそめ@コンソメ!きゆうてーるのこんそめあらふらんせす@---・牛テールのコンソメアラフランセズ}
\index{きゆうてーるのこんそめあらふらんせす@牛テールのコンソメアラフランセズ}

\hypertarget{consomme-rabelais}{%
\subsubsection{コンソメ・ラブレ}\label{consomme-rabelais}}

\frsub{Consommé Rabelais}

\index{consomme@consommé!rabelais@--- Rabelais}
\index{rabelais@Rabelais!consomme@consommé ---}
\index{こんそめ@コンソメ!らふれ@---・ラブレ}
\index{らふれ@ラブレ!こんそめ@コンソメ・---}

\hypertarget{consomme-rachel}{%
\subsubsection{コンソメ・ラケル}\label{consomme-rachel}}

\frsub{Consommé Rachel}

\index{consomme@consommé!rachel@--- Rachel}
\index{rachel@Rachel!consomme@consommé ---}
\index{こんそめ@コンソメ!らける@---・ラケル}
\index{らける@ラケル!こんそめ@コンソメ・---}

\hypertarget{consomme-aux-raviolis}{%
\subsubsection{コンソメ・ラヴィオリ入り}\label{consomme-aux-raviolis}}

\frsub{Consommé aux Raviolis}

\index{consomme@consommé!au ravioli@--- au(x) ravioli(s)}
\index{ravioli@!consomme@consommé aux ---(s)}
\index{こんそめ@コンソメ!らういおりいり@---・ラヴィオリ入り}
\index{らういおりいり@ラヴィオリ入り!こんそめ@コンソメ・---}

\hypertarget{consomme-recamier}{%
\subsubsection{コンソメ・レカミエ}\label{consomme-recamier}}

\frsub{Consommé Récamier}

\index{consomme@consommé!recamier@--- Récamier}
\index{recamier@Récamier!consomme@consommé ---}
\index{こんそめ@コンソメ!れかみえ@---・レカミエ}
\index{れかみえ@レカミエ!こんそめ@コンソメ・---}

\hypertarget{consomme-a-la-reine}{%
\subsubsection{コンソメ・女王風}\label{consomme-a-la-reine}}

\frsub{Consommé à la Reine}

\index{consomme@consommé!a la reine@--- à la Reine}
\index{reine@Reine!consomme@consommé à la ---}
\index{こんそめ@コンソメ!しよおうふう@---・女王風}
\index{しよおうふう@女王風!こんそめ@コンソメ・---}

\hypertarget{consomme-renaissance}{%
\subsubsection{コンソメ・ルネッサンス}\label{consomme-renaissance}}

\frsub{Consommé Renaissance}

\index{consomme@consommé!renaissance@--- Renaissance}
\index{renaissance@Renaissance!consomme@consommé ---}
\index{こんそめ@コンソメ!るねつさんす@---・ルネッサンス}
\index{るねつさんす@ルネッサンス!こんそめ@コンソメ・---}

\hypertarget{consomme-rossini}{%
\subsubsection{コンソメ・ロッシーニ}\label{consomme-rossini}}

\frsub{Consommé Rossini}

\index{consomme@consommé!rossini@--- Rossini}
\index{rossini@Rossini!consomme@consommé ---}
\index{こんそめ@コンソメ!ろつしーに@---・ロッシーニ}
\index{ろつしーに@ロッシーニ!こんそめ@コンソメ・---}

\hypertarget{consomme-a-la-royale}{%
\subsubsection{コンソメ・ロワイヤル}\label{consomme-a-la-royale}}

\frsub{Consommé à la Royale}

\index{consomme@consommé!royale@--- à la Royale}
\index{royale@Royale!consomme@consommé à la ---}
\index{こんそめ@コンソメ!ろわいやる@---・ロワイヤル}
\index{ろわいやる@ロワイヤル!こんそめ@コンソメ・---}

\hypertarget{consomme-au-sagou}{%
\subsubsection{コンソメ・サゴ澱粉入り}\label{consomme-au-sagou}}

\frsub{Consommé au Sagou}

\index{consomme@consommé!sagou@--- au Sagou}
\index{sagou@Sagou!consomme@consommé au ---}
\index{こんそめ@コンソメ!さこてんふんいり@---・サゴ澱粉入り}
\index{さこてんふんいり@サゴ澱粉入り!こんそめ@コンソメ・---}

\hypertarget{consomme-saint-hubert}{%
\subsubsection{コンソメ・サンテュベール}\label{consomme-saint-hubert}}

\frsub{Consommé Saint Hubert}

\index{consomme@consommé!saint hubert@--- Saint Hubert}
\index{saint hubert@Saint Hubert!consomme@consommé ---}
\index{こんそめ@コンソメ!さんてゆべーる@---・サンテュベール}
\index{さんてゆべーる@サンテュベール!こんそめ@コンソメ・---}

\hypertarget{consomme-au-salep}{%
\subsubsection{コンソメ・サレプ粉入り}\label{consomme-au-salep}}

\frsub{Consommé au Salep}

\index{consomme@consommé!salep@--- au Salep}
\index{salep@Salep!consomme@consommé au ---}
\index{こんそめ@コンソメ!されふこいり@---・サレプ粉入り}
\index{されふこいり@サレプ粉入り!こんそめ@コンソメ・---}

\hypertarget{consomme-sapho}{%
\subsubsection{コンソメ・サフォ}\label{consomme-sapho}}

\frsub{Consommé Sapho}

\index{consomme@consommé!sapho@--- Sapho}
\index{sapho@Sapho!consomme@consommé ---}
\index{こんそめ@コンソメ!さふお@---・サフォ}
\index{さふお@サフォ!こんそめ@コンソメ・---}

\hypertarget{potage-sarah-bernhardt}{%
\subsubsection{コンソメ・サラ・ベルンハルト風}\label{potage-sarah-bernhardt}}

\frsub{Potage Sarah Bernhardt}

\index{potage sarah bernhardt@Potage Sarah Bernhardt}
\index{こんそめ@コンソメ!さらへるんはるとふう@---・サラ・ベルンハルト風}
\index{さらへるんはるとふう@サラ・ベルンハルト風!こんそめ@コンソメ・---}

\hypertarget{consomme-severine}{%
\subsubsection{コンソメ・セヴェリン}\label{consomme-severine}}

\frsub{Consommé Séverine}

\index{consomme@consommé!severine@--- Séverine}
\index{severine@Séverine!consomme@consommé ---}
\index{こんそめ@コンソメ!せうえりん@---・セヴェリン}
\index{せうえりん@セヴェリン!こんそめ@コンソメ・---}

\hypertarget{consomme-sevigne}{%
\subsubsection{コンソメ・セヴィニェ}\label{consomme-sevigne}}

\frsub{Consommé Sévigné}

\index{consomme@consommé!sevigne@--- Sévigné}
\index{sevigne@Sévigné!consomme@consommé ---}
\index{こんそめ@コンソメ!せういにえ@---・セヴィニェ}
\index{せういにえ@セヴィニェ!こんそめ@コンソメ・---}

\hypertarget{consomme-solange}{%
\subsubsection{コンソメ・ソランジュ}\label{consomme-solange}}

\frsub{Consommé Solange}

\index{consomme@consommé!solange@--- Solange}
\index{solange@Solange!consomme@consommé ---}
\index{こんそめ@コンソメ!そらんしゆ@---・ソランジュ}
\index{そらんしゆ@ソランジュ!こんそめ@コンソメ・---}

\hypertarget{consomme-stael}{%
\subsubsection{コンソメ・スタール}\label{consomme-stael}}

\frsub{Consommé Staël}

\index{consomme@consommé!stael@--- Staël}
\index{stael@Staël!consomme@consommé ---}
\index{こんそめ@コンソメ!すたーる@---・スタール}
\index{すたーる@スタール!こんそめ@コンソメ・---}

\hypertarget{consomme-stanley}{%
\subsubsection{コンソメ・スタンレイ}\label{consomme-stanley}}

\frsub{Consommé Stanley}

\index{consomme@consommé!stanley@--- Stanley}
\index{stanley@Stanley!consomme@consommé ---}
\index{こんそめ@コンソメ!すたんれい@---・スタンレイ}
\index{すたんれい@スタンレイ!こんそめ@コンソメ・---}

\hypertarget{consomme-suzette}{%
\subsubsection{コンソメ・シュゼット}\label{consomme-suzette}}

\frsub{Consommé Suzette}

\index{consomme@consommé!suzette@--- Suzette}
\index{suzette@Suzette!consomme@consommé ---}
\index{こんそめ@コンソメ!しゆせつと@---・シュゼット}
\index{しゆせつと@シュゼット!こんそめ@コンソメ・---}

\hypertarget{consomme-talleyrand}{%
\subsubsection{コンソメ・タレーラン}\label{consomme-talleyrand}}

\frsub{Consommé Talleyrand}

\index{consomme@consommé!talleyrand@--- Talleyrand}
\index{talleyrand@Talleyrand!consomme@consommé ---}
\index{こんそめ@コンソメ!たれーらん@---・タレーラン}
\index{たれーらん@タレーラン!こんそめ@コンソメ・---}

\hypertarget{consomme-au-tapioca}{%
\subsubsection{コンソメ・タピオカ入り}\label{consomme-au-tapioca}}

\frsub{Consommé au Tapioca}

\index{consomme@consommé!au tapioca@--- au Tapioca}
\index{tapioca@Tapioca!consomme@consommé au ---}
\index{こんそめ@コンソメ!たひおかいり@---・タピオカ入り}
\index{たひおかいり@タピオカ入り!こんそめ@コンソメ・---}

\hypertarget{consomme-theodora}{%
\subsubsection{コンソメ・テオドラ}\label{consomme-theodora}}

\frsub{Consommé Théodora}

\index{consomme@consommé!theodora@--- Théodora}
\index{theodora@Théodora!consomme@consommé ---}
\index{こんそめ@コンソメ!ておとら@---・テオドラ}
\index{ておとら@テオドラ!こんそめ@コンソメ・---}

\hypertarget{consomme-toreador}{%
\subsubsection{コンソメ・闘牛士風}\label{consomme-toreador}}

\frsub{Consommé Toréador}

\index{consomme@consommé!toreador@--- Toréador}
\index{toreador@Toréador!consomme@consommé ---}
\index{こんそめ@コンソメ!とうきゆうしふう@---・闘牛士風}
\index{とうきゆうしふう@闘牛士風!こんそめ@コンソメ・---}

\hypertarget{consomme-tosca}{%
\subsubsection{コンソメ・トスカ}\label{consomme-tosca}}

\frsub{Consommé Tosca}

\index{consomme@consommé!tosca@--- Tosca}
\index{tosca@Tosca!consomme@consommé ---}
\index{こんそめ@コンソメ!とすか@---・トスカ}
\index{とすか@トスカ!こんそめ@コンソメ・---}

\hypertarget{consomme-toulousaine}{%
\subsubsection{コンソメ・トゥールーズ風}\label{consomme-toulousaine}}

\frsub{Consommé Toulousaine}

\index{consomme@consommé!toulousaine@--- Toulousaine}
\index{toulousaine@Toulousaine!consomme@consommé ---}
\index{こんそめ@コンソメ!とうーるーすふう@---・トゥールーズ風}
\index{とうーるーすふう@トゥールーズ風!こんそめ@コンソメ・---}

\hypertarget{consomme-a-la-trevise}{%
\subsubsection{コンソメ・トレヴィーゾ風}\label{consomme-a-la-trevise}}

\frsub{Consommé à la Trévise}

\index{consomme@consommé!a la trevise@--- à la Trévise}
\index{trevise@Trévise!consomme@consommé à la ---}
\index{こんそめ@コンソメ!とれういーそふう@---・トレヴィーゾ風}
\index{とれうーそふう@トレヴィーゾ風!こんそめ@コンソメ・---}

\hypertarget{consomme-tyrolienne}{%
\subsubsection{コンソメ・チロリエンヌ}\label{consomme-tyrolienne}}

\frsub{Consommé Tyrolienne}

\index{consomme@consommé!tyrolienne@--- Tyrolienne}
\index{tyrolienne@Tyrolienne!consomme@consommé ---}
\index{こんそめ@コンソメ!ちろりえんぬ@---・チロリエンヌ}
\index{ちろりえんぬ@チロリエンヌ!こんそめ@コンソメ・---}

\hypertarget{consomme-d-uzes}{%
\subsubsection{コンソメ・ユゼス}\label{consomme-d-uzes}}

\frsub{Consommé d'Uzès}

\index{consomme@consommé!d'uzes@--- d'Uzès}
\index{uzes@Uzès!consomme@consommé d'---}
\index{こんそめ@コンソメ!ゆせす@---・ユゼス}
\index{ゆせす@ユゼス!こんそめ@コンソメ・---}

\hypertarget{consomme-valromey}{%
\subsubsection{コンソメ・ヴァルロメ}\label{consomme-valromey}}

\frsub{Consommé Valromey}

\index{consomme@consommé!valromey@--- Valromey}
\index{valromey@Valromey!consomme@consommé ---}
\index{こんそめ@コンソメ!うあるろめ@---・ヴァルロメ}
\index{うあるろめ@ヴァルロメ!こんそめ@コンソメ・---}

\hypertarget{consomme-vendome}{%
\subsubsection{コンソメ・ヴァンドーム}\label{consomme-vendome}}

\frsub{Consommé Vendôme}

\index{consomme@consommé!vendome@--- Vendôme}
\index{vendome@Vendôme!consomme@consommé ---}
\index{こんそめ@コンソメ!うあんとーむ@---・ヴァンドーム}
\index{うあんとーむ@ヴァンドーム!こんそめ@コンソメ・---}

\hypertarget{consomme-verdi}{%
\subsubsection{コンソメ・ヴェルディ}\label{consomme-verdi}}

\frsub{Consommé Verdi}

\index{consomme@consommé!verdi@--- Verdi}
\index{verdi@Verdi!consomme@consommé ---}
\index{こんそめ@コンソメ!うえるてい@---・ヴェルディ}
\index{うえるてい@ヴェルディ!こんそめ@コンソメ・---}

\hypertarget{consomme-vermandoise}{%
\subsubsection{コンソメ・ヴェルマンドワズ}\label{consomme-vermandoise}}

\frsub{Consommé Vermandoise}

\index{consomme@consommé!vermandoise@--- Vermandoise}
\index{vermandoise@Vermandoise!consomme@consommé ---}
\index{こんそめ@コンソメ!うえるまんとわす@---・ヴェルマンドワズ}
\index{うえるまんとわす@ヴェルマンドワズ!こんそめ@コンソメ・---}

\hypertarget{consomme-au-vermicelle}{%
\subsubsection{コンソメ・ヴァミセリ入り}\label{consomme-au-vermicelle}}

\frsub{Consommé au Vermicelle}

\index{consomme@consommé!au vermicelle@--- au Vermicelle}
\index{vermicelle@Vermicelle!consomme@consommé au ---}
\index{こんそめ@コンソメ!うあみせりいり@---・ヴァミセリ入り}
\index{うあみせりいり@ヴァミセリ入り!こんそめ@コンソメ・---}

\hypertarget{consomme-des-viveurs}{%
\subsubsection{コンソメ・ヴィブール}\label{consomme-des-viveurs}}

\frsub{Consommé des Viveurs}

\index{consomme@consommé!des viveurs@--- des Viveurs}
\index{viveurs@Viveurs!consomme@consommé des---}
\index{こんそめ@コンソメ!ういふーる@---・ヴィブール}
\index{ういふーる@ヴィブール!こんそめ@コンソメ・---}

\hypertarget{consomme-warwick}{%
\subsubsection{コンソメ・ヴァルヴィック}\label{consomme-warwick}}

\frsub{Consommé Warwick}

\index{consomme@consommé!warwick@--- Warwick}
\index{warwick@Warwick!consomme@consommé ---}
\index{こんそめ@コンソメ!うあるういつく@---・ヴァルヴィック}
\index{うあるういつく@ヴァルヴィック!こんそめ@コンソメ・---}

\hypertarget{consomme-washington}{%
\subsubsection{コンソメ・ワシントン}\label{consomme-washington}}

\frsub{Consommé Washington}

\index{consomme@consommé!washington@--- Washington}
\index{washington@Washington!consomme@consommé ---}
\index{こんそめ@コンソメ!わしんとん@---・ワシントン}
\index{わしんとん@ワシントン!こんそめ@コンソメ・---}

\hypertarget{consomme-wladimir}{%
\subsubsection{コンソメ・ワディミール}\label{consomme-wladimir}}

\frsub{Consommé Wladimir}

\index{consomme@consommé!wladimir@--- Wladimir}
\index{wladimir@Wladimir!consomme@consommé ---}
\index{こんそめ@コンソメ!わていみーる@---・ワディミール}
\index{わていみーる@ワディミール!こんそめ@コンソメ・---}

\hypertarget{consomme-yvetot}{%
\subsubsection{コンソメ・イヴト}\label{consomme-yvetot}}

\frsub{Consommé Yvetot}

\index{consomme@consommé!yvetot@--- Yvetot}
\index{yvetot@vetot!consomme@consommé ---}
\index{こんそめ@コンソメ!いうと@---・イヴト}
\index{いうと@イヴト!こんそめ@コンソメ・---}

\hypertarget{consomme-zola}{%
\subsubsection{コンソメ・ゾラ}\label{consomme-zola}}

\frsub{Consommé Zola}

\index{consomme@consommé!zola@--- Zola}
\index{zola@Zola!consomme@consommé ---}
\index{こんそめ@コンソメ!そら@---・ゾラ}
\index{そら@ゾラ!こんそめ@コンソメ・---}

\hypertarget{consomme-zorilla}{%
\subsubsection{コンソメ・ゾリヤ}\label{consomme-zorilla}}

\frsub{Consommé Zorilla}

\index{consomme@consommé!zorilla@--- Zorilla}
\index{zorilla@Zorilla!consomme@consommé ---}
\index{こんそめ@コンソメ!そりや@---・ゾリヤ}
\index{そりや@ゾリヤ!こんそめ@コンソメ・---}
\end{recette}%newpage
%\hypertarget{consomme-divers-speciaux-pour-soupers}{%
\section{夜食用のコンソメなど}\label{consomme-divers-speciaux-pour-soupers}}

\frsec{Consommés divers, spéciaux pour Soupers}
\begin{recette}
\hypertarget{consome-a-l-essence-de-caille}{%
\subsubsection{コンソメ・鶉のエッセンス入り}\label{consome-a-l-essence-de-caille}}

\frsub{Consommé à l'essence de Caille}

\index{consomme@consommé!caille@--- à l'essence de Caille}
\index{caille@caille!consomme@consommé à l'essence de ---}
\index{こんそめ@コンソメ!うすらのえつせんすいり@鶉のエッセンス入り}
\index{うすら@鶉!こんそめえつせんすいり@コンソメ・---のエッセンス入り}

コンソメ1 Lあたり鶉4羽を用いる。鶉をローストし、胸肉は別に取り置いて別
の用途に使うこと。残りの部位を使ってコンソメをクラリフィエする。

\hypertarget{consome-a-l-essence-de-celeri}{%
\subsubsection{コンソメ・セロリのエッセンス入り}\label{consome-a-l-essence-de-celeri}}

\frsub{Consommé à l'essence de Céleri}

\index{consomme@consommé!celeri@--- à l'essence de Céleri}
\index{celeri@céleri!consomme@consommé à l'essence de ---}
\index{こんそめ@コンソメ!せろりのえつせんす@---・セロリのエッセンス入り}
\index{せろり@せろり!こんそめえつせんすいり@コンソメ・---のエッセンス入り}

コンソメに風味を付けるために必要なセロリの量は、概ね 1 Lあたり 100
gとして、クラリフィエの際に加える。

\hypertarget{consome-a-l-essence-d-estragon}{%
\subsubsection{コンソメ・エストラゴンのエッセンス入り}\label{consome-a-l-essence-d-estragon}}

\frsub{Consommé à l'essence d'Estragon}

\index{consomme@consommé!estragon@--- à l'essence d'Estragon}
\index{estragon@Estragon!consomme@consommé à l'essence d' ---}
\index{こんそめ@コンソメ!えすとらこんのえつせんす@エストラゴンのエッセンス入り}
\index{えすとらこん@エストラゴン!こんそめえつせんすいり@コンソメ・---のエッセンス入り}

この香草は使い過ぎないようにするのががいい。風味付けには1
Lあたり数枚の葉で充分。

\hypertarget{consome-ivan-pour-soupers}{%
\subsubsection{コンソメ・イヴァン}\label{consome-ivan-pour-soupers}}

\frsub{Consommé Ivan}

\index{consomme@consommé!ivan@--- Ivan (pour soupers)}
\index{ivan@Ivan!consomme@consommé --- (pour soupers)}
\index{こんそめ@コンソメ!いうあんやしよくよう@イヴァン(夜食用)}
\index{いうあん@イヴァン!こんそめやしよくよう@コンソメ・---(夜食用)}

\protect\hyperlink{consomme-ivan}{温かいコンソメの節のアルファベット順に出ているレシピ}参照。

\hypertarget{consome-a-l-essence-de-morille}{%
\subsubsection{コンソメ・モリーユのエッセンス入り}\label{consome-a-l-essence-de-morille}}

\frsub{Consommé à l'essence de Morille}

\index{consomme@consommé!morille@--- à l'essence de Morille}
\index{morille@morille!consomme@consommé à l'essence de ---}
\index{こんそめ@コンソメ!もりーゆのえつせんすいり@---・モリーユのエッセンス入り}
\index{もりーゆ@モリーユ!こんそめえつせんすいり@コンソメ・---のエッセンス入り}

コンソメ1 Lあたり、生なら150 g、乾燥なら90 gの網笠茸を用意する。ごく薄
くスライスし、さらに細かくすり潰して加える。
\end{recette}%newpage
%\href{原稿下準備20180414五島、連載からコピー}{} \href{訳と注釈}{}
\href{未、原文対照チェック}{} \href{未、日本語表現校正}{}
\href{未、その他修正}{} \href{未、原稿最終校正}{}

\hypertarget{ux3068ux308dux307fux3092ux4ed8ux3051ux305fux30ddux30bfux30fcux30b8ux30e5}{%
\section{とろみを付けたポタージュ}\label{ux3068ux308dux307fux3092ux4ed8ux3051ux305fux30ddux30bfux30fcux30b8ux30e5}}

\frsec{Potages Liés}

\hypertarget{ux30ddux30bfux30fcux30b8ux30e5ux30d4ux30e5ux30ec}{%
\subsection{ポタージュ・ピュレ}\label{ux30ddux30bfux30fcux30b8ux30e5ux30d4ux30e5ux30ec}}

\frsecb{les Purées}

主素材とつなぎ:ポタージュ・ピュレの主素材として用いるのは次のとおり。
1種類または数種を組み合わせた野菜、鶏、ジビエ、甲殻類。

ほぼ全てのポタージュ・ピュレにはつなぎを加える。すなわち、

米\ldots{}\ldots{}鶏、甲殻類のポタージュ・ピュレおよび野菜のポタージュ・ピュレ
のいくつか。

じゃがいも\ldots{}\ldots{}香草や、かぼちゃのように水分の多い野菜のポタージュ・
ピュレ。

レンズ豆\ldots{}\ldots{}ジビエのポタージュ・ピュレ。

バターで揚げたクルトン\ldots{}\ldots{}クラシックなポタージュ・ピュレ。

昔の料理では、他にもつなぎに用いるものはあったが、とりわけクーリとビス
ク\footnote{「昔の料理」におけるビスクは甲殻類のポタージュ・ピュレのことで
  はなく、鳩などの煮込み料理のこと(本連載「ポタージュ(1)」2012年6月
  号p.115 参照)。}には、クルトンが主に用いられていた\footnote{中世〜18世紀には、とろみをつけるために、硬くなったパンを加えて
  弱火で煮込む(mitonnerミトネ)ことが一般的だった。}。とてもまろやかな仕上り
になるので、現代でもこの手法を用いる価値はある。

いんげん豆やレンズ豆、じゃがいものようなでんぷん質の素材のポタージュ・
ピュレにはつなぎを加える必要はない。主素材である野菜それ自体がつなぎと
なるからだ。

加える液体とつなぎの分量:ポタージュ・ピュレに加える液体は、主素材の種
類に応じて、白いコンソメ、ジビエのコンソメ、魚のコンソメを用いる。野菜
のポタージュ・ピュレでは牛乳を用いる場合もある。

加える液体\ldots{}\ldots{}ベースとなるピュレ1 Lに対して2 L。

つなぎ\ldots{}\ldots{}

\begin{enumerate}
\def\labelenumi{\arabic{enumi}.}
\item
  米\ldots{}\ldots{}野菜500 gあたり85〜120 g。鶏、ジビエ、甲殻類の身500
  gあ たり75〜100 g。
\item
  レンズ豆\ldots{}\ldots{}ジビエの肉500 gあたり190 g。
\item
  じゃがいも\ldots{}\ldots{}香草と野菜500 gあたり250 g。
\item
  バターで揚げたクルトン\ldots{}\ldots{}野菜または甲殻類の身500
  gあたり270 g。
\end{enumerate}

作業と仕上げ:野菜は次のいずれかの方法で処理する。(a)薄切りにした野菜
600〜700 gあたり80〜100 gのバターでエテュヴェする。(b)薄切りにした野菜を
湯通し\footnote{原文 blanchir
  (ブランシール)。下茹でする、湯がくこと。野菜類の
  ブランシールは塩を加えた湯で行なうが、素材の性質により2種に分けら
  れる。ひとつは大量の湯で素材に完全に火が通るまで茹でること。もうひ
  とつは刳味(アク)を除くための下茹で、湯通し(原書p.726)。ここでは後 者。}してからバターでエテュヴェする。どちらの方法を用いるかは、
本書では個々のルセットに記してある。

ジビエはサルミを調理する際と同様にロティール\footnote{鶏、猟鳥の胸肉の部分を豚背脂のシートで包んでセニャンにロティー
  ルする。なお、「サルミ」は古くは「猟鳥肉の煮込み」の意であった。
  『ル・ギード・キュリネール』では、ロティールした猟鳥の肉を切り分け
  て保温し、摺り潰したガラと端肉を煮込んで作ったソースと合わせる(本
  連載「雉のサルミ」2011年11月号pp.128-129 参照)。}してから、レンズ豆と
ともに煮る。火が通ったら骨を外す。肉とレンズ豆を摺り潰し、布漉しした後、
濃さを調節する。

鶏は白いコンソメでポシェする。つなぎに用いる米も一緒に煮る。火が通った
ら骨を外し、その後はジビエのピュレと同様にする。鶏およびジビエにちょう
ど火が通ったところで、浮き実にする分の胸肉は別にとっておくこと。

野菜のポタージュ・ピュレは、濃さを調節したらデプイエ、つまり微沸騰の状
態で25〜30分間かけて不純物を取り除く。

このデプイエの作業の際、時折、冷たいコンソメを若干量加えるとよい。ピュ
レの中に紛れている不純物が表面に浮かび上がって、取り除きやすくなる。

鶏、ジビエ、甲殻類のピュレは沸騰したら湯煎にかける。デプイエする必要は
ない。

どのポタージュ・ピュレも、仕上げにバターを加える直前に、目の細かいシノ
ワで漉すこと。

仕上げは提供直前に行なう。火から外し、ポタージュ1ℓあたり80〜100
gのバター を加える。

つなぎに白いんげん豆、じゃがいも、米などのような白いでんぷん質やクルト
ンを用いるポタージュは、さらにつなぎとして卵黄を補ってもよい。

バターを加えたら、再沸騰させてはいけないと肝に銘じること。沸騰するとバ
ターの風味が失なわれてしまう。ポタージュにおいて、バターの風味は明瞭で
フレッシュでなくてはいけない。

(略)

ピュレの展開\footnote{本連載「ポタージュ(1)」2012年6月号 p.115 参照。}:以下に記す方法で、ピュレの多くはポタージュ・ヴルテ、
ポタージュ・クレームにすることが出来る。ポタージュ・ピュレに用いるつな
ぎの代わりに、鶏または魚のヴルテや薄いソース・ベシャメルを主素材に加え
るのだ。

ただし、素材によっては、ポタージュ・ピュレ以外の仕立てに出来ないものも
ある。

\hypertarget{ux30ddux30bfux30fcux30b8ux30e5ux30f4ux30ebux30c6}{%
\subsection{ポタージュ・ヴルテ}\label{ux30ddux30bfux30fcux30b8ux30e5ux30f4ux30ebux30c6}}

\frsecb{les Veloutés}

ベースとなるヴルテ\footnote{基本ソースとしてのヴルテ(原書
  p.15)がベースとなる。}:

\begin{enumerate}
\def\labelenumi{\arabic{enumi}.}
\item
  野菜のポタージュ・ヴルテの場合は、やや薄い通常のヴルテ。
\item
  鶏や魚のポタージュ・ヴルテの場合は、それぞれ対応するヴルテ。
\end{enumerate}

ポタージュのベースにするヴルテは、主素材となる野菜、鶏、ジビエおよび魚
に応じて、通常の白いコンソメ、鶏のコンソメ、ジビエのコンソメ、魚のコン
ソメ1ℓあたり白いルゥ100 gを加えて作る。

材料比率:この方法で作るポタージュは全て、次の分量配分となる。

・ベースとなるヴルテはポタージュ全体の半量。

・ポタージュの性格を決めるピュレは全体の\textbf{1/4}。

・濃さを整えるのに加えるコンソメも\textbf{1/4}。ただし、つなぎとして加える
生クリームの分量もこれに含める。

例えば、仕上がり2ℓの「ポタージュ・ヴルテ王妃風\footnote{à la reine
  (ア・ラ・レーヌ)優美で繊細な料理に用いる表現。この名
  称の料理には鶏を素材としたものが多い。「ポタージュ・ピュレ王妃風」
  のルセットは原書p.146。}」の場合には分量は 次のようになる。

\begin{quote}
鶏のヴルテ\textbf{1}ℓ。鶏のピュレ\textbf{5dl}。仕上げに加える白いコンソメ
\textbf{3dl}。つなぎ\textbf{(}生クリーム\textbf{)2dl}。計\textbf{2}ℓ。
\end{quote}

作業:

\begin{enumerate}
\def\labelenumi{(\arabic{enumi})}
\item
  主素材が鶏や魚の場合は、予め骨を外してからベースとなるヴルテで素材
  を煮る。次に、肉を取り出して摺り潰し、肉を煮たヴルテでのばしてから布漉
  しする。このピュレにコンソメを加えて濃さを整える。
\item
  野菜の場合は、素材の性質に応じて、湯通ししたものをバターでエテュヴェ
  するか、生の野菜をバターでエテュヴェしてから、ベースとなるヴルテに加え
  る。野菜に火が通った後は上記と同様にする。
\item
  甲殻類の場合は、通常どおりミルポワを用いて火を通し、細かく摺り潰し
  てからベースとなるヴルテに加えて煮、布漉しする。
\end{enumerate}

つなぎと仕上げ:ポタージュ・ヴルテのつなぎには、仕上り1ℓあたり卵黄3ヶ
と生クリーム1dlを加える。

提供直前に、鍋を火から外して、1ℓあたりバター80〜100 gを加えて仕上げる。
(略)

\hypertarget{ux30ddux30bfux30fcux30b8ux30e5ux30afux30ecux30fcux30e0}{%
\subsection{ポタージュ・クレーム}\label{ux30ddux30bfux30fcux30b8ux30e5ux30afux30ecux30fcux30e0}}

\frsecb{les Crèmes}

ポタージュ・クレームの作り方はポタージュ・ヴルテと同じだが、以下の点が
違う。

\begin{enumerate}
\def\labelenumi{(\arabic{enumi})}
\item
  ヴルテではなく薄いソース・ベシャメルをベースとして用いる。牛乳1ℓあ
  たり白いルゥ100 gで作る。
\item
  多くの場合、仕上げに濃さを調節する際、コンソメではなく牛乳を加える。
\end{enumerate}

材料比率:ポタージュ・ヴルテと同様。つまり、ベシャメルはポタージュ全体
の半量、ポタージュの性格を決めるピュレが1/4、濃さを整えるための白いコ
ンソメまたは牛乳が1/4(仕上げに加える生クリームもこれに含める)。

作業:主素材が鶏、ジビエ、野菜、甲殻類いずれの場合も、作業はポタージュ・
ヴルテの項で示したのと同じ。(略)

仕上げ:提供直前に、ポタージュ1ℓあたり2dlの生クリームを加える。

原注:ポタージュ・ヴルテもポタージュ・クレームもデプイエは行なわない。
ポタージュの濃さを整えたら、沸騰寸前まで温め、湯煎にかけて保温しておく。
表面が乾かないようバターのかけら数片を載せる。ポタージュ・ヴルテは、供
する前に卵黄、生クリーム、バターを加えて仕上げる。ポタージュ・クレーム
の仕上げは供する前に生クリームだけを加える。
%newpage
%\input{../03-potages/03-07-pp139-146}
%\input{../03-potages/03-08-pp146-150}
%\input{../03-potages/03-09-pp151-156}
%\input{../03-potages/03-10-pp157-162}
%\input{../03-potages/03-11-pp163-169}
%\input{../03-potages/03-12-pp170-171}
%\input{../03-potages/03-13-pp172-185}
 


%%% chapitre iv. hors-d'oeuvres
%%%% エスコフィエ『料理の手引き』全注解
% 五島 学

\href{✓原稿下準備なし}{} \href{訳と注釈\%2020180420進行中}{}
\href{未、原文対照チェック}{} \href{未、日本語表現校正}{}
\href{未、注釈チェク}{} \href{未、原稿最終校正}{}

\hypertarget{poissons}{%
\chapter{VI 魚料理 Poissons}\label{poissons}}

\hypertarget{serie-de-courts-bouillons-de-poisson}{%
\section{クールブイヨン}\label{serie-de-courts-bouillons-de-poisson}}

\begin{recette}
\hypertarget{ux30afux30fcux30ebux30d6ux30a4ux30e8ux30f3-a}{%
\subsubsection{クールブイヨン
A}\label{ux30afux30fcux30ebux30d6ux30a4ux30e8ux30f3-a}}

水5 L に対し、ヴィネガー2.5 dL、粗塩60 g、薄切りにしたにんじん600
gと玉ねぎ500 g、タイム1枝、ローリエの小さな葉2枚、パセリの茎100
g、粒こしょう20
g(こしょうを加えるのはクールブイヨンを漉す10分前)。材料を全て鍋に入れ、火にかけて1時間弱火で煮、漉す(原書
p.277)。

\hypertarget{ux30afux30fcux30ebux30d6ux30a4ux30e8ux30f3-b-4-ux9c52ux3046ux306aux304eux30d6ux30edux30b7ux30a7ux7b49-ux539fux66f8p.277}{%
\subsubsection[クールブイヨン B (鱒、うなぎ、ブロシェ等)
原書p.277]{\texorpdfstring{クールブイヨン B \footnote{クールブイヨンは用途に応じ、AからEまでの5種が挙げられている(原書pp.277-278)。}
(鱒、うなぎ、ブロシェ等)
原書p.277}{クールブイヨン B  (鱒、うなぎ、ブロシェ等) 原書p.277}}\label{ux30afux30fcux30ebux30d6ux30a4ux30e8ux30f3-b-4-ux9c52ux3046ux306aux304eux30d6ux30edux30b7ux30a7ux7b49-ux539fux66f8p.277}}

5 L 分の材料\ldots{}\ldots{}白ワイン2.5 L 。水2.5 L
。薄切りにした玉ねぎ600 g。パセリの茎80g
。タイムの小枝1本。ローリエの葉(小) \(\frac{1}{2}\) 枚。粗塩
60g。大粒のこしょう15 g(クールブイヨンを漉す10分前に加える)。

作業手順\ldots{}\ldots{}作業:液体、香味素材、調味料を鍋に入れ、沸かす。弱火で30分程煮て、漉す。

\href{欠落アリ}{}

原注:クールブイヨンBとC\footnote{クールブイヨンBの白ワインを赤ワインに代え、香味素材としてにんじん400gを加える。鱒、鯉、マトロート用(原書pp.277-278)。}で調理した魚はクールブイヨン添えとして供する。つまり、少量の煮汁とクールブイヨンに用いた野菜を添える。野菜はよく火が通っていること。煮汁はしっかり煮詰め、提供直前に新鮮なバター少量を加えて仕上げる。
\end{recette}
\hypertarget{ux30afux30fcux30ebux30d6ux30a4ux30e8ux30f3ux306eux4f7fux3044ux65b9-ux539fux66f8-p.278}{%
\subsection{クールブイヨンの使い方 原書
p.278}\label{ux30afux30fcux30ebux30d6ux30a4ux30e8ux30f3ux306eux4f7fux3044ux65b9-ux539fux66f8-p.278}}

\begin{enumerate}
\def\labelenumi{\arabic{enumi}.}
\item
  加熱時間が30分以内の場合は、クールブイヨンは必ず事前に用意しておくこと。
\item
  加熱時間が30分を越える場合は、クールブイヨンの材料は冷たい状態のままで合わせせておく。香味素材はポワソニエールの網の下に入れる。
\item
  ごく少量のクールブイヨンでポシェ\footnote{原文 pochage à court
    mouillement『ル・ギード・キュリネール』では、この表現はテュルボタン(小型のテュルボ)、バルビュ、舌びらめ等の平たい魚をポシェする際に用いられる。本連載「舌びらめのボヌ・ファム」
    2011年3月号pp.110-111 参照。}する場合、材料は(白または赤ワインを含む場合も)魚を火にかける際に合わせる。クールブイヨンの量は魚の
  \(\frac{1}{3}\)
  の高さとし、加熱中ひんぱんに煮汁を魚にかけてやること\footnote{arroser
    アロゼ。}。この調理法の場合は通常、クールブイヨンは上で記したように
  \footnote{「クールブイヨンB」原注。}、提供直前に軽くバターを加えて仕上げ、魚に添える。
\item
  冷製にする場合は、必ずクールブイヨンに魚が浸った状態で冷ますこと。当然ながら、火にかけている時間は短かくなる\footnote{余熱で火が通るため。}。
\end{enumerate}

\hypertarget{ux539fux6ce8}{%
\subparagraph{【原注】}\label{ux539fux6ce8}}

いくつかの魚種の加熱時間は該当する項で示してある。

\hypertarget{ux9b5aux306eux8abfux7406ux6cd5}{%
\section{魚の調理法}\label{ux9b5aux306eux8abfux7406ux6cd5}}

魚料理は全て、下記のいずれかの調理法による。

\begin{enumerate}
\def\labelenumi{\arabic{enumi}.}
\item
  塩水(湯)またはクールブイヨン\footnote{court-bouillon直訳は「量の少ない煮汁」。魚の他、甲殻類、鶏などの白身肉、野菜などをポシェするのに用いる。とりわけ魚や鶏を丸ごとポシェする場合には、その名称のとおり、できるだけ少量でポシェする必要がある。また、ポシェに用いたクールブイヨンをベースにソースを作る場合が多い。}Bを用いたポシェ\ldots{}\ldots{}大きな魚丸ごと、および切り身。
\item
  ごく少量のクールブイヨンを用いたポシェ\ldots{}\ldots{}魚のフィレ、またはやや小さい魚。
\item
  ブレゼ\ldots{}\ldots{}もっぱら大きな魚。
\item
  オ・ブルー\footnote{比較的小さめの淡水魚に主として用いられる調理法。生きたままの魚の表面のぬめりをとらないように洗い、内臓を取り除いたらすぐに、塩とヴィネガーを加えたクールブイヨンで茹でる。冷製、温製どちらでも供する。原書p.281参照。}\ldots{}\ldots{}とりわけ\ruby{鱒}{ます}、鯉、ブロシェ\footnote{川かますの一種。本連載「ブロシェのクネル」2011年10月号
    pp.124-125 参照。}に合う。
\item
  揚げもの\ldots{}\ldots{}もっぱら小さい魚、切り身。
\item
  ムニエール\ldots{}\ldots{}揚げものにするのと同じ小さい魚、切り身。
\item
  グリエ\ldots{}\ldots{}小さい魚、および切り身。
\item
  グラタン\ldots{}\ldots{}小さい魚、切り身。
\end{enumerate}

\hypertarget{ux5869ux6c34ux6e6fux304aux3088ux3073ux30afux30fcux30ebux30d6ux30a4ux30e8ux30f3bux3092ux7528ux3044ux305fux52a0ux71b1ux8abfux7406}{%
\subsection{塩水(湯)およびクールブイヨンBを用いた加熱調理}\label{ux5869ux6c34ux6e6fux304aux3088ux3073ux30afux30fcux30ebux30d6ux30a4ux30e8ux30f3bux3092ux7528ux3044ux305fux52a0ux71b1ux8abfux7406}}

魚を丸ごと調理する場合は、魚に合ったポワソニエール\footnote{大きな魚を丸ごと煮るための細長い鍋。魚の形を崩さずに取り出せるよう、中に専用の網を敷いて使う。似たものに、舌びらめ等の平たい魚にぴったり合う菱形をしたテュルボティエールがある。いずれも、できるだけ少量の煮汁で魚を加熱できるように工夫されたもの。(図参照)}を用いる。魚を掃除し(テュルボは水にさらして血抜きをし)、ひれ等を切り落して形を整え、ポワソニエールの網に乗せる。魚種に応じて塩水または冷たいクールブイヨンをかぶるまで注ぐ。強火にかけて沸騰したらすぐにレンジの火の弱いところに鍋を移動させ、ポシェする。

切り身(薄すぎは絶対にいけない)の場合、沸騰した液体(塩湯またはクールブイヨン)に投入したらすぐにレンジの火の弱いところに鍋を移動させ、沸騰しない程度の温度でゆっくりと火を通す。

こうするのは、魚の身のエキスを閉じこめるためである。冷水から火にかけた場合にはエキスの大部分が流れ出してしまう。大きな魚丸ごとの場合にはこのやり方はしない。沸騰した液体に魚を投入すると身が収縮するので、大きな魚の場合は身が割れたり形が崩れたりするからだ。

塩湯あるいはクールブイヨンでポシェした魚は、ナフキンまたは専用の網に盛る。周囲をパセリで飾り、塩茹でしたじゃがいもと1種類または数種のソースを添えて供する。ガルニテュールがパセリのみの場合、魚の周囲にパセリを飾るのは客に料理を見せる\footnote{当時の宴席で主流だったロシア式サーヴィスでは、大きな銀盆に盛った料理をまず食客に見せてから、とり分けて給仕する。}直前にすること。どんな場合でも、ガルニテュールを添えたらクロッシュ\footnote{銀または陶製の保温用皿カバー。}は被せないこと。

\hypertarget{ux3054ux304fux5c11ux91cfux306eux30afux30fcux30ebux30d6ux30a4ux30e8ux30f3ux3092ux7528ux3044ux305fux30ddux30b7ux30a7-ux539fux66f8-pp.279-280}{%
\subsection{ごく少量のクールブイヨンを用いたポシェ 原書
pp.279-280}\label{ux3054ux304fux5c11ux91cfux306eux30afux30fcux30ebux30d6ux30a4ux30e8ux30f3ux3092ux7528ux3044ux305fux30ddux30b7ux30a7-ux539fux66f8-pp.279-280}}

この火入れの方法は主としてテュルボタン、バルビュ、舌びらめ、丸ごとおよびそれぞれの魚のフィレで用いる。バターを塗った天板あるいはソテ鍋に魚丸ごとあるいはそのフィレを置き、軽く塩をして、所要量の魚のフュメかマッシュルームの煮汁を注ぐ。フュメとマッシュルームの煮汁を合わせたものを用いる場合もある。蓋をして、中温のオーヴンに入れる。魚丸ごとの場合は時折煮汁をかけてやる。

魚(丸ごとでもフィレでも)に火が通ったら、注意して汁気をきり、皿に盛る。ガルニテュールを含む料理の場合、ガルニテュールを魚の周囲に盛り、ソースをかける\footnote{原文通りの順で訳したが、実際にはソースをかけてからガルニテュールを盛ったほうが良い場合もあるだろう。}。多くの場合、魚の煮汁を煮詰めてソースに加える。

\href{欠落アリ}{}

\hypertarget{ux9b5aux306eux30d6ux30ecux30bc17-ux539fux66f8-p.280}{%
\subsection[魚のブレゼ 原書 p.280]{\texorpdfstring{魚のブレゼ\footnote{ここではブレゼの語が限定的な意味で用いられていることに注意。牛のアロワイヨのような大きな塊肉のブレゼと同様の調理法、ということである。それは、香味素材を色づくまで炒めてから用いることや、主素材に豚背脂やトリュフをピケ針で差したり、豚背脂のシートで覆って加熱するという点によく表れている。ただし、これらは必須というわけではないため、事実上は「ごく少量のクールブイヨンを用いたポシェ」と区別がつきにくい。実際、モンタニェ『ラルース・ガストロノミーク』初版では、魚のブレゼについて「本来的な意味でのブレゼというよりは、ごく少量のクールブイヨンを用いたポシェである」と述べられている。逆に言えば、こんにち魚の加熱方法についてしばしば「ブレゼ」と呼ばれているものが、エスコフィエやモンタニェにおいては「少量のクールブイヨンを用いたポシェ」と表現されていたということである。}
原書
p.280}{魚のブレゼ 原書 p.280}}\label{ux9b5aux306eux30d6ux30ecux30bc17-ux539fux66f8-p.280}}

この調理法を用いるのは通常、丸ごとまたは筒切りにした鮭、大ぶりの鱒、テュルボ、テュルボタンのうち大きなもの、である。

場合によっては、魚の片面に、小さく切った豚背脂、トリュフ、コルニション、にんじん等をピケ針で差し込む。

香味素材等\footnote{原文 fonds de braisage フォン・ド・ブレザージュ
  (fonds de
  braiseフォン・ド・ブレーズ、とも)。通常は、厚い輪切りにしたにんじんと玉ねぎをバターか獣脂で色づくまで炒め、ブーケガルニ、下茹でした豚皮を合わせる(原書pp.394-395)。また、これを用いた煮汁のことも指す。}は肉料理のブレゼの場合と同じように用意するが、豚皮は用いない。提供方法に応じて、白または赤ワインと軽い魚のフュメ同量ずつを、魚の厚みの
\(\frac{3}{4}\) またはひたひたの高さまで注ぐ。厳密に肉断ち \footnote{カトリックの生活習慣として、四旬節(復活祭までの46日間)および週1
  回程度、肉類を食べないということが行なわれた(本連載2012年5月号「ソース・エスパニョル(4)」p.110、訳注1参照)。}のための仕立てにする場合を除いて、薄くスライスした豚背脂のシートを魚にかぶせる。加熱中\footnote{鍋を火にかけ、沸騰したら蓋をして中火のオーヴンに入れ、加熱する。}こまめに煮汁を魚にかけてやる\footnote{arroser
  アロゼ。}。また、完全には蓋をせず、加熱中に煮汁が煮詰まるようにしてやる。

ほぼ火が通ったら、鍋の蓋をとり、魚にかけた煮汁の水分をオーヴンの熱で蒸発させて表面につやを出す\footnote{glacer
  グラセ。}。魚を鍋から出して汁気をきり、皿に盛り保温しておく。

煮汁\footnote{原文 fonds de braisage (訳注2参照)。}を漉し、しばらく休ませたら浮き脂を取り除き、必要なら煮詰める。これを加えてソースを仕上げる。

魚のブレゼには通常、各ルセットに示してあるガルニテュールを添える。

\hypertarget{ux30aaux30d6ux30ebux30fc36-ux539fux66f8-pp.280-281}{%
\subsection[オ・ブルー 原書
pp.280-281]{\texorpdfstring{オ・ブルー\footnote{au bleu
  ヴィネガーを加えることで魚の表面のぬめりが青みがかることから。} 原書
pp.280-281}{オ・ブルー 原書 pp.280-281}}\label{ux30aaux30d6ux30ebux30fc36-ux539fux66f8-pp.280-281}}

オ・ブルーは鱒、鯉、ブロシェ\footnote{川かますの一種。}のみに用いられる特殊な調理法で、基本的なポイントは以下のとおり。

\begin{enumerate}
\def\labelenumi{\arabic{enumi}.}
\item
  必ず、生きた魚を使う。
\item
  魚の表面のぬめりをとらないように、なるべく手で触れずに、わたを抜く。鱗も引かない。
\item
  魚が大きい場合は、専用の網を敷いたポワソニーエルに入れ、「沸騰したヴィネガーをかける」。ヴィネガーは、通常のクールブイヨンに加える分量\footnote{以下の「クールブイヨンA」の分量比率を参照。}。次に、ヴィネガーを入れずに用意した温かい\footnote{原文
    tiède ぬるい、温かい。}クールブイヨンを注ぎ入れる。これは、なるべく身が割れないようにするためである。その後は通常どおり加熱する\footnote{レンジで沸騰させたらオーヴンに入れる。}。
\item
  小さい鱒の場合は、生きたままのものを手早く中抜きし、塩、ヴィネガーを加えただけの沸騰したクールブイヨンで煮る。
\item
  オ・ブルーは冷製、温製どちらの仕立てにしてもいい。実際の作り方の項で示してあるソースを添えて供する。
\end{enumerate}

\href{欠落アリ}{}

\hypertarget{ux30e0ux30cbux30a8ux30fcux30eb43-ux539fux66f8-p.282}{%
\subsection[ムニエール 原書 p.282]{\texorpdfstring{ムニエール\footnote{à
  la meunière 「粉挽き職人風」の意。} 原書
p.282}{ムニエール 原書 p.282}}\label{ux30e0ux30cbux30a8ux30fcux30eb43-ux539fux66f8-p.282}}

ムニエールは素晴らしい調理法だが、小型の魚と、大きな魚の場合は切り身にしか用いない。とはいえ、丁寧にやれば1.5
kg以下のテュルボタンはムニエールで調理できる。

魚丸ごと、あるいは切り身、フィレに味つけをして小麦粉をまぶし、バターを熱したフライパンで焼く。

魚が小さい場合は普通のバターでいいが、大きい場合は澄ましバターを使った方がいい。

魚の両面を焼き、程良く火が通ったら、予め熱しておいた皿に盛る。

飾り切りにした半割りのレモンを添えて、そのまま供することも可能である。ただし、このような提供方法の場合は本来の「ムニエール」と区別するために「黄金色に焼いた\footnote{doré
  (ドレ)
  一般的な色の表現として「黄金色」の意だが、ムニエールの場合、通常は大きい魚についてのみこの表現を用いる。}」と表現する。

「ムニエール」の場合には、焼き上がった魚に少量のレモン汁をふり、塩、こしょう少々で味を整える。粗みじん切りにして湯通ししたパセリを魚の表面に散らし、焦がしバターをかけてすぐに供する。湯通ししたパセリの水分に熱いバターが触れて泡がたつので、それが消えないうちに客に料理を見せるようにする。

%\input{06-poissons/06-02-p-284}
%\input{06-poissons/06-03-pp285-286}
%\input{06-poissons/06-04-p286}
%\input{06-poissons/06-05-pp286-290}

%\href{✓原稿下準備なし}{} \href{訳と注釈\%2020180420進行中}{}
\href{未、原文対照チェック}{} \href{未、日本語表現校正}{}
\href{未、注釈チェク}{} \href{未、原稿最終校正}{}

\hypertarget{poissons}{%
\chapter{VI 魚料理 Poissons}\label{poissons}}

\hypertarget{serie-de-courts-bouillons-de-poisson}{%
\section{クールブイヨン}\label{serie-de-courts-bouillons-de-poisson}}

\begin{recette}
\hypertarget{ux30afux30fcux30ebux30d6ux30a4ux30e8ux30f3-a}{%
\subsubsection{クールブイヨン
A}\label{ux30afux30fcux30ebux30d6ux30a4ux30e8ux30f3-a}}

水5 L に対し、ヴィネガー2.5 dL、粗塩60 g、薄切りにしたにんじん600
gと玉ねぎ500 g、タイム1枝、ローリエの小さな葉2枚、パセリの茎100
g、粒こしょう20
g(こしょうを加えるのはクールブイヨンを漉す10分前)。材料を全て鍋に入れ、火にかけて1時間弱火で煮、漉す(原書
p.277)。

\hypertarget{ux30afux30fcux30ebux30d6ux30a4ux30e8ux30f3-b-4-ux9c52ux3046ux306aux304eux30d6ux30edux30b7ux30a7ux7b49-ux539fux66f8p.277}{%
\subsubsection[クールブイヨン B (鱒、うなぎ、ブロシェ等)
原書p.277]{\texorpdfstring{クールブイヨン B \footnote{クールブイヨンは用途に応じ、AからEまでの5種が挙げられている(原書pp.277-278)。}
(鱒、うなぎ、ブロシェ等)
原書p.277}{クールブイヨン B  (鱒、うなぎ、ブロシェ等) 原書p.277}}\label{ux30afux30fcux30ebux30d6ux30a4ux30e8ux30f3-b-4-ux9c52ux3046ux306aux304eux30d6ux30edux30b7ux30a7ux7b49-ux539fux66f8p.277}}

5 L 分の材料\ldots{}\ldots{}白ワイン2.5 L 。水2.5 L
。薄切りにした玉ねぎ600 g。パセリの茎80g
。タイムの小枝1本。ローリエの葉(小) \(\frac{1}{2}\) 枚。粗塩
60g。大粒のこしょう15 g(クールブイヨンを漉す10分前に加える)。

作業手順\ldots{}\ldots{}作業:液体、香味素材、調味料を鍋に入れ、沸かす。弱火で30分程煮て、漉す。

\href{欠落アリ}{}

原注:クールブイヨンBとC\footnote{クールブイヨンBの白ワインを赤ワインに代え、香味素材としてにんじん400gを加える。鱒、鯉、マトロート用(原書pp.277-278)。}で調理した魚はクールブイヨン添えとして供する。つまり、少量の煮汁とクールブイヨンに用いた野菜を添える。野菜はよく火が通っていること。煮汁はしっかり煮詰め、提供直前に新鮮なバター少量を加えて仕上げる。
\end{recette}
\hypertarget{ux30afux30fcux30ebux30d6ux30a4ux30e8ux30f3ux306eux4f7fux3044ux65b9-ux539fux66f8-p.278}{%
\subsection{クールブイヨンの使い方 原書
p.278}\label{ux30afux30fcux30ebux30d6ux30a4ux30e8ux30f3ux306eux4f7fux3044ux65b9-ux539fux66f8-p.278}}

\begin{enumerate}
\def\labelenumi{\arabic{enumi}.}
\item
  加熱時間が30分以内の場合は、クールブイヨンは必ず事前に用意しておくこと。
\item
  加熱時間が30分を越える場合は、クールブイヨンの材料は冷たい状態のままで合わせせておく。香味素材はポワソニエールの網の下に入れる。
\item
  ごく少量のクールブイヨンでポシェ\footnote{原文 pochage à court
    mouillement『ル・ギード・キュリネール』では、この表現はテュルボタン(小型のテュルボ)、バルビュ、舌びらめ等の平たい魚をポシェする際に用いられる。本連載「舌びらめのボヌ・ファム」
    2011年3月号pp.110-111 参照。}する場合、材料は(白または赤ワインを含む場合も)魚を火にかける際に合わせる。クールブイヨンの量は魚の
  \(\frac{1}{3}\)
  の高さとし、加熱中ひんぱんに煮汁を魚にかけてやること\footnote{arroser
    アロゼ。}。この調理法の場合は通常、クールブイヨンは上で記したように
  \footnote{「クールブイヨンB」原注。}、提供直前に軽くバターを加えて仕上げ、魚に添える。
\item
  冷製にする場合は、必ずクールブイヨンに魚が浸った状態で冷ますこと。当然ながら、火にかけている時間は短かくなる\footnote{余熱で火が通るため。}。
\end{enumerate}

\hypertarget{ux539fux6ce8}{%
\subparagraph{【原注】}\label{ux539fux6ce8}}

いくつかの魚種の加熱時間は該当する項で示してある。

\hypertarget{ux9b5aux306eux8abfux7406ux6cd5}{%
\section{魚の調理法}\label{ux9b5aux306eux8abfux7406ux6cd5}}

魚料理は全て、下記のいずれかの調理法による。

\begin{enumerate}
\def\labelenumi{\arabic{enumi}.}
\item
  塩水(湯)またはクールブイヨン\footnote{court-bouillon直訳は「量の少ない煮汁」。魚の他、甲殻類、鶏などの白身肉、野菜などをポシェするのに用いる。とりわけ魚や鶏を丸ごとポシェする場合には、その名称のとおり、できるだけ少量でポシェする必要がある。また、ポシェに用いたクールブイヨンをベースにソースを作る場合が多い。}Bを用いたポシェ\ldots{}\ldots{}大きな魚丸ごと、および切り身。
\item
  ごく少量のクールブイヨンを用いたポシェ\ldots{}\ldots{}魚のフィレ、またはやや小さい魚。
\item
  ブレゼ\ldots{}\ldots{}もっぱら大きな魚。
\item
  オ・ブルー\footnote{比較的小さめの淡水魚に主として用いられる調理法。生きたままの魚の表面のぬめりをとらないように洗い、内臓を取り除いたらすぐに、塩とヴィネガーを加えたクールブイヨンで茹でる。冷製、温製どちらでも供する。原書p.281参照。}\ldots{}\ldots{}とりわけ\ruby{鱒}{ます}、鯉、ブロシェ\footnote{川かますの一種。本連載「ブロシェのクネル」2011年10月号
    pp.124-125 参照。}に合う。
\item
  揚げもの\ldots{}\ldots{}もっぱら小さい魚、切り身。
\item
  ムニエール\ldots{}\ldots{}揚げものにするのと同じ小さい魚、切り身。
\item
  グリエ\ldots{}\ldots{}小さい魚、および切り身。
\item
  グラタン\ldots{}\ldots{}小さい魚、切り身。
\end{enumerate}

\hypertarget{ux5869ux6c34ux6e6fux304aux3088ux3073ux30afux30fcux30ebux30d6ux30a4ux30e8ux30f3bux3092ux7528ux3044ux305fux52a0ux71b1ux8abfux7406}{%
\subsection{塩水(湯)およびクールブイヨンBを用いた加熱調理}\label{ux5869ux6c34ux6e6fux304aux3088ux3073ux30afux30fcux30ebux30d6ux30a4ux30e8ux30f3bux3092ux7528ux3044ux305fux52a0ux71b1ux8abfux7406}}

魚を丸ごと調理する場合は、魚に合ったポワソニエール\footnote{大きな魚を丸ごと煮るための細長い鍋。魚の形を崩さずに取り出せるよう、中に専用の網を敷いて使う。似たものに、舌びらめ等の平たい魚にぴったり合う菱形をしたテュルボティエールがある。いずれも、できるだけ少量の煮汁で魚を加熱できるように工夫されたもの。(図参照)}を用いる。魚を掃除し(テュルボは水にさらして血抜きをし)、ひれ等を切り落して形を整え、ポワソニエールの網に乗せる。魚種に応じて塩水または冷たいクールブイヨンをかぶるまで注ぐ。強火にかけて沸騰したらすぐにレンジの火の弱いところに鍋を移動させ、ポシェする。

切り身(薄すぎは絶対にいけない)の場合、沸騰した液体(塩湯またはクールブイヨン)に投入したらすぐにレンジの火の弱いところに鍋を移動させ、沸騰しない程度の温度でゆっくりと火を通す。

こうするのは、魚の身のエキスを閉じこめるためである。冷水から火にかけた場合にはエキスの大部分が流れ出してしまう。大きな魚丸ごとの場合にはこのやり方はしない。沸騰した液体に魚を投入すると身が収縮するので、大きな魚の場合は身が割れたり形が崩れたりするからだ。

塩湯あるいはクールブイヨンでポシェした魚は、ナフキンまたは専用の網に盛る。周囲をパセリで飾り、塩茹でしたじゃがいもと1種類または数種のソースを添えて供する。ガルニテュールがパセリのみの場合、魚の周囲にパセリを飾るのは客に料理を見せる\footnote{当時の宴席で主流だったロシア式サーヴィスでは、大きな銀盆に盛った料理をまず食客に見せてから、とり分けて給仕する。}直前にすること。どんな場合でも、ガルニテュールを添えたらクロッシュ\footnote{銀または陶製の保温用皿カバー。}は被せないこと。

\hypertarget{ux3054ux304fux5c11ux91cfux306eux30afux30fcux30ebux30d6ux30a4ux30e8ux30f3ux3092ux7528ux3044ux305fux30ddux30b7ux30a7-ux539fux66f8-pp.279-280}{%
\subsection{ごく少量のクールブイヨンを用いたポシェ 原書
pp.279-280}\label{ux3054ux304fux5c11ux91cfux306eux30afux30fcux30ebux30d6ux30a4ux30e8ux30f3ux3092ux7528ux3044ux305fux30ddux30b7ux30a7-ux539fux66f8-pp.279-280}}

この火入れの方法は主としてテュルボタン、バルビュ、舌びらめ、丸ごとおよびそれぞれの魚のフィレで用いる。バターを塗った天板あるいはソテ鍋に魚丸ごとあるいはそのフィレを置き、軽く塩をして、所要量の魚のフュメかマッシュルームの煮汁を注ぐ。フュメとマッシュルームの煮汁を合わせたものを用いる場合もある。蓋をして、中温のオーヴンに入れる。魚丸ごとの場合は時折煮汁をかけてやる。

魚(丸ごとでもフィレでも)に火が通ったら、注意して汁気をきり、皿に盛る。ガルニテュールを含む料理の場合、ガルニテュールを魚の周囲に盛り、ソースをかける\footnote{原文通りの順で訳したが、実際にはソースをかけてからガルニテュールを盛ったほうが良い場合もあるだろう。}。多くの場合、魚の煮汁を煮詰めてソースに加える。

\href{欠落アリ}{}

\hypertarget{ux9b5aux306eux30d6ux30ecux30bc17-ux539fux66f8-p.280}{%
\subsection[魚のブレゼ 原書 p.280]{\texorpdfstring{魚のブレゼ\footnote{ここではブレゼの語が限定的な意味で用いられていることに注意。牛のアロワイヨのような大きな塊肉のブレゼと同様の調理法、ということである。それは、香味素材を色づくまで炒めてから用いることや、主素材に豚背脂やトリュフをピケ針で差したり、豚背脂のシートで覆って加熱するという点によく表れている。ただし、これらは必須というわけではないため、事実上は「ごく少量のクールブイヨンを用いたポシェ」と区別がつきにくい。実際、モンタニェ『ラルース・ガストロノミーク』初版では、魚のブレゼについて「本来的な意味でのブレゼというよりは、ごく少量のクールブイヨンを用いたポシェである」と述べられている。逆に言えば、こんにち魚の加熱方法についてしばしば「ブレゼ」と呼ばれているものが、エスコフィエやモンタニェにおいては「少量のクールブイヨンを用いたポシェ」と表現されていたということである。}
原書
p.280}{魚のブレゼ 原書 p.280}}\label{ux9b5aux306eux30d6ux30ecux30bc17-ux539fux66f8-p.280}}

この調理法を用いるのは通常、丸ごとまたは筒切りにした鮭、大ぶりの鱒、テュルボ、テュルボタンのうち大きなもの、である。

場合によっては、魚の片面に、小さく切った豚背脂、トリュフ、コルニション、にんじん等をピケ針で差し込む。

香味素材等\footnote{原文 fonds de braisage フォン・ド・ブレザージュ
  (fonds de
  braiseフォン・ド・ブレーズ、とも)。通常は、厚い輪切りにしたにんじんと玉ねぎをバターか獣脂で色づくまで炒め、ブーケガルニ、下茹でした豚皮を合わせる(原書pp.394-395)。また、これを用いた煮汁のことも指す。}は肉料理のブレゼの場合と同じように用意するが、豚皮は用いない。提供方法に応じて、白または赤ワインと軽い魚のフュメ同量ずつを、魚の厚みの
\(\frac{3}{4}\) またはひたひたの高さまで注ぐ。厳密に肉断ち \footnote{カトリックの生活習慣として、四旬節(復活祭までの46日間)および週1
  回程度、肉類を食べないということが行なわれた(本連載2012年5月号「ソース・エスパニョル(4)」p.110、訳注1参照)。}のための仕立てにする場合を除いて、薄くスライスした豚背脂のシートを魚にかぶせる。加熱中\footnote{鍋を火にかけ、沸騰したら蓋をして中火のオーヴンに入れ、加熱する。}こまめに煮汁を魚にかけてやる\footnote{arroser
  アロゼ。}。また、完全には蓋をせず、加熱中に煮汁が煮詰まるようにしてやる。

ほぼ火が通ったら、鍋の蓋をとり、魚にかけた煮汁の水分をオーヴンの熱で蒸発させて表面につやを出す\footnote{glacer
  グラセ。}。魚を鍋から出して汁気をきり、皿に盛り保温しておく。

煮汁\footnote{原文 fonds de braisage (訳注2参照)。}を漉し、しばらく休ませたら浮き脂を取り除き、必要なら煮詰める。これを加えてソースを仕上げる。

魚のブレゼには通常、各ルセットに示してあるガルニテュールを添える。

\hypertarget{ux30aaux30d6ux30ebux30fc36-ux539fux66f8-pp.280-281}{%
\subsection[オ・ブルー 原書
pp.280-281]{\texorpdfstring{オ・ブルー\footnote{au bleu
  ヴィネガーを加えることで魚の表面のぬめりが青みがかることから。} 原書
pp.280-281}{オ・ブルー 原書 pp.280-281}}\label{ux30aaux30d6ux30ebux30fc36-ux539fux66f8-pp.280-281}}

オ・ブルーは鱒、鯉、ブロシェ\footnote{川かますの一種。}のみに用いられる特殊な調理法で、基本的なポイントは以下のとおり。

\begin{enumerate}
\def\labelenumi{\arabic{enumi}.}
\item
  必ず、生きた魚を使う。
\item
  魚の表面のぬめりをとらないように、なるべく手で触れずに、わたを抜く。鱗も引かない。
\item
  魚が大きい場合は、専用の網を敷いたポワソニーエルに入れ、「沸騰したヴィネガーをかける」。ヴィネガーは、通常のクールブイヨンに加える分量\footnote{以下の「クールブイヨンA」の分量比率を参照。}。次に、ヴィネガーを入れずに用意した温かい\footnote{原文
    tiède ぬるい、温かい。}クールブイヨンを注ぎ入れる。これは、なるべく身が割れないようにするためである。その後は通常どおり加熱する\footnote{レンジで沸騰させたらオーヴンに入れる。}。
\item
  小さい鱒の場合は、生きたままのものを手早く中抜きし、塩、ヴィネガーを加えただけの沸騰したクールブイヨンで煮る。
\item
  オ・ブルーは冷製、温製どちらの仕立てにしてもいい。実際の作り方の項で示してあるソースを添えて供する。
\end{enumerate}

\href{欠落アリ}{}

\hypertarget{ux30e0ux30cbux30a8ux30fcux30eb43-ux539fux66f8-p.282}{%
\subsection[ムニエール 原書 p.282]{\texorpdfstring{ムニエール\footnote{à
  la meunière 「粉挽き職人風」の意。} 原書
p.282}{ムニエール 原書 p.282}}\label{ux30e0ux30cbux30a8ux30fcux30eb43-ux539fux66f8-p.282}}

ムニエールは素晴らしい調理法だが、小型の魚と、大きな魚の場合は切り身にしか用いない。とはいえ、丁寧にやれば1.5
kg以下のテュルボタンはムニエールで調理できる。

魚丸ごと、あるいは切り身、フィレに味つけをして小麦粉をまぶし、バターを熱したフライパンで焼く。

魚が小さい場合は普通のバターでいいが、大きい場合は澄ましバターを使った方がいい。

魚の両面を焼き、程良く火が通ったら、予め熱しておいた皿に盛る。

飾り切りにした半割りのレモンを添えて、そのまま供することも可能である。ただし、このような提供方法の場合は本来の「ムニエール」と区別するために「黄金色に焼いた\footnote{doré
  (ドレ)
  一般的な色の表現として「黄金色」の意だが、ムニエールの場合、通常は大きい魚についてのみこの表現を用いる。}」と表現する。

「ムニエール」の場合には、焼き上がった魚に少量のレモン汁をふり、塩、こしょう少々で味を整える。粗みじん切りにして湯通ししたパセリを魚の表面に散らし、焦がしバターをかけてすぐに供する。湯通ししたパセリの水分に熱いバターが触れて泡がたつので、それが消えないうちに客に料理を見せるようにする。

%\href{✓原稿下準備なし}{} \href{訳と注釈\%2020180420進行中}{}
\href{未、原文対照チェック}{} \href{未、日本語表現校正}{}
\href{未、注釈チェク}{} \href{未、原稿最終校正}{}

\hypertarget{poissons}{%
\chapter{VI 魚料理 Poissons}\label{poissons}}

\hypertarget{serie-de-courts-bouillons-de-poisson}{%
\section{クールブイヨン}\label{serie-de-courts-bouillons-de-poisson}}

\begin{recette}
\hypertarget{ux30afux30fcux30ebux30d6ux30a4ux30e8ux30f3-a}{%
\subsubsection{クールブイヨン
A}\label{ux30afux30fcux30ebux30d6ux30a4ux30e8ux30f3-a}}

水5 L に対し、ヴィネガー2.5 dL、粗塩60 g、薄切りにしたにんじん600
gと玉ねぎ500 g、タイム1枝、ローリエの小さな葉2枚、パセリの茎100
g、粒こしょう20
g(こしょうを加えるのはクールブイヨンを漉す10分前)。材料を全て鍋に入れ、火にかけて1時間弱火で煮、漉す(原書
p.277)。

\hypertarget{ux30afux30fcux30ebux30d6ux30a4ux30e8ux30f3-b-4-ux9c52ux3046ux306aux304eux30d6ux30edux30b7ux30a7ux7b49-ux539fux66f8p.277}{%
\subsubsection[クールブイヨン B (鱒、うなぎ、ブロシェ等)
原書p.277]{\texorpdfstring{クールブイヨン B \footnote{クールブイヨンは用途に応じ、AからEまでの5種が挙げられている(原書pp.277-278)。}
(鱒、うなぎ、ブロシェ等)
原書p.277}{クールブイヨン B  (鱒、うなぎ、ブロシェ等) 原書p.277}}\label{ux30afux30fcux30ebux30d6ux30a4ux30e8ux30f3-b-4-ux9c52ux3046ux306aux304eux30d6ux30edux30b7ux30a7ux7b49-ux539fux66f8p.277}}

5 L 分の材料\ldots{}\ldots{}白ワイン2.5 L 。水2.5 L
。薄切りにした玉ねぎ600 g。パセリの茎80g
。タイムの小枝1本。ローリエの葉(小) \(\frac{1}{2}\) 枚。粗塩
60g。大粒のこしょう15 g(クールブイヨンを漉す10分前に加える)。

作業手順\ldots{}\ldots{}作業:液体、香味素材、調味料を鍋に入れ、沸かす。弱火で30分程煮て、漉す。

\href{欠落アリ}{}

原注:クールブイヨンBとC\footnote{クールブイヨンBの白ワインを赤ワインに代え、香味素材としてにんじん400gを加える。鱒、鯉、マトロート用(原書pp.277-278)。}で調理した魚はクールブイヨン添えとして供する。つまり、少量の煮汁とクールブイヨンに用いた野菜を添える。野菜はよく火が通っていること。煮汁はしっかり煮詰め、提供直前に新鮮なバター少量を加えて仕上げる。
\end{recette}
\hypertarget{ux30afux30fcux30ebux30d6ux30a4ux30e8ux30f3ux306eux4f7fux3044ux65b9-ux539fux66f8-p.278}{%
\subsection{クールブイヨンの使い方 原書
p.278}\label{ux30afux30fcux30ebux30d6ux30a4ux30e8ux30f3ux306eux4f7fux3044ux65b9-ux539fux66f8-p.278}}

\begin{enumerate}
\def\labelenumi{\arabic{enumi}.}
\item
  加熱時間が30分以内の場合は、クールブイヨンは必ず事前に用意しておくこと。
\item
  加熱時間が30分を越える場合は、クールブイヨンの材料は冷たい状態のままで合わせせておく。香味素材はポワソニエールの網の下に入れる。
\item
  ごく少量のクールブイヨンでポシェ\footnote{原文 pochage à court
    mouillement『ル・ギード・キュリネール』では、この表現はテュルボタン(小型のテュルボ)、バルビュ、舌びらめ等の平たい魚をポシェする際に用いられる。本連載「舌びらめのボヌ・ファム」
    2011年3月号pp.110-111 参照。}する場合、材料は(白または赤ワインを含む場合も)魚を火にかける際に合わせる。クールブイヨンの量は魚の
  \(\frac{1}{3}\)
  の高さとし、加熱中ひんぱんに煮汁を魚にかけてやること\footnote{arroser
    アロゼ。}。この調理法の場合は通常、クールブイヨンは上で記したように
  \footnote{「クールブイヨンB」原注。}、提供直前に軽くバターを加えて仕上げ、魚に添える。
\item
  冷製にする場合は、必ずクールブイヨンに魚が浸った状態で冷ますこと。当然ながら、火にかけている時間は短かくなる\footnote{余熱で火が通るため。}。
\end{enumerate}

\hypertarget{ux539fux6ce8}{%
\subparagraph{【原注】}\label{ux539fux6ce8}}

いくつかの魚種の加熱時間は該当する項で示してある。

\hypertarget{ux9b5aux306eux8abfux7406ux6cd5}{%
\section{魚の調理法}\label{ux9b5aux306eux8abfux7406ux6cd5}}

魚料理は全て、下記のいずれかの調理法による。

\begin{enumerate}
\def\labelenumi{\arabic{enumi}.}
\item
  塩水(湯)またはクールブイヨン\footnote{court-bouillon直訳は「量の少ない煮汁」。魚の他、甲殻類、鶏などの白身肉、野菜などをポシェするのに用いる。とりわけ魚や鶏を丸ごとポシェする場合には、その名称のとおり、できるだけ少量でポシェする必要がある。また、ポシェに用いたクールブイヨンをベースにソースを作る場合が多い。}Bを用いたポシェ\ldots{}\ldots{}大きな魚丸ごと、および切り身。
\item
  ごく少量のクールブイヨンを用いたポシェ\ldots{}\ldots{}魚のフィレ、またはやや小さい魚。
\item
  ブレゼ\ldots{}\ldots{}もっぱら大きな魚。
\item
  オ・ブルー\footnote{比較的小さめの淡水魚に主として用いられる調理法。生きたままの魚の表面のぬめりをとらないように洗い、内臓を取り除いたらすぐに、塩とヴィネガーを加えたクールブイヨンで茹でる。冷製、温製どちらでも供する。原書p.281参照。}\ldots{}\ldots{}とりわけ\ruby{鱒}{ます}、鯉、ブロシェ\footnote{川かますの一種。本連載「ブロシェのクネル」2011年10月号
    pp.124-125 参照。}に合う。
\item
  揚げもの\ldots{}\ldots{}もっぱら小さい魚、切り身。
\item
  ムニエール\ldots{}\ldots{}揚げものにするのと同じ小さい魚、切り身。
\item
  グリエ\ldots{}\ldots{}小さい魚、および切り身。
\item
  グラタン\ldots{}\ldots{}小さい魚、切り身。
\end{enumerate}

\hypertarget{ux5869ux6c34ux6e6fux304aux3088ux3073ux30afux30fcux30ebux30d6ux30a4ux30e8ux30f3bux3092ux7528ux3044ux305fux52a0ux71b1ux8abfux7406}{%
\subsection{塩水(湯)およびクールブイヨンBを用いた加熱調理}\label{ux5869ux6c34ux6e6fux304aux3088ux3073ux30afux30fcux30ebux30d6ux30a4ux30e8ux30f3bux3092ux7528ux3044ux305fux52a0ux71b1ux8abfux7406}}

魚を丸ごと調理する場合は、魚に合ったポワソニエール\footnote{大きな魚を丸ごと煮るための細長い鍋。魚の形を崩さずに取り出せるよう、中に専用の網を敷いて使う。似たものに、舌びらめ等の平たい魚にぴったり合う菱形をしたテュルボティエールがある。いずれも、できるだけ少量の煮汁で魚を加熱できるように工夫されたもの。(図参照)}を用いる。魚を掃除し(テュルボは水にさらして血抜きをし)、ひれ等を切り落して形を整え、ポワソニエールの網に乗せる。魚種に応じて塩水または冷たいクールブイヨンをかぶるまで注ぐ。強火にかけて沸騰したらすぐにレンジの火の弱いところに鍋を移動させ、ポシェする。

切り身(薄すぎは絶対にいけない)の場合、沸騰した液体(塩湯またはクールブイヨン)に投入したらすぐにレンジの火の弱いところに鍋を移動させ、沸騰しない程度の温度でゆっくりと火を通す。

こうするのは、魚の身のエキスを閉じこめるためである。冷水から火にかけた場合にはエキスの大部分が流れ出してしまう。大きな魚丸ごとの場合にはこのやり方はしない。沸騰した液体に魚を投入すると身が収縮するので、大きな魚の場合は身が割れたり形が崩れたりするからだ。

塩湯あるいはクールブイヨンでポシェした魚は、ナフキンまたは専用の網に盛る。周囲をパセリで飾り、塩茹でしたじゃがいもと1種類または数種のソースを添えて供する。ガルニテュールがパセリのみの場合、魚の周囲にパセリを飾るのは客に料理を見せる\footnote{当時の宴席で主流だったロシア式サーヴィスでは、大きな銀盆に盛った料理をまず食客に見せてから、とり分けて給仕する。}直前にすること。どんな場合でも、ガルニテュールを添えたらクロッシュ\footnote{銀または陶製の保温用皿カバー。}は被せないこと。

\hypertarget{ux3054ux304fux5c11ux91cfux306eux30afux30fcux30ebux30d6ux30a4ux30e8ux30f3ux3092ux7528ux3044ux305fux30ddux30b7ux30a7-ux539fux66f8-pp.279-280}{%
\subsection{ごく少量のクールブイヨンを用いたポシェ 原書
pp.279-280}\label{ux3054ux304fux5c11ux91cfux306eux30afux30fcux30ebux30d6ux30a4ux30e8ux30f3ux3092ux7528ux3044ux305fux30ddux30b7ux30a7-ux539fux66f8-pp.279-280}}

この火入れの方法は主としてテュルボタン、バルビュ、舌びらめ、丸ごとおよびそれぞれの魚のフィレで用いる。バターを塗った天板あるいはソテ鍋に魚丸ごとあるいはそのフィレを置き、軽く塩をして、所要量の魚のフュメかマッシュルームの煮汁を注ぐ。フュメとマッシュルームの煮汁を合わせたものを用いる場合もある。蓋をして、中温のオーヴンに入れる。魚丸ごとの場合は時折煮汁をかけてやる。

魚(丸ごとでもフィレでも)に火が通ったら、注意して汁気をきり、皿に盛る。ガルニテュールを含む料理の場合、ガルニテュールを魚の周囲に盛り、ソースをかける\footnote{原文通りの順で訳したが、実際にはソースをかけてからガルニテュールを盛ったほうが良い場合もあるだろう。}。多くの場合、魚の煮汁を煮詰めてソースに加える。

\href{欠落アリ}{}

\hypertarget{ux9b5aux306eux30d6ux30ecux30bc17-ux539fux66f8-p.280}{%
\subsection[魚のブレゼ 原書 p.280]{\texorpdfstring{魚のブレゼ\footnote{ここではブレゼの語が限定的な意味で用いられていることに注意。牛のアロワイヨのような大きな塊肉のブレゼと同様の調理法、ということである。それは、香味素材を色づくまで炒めてから用いることや、主素材に豚背脂やトリュフをピケ針で差したり、豚背脂のシートで覆って加熱するという点によく表れている。ただし、これらは必須というわけではないため、事実上は「ごく少量のクールブイヨンを用いたポシェ」と区別がつきにくい。実際、モンタニェ『ラルース・ガストロノミーク』初版では、魚のブレゼについて「本来的な意味でのブレゼというよりは、ごく少量のクールブイヨンを用いたポシェである」と述べられている。逆に言えば、こんにち魚の加熱方法についてしばしば「ブレゼ」と呼ばれているものが、エスコフィエやモンタニェにおいては「少量のクールブイヨンを用いたポシェ」と表現されていたということである。}
原書
p.280}{魚のブレゼ 原書 p.280}}\label{ux9b5aux306eux30d6ux30ecux30bc17-ux539fux66f8-p.280}}

この調理法を用いるのは通常、丸ごとまたは筒切りにした鮭、大ぶりの鱒、テュルボ、テュルボタンのうち大きなもの、である。

場合によっては、魚の片面に、小さく切った豚背脂、トリュフ、コルニション、にんじん等をピケ針で差し込む。

香味素材等\footnote{原文 fonds de braisage フォン・ド・ブレザージュ
  (fonds de
  braiseフォン・ド・ブレーズ、とも)。通常は、厚い輪切りにしたにんじんと玉ねぎをバターか獣脂で色づくまで炒め、ブーケガルニ、下茹でした豚皮を合わせる(原書pp.394-395)。また、これを用いた煮汁のことも指す。}は肉料理のブレゼの場合と同じように用意するが、豚皮は用いない。提供方法に応じて、白または赤ワインと軽い魚のフュメ同量ずつを、魚の厚みの
\(\frac{3}{4}\) またはひたひたの高さまで注ぐ。厳密に肉断ち \footnote{カトリックの生活習慣として、四旬節(復活祭までの46日間)および週1
  回程度、肉類を食べないということが行なわれた(本連載2012年5月号「ソース・エスパニョル(4)」p.110、訳注1参照)。}のための仕立てにする場合を除いて、薄くスライスした豚背脂のシートを魚にかぶせる。加熱中\footnote{鍋を火にかけ、沸騰したら蓋をして中火のオーヴンに入れ、加熱する。}こまめに煮汁を魚にかけてやる\footnote{arroser
  アロゼ。}。また、完全には蓋をせず、加熱中に煮汁が煮詰まるようにしてやる。

ほぼ火が通ったら、鍋の蓋をとり、魚にかけた煮汁の水分をオーヴンの熱で蒸発させて表面につやを出す\footnote{glacer
  グラセ。}。魚を鍋から出して汁気をきり、皿に盛り保温しておく。

煮汁\footnote{原文 fonds de braisage (訳注2参照)。}を漉し、しばらく休ませたら浮き脂を取り除き、必要なら煮詰める。これを加えてソースを仕上げる。

魚のブレゼには通常、各ルセットに示してあるガルニテュールを添える。

\hypertarget{ux30aaux30d6ux30ebux30fc36-ux539fux66f8-pp.280-281}{%
\subsection[オ・ブルー 原書
pp.280-281]{\texorpdfstring{オ・ブルー\footnote{au bleu
  ヴィネガーを加えることで魚の表面のぬめりが青みがかることから。} 原書
pp.280-281}{オ・ブルー 原書 pp.280-281}}\label{ux30aaux30d6ux30ebux30fc36-ux539fux66f8-pp.280-281}}

オ・ブルーは鱒、鯉、ブロシェ\footnote{川かますの一種。}のみに用いられる特殊な調理法で、基本的なポイントは以下のとおり。

\begin{enumerate}
\def\labelenumi{\arabic{enumi}.}
\item
  必ず、生きた魚を使う。
\item
  魚の表面のぬめりをとらないように、なるべく手で触れずに、わたを抜く。鱗も引かない。
\item
  魚が大きい場合は、専用の網を敷いたポワソニーエルに入れ、「沸騰したヴィネガーをかける」。ヴィネガーは、通常のクールブイヨンに加える分量\footnote{以下の「クールブイヨンA」の分量比率を参照。}。次に、ヴィネガーを入れずに用意した温かい\footnote{原文
    tiède ぬるい、温かい。}クールブイヨンを注ぎ入れる。これは、なるべく身が割れないようにするためである。その後は通常どおり加熱する\footnote{レンジで沸騰させたらオーヴンに入れる。}。
\item
  小さい鱒の場合は、生きたままのものを手早く中抜きし、塩、ヴィネガーを加えただけの沸騰したクールブイヨンで煮る。
\item
  オ・ブルーは冷製、温製どちらの仕立てにしてもいい。実際の作り方の項で示してあるソースを添えて供する。
\end{enumerate}

\href{欠落アリ}{}

\hypertarget{ux30e0ux30cbux30a8ux30fcux30eb43-ux539fux66f8-p.282}{%
\subsection[ムニエール 原書 p.282]{\texorpdfstring{ムニエール\footnote{à
  la meunière 「粉挽き職人風」の意。} 原書
p.282}{ムニエール 原書 p.282}}\label{ux30e0ux30cbux30a8ux30fcux30eb43-ux539fux66f8-p.282}}

ムニエールは素晴らしい調理法だが、小型の魚と、大きな魚の場合は切り身にしか用いない。とはいえ、丁寧にやれば1.5
kg以下のテュルボタンはムニエールで調理できる。

魚丸ごと、あるいは切り身、フィレに味つけをして小麦粉をまぶし、バターを熱したフライパンで焼く。

魚が小さい場合は普通のバターでいいが、大きい場合は澄ましバターを使った方がいい。

魚の両面を焼き、程良く火が通ったら、予め熱しておいた皿に盛る。

飾り切りにした半割りのレモンを添えて、そのまま供することも可能である。ただし、このような提供方法の場合は本来の「ムニエール」と区別するために「黄金色に焼いた\footnote{doré
  (ドレ)
  一般的な色の表現として「黄金色」の意だが、ムニエールの場合、通常は大きい魚についてのみこの表現を用いる。}」と表現する。

「ムニエール」の場合には、焼き上がった魚に少量のレモン汁をふり、塩、こしょう少々で味を整える。粗みじん切りにして湯通ししたパセリを魚の表面に散らし、焦がしバターをかけてすぐに供する。湯通ししたパセリの水分に熱いバターが触れて泡がたつので、それが消えないうちに客に料理を見せるようにする。

%\href{✓原稿下準備なし}{} \href{訳と注釈\%2020180420進行中}{}
\href{未、原文対照チェック}{} \href{未、日本語表現校正}{}
\href{未、注釈チェク}{} \href{未、原稿最終校正}{}

\hypertarget{poissons}{%
\chapter{VI 魚料理 Poissons}\label{poissons}}

\hypertarget{serie-de-courts-bouillons-de-poisson}{%
\section{クールブイヨン}\label{serie-de-courts-bouillons-de-poisson}}

\begin{recette}
\hypertarget{ux30afux30fcux30ebux30d6ux30a4ux30e8ux30f3-a}{%
\subsubsection{クールブイヨン
A}\label{ux30afux30fcux30ebux30d6ux30a4ux30e8ux30f3-a}}

水5 L に対し、ヴィネガー2.5 dL、粗塩60 g、薄切りにしたにんじん600
gと玉ねぎ500 g、タイム1枝、ローリエの小さな葉2枚、パセリの茎100
g、粒こしょう20
g(こしょうを加えるのはクールブイヨンを漉す10分前)。材料を全て鍋に入れ、火にかけて1時間弱火で煮、漉す(原書
p.277)。

\hypertarget{ux30afux30fcux30ebux30d6ux30a4ux30e8ux30f3-b-4-ux9c52ux3046ux306aux304eux30d6ux30edux30b7ux30a7ux7b49-ux539fux66f8p.277}{%
\subsubsection[クールブイヨン B (鱒、うなぎ、ブロシェ等)
原書p.277]{\texorpdfstring{クールブイヨン B \footnote{クールブイヨンは用途に応じ、AからEまでの5種が挙げられている(原書pp.277-278)。}
(鱒、うなぎ、ブロシェ等)
原書p.277}{クールブイヨン B  (鱒、うなぎ、ブロシェ等) 原書p.277}}\label{ux30afux30fcux30ebux30d6ux30a4ux30e8ux30f3-b-4-ux9c52ux3046ux306aux304eux30d6ux30edux30b7ux30a7ux7b49-ux539fux66f8p.277}}

5 L 分の材料\ldots{}\ldots{}白ワイン2.5 L 。水2.5 L
。薄切りにした玉ねぎ600 g。パセリの茎80g
。タイムの小枝1本。ローリエの葉(小) \(\frac{1}{2}\) 枚。粗塩
60g。大粒のこしょう15 g(クールブイヨンを漉す10分前に加える)。

作業手順\ldots{}\ldots{}作業:液体、香味素材、調味料を鍋に入れ、沸かす。弱火で30分程煮て、漉す。

\href{欠落アリ}{}

原注:クールブイヨンBとC\footnote{クールブイヨンBの白ワインを赤ワインに代え、香味素材としてにんじん400gを加える。鱒、鯉、マトロート用(原書pp.277-278)。}で調理した魚はクールブイヨン添えとして供する。つまり、少量の煮汁とクールブイヨンに用いた野菜を添える。野菜はよく火が通っていること。煮汁はしっかり煮詰め、提供直前に新鮮なバター少量を加えて仕上げる。
\end{recette}
\hypertarget{ux30afux30fcux30ebux30d6ux30a4ux30e8ux30f3ux306eux4f7fux3044ux65b9-ux539fux66f8-p.278}{%
\subsection{クールブイヨンの使い方 原書
p.278}\label{ux30afux30fcux30ebux30d6ux30a4ux30e8ux30f3ux306eux4f7fux3044ux65b9-ux539fux66f8-p.278}}

\begin{enumerate}
\def\labelenumi{\arabic{enumi}.}
\item
  加熱時間が30分以内の場合は、クールブイヨンは必ず事前に用意しておくこと。
\item
  加熱時間が30分を越える場合は、クールブイヨンの材料は冷たい状態のままで合わせせておく。香味素材はポワソニエールの網の下に入れる。
\item
  ごく少量のクールブイヨンでポシェ\footnote{原文 pochage à court
    mouillement『ル・ギード・キュリネール』では、この表現はテュルボタン(小型のテュルボ)、バルビュ、舌びらめ等の平たい魚をポシェする際に用いられる。本連載「舌びらめのボヌ・ファム」
    2011年3月号pp.110-111 参照。}する場合、材料は(白または赤ワインを含む場合も)魚を火にかける際に合わせる。クールブイヨンの量は魚の
  \(\frac{1}{3}\)
  の高さとし、加熱中ひんぱんに煮汁を魚にかけてやること\footnote{arroser
    アロゼ。}。この調理法の場合は通常、クールブイヨンは上で記したように
  \footnote{「クールブイヨンB」原注。}、提供直前に軽くバターを加えて仕上げ、魚に添える。
\item
  冷製にする場合は、必ずクールブイヨンに魚が浸った状態で冷ますこと。当然ながら、火にかけている時間は短かくなる\footnote{余熱で火が通るため。}。
\end{enumerate}

\hypertarget{ux539fux6ce8}{%
\subparagraph{【原注】}\label{ux539fux6ce8}}

いくつかの魚種の加熱時間は該当する項で示してある。

\hypertarget{ux9b5aux306eux8abfux7406ux6cd5}{%
\section{魚の調理法}\label{ux9b5aux306eux8abfux7406ux6cd5}}

魚料理は全て、下記のいずれかの調理法による。

\begin{enumerate}
\def\labelenumi{\arabic{enumi}.}
\item
  塩水(湯)またはクールブイヨン\footnote{court-bouillon直訳は「量の少ない煮汁」。魚の他、甲殻類、鶏などの白身肉、野菜などをポシェするのに用いる。とりわけ魚や鶏を丸ごとポシェする場合には、その名称のとおり、できるだけ少量でポシェする必要がある。また、ポシェに用いたクールブイヨンをベースにソースを作る場合が多い。}Bを用いたポシェ\ldots{}\ldots{}大きな魚丸ごと、および切り身。
\item
  ごく少量のクールブイヨンを用いたポシェ\ldots{}\ldots{}魚のフィレ、またはやや小さい魚。
\item
  ブレゼ\ldots{}\ldots{}もっぱら大きな魚。
\item
  オ・ブルー\footnote{比較的小さめの淡水魚に主として用いられる調理法。生きたままの魚の表面のぬめりをとらないように洗い、内臓を取り除いたらすぐに、塩とヴィネガーを加えたクールブイヨンで茹でる。冷製、温製どちらでも供する。原書p.281参照。}\ldots{}\ldots{}とりわけ\ruby{鱒}{ます}、鯉、ブロシェ\footnote{川かますの一種。本連載「ブロシェのクネル」2011年10月号
    pp.124-125 参照。}に合う。
\item
  揚げもの\ldots{}\ldots{}もっぱら小さい魚、切り身。
\item
  ムニエール\ldots{}\ldots{}揚げものにするのと同じ小さい魚、切り身。
\item
  グリエ\ldots{}\ldots{}小さい魚、および切り身。
\item
  グラタン\ldots{}\ldots{}小さい魚、切り身。
\end{enumerate}

\hypertarget{ux5869ux6c34ux6e6fux304aux3088ux3073ux30afux30fcux30ebux30d6ux30a4ux30e8ux30f3bux3092ux7528ux3044ux305fux52a0ux71b1ux8abfux7406}{%
\subsection{塩水(湯)およびクールブイヨンBを用いた加熱調理}\label{ux5869ux6c34ux6e6fux304aux3088ux3073ux30afux30fcux30ebux30d6ux30a4ux30e8ux30f3bux3092ux7528ux3044ux305fux52a0ux71b1ux8abfux7406}}

魚を丸ごと調理する場合は、魚に合ったポワソニエール\footnote{大きな魚を丸ごと煮るための細長い鍋。魚の形を崩さずに取り出せるよう、中に専用の網を敷いて使う。似たものに、舌びらめ等の平たい魚にぴったり合う菱形をしたテュルボティエールがある。いずれも、できるだけ少量の煮汁で魚を加熱できるように工夫されたもの。(図参照)}を用いる。魚を掃除し(テュルボは水にさらして血抜きをし)、ひれ等を切り落して形を整え、ポワソニエールの網に乗せる。魚種に応じて塩水または冷たいクールブイヨンをかぶるまで注ぐ。強火にかけて沸騰したらすぐにレンジの火の弱いところに鍋を移動させ、ポシェする。

切り身(薄すぎは絶対にいけない)の場合、沸騰した液体(塩湯またはクールブイヨン)に投入したらすぐにレンジの火の弱いところに鍋を移動させ、沸騰しない程度の温度でゆっくりと火を通す。

こうするのは、魚の身のエキスを閉じこめるためである。冷水から火にかけた場合にはエキスの大部分が流れ出してしまう。大きな魚丸ごとの場合にはこのやり方はしない。沸騰した液体に魚を投入すると身が収縮するので、大きな魚の場合は身が割れたり形が崩れたりするからだ。

塩湯あるいはクールブイヨンでポシェした魚は、ナフキンまたは専用の網に盛る。周囲をパセリで飾り、塩茹でしたじゃがいもと1種類または数種のソースを添えて供する。ガルニテュールがパセリのみの場合、魚の周囲にパセリを飾るのは客に料理を見せる\footnote{当時の宴席で主流だったロシア式サーヴィスでは、大きな銀盆に盛った料理をまず食客に見せてから、とり分けて給仕する。}直前にすること。どんな場合でも、ガルニテュールを添えたらクロッシュ\footnote{銀または陶製の保温用皿カバー。}は被せないこと。

\hypertarget{ux3054ux304fux5c11ux91cfux306eux30afux30fcux30ebux30d6ux30a4ux30e8ux30f3ux3092ux7528ux3044ux305fux30ddux30b7ux30a7-ux539fux66f8-pp.279-280}{%
\subsection{ごく少量のクールブイヨンを用いたポシェ 原書
pp.279-280}\label{ux3054ux304fux5c11ux91cfux306eux30afux30fcux30ebux30d6ux30a4ux30e8ux30f3ux3092ux7528ux3044ux305fux30ddux30b7ux30a7-ux539fux66f8-pp.279-280}}

この火入れの方法は主としてテュルボタン、バルビュ、舌びらめ、丸ごとおよびそれぞれの魚のフィレで用いる。バターを塗った天板あるいはソテ鍋に魚丸ごとあるいはそのフィレを置き、軽く塩をして、所要量の魚のフュメかマッシュルームの煮汁を注ぐ。フュメとマッシュルームの煮汁を合わせたものを用いる場合もある。蓋をして、中温のオーヴンに入れる。魚丸ごとの場合は時折煮汁をかけてやる。

魚(丸ごとでもフィレでも)に火が通ったら、注意して汁気をきり、皿に盛る。ガルニテュールを含む料理の場合、ガルニテュールを魚の周囲に盛り、ソースをかける\footnote{原文通りの順で訳したが、実際にはソースをかけてからガルニテュールを盛ったほうが良い場合もあるだろう。}。多くの場合、魚の煮汁を煮詰めてソースに加える。

\href{欠落アリ}{}

\hypertarget{ux9b5aux306eux30d6ux30ecux30bc17-ux539fux66f8-p.280}{%
\subsection[魚のブレゼ 原書 p.280]{\texorpdfstring{魚のブレゼ\footnote{ここではブレゼの語が限定的な意味で用いられていることに注意。牛のアロワイヨのような大きな塊肉のブレゼと同様の調理法、ということである。それは、香味素材を色づくまで炒めてから用いることや、主素材に豚背脂やトリュフをピケ針で差したり、豚背脂のシートで覆って加熱するという点によく表れている。ただし、これらは必須というわけではないため、事実上は「ごく少量のクールブイヨンを用いたポシェ」と区別がつきにくい。実際、モンタニェ『ラルース・ガストロノミーク』初版では、魚のブレゼについて「本来的な意味でのブレゼというよりは、ごく少量のクールブイヨンを用いたポシェである」と述べられている。逆に言えば、こんにち魚の加熱方法についてしばしば「ブレゼ」と呼ばれているものが、エスコフィエやモンタニェにおいては「少量のクールブイヨンを用いたポシェ」と表現されていたということである。}
原書
p.280}{魚のブレゼ 原書 p.280}}\label{ux9b5aux306eux30d6ux30ecux30bc17-ux539fux66f8-p.280}}

この調理法を用いるのは通常、丸ごとまたは筒切りにした鮭、大ぶりの鱒、テュルボ、テュルボタンのうち大きなもの、である。

場合によっては、魚の片面に、小さく切った豚背脂、トリュフ、コルニション、にんじん等をピケ針で差し込む。

香味素材等\footnote{原文 fonds de braisage フォン・ド・ブレザージュ
  (fonds de
  braiseフォン・ド・ブレーズ、とも)。通常は、厚い輪切りにしたにんじんと玉ねぎをバターか獣脂で色づくまで炒め、ブーケガルニ、下茹でした豚皮を合わせる(原書pp.394-395)。また、これを用いた煮汁のことも指す。}は肉料理のブレゼの場合と同じように用意するが、豚皮は用いない。提供方法に応じて、白または赤ワインと軽い魚のフュメ同量ずつを、魚の厚みの
\(\frac{3}{4}\) またはひたひたの高さまで注ぐ。厳密に肉断ち \footnote{カトリックの生活習慣として、四旬節(復活祭までの46日間)および週1
  回程度、肉類を食べないということが行なわれた(本連載2012年5月号「ソース・エスパニョル(4)」p.110、訳注1参照)。}のための仕立てにする場合を除いて、薄くスライスした豚背脂のシートを魚にかぶせる。加熱中\footnote{鍋を火にかけ、沸騰したら蓋をして中火のオーヴンに入れ、加熱する。}こまめに煮汁を魚にかけてやる\footnote{arroser
  アロゼ。}。また、完全には蓋をせず、加熱中に煮汁が煮詰まるようにしてやる。

ほぼ火が通ったら、鍋の蓋をとり、魚にかけた煮汁の水分をオーヴンの熱で蒸発させて表面につやを出す\footnote{glacer
  グラセ。}。魚を鍋から出して汁気をきり、皿に盛り保温しておく。

煮汁\footnote{原文 fonds de braisage (訳注2参照)。}を漉し、しばらく休ませたら浮き脂を取り除き、必要なら煮詰める。これを加えてソースを仕上げる。

魚のブレゼには通常、各ルセットに示してあるガルニテュールを添える。

\hypertarget{ux30aaux30d6ux30ebux30fc36-ux539fux66f8-pp.280-281}{%
\subsection[オ・ブルー 原書
pp.280-281]{\texorpdfstring{オ・ブルー\footnote{au bleu
  ヴィネガーを加えることで魚の表面のぬめりが青みがかることから。} 原書
pp.280-281}{オ・ブルー 原書 pp.280-281}}\label{ux30aaux30d6ux30ebux30fc36-ux539fux66f8-pp.280-281}}

オ・ブルーは鱒、鯉、ブロシェ\footnote{川かますの一種。}のみに用いられる特殊な調理法で、基本的なポイントは以下のとおり。

\begin{enumerate}
\def\labelenumi{\arabic{enumi}.}
\item
  必ず、生きた魚を使う。
\item
  魚の表面のぬめりをとらないように、なるべく手で触れずに、わたを抜く。鱗も引かない。
\item
  魚が大きい場合は、専用の網を敷いたポワソニーエルに入れ、「沸騰したヴィネガーをかける」。ヴィネガーは、通常のクールブイヨンに加える分量\footnote{以下の「クールブイヨンA」の分量比率を参照。}。次に、ヴィネガーを入れずに用意した温かい\footnote{原文
    tiède ぬるい、温かい。}クールブイヨンを注ぎ入れる。これは、なるべく身が割れないようにするためである。その後は通常どおり加熱する\footnote{レンジで沸騰させたらオーヴンに入れる。}。
\item
  小さい鱒の場合は、生きたままのものを手早く中抜きし、塩、ヴィネガーを加えただけの沸騰したクールブイヨンで煮る。
\item
  オ・ブルーは冷製、温製どちらの仕立てにしてもいい。実際の作り方の項で示してあるソースを添えて供する。
\end{enumerate}

\href{欠落アリ}{}

\hypertarget{ux30e0ux30cbux30a8ux30fcux30eb43-ux539fux66f8-p.282}{%
\subsection[ムニエール 原書 p.282]{\texorpdfstring{ムニエール\footnote{à
  la meunière 「粉挽き職人風」の意。} 原書
p.282}{ムニエール 原書 p.282}}\label{ux30e0ux30cbux30a8ux30fcux30eb43-ux539fux66f8-p.282}}

ムニエールは素晴らしい調理法だが、小型の魚と、大きな魚の場合は切り身にしか用いない。とはいえ、丁寧にやれば1.5
kg以下のテュルボタンはムニエールで調理できる。

魚丸ごと、あるいは切り身、フィレに味つけをして小麦粉をまぶし、バターを熱したフライパンで焼く。

魚が小さい場合は普通のバターでいいが、大きい場合は澄ましバターを使った方がいい。

魚の両面を焼き、程良く火が通ったら、予め熱しておいた皿に盛る。

飾り切りにした半割りのレモンを添えて、そのまま供することも可能である。ただし、このような提供方法の場合は本来の「ムニエール」と区別するために「黄金色に焼いた\footnote{doré
  (ドレ)
  一般的な色の表現として「黄金色」の意だが、ムニエールの場合、通常は大きい魚についてのみこの表現を用いる。}」と表現する。

「ムニエール」の場合には、焼き上がった魚に少量のレモン汁をふり、塩、こしょう少々で味を整える。粗みじん切りにして湯通ししたパセリを魚の表面に散らし、焦がしバターをかけてすぐに供する。湯通ししたパセリの水分に熱いバターが触れて泡がたつので、それが消えないうちに客に料理を見せるようにする。



%%% chapitre v. oeufs
%%%% エスコフィエ『料理の手引き』全注解
% 五島 学

\href{✓原稿下準備なし}{} \href{訳と注釈\%2020180420進行中}{}
\href{未、原文対照チェック}{} \href{未、日本語表現校正}{}
\href{未、注釈チェク}{} \href{未、原稿最終校正}{}

\hypertarget{poissons}{%
\chapter{VI 魚料理 Poissons}\label{poissons}}

\hypertarget{serie-de-courts-bouillons-de-poisson}{%
\section{クールブイヨン}\label{serie-de-courts-bouillons-de-poisson}}

\begin{recette}
\hypertarget{ux30afux30fcux30ebux30d6ux30a4ux30e8ux30f3-a}{%
\subsubsection{クールブイヨン
A}\label{ux30afux30fcux30ebux30d6ux30a4ux30e8ux30f3-a}}

水5 L に対し、ヴィネガー2.5 dL、粗塩60 g、薄切りにしたにんじん600
gと玉ねぎ500 g、タイム1枝、ローリエの小さな葉2枚、パセリの茎100
g、粒こしょう20
g(こしょうを加えるのはクールブイヨンを漉す10分前)。材料を全て鍋に入れ、火にかけて1時間弱火で煮、漉す(原書
p.277)。

\hypertarget{ux30afux30fcux30ebux30d6ux30a4ux30e8ux30f3-b-4-ux9c52ux3046ux306aux304eux30d6ux30edux30b7ux30a7ux7b49-ux539fux66f8p.277}{%
\subsubsection[クールブイヨン B (鱒、うなぎ、ブロシェ等)
原書p.277]{\texorpdfstring{クールブイヨン B \footnote{クールブイヨンは用途に応じ、AからEまでの5種が挙げられている(原書pp.277-278)。}
(鱒、うなぎ、ブロシェ等)
原書p.277}{クールブイヨン B  (鱒、うなぎ、ブロシェ等) 原書p.277}}\label{ux30afux30fcux30ebux30d6ux30a4ux30e8ux30f3-b-4-ux9c52ux3046ux306aux304eux30d6ux30edux30b7ux30a7ux7b49-ux539fux66f8p.277}}

5 L 分の材料\ldots{}\ldots{}白ワイン2.5 L 。水2.5 L
。薄切りにした玉ねぎ600 g。パセリの茎80g
。タイムの小枝1本。ローリエの葉(小) \(\frac{1}{2}\) 枚。粗塩
60g。大粒のこしょう15 g(クールブイヨンを漉す10分前に加える)。

作業手順\ldots{}\ldots{}作業:液体、香味素材、調味料を鍋に入れ、沸かす。弱火で30分程煮て、漉す。

\href{欠落アリ}{}

原注:クールブイヨンBとC\footnote{クールブイヨンBの白ワインを赤ワインに代え、香味素材としてにんじん400gを加える。鱒、鯉、マトロート用(原書pp.277-278)。}で調理した魚はクールブイヨン添えとして供する。つまり、少量の煮汁とクールブイヨンに用いた野菜を添える。野菜はよく火が通っていること。煮汁はしっかり煮詰め、提供直前に新鮮なバター少量を加えて仕上げる。
\end{recette}
\hypertarget{ux30afux30fcux30ebux30d6ux30a4ux30e8ux30f3ux306eux4f7fux3044ux65b9-ux539fux66f8-p.278}{%
\subsection{クールブイヨンの使い方 原書
p.278}\label{ux30afux30fcux30ebux30d6ux30a4ux30e8ux30f3ux306eux4f7fux3044ux65b9-ux539fux66f8-p.278}}

\begin{enumerate}
\def\labelenumi{\arabic{enumi}.}
\item
  加熱時間が30分以内の場合は、クールブイヨンは必ず事前に用意しておくこと。
\item
  加熱時間が30分を越える場合は、クールブイヨンの材料は冷たい状態のままで合わせせておく。香味素材はポワソニエールの網の下に入れる。
\item
  ごく少量のクールブイヨンでポシェ\footnote{原文 pochage à court
    mouillement『ル・ギード・キュリネール』では、この表現はテュルボタン(小型のテュルボ)、バルビュ、舌びらめ等の平たい魚をポシェする際に用いられる。本連載「舌びらめのボヌ・ファム」
    2011年3月号pp.110-111 参照。}する場合、材料は(白または赤ワインを含む場合も)魚を火にかける際に合わせる。クールブイヨンの量は魚の
  \(\frac{1}{3}\)
  の高さとし、加熱中ひんぱんに煮汁を魚にかけてやること\footnote{arroser
    アロゼ。}。この調理法の場合は通常、クールブイヨンは上で記したように
  \footnote{「クールブイヨンB」原注。}、提供直前に軽くバターを加えて仕上げ、魚に添える。
\item
  冷製にする場合は、必ずクールブイヨンに魚が浸った状態で冷ますこと。当然ながら、火にかけている時間は短かくなる\footnote{余熱で火が通るため。}。
\end{enumerate}

\hypertarget{ux539fux6ce8}{%
\subparagraph{【原注】}\label{ux539fux6ce8}}

いくつかの魚種の加熱時間は該当する項で示してある。

\hypertarget{ux9b5aux306eux8abfux7406ux6cd5}{%
\section{魚の調理法}\label{ux9b5aux306eux8abfux7406ux6cd5}}

魚料理は全て、下記のいずれかの調理法による。

\begin{enumerate}
\def\labelenumi{\arabic{enumi}.}
\item
  塩水(湯)またはクールブイヨン\footnote{court-bouillon直訳は「量の少ない煮汁」。魚の他、甲殻類、鶏などの白身肉、野菜などをポシェするのに用いる。とりわけ魚や鶏を丸ごとポシェする場合には、その名称のとおり、できるだけ少量でポシェする必要がある。また、ポシェに用いたクールブイヨンをベースにソースを作る場合が多い。}Bを用いたポシェ\ldots{}\ldots{}大きな魚丸ごと、および切り身。
\item
  ごく少量のクールブイヨンを用いたポシェ\ldots{}\ldots{}魚のフィレ、またはやや小さい魚。
\item
  ブレゼ\ldots{}\ldots{}もっぱら大きな魚。
\item
  オ・ブルー\footnote{比較的小さめの淡水魚に主として用いられる調理法。生きたままの魚の表面のぬめりをとらないように洗い、内臓を取り除いたらすぐに、塩とヴィネガーを加えたクールブイヨンで茹でる。冷製、温製どちらでも供する。原書p.281参照。}\ldots{}\ldots{}とりわけ\ruby{鱒}{ます}、鯉、ブロシェ\footnote{川かますの一種。本連載「ブロシェのクネル」2011年10月号
    pp.124-125 参照。}に合う。
\item
  揚げもの\ldots{}\ldots{}もっぱら小さい魚、切り身。
\item
  ムニエール\ldots{}\ldots{}揚げものにするのと同じ小さい魚、切り身。
\item
  グリエ\ldots{}\ldots{}小さい魚、および切り身。
\item
  グラタン\ldots{}\ldots{}小さい魚、切り身。
\end{enumerate}

\hypertarget{ux5869ux6c34ux6e6fux304aux3088ux3073ux30afux30fcux30ebux30d6ux30a4ux30e8ux30f3bux3092ux7528ux3044ux305fux52a0ux71b1ux8abfux7406}{%
\subsection{塩水(湯)およびクールブイヨンBを用いた加熱調理}\label{ux5869ux6c34ux6e6fux304aux3088ux3073ux30afux30fcux30ebux30d6ux30a4ux30e8ux30f3bux3092ux7528ux3044ux305fux52a0ux71b1ux8abfux7406}}

魚を丸ごと調理する場合は、魚に合ったポワソニエール\footnote{大きな魚を丸ごと煮るための細長い鍋。魚の形を崩さずに取り出せるよう、中に専用の網を敷いて使う。似たものに、舌びらめ等の平たい魚にぴったり合う菱形をしたテュルボティエールがある。いずれも、できるだけ少量の煮汁で魚を加熱できるように工夫されたもの。(図参照)}を用いる。魚を掃除し(テュルボは水にさらして血抜きをし)、ひれ等を切り落して形を整え、ポワソニエールの網に乗せる。魚種に応じて塩水または冷たいクールブイヨンをかぶるまで注ぐ。強火にかけて沸騰したらすぐにレンジの火の弱いところに鍋を移動させ、ポシェする。

切り身(薄すぎは絶対にいけない)の場合、沸騰した液体(塩湯またはクールブイヨン)に投入したらすぐにレンジの火の弱いところに鍋を移動させ、沸騰しない程度の温度でゆっくりと火を通す。

こうするのは、魚の身のエキスを閉じこめるためである。冷水から火にかけた場合にはエキスの大部分が流れ出してしまう。大きな魚丸ごとの場合にはこのやり方はしない。沸騰した液体に魚を投入すると身が収縮するので、大きな魚の場合は身が割れたり形が崩れたりするからだ。

塩湯あるいはクールブイヨンでポシェした魚は、ナフキンまたは専用の網に盛る。周囲をパセリで飾り、塩茹でしたじゃがいもと1種類または数種のソースを添えて供する。ガルニテュールがパセリのみの場合、魚の周囲にパセリを飾るのは客に料理を見せる\footnote{当時の宴席で主流だったロシア式サーヴィスでは、大きな銀盆に盛った料理をまず食客に見せてから、とり分けて給仕する。}直前にすること。どんな場合でも、ガルニテュールを添えたらクロッシュ\footnote{銀または陶製の保温用皿カバー。}は被せないこと。

\hypertarget{ux3054ux304fux5c11ux91cfux306eux30afux30fcux30ebux30d6ux30a4ux30e8ux30f3ux3092ux7528ux3044ux305fux30ddux30b7ux30a7-ux539fux66f8-pp.279-280}{%
\subsection{ごく少量のクールブイヨンを用いたポシェ 原書
pp.279-280}\label{ux3054ux304fux5c11ux91cfux306eux30afux30fcux30ebux30d6ux30a4ux30e8ux30f3ux3092ux7528ux3044ux305fux30ddux30b7ux30a7-ux539fux66f8-pp.279-280}}

この火入れの方法は主としてテュルボタン、バルビュ、舌びらめ、丸ごとおよびそれぞれの魚のフィレで用いる。バターを塗った天板あるいはソテ鍋に魚丸ごとあるいはそのフィレを置き、軽く塩をして、所要量の魚のフュメかマッシュルームの煮汁を注ぐ。フュメとマッシュルームの煮汁を合わせたものを用いる場合もある。蓋をして、中温のオーヴンに入れる。魚丸ごとの場合は時折煮汁をかけてやる。

魚(丸ごとでもフィレでも)に火が通ったら、注意して汁気をきり、皿に盛る。ガルニテュールを含む料理の場合、ガルニテュールを魚の周囲に盛り、ソースをかける\footnote{原文通りの順で訳したが、実際にはソースをかけてからガルニテュールを盛ったほうが良い場合もあるだろう。}。多くの場合、魚の煮汁を煮詰めてソースに加える。

\href{欠落アリ}{}

\hypertarget{ux9b5aux306eux30d6ux30ecux30bc17-ux539fux66f8-p.280}{%
\subsection[魚のブレゼ 原書 p.280]{\texorpdfstring{魚のブレゼ\footnote{ここではブレゼの語が限定的な意味で用いられていることに注意。牛のアロワイヨのような大きな塊肉のブレゼと同様の調理法、ということである。それは、香味素材を色づくまで炒めてから用いることや、主素材に豚背脂やトリュフをピケ針で差したり、豚背脂のシートで覆って加熱するという点によく表れている。ただし、これらは必須というわけではないため、事実上は「ごく少量のクールブイヨンを用いたポシェ」と区別がつきにくい。実際、モンタニェ『ラルース・ガストロノミーク』初版では、魚のブレゼについて「本来的な意味でのブレゼというよりは、ごく少量のクールブイヨンを用いたポシェである」と述べられている。逆に言えば、こんにち魚の加熱方法についてしばしば「ブレゼ」と呼ばれているものが、エスコフィエやモンタニェにおいては「少量のクールブイヨンを用いたポシェ」と表現されていたということである。}
原書
p.280}{魚のブレゼ 原書 p.280}}\label{ux9b5aux306eux30d6ux30ecux30bc17-ux539fux66f8-p.280}}

この調理法を用いるのは通常、丸ごとまたは筒切りにした鮭、大ぶりの鱒、テュルボ、テュルボタンのうち大きなもの、である。

場合によっては、魚の片面に、小さく切った豚背脂、トリュフ、コルニション、にんじん等をピケ針で差し込む。

香味素材等\footnote{原文 fonds de braisage フォン・ド・ブレザージュ
  (fonds de
  braiseフォン・ド・ブレーズ、とも)。通常は、厚い輪切りにしたにんじんと玉ねぎをバターか獣脂で色づくまで炒め、ブーケガルニ、下茹でした豚皮を合わせる(原書pp.394-395)。また、これを用いた煮汁のことも指す。}は肉料理のブレゼの場合と同じように用意するが、豚皮は用いない。提供方法に応じて、白または赤ワインと軽い魚のフュメ同量ずつを、魚の厚みの
\(\frac{3}{4}\) またはひたひたの高さまで注ぐ。厳密に肉断ち \footnote{カトリックの生活習慣として、四旬節(復活祭までの46日間)および週1
  回程度、肉類を食べないということが行なわれた(本連載2012年5月号「ソース・エスパニョル(4)」p.110、訳注1参照)。}のための仕立てにする場合を除いて、薄くスライスした豚背脂のシートを魚にかぶせる。加熱中\footnote{鍋を火にかけ、沸騰したら蓋をして中火のオーヴンに入れ、加熱する。}こまめに煮汁を魚にかけてやる\footnote{arroser
  アロゼ。}。また、完全には蓋をせず、加熱中に煮汁が煮詰まるようにしてやる。

ほぼ火が通ったら、鍋の蓋をとり、魚にかけた煮汁の水分をオーヴンの熱で蒸発させて表面につやを出す\footnote{glacer
  グラセ。}。魚を鍋から出して汁気をきり、皿に盛り保温しておく。

煮汁\footnote{原文 fonds de braisage (訳注2参照)。}を漉し、しばらく休ませたら浮き脂を取り除き、必要なら煮詰める。これを加えてソースを仕上げる。

魚のブレゼには通常、各ルセットに示してあるガルニテュールを添える。

\hypertarget{ux30aaux30d6ux30ebux30fc36-ux539fux66f8-pp.280-281}{%
\subsection[オ・ブルー 原書
pp.280-281]{\texorpdfstring{オ・ブルー\footnote{au bleu
  ヴィネガーを加えることで魚の表面のぬめりが青みがかることから。} 原書
pp.280-281}{オ・ブルー 原書 pp.280-281}}\label{ux30aaux30d6ux30ebux30fc36-ux539fux66f8-pp.280-281}}

オ・ブルーは鱒、鯉、ブロシェ\footnote{川かますの一種。}のみに用いられる特殊な調理法で、基本的なポイントは以下のとおり。

\begin{enumerate}
\def\labelenumi{\arabic{enumi}.}
\item
  必ず、生きた魚を使う。
\item
  魚の表面のぬめりをとらないように、なるべく手で触れずに、わたを抜く。鱗も引かない。
\item
  魚が大きい場合は、専用の網を敷いたポワソニーエルに入れ、「沸騰したヴィネガーをかける」。ヴィネガーは、通常のクールブイヨンに加える分量\footnote{以下の「クールブイヨンA」の分量比率を参照。}。次に、ヴィネガーを入れずに用意した温かい\footnote{原文
    tiède ぬるい、温かい。}クールブイヨンを注ぎ入れる。これは、なるべく身が割れないようにするためである。その後は通常どおり加熱する\footnote{レンジで沸騰させたらオーヴンに入れる。}。
\item
  小さい鱒の場合は、生きたままのものを手早く中抜きし、塩、ヴィネガーを加えただけの沸騰したクールブイヨンで煮る。
\item
  オ・ブルーは冷製、温製どちらの仕立てにしてもいい。実際の作り方の項で示してあるソースを添えて供する。
\end{enumerate}

\href{欠落アリ}{}

\hypertarget{ux30e0ux30cbux30a8ux30fcux30eb43-ux539fux66f8-p.282}{%
\subsection[ムニエール 原書 p.282]{\texorpdfstring{ムニエール\footnote{à
  la meunière 「粉挽き職人風」の意。} 原書
p.282}{ムニエール 原書 p.282}}\label{ux30e0ux30cbux30a8ux30fcux30eb43-ux539fux66f8-p.282}}

ムニエールは素晴らしい調理法だが、小型の魚と、大きな魚の場合は切り身にしか用いない。とはいえ、丁寧にやれば1.5
kg以下のテュルボタンはムニエールで調理できる。

魚丸ごと、あるいは切り身、フィレに味つけをして小麦粉をまぶし、バターを熱したフライパンで焼く。

魚が小さい場合は普通のバターでいいが、大きい場合は澄ましバターを使った方がいい。

魚の両面を焼き、程良く火が通ったら、予め熱しておいた皿に盛る。

飾り切りにした半割りのレモンを添えて、そのまま供することも可能である。ただし、このような提供方法の場合は本来の「ムニエール」と区別するために「黄金色に焼いた\footnote{doré
  (ドレ)
  一般的な色の表現として「黄金色」の意だが、ムニエールの場合、通常は大きい魚についてのみこの表現を用いる。}」と表現する。

「ムニエール」の場合には、焼き上がった魚に少量のレモン汁をふり、塩、こしょう少々で味を整える。粗みじん切りにして湯通ししたパセリを魚の表面に散らし、焦がしバターをかけてすぐに供する。湯通ししたパセリの水分に熱いバターが触れて泡がたつので、それが消えないうちに客に料理を見せるようにする。

%\input{06-poissons/06-02-p-284}
%\input{06-poissons/06-03-pp285-286}
%\input{06-poissons/06-04-p286}
%\input{06-poissons/06-05-pp286-290}

%\href{✓原稿下準備なし}{} \href{訳と注釈\%2020180420進行中}{}
\href{未、原文対照チェック}{} \href{未、日本語表現校正}{}
\href{未、注釈チェク}{} \href{未、原稿最終校正}{}

\hypertarget{poissons}{%
\chapter{VI 魚料理 Poissons}\label{poissons}}

\hypertarget{serie-de-courts-bouillons-de-poisson}{%
\section{クールブイヨン}\label{serie-de-courts-bouillons-de-poisson}}

\begin{recette}
\hypertarget{ux30afux30fcux30ebux30d6ux30a4ux30e8ux30f3-a}{%
\subsubsection{クールブイヨン
A}\label{ux30afux30fcux30ebux30d6ux30a4ux30e8ux30f3-a}}

水5 L に対し、ヴィネガー2.5 dL、粗塩60 g、薄切りにしたにんじん600
gと玉ねぎ500 g、タイム1枝、ローリエの小さな葉2枚、パセリの茎100
g、粒こしょう20
g(こしょうを加えるのはクールブイヨンを漉す10分前)。材料を全て鍋に入れ、火にかけて1時間弱火で煮、漉す(原書
p.277)。

\hypertarget{ux30afux30fcux30ebux30d6ux30a4ux30e8ux30f3-b-4-ux9c52ux3046ux306aux304eux30d6ux30edux30b7ux30a7ux7b49-ux539fux66f8p.277}{%
\subsubsection[クールブイヨン B (鱒、うなぎ、ブロシェ等)
原書p.277]{\texorpdfstring{クールブイヨン B \footnote{クールブイヨンは用途に応じ、AからEまでの5種が挙げられている(原書pp.277-278)。}
(鱒、うなぎ、ブロシェ等)
原書p.277}{クールブイヨン B  (鱒、うなぎ、ブロシェ等) 原書p.277}}\label{ux30afux30fcux30ebux30d6ux30a4ux30e8ux30f3-b-4-ux9c52ux3046ux306aux304eux30d6ux30edux30b7ux30a7ux7b49-ux539fux66f8p.277}}

5 L 分の材料\ldots{}\ldots{}白ワイン2.5 L 。水2.5 L
。薄切りにした玉ねぎ600 g。パセリの茎80g
。タイムの小枝1本。ローリエの葉(小) \(\frac{1}{2}\) 枚。粗塩
60g。大粒のこしょう15 g(クールブイヨンを漉す10分前に加える)。

作業手順\ldots{}\ldots{}作業:液体、香味素材、調味料を鍋に入れ、沸かす。弱火で30分程煮て、漉す。

\href{欠落アリ}{}

原注:クールブイヨンBとC\footnote{クールブイヨンBの白ワインを赤ワインに代え、香味素材としてにんじん400gを加える。鱒、鯉、マトロート用(原書pp.277-278)。}で調理した魚はクールブイヨン添えとして供する。つまり、少量の煮汁とクールブイヨンに用いた野菜を添える。野菜はよく火が通っていること。煮汁はしっかり煮詰め、提供直前に新鮮なバター少量を加えて仕上げる。
\end{recette}
\hypertarget{ux30afux30fcux30ebux30d6ux30a4ux30e8ux30f3ux306eux4f7fux3044ux65b9-ux539fux66f8-p.278}{%
\subsection{クールブイヨンの使い方 原書
p.278}\label{ux30afux30fcux30ebux30d6ux30a4ux30e8ux30f3ux306eux4f7fux3044ux65b9-ux539fux66f8-p.278}}

\begin{enumerate}
\def\labelenumi{\arabic{enumi}.}
\item
  加熱時間が30分以内の場合は、クールブイヨンは必ず事前に用意しておくこと。
\item
  加熱時間が30分を越える場合は、クールブイヨンの材料は冷たい状態のままで合わせせておく。香味素材はポワソニエールの網の下に入れる。
\item
  ごく少量のクールブイヨンでポシェ\footnote{原文 pochage à court
    mouillement『ル・ギード・キュリネール』では、この表現はテュルボタン(小型のテュルボ)、バルビュ、舌びらめ等の平たい魚をポシェする際に用いられる。本連載「舌びらめのボヌ・ファム」
    2011年3月号pp.110-111 参照。}する場合、材料は(白または赤ワインを含む場合も)魚を火にかける際に合わせる。クールブイヨンの量は魚の
  \(\frac{1}{3}\)
  の高さとし、加熱中ひんぱんに煮汁を魚にかけてやること\footnote{arroser
    アロゼ。}。この調理法の場合は通常、クールブイヨンは上で記したように
  \footnote{「クールブイヨンB」原注。}、提供直前に軽くバターを加えて仕上げ、魚に添える。
\item
  冷製にする場合は、必ずクールブイヨンに魚が浸った状態で冷ますこと。当然ながら、火にかけている時間は短かくなる\footnote{余熱で火が通るため。}。
\end{enumerate}

\hypertarget{ux539fux6ce8}{%
\subparagraph{【原注】}\label{ux539fux6ce8}}

いくつかの魚種の加熱時間は該当する項で示してある。

\hypertarget{ux9b5aux306eux8abfux7406ux6cd5}{%
\section{魚の調理法}\label{ux9b5aux306eux8abfux7406ux6cd5}}

魚料理は全て、下記のいずれかの調理法による。

\begin{enumerate}
\def\labelenumi{\arabic{enumi}.}
\item
  塩水(湯)またはクールブイヨン\footnote{court-bouillon直訳は「量の少ない煮汁」。魚の他、甲殻類、鶏などの白身肉、野菜などをポシェするのに用いる。とりわけ魚や鶏を丸ごとポシェする場合には、その名称のとおり、できるだけ少量でポシェする必要がある。また、ポシェに用いたクールブイヨンをベースにソースを作る場合が多い。}Bを用いたポシェ\ldots{}\ldots{}大きな魚丸ごと、および切り身。
\item
  ごく少量のクールブイヨンを用いたポシェ\ldots{}\ldots{}魚のフィレ、またはやや小さい魚。
\item
  ブレゼ\ldots{}\ldots{}もっぱら大きな魚。
\item
  オ・ブルー\footnote{比較的小さめの淡水魚に主として用いられる調理法。生きたままの魚の表面のぬめりをとらないように洗い、内臓を取り除いたらすぐに、塩とヴィネガーを加えたクールブイヨンで茹でる。冷製、温製どちらでも供する。原書p.281参照。}\ldots{}\ldots{}とりわけ\ruby{鱒}{ます}、鯉、ブロシェ\footnote{川かますの一種。本連載「ブロシェのクネル」2011年10月号
    pp.124-125 参照。}に合う。
\item
  揚げもの\ldots{}\ldots{}もっぱら小さい魚、切り身。
\item
  ムニエール\ldots{}\ldots{}揚げものにするのと同じ小さい魚、切り身。
\item
  グリエ\ldots{}\ldots{}小さい魚、および切り身。
\item
  グラタン\ldots{}\ldots{}小さい魚、切り身。
\end{enumerate}

\hypertarget{ux5869ux6c34ux6e6fux304aux3088ux3073ux30afux30fcux30ebux30d6ux30a4ux30e8ux30f3bux3092ux7528ux3044ux305fux52a0ux71b1ux8abfux7406}{%
\subsection{塩水(湯)およびクールブイヨンBを用いた加熱調理}\label{ux5869ux6c34ux6e6fux304aux3088ux3073ux30afux30fcux30ebux30d6ux30a4ux30e8ux30f3bux3092ux7528ux3044ux305fux52a0ux71b1ux8abfux7406}}

魚を丸ごと調理する場合は、魚に合ったポワソニエール\footnote{大きな魚を丸ごと煮るための細長い鍋。魚の形を崩さずに取り出せるよう、中に専用の網を敷いて使う。似たものに、舌びらめ等の平たい魚にぴったり合う菱形をしたテュルボティエールがある。いずれも、できるだけ少量の煮汁で魚を加熱できるように工夫されたもの。(図参照)}を用いる。魚を掃除し(テュルボは水にさらして血抜きをし)、ひれ等を切り落して形を整え、ポワソニエールの網に乗せる。魚種に応じて塩水または冷たいクールブイヨンをかぶるまで注ぐ。強火にかけて沸騰したらすぐにレンジの火の弱いところに鍋を移動させ、ポシェする。

切り身(薄すぎは絶対にいけない)の場合、沸騰した液体(塩湯またはクールブイヨン)に投入したらすぐにレンジの火の弱いところに鍋を移動させ、沸騰しない程度の温度でゆっくりと火を通す。

こうするのは、魚の身のエキスを閉じこめるためである。冷水から火にかけた場合にはエキスの大部分が流れ出してしまう。大きな魚丸ごとの場合にはこのやり方はしない。沸騰した液体に魚を投入すると身が収縮するので、大きな魚の場合は身が割れたり形が崩れたりするからだ。

塩湯あるいはクールブイヨンでポシェした魚は、ナフキンまたは専用の網に盛る。周囲をパセリで飾り、塩茹でしたじゃがいもと1種類または数種のソースを添えて供する。ガルニテュールがパセリのみの場合、魚の周囲にパセリを飾るのは客に料理を見せる\footnote{当時の宴席で主流だったロシア式サーヴィスでは、大きな銀盆に盛った料理をまず食客に見せてから、とり分けて給仕する。}直前にすること。どんな場合でも、ガルニテュールを添えたらクロッシュ\footnote{銀または陶製の保温用皿カバー。}は被せないこと。

\hypertarget{ux3054ux304fux5c11ux91cfux306eux30afux30fcux30ebux30d6ux30a4ux30e8ux30f3ux3092ux7528ux3044ux305fux30ddux30b7ux30a7-ux539fux66f8-pp.279-280}{%
\subsection{ごく少量のクールブイヨンを用いたポシェ 原書
pp.279-280}\label{ux3054ux304fux5c11ux91cfux306eux30afux30fcux30ebux30d6ux30a4ux30e8ux30f3ux3092ux7528ux3044ux305fux30ddux30b7ux30a7-ux539fux66f8-pp.279-280}}

この火入れの方法は主としてテュルボタン、バルビュ、舌びらめ、丸ごとおよびそれぞれの魚のフィレで用いる。バターを塗った天板あるいはソテ鍋に魚丸ごとあるいはそのフィレを置き、軽く塩をして、所要量の魚のフュメかマッシュルームの煮汁を注ぐ。フュメとマッシュルームの煮汁を合わせたものを用いる場合もある。蓋をして、中温のオーヴンに入れる。魚丸ごとの場合は時折煮汁をかけてやる。

魚(丸ごとでもフィレでも)に火が通ったら、注意して汁気をきり、皿に盛る。ガルニテュールを含む料理の場合、ガルニテュールを魚の周囲に盛り、ソースをかける\footnote{原文通りの順で訳したが、実際にはソースをかけてからガルニテュールを盛ったほうが良い場合もあるだろう。}。多くの場合、魚の煮汁を煮詰めてソースに加える。

\href{欠落アリ}{}

\hypertarget{ux9b5aux306eux30d6ux30ecux30bc17-ux539fux66f8-p.280}{%
\subsection[魚のブレゼ 原書 p.280]{\texorpdfstring{魚のブレゼ\footnote{ここではブレゼの語が限定的な意味で用いられていることに注意。牛のアロワイヨのような大きな塊肉のブレゼと同様の調理法、ということである。それは、香味素材を色づくまで炒めてから用いることや、主素材に豚背脂やトリュフをピケ針で差したり、豚背脂のシートで覆って加熱するという点によく表れている。ただし、これらは必須というわけではないため、事実上は「ごく少量のクールブイヨンを用いたポシェ」と区別がつきにくい。実際、モンタニェ『ラルース・ガストロノミーク』初版では、魚のブレゼについて「本来的な意味でのブレゼというよりは、ごく少量のクールブイヨンを用いたポシェである」と述べられている。逆に言えば、こんにち魚の加熱方法についてしばしば「ブレゼ」と呼ばれているものが、エスコフィエやモンタニェにおいては「少量のクールブイヨンを用いたポシェ」と表現されていたということである。}
原書
p.280}{魚のブレゼ 原書 p.280}}\label{ux9b5aux306eux30d6ux30ecux30bc17-ux539fux66f8-p.280}}

この調理法を用いるのは通常、丸ごとまたは筒切りにした鮭、大ぶりの鱒、テュルボ、テュルボタンのうち大きなもの、である。

場合によっては、魚の片面に、小さく切った豚背脂、トリュフ、コルニション、にんじん等をピケ針で差し込む。

香味素材等\footnote{原文 fonds de braisage フォン・ド・ブレザージュ
  (fonds de
  braiseフォン・ド・ブレーズ、とも)。通常は、厚い輪切りにしたにんじんと玉ねぎをバターか獣脂で色づくまで炒め、ブーケガルニ、下茹でした豚皮を合わせる(原書pp.394-395)。また、これを用いた煮汁のことも指す。}は肉料理のブレゼの場合と同じように用意するが、豚皮は用いない。提供方法に応じて、白または赤ワインと軽い魚のフュメ同量ずつを、魚の厚みの
\(\frac{3}{4}\) またはひたひたの高さまで注ぐ。厳密に肉断ち \footnote{カトリックの生活習慣として、四旬節(復活祭までの46日間)および週1
  回程度、肉類を食べないということが行なわれた(本連載2012年5月号「ソース・エスパニョル(4)」p.110、訳注1参照)。}のための仕立てにする場合を除いて、薄くスライスした豚背脂のシートを魚にかぶせる。加熱中\footnote{鍋を火にかけ、沸騰したら蓋をして中火のオーヴンに入れ、加熱する。}こまめに煮汁を魚にかけてやる\footnote{arroser
  アロゼ。}。また、完全には蓋をせず、加熱中に煮汁が煮詰まるようにしてやる。

ほぼ火が通ったら、鍋の蓋をとり、魚にかけた煮汁の水分をオーヴンの熱で蒸発させて表面につやを出す\footnote{glacer
  グラセ。}。魚を鍋から出して汁気をきり、皿に盛り保温しておく。

煮汁\footnote{原文 fonds de braisage (訳注2参照)。}を漉し、しばらく休ませたら浮き脂を取り除き、必要なら煮詰める。これを加えてソースを仕上げる。

魚のブレゼには通常、各ルセットに示してあるガルニテュールを添える。

\hypertarget{ux30aaux30d6ux30ebux30fc36-ux539fux66f8-pp.280-281}{%
\subsection[オ・ブルー 原書
pp.280-281]{\texorpdfstring{オ・ブルー\footnote{au bleu
  ヴィネガーを加えることで魚の表面のぬめりが青みがかることから。} 原書
pp.280-281}{オ・ブルー 原書 pp.280-281}}\label{ux30aaux30d6ux30ebux30fc36-ux539fux66f8-pp.280-281}}

オ・ブルーは鱒、鯉、ブロシェ\footnote{川かますの一種。}のみに用いられる特殊な調理法で、基本的なポイントは以下のとおり。

\begin{enumerate}
\def\labelenumi{\arabic{enumi}.}
\item
  必ず、生きた魚を使う。
\item
  魚の表面のぬめりをとらないように、なるべく手で触れずに、わたを抜く。鱗も引かない。
\item
  魚が大きい場合は、専用の網を敷いたポワソニーエルに入れ、「沸騰したヴィネガーをかける」。ヴィネガーは、通常のクールブイヨンに加える分量\footnote{以下の「クールブイヨンA」の分量比率を参照。}。次に、ヴィネガーを入れずに用意した温かい\footnote{原文
    tiède ぬるい、温かい。}クールブイヨンを注ぎ入れる。これは、なるべく身が割れないようにするためである。その後は通常どおり加熱する\footnote{レンジで沸騰させたらオーヴンに入れる。}。
\item
  小さい鱒の場合は、生きたままのものを手早く中抜きし、塩、ヴィネガーを加えただけの沸騰したクールブイヨンで煮る。
\item
  オ・ブルーは冷製、温製どちらの仕立てにしてもいい。実際の作り方の項で示してあるソースを添えて供する。
\end{enumerate}

\href{欠落アリ}{}

\hypertarget{ux30e0ux30cbux30a8ux30fcux30eb43-ux539fux66f8-p.282}{%
\subsection[ムニエール 原書 p.282]{\texorpdfstring{ムニエール\footnote{à
  la meunière 「粉挽き職人風」の意。} 原書
p.282}{ムニエール 原書 p.282}}\label{ux30e0ux30cbux30a8ux30fcux30eb43-ux539fux66f8-p.282}}

ムニエールは素晴らしい調理法だが、小型の魚と、大きな魚の場合は切り身にしか用いない。とはいえ、丁寧にやれば1.5
kg以下のテュルボタンはムニエールで調理できる。

魚丸ごと、あるいは切り身、フィレに味つけをして小麦粉をまぶし、バターを熱したフライパンで焼く。

魚が小さい場合は普通のバターでいいが、大きい場合は澄ましバターを使った方がいい。

魚の両面を焼き、程良く火が通ったら、予め熱しておいた皿に盛る。

飾り切りにした半割りのレモンを添えて、そのまま供することも可能である。ただし、このような提供方法の場合は本来の「ムニエール」と区別するために「黄金色に焼いた\footnote{doré
  (ドレ)
  一般的な色の表現として「黄金色」の意だが、ムニエールの場合、通常は大きい魚についてのみこの表現を用いる。}」と表現する。

「ムニエール」の場合には、焼き上がった魚に少量のレモン汁をふり、塩、こしょう少々で味を整える。粗みじん切りにして湯通ししたパセリを魚の表面に散らし、焦がしバターをかけてすぐに供する。湯通ししたパセリの水分に熱いバターが触れて泡がたつので、それが消えないうちに客に料理を見せるようにする。

%\href{✓原稿下準備なし}{} \href{訳と注釈\%2020180420進行中}{}
\href{未、原文対照チェック}{} \href{未、日本語表現校正}{}
\href{未、注釈チェク}{} \href{未、原稿最終校正}{}

\hypertarget{poissons}{%
\chapter{VI 魚料理 Poissons}\label{poissons}}

\hypertarget{serie-de-courts-bouillons-de-poisson}{%
\section{クールブイヨン}\label{serie-de-courts-bouillons-de-poisson}}

\begin{recette}
\hypertarget{ux30afux30fcux30ebux30d6ux30a4ux30e8ux30f3-a}{%
\subsubsection{クールブイヨン
A}\label{ux30afux30fcux30ebux30d6ux30a4ux30e8ux30f3-a}}

水5 L に対し、ヴィネガー2.5 dL、粗塩60 g、薄切りにしたにんじん600
gと玉ねぎ500 g、タイム1枝、ローリエの小さな葉2枚、パセリの茎100
g、粒こしょう20
g(こしょうを加えるのはクールブイヨンを漉す10分前)。材料を全て鍋に入れ、火にかけて1時間弱火で煮、漉す(原書
p.277)。

\hypertarget{ux30afux30fcux30ebux30d6ux30a4ux30e8ux30f3-b-4-ux9c52ux3046ux306aux304eux30d6ux30edux30b7ux30a7ux7b49-ux539fux66f8p.277}{%
\subsubsection[クールブイヨン B (鱒、うなぎ、ブロシェ等)
原書p.277]{\texorpdfstring{クールブイヨン B \footnote{クールブイヨンは用途に応じ、AからEまでの5種が挙げられている(原書pp.277-278)。}
(鱒、うなぎ、ブロシェ等)
原書p.277}{クールブイヨン B  (鱒、うなぎ、ブロシェ等) 原書p.277}}\label{ux30afux30fcux30ebux30d6ux30a4ux30e8ux30f3-b-4-ux9c52ux3046ux306aux304eux30d6ux30edux30b7ux30a7ux7b49-ux539fux66f8p.277}}

5 L 分の材料\ldots{}\ldots{}白ワイン2.5 L 。水2.5 L
。薄切りにした玉ねぎ600 g。パセリの茎80g
。タイムの小枝1本。ローリエの葉(小) \(\frac{1}{2}\) 枚。粗塩
60g。大粒のこしょう15 g(クールブイヨンを漉す10分前に加える)。

作業手順\ldots{}\ldots{}作業:液体、香味素材、調味料を鍋に入れ、沸かす。弱火で30分程煮て、漉す。

\href{欠落アリ}{}

原注:クールブイヨンBとC\footnote{クールブイヨンBの白ワインを赤ワインに代え、香味素材としてにんじん400gを加える。鱒、鯉、マトロート用(原書pp.277-278)。}で調理した魚はクールブイヨン添えとして供する。つまり、少量の煮汁とクールブイヨンに用いた野菜を添える。野菜はよく火が通っていること。煮汁はしっかり煮詰め、提供直前に新鮮なバター少量を加えて仕上げる。
\end{recette}
\hypertarget{ux30afux30fcux30ebux30d6ux30a4ux30e8ux30f3ux306eux4f7fux3044ux65b9-ux539fux66f8-p.278}{%
\subsection{クールブイヨンの使い方 原書
p.278}\label{ux30afux30fcux30ebux30d6ux30a4ux30e8ux30f3ux306eux4f7fux3044ux65b9-ux539fux66f8-p.278}}

\begin{enumerate}
\def\labelenumi{\arabic{enumi}.}
\item
  加熱時間が30分以内の場合は、クールブイヨンは必ず事前に用意しておくこと。
\item
  加熱時間が30分を越える場合は、クールブイヨンの材料は冷たい状態のままで合わせせておく。香味素材はポワソニエールの網の下に入れる。
\item
  ごく少量のクールブイヨンでポシェ\footnote{原文 pochage à court
    mouillement『ル・ギード・キュリネール』では、この表現はテュルボタン(小型のテュルボ)、バルビュ、舌びらめ等の平たい魚をポシェする際に用いられる。本連載「舌びらめのボヌ・ファム」
    2011年3月号pp.110-111 参照。}する場合、材料は(白または赤ワインを含む場合も)魚を火にかける際に合わせる。クールブイヨンの量は魚の
  \(\frac{1}{3}\)
  の高さとし、加熱中ひんぱんに煮汁を魚にかけてやること\footnote{arroser
    アロゼ。}。この調理法の場合は通常、クールブイヨンは上で記したように
  \footnote{「クールブイヨンB」原注。}、提供直前に軽くバターを加えて仕上げ、魚に添える。
\item
  冷製にする場合は、必ずクールブイヨンに魚が浸った状態で冷ますこと。当然ながら、火にかけている時間は短かくなる\footnote{余熱で火が通るため。}。
\end{enumerate}

\hypertarget{ux539fux6ce8}{%
\subparagraph{【原注】}\label{ux539fux6ce8}}

いくつかの魚種の加熱時間は該当する項で示してある。

\hypertarget{ux9b5aux306eux8abfux7406ux6cd5}{%
\section{魚の調理法}\label{ux9b5aux306eux8abfux7406ux6cd5}}

魚料理は全て、下記のいずれかの調理法による。

\begin{enumerate}
\def\labelenumi{\arabic{enumi}.}
\item
  塩水(湯)またはクールブイヨン\footnote{court-bouillon直訳は「量の少ない煮汁」。魚の他、甲殻類、鶏などの白身肉、野菜などをポシェするのに用いる。とりわけ魚や鶏を丸ごとポシェする場合には、その名称のとおり、できるだけ少量でポシェする必要がある。また、ポシェに用いたクールブイヨンをベースにソースを作る場合が多い。}Bを用いたポシェ\ldots{}\ldots{}大きな魚丸ごと、および切り身。
\item
  ごく少量のクールブイヨンを用いたポシェ\ldots{}\ldots{}魚のフィレ、またはやや小さい魚。
\item
  ブレゼ\ldots{}\ldots{}もっぱら大きな魚。
\item
  オ・ブルー\footnote{比較的小さめの淡水魚に主として用いられる調理法。生きたままの魚の表面のぬめりをとらないように洗い、内臓を取り除いたらすぐに、塩とヴィネガーを加えたクールブイヨンで茹でる。冷製、温製どちらでも供する。原書p.281参照。}\ldots{}\ldots{}とりわけ\ruby{鱒}{ます}、鯉、ブロシェ\footnote{川かますの一種。本連載「ブロシェのクネル」2011年10月号
    pp.124-125 参照。}に合う。
\item
  揚げもの\ldots{}\ldots{}もっぱら小さい魚、切り身。
\item
  ムニエール\ldots{}\ldots{}揚げものにするのと同じ小さい魚、切り身。
\item
  グリエ\ldots{}\ldots{}小さい魚、および切り身。
\item
  グラタン\ldots{}\ldots{}小さい魚、切り身。
\end{enumerate}

\hypertarget{ux5869ux6c34ux6e6fux304aux3088ux3073ux30afux30fcux30ebux30d6ux30a4ux30e8ux30f3bux3092ux7528ux3044ux305fux52a0ux71b1ux8abfux7406}{%
\subsection{塩水(湯)およびクールブイヨンBを用いた加熱調理}\label{ux5869ux6c34ux6e6fux304aux3088ux3073ux30afux30fcux30ebux30d6ux30a4ux30e8ux30f3bux3092ux7528ux3044ux305fux52a0ux71b1ux8abfux7406}}

魚を丸ごと調理する場合は、魚に合ったポワソニエール\footnote{大きな魚を丸ごと煮るための細長い鍋。魚の形を崩さずに取り出せるよう、中に専用の網を敷いて使う。似たものに、舌びらめ等の平たい魚にぴったり合う菱形をしたテュルボティエールがある。いずれも、できるだけ少量の煮汁で魚を加熱できるように工夫されたもの。(図参照)}を用いる。魚を掃除し(テュルボは水にさらして血抜きをし)、ひれ等を切り落して形を整え、ポワソニエールの網に乗せる。魚種に応じて塩水または冷たいクールブイヨンをかぶるまで注ぐ。強火にかけて沸騰したらすぐにレンジの火の弱いところに鍋を移動させ、ポシェする。

切り身(薄すぎは絶対にいけない)の場合、沸騰した液体(塩湯またはクールブイヨン)に投入したらすぐにレンジの火の弱いところに鍋を移動させ、沸騰しない程度の温度でゆっくりと火を通す。

こうするのは、魚の身のエキスを閉じこめるためである。冷水から火にかけた場合にはエキスの大部分が流れ出してしまう。大きな魚丸ごとの場合にはこのやり方はしない。沸騰した液体に魚を投入すると身が収縮するので、大きな魚の場合は身が割れたり形が崩れたりするからだ。

塩湯あるいはクールブイヨンでポシェした魚は、ナフキンまたは専用の網に盛る。周囲をパセリで飾り、塩茹でしたじゃがいもと1種類または数種のソースを添えて供する。ガルニテュールがパセリのみの場合、魚の周囲にパセリを飾るのは客に料理を見せる\footnote{当時の宴席で主流だったロシア式サーヴィスでは、大きな銀盆に盛った料理をまず食客に見せてから、とり分けて給仕する。}直前にすること。どんな場合でも、ガルニテュールを添えたらクロッシュ\footnote{銀または陶製の保温用皿カバー。}は被せないこと。

\hypertarget{ux3054ux304fux5c11ux91cfux306eux30afux30fcux30ebux30d6ux30a4ux30e8ux30f3ux3092ux7528ux3044ux305fux30ddux30b7ux30a7-ux539fux66f8-pp.279-280}{%
\subsection{ごく少量のクールブイヨンを用いたポシェ 原書
pp.279-280}\label{ux3054ux304fux5c11ux91cfux306eux30afux30fcux30ebux30d6ux30a4ux30e8ux30f3ux3092ux7528ux3044ux305fux30ddux30b7ux30a7-ux539fux66f8-pp.279-280}}

この火入れの方法は主としてテュルボタン、バルビュ、舌びらめ、丸ごとおよびそれぞれの魚のフィレで用いる。バターを塗った天板あるいはソテ鍋に魚丸ごとあるいはそのフィレを置き、軽く塩をして、所要量の魚のフュメかマッシュルームの煮汁を注ぐ。フュメとマッシュルームの煮汁を合わせたものを用いる場合もある。蓋をして、中温のオーヴンに入れる。魚丸ごとの場合は時折煮汁をかけてやる。

魚(丸ごとでもフィレでも)に火が通ったら、注意して汁気をきり、皿に盛る。ガルニテュールを含む料理の場合、ガルニテュールを魚の周囲に盛り、ソースをかける\footnote{原文通りの順で訳したが、実際にはソースをかけてからガルニテュールを盛ったほうが良い場合もあるだろう。}。多くの場合、魚の煮汁を煮詰めてソースに加える。

\href{欠落アリ}{}

\hypertarget{ux9b5aux306eux30d6ux30ecux30bc17-ux539fux66f8-p.280}{%
\subsection[魚のブレゼ 原書 p.280]{\texorpdfstring{魚のブレゼ\footnote{ここではブレゼの語が限定的な意味で用いられていることに注意。牛のアロワイヨのような大きな塊肉のブレゼと同様の調理法、ということである。それは、香味素材を色づくまで炒めてから用いることや、主素材に豚背脂やトリュフをピケ針で差したり、豚背脂のシートで覆って加熱するという点によく表れている。ただし、これらは必須というわけではないため、事実上は「ごく少量のクールブイヨンを用いたポシェ」と区別がつきにくい。実際、モンタニェ『ラルース・ガストロノミーク』初版では、魚のブレゼについて「本来的な意味でのブレゼというよりは、ごく少量のクールブイヨンを用いたポシェである」と述べられている。逆に言えば、こんにち魚の加熱方法についてしばしば「ブレゼ」と呼ばれているものが、エスコフィエやモンタニェにおいては「少量のクールブイヨンを用いたポシェ」と表現されていたということである。}
原書
p.280}{魚のブレゼ 原書 p.280}}\label{ux9b5aux306eux30d6ux30ecux30bc17-ux539fux66f8-p.280}}

この調理法を用いるのは通常、丸ごとまたは筒切りにした鮭、大ぶりの鱒、テュルボ、テュルボタンのうち大きなもの、である。

場合によっては、魚の片面に、小さく切った豚背脂、トリュフ、コルニション、にんじん等をピケ針で差し込む。

香味素材等\footnote{原文 fonds de braisage フォン・ド・ブレザージュ
  (fonds de
  braiseフォン・ド・ブレーズ、とも)。通常は、厚い輪切りにしたにんじんと玉ねぎをバターか獣脂で色づくまで炒め、ブーケガルニ、下茹でした豚皮を合わせる(原書pp.394-395)。また、これを用いた煮汁のことも指す。}は肉料理のブレゼの場合と同じように用意するが、豚皮は用いない。提供方法に応じて、白または赤ワインと軽い魚のフュメ同量ずつを、魚の厚みの
\(\frac{3}{4}\) またはひたひたの高さまで注ぐ。厳密に肉断ち \footnote{カトリックの生活習慣として、四旬節(復活祭までの46日間)および週1
  回程度、肉類を食べないということが行なわれた(本連載2012年5月号「ソース・エスパニョル(4)」p.110、訳注1参照)。}のための仕立てにする場合を除いて、薄くスライスした豚背脂のシートを魚にかぶせる。加熱中\footnote{鍋を火にかけ、沸騰したら蓋をして中火のオーヴンに入れ、加熱する。}こまめに煮汁を魚にかけてやる\footnote{arroser
  アロゼ。}。また、完全には蓋をせず、加熱中に煮汁が煮詰まるようにしてやる。

ほぼ火が通ったら、鍋の蓋をとり、魚にかけた煮汁の水分をオーヴンの熱で蒸発させて表面につやを出す\footnote{glacer
  グラセ。}。魚を鍋から出して汁気をきり、皿に盛り保温しておく。

煮汁\footnote{原文 fonds de braisage (訳注2参照)。}を漉し、しばらく休ませたら浮き脂を取り除き、必要なら煮詰める。これを加えてソースを仕上げる。

魚のブレゼには通常、各ルセットに示してあるガルニテュールを添える。

\hypertarget{ux30aaux30d6ux30ebux30fc36-ux539fux66f8-pp.280-281}{%
\subsection[オ・ブルー 原書
pp.280-281]{\texorpdfstring{オ・ブルー\footnote{au bleu
  ヴィネガーを加えることで魚の表面のぬめりが青みがかることから。} 原書
pp.280-281}{オ・ブルー 原書 pp.280-281}}\label{ux30aaux30d6ux30ebux30fc36-ux539fux66f8-pp.280-281}}

オ・ブルーは鱒、鯉、ブロシェ\footnote{川かますの一種。}のみに用いられる特殊な調理法で、基本的なポイントは以下のとおり。

\begin{enumerate}
\def\labelenumi{\arabic{enumi}.}
\item
  必ず、生きた魚を使う。
\item
  魚の表面のぬめりをとらないように、なるべく手で触れずに、わたを抜く。鱗も引かない。
\item
  魚が大きい場合は、専用の網を敷いたポワソニーエルに入れ、「沸騰したヴィネガーをかける」。ヴィネガーは、通常のクールブイヨンに加える分量\footnote{以下の「クールブイヨンA」の分量比率を参照。}。次に、ヴィネガーを入れずに用意した温かい\footnote{原文
    tiède ぬるい、温かい。}クールブイヨンを注ぎ入れる。これは、なるべく身が割れないようにするためである。その後は通常どおり加熱する\footnote{レンジで沸騰させたらオーヴンに入れる。}。
\item
  小さい鱒の場合は、生きたままのものを手早く中抜きし、塩、ヴィネガーを加えただけの沸騰したクールブイヨンで煮る。
\item
  オ・ブルーは冷製、温製どちらの仕立てにしてもいい。実際の作り方の項で示してあるソースを添えて供する。
\end{enumerate}

\href{欠落アリ}{}

\hypertarget{ux30e0ux30cbux30a8ux30fcux30eb43-ux539fux66f8-p.282}{%
\subsection[ムニエール 原書 p.282]{\texorpdfstring{ムニエール\footnote{à
  la meunière 「粉挽き職人風」の意。} 原書
p.282}{ムニエール 原書 p.282}}\label{ux30e0ux30cbux30a8ux30fcux30eb43-ux539fux66f8-p.282}}

ムニエールは素晴らしい調理法だが、小型の魚と、大きな魚の場合は切り身にしか用いない。とはいえ、丁寧にやれば1.5
kg以下のテュルボタンはムニエールで調理できる。

魚丸ごと、あるいは切り身、フィレに味つけをして小麦粉をまぶし、バターを熱したフライパンで焼く。

魚が小さい場合は普通のバターでいいが、大きい場合は澄ましバターを使った方がいい。

魚の両面を焼き、程良く火が通ったら、予め熱しておいた皿に盛る。

飾り切りにした半割りのレモンを添えて、そのまま供することも可能である。ただし、このような提供方法の場合は本来の「ムニエール」と区別するために「黄金色に焼いた\footnote{doré
  (ドレ)
  一般的な色の表現として「黄金色」の意だが、ムニエールの場合、通常は大きい魚についてのみこの表現を用いる。}」と表現する。

「ムニエール」の場合には、焼き上がった魚に少量のレモン汁をふり、塩、こしょう少々で味を整える。粗みじん切りにして湯通ししたパセリを魚の表面に散らし、焦がしバターをかけてすぐに供する。湯通ししたパセリの水分に熱いバターが触れて泡がたつので、それが消えないうちに客に料理を見せるようにする。

%\href{✓原稿下準備なし}{} \href{訳と注釈\%2020180420進行中}{}
\href{未、原文対照チェック}{} \href{未、日本語表現校正}{}
\href{未、注釈チェク}{} \href{未、原稿最終校正}{}

\hypertarget{poissons}{%
\chapter{VI 魚料理 Poissons}\label{poissons}}

\hypertarget{serie-de-courts-bouillons-de-poisson}{%
\section{クールブイヨン}\label{serie-de-courts-bouillons-de-poisson}}

\begin{recette}
\hypertarget{ux30afux30fcux30ebux30d6ux30a4ux30e8ux30f3-a}{%
\subsubsection{クールブイヨン
A}\label{ux30afux30fcux30ebux30d6ux30a4ux30e8ux30f3-a}}

水5 L に対し、ヴィネガー2.5 dL、粗塩60 g、薄切りにしたにんじん600
gと玉ねぎ500 g、タイム1枝、ローリエの小さな葉2枚、パセリの茎100
g、粒こしょう20
g(こしょうを加えるのはクールブイヨンを漉す10分前)。材料を全て鍋に入れ、火にかけて1時間弱火で煮、漉す(原書
p.277)。

\hypertarget{ux30afux30fcux30ebux30d6ux30a4ux30e8ux30f3-b-4-ux9c52ux3046ux306aux304eux30d6ux30edux30b7ux30a7ux7b49-ux539fux66f8p.277}{%
\subsubsection[クールブイヨン B (鱒、うなぎ、ブロシェ等)
原書p.277]{\texorpdfstring{クールブイヨン B \footnote{クールブイヨンは用途に応じ、AからEまでの5種が挙げられている(原書pp.277-278)。}
(鱒、うなぎ、ブロシェ等)
原書p.277}{クールブイヨン B  (鱒、うなぎ、ブロシェ等) 原書p.277}}\label{ux30afux30fcux30ebux30d6ux30a4ux30e8ux30f3-b-4-ux9c52ux3046ux306aux304eux30d6ux30edux30b7ux30a7ux7b49-ux539fux66f8p.277}}

5 L 分の材料\ldots{}\ldots{}白ワイン2.5 L 。水2.5 L
。薄切りにした玉ねぎ600 g。パセリの茎80g
。タイムの小枝1本。ローリエの葉(小) \(\frac{1}{2}\) 枚。粗塩
60g。大粒のこしょう15 g(クールブイヨンを漉す10分前に加える)。

作業手順\ldots{}\ldots{}作業:液体、香味素材、調味料を鍋に入れ、沸かす。弱火で30分程煮て、漉す。

\href{欠落アリ}{}

原注:クールブイヨンBとC\footnote{クールブイヨンBの白ワインを赤ワインに代え、香味素材としてにんじん400gを加える。鱒、鯉、マトロート用(原書pp.277-278)。}で調理した魚はクールブイヨン添えとして供する。つまり、少量の煮汁とクールブイヨンに用いた野菜を添える。野菜はよく火が通っていること。煮汁はしっかり煮詰め、提供直前に新鮮なバター少量を加えて仕上げる。
\end{recette}
\hypertarget{ux30afux30fcux30ebux30d6ux30a4ux30e8ux30f3ux306eux4f7fux3044ux65b9-ux539fux66f8-p.278}{%
\subsection{クールブイヨンの使い方 原書
p.278}\label{ux30afux30fcux30ebux30d6ux30a4ux30e8ux30f3ux306eux4f7fux3044ux65b9-ux539fux66f8-p.278}}

\begin{enumerate}
\def\labelenumi{\arabic{enumi}.}
\item
  加熱時間が30分以内の場合は、クールブイヨンは必ず事前に用意しておくこと。
\item
  加熱時間が30分を越える場合は、クールブイヨンの材料は冷たい状態のままで合わせせておく。香味素材はポワソニエールの網の下に入れる。
\item
  ごく少量のクールブイヨンでポシェ\footnote{原文 pochage à court
    mouillement『ル・ギード・キュリネール』では、この表現はテュルボタン(小型のテュルボ)、バルビュ、舌びらめ等の平たい魚をポシェする際に用いられる。本連載「舌びらめのボヌ・ファム」
    2011年3月号pp.110-111 参照。}する場合、材料は(白または赤ワインを含む場合も)魚を火にかける際に合わせる。クールブイヨンの量は魚の
  \(\frac{1}{3}\)
  の高さとし、加熱中ひんぱんに煮汁を魚にかけてやること\footnote{arroser
    アロゼ。}。この調理法の場合は通常、クールブイヨンは上で記したように
  \footnote{「クールブイヨンB」原注。}、提供直前に軽くバターを加えて仕上げ、魚に添える。
\item
  冷製にする場合は、必ずクールブイヨンに魚が浸った状態で冷ますこと。当然ながら、火にかけている時間は短かくなる\footnote{余熱で火が通るため。}。
\end{enumerate}

\hypertarget{ux539fux6ce8}{%
\subparagraph{【原注】}\label{ux539fux6ce8}}

いくつかの魚種の加熱時間は該当する項で示してある。

\hypertarget{ux9b5aux306eux8abfux7406ux6cd5}{%
\section{魚の調理法}\label{ux9b5aux306eux8abfux7406ux6cd5}}

魚料理は全て、下記のいずれかの調理法による。

\begin{enumerate}
\def\labelenumi{\arabic{enumi}.}
\item
  塩水(湯)またはクールブイヨン\footnote{court-bouillon直訳は「量の少ない煮汁」。魚の他、甲殻類、鶏などの白身肉、野菜などをポシェするのに用いる。とりわけ魚や鶏を丸ごとポシェする場合には、その名称のとおり、できるだけ少量でポシェする必要がある。また、ポシェに用いたクールブイヨンをベースにソースを作る場合が多い。}Bを用いたポシェ\ldots{}\ldots{}大きな魚丸ごと、および切り身。
\item
  ごく少量のクールブイヨンを用いたポシェ\ldots{}\ldots{}魚のフィレ、またはやや小さい魚。
\item
  ブレゼ\ldots{}\ldots{}もっぱら大きな魚。
\item
  オ・ブルー\footnote{比較的小さめの淡水魚に主として用いられる調理法。生きたままの魚の表面のぬめりをとらないように洗い、内臓を取り除いたらすぐに、塩とヴィネガーを加えたクールブイヨンで茹でる。冷製、温製どちらでも供する。原書p.281参照。}\ldots{}\ldots{}とりわけ\ruby{鱒}{ます}、鯉、ブロシェ\footnote{川かますの一種。本連載「ブロシェのクネル」2011年10月号
    pp.124-125 参照。}に合う。
\item
  揚げもの\ldots{}\ldots{}もっぱら小さい魚、切り身。
\item
  ムニエール\ldots{}\ldots{}揚げものにするのと同じ小さい魚、切り身。
\item
  グリエ\ldots{}\ldots{}小さい魚、および切り身。
\item
  グラタン\ldots{}\ldots{}小さい魚、切り身。
\end{enumerate}

\hypertarget{ux5869ux6c34ux6e6fux304aux3088ux3073ux30afux30fcux30ebux30d6ux30a4ux30e8ux30f3bux3092ux7528ux3044ux305fux52a0ux71b1ux8abfux7406}{%
\subsection{塩水(湯)およびクールブイヨンBを用いた加熱調理}\label{ux5869ux6c34ux6e6fux304aux3088ux3073ux30afux30fcux30ebux30d6ux30a4ux30e8ux30f3bux3092ux7528ux3044ux305fux52a0ux71b1ux8abfux7406}}

魚を丸ごと調理する場合は、魚に合ったポワソニエール\footnote{大きな魚を丸ごと煮るための細長い鍋。魚の形を崩さずに取り出せるよう、中に専用の網を敷いて使う。似たものに、舌びらめ等の平たい魚にぴったり合う菱形をしたテュルボティエールがある。いずれも、できるだけ少量の煮汁で魚を加熱できるように工夫されたもの。(図参照)}を用いる。魚を掃除し(テュルボは水にさらして血抜きをし)、ひれ等を切り落して形を整え、ポワソニエールの網に乗せる。魚種に応じて塩水または冷たいクールブイヨンをかぶるまで注ぐ。強火にかけて沸騰したらすぐにレンジの火の弱いところに鍋を移動させ、ポシェする。

切り身(薄すぎは絶対にいけない)の場合、沸騰した液体(塩湯またはクールブイヨン)に投入したらすぐにレンジの火の弱いところに鍋を移動させ、沸騰しない程度の温度でゆっくりと火を通す。

こうするのは、魚の身のエキスを閉じこめるためである。冷水から火にかけた場合にはエキスの大部分が流れ出してしまう。大きな魚丸ごとの場合にはこのやり方はしない。沸騰した液体に魚を投入すると身が収縮するので、大きな魚の場合は身が割れたり形が崩れたりするからだ。

塩湯あるいはクールブイヨンでポシェした魚は、ナフキンまたは専用の網に盛る。周囲をパセリで飾り、塩茹でしたじゃがいもと1種類または数種のソースを添えて供する。ガルニテュールがパセリのみの場合、魚の周囲にパセリを飾るのは客に料理を見せる\footnote{当時の宴席で主流だったロシア式サーヴィスでは、大きな銀盆に盛った料理をまず食客に見せてから、とり分けて給仕する。}直前にすること。どんな場合でも、ガルニテュールを添えたらクロッシュ\footnote{銀または陶製の保温用皿カバー。}は被せないこと。

\hypertarget{ux3054ux304fux5c11ux91cfux306eux30afux30fcux30ebux30d6ux30a4ux30e8ux30f3ux3092ux7528ux3044ux305fux30ddux30b7ux30a7-ux539fux66f8-pp.279-280}{%
\subsection{ごく少量のクールブイヨンを用いたポシェ 原書
pp.279-280}\label{ux3054ux304fux5c11ux91cfux306eux30afux30fcux30ebux30d6ux30a4ux30e8ux30f3ux3092ux7528ux3044ux305fux30ddux30b7ux30a7-ux539fux66f8-pp.279-280}}

この火入れの方法は主としてテュルボタン、バルビュ、舌びらめ、丸ごとおよびそれぞれの魚のフィレで用いる。バターを塗った天板あるいはソテ鍋に魚丸ごとあるいはそのフィレを置き、軽く塩をして、所要量の魚のフュメかマッシュルームの煮汁を注ぐ。フュメとマッシュルームの煮汁を合わせたものを用いる場合もある。蓋をして、中温のオーヴンに入れる。魚丸ごとの場合は時折煮汁をかけてやる。

魚(丸ごとでもフィレでも)に火が通ったら、注意して汁気をきり、皿に盛る。ガルニテュールを含む料理の場合、ガルニテュールを魚の周囲に盛り、ソースをかける\footnote{原文通りの順で訳したが、実際にはソースをかけてからガルニテュールを盛ったほうが良い場合もあるだろう。}。多くの場合、魚の煮汁を煮詰めてソースに加える。

\href{欠落アリ}{}

\hypertarget{ux9b5aux306eux30d6ux30ecux30bc17-ux539fux66f8-p.280}{%
\subsection[魚のブレゼ 原書 p.280]{\texorpdfstring{魚のブレゼ\footnote{ここではブレゼの語が限定的な意味で用いられていることに注意。牛のアロワイヨのような大きな塊肉のブレゼと同様の調理法、ということである。それは、香味素材を色づくまで炒めてから用いることや、主素材に豚背脂やトリュフをピケ針で差したり、豚背脂のシートで覆って加熱するという点によく表れている。ただし、これらは必須というわけではないため、事実上は「ごく少量のクールブイヨンを用いたポシェ」と区別がつきにくい。実際、モンタニェ『ラルース・ガストロノミーク』初版では、魚のブレゼについて「本来的な意味でのブレゼというよりは、ごく少量のクールブイヨンを用いたポシェである」と述べられている。逆に言えば、こんにち魚の加熱方法についてしばしば「ブレゼ」と呼ばれているものが、エスコフィエやモンタニェにおいては「少量のクールブイヨンを用いたポシェ」と表現されていたということである。}
原書
p.280}{魚のブレゼ 原書 p.280}}\label{ux9b5aux306eux30d6ux30ecux30bc17-ux539fux66f8-p.280}}

この調理法を用いるのは通常、丸ごとまたは筒切りにした鮭、大ぶりの鱒、テュルボ、テュルボタンのうち大きなもの、である。

場合によっては、魚の片面に、小さく切った豚背脂、トリュフ、コルニション、にんじん等をピケ針で差し込む。

香味素材等\footnote{原文 fonds de braisage フォン・ド・ブレザージュ
  (fonds de
  braiseフォン・ド・ブレーズ、とも)。通常は、厚い輪切りにしたにんじんと玉ねぎをバターか獣脂で色づくまで炒め、ブーケガルニ、下茹でした豚皮を合わせる(原書pp.394-395)。また、これを用いた煮汁のことも指す。}は肉料理のブレゼの場合と同じように用意するが、豚皮は用いない。提供方法に応じて、白または赤ワインと軽い魚のフュメ同量ずつを、魚の厚みの
\(\frac{3}{4}\) またはひたひたの高さまで注ぐ。厳密に肉断ち \footnote{カトリックの生活習慣として、四旬節(復活祭までの46日間)および週1
  回程度、肉類を食べないということが行なわれた(本連載2012年5月号「ソース・エスパニョル(4)」p.110、訳注1参照)。}のための仕立てにする場合を除いて、薄くスライスした豚背脂のシートを魚にかぶせる。加熱中\footnote{鍋を火にかけ、沸騰したら蓋をして中火のオーヴンに入れ、加熱する。}こまめに煮汁を魚にかけてやる\footnote{arroser
  アロゼ。}。また、完全には蓋をせず、加熱中に煮汁が煮詰まるようにしてやる。

ほぼ火が通ったら、鍋の蓋をとり、魚にかけた煮汁の水分をオーヴンの熱で蒸発させて表面につやを出す\footnote{glacer
  グラセ。}。魚を鍋から出して汁気をきり、皿に盛り保温しておく。

煮汁\footnote{原文 fonds de braisage (訳注2参照)。}を漉し、しばらく休ませたら浮き脂を取り除き、必要なら煮詰める。これを加えてソースを仕上げる。

魚のブレゼには通常、各ルセットに示してあるガルニテュールを添える。

\hypertarget{ux30aaux30d6ux30ebux30fc36-ux539fux66f8-pp.280-281}{%
\subsection[オ・ブルー 原書
pp.280-281]{\texorpdfstring{オ・ブルー\footnote{au bleu
  ヴィネガーを加えることで魚の表面のぬめりが青みがかることから。} 原書
pp.280-281}{オ・ブルー 原書 pp.280-281}}\label{ux30aaux30d6ux30ebux30fc36-ux539fux66f8-pp.280-281}}

オ・ブルーは鱒、鯉、ブロシェ\footnote{川かますの一種。}のみに用いられる特殊な調理法で、基本的なポイントは以下のとおり。

\begin{enumerate}
\def\labelenumi{\arabic{enumi}.}
\item
  必ず、生きた魚を使う。
\item
  魚の表面のぬめりをとらないように、なるべく手で触れずに、わたを抜く。鱗も引かない。
\item
  魚が大きい場合は、専用の網を敷いたポワソニーエルに入れ、「沸騰したヴィネガーをかける」。ヴィネガーは、通常のクールブイヨンに加える分量\footnote{以下の「クールブイヨンA」の分量比率を参照。}。次に、ヴィネガーを入れずに用意した温かい\footnote{原文
    tiède ぬるい、温かい。}クールブイヨンを注ぎ入れる。これは、なるべく身が割れないようにするためである。その後は通常どおり加熱する\footnote{レンジで沸騰させたらオーヴンに入れる。}。
\item
  小さい鱒の場合は、生きたままのものを手早く中抜きし、塩、ヴィネガーを加えただけの沸騰したクールブイヨンで煮る。
\item
  オ・ブルーは冷製、温製どちらの仕立てにしてもいい。実際の作り方の項で示してあるソースを添えて供する。
\end{enumerate}

\href{欠落アリ}{}

\hypertarget{ux30e0ux30cbux30a8ux30fcux30eb43-ux539fux66f8-p.282}{%
\subsection[ムニエール 原書 p.282]{\texorpdfstring{ムニエール\footnote{à
  la meunière 「粉挽き職人風」の意。} 原書
p.282}{ムニエール 原書 p.282}}\label{ux30e0ux30cbux30a8ux30fcux30eb43-ux539fux66f8-p.282}}

ムニエールは素晴らしい調理法だが、小型の魚と、大きな魚の場合は切り身にしか用いない。とはいえ、丁寧にやれば1.5
kg以下のテュルボタンはムニエールで調理できる。

魚丸ごと、あるいは切り身、フィレに味つけをして小麦粉をまぶし、バターを熱したフライパンで焼く。

魚が小さい場合は普通のバターでいいが、大きい場合は澄ましバターを使った方がいい。

魚の両面を焼き、程良く火が通ったら、予め熱しておいた皿に盛る。

飾り切りにした半割りのレモンを添えて、そのまま供することも可能である。ただし、このような提供方法の場合は本来の「ムニエール」と区別するために「黄金色に焼いた\footnote{doré
  (ドレ)
  一般的な色の表現として「黄金色」の意だが、ムニエールの場合、通常は大きい魚についてのみこの表現を用いる。}」と表現する。

「ムニエール」の場合には、焼き上がった魚に少量のレモン汁をふり、塩、こしょう少々で味を整える。粗みじん切りにして湯通ししたパセリを魚の表面に散らし、焦がしバターをかけてすぐに供する。湯通ししたパセリの水分に熱いバターが触れて泡がたつので、それが消えないうちに客に料理を見せるようにする。




%%% chapitre vi. poissons
%% エスコフィエ『料理の手引き』全注解
% 五島 学

\href{✓原稿下準備なし}{} \href{訳と注釈\%2020180420進行中}{}
\href{未、原文対照チェック}{} \href{未、日本語表現校正}{}
\href{未、注釈チェク}{} \href{未、原稿最終校正}{}

\hypertarget{poissons}{%
\chapter{VI 魚料理 Poissons}\label{poissons}}

\hypertarget{serie-de-courts-bouillons-de-poisson}{%
\section{クールブイヨン}\label{serie-de-courts-bouillons-de-poisson}}

\begin{recette}
\hypertarget{ux30afux30fcux30ebux30d6ux30a4ux30e8ux30f3-a}{%
\subsubsection{クールブイヨン
A}\label{ux30afux30fcux30ebux30d6ux30a4ux30e8ux30f3-a}}

水5 L に対し、ヴィネガー2.5 dL、粗塩60 g、薄切りにしたにんじん600
gと玉ねぎ500 g、タイム1枝、ローリエの小さな葉2枚、パセリの茎100
g、粒こしょう20
g(こしょうを加えるのはクールブイヨンを漉す10分前)。材料を全て鍋に入れ、火にかけて1時間弱火で煮、漉す(原書
p.277)。

\hypertarget{ux30afux30fcux30ebux30d6ux30a4ux30e8ux30f3-b-4-ux9c52ux3046ux306aux304eux30d6ux30edux30b7ux30a7ux7b49-ux539fux66f8p.277}{%
\subsubsection[クールブイヨン B (鱒、うなぎ、ブロシェ等)
原書p.277]{\texorpdfstring{クールブイヨン B \footnote{クールブイヨンは用途に応じ、AからEまでの5種が挙げられている(原書pp.277-278)。}
(鱒、うなぎ、ブロシェ等)
原書p.277}{クールブイヨン B  (鱒、うなぎ、ブロシェ等) 原書p.277}}\label{ux30afux30fcux30ebux30d6ux30a4ux30e8ux30f3-b-4-ux9c52ux3046ux306aux304eux30d6ux30edux30b7ux30a7ux7b49-ux539fux66f8p.277}}

5 L 分の材料\ldots{}\ldots{}白ワイン2.5 L 。水2.5 L
。薄切りにした玉ねぎ600 g。パセリの茎80g
。タイムの小枝1本。ローリエの葉(小) \(\frac{1}{2}\) 枚。粗塩
60g。大粒のこしょう15 g(クールブイヨンを漉す10分前に加える)。

作業手順\ldots{}\ldots{}作業:液体、香味素材、調味料を鍋に入れ、沸かす。弱火で30分程煮て、漉す。

\href{欠落アリ}{}

原注:クールブイヨンBとC\footnote{クールブイヨンBの白ワインを赤ワインに代え、香味素材としてにんじん400gを加える。鱒、鯉、マトロート用(原書pp.277-278)。}で調理した魚はクールブイヨン添えとして供する。つまり、少量の煮汁とクールブイヨンに用いた野菜を添える。野菜はよく火が通っていること。煮汁はしっかり煮詰め、提供直前に新鮮なバター少量を加えて仕上げる。
\end{recette}
\hypertarget{ux30afux30fcux30ebux30d6ux30a4ux30e8ux30f3ux306eux4f7fux3044ux65b9-ux539fux66f8-p.278}{%
\subsection{クールブイヨンの使い方 原書
p.278}\label{ux30afux30fcux30ebux30d6ux30a4ux30e8ux30f3ux306eux4f7fux3044ux65b9-ux539fux66f8-p.278}}

\begin{enumerate}
\def\labelenumi{\arabic{enumi}.}
\item
  加熱時間が30分以内の場合は、クールブイヨンは必ず事前に用意しておくこと。
\item
  加熱時間が30分を越える場合は、クールブイヨンの材料は冷たい状態のままで合わせせておく。香味素材はポワソニエールの網の下に入れる。
\item
  ごく少量のクールブイヨンでポシェ\footnote{原文 pochage à court
    mouillement『ル・ギード・キュリネール』では、この表現はテュルボタン(小型のテュルボ)、バルビュ、舌びらめ等の平たい魚をポシェする際に用いられる。本連載「舌びらめのボヌ・ファム」
    2011年3月号pp.110-111 参照。}する場合、材料は(白または赤ワインを含む場合も)魚を火にかける際に合わせる。クールブイヨンの量は魚の
  \(\frac{1}{3}\)
  の高さとし、加熱中ひんぱんに煮汁を魚にかけてやること\footnote{arroser
    アロゼ。}。この調理法の場合は通常、クールブイヨンは上で記したように
  \footnote{「クールブイヨンB」原注。}、提供直前に軽くバターを加えて仕上げ、魚に添える。
\item
  冷製にする場合は、必ずクールブイヨンに魚が浸った状態で冷ますこと。当然ながら、火にかけている時間は短かくなる\footnote{余熱で火が通るため。}。
\end{enumerate}

\hypertarget{ux539fux6ce8}{%
\subparagraph{【原注】}\label{ux539fux6ce8}}

いくつかの魚種の加熱時間は該当する項で示してある。

\hypertarget{ux9b5aux306eux8abfux7406ux6cd5}{%
\section{魚の調理法}\label{ux9b5aux306eux8abfux7406ux6cd5}}

魚料理は全て、下記のいずれかの調理法による。

\begin{enumerate}
\def\labelenumi{\arabic{enumi}.}
\item
  塩水(湯)またはクールブイヨン\footnote{court-bouillon直訳は「量の少ない煮汁」。魚の他、甲殻類、鶏などの白身肉、野菜などをポシェするのに用いる。とりわけ魚や鶏を丸ごとポシェする場合には、その名称のとおり、できるだけ少量でポシェする必要がある。また、ポシェに用いたクールブイヨンをベースにソースを作る場合が多い。}Bを用いたポシェ\ldots{}\ldots{}大きな魚丸ごと、および切り身。
\item
  ごく少量のクールブイヨンを用いたポシェ\ldots{}\ldots{}魚のフィレ、またはやや小さい魚。
\item
  ブレゼ\ldots{}\ldots{}もっぱら大きな魚。
\item
  オ・ブルー\footnote{比較的小さめの淡水魚に主として用いられる調理法。生きたままの魚の表面のぬめりをとらないように洗い、内臓を取り除いたらすぐに、塩とヴィネガーを加えたクールブイヨンで茹でる。冷製、温製どちらでも供する。原書p.281参照。}\ldots{}\ldots{}とりわけ\ruby{鱒}{ます}、鯉、ブロシェ\footnote{川かますの一種。本連載「ブロシェのクネル」2011年10月号
    pp.124-125 参照。}に合う。
\item
  揚げもの\ldots{}\ldots{}もっぱら小さい魚、切り身。
\item
  ムニエール\ldots{}\ldots{}揚げものにするのと同じ小さい魚、切り身。
\item
  グリエ\ldots{}\ldots{}小さい魚、および切り身。
\item
  グラタン\ldots{}\ldots{}小さい魚、切り身。
\end{enumerate}

\hypertarget{ux5869ux6c34ux6e6fux304aux3088ux3073ux30afux30fcux30ebux30d6ux30a4ux30e8ux30f3bux3092ux7528ux3044ux305fux52a0ux71b1ux8abfux7406}{%
\subsection{塩水(湯)およびクールブイヨンBを用いた加熱調理}\label{ux5869ux6c34ux6e6fux304aux3088ux3073ux30afux30fcux30ebux30d6ux30a4ux30e8ux30f3bux3092ux7528ux3044ux305fux52a0ux71b1ux8abfux7406}}

魚を丸ごと調理する場合は、魚に合ったポワソニエール\footnote{大きな魚を丸ごと煮るための細長い鍋。魚の形を崩さずに取り出せるよう、中に専用の網を敷いて使う。似たものに、舌びらめ等の平たい魚にぴったり合う菱形をしたテュルボティエールがある。いずれも、できるだけ少量の煮汁で魚を加熱できるように工夫されたもの。(図参照)}を用いる。魚を掃除し(テュルボは水にさらして血抜きをし)、ひれ等を切り落して形を整え、ポワソニエールの網に乗せる。魚種に応じて塩水または冷たいクールブイヨンをかぶるまで注ぐ。強火にかけて沸騰したらすぐにレンジの火の弱いところに鍋を移動させ、ポシェする。

切り身(薄すぎは絶対にいけない)の場合、沸騰した液体(塩湯またはクールブイヨン)に投入したらすぐにレンジの火の弱いところに鍋を移動させ、沸騰しない程度の温度でゆっくりと火を通す。

こうするのは、魚の身のエキスを閉じこめるためである。冷水から火にかけた場合にはエキスの大部分が流れ出してしまう。大きな魚丸ごとの場合にはこのやり方はしない。沸騰した液体に魚を投入すると身が収縮するので、大きな魚の場合は身が割れたり形が崩れたりするからだ。

塩湯あるいはクールブイヨンでポシェした魚は、ナフキンまたは専用の網に盛る。周囲をパセリで飾り、塩茹でしたじゃがいもと1種類または数種のソースを添えて供する。ガルニテュールがパセリのみの場合、魚の周囲にパセリを飾るのは客に料理を見せる\footnote{当時の宴席で主流だったロシア式サーヴィスでは、大きな銀盆に盛った料理をまず食客に見せてから、とり分けて給仕する。}直前にすること。どんな場合でも、ガルニテュールを添えたらクロッシュ\footnote{銀または陶製の保温用皿カバー。}は被せないこと。

\hypertarget{ux3054ux304fux5c11ux91cfux306eux30afux30fcux30ebux30d6ux30a4ux30e8ux30f3ux3092ux7528ux3044ux305fux30ddux30b7ux30a7-ux539fux66f8-pp.279-280}{%
\subsection{ごく少量のクールブイヨンを用いたポシェ 原書
pp.279-280}\label{ux3054ux304fux5c11ux91cfux306eux30afux30fcux30ebux30d6ux30a4ux30e8ux30f3ux3092ux7528ux3044ux305fux30ddux30b7ux30a7-ux539fux66f8-pp.279-280}}

この火入れの方法は主としてテュルボタン、バルビュ、舌びらめ、丸ごとおよびそれぞれの魚のフィレで用いる。バターを塗った天板あるいはソテ鍋に魚丸ごとあるいはそのフィレを置き、軽く塩をして、所要量の魚のフュメかマッシュルームの煮汁を注ぐ。フュメとマッシュルームの煮汁を合わせたものを用いる場合もある。蓋をして、中温のオーヴンに入れる。魚丸ごとの場合は時折煮汁をかけてやる。

魚(丸ごとでもフィレでも)に火が通ったら、注意して汁気をきり、皿に盛る。ガルニテュールを含む料理の場合、ガルニテュールを魚の周囲に盛り、ソースをかける\footnote{原文通りの順で訳したが、実際にはソースをかけてからガルニテュールを盛ったほうが良い場合もあるだろう。}。多くの場合、魚の煮汁を煮詰めてソースに加える。

\href{欠落アリ}{}

\hypertarget{ux9b5aux306eux30d6ux30ecux30bc17-ux539fux66f8-p.280}{%
\subsection[魚のブレゼ 原書 p.280]{\texorpdfstring{魚のブレゼ\footnote{ここではブレゼの語が限定的な意味で用いられていることに注意。牛のアロワイヨのような大きな塊肉のブレゼと同様の調理法、ということである。それは、香味素材を色づくまで炒めてから用いることや、主素材に豚背脂やトリュフをピケ針で差したり、豚背脂のシートで覆って加熱するという点によく表れている。ただし、これらは必須というわけではないため、事実上は「ごく少量のクールブイヨンを用いたポシェ」と区別がつきにくい。実際、モンタニェ『ラルース・ガストロノミーク』初版では、魚のブレゼについて「本来的な意味でのブレゼというよりは、ごく少量のクールブイヨンを用いたポシェである」と述べられている。逆に言えば、こんにち魚の加熱方法についてしばしば「ブレゼ」と呼ばれているものが、エスコフィエやモンタニェにおいては「少量のクールブイヨンを用いたポシェ」と表現されていたということである。}
原書
p.280}{魚のブレゼ 原書 p.280}}\label{ux9b5aux306eux30d6ux30ecux30bc17-ux539fux66f8-p.280}}

この調理法を用いるのは通常、丸ごとまたは筒切りにした鮭、大ぶりの鱒、テュルボ、テュルボタンのうち大きなもの、である。

場合によっては、魚の片面に、小さく切った豚背脂、トリュフ、コルニション、にんじん等をピケ針で差し込む。

香味素材等\footnote{原文 fonds de braisage フォン・ド・ブレザージュ
  (fonds de
  braiseフォン・ド・ブレーズ、とも)。通常は、厚い輪切りにしたにんじんと玉ねぎをバターか獣脂で色づくまで炒め、ブーケガルニ、下茹でした豚皮を合わせる(原書pp.394-395)。また、これを用いた煮汁のことも指す。}は肉料理のブレゼの場合と同じように用意するが、豚皮は用いない。提供方法に応じて、白または赤ワインと軽い魚のフュメ同量ずつを、魚の厚みの
\(\frac{3}{4}\) またはひたひたの高さまで注ぐ。厳密に肉断ち \footnote{カトリックの生活習慣として、四旬節(復活祭までの46日間)および週1
  回程度、肉類を食べないということが行なわれた(本連載2012年5月号「ソース・エスパニョル(4)」p.110、訳注1参照)。}のための仕立てにする場合を除いて、薄くスライスした豚背脂のシートを魚にかぶせる。加熱中\footnote{鍋を火にかけ、沸騰したら蓋をして中火のオーヴンに入れ、加熱する。}こまめに煮汁を魚にかけてやる\footnote{arroser
  アロゼ。}。また、完全には蓋をせず、加熱中に煮汁が煮詰まるようにしてやる。

ほぼ火が通ったら、鍋の蓋をとり、魚にかけた煮汁の水分をオーヴンの熱で蒸発させて表面につやを出す\footnote{glacer
  グラセ。}。魚を鍋から出して汁気をきり、皿に盛り保温しておく。

煮汁\footnote{原文 fonds de braisage (訳注2参照)。}を漉し、しばらく休ませたら浮き脂を取り除き、必要なら煮詰める。これを加えてソースを仕上げる。

魚のブレゼには通常、各ルセットに示してあるガルニテュールを添える。

\hypertarget{ux30aaux30d6ux30ebux30fc36-ux539fux66f8-pp.280-281}{%
\subsection[オ・ブルー 原書
pp.280-281]{\texorpdfstring{オ・ブルー\footnote{au bleu
  ヴィネガーを加えることで魚の表面のぬめりが青みがかることから。} 原書
pp.280-281}{オ・ブルー 原書 pp.280-281}}\label{ux30aaux30d6ux30ebux30fc36-ux539fux66f8-pp.280-281}}

オ・ブルーは鱒、鯉、ブロシェ\footnote{川かますの一種。}のみに用いられる特殊な調理法で、基本的なポイントは以下のとおり。

\begin{enumerate}
\def\labelenumi{\arabic{enumi}.}
\item
  必ず、生きた魚を使う。
\item
  魚の表面のぬめりをとらないように、なるべく手で触れずに、わたを抜く。鱗も引かない。
\item
  魚が大きい場合は、専用の網を敷いたポワソニーエルに入れ、「沸騰したヴィネガーをかける」。ヴィネガーは、通常のクールブイヨンに加える分量\footnote{以下の「クールブイヨンA」の分量比率を参照。}。次に、ヴィネガーを入れずに用意した温かい\footnote{原文
    tiède ぬるい、温かい。}クールブイヨンを注ぎ入れる。これは、なるべく身が割れないようにするためである。その後は通常どおり加熱する\footnote{レンジで沸騰させたらオーヴンに入れる。}。
\item
  小さい鱒の場合は、生きたままのものを手早く中抜きし、塩、ヴィネガーを加えただけの沸騰したクールブイヨンで煮る。
\item
  オ・ブルーは冷製、温製どちらの仕立てにしてもいい。実際の作り方の項で示してあるソースを添えて供する。
\end{enumerate}

\href{欠落アリ}{}

\hypertarget{ux30e0ux30cbux30a8ux30fcux30eb43-ux539fux66f8-p.282}{%
\subsection[ムニエール 原書 p.282]{\texorpdfstring{ムニエール\footnote{à
  la meunière 「粉挽き職人風」の意。} 原書
p.282}{ムニエール 原書 p.282}}\label{ux30e0ux30cbux30a8ux30fcux30eb43-ux539fux66f8-p.282}}

ムニエールは素晴らしい調理法だが、小型の魚と、大きな魚の場合は切り身にしか用いない。とはいえ、丁寧にやれば1.5
kg以下のテュルボタンはムニエールで調理できる。

魚丸ごと、あるいは切り身、フィレに味つけをして小麦粉をまぶし、バターを熱したフライパンで焼く。

魚が小さい場合は普通のバターでいいが、大きい場合は澄ましバターを使った方がいい。

魚の両面を焼き、程良く火が通ったら、予め熱しておいた皿に盛る。

飾り切りにした半割りのレモンを添えて、そのまま供することも可能である。ただし、このような提供方法の場合は本来の「ムニエール」と区別するために「黄金色に焼いた\footnote{doré
  (ドレ)
  一般的な色の表現として「黄金色」の意だが、ムニエールの場合、通常は大きい魚についてのみこの表現を用いる。}」と表現する。

「ムニエール」の場合には、焼き上がった魚に少量のレモン汁をふり、塩、こしょう少々で味を整える。粗みじん切りにして湯通ししたパセリを魚の表面に散らし、焦がしバターをかけてすぐに供する。湯通ししたパセリの水分に熱いバターが触れて泡がたつので、それが消えないうちに客に料理を見せるようにする。

%\input{06-poissons/06-02-p-284}
%\input{06-poissons/06-03-pp285-286}
%\input{06-poissons/06-04-p286}
%\input{06-poissons/06-05-pp286-290}

%\href{✓原稿下準備なし}{} \href{訳と注釈\%2020180420進行中}{}
\href{未、原文対照チェック}{} \href{未、日本語表現校正}{}
\href{未、注釈チェク}{} \href{未、原稿最終校正}{}

\hypertarget{poissons}{%
\chapter{VI 魚料理 Poissons}\label{poissons}}

\hypertarget{serie-de-courts-bouillons-de-poisson}{%
\section{クールブイヨン}\label{serie-de-courts-bouillons-de-poisson}}

\begin{recette}
\hypertarget{ux30afux30fcux30ebux30d6ux30a4ux30e8ux30f3-a}{%
\subsubsection{クールブイヨン
A}\label{ux30afux30fcux30ebux30d6ux30a4ux30e8ux30f3-a}}

水5 L に対し、ヴィネガー2.5 dL、粗塩60 g、薄切りにしたにんじん600
gと玉ねぎ500 g、タイム1枝、ローリエの小さな葉2枚、パセリの茎100
g、粒こしょう20
g(こしょうを加えるのはクールブイヨンを漉す10分前)。材料を全て鍋に入れ、火にかけて1時間弱火で煮、漉す(原書
p.277)。

\hypertarget{ux30afux30fcux30ebux30d6ux30a4ux30e8ux30f3-b-4-ux9c52ux3046ux306aux304eux30d6ux30edux30b7ux30a7ux7b49-ux539fux66f8p.277}{%
\subsubsection[クールブイヨン B (鱒、うなぎ、ブロシェ等)
原書p.277]{\texorpdfstring{クールブイヨン B \footnote{クールブイヨンは用途に応じ、AからEまでの5種が挙げられている(原書pp.277-278)。}
(鱒、うなぎ、ブロシェ等)
原書p.277}{クールブイヨン B  (鱒、うなぎ、ブロシェ等) 原書p.277}}\label{ux30afux30fcux30ebux30d6ux30a4ux30e8ux30f3-b-4-ux9c52ux3046ux306aux304eux30d6ux30edux30b7ux30a7ux7b49-ux539fux66f8p.277}}

5 L 分の材料\ldots{}\ldots{}白ワイン2.5 L 。水2.5 L
。薄切りにした玉ねぎ600 g。パセリの茎80g
。タイムの小枝1本。ローリエの葉(小) \(\frac{1}{2}\) 枚。粗塩
60g。大粒のこしょう15 g(クールブイヨンを漉す10分前に加える)。

作業手順\ldots{}\ldots{}作業:液体、香味素材、調味料を鍋に入れ、沸かす。弱火で30分程煮て、漉す。

\href{欠落アリ}{}

原注:クールブイヨンBとC\footnote{クールブイヨンBの白ワインを赤ワインに代え、香味素材としてにんじん400gを加える。鱒、鯉、マトロート用(原書pp.277-278)。}で調理した魚はクールブイヨン添えとして供する。つまり、少量の煮汁とクールブイヨンに用いた野菜を添える。野菜はよく火が通っていること。煮汁はしっかり煮詰め、提供直前に新鮮なバター少量を加えて仕上げる。
\end{recette}
\hypertarget{ux30afux30fcux30ebux30d6ux30a4ux30e8ux30f3ux306eux4f7fux3044ux65b9-ux539fux66f8-p.278}{%
\subsection{クールブイヨンの使い方 原書
p.278}\label{ux30afux30fcux30ebux30d6ux30a4ux30e8ux30f3ux306eux4f7fux3044ux65b9-ux539fux66f8-p.278}}

\begin{enumerate}
\def\labelenumi{\arabic{enumi}.}
\item
  加熱時間が30分以内の場合は、クールブイヨンは必ず事前に用意しておくこと。
\item
  加熱時間が30分を越える場合は、クールブイヨンの材料は冷たい状態のままで合わせせておく。香味素材はポワソニエールの網の下に入れる。
\item
  ごく少量のクールブイヨンでポシェ\footnote{原文 pochage à court
    mouillement『ル・ギード・キュリネール』では、この表現はテュルボタン(小型のテュルボ)、バルビュ、舌びらめ等の平たい魚をポシェする際に用いられる。本連載「舌びらめのボヌ・ファム」
    2011年3月号pp.110-111 参照。}する場合、材料は(白または赤ワインを含む場合も)魚を火にかける際に合わせる。クールブイヨンの量は魚の
  \(\frac{1}{3}\)
  の高さとし、加熱中ひんぱんに煮汁を魚にかけてやること\footnote{arroser
    アロゼ。}。この調理法の場合は通常、クールブイヨンは上で記したように
  \footnote{「クールブイヨンB」原注。}、提供直前に軽くバターを加えて仕上げ、魚に添える。
\item
  冷製にする場合は、必ずクールブイヨンに魚が浸った状態で冷ますこと。当然ながら、火にかけている時間は短かくなる\footnote{余熱で火が通るため。}。
\end{enumerate}

\hypertarget{ux539fux6ce8}{%
\subparagraph{【原注】}\label{ux539fux6ce8}}

いくつかの魚種の加熱時間は該当する項で示してある。

\hypertarget{ux9b5aux306eux8abfux7406ux6cd5}{%
\section{魚の調理法}\label{ux9b5aux306eux8abfux7406ux6cd5}}

魚料理は全て、下記のいずれかの調理法による。

\begin{enumerate}
\def\labelenumi{\arabic{enumi}.}
\item
  塩水(湯)またはクールブイヨン\footnote{court-bouillon直訳は「量の少ない煮汁」。魚の他、甲殻類、鶏などの白身肉、野菜などをポシェするのに用いる。とりわけ魚や鶏を丸ごとポシェする場合には、その名称のとおり、できるだけ少量でポシェする必要がある。また、ポシェに用いたクールブイヨンをベースにソースを作る場合が多い。}Bを用いたポシェ\ldots{}\ldots{}大きな魚丸ごと、および切り身。
\item
  ごく少量のクールブイヨンを用いたポシェ\ldots{}\ldots{}魚のフィレ、またはやや小さい魚。
\item
  ブレゼ\ldots{}\ldots{}もっぱら大きな魚。
\item
  オ・ブルー\footnote{比較的小さめの淡水魚に主として用いられる調理法。生きたままの魚の表面のぬめりをとらないように洗い、内臓を取り除いたらすぐに、塩とヴィネガーを加えたクールブイヨンで茹でる。冷製、温製どちらでも供する。原書p.281参照。}\ldots{}\ldots{}とりわけ\ruby{鱒}{ます}、鯉、ブロシェ\footnote{川かますの一種。本連載「ブロシェのクネル」2011年10月号
    pp.124-125 参照。}に合う。
\item
  揚げもの\ldots{}\ldots{}もっぱら小さい魚、切り身。
\item
  ムニエール\ldots{}\ldots{}揚げものにするのと同じ小さい魚、切り身。
\item
  グリエ\ldots{}\ldots{}小さい魚、および切り身。
\item
  グラタン\ldots{}\ldots{}小さい魚、切り身。
\end{enumerate}

\hypertarget{ux5869ux6c34ux6e6fux304aux3088ux3073ux30afux30fcux30ebux30d6ux30a4ux30e8ux30f3bux3092ux7528ux3044ux305fux52a0ux71b1ux8abfux7406}{%
\subsection{塩水(湯)およびクールブイヨンBを用いた加熱調理}\label{ux5869ux6c34ux6e6fux304aux3088ux3073ux30afux30fcux30ebux30d6ux30a4ux30e8ux30f3bux3092ux7528ux3044ux305fux52a0ux71b1ux8abfux7406}}

魚を丸ごと調理する場合は、魚に合ったポワソニエール\footnote{大きな魚を丸ごと煮るための細長い鍋。魚の形を崩さずに取り出せるよう、中に専用の網を敷いて使う。似たものに、舌びらめ等の平たい魚にぴったり合う菱形をしたテュルボティエールがある。いずれも、できるだけ少量の煮汁で魚を加熱できるように工夫されたもの。(図参照)}を用いる。魚を掃除し(テュルボは水にさらして血抜きをし)、ひれ等を切り落して形を整え、ポワソニエールの網に乗せる。魚種に応じて塩水または冷たいクールブイヨンをかぶるまで注ぐ。強火にかけて沸騰したらすぐにレンジの火の弱いところに鍋を移動させ、ポシェする。

切り身(薄すぎは絶対にいけない)の場合、沸騰した液体(塩湯またはクールブイヨン)に投入したらすぐにレンジの火の弱いところに鍋を移動させ、沸騰しない程度の温度でゆっくりと火を通す。

こうするのは、魚の身のエキスを閉じこめるためである。冷水から火にかけた場合にはエキスの大部分が流れ出してしまう。大きな魚丸ごとの場合にはこのやり方はしない。沸騰した液体に魚を投入すると身が収縮するので、大きな魚の場合は身が割れたり形が崩れたりするからだ。

塩湯あるいはクールブイヨンでポシェした魚は、ナフキンまたは専用の網に盛る。周囲をパセリで飾り、塩茹でしたじゃがいもと1種類または数種のソースを添えて供する。ガルニテュールがパセリのみの場合、魚の周囲にパセリを飾るのは客に料理を見せる\footnote{当時の宴席で主流だったロシア式サーヴィスでは、大きな銀盆に盛った料理をまず食客に見せてから、とり分けて給仕する。}直前にすること。どんな場合でも、ガルニテュールを添えたらクロッシュ\footnote{銀または陶製の保温用皿カバー。}は被せないこと。

\hypertarget{ux3054ux304fux5c11ux91cfux306eux30afux30fcux30ebux30d6ux30a4ux30e8ux30f3ux3092ux7528ux3044ux305fux30ddux30b7ux30a7-ux539fux66f8-pp.279-280}{%
\subsection{ごく少量のクールブイヨンを用いたポシェ 原書
pp.279-280}\label{ux3054ux304fux5c11ux91cfux306eux30afux30fcux30ebux30d6ux30a4ux30e8ux30f3ux3092ux7528ux3044ux305fux30ddux30b7ux30a7-ux539fux66f8-pp.279-280}}

この火入れの方法は主としてテュルボタン、バルビュ、舌びらめ、丸ごとおよびそれぞれの魚のフィレで用いる。バターを塗った天板あるいはソテ鍋に魚丸ごとあるいはそのフィレを置き、軽く塩をして、所要量の魚のフュメかマッシュルームの煮汁を注ぐ。フュメとマッシュルームの煮汁を合わせたものを用いる場合もある。蓋をして、中温のオーヴンに入れる。魚丸ごとの場合は時折煮汁をかけてやる。

魚(丸ごとでもフィレでも)に火が通ったら、注意して汁気をきり、皿に盛る。ガルニテュールを含む料理の場合、ガルニテュールを魚の周囲に盛り、ソースをかける\footnote{原文通りの順で訳したが、実際にはソースをかけてからガルニテュールを盛ったほうが良い場合もあるだろう。}。多くの場合、魚の煮汁を煮詰めてソースに加える。

\href{欠落アリ}{}

\hypertarget{ux9b5aux306eux30d6ux30ecux30bc17-ux539fux66f8-p.280}{%
\subsection[魚のブレゼ 原書 p.280]{\texorpdfstring{魚のブレゼ\footnote{ここではブレゼの語が限定的な意味で用いられていることに注意。牛のアロワイヨのような大きな塊肉のブレゼと同様の調理法、ということである。それは、香味素材を色づくまで炒めてから用いることや、主素材に豚背脂やトリュフをピケ針で差したり、豚背脂のシートで覆って加熱するという点によく表れている。ただし、これらは必須というわけではないため、事実上は「ごく少量のクールブイヨンを用いたポシェ」と区別がつきにくい。実際、モンタニェ『ラルース・ガストロノミーク』初版では、魚のブレゼについて「本来的な意味でのブレゼというよりは、ごく少量のクールブイヨンを用いたポシェである」と述べられている。逆に言えば、こんにち魚の加熱方法についてしばしば「ブレゼ」と呼ばれているものが、エスコフィエやモンタニェにおいては「少量のクールブイヨンを用いたポシェ」と表現されていたということである。}
原書
p.280}{魚のブレゼ 原書 p.280}}\label{ux9b5aux306eux30d6ux30ecux30bc17-ux539fux66f8-p.280}}

この調理法を用いるのは通常、丸ごとまたは筒切りにした鮭、大ぶりの鱒、テュルボ、テュルボタンのうち大きなもの、である。

場合によっては、魚の片面に、小さく切った豚背脂、トリュフ、コルニション、にんじん等をピケ針で差し込む。

香味素材等\footnote{原文 fonds de braisage フォン・ド・ブレザージュ
  (fonds de
  braiseフォン・ド・ブレーズ、とも)。通常は、厚い輪切りにしたにんじんと玉ねぎをバターか獣脂で色づくまで炒め、ブーケガルニ、下茹でした豚皮を合わせる(原書pp.394-395)。また、これを用いた煮汁のことも指す。}は肉料理のブレゼの場合と同じように用意するが、豚皮は用いない。提供方法に応じて、白または赤ワインと軽い魚のフュメ同量ずつを、魚の厚みの
\(\frac{3}{4}\) またはひたひたの高さまで注ぐ。厳密に肉断ち \footnote{カトリックの生活習慣として、四旬節(復活祭までの46日間)および週1
  回程度、肉類を食べないということが行なわれた(本連載2012年5月号「ソース・エスパニョル(4)」p.110、訳注1参照)。}のための仕立てにする場合を除いて、薄くスライスした豚背脂のシートを魚にかぶせる。加熱中\footnote{鍋を火にかけ、沸騰したら蓋をして中火のオーヴンに入れ、加熱する。}こまめに煮汁を魚にかけてやる\footnote{arroser
  アロゼ。}。また、完全には蓋をせず、加熱中に煮汁が煮詰まるようにしてやる。

ほぼ火が通ったら、鍋の蓋をとり、魚にかけた煮汁の水分をオーヴンの熱で蒸発させて表面につやを出す\footnote{glacer
  グラセ。}。魚を鍋から出して汁気をきり、皿に盛り保温しておく。

煮汁\footnote{原文 fonds de braisage (訳注2参照)。}を漉し、しばらく休ませたら浮き脂を取り除き、必要なら煮詰める。これを加えてソースを仕上げる。

魚のブレゼには通常、各ルセットに示してあるガルニテュールを添える。

\hypertarget{ux30aaux30d6ux30ebux30fc36-ux539fux66f8-pp.280-281}{%
\subsection[オ・ブルー 原書
pp.280-281]{\texorpdfstring{オ・ブルー\footnote{au bleu
  ヴィネガーを加えることで魚の表面のぬめりが青みがかることから。} 原書
pp.280-281}{オ・ブルー 原書 pp.280-281}}\label{ux30aaux30d6ux30ebux30fc36-ux539fux66f8-pp.280-281}}

オ・ブルーは鱒、鯉、ブロシェ\footnote{川かますの一種。}のみに用いられる特殊な調理法で、基本的なポイントは以下のとおり。

\begin{enumerate}
\def\labelenumi{\arabic{enumi}.}
\item
  必ず、生きた魚を使う。
\item
  魚の表面のぬめりをとらないように、なるべく手で触れずに、わたを抜く。鱗も引かない。
\item
  魚が大きい場合は、専用の網を敷いたポワソニーエルに入れ、「沸騰したヴィネガーをかける」。ヴィネガーは、通常のクールブイヨンに加える分量\footnote{以下の「クールブイヨンA」の分量比率を参照。}。次に、ヴィネガーを入れずに用意した温かい\footnote{原文
    tiède ぬるい、温かい。}クールブイヨンを注ぎ入れる。これは、なるべく身が割れないようにするためである。その後は通常どおり加熱する\footnote{レンジで沸騰させたらオーヴンに入れる。}。
\item
  小さい鱒の場合は、生きたままのものを手早く中抜きし、塩、ヴィネガーを加えただけの沸騰したクールブイヨンで煮る。
\item
  オ・ブルーは冷製、温製どちらの仕立てにしてもいい。実際の作り方の項で示してあるソースを添えて供する。
\end{enumerate}

\href{欠落アリ}{}

\hypertarget{ux30e0ux30cbux30a8ux30fcux30eb43-ux539fux66f8-p.282}{%
\subsection[ムニエール 原書 p.282]{\texorpdfstring{ムニエール\footnote{à
  la meunière 「粉挽き職人風」の意。} 原書
p.282}{ムニエール 原書 p.282}}\label{ux30e0ux30cbux30a8ux30fcux30eb43-ux539fux66f8-p.282}}

ムニエールは素晴らしい調理法だが、小型の魚と、大きな魚の場合は切り身にしか用いない。とはいえ、丁寧にやれば1.5
kg以下のテュルボタンはムニエールで調理できる。

魚丸ごと、あるいは切り身、フィレに味つけをして小麦粉をまぶし、バターを熱したフライパンで焼く。

魚が小さい場合は普通のバターでいいが、大きい場合は澄ましバターを使った方がいい。

魚の両面を焼き、程良く火が通ったら、予め熱しておいた皿に盛る。

飾り切りにした半割りのレモンを添えて、そのまま供することも可能である。ただし、このような提供方法の場合は本来の「ムニエール」と区別するために「黄金色に焼いた\footnote{doré
  (ドレ)
  一般的な色の表現として「黄金色」の意だが、ムニエールの場合、通常は大きい魚についてのみこの表現を用いる。}」と表現する。

「ムニエール」の場合には、焼き上がった魚に少量のレモン汁をふり、塩、こしょう少々で味を整える。粗みじん切りにして湯通ししたパセリを魚の表面に散らし、焦がしバターをかけてすぐに供する。湯通ししたパセリの水分に熱いバターが触れて泡がたつので、それが消えないうちに客に料理を見せるようにする。

%\href{✓原稿下準備なし}{} \href{訳と注釈\%2020180420進行中}{}
\href{未、原文対照チェック}{} \href{未、日本語表現校正}{}
\href{未、注釈チェク}{} \href{未、原稿最終校正}{}

\hypertarget{poissons}{%
\chapter{VI 魚料理 Poissons}\label{poissons}}

\hypertarget{serie-de-courts-bouillons-de-poisson}{%
\section{クールブイヨン}\label{serie-de-courts-bouillons-de-poisson}}

\begin{recette}
\hypertarget{ux30afux30fcux30ebux30d6ux30a4ux30e8ux30f3-a}{%
\subsubsection{クールブイヨン
A}\label{ux30afux30fcux30ebux30d6ux30a4ux30e8ux30f3-a}}

水5 L に対し、ヴィネガー2.5 dL、粗塩60 g、薄切りにしたにんじん600
gと玉ねぎ500 g、タイム1枝、ローリエの小さな葉2枚、パセリの茎100
g、粒こしょう20
g(こしょうを加えるのはクールブイヨンを漉す10分前)。材料を全て鍋に入れ、火にかけて1時間弱火で煮、漉す(原書
p.277)。

\hypertarget{ux30afux30fcux30ebux30d6ux30a4ux30e8ux30f3-b-4-ux9c52ux3046ux306aux304eux30d6ux30edux30b7ux30a7ux7b49-ux539fux66f8p.277}{%
\subsubsection[クールブイヨン B (鱒、うなぎ、ブロシェ等)
原書p.277]{\texorpdfstring{クールブイヨン B \footnote{クールブイヨンは用途に応じ、AからEまでの5種が挙げられている(原書pp.277-278)。}
(鱒、うなぎ、ブロシェ等)
原書p.277}{クールブイヨン B  (鱒、うなぎ、ブロシェ等) 原書p.277}}\label{ux30afux30fcux30ebux30d6ux30a4ux30e8ux30f3-b-4-ux9c52ux3046ux306aux304eux30d6ux30edux30b7ux30a7ux7b49-ux539fux66f8p.277}}

5 L 分の材料\ldots{}\ldots{}白ワイン2.5 L 。水2.5 L
。薄切りにした玉ねぎ600 g。パセリの茎80g
。タイムの小枝1本。ローリエの葉(小) \(\frac{1}{2}\) 枚。粗塩
60g。大粒のこしょう15 g(クールブイヨンを漉す10分前に加える)。

作業手順\ldots{}\ldots{}作業:液体、香味素材、調味料を鍋に入れ、沸かす。弱火で30分程煮て、漉す。

\href{欠落アリ}{}

原注:クールブイヨンBとC\footnote{クールブイヨンBの白ワインを赤ワインに代え、香味素材としてにんじん400gを加える。鱒、鯉、マトロート用(原書pp.277-278)。}で調理した魚はクールブイヨン添えとして供する。つまり、少量の煮汁とクールブイヨンに用いた野菜を添える。野菜はよく火が通っていること。煮汁はしっかり煮詰め、提供直前に新鮮なバター少量を加えて仕上げる。
\end{recette}
\hypertarget{ux30afux30fcux30ebux30d6ux30a4ux30e8ux30f3ux306eux4f7fux3044ux65b9-ux539fux66f8-p.278}{%
\subsection{クールブイヨンの使い方 原書
p.278}\label{ux30afux30fcux30ebux30d6ux30a4ux30e8ux30f3ux306eux4f7fux3044ux65b9-ux539fux66f8-p.278}}

\begin{enumerate}
\def\labelenumi{\arabic{enumi}.}
\item
  加熱時間が30分以内の場合は、クールブイヨンは必ず事前に用意しておくこと。
\item
  加熱時間が30分を越える場合は、クールブイヨンの材料は冷たい状態のままで合わせせておく。香味素材はポワソニエールの網の下に入れる。
\item
  ごく少量のクールブイヨンでポシェ\footnote{原文 pochage à court
    mouillement『ル・ギード・キュリネール』では、この表現はテュルボタン(小型のテュルボ)、バルビュ、舌びらめ等の平たい魚をポシェする際に用いられる。本連載「舌びらめのボヌ・ファム」
    2011年3月号pp.110-111 参照。}する場合、材料は(白または赤ワインを含む場合も)魚を火にかける際に合わせる。クールブイヨンの量は魚の
  \(\frac{1}{3}\)
  の高さとし、加熱中ひんぱんに煮汁を魚にかけてやること\footnote{arroser
    アロゼ。}。この調理法の場合は通常、クールブイヨンは上で記したように
  \footnote{「クールブイヨンB」原注。}、提供直前に軽くバターを加えて仕上げ、魚に添える。
\item
  冷製にする場合は、必ずクールブイヨンに魚が浸った状態で冷ますこと。当然ながら、火にかけている時間は短かくなる\footnote{余熱で火が通るため。}。
\end{enumerate}

\hypertarget{ux539fux6ce8}{%
\subparagraph{【原注】}\label{ux539fux6ce8}}

いくつかの魚種の加熱時間は該当する項で示してある。

\hypertarget{ux9b5aux306eux8abfux7406ux6cd5}{%
\section{魚の調理法}\label{ux9b5aux306eux8abfux7406ux6cd5}}

魚料理は全て、下記のいずれかの調理法による。

\begin{enumerate}
\def\labelenumi{\arabic{enumi}.}
\item
  塩水(湯)またはクールブイヨン\footnote{court-bouillon直訳は「量の少ない煮汁」。魚の他、甲殻類、鶏などの白身肉、野菜などをポシェするのに用いる。とりわけ魚や鶏を丸ごとポシェする場合には、その名称のとおり、できるだけ少量でポシェする必要がある。また、ポシェに用いたクールブイヨンをベースにソースを作る場合が多い。}Bを用いたポシェ\ldots{}\ldots{}大きな魚丸ごと、および切り身。
\item
  ごく少量のクールブイヨンを用いたポシェ\ldots{}\ldots{}魚のフィレ、またはやや小さい魚。
\item
  ブレゼ\ldots{}\ldots{}もっぱら大きな魚。
\item
  オ・ブルー\footnote{比較的小さめの淡水魚に主として用いられる調理法。生きたままの魚の表面のぬめりをとらないように洗い、内臓を取り除いたらすぐに、塩とヴィネガーを加えたクールブイヨンで茹でる。冷製、温製どちらでも供する。原書p.281参照。}\ldots{}\ldots{}とりわけ\ruby{鱒}{ます}、鯉、ブロシェ\footnote{川かますの一種。本連載「ブロシェのクネル」2011年10月号
    pp.124-125 参照。}に合う。
\item
  揚げもの\ldots{}\ldots{}もっぱら小さい魚、切り身。
\item
  ムニエール\ldots{}\ldots{}揚げものにするのと同じ小さい魚、切り身。
\item
  グリエ\ldots{}\ldots{}小さい魚、および切り身。
\item
  グラタン\ldots{}\ldots{}小さい魚、切り身。
\end{enumerate}

\hypertarget{ux5869ux6c34ux6e6fux304aux3088ux3073ux30afux30fcux30ebux30d6ux30a4ux30e8ux30f3bux3092ux7528ux3044ux305fux52a0ux71b1ux8abfux7406}{%
\subsection{塩水(湯)およびクールブイヨンBを用いた加熱調理}\label{ux5869ux6c34ux6e6fux304aux3088ux3073ux30afux30fcux30ebux30d6ux30a4ux30e8ux30f3bux3092ux7528ux3044ux305fux52a0ux71b1ux8abfux7406}}

魚を丸ごと調理する場合は、魚に合ったポワソニエール\footnote{大きな魚を丸ごと煮るための細長い鍋。魚の形を崩さずに取り出せるよう、中に専用の網を敷いて使う。似たものに、舌びらめ等の平たい魚にぴったり合う菱形をしたテュルボティエールがある。いずれも、できるだけ少量の煮汁で魚を加熱できるように工夫されたもの。(図参照)}を用いる。魚を掃除し(テュルボは水にさらして血抜きをし)、ひれ等を切り落して形を整え、ポワソニエールの網に乗せる。魚種に応じて塩水または冷たいクールブイヨンをかぶるまで注ぐ。強火にかけて沸騰したらすぐにレンジの火の弱いところに鍋を移動させ、ポシェする。

切り身(薄すぎは絶対にいけない)の場合、沸騰した液体(塩湯またはクールブイヨン)に投入したらすぐにレンジの火の弱いところに鍋を移動させ、沸騰しない程度の温度でゆっくりと火を通す。

こうするのは、魚の身のエキスを閉じこめるためである。冷水から火にかけた場合にはエキスの大部分が流れ出してしまう。大きな魚丸ごとの場合にはこのやり方はしない。沸騰した液体に魚を投入すると身が収縮するので、大きな魚の場合は身が割れたり形が崩れたりするからだ。

塩湯あるいはクールブイヨンでポシェした魚は、ナフキンまたは専用の網に盛る。周囲をパセリで飾り、塩茹でしたじゃがいもと1種類または数種のソースを添えて供する。ガルニテュールがパセリのみの場合、魚の周囲にパセリを飾るのは客に料理を見せる\footnote{当時の宴席で主流だったロシア式サーヴィスでは、大きな銀盆に盛った料理をまず食客に見せてから、とり分けて給仕する。}直前にすること。どんな場合でも、ガルニテュールを添えたらクロッシュ\footnote{銀または陶製の保温用皿カバー。}は被せないこと。

\hypertarget{ux3054ux304fux5c11ux91cfux306eux30afux30fcux30ebux30d6ux30a4ux30e8ux30f3ux3092ux7528ux3044ux305fux30ddux30b7ux30a7-ux539fux66f8-pp.279-280}{%
\subsection{ごく少量のクールブイヨンを用いたポシェ 原書
pp.279-280}\label{ux3054ux304fux5c11ux91cfux306eux30afux30fcux30ebux30d6ux30a4ux30e8ux30f3ux3092ux7528ux3044ux305fux30ddux30b7ux30a7-ux539fux66f8-pp.279-280}}

この火入れの方法は主としてテュルボタン、バルビュ、舌びらめ、丸ごとおよびそれぞれの魚のフィレで用いる。バターを塗った天板あるいはソテ鍋に魚丸ごとあるいはそのフィレを置き、軽く塩をして、所要量の魚のフュメかマッシュルームの煮汁を注ぐ。フュメとマッシュルームの煮汁を合わせたものを用いる場合もある。蓋をして、中温のオーヴンに入れる。魚丸ごとの場合は時折煮汁をかけてやる。

魚(丸ごとでもフィレでも)に火が通ったら、注意して汁気をきり、皿に盛る。ガルニテュールを含む料理の場合、ガルニテュールを魚の周囲に盛り、ソースをかける\footnote{原文通りの順で訳したが、実際にはソースをかけてからガルニテュールを盛ったほうが良い場合もあるだろう。}。多くの場合、魚の煮汁を煮詰めてソースに加える。

\href{欠落アリ}{}

\hypertarget{ux9b5aux306eux30d6ux30ecux30bc17-ux539fux66f8-p.280}{%
\subsection[魚のブレゼ 原書 p.280]{\texorpdfstring{魚のブレゼ\footnote{ここではブレゼの語が限定的な意味で用いられていることに注意。牛のアロワイヨのような大きな塊肉のブレゼと同様の調理法、ということである。それは、香味素材を色づくまで炒めてから用いることや、主素材に豚背脂やトリュフをピケ針で差したり、豚背脂のシートで覆って加熱するという点によく表れている。ただし、これらは必須というわけではないため、事実上は「ごく少量のクールブイヨンを用いたポシェ」と区別がつきにくい。実際、モンタニェ『ラルース・ガストロノミーク』初版では、魚のブレゼについて「本来的な意味でのブレゼというよりは、ごく少量のクールブイヨンを用いたポシェである」と述べられている。逆に言えば、こんにち魚の加熱方法についてしばしば「ブレゼ」と呼ばれているものが、エスコフィエやモンタニェにおいては「少量のクールブイヨンを用いたポシェ」と表現されていたということである。}
原書
p.280}{魚のブレゼ 原書 p.280}}\label{ux9b5aux306eux30d6ux30ecux30bc17-ux539fux66f8-p.280}}

この調理法を用いるのは通常、丸ごとまたは筒切りにした鮭、大ぶりの鱒、テュルボ、テュルボタンのうち大きなもの、である。

場合によっては、魚の片面に、小さく切った豚背脂、トリュフ、コルニション、にんじん等をピケ針で差し込む。

香味素材等\footnote{原文 fonds de braisage フォン・ド・ブレザージュ
  (fonds de
  braiseフォン・ド・ブレーズ、とも)。通常は、厚い輪切りにしたにんじんと玉ねぎをバターか獣脂で色づくまで炒め、ブーケガルニ、下茹でした豚皮を合わせる(原書pp.394-395)。また、これを用いた煮汁のことも指す。}は肉料理のブレゼの場合と同じように用意するが、豚皮は用いない。提供方法に応じて、白または赤ワインと軽い魚のフュメ同量ずつを、魚の厚みの
\(\frac{3}{4}\) またはひたひたの高さまで注ぐ。厳密に肉断ち \footnote{カトリックの生活習慣として、四旬節(復活祭までの46日間)および週1
  回程度、肉類を食べないということが行なわれた(本連載2012年5月号「ソース・エスパニョル(4)」p.110、訳注1参照)。}のための仕立てにする場合を除いて、薄くスライスした豚背脂のシートを魚にかぶせる。加熱中\footnote{鍋を火にかけ、沸騰したら蓋をして中火のオーヴンに入れ、加熱する。}こまめに煮汁を魚にかけてやる\footnote{arroser
  アロゼ。}。また、完全には蓋をせず、加熱中に煮汁が煮詰まるようにしてやる。

ほぼ火が通ったら、鍋の蓋をとり、魚にかけた煮汁の水分をオーヴンの熱で蒸発させて表面につやを出す\footnote{glacer
  グラセ。}。魚を鍋から出して汁気をきり、皿に盛り保温しておく。

煮汁\footnote{原文 fonds de braisage (訳注2参照)。}を漉し、しばらく休ませたら浮き脂を取り除き、必要なら煮詰める。これを加えてソースを仕上げる。

魚のブレゼには通常、各ルセットに示してあるガルニテュールを添える。

\hypertarget{ux30aaux30d6ux30ebux30fc36-ux539fux66f8-pp.280-281}{%
\subsection[オ・ブルー 原書
pp.280-281]{\texorpdfstring{オ・ブルー\footnote{au bleu
  ヴィネガーを加えることで魚の表面のぬめりが青みがかることから。} 原書
pp.280-281}{オ・ブルー 原書 pp.280-281}}\label{ux30aaux30d6ux30ebux30fc36-ux539fux66f8-pp.280-281}}

オ・ブルーは鱒、鯉、ブロシェ\footnote{川かますの一種。}のみに用いられる特殊な調理法で、基本的なポイントは以下のとおり。

\begin{enumerate}
\def\labelenumi{\arabic{enumi}.}
\item
  必ず、生きた魚を使う。
\item
  魚の表面のぬめりをとらないように、なるべく手で触れずに、わたを抜く。鱗も引かない。
\item
  魚が大きい場合は、専用の網を敷いたポワソニーエルに入れ、「沸騰したヴィネガーをかける」。ヴィネガーは、通常のクールブイヨンに加える分量\footnote{以下の「クールブイヨンA」の分量比率を参照。}。次に、ヴィネガーを入れずに用意した温かい\footnote{原文
    tiède ぬるい、温かい。}クールブイヨンを注ぎ入れる。これは、なるべく身が割れないようにするためである。その後は通常どおり加熱する\footnote{レンジで沸騰させたらオーヴンに入れる。}。
\item
  小さい鱒の場合は、生きたままのものを手早く中抜きし、塩、ヴィネガーを加えただけの沸騰したクールブイヨンで煮る。
\item
  オ・ブルーは冷製、温製どちらの仕立てにしてもいい。実際の作り方の項で示してあるソースを添えて供する。
\end{enumerate}

\href{欠落アリ}{}

\hypertarget{ux30e0ux30cbux30a8ux30fcux30eb43-ux539fux66f8-p.282}{%
\subsection[ムニエール 原書 p.282]{\texorpdfstring{ムニエール\footnote{à
  la meunière 「粉挽き職人風」の意。} 原書
p.282}{ムニエール 原書 p.282}}\label{ux30e0ux30cbux30a8ux30fcux30eb43-ux539fux66f8-p.282}}

ムニエールは素晴らしい調理法だが、小型の魚と、大きな魚の場合は切り身にしか用いない。とはいえ、丁寧にやれば1.5
kg以下のテュルボタンはムニエールで調理できる。

魚丸ごと、あるいは切り身、フィレに味つけをして小麦粉をまぶし、バターを熱したフライパンで焼く。

魚が小さい場合は普通のバターでいいが、大きい場合は澄ましバターを使った方がいい。

魚の両面を焼き、程良く火が通ったら、予め熱しておいた皿に盛る。

飾り切りにした半割りのレモンを添えて、そのまま供することも可能である。ただし、このような提供方法の場合は本来の「ムニエール」と区別するために「黄金色に焼いた\footnote{doré
  (ドレ)
  一般的な色の表現として「黄金色」の意だが、ムニエールの場合、通常は大きい魚についてのみこの表現を用いる。}」と表現する。

「ムニエール」の場合には、焼き上がった魚に少量のレモン汁をふり、塩、こしょう少々で味を整える。粗みじん切りにして湯通ししたパセリを魚の表面に散らし、焦がしバターをかけてすぐに供する。湯通ししたパセリの水分に熱いバターが触れて泡がたつので、それが消えないうちに客に料理を見せるようにする。

%\href{✓原稿下準備なし}{} \href{訳と注釈\%2020180420進行中}{}
\href{未、原文対照チェック}{} \href{未、日本語表現校正}{}
\href{未、注釈チェク}{} \href{未、原稿最終校正}{}

\hypertarget{poissons}{%
\chapter{VI 魚料理 Poissons}\label{poissons}}

\hypertarget{serie-de-courts-bouillons-de-poisson}{%
\section{クールブイヨン}\label{serie-de-courts-bouillons-de-poisson}}

\begin{recette}
\hypertarget{ux30afux30fcux30ebux30d6ux30a4ux30e8ux30f3-a}{%
\subsubsection{クールブイヨン
A}\label{ux30afux30fcux30ebux30d6ux30a4ux30e8ux30f3-a}}

水5 L に対し、ヴィネガー2.5 dL、粗塩60 g、薄切りにしたにんじん600
gと玉ねぎ500 g、タイム1枝、ローリエの小さな葉2枚、パセリの茎100
g、粒こしょう20
g(こしょうを加えるのはクールブイヨンを漉す10分前)。材料を全て鍋に入れ、火にかけて1時間弱火で煮、漉す(原書
p.277)。

\hypertarget{ux30afux30fcux30ebux30d6ux30a4ux30e8ux30f3-b-4-ux9c52ux3046ux306aux304eux30d6ux30edux30b7ux30a7ux7b49-ux539fux66f8p.277}{%
\subsubsection[クールブイヨン B (鱒、うなぎ、ブロシェ等)
原書p.277]{\texorpdfstring{クールブイヨン B \footnote{クールブイヨンは用途に応じ、AからEまでの5種が挙げられている(原書pp.277-278)。}
(鱒、うなぎ、ブロシェ等)
原書p.277}{クールブイヨン B  (鱒、うなぎ、ブロシェ等) 原書p.277}}\label{ux30afux30fcux30ebux30d6ux30a4ux30e8ux30f3-b-4-ux9c52ux3046ux306aux304eux30d6ux30edux30b7ux30a7ux7b49-ux539fux66f8p.277}}

5 L 分の材料\ldots{}\ldots{}白ワイン2.5 L 。水2.5 L
。薄切りにした玉ねぎ600 g。パセリの茎80g
。タイムの小枝1本。ローリエの葉(小) \(\frac{1}{2}\) 枚。粗塩
60g。大粒のこしょう15 g(クールブイヨンを漉す10分前に加える)。

作業手順\ldots{}\ldots{}作業:液体、香味素材、調味料を鍋に入れ、沸かす。弱火で30分程煮て、漉す。

\href{欠落アリ}{}

原注:クールブイヨンBとC\footnote{クールブイヨンBの白ワインを赤ワインに代え、香味素材としてにんじん400gを加える。鱒、鯉、マトロート用(原書pp.277-278)。}で調理した魚はクールブイヨン添えとして供する。つまり、少量の煮汁とクールブイヨンに用いた野菜を添える。野菜はよく火が通っていること。煮汁はしっかり煮詰め、提供直前に新鮮なバター少量を加えて仕上げる。
\end{recette}
\hypertarget{ux30afux30fcux30ebux30d6ux30a4ux30e8ux30f3ux306eux4f7fux3044ux65b9-ux539fux66f8-p.278}{%
\subsection{クールブイヨンの使い方 原書
p.278}\label{ux30afux30fcux30ebux30d6ux30a4ux30e8ux30f3ux306eux4f7fux3044ux65b9-ux539fux66f8-p.278}}

\begin{enumerate}
\def\labelenumi{\arabic{enumi}.}
\item
  加熱時間が30分以内の場合は、クールブイヨンは必ず事前に用意しておくこと。
\item
  加熱時間が30分を越える場合は、クールブイヨンの材料は冷たい状態のままで合わせせておく。香味素材はポワソニエールの網の下に入れる。
\item
  ごく少量のクールブイヨンでポシェ\footnote{原文 pochage à court
    mouillement『ル・ギード・キュリネール』では、この表現はテュルボタン(小型のテュルボ)、バルビュ、舌びらめ等の平たい魚をポシェする際に用いられる。本連載「舌びらめのボヌ・ファム」
    2011年3月号pp.110-111 参照。}する場合、材料は(白または赤ワインを含む場合も)魚を火にかける際に合わせる。クールブイヨンの量は魚の
  \(\frac{1}{3}\)
  の高さとし、加熱中ひんぱんに煮汁を魚にかけてやること\footnote{arroser
    アロゼ。}。この調理法の場合は通常、クールブイヨンは上で記したように
  \footnote{「クールブイヨンB」原注。}、提供直前に軽くバターを加えて仕上げ、魚に添える。
\item
  冷製にする場合は、必ずクールブイヨンに魚が浸った状態で冷ますこと。当然ながら、火にかけている時間は短かくなる\footnote{余熱で火が通るため。}。
\end{enumerate}

\hypertarget{ux539fux6ce8}{%
\subparagraph{【原注】}\label{ux539fux6ce8}}

いくつかの魚種の加熱時間は該当する項で示してある。

\hypertarget{ux9b5aux306eux8abfux7406ux6cd5}{%
\section{魚の調理法}\label{ux9b5aux306eux8abfux7406ux6cd5}}

魚料理は全て、下記のいずれかの調理法による。

\begin{enumerate}
\def\labelenumi{\arabic{enumi}.}
\item
  塩水(湯)またはクールブイヨン\footnote{court-bouillon直訳は「量の少ない煮汁」。魚の他、甲殻類、鶏などの白身肉、野菜などをポシェするのに用いる。とりわけ魚や鶏を丸ごとポシェする場合には、その名称のとおり、できるだけ少量でポシェする必要がある。また、ポシェに用いたクールブイヨンをベースにソースを作る場合が多い。}Bを用いたポシェ\ldots{}\ldots{}大きな魚丸ごと、および切り身。
\item
  ごく少量のクールブイヨンを用いたポシェ\ldots{}\ldots{}魚のフィレ、またはやや小さい魚。
\item
  ブレゼ\ldots{}\ldots{}もっぱら大きな魚。
\item
  オ・ブルー\footnote{比較的小さめの淡水魚に主として用いられる調理法。生きたままの魚の表面のぬめりをとらないように洗い、内臓を取り除いたらすぐに、塩とヴィネガーを加えたクールブイヨンで茹でる。冷製、温製どちらでも供する。原書p.281参照。}\ldots{}\ldots{}とりわけ\ruby{鱒}{ます}、鯉、ブロシェ\footnote{川かますの一種。本連載「ブロシェのクネル」2011年10月号
    pp.124-125 参照。}に合う。
\item
  揚げもの\ldots{}\ldots{}もっぱら小さい魚、切り身。
\item
  ムニエール\ldots{}\ldots{}揚げものにするのと同じ小さい魚、切り身。
\item
  グリエ\ldots{}\ldots{}小さい魚、および切り身。
\item
  グラタン\ldots{}\ldots{}小さい魚、切り身。
\end{enumerate}

\hypertarget{ux5869ux6c34ux6e6fux304aux3088ux3073ux30afux30fcux30ebux30d6ux30a4ux30e8ux30f3bux3092ux7528ux3044ux305fux52a0ux71b1ux8abfux7406}{%
\subsection{塩水(湯)およびクールブイヨンBを用いた加熱調理}\label{ux5869ux6c34ux6e6fux304aux3088ux3073ux30afux30fcux30ebux30d6ux30a4ux30e8ux30f3bux3092ux7528ux3044ux305fux52a0ux71b1ux8abfux7406}}

魚を丸ごと調理する場合は、魚に合ったポワソニエール\footnote{大きな魚を丸ごと煮るための細長い鍋。魚の形を崩さずに取り出せるよう、中に専用の網を敷いて使う。似たものに、舌びらめ等の平たい魚にぴったり合う菱形をしたテュルボティエールがある。いずれも、できるだけ少量の煮汁で魚を加熱できるように工夫されたもの。(図参照)}を用いる。魚を掃除し(テュルボは水にさらして血抜きをし)、ひれ等を切り落して形を整え、ポワソニエールの網に乗せる。魚種に応じて塩水または冷たいクールブイヨンをかぶるまで注ぐ。強火にかけて沸騰したらすぐにレンジの火の弱いところに鍋を移動させ、ポシェする。

切り身(薄すぎは絶対にいけない)の場合、沸騰した液体(塩湯またはクールブイヨン)に投入したらすぐにレンジの火の弱いところに鍋を移動させ、沸騰しない程度の温度でゆっくりと火を通す。

こうするのは、魚の身のエキスを閉じこめるためである。冷水から火にかけた場合にはエキスの大部分が流れ出してしまう。大きな魚丸ごとの場合にはこのやり方はしない。沸騰した液体に魚を投入すると身が収縮するので、大きな魚の場合は身が割れたり形が崩れたりするからだ。

塩湯あるいはクールブイヨンでポシェした魚は、ナフキンまたは専用の網に盛る。周囲をパセリで飾り、塩茹でしたじゃがいもと1種類または数種のソースを添えて供する。ガルニテュールがパセリのみの場合、魚の周囲にパセリを飾るのは客に料理を見せる\footnote{当時の宴席で主流だったロシア式サーヴィスでは、大きな銀盆に盛った料理をまず食客に見せてから、とり分けて給仕する。}直前にすること。どんな場合でも、ガルニテュールを添えたらクロッシュ\footnote{銀または陶製の保温用皿カバー。}は被せないこと。

\hypertarget{ux3054ux304fux5c11ux91cfux306eux30afux30fcux30ebux30d6ux30a4ux30e8ux30f3ux3092ux7528ux3044ux305fux30ddux30b7ux30a7-ux539fux66f8-pp.279-280}{%
\subsection{ごく少量のクールブイヨンを用いたポシェ 原書
pp.279-280}\label{ux3054ux304fux5c11ux91cfux306eux30afux30fcux30ebux30d6ux30a4ux30e8ux30f3ux3092ux7528ux3044ux305fux30ddux30b7ux30a7-ux539fux66f8-pp.279-280}}

この火入れの方法は主としてテュルボタン、バルビュ、舌びらめ、丸ごとおよびそれぞれの魚のフィレで用いる。バターを塗った天板あるいはソテ鍋に魚丸ごとあるいはそのフィレを置き、軽く塩をして、所要量の魚のフュメかマッシュルームの煮汁を注ぐ。フュメとマッシュルームの煮汁を合わせたものを用いる場合もある。蓋をして、中温のオーヴンに入れる。魚丸ごとの場合は時折煮汁をかけてやる。

魚(丸ごとでもフィレでも)に火が通ったら、注意して汁気をきり、皿に盛る。ガルニテュールを含む料理の場合、ガルニテュールを魚の周囲に盛り、ソースをかける\footnote{原文通りの順で訳したが、実際にはソースをかけてからガルニテュールを盛ったほうが良い場合もあるだろう。}。多くの場合、魚の煮汁を煮詰めてソースに加える。

\href{欠落アリ}{}

\hypertarget{ux9b5aux306eux30d6ux30ecux30bc17-ux539fux66f8-p.280}{%
\subsection[魚のブレゼ 原書 p.280]{\texorpdfstring{魚のブレゼ\footnote{ここではブレゼの語が限定的な意味で用いられていることに注意。牛のアロワイヨのような大きな塊肉のブレゼと同様の調理法、ということである。それは、香味素材を色づくまで炒めてから用いることや、主素材に豚背脂やトリュフをピケ針で差したり、豚背脂のシートで覆って加熱するという点によく表れている。ただし、これらは必須というわけではないため、事実上は「ごく少量のクールブイヨンを用いたポシェ」と区別がつきにくい。実際、モンタニェ『ラルース・ガストロノミーク』初版では、魚のブレゼについて「本来的な意味でのブレゼというよりは、ごく少量のクールブイヨンを用いたポシェである」と述べられている。逆に言えば、こんにち魚の加熱方法についてしばしば「ブレゼ」と呼ばれているものが、エスコフィエやモンタニェにおいては「少量のクールブイヨンを用いたポシェ」と表現されていたということである。}
原書
p.280}{魚のブレゼ 原書 p.280}}\label{ux9b5aux306eux30d6ux30ecux30bc17-ux539fux66f8-p.280}}

この調理法を用いるのは通常、丸ごとまたは筒切りにした鮭、大ぶりの鱒、テュルボ、テュルボタンのうち大きなもの、である。

場合によっては、魚の片面に、小さく切った豚背脂、トリュフ、コルニション、にんじん等をピケ針で差し込む。

香味素材等\footnote{原文 fonds de braisage フォン・ド・ブレザージュ
  (fonds de
  braiseフォン・ド・ブレーズ、とも)。通常は、厚い輪切りにしたにんじんと玉ねぎをバターか獣脂で色づくまで炒め、ブーケガルニ、下茹でした豚皮を合わせる(原書pp.394-395)。また、これを用いた煮汁のことも指す。}は肉料理のブレゼの場合と同じように用意するが、豚皮は用いない。提供方法に応じて、白または赤ワインと軽い魚のフュメ同量ずつを、魚の厚みの
\(\frac{3}{4}\) またはひたひたの高さまで注ぐ。厳密に肉断ち \footnote{カトリックの生活習慣として、四旬節(復活祭までの46日間)および週1
  回程度、肉類を食べないということが行なわれた(本連載2012年5月号「ソース・エスパニョル(4)」p.110、訳注1参照)。}のための仕立てにする場合を除いて、薄くスライスした豚背脂のシートを魚にかぶせる。加熱中\footnote{鍋を火にかけ、沸騰したら蓋をして中火のオーヴンに入れ、加熱する。}こまめに煮汁を魚にかけてやる\footnote{arroser
  アロゼ。}。また、完全には蓋をせず、加熱中に煮汁が煮詰まるようにしてやる。

ほぼ火が通ったら、鍋の蓋をとり、魚にかけた煮汁の水分をオーヴンの熱で蒸発させて表面につやを出す\footnote{glacer
  グラセ。}。魚を鍋から出して汁気をきり、皿に盛り保温しておく。

煮汁\footnote{原文 fonds de braisage (訳注2参照)。}を漉し、しばらく休ませたら浮き脂を取り除き、必要なら煮詰める。これを加えてソースを仕上げる。

魚のブレゼには通常、各ルセットに示してあるガルニテュールを添える。

\hypertarget{ux30aaux30d6ux30ebux30fc36-ux539fux66f8-pp.280-281}{%
\subsection[オ・ブルー 原書
pp.280-281]{\texorpdfstring{オ・ブルー\footnote{au bleu
  ヴィネガーを加えることで魚の表面のぬめりが青みがかることから。} 原書
pp.280-281}{オ・ブルー 原書 pp.280-281}}\label{ux30aaux30d6ux30ebux30fc36-ux539fux66f8-pp.280-281}}

オ・ブルーは鱒、鯉、ブロシェ\footnote{川かますの一種。}のみに用いられる特殊な調理法で、基本的なポイントは以下のとおり。

\begin{enumerate}
\def\labelenumi{\arabic{enumi}.}
\item
  必ず、生きた魚を使う。
\item
  魚の表面のぬめりをとらないように、なるべく手で触れずに、わたを抜く。鱗も引かない。
\item
  魚が大きい場合は、専用の網を敷いたポワソニーエルに入れ、「沸騰したヴィネガーをかける」。ヴィネガーは、通常のクールブイヨンに加える分量\footnote{以下の「クールブイヨンA」の分量比率を参照。}。次に、ヴィネガーを入れずに用意した温かい\footnote{原文
    tiède ぬるい、温かい。}クールブイヨンを注ぎ入れる。これは、なるべく身が割れないようにするためである。その後は通常どおり加熱する\footnote{レンジで沸騰させたらオーヴンに入れる。}。
\item
  小さい鱒の場合は、生きたままのものを手早く中抜きし、塩、ヴィネガーを加えただけの沸騰したクールブイヨンで煮る。
\item
  オ・ブルーは冷製、温製どちらの仕立てにしてもいい。実際の作り方の項で示してあるソースを添えて供する。
\end{enumerate}

\href{欠落アリ}{}

\hypertarget{ux30e0ux30cbux30a8ux30fcux30eb43-ux539fux66f8-p.282}{%
\subsection[ムニエール 原書 p.282]{\texorpdfstring{ムニエール\footnote{à
  la meunière 「粉挽き職人風」の意。} 原書
p.282}{ムニエール 原書 p.282}}\label{ux30e0ux30cbux30a8ux30fcux30eb43-ux539fux66f8-p.282}}

ムニエールは素晴らしい調理法だが、小型の魚と、大きな魚の場合は切り身にしか用いない。とはいえ、丁寧にやれば1.5
kg以下のテュルボタンはムニエールで調理できる。

魚丸ごと、あるいは切り身、フィレに味つけをして小麦粉をまぶし、バターを熱したフライパンで焼く。

魚が小さい場合は普通のバターでいいが、大きい場合は澄ましバターを使った方がいい。

魚の両面を焼き、程良く火が通ったら、予め熱しておいた皿に盛る。

飾り切りにした半割りのレモンを添えて、そのまま供することも可能である。ただし、このような提供方法の場合は本来の「ムニエール」と区別するために「黄金色に焼いた\footnote{doré
  (ドレ)
  一般的な色の表現として「黄金色」の意だが、ムニエールの場合、通常は大きい魚についてのみこの表現を用いる。}」と表現する。

「ムニエール」の場合には、焼き上がった魚に少量のレモン汁をふり、塩、こしょう少々で味を整える。粗みじん切りにして湯通ししたパセリを魚の表面に散らし、焦がしバターをかけてすぐに供する。湯通ししたパセリの水分に熱いバターが触れて泡がたつので、それが消えないうちに客に料理を見せるようにする。



%%% chapitre vii. relevés et entrées
%% エスコフィエ『料理の手引き』全注解
% 五島 学


\href{✓原稿下準備なし}{} \href{訳と注釈\%2020180424古い原稿のコピペ}{}
\href{未、原文対照チェック}{} \href{未、日本語表現校正}{}
\href{未、注釈チェク}{} \href{未、原稿最終校正}{}

\hypertarget{vii.ux8089ux6599ux7406-relevuxe9s-et-entruxe9es}{%
\chapter{VII. 肉料理 Relevés et
Entrées}\label{vii.ux8089ux6599ux7406-relevuxe9s-et-entruxe9es}}

\begin{center}
\frsec{ブレゼ、ポワレ、ソテ、ポシェの基礎、下茹で\\グラタン、グリル、揚げものの調理理論}
\end{center}
\normalsize
\vspace{1\zw}
\frsec{Princie généraux de la Conduite de Braisée --- des Poêlés --- des Sautés --- et des Pochés --- Blanchissages\\\vspace{.5ex}Théorie des Gratins, des Grillades et des Fritures}
\normalsize

\hypertarget{les-braises}{%
\section{ブレゼ}\label{les-braises}}

いろいろな調理法の中でも、ブレゼはとりわけコストがかかり、しかもきわめて高度な技術が要求される。この調理法を習得するには時間をかけて、注意深く実践を重ねていくしかない。細心の注意を払って調理しなければならないのは勿論のこと、他の調理法もそうだが、主素材となる肉の品質は非常に重要だ。美味しいブレゼを作るにはさらに、加熱の際に上質のフォンを用い、香味素材
(フォン・ド・ブレーズ)もきちんと仕込みをしておく必要がある。

\hypertarget{ux30d6ux30ecux30bcux306bux7528ux3044ux308bux8089}{%
\subsection{ブレゼに用いる肉}\label{ux30d6ux30ecux30bcux306bux7528ux3044ux308bux8089}}

ブレゼの通常の調理法としてここで述べるのは、牛肉、羊肉を用いる。仔牛、乳飲み仔羊、家禽のブレゼについては後述。ブレゼの素材は、ロティの場合と違って、若い畜肉でなくていい。牛の場合は3〜6才、羊は1〜2才のものが最良だ。上記の年齢を過ぎたものはいい肉質を期待できない。そのような肉を使う場合は加熱時間をかなり長くとる必要が出てくるし、それでも大抵は、筋っぽくてぱさついた仕上りになってしまう。実際問題として、料理に老齢の畜肉を用いるのは、コンソメと各種フォンをとる場合のみと考えること。

\hypertarget{ux30e9ux30ebux30c7}{%
\subsection{ラルデ}\label{ux30e9ux30ebux30c7}}

アロワイヨや背肉の塊は、ペルシエつまり脂肪の筋が入ったものであれば良質で柔らかい。牛や羊のもも肉の場合は事情が異なる。牛、羊のもも肉はそれ自体が脂をあまり含んでいないため、長時間加熱するとぱさついてしまう。脂を補うために、約1cm角の棒状に切った豚背脂を、繊維方向に沿って肉の内部に刺し込む(ラルデ)。背脂は予{[}あらかじ{]}め、こしょう、ナツメグ、その他の香辛料で味つけし、パセリのみじん切りを振ってから、コニャック適量で2時間マリネしておくこと。

\hypertarget{ux30deux30eaux30cd}{%
\subsection{マリネ}\label{ux30deux30eaux30cd}}

背脂を刺す場合も、そうでない場合も、ワインと香味素材で素材を数時間マリネする。ワインは煮汁として、香味素材はフォン・ド・ブレーズとして用いることになる。マリネする前に、素材を塩、こしょう、その他の香辛料で味つけする。肉を転がし、調味料がよく浸み込むようにする。素材が丁度入る大きさの容器の底に香味素材を敷いて、その上に肉を置き、さらに香味素材で覆う。煮汁に用いるワインをひたひたの高さまで注ぐ。一般的には、白でも赤でも普通のワインを用いる。分量は肉1kgあたり3㎗。5〜6時間マリネする。途中、何度か肉を裏返す。

\hypertarget{ux9999ux5473ux7d20ux6750ux30d5ux30a9ux30f3ux30c9ux30d6ux30ecux30fcux30ba}{%
\subsection{香味素材(フォン・ド・ブレーズ)}\label{ux9999ux5473ux7d20ux6750ux30d5ux30a9ux30f3ux30c9ux30d6ux30ecux30fcux30ba}}

にんじんと玉ねぎを厚めの輪切りにし、バターまたはグレス・ド・マルミートを用いて強火でこんがり炒める(野菜の量は肉1kgあたり各60g)。ブーケガルニ
(にんにくを入れること)。肉1kgあたり50gの生豚皮(下茹でしておく)。

\hypertarget{ux30eaux30bdux30ecux30d6ux30ecux30bc}{%
\subsection{リソレ、ブレゼ}\label{ux30eaux30bdux30ecux30d6ux30ecux30bc}}

いい具合にマリネした状態になったら、肉を取り出して網にあげ、30分間水気をきる。さらに布などで水気を拭き取る。丁度いい大きさの、厚手の片手鍋またはブレゼ鍋に、不純物を取り除いたグレス・ド・マルミートを熱する。素材を鍋に入れ、強火で表面をまんべんなくこんがりと焼く(リソレ)。こうすることで、肉に一種の鎧をまとわせ、肉汁があまり早く外に流れ出ないようにする。そうでないと、ブレゼではなくブイイになってしまう。素材が大きければ大きい程、表面の焼き固めた層がそれだけ丈夫でなくてはいけないから、素材の表面を焼く時間は長くなることになる。

表面を焼き固めたら素材を鍋から取り出す。脂の少ない肉の場合は、豚背脂のシートでくるみ、紐で縛る。牛のアロワイヨや背肉の場合、この作業は必要ない。そのままの状態で、脂の層に覆われているからだ。

素材の大きさにぴったり合うサイズの鍋に、香味野菜(マリネに用いたもの)、豚皮、ブーケガルニを入れ、その上に素材を置く。マリネに用いたワインを注ぎ、強火にかけてシロップ状になるまで煮詰める。上質の茶色いフォンを素材がかぶるまで注ぎ、沸騰させる。鍋に蓋をして中火のオーヴンに入れ、安定した微沸騰の状態を保つようにする。

細いブリデ針を深く刺してみて、穴から血が上ってこなくなるまで火を通す。この時点で、ブレゼの第1段階は終了。以下が第2段階だが、その調理メカニズムについては後述する。

煮汁については、目的に応じて以下のどちらかの方法をとる。

煮汁を澄んだままにしておきたい場合は、肉を充分な大きさのきれいな別鍋に移し、布漉しした煮汁を注ぐ。この鍋をオーヴンに入れ、こまめに煮汁をかけてやりながら加熱を仕上げる。

通常の「とろみを付けたジュ」と同様に、コーンスターチでとろみを付ける。ブレゼのソースとして仕上げる場合は、煮汁を半量まで煮詰める。その⅔量のソース・エスパニョルと⅓量のトマトピュレまたは同等量の生のトマトを加え、もとの煮汁と同じ量にする。これを、上で述べたように別鍋に移しておいた肉にかけてオーヴンに入れ、こまめにソースをかけてやりながら加熱を仕上げる。

肉にナイフを刺してみて、抵抗なく入っていくようであれば、ほどよく火が通っているので、肉をソースから取り出す。ソースは布で漉し、脂がすっかり表面に浮いてくるまで10分程休ませる。浮いてきた脂は徹底的に取り除く。最後に、ソースが濃いようなら上質のフォン少量を加え、薄過ぎるようなら煮詰めて、ソースを仕上げる。

\hypertarget{ux30d6ux30ecux30bcux7b2cux6bb5ux968e-ux306eux8abfux7406ux30e1ux30abux30cbux30baux30e0-ux539fux66f8-p.396}{%
\subsection{ブレゼ第2段階 の調理メカニズム 原書
p.396}\label{ux30d6ux30ecux30bcux7b2cux6bb5ux968e-ux306eux8abfux7406ux30e1ux30abux30cbux30baux30e0-ux539fux66f8-p.396}}

作業の第1段階のところで述べたように、肉の塊が大きければ大きい程、表面をしっかり焼き固める必要がある。表面を焼き固めるのは、外に逃げ出そうとする肉汁を内側に押し返すことと、表面に一種の鎧を作ることが目的だ。この鎧は、加熱が進むにつれ周囲から中心に向かって厚いものになっていく。

鍋の液体が熱せられると、肉の筋繊維が締まり肉汁は中心に向かうが、やがて熱が中心まで届くと、そこに圧縮された肉汁は分解して余計な水分が分離される。水分は蒸気となり、筋繊維を膨らませてほぐすのだ。

つまり、第1段階では明らかに、肉の塊の中心に向かって肉汁が濃縮されていく。

第2段階ではこれとは逆向きの現象が起こる。

肉の塊の中心部に集まった肉汁の水分が蒸発するだけの温度に達すると、筋繊維がほぐれ始める。肉汁から分離した水分が、逃げ場がないために蒸気圧をかけるのだ。筋繊維はかなりの圧力の影響を受けるわけだが、この圧力は第1段階とは逆に、肉の中心から周囲に向かうことになる。

だから、加熱が進み、肉の内部の圧力が高まるとその結果、筋繊維はゆるんでいく。肉汁の水分は、外側の焼き固めた層に少しずつ達してその筋繊維をゆるめ、内部の肉汁が流れ出す通り道が出来る。肉汁はソースと混ざり、同時に、毛管現象によってソースも肉の内部に浸み込む。ここが、ブレゼの調理でもっとも注意を払わなければならないところだ。作業は最終段階で、煮汁はかなり煮詰まってきて、肉を覆ってはいない。
何も覆うものがないから肉はとてもすぐに乾燥してしまうから、こまめに煮汁をかけてやり、肉を裏返す。肉の組織が常にソースを吸い込んだ状態にしてやるのだ。こうすることで、他の調理法とは一線を画すブレゼの特徴、柔らかくてとろけるような仕上りとなる。

\hypertarget{ux7167ux308aux3092ux3064ux3051ux308bux30b0ux30e9ux30bb}{%
\subsection{照りをつける(グラセ)}\label{ux7167ux308aux3092ux3064ux3051ux308bux30b0ux30e9ux30bb}}

ブレゼを塊のまま客にプレゼンテーションする場合には絶対に照りをつける必要があるが、切り分けてから供する場合には必須ではないし、不要とも言える。

照りをつける場合、ほど良く火が通ったらすぐに鍋から肉を取り出し、平鍋に移してオーヴンの入口近くに入れる。煮汁かジュ、ソースを軽くかけると、オーヴンの熱で煮詰まって、表面に薄い膜が出来る。この作業を、肉が艶のある層ですっかり覆われるまで繰り返す。オーヴンから出したら皿に盛り、供するまでクロッシュをかぶせておく。

\hypertarget{ux6ce8ux610fux4e8bux9805}{%
\subsection{注意事項}\label{ux6ce8ux610fux4e8bux9805}}

例えばブフ・アラモードのように野菜を添える場合には、バターで色良く炒めてから、手順の第2段階で肉と一緒に煮るか、あるいは、煮汁の一部を取り分けて肉とは別に煮る。

一番いいのは最初のやり方だが、緻密な盛り付けをするには向かない。だから、臨機応変にどちらがいいか判断する必要がある。ブレゼの作り方について、一般的に行なわれてはいるけれども絶対に間違っていることが2つある。ひとつ目は、香味野菜(フォン・ド・ブレーズ)をパンセすることだ。

香味野菜をパンセするというのは、予{[}あらかじ{]}め色良く炒めておりた野菜の上に、焼き色を付けた肉を置くのではなく、ブレゼ鍋の底に生のままの野菜を並べ、その上に肉を置くが、多くの場合は肉にも焼き色を付けない。溶かしたグレス・ド・マルミートを少量かけてやり、野菜が鍋底に軽く焦げ付くまで加熱する。----
厳密に言えば、上手にやるならこの方法も許容されるだろう。けれど、片面しか焼き色を付けていない野菜では両面に焼き色を付けた場合ほどは風味は出ないし、その上、加熱時間が長過ぎるとほとんど黒焦げになりかねず、苦味が出てソースの風味を損ねてしまう。

実際のところ、このパンセという作業は、煮汁に用いるフォンを事前に仕込んでおくこともせず、そのフォンの材料をブレゼそれ自体と一緒に煮ていた大昔の料理のうわべだけ真似たものに過ぎない。

昔のブレゼの作り方は素晴しいけれど、とてもコストのかかるものだった。というのも、厚くスライスした生ハムと仔牛のもも肉にのせて主素材を同時にブレゼしていたのだ。コストの問題でこのやり方でブレゼを作らなくなって久しいのだが、本質的なところは無視して、形式的な手順だけが慣習として残ったのだ。しかも、昔はフォンの材料に肉を用いていたのを各種の獣骨で代用するようになってしまったのだから、誤りとしてますますひどいものになってしまった。

そこで、第2の誤りである。

よく知られているように、ブレゼに最もよく使われるのは仔牛の骨だが、それでさえ完全に煮出すのには10〜12時間かかる。まず5〜6時間煮てフォンをとった骨に、さらに液体を注いで6時間煮た方が、5〜6時間煮ただけのフォンよりも多くのグラス・ド・ヴィアンドが得られるのがその証拠だ。2番のフォンで作ったグラス・ド・ヴィアンドは風味は劣る一方で、ゼラチン質が多いのは事実だ。ブレゼに用いるフォンとしては、このゼラチン質は風味の要素に負けず劣らず意味がある。ゼラチン質によってブレゼはなめらかでとろけるような口当たりになる。これは他のものでは代用できないし、ソースの出来を決めるものなのだ。

ブレゼに生の骨を加えたとしても、ブレゼの加熱時間は最大でも4〜5時間以上には出来ないのだから、肉に火が通った時点で骨は表面しか煮えていないわけだ。骨に含まれているもののほとんどは煮出されないままになってしまう。----
つまり、ブレゼに骨を加えても全く意味がないのだ。

この間違った方法には、また別の欠点もある。素材を煮るのに大量の液体を用いなければならないということだ。ブレゼというのはソースがしっかり煮詰められてこくがないと完全ではない、というのは誰もが認めるところだろう。煮汁が多ければ多い程、ソースは薄味になるし、結果的に煮汁で肉を洗うようなことになってしまう。

だからこそ、既に述べたように、ブレゼでは素材の大きさにぴったり合った容量の鍋を用いるべきなのだ。肉は始めすっかり煮汁に浸っていなければならないわけだから、鍋の大きさが丁度良ければそれだけ煮汁の絶対量は少なくなり、素材から溶け出すものも加わってフォンに一層こくが出るのだ。

\hypertarget{ux767dux8eabux8089ux306eux30d6ux30ecux30bc}{%
\section{白身肉のブレゼ}\label{ux767dux8eabux8089ux306eux30d6ux30ecux30bc}}

当今の白身肉のブレゼは、本来的な意味ではブレゼとは呼べない。赤身肉のブレゼの作り方は2段階で作業を行なうのが特徴だが、その1段階で火入れを止めてしまうからである。

昔の料理ではブレゼの2段階の作業が行なわれていなかったのは事実だ。が、仔牛等の大きな塊肉はスプーンで切れるくらいまでよく火を通すことが多かったのだ。---
現代ではこうした調理は行なわれなくなってしまい、名称だけが残ったわけだ。

白身肉のブレゼは次のとおり。仔牛の背肉、鞍下肉、腰肉、もも肉。フリカンド、リ・ド・ヴォ、若い七面鳥、肥鶏。頻度は少ないが、仔羊のドゥーブルやバロン、鞍下肉でもよく作られる。

上記の肉すべて同じ作り方にするが、加熱時間は肉の大きさによって変わるので注意。

ブレゼに用いる香味素材は赤身肉のブレゼと同じだが、野菜は色づかないようバターで軽く炒めるだけにすること。煮汁には必ず白いフォンを用いる。

\hypertarget{ux767dux8eabux8089ux306eux30d6ux30ecux30bcux306eux4f5cux696dux624bux9806}{%
\subsection{白身肉のブレゼの作業手順}\label{ux767dux8eabux8089ux306eux30d6ux30ecux30bcux306eux4f5cux696dux624bux9806}}

リ・ド・ヴォは調理前に必ず下茹でするので別だが、ブレゼする白身肉や家禽は表面を全て、軽く色づく程度まで焼き固めてもいい。その方がパサつきにくくなる。とはいえ、この作業は省いても良い。

次に、底に香味素材を敷いた鍋に肉を入れる。鍋の大きさは肉がちょうど入るくらいで、蓋をした際に肉が蓋に当たらない程度の深さのものを用いる。

この鍋に仔牛のフォン少量を注ぎ、蓋をして弱火で煮詰める。再び同量のフォンを注いで同じように煮詰める。それから肉の半分の高さまで煮汁を注ぐ。沸騰したら弱火のオーヴンに入れる。弱火といっても、煮汁が微沸騰の状態を保つ程度の温度であること。

加熱中は肉の表面が乾かないように、こまめに煮汁を肉にかけてやること。フォンにはゼラチン質が多く含まれているので、表面に塗膜のようなものが出来て、熱による肉汁の蒸散を防いでくれる。肉の表面をごく軽く焼き固めただけでは保護層が充分に出来ていないからだ。

そのために、最終的に煮汁を注ぐ前に少量のフォンを煮詰めておいたわけだ。肉を鍋に入れてそのまま煮汁を注いだとして、上で述べたような膜が出来るほどフォンが濃くならないだろうから、肉はひどくパサついてしまうだろう。

白身肉のブレゼで火の通り具合をみるには、ブリデ針を深く刺す。穴から透明の肉汁が上がってくるようになったら程良く火が通っている。透明の肉汁が上がってくるというのは、肉の中心まで火が通って血が分解された証拠なのだ。

この火の通し加減という点で、赤身肉のブレゼと白身肉のブレゼは大きく異なるわけだ。実際のところ、白身肉のブレゼの火入れ加減はほとんどロティに近い。だから家禽とごく若い畜肉の脂がのっていて柔らかいものしか使わないのだ。というのも、この料理では、ロティと同じくらいの程良い火入れ加減を少しでも越えたらたちまちおいしくなくなってしまうからだ。

白身肉のブレゼは通常、照りをつけてやる。とりわけ、細く切った豚背脂をピケ針で刺している場合には照りをつけてやったほうがいい。豚背脂を刺すのは昔と比べて減ったが、まだこの方法を採る者も多い。

\hypertarget{les-poches}{%
\section{ポシェ}\label{les-poches}}

こんな表現が成り立つならの話だが、ポシェとは「沸騰させないで作るブイイ」とするのが最も正しい定義だろう。

「ポシェ」という用語は広い意味では、何らかの液体を用いて弱火でゆっくり火を通すことを指す。液体の量が多いか少ないかは問題とならない。だから、大きなテュルボや鮭を丸ごとクールブイヨンで煮るのはもとより、舌びらめの切り身を少量の魚のフュメで煮る場合や、温製のムースやムスリーヌ、クネル、クレーム、ロワイヤル等々の加熱についても、「ポシェ」と呼ばれる。

このようにポシェする対象は多岐にわたるので、それぞれの加熱時間は大きく異なる。けれども、全てに共通して絶対守るべき原則がある。ポシェする液体は決して沸騰させない、沸騰寸前の温度にするということだ。

もうひとつ大事なことだが、魚や鶏を丸ごとポシェする際は液体が冷たい状態で火にかけ、手早く所定の温度まで上げるようにする。ごく少量の液体で魚や鶏の切り身をポシェする場合も同様にしていい。これに対し、他のポシェの場合は事前に所定の温度にしておいた液体に投入する手順となる。

\hypertarget{preparation-des-volailles-a-pocher}{%
\subsection{鶏のポシェの下ごしらえ}\label{preparation-des-volailles-a-pocher}}

\frsecb{Préparation des Volailles à pocher}

鶏は下処理の後、指示があれば詰め物をし、ブリデ針を用い糸で手羽と脚を畳み込むように縛る。

細かく切ったトリュフ、ハム、ラング・エカルラートを鶏の胸や脚に刺す場合には、半割りにしたレモンで擦ってから、沸騰した白いフォンに数分間浸してやる。

こうすることで皮が締まり、トリュフ等を刺す作業がやり易くなる。

\hypertarget{pochage-de-la-volaille}{%
\subsection{鶏のポシェ}\label{pochage-de-la-volaille}}

\frsecb{Pochage de la Volaille}

詰め物をしたり、トリュフ等を刺すのは必要な時だけだが、どんな場合でも豚背脂のシートで包んでやる。丁度いい大きさの鍋に素材を入れ、事前に仕込んでおいた白いフォンをかぶるまで注ぐ。

火にかけて沸騰したらアクを引き、蓋をして所定の温度、つまり目で見てほとんどわからない程度の微沸騰の状態を保つようにする。熱がだんだん伝わって鶏に火を通すにはこれで充分な温度だ。

明らかな沸騰状態にしてしまうといろいろ不都合が起きる。とりわけ (1)
水分の蒸発が激しすぎて煮汁が煮詰まり、澄んだ状態を保てなくなる。(2)
詰め物をしている場合は特に、皮が弾けやすくなる。

鶏のポシェで火の通り加減を見るには、ドラムスティックに近い腿の裏側を刺してみる。完全に透明な汁が上がってくれば程良く火が通っている。

\hypertarget{nota-sur-les-pochages-de-volaille}{%
\subsection{注意事項}\label{nota-sur-les-pochages-de-volaille}}

\begin{itemize}
\item
  鶏をポシェするのに丁度いい大きさの鍋を使うべき理由は\ldots{}\ldots{}

  \begin{enumerate}
  \def\labelenumi{\arabic{enumi}.}
  \item
    加熱中、素材が常にフォンに浸っていなければならない。
  \item
    煮汁そのものをソースに用いるので、煮汁の全体量が少なければ少ない程、鶏から流れる肉汁が薄まりにくくなる。結果として、ソースの風味が良くなる。
  \end{enumerate}
\item
  {[}その他{]}\footnote{原書ではここに項目名はないが、箇条書きを見やすくするために訳者が補った。}  

  \begin{enumerate}
  \def\labelenumi{\arabic{enumi}.}
  \item
    ポシェに使う白いフォンは必ず事前に仕込んでおく。充分に澄んだフォンを用いること。
  \item
    もしもフォンをとる材料と鶏を一緒に火にかけたら、フォンの材料がどんなにたくさんでも、いい結果は得られないだろう。理由は、鶏の加熱時間は最大でも1時間〜1時間半であるのに対して、フォンの材料から香りと栄養素を充分に引き出すには最低6時間はかかる。その結果、単なるお湯に近い液体で鶏のポシェが完了し、その煮汁から作ったソースも味気ないものになってしまうだろう。
  \end{enumerate}
\end{itemize}

\hypertarget{les-poeles}{%
\section{ポワレ}\label{les-poeles}}

\frsec{Les Poêlés}

ポワレは事実上ロティの一種と言える。ロティもポワレも目指す火入れは同じだ。

ここで記すポワレは次のような古い調理法を単純化したものだ。古い調理法では、あらかじめ素材の表面に焼き色をつけ、たっぷりのマティニョンで覆ってから豚背脂のシートやバターを塗った紙で包み、オーヴンまたは串を刺して直火で、溶かしバターをかけながら焼いていた。

火が通ったらすぐに包みを外して脂をきる。マティニョンをブレゼ鍋または片手鍋に移し、マデイラと煮詰めたフォンを加える。

マティニョンの香味がフォンに移ったら、フォンを漉し、提供直前に浮き脂を取り除いて仕上げていた。

家禽を丸ごと調理する仕立てのいくつかについては、今なおこの古い方法で作る価値がある。

\hypertarget{ux30ddux30efux30ecux306eux4f5cux696dux624bux9806}{%
\subsection{ポワレの作業手順}\label{ux30ddux30efux30ecux306eux4f5cux696dux624bux9806}}

素材に対して余裕のある大きさの厚手の深鍋の底にマティニョンを敷きつめる
(マティニョンについては「ガルニテュールの仕込み」参照)。

畜肉あるいは家禽にしっかり味付けをし、野菜の上に置く。溶かしバターをたっぷりかけてやる。鍋に蓋をして、やや高温のオーヴンに入れる。

そうして、小まめにバターをかけながら、蓋をした状態でじっくり火を入れる。

火が通ったら鍋の蓋を取り、オーヴンの熱で素材に焼き色をつける。皿に移し、クロッシュをかぶせて保温しておく。

野菜(焦げていないこと)に充分な量の煮詰めた澄んだフォンを注ぐ。弱火で10
分間煮てから布で漉し、丁寧に浮き脂を取り除く。これをソース容器に入れ、主素材の周囲にガルニテュールを盛って供する。

\hypertarget{ux30ddux30efux30ecux306bux3064ux3044ux3066ux306eux6ce8ux610f}{%
\subsection{ポワレについての注意}\label{ux30ddux30efux30ecux306bux3064ux3044ux3066ux306eux6ce8ux610f}}

\begin{enumerate}
\def\labelenumi{\arabic{enumi}.}
\item
  ポワレは火入れに液体を用いないのが重要ポイント。液体を用いたら白身肉のブレゼと同じ風味になってしまう。ポワレは火入れにバターしか用いない。ただし、雉、ペルドロ、うずら等の猟鳥のポワレでは概ね火が通った時点で少量のコニャックを注いでフランベする。
\item
  フォンを注ぐ前に野菜から脂を取り除かないことも大事なポイントだ。
\end{enumerate}

実際、ポワレに用いたバターには主素材と野菜の風味が溶け込んでいるわけだ。この風味を取り出すためには、フォンを注いで最低10分以上バターと接しているようにする必要がある。その後であれば、バターを取り除いてもフォンの香味を損なうことはない。

\hypertarget{ux7279ux6b8aux306aux30ddux30efux30ec-ux30abux30b9ux30edux30fcux30ebux30b3ux30b3ux30c3ux30c8}{%
\subsection{特殊なポワレ ----
カスロール、ココット}\label{ux7279ux6b8aux306aux30ddux30efux30ec-ux30abux30b9ux30edux30fcux30ebux30b3ux30b3ux30c3ux30c8}}

カスロール、ココットという調理は畜肉、家禽、ジビエを専用の陶製の鍋で火入れし、鍋ごと供するが、これはまさにポワレそのものと言える。

一般的に、「カスロール」は野菜を加えずバターだけを用いて素材に火を通す。素材に程良く火が通ったら主素材を取り出し、鍋に仔牛のフォン少量を注ぐ。数分間沸かしてから、浮いている余分なバターを取り除く。主素材を鍋に戻し、保温しておくが、沸騰させないこと。

「ココット」も同様に調理するが、マッシュルーム、アーティチョークの萼の基底部、小玉ねぎ、にんじん、かぶ等の野菜を加えて調理する。野菜はそれぞれの性質に応じて形を整え、バターで炒めて半ば火を通しておくこと。

出来るだけ新野菜を使うようにすること。野菜は主素材の周りに入れるが、主素材と同時に火入れが終了するタイミングで加えること。「ココット」に用いる陶製の鍋はある程度使い込んだものの方がいい。重曹や洗剤は用いずに水できれいに洗って手入れすること。

新品の鍋を用いなければならない場合、軽く沸かした湯をいっぱいに注ぎ、12
時間以上は微沸騰の状態を保つようにしてやる。その後、水気を拭き取る。さらに、冷水をいっぱいに注ぎしばらく放置してから使用すること。

\hypertarget{ux30bdux30c6-les-sautes}{%
\section{ソテ \{les-sautes\}}\label{ux30bdux30c6-les-sautes}}

ソテと呼ばれる調理法は水を使わないで火入れをするのが特徴だ。バター、植物油や精製した獣脂だけを用いる。

ソテに用いるのは、捌いた家禽やジビエ、あるいはソテに適するようにカットした畜肉である。

ソテに用いる素材は全て、強く熱した油脂で表面を焼き固める。表面に層を作り、肉汁を外に流れ出させずに内部に留めておくのが目的だ。この作業はとりわけ牛や羊のような赤身肉の場合に行なう。

家禽とジビエのソテは、素材に焼き色をつけたらソテ鍋に蓋をしてレンジで、あるいは蓋をせずオーヴンに入れてロティールと同様に焼き脂をかけながら火入れを仕上げる。

次に、素材を鍋から取り出してデグラセする。主素材を鍋に戻してソースあるいは付合せとからめる場合はごく短時間、つまりソースの風味がなじむ程度にとどめる。

トゥルヌド、ノワゼート、コトレート、フィレ、アントルコートのような赤身肉のソテでは、少量の澄ましバターで表面を焼き固め、そのままレンジで加熱を行なう。

表面を焼き固める際には、素材が小さく、薄く切ったものであればそれだけ強火にする。

焼いていない生のままの面に血が滲み出てきたら裏返す。先に焼いた面にピンク色の肉汁が出てきたら程良く火が通っている。

ソテ鍋から肉を取り出し、脂を捨ててから、ソースの一部となるワイン等の液体を鍋に注いで沸騰させ、鍋底についた肉汁を溶かし出す。こうしてデグラセした液体にソースを加える
----
場合によってはその逆で、別途用意しておいたソースやガルニテュールに加えることもある。仕立てとしての「ソテ」ではデグラセを必ず行なうこと。

ソテ鍋は素材の大きさにぴったり合ったものを使用する。大き過ぎると、肉が接していない部分が高温になり、デグラセが上手く出来なくなってしまう。デグラセは、肉を焼いた際に流れ出て固形化した肉汁を液体で溶かし出し、それによってソースがおいしくなるわけだから、デグラセが上手く出来ないとソースがおいしくなくなってしまうのだ。

仔牛、仔羊のような白身肉のソテは、まず表面を焼き固めてから弱火で火を通す。

白身肉の他の調理法と同様、しっかりと火を通すこと。

「ソテ」という名称は、ソテとブレゼ両方の特徴を兼ね備えた料理にも使われる。これは実際のところ「ラグー」と呼ぶのがふさわしいものだ。

こうした料理には牛、仔牛、仔羊、ジビエ等が用いられる。本書ではエストゥファード、グーラーシュ、仔牛のソテ、仔羊のソテ、カルボナード、ナヴァラン、シヴェ等の名称でまとめてある。

調理の第1段階では通常のソテと同様に小さめに切った肉に焼き色を付ける。第2段階はソースやガルニテュールと合わせて時間をかけて火を通すという点でブレゼと良く似ている。

\hypertarget{ux4e0bux8339ux3067}{%
\section{下茹で}\label{ux4e0bux8339ux3067}}

\hypertarget{ux30b0ux30e9ux30bfux30f3}{%
\section{グラタン}\label{ux30b0ux30e9ux30bfux30f3}}

\hypertarget{ux30b0ux30eaux30eb}{%
\section{グリル}\label{ux30b0ux30eaux30eb}}

グリルにおける調理上の働きとして重要なのは「凝縮」である。

グリルで重要なのは肉汁であり、殆どの場合、肉汁を内部に凝縮させることをまず目指すべきだ。

グリルとは要するに直火で焼くロティと同じであり、人類が料理ということを始めた遥か遠い起源に遡れるだろう。原始人の堅い頭に最初に生じたこのグリエというアイデアは、「もっと美味しいものを食べたい」という本能的な欲求から生まれた進化であり、食べ物を加熱調理するのに用いられた初めての方法なのだ。

やがて、その当然の帰結としてグリルから串焼きのロティが発生した。その頃には既に人類は本能ではなく知性で考えるようになっており、理屈によって結果を推論し、実地から結論を導き出すようになっていた。こうして、料理は進歩への道を歩みはじめたのだ。

\hypertarget{ux30b0ux30eaux30ebux306eux71b1ux6e90}{%
\subsection{グリルの熱源}\label{ux30b0ux30eaux30ebux306eux71b1ux6e90}}

もっとも一般的に用いられており、明らかに最良のものと言える燃料は熾火、または小さめの木炭である。どんな種類の燃料でも、重要なのは煙を出さないということだ。火力を強めるために風を送る場合は、その風で煙が外に流れ去るようにしてやる。

ましてや、あまりないことだが、自然と火がくすぶってしまい人工的に風を送ってやらねばならない時でも、煙が出てはいけない。熱源以外の物が燃えたり、炭に脂が落ちて煙が出てしまったら、人工的に風を送ろうと、強い風が吹こうと、きちんと煙を追い出せない限りは、どうしようもなく不味いグリルになってしまうからだ。

とはいえ、他の種類の熱源をグリルに用いても構わない。本書はこの点で絶対を主張はしない。反対に、適切に使うならばどんな熱源でも良いと言える。

\hypertarget{ux706bux5e8a}{%
\subsection{火床}\label{ux706bux5e8a}}

火床あるいはグリル台の造りも重要だ。グリルする素材の性質や大きさはもとより、状況によって火力を強めたり弱めたり自在に出来なければならない。

だから火床は炉の中央に平らに配する。ただし火力の強弱をつける必要に応じて厚みには変化をつけられるようにする。また、風が当たる側はやや高くして、熱がまんべんなく均等に行きわたるようにする。

焼き網は必ず前もって火床に据え、素材をのせる時には充分に熱くなっておくようにする。網を熱くしておかないと素材が貼り付き、裏返す際に素材を壊してしまうことになる。

\hypertarget{ux30b0ux30eaux30ebux306eux5206ux985e}{%
\subsection{グリルの分類}\label{ux30b0ux30eaux30ebux306eux5206ux985e}}

グリルは4種に分類され、それぞれ注意すべき点が異なる。

\begin{itemize}
\tightlist
\item
  赤身肉のグリル(牛、羊、ジビエ)
\item
  白身肉のグリル(仔牛、仔羊、家禽)
\item
  魚のグリル
\item
  パン粉衣をつけたグリル。パン粉のみをまぶす場合と、イギリス風パン粉衣を用いる場合がある。
\end{itemize}

\hypertarget{ux8d64ux8eabux8089ux306eux30b0ux30eaux30eb}{%
\subsection{赤身肉のグリル}\label{ux8d64ux8eabux8089ux306eux30b0ux30eaux30eb}}

グリルの作業は何よりもまず、各素材に適切な加熱温度を見きわめることから始まる。

素材が大きければ大きい程、肉汁が多ければ多い程、強火で表面をしっかり焼き固める必要がある。

この「リソレ」の役割と利点については「ブレゼ」のページで既に述べたが、グリルに関してもう一度おさらいしておこう。

牛や羊のような赤身肉を大きく切ったものをグリルする場合、上質の素材で肉汁の豊富なものであればなおのこと、しっかりした層が出来るよう表面を焼き固める必要がある。

内部の肉汁が多ければそれだけ、表面の焼き固めた層に向かう圧力は強くなる。肉汁が熱されるにつれてこの圧力は強くなる。

素材の内部にゆっくり熱が伝わるように火力を上手く調節していれば、次のような調理メカニズムとなる。

肉の火に接している側は、焼き網を通った熱によって線維が収縮し、肉の内部に熱が伝わっていく。熱は層をなすように肉の内部に広がり、肉汁を逆流させる。しまいには肉汁が肉の反対側に逹し、火にあたっていない生のままの面に滲み出てくる。このタイミングで肉を裏返し、焼いていない面について同じプロセスを行なう。肉汁が上に向かって逆流して最初に焼いた面でいったん止まり、そこから血の雫が浮かび上がってきたら、程良く火が通っている。

素材が大きい場合、表面をしっかり焼き固めたらすぐに火を弱め、熱がゆっくりと内部に伝わるようにする。同じ火の強さのまま焼き続けたら、焼き固めた表面がすぐに焦げてしまい、熱が肉の内部に伝わるのを邪魔する結果となる。外側は黒焦げなのに内部はまるで生という状態になってしまう。

あまり厚みのない肉をグリルする場合は、強火で表面を焼き固め、数分間そのまま焼けば程良く火が通る。火を弱める必要はない。例)
ランプやシャトブリヤンに程良く火を通すには、まず表面を強火で焼き固めて肉汁が流れ出ないようにしてから、弱めの火加減にしてやり、熱がゆっくり伝わって中まで火が通るようにする。

トゥルヌドやフィレミニョン、ノワゼート、コトレートのような小さいカットの肉の場合は、必ず表面を強火で焼き固め、そのままの火の強さで焼き上げる。厚みがないために熱が中まですぐに伝わるからである。

\hypertarget{ux8d64ux8eabux8089ux306eux30b0ux30eaux30ebux306eux969bux306bux6c17ux3092ux3064ux3051ux308bux3079ux304dux3053ux3068}{%
\subsection{赤身肉のグリルの際に気をつけるべきこと}\label{ux8d64ux8eabux8089ux306eux30b0ux30eaux30ebux306eux969bux306bux6c17ux3092ux3064ux3051ux308bux3079ux304dux3053ux3068}}

肉を焼き網にのせる前に、澄ましバターを刷毛でまんべんなく塗っておく。焼いている間も同様に澄ましバターを小まめに塗り、火が当たっている部分が乾かないようにする。

肉を裏返すにはパレットナイフか、出来たらグリル用のトングを使う。肉用フォーク等で刺して裏返すのは避けるべきだ。肉を刺せば肉汁の流れ出る通り道が出来るわけだから、それまで気を配ってきたこと全てをわざわざ台無しにしてしまうことになる。

\hypertarget{ux7a0bux826fux3044ux706bux306eux901aux308aux5177ux5408}{%
\subsection{程良い火の通り具合}\label{ux7a0bux826fux3044ux706bux306eux901aux308aux5177ux5408}}

赤身肉の場合、火の通り具合は指で表面に触れてみて押すと抵抗を感じる程度が、内部まで熱が伝わって程良く火が通っている目安となる。反対に、指で押しても全然抵抗を感じないようなら、熱がまだ内部まで伝わっていない。肉の焼き固めた表面にピンク色の肉汁がわずかに浮び上ってくる状態がいちばん確かな目印だ。

\hypertarget{ux767dux8eabux8089ux306eux30b0ux30eaux30eb}{%
\subsection{白身肉のグリル}\label{ux767dux8eabux8089ux306eux30b0ux30eaux30eb}}

赤身肉の場合は必ず表面を強火で焼き固めるが、白身肉の場合その必要はまったくない。仔牛や仔羊のような白身肉は肉汁がアルブミンの状態でしか存在しておらず、焼く際に凝縮させるよう気を配る必要がないからだ。

白身肉のグリルは弱めの火加減で、肉に火を通しながら同時に表面にきれいな焼き色がつくようにする。焼いている間、小まめにバターをかけてやり、表面が乾かないようにする。

浸み出てくる肉汁がすっかり透明になったら、程良く火が通っている。

\hypertarget{ux9b5aux306eux30b0ux30eaux30eb}{%
\subsection{魚のグリル}\label{ux9b5aux306eux30b0ux30eaux30eb}}

魚は大小にかかわらず、バターか植物油をたっぷりかけてから、やや弱めの火加減でグリルする。火入れの最中も小まめにバターまたは油をかけてやること。

魚の場合は、骨から簡単に身が離れるようになったら程良く火が通っている。

鯖{[}さば{]}、ルージェ、鰊{[}にしん{]}のような脂ののった魚以外は、焼く前に小麦粉をまぶしつけ、溶かしバターをかける。こうして魚を黄金色の殻で覆うようにすると、身が乾くのを防ぐと同時に、見た目も良くなる。

\hypertarget{ux30a4ux30aeux30eaux30b9ux98a8ux307eux305fux306fux30d0ux30bfux30fcux3067ux30d1ux30f3ux7c89ux8863ux3092ux3064ux3051ux305fux30b0ux30eaux30eb}{%
\subsection{イギリス風またはバターでパン粉衣をつけたグリル}\label{ux30a4ux30aeux30eaux30b9ux98a8ux307eux305fux306fux30d0ux30bfux30fcux3067ux30d1ux30f3ux7c89ux8863ux3092ux3064ux3051ux305fux30b0ux30eaux30eb}}

一般的に小さな素材でしかこの方法は用いないが、ごく弱火でグリルし、主素材の火入れとパン粉衣の色づけが同時に行なわれるようにする。

加熱中は小まめに澄ましバターをかけてやる。衣は素材の肉汁を閉じ込めるためのものなので、裏返す時は衣を壊さないように注意すること。

\hypertarget{ux63daux3052ux3082ux306e}{%
\section{揚げもの}\label{ux63daux3052ux3082ux306e}}






%%% chapitre xiii legumes
%\href{未、原文対照チェック}{} \href{未、日本語表現校正}{}
\href{未、その他修正}{} \href{未、原稿最終校正}{}

\hypertarget{legumes-farineux-et-pates-alimentaires}{%
\chapter{XIII. 野菜料理・パスタなど}\label{legumes-farineux-et-pates-alimentaires}}

\frchap{Légumes --- Farineux et Pâtes alimentaires}

\hypertarget{serie-des-legumes}{%
\section{野菜料理}\label{serie-des-legumes}}

\begin{center}
\headfont\large 下拵えと注意事項\label{observations-sur-les-operations-preliminaires}

\normalfont\textit{Observation sur les Opérations préliminaires.}
\end{center}

\normalfont\normalsize

\hypertarget{blanchissage}{%
\subsection[湯がく・下茹で]{\texorpdfstring{湯がく・下茹で\footnote{blanchir(ブロンシール)。もともとの意味は「白くする」.食材を冷蔵保存出来なかった中世には肉類はいったん下茹でしてから調理するのが一般的であり,肉を下茹ですると表面が白くなることからこの用語が用いられるようになった.素材の種類によっては白く茹であげるために単なる湯ではなく「ブラン」を用いる場合もある.これは、水1ℓにスプーン1杯強の小麦粉,塩6gとスプーン2杯の溶いて沸かす.クローヴを刺した玉ねぎ1ヶとブーケガルニ,下茹でする素材,空気に素材が触れて変色するのを防ぐための脂を入れる.脂は,牛あるいは仔牛の生のケンネ脂を細かく刻んだもの.必要なら脂を事前に冷水にさらして血等の夾雑物がないようにしておくこと.
  (原注)
  野菜の下茹で用のブランには,ヴィネガーではなくレモン汁を用いたほうがいい(原書
  p.405).}}{湯がく・下茹で}}\label{blanchissage}}

この作業は2つの目的で行なう。第1に、例えばほうれんそう、プティポワ、さやいんげん等の一般的な葉物野菜では、完全に火を通すのが目的。たっぷりの湯で手早く茹で、クロロフィルすなわち葉緑素を失わないようにすること。第
2は、野菜に自然にあるえぐ味を消す目的。例えばキャベツ、セロリ、シコレ等。原則的に、新野菜は下茹でしない。下茹でで完全に火を通してしまう野菜については、1リットルあたり7gの塩を湯に加えること。

\hypertarget{rafraichissage}{%
\subsection[冷水にはなす]{\texorpdfstring{冷水にはなす\footnote{rafraîchir
  ラフレシール}}{冷水にはなす}}\label{rafraichissage}}

湯がいた後、冷水にとるのは、野菜をブレゼにする場合と、オペレーションの都合から事前に茹でておかなければならない場合のみ。ただし、後でバターやクリームで合える場合は、冷水にとると風味が失われる。

\hypertarget{cuisson-des-legumes-a-l-anglaise}{%
\subsection{アラングレーズ}\label{cuisson-des-legumes-a-l-anglaise}}

沸騰した湯で茹でるのみ。次によく水切りをして、さらに水気をとばす。深皿に盛りつけ、貝殻形のバターを添えて供する。味付けは食べ手がする。葉物野菜なら何でもこのイギリス風に調理、提供可能。

\hypertarget{cuisson-des-legumes-secs}{%
\subsection{乾燥豆}\label{cuisson-des-legumes-secs}}

乾燥豆を水でもどしておくのはよろしくない。その年に穫れた良質のものなら、水から弱火でゆっくり沸かして茹でればいい。あく取りをして香味野菜\footnote{ベシャメルソース1ℓに生クリーム2dlを加え,火にかけてへらで混ぜながら
  ¾ℓになるまで煮つめる.
  布で漉し,クレーム・ドゥーブル1½dlとレモン汁½個分を少しずつ加えていき、濃さを調節する(原書p.32)。}を加え、蓋をしてごく弱火で茹でる。

あまりにも古い豆や品質が劣るものは水でもどす。ただし、豆が膨れるのに必要な時間きっかり、すなわち1時間半から2時間とすること。

何時間も水につけておくと、発酵が始まってしまう。そうなると豆の組織が損なわれて使いものにならなくなってしまうことさえある。

\hypertarget{braisage-des-legumes}{%
\subsection{野菜のブレゼ}\label{braisage-des-legumes}}

野菜は事前に湯がいて冷水にとる。形を整える。鍋の底と周囲に豚背脂のシートを張り、野菜を入れる。上面を背脂で覆う。鍋に蓋をして弱火でかるく汗をかかせるように蒸し煮した後、ひたひたまで白いフォンを注ぐ。鍋に蓋をし、中温のオーヴンに入れる。火が通ったら、野菜の水気をきり、用途に合わせて形を整える。その後すぐに使う場合は、煮汁の浮き脂を取り除いて煮つめ、野菜とともにソテ鍋で保温する。事前に仕込んでおく場合には、鍋から皿あるいは専用の容器に移す。煮汁は浮き脂を取り除かずにそのまま加える。バターを塗った紙で覆ってストックする。

\hypertarget{sauce-des-legumes-braises}{%
\subsection{野菜のブレゼのソース}\label{sauce-des-legumes-braises}}

ブレゼの煮汁を煮詰め、浮き脂を丁寧に取り除いて使う。場合によってはグラスドヴィアンド、あるいは相応量のドゥミ・グラスを加える。どちらの場合にも、

ソースがまろやかになるようバターを加えて仕上げる。必要ならレモン汁数滴も加える。

\hypertarget{liaison-des-legumes-vert-au-beurre}{%
\subsection{葉野菜をバターであえる}\label{liaison-des-legumes-vert-au-beurre}}

茹でた野菜はしっかりと水をきっておく。味付けをしてバターを加え、鍋をあおるようにしてバターが野菜全体にまわるようにする。バターの風味を失なわないためには、「火から外した状態」でバターを加えること。

\hypertarget{liaison-des-legumes-a-la-creme}{%
\subsection{クリームであえる}\label{liaison-des-legumes-a-la-creme}}

この方法で調理する場合は、野菜に火を通す際、やや歯応えを残しておくこと。

しっかり水気をきり、野菜を鍋に入れる。沸かした生クリームを野菜が上に顔を出す程度に加える。

時々、よくかきまぜながら加熱する。

クリームがほぼすっかり煮詰まったら、バターとレモン汁少量を加える。必要なら、生クリームにソース・クレーム\footnote{blanchir(ブロンシール)。もともとの意味は「白くする」.食材を冷蔵保存出来なかった中世には肉類はいったん下茹でしてから調理するのが一般的であり,肉を下茹ですると表面が白くなることからこの用語が用いられるようになった.素材の種類によっては白く茹であげるために単なる湯ではなく「ブラン」を用いる場合もある.これは、水1ℓにスプーン1杯強の小麦粉,塩6gとスプーン2杯の溶いて沸かす.クローヴを刺した玉ねぎ1ヶとブーケガルニ,下茹でする素材,空気に素材が触れて変色するのを防ぐための脂を入れる.脂は,牛あるいは仔牛の生のケンネ脂を細かく刻んだもの.必要なら脂を事前に冷水にさらして血等の夾雑物がないようにしておくこと.
  (原注)
  野菜の下茹で用のブランには,ヴィネガーではなくレモン汁を用いたほうがいい(原書
  p.405).}を足してもいい。

\hypertarget{cremes-et-puree-de-legumes}{%
\subsection{野菜のクレームとピュレ}\label{cremes-et-puree-de-legumes}}

乾燥豆とでんぷん質の野菜は裏漉ししてピュレにする。次にバター1かけを加えて火にかけ、水気をとばす。牛乳か生クリームを加えて濃さを調節する。

さやいんげん、カリフラワー等のように水分の多い野菜をピュレにする場合は、濃さを出すため、味のバランスのとれたでんぷん質の野菜のピュレを加えること。

野菜の「クレーム」にする場合は、でんぷん質の野菜のピュレではなく、濃く仕上げたソース・ベシャメルを加える。
\newpage
\href{未、原文対照チェック}{} \href{未、日本語表現校正}{}
\href{未、その他修正}{} \href{未、原稿最終校正}{}

\begin{Main}

\hypertarget{artichauts}{%
\subsection[アーティチョーク]{\texorpdfstring{アーティチョーク\footnote{artichaut
  (アルティショー)キク科の多年草で、草丈は1
  m程にもなる。フランス語としては、16世紀初頭には carchoffle あるいは
  artichault
  の綴りで記録されている。しばしば、カトリーヌ・ド・メディシスがイタリアからフランスに「紹介した」とか「もたらした」といわれるが、これは俗説であり、それ以前からフランスでも知られていたし、南フランスでは栽培されていた。実際のところ、カトリーヌ・ド・メディシスはフランスの王宮においてアーティチョークの料理が流行するきっかけ程度には普及に貢献したのだろう。16世紀末オリヴィエ・ド・セール『農業経営論』において既に、南フランスの気候を活かした周年栽培の方法が記されており、その方法論の基礎はこんにちでも変化していない。食材としては、主に花蕾を利用する。固く厚みのある花弁のような花萼に覆われており、ある程度成熟したものは花萼をすべてナイフで切り落して取り除き、皿のような形状にしたものを加熱したのちに、タルトレットのようにアパレイユを詰めるなどする。この基底部をfond
  d'artichaut
  (フォンダルティショー)と呼ぶ。やや若どりのものは花萼も下半分は加熱すれば柔らかいため、花蕾の上部は切り落して、半割りまたは四つ割りにして加熱調理する。これは
  coeur d'artichaut
  (クールダルティショー)と呼ばれる。さらにごく若どりのものは生食も可能であり、poivrade
  (ポワヴラード)と呼ばれる。ブルターニュ産のものが有名で、大きな花蕾を付ける品種が中心であり、南フランス産のものは比較的小ぶりで、紫色の品種が代表的。日本には明治時代に伝わり、何度も生産が試みられているが、一般野菜としての需要を喚起することが出来ずにいるため、現在も輸入品が中心。}}{アーティチョーク}}\label{artichauts}}

\index{artichaut@artichaut|(}
\index{あーていちよーく@アーティチョーク|(}

\end{Main}

\begin{recette}

\hypertarget{artichauts-a-la-barigoule}{%
\subsubsection{アーティチョーク・バリグール}\label{artichauts-a-la-barigoule}}

\frsub{Artichauts à la Barigoule}

\index{artichaut@artichautl!barigoule@--- à la Barigoule}
\index{barigoule@barigoule!artichauts@Artichauts à la ---}
\index{あーていちよーく@アーティチョーク!はりくーる@---・バリグール}
\index{はりくーる@バリグール!あーていちよーく@アーティチョーク・バリグール}

\index{artichaut@artichaut|)}
\index{あーていちよーく@アーティチョーク|)}

\end{recette}

\begin{main}

\hypertarget{asperges}{%
\subsection{アスパラガス}\label{asperges}}

\begin{frsecbenv}

Asperges

\end{frsecbenv}

アスパラガスは主に4種類\footnote{variété
  (ヴァリエテ)野菜についての場合通常は「品種」と訳すが、ここではバラエティ、種類くらいの意。なお、いわゆる「アスパラソバージュ」asperges
  sauvages
  (アスペルジュソヴァージュ)はまったくの別種であり、ここには含まれていない。}に分けられる。フランス産アスパラガスの典型的な品種、アルジャントゥイユ\footnote{Argenteuil
  (アルジョントゥイユ)。パリ近郊の地名。かつてここでアスパラガスの生産が盛んだったという。またこの地名を冠した品種が21
  世紀になった現在でも主流であることは事実。穂先がやや紫がかる傾向にあるが基本的には緑色の品種。}。グリーンアスパラガス\footnote{現在ではオランダの種苗会社が交配したF1品種が増えている。}。ジェノヴァ産紫アスパラガス\footnote{同上。ただしこの紫色は茹でると失なわれる。}、イタリア産アスパラガスの典型で繊細な風味だがややえぐ味がある。ベルギー産ホワイトアスパラガス\footnote{ホワイトアスパラガスは「品種」ではなく栽培方法が異なる(軟白の工程が入る)。理屈のうえではどんな品種であってもホワイトアスパラガスにすることは可能。秋のうちに地上部を切り落し、根株を中心にプラウ(鋤の一種)などを用いて土を盛り上げる。全体に大きなかまぼこのような格好になる。春になると、地中深くの根株から伸びてきたアスパラガスの茎は日光に当たっていないので軟白されている。それを地上に出る直前に収穫する。収穫は一定期間で終了させ、盛り上げた土を平らに戻し、緑の茎葉を茂らせて翌年のために根株を養成する。トラクタがプラウを曳けるようかなり条間を広くとる必要があり、単位面積あたりの収量は低い。かつては缶詰の材料として北海道で盛んに栽培されていたが、だんだん生産量が落ちている。近年日本ではハウス栽培で土盛りをせずトンネルに遮光率100%のシートで暗闇を作って栽培する方式が増えつつある。いずれもフランス料理においてあまり高い評価を得られていないのは、品種の選定と栽培方法に負うところが大きいだろう。}、これも繊細な風味だが、輸送による劣化がはげしい。

\end{main}


\href{未、原文対照チェック}{} \href{未、日本語表現校正}{}
\href{未、その他修正}{} \href{未、原稿最終校正}{}

\hypertarget{legumes-farineux-et-pates-alimentaires}{%
\chapter{XIII. 野菜料理・パスタなど}\label{legumes-farineux-et-pates-alimentaires}}

\frchap{Légumes --- Farineux et Pâtes alimentaires}

\hypertarget{serie-des-legumes}{%
\section{野菜料理}\label{serie-des-legumes}}

\begin{center}
\headfont\large 下拵えと注意事項\label{observations-sur-les-operations-preliminaires}

\normalfont\textit{Observation sur les Opérations préliminaires.}
\end{center}

\normalfont\normalsize

\hypertarget{blanchissage}{%
\subsection[湯がく・下茹で]{\texorpdfstring{湯がく・下茹で\footnote{blanchir(ブロンシール)。もともとの意味は「白くする」.食材を冷蔵保存出来なかった中世には肉類はいったん下茹でしてから調理するのが一般的であり,肉を下茹ですると表面が白くなることからこの用語が用いられるようになった.素材の種類によっては白く茹であげるために単なる湯ではなく「ブラン」を用いる場合もある.これは、水1ℓにスプーン1杯強の小麦粉,塩6gとスプーン2杯の溶いて沸かす.クローヴを刺した玉ねぎ1ヶとブーケガルニ,下茹でする素材,空気に素材が触れて変色するのを防ぐための脂を入れる.脂は,牛あるいは仔牛の生のケンネ脂を細かく刻んだもの.必要なら脂を事前に冷水にさらして血等の夾雑物がないようにしておくこと.
  (原注)
  野菜の下茹で用のブランには,ヴィネガーではなくレモン汁を用いたほうがいい(原書
  p.405).}}{湯がく・下茹で}}\label{blanchissage}}

この作業は2つの目的で行なう。第1に、例えばほうれんそう、プティポワ、さやいんげん等の一般的な葉物野菜では、完全に火を通すのが目的。たっぷりの湯で手早く茹で、クロロフィルすなわち葉緑素を失わないようにすること。第
2は、野菜に自然にあるえぐ味を消す目的。例えばキャベツ、セロリ、シコレ等。原則的に、新野菜は下茹でしない。下茹でで完全に火を通してしまう野菜については、1リットルあたり7gの塩を湯に加えること。

\hypertarget{rafraichissage}{%
\subsection[冷水にはなす]{\texorpdfstring{冷水にはなす\footnote{rafraîchir
  ラフレシール}}{冷水にはなす}}\label{rafraichissage}}

湯がいた後、冷水にとるのは、野菜をブレゼにする場合と、オペレーションの都合から事前に茹でておかなければならない場合のみ。ただし、後でバターやクリームで合える場合は、冷水にとると風味が失われる。

\hypertarget{cuisson-des-legumes-a-l-anglaise}{%
\subsection{アラングレーズ}\label{cuisson-des-legumes-a-l-anglaise}}

沸騰した湯で茹でるのみ。次によく水切りをして、さらに水気をとばす。深皿に盛りつけ、貝殻形のバターを添えて供する。味付けは食べ手がする。葉物野菜なら何でもこのイギリス風に調理、提供可能。

\hypertarget{cuisson-des-legumes-secs}{%
\subsection{乾燥豆}\label{cuisson-des-legumes-secs}}

乾燥豆を水でもどしておくのはよろしくない。その年に穫れた良質のものなら、水から弱火でゆっくり沸かして茹でればいい。あく取りをして香味野菜\footnote{ベシャメルソース1ℓに生クリーム2dlを加え,火にかけてへらで混ぜながら
  ¾ℓになるまで煮つめる.
  布で漉し,クレーム・ドゥーブル1½dlとレモン汁½個分を少しずつ加えていき、濃さを調節する(原書p.32)。}を加え、蓋をしてごく弱火で茹でる。

あまりにも古い豆や品質が劣るものは水でもどす。ただし、豆が膨れるのに必要な時間きっかり、すなわち1時間半から2時間とすること。

何時間も水につけておくと、発酵が始まってしまう。そうなると豆の組織が損なわれて使いものにならなくなってしまうことさえある。

\hypertarget{braisage-des-legumes}{%
\subsection{野菜のブレゼ}\label{braisage-des-legumes}}

野菜は事前に湯がいて冷水にとる。形を整える。鍋の底と周囲に豚背脂のシートを張り、野菜を入れる。上面を背脂で覆う。鍋に蓋をして弱火でかるく汗をかかせるように蒸し煮した後、ひたひたまで白いフォンを注ぐ。鍋に蓋をし、中温のオーヴンに入れる。火が通ったら、野菜の水気をきり、用途に合わせて形を整える。その後すぐに使う場合は、煮汁の浮き脂を取り除いて煮つめ、野菜とともにソテ鍋で保温する。事前に仕込んでおく場合には、鍋から皿あるいは専用の容器に移す。煮汁は浮き脂を取り除かずにそのまま加える。バターを塗った紙で覆ってストックする。

\hypertarget{sauce-des-legumes-braises}{%
\subsection{野菜のブレゼのソース}\label{sauce-des-legumes-braises}}

ブレゼの煮汁を煮詰め、浮き脂を丁寧に取り除いて使う。場合によってはグラスドヴィアンド、あるいは相応量のドゥミ・グラスを加える。どちらの場合にも、

ソースがまろやかになるようバターを加えて仕上げる。必要ならレモン汁数滴も加える。

\hypertarget{liaison-des-legumes-vert-au-beurre}{%
\subsection{葉野菜をバターであえる}\label{liaison-des-legumes-vert-au-beurre}}

茹でた野菜はしっかりと水をきっておく。味付けをしてバターを加え、鍋をあおるようにしてバターが野菜全体にまわるようにする。バターの風味を失なわないためには、「火から外した状態」でバターを加えること。

\hypertarget{liaison-des-legumes-a-la-creme}{%
\subsection{クリームであえる}\label{liaison-des-legumes-a-la-creme}}

この方法で調理する場合は、野菜に火を通す際、やや歯応えを残しておくこと。

しっかり水気をきり、野菜を鍋に入れる。沸かした生クリームを野菜が上に顔を出す程度に加える。

時々、よくかきまぜながら加熱する。

クリームがほぼすっかり煮詰まったら、バターとレモン汁少量を加える。必要なら、生クリームにソース・クレーム\footnote{blanchir(ブロンシール)。もともとの意味は「白くする」.食材を冷蔵保存出来なかった中世には肉類はいったん下茹でしてから調理するのが一般的であり,肉を下茹ですると表面が白くなることからこの用語が用いられるようになった.素材の種類によっては白く茹であげるために単なる湯ではなく「ブラン」を用いる場合もある.これは、水1ℓにスプーン1杯強の小麦粉,塩6gとスプーン2杯の溶いて沸かす.クローヴを刺した玉ねぎ1ヶとブーケガルニ,下茹でする素材,空気に素材が触れて変色するのを防ぐための脂を入れる.脂は,牛あるいは仔牛の生のケンネ脂を細かく刻んだもの.必要なら脂を事前に冷水にさらして血等の夾雑物がないようにしておくこと.
  (原注)
  野菜の下茹で用のブランには,ヴィネガーではなくレモン汁を用いたほうがいい(原書
  p.405).}を足してもいい。

\hypertarget{cremes-et-puree-de-legumes}{%
\subsection{野菜のクレームとピュレ}\label{cremes-et-puree-de-legumes}}

乾燥豆とでんぷん質の野菜は裏漉ししてピュレにする。次にバター1かけを加えて火にかけ、水気をとばす。牛乳か生クリームを加えて濃さを調節する。

さやいんげん、カリフラワー等のように水分の多い野菜をピュレにする場合は、濃さを出すため、味のバランスのとれたでんぷん質の野菜のピュレを加えること。

野菜の「クレーム」にする場合は、でんぷん質の野菜のピュレではなく、濃く仕上げたソース・ベシャメルを加える。
%
\href{未、原文対照チェック}{} \href{未、日本語表現校正}{}
\href{未、その他修正}{} \href{未、原稿最終校正}{}

\begin{Main}

\hypertarget{artichauts}{%
\subsection[アーティチョーク]{\texorpdfstring{アーティチョーク\footnote{artichaut
  (アルティショー)キク科の多年草で、草丈は1
  m程にもなる。フランス語としては、16世紀初頭には carchoffle あるいは
  artichault
  の綴りで記録されている。しばしば、カトリーヌ・ド・メディシスがイタリアからフランスに「紹介した」とか「もたらした」といわれるが、これは俗説であり、それ以前からフランスでも知られていたし、南フランスでは栽培されていた。実際のところ、カトリーヌ・ド・メディシスはフランスの王宮においてアーティチョークの料理が流行するきっかけ程度には普及に貢献したのだろう。16世紀末オリヴィエ・ド・セール『農業経営論』において既に、南フランスの気候を活かした周年栽培の方法が記されており、その方法論の基礎はこんにちでも変化していない。食材としては、主に花蕾を利用する。固く厚みのある花弁のような花萼に覆われており、ある程度成熟したものは花萼をすべてナイフで切り落して取り除き、皿のような形状にしたものを加熱したのちに、タルトレットのようにアパレイユを詰めるなどする。この基底部をfond
  d'artichaut
  (フォンダルティショー)と呼ぶ。やや若どりのものは花萼も下半分は加熱すれば柔らかいため、花蕾の上部は切り落して、半割りまたは四つ割りにして加熱調理する。これは
  coeur d'artichaut
  (クールダルティショー)と呼ばれる。さらにごく若どりのものは生食も可能であり、poivrade
  (ポワヴラード)と呼ばれる。ブルターニュ産のものが有名で、大きな花蕾を付ける品種が中心であり、南フランス産のものは比較的小ぶりで、紫色の品種が代表的。日本には明治時代に伝わり、何度も生産が試みられているが、一般野菜としての需要を喚起することが出来ずにいるため、現在も輸入品が中心。}}{アーティチョーク}}\label{artichauts}}

\index{artichaut@artichaut|(}
\index{あーていちよーく@アーティチョーク|(}

\end{Main}

\begin{recette}

\hypertarget{artichauts-a-la-barigoule}{%
\subsubsection{アーティチョーク・バリグール}\label{artichauts-a-la-barigoule}}

\frsub{Artichauts à la Barigoule}

\index{artichaut@artichautl!barigoule@--- à la Barigoule}
\index{barigoule@barigoule!artichauts@Artichauts à la ---}
\index{あーていちよーく@アーティチョーク!はりくーる@---・バリグール}
\index{はりくーる@バリグール!あーていちよーく@アーティチョーク・バリグール}

\index{artichaut@artichaut|)}
\index{あーていちよーく@アーティチョーク|)}

\end{recette}

\begin{main}

\hypertarget{asperges}{%
\subsection{アスパラガス}\label{asperges}}

\begin{frsecbenv}

Asperges

\end{frsecbenv}

アスパラガスは主に4種類\footnote{variété
  (ヴァリエテ)野菜についての場合通常は「品種」と訳すが、ここではバラエティ、種類くらいの意。なお、いわゆる「アスパラソバージュ」asperges
  sauvages
  (アスペルジュソヴァージュ)はまったくの別種であり、ここには含まれていない。}に分けられる。フランス産アスパラガスの典型的な品種、アルジャントゥイユ\footnote{Argenteuil
  (アルジョントゥイユ)。パリ近郊の地名。かつてここでアスパラガスの生産が盛んだったという。またこの地名を冠した品種が21
  世紀になった現在でも主流であることは事実。穂先がやや紫がかる傾向にあるが基本的には緑色の品種。}。グリーンアスパラガス\footnote{現在ではオランダの種苗会社が交配したF1品種が増えている。}。ジェノヴァ産紫アスパラガス\footnote{同上。ただしこの紫色は茹でると失なわれる。}、イタリア産アスパラガスの典型で繊細な風味だがややえぐ味がある。ベルギー産ホワイトアスパラガス\footnote{ホワイトアスパラガスは「品種」ではなく栽培方法が異なる(軟白の工程が入る)。理屈のうえではどんな品種であってもホワイトアスパラガスにすることは可能。秋のうちに地上部を切り落し、根株を中心にプラウ(鋤の一種)などを用いて土を盛り上げる。全体に大きなかまぼこのような格好になる。春になると、地中深くの根株から伸びてきたアスパラガスの茎は日光に当たっていないので軟白されている。それを地上に出る直前に収穫する。収穫は一定期間で終了させ、盛り上げた土を平らに戻し、緑の茎葉を茂らせて翌年のために根株を養成する。トラクタがプラウを曳けるようかなり条間を広くとる必要があり、単位面積あたりの収量は低い。かつては缶詰の材料として北海道で盛んに栽培されていたが、だんだん生産量が落ちている。近年日本ではハウス栽培で土盛りをせずトンネルに遮光率100%のシートで暗闇を作って栽培する方式が増えつつある。いずれもフランス料理においてあまり高い評価を得られていないのは、品種の選定と栽培方法に負うところが大きいだろう。}、これも繊細な風味だが、輸送による劣化がはげしい。

\end{main}

%\begin{main}

\hypertarget{legumes-farineux-et-pates-alimentaires}{%
\chapter{XIII. 野菜料理・パスタ}\label{legumes-farineux-et-pates-alimentaires}}

\begin{frchapenv}

Légumes --- Farineux et Pâtes alimentaires

\end{frchapenv}

\hypertarget{serie-des-legumes}{%
\section{野菜料理}\label{serie-des-legumes}}

\begin{frsecenv}

Série des Légumes

\end{frsecenv}

\begin{center}
\headfont\large 野菜の仕込みにおける注意事項\label{observations-sur-les-operations-preliminaires}

\normalfont\textit{Observation sur les Opérations préliminaires.}
\end{center}

\normalfont\normalsize

\hypertarget{blanchissage}{%
\subsection[下茹で]{\texorpdfstring{下茹で\footnote{blanchir(ブロンシール)。もともとの意味は「白くする」。古代ローマ時代から17世紀頃まで、肉類は最初に下茹でしてから調理するのが一般的だった。赤身肉を茹でると表面が白くなることからこの用語が用いられるようになった。野菜にかぎらず、素材によっては白く茹であげるために単なる湯ではなく「ブラン」を用いる場合もある。以下は本書、肉料理の概説部分にあるブランの要旨。水
  1
  Lに大さじ1杯強の小麦粉、塩6gとスプーン2杯の溶いて沸かす。クローヴを刺した玉ねぎ1個とブーケガルニ、下茹でする素材、さらに空気に素材が触れて変色するのを防ぐために獣脂を加える。脂は牛あるいは仔牛の生のケンネ脂を細かく刻んだものを使う。脂を事前に冷水にさらして血等の不純物が残っていないようにしておくこと。また、原注において、野菜の下茹で用のブランには,ヴィネガーではなくレモン汁を用いたほうがいいと述べてある(\protect\hyperlink{blanc-pour-viandes-et-certains-legumes}{肉およびある種の野菜に用いるブラン}
  参照。)}}{下茹で}}\label{blanchissage}}

\begin{frsecbenv}

Blanchissage

\end{frsecbenv}

この作業を行なう場合、2つのケースに分けられる。第1に、例えばほうれんそう、プチポワ、アリコヴェール等の一般的な青物野菜の場合、完全に火を通すのが目的。たっぷりの湯で手早く茹で、クロロフィルすなわち葉緑素を失わないよう注意すること。第2は、野菜に本来あるえぐ味を消す目的\footnote{いわゆる「アク」だが、例えばサヴォイキャベツの場合など、しっかり下茹でをしないと、後の調理の段階で変色することがある。厳冬期に霜に何度もあたったものなら下茹で時間は数分〜10分程度で済むが、それ以外の時期のものは2時間〜3時間の下茹でが必要(丸ごと下茹でする場合)。冬季の霜にあたった状態を人為的に作りだす、つまり冷凍庫に入れて水分が凍る際に膨張する力を利用して細胞壁を破壊すれば、下茹で時間は少なくて済む。シコレの場合、軟白されているものはそのまま生食可能なくらいアクも苦味も少ないので、ここでは軟白されていない緑のものを指している。セロリは品種によって風味やアク、苦味の強さが違うので注意。とりわけ日本ではコーネルという生食に適した品種の栽培が多いため、下茹ではあまりしっかり行なう必要もないだろう。逆に、サルシフィなどは変色を防ぐためにレモン果汁を用いるが、この変色はアクではなく、同じキク科の野菜であるレタス類の切り口が空気に触れて変色するのとまったく同じ現象。ただし、サルシフィは長時間加熱することで線維が柔らかく美味しくなるので、調理によっては充分に下茹でをしたほうがいい。また、ほうれんそうやブレットなどアカザ科の野菜にはシュウ酸が含まれていて結石の原因となるが、これも下茹ですることで容易にほとんどを除去出来るので下茹でが必須。いっぽう、プチポワのごく若どりのものについては新鮮なものに限られるが少量なら生食可能。冷凍品や大きく育った豆は下茹でが必須。アリコヴェールも同様に、わずかだがレクチンが含まれているので下茹で必須。これは他の豆類も同様なので注意。}。例えばキャベツ、セロリ、シコレ等。原則的に、若どりの野菜は下茹でしない。下茹でで完全に火を通してしまう野菜については、1リットルあたり7gの塩を湯に加えること。

\hypertarget{rafraichissage}{%
\subsection[冷水にはなす]{\texorpdfstring{冷水にはなす\footnote{rafraîchir
  (ラフレシール)。原義は、リフレッシュさせる。}}{冷水にはなす}}\label{rafraichissage}}

\begin{frsecbenv}

Rafraîchissage

\end{frsecbenv}

下茹で後に冷水にとるのは、野菜をブレゼにする場合と、オペレーションの都合から事前に茹でておかなければならない場合だけにすること。また、後でバターやクリームであえる場合、冷水にとると風味が負けてしまうので注意。

\hypertarget{cuisson-des-legumes-a-l-anglaise}{%
\subsection[アラングレーズ]{\texorpdfstring{アラングレーズ\footnote{à
  l'anglaise
  (アロングレーズ)。イギリス風、の意だが、常識的に考えて、イギリスにおいてのみ野菜をただ茹でるという調理法が一般的というわけではない。1907年に刊行された本書の英語版
  \emph{A guide to modern cookery}
  において、イギリス式パン粉衣については pané à l'anglaise, treated à
  l'anglaise として説明がなされているが
  (pp.70-71)、野菜類を茹ですることについての言及は見られない。つまり、この表現が通用したのはフランスにおいてのみ、ということになる。}}{アラングレーズ}}\label{cuisson-des-legumes-a-l-anglaise}}

\begin{frsecbenv}

Cuisson des légumes à l'anglaise

\end{frsecbenv}

沸騰した湯で茹でるだけでいい。次によく湯切りをして、さらに水気をとばす。野菜料理用の深皿に盛りつけ、貝殻形のバターを添えて供する。味付けは食べ手自身に行なっていただく。葉物野菜なら何でもこのアラングレーズで調理、提供可能。

\hypertarget{cuisson-des-legumes-secs}{%
\subsection[乾燥豆]{\texorpdfstring{乾燥豆\footnote{légume
  (レギューム)という語は古い時代には「豆類」を意味していた。その他の野菜は
  feuilles (フイユ、葉)とか racine
  (ラシーヌ、根=根菜)などと呼ばれることが多かった。légume
  が野菜の総称となったのは比較的新しい時代のことであり、現代でもイタリア語ではlégumes
  に対応する legumi
  (レグーミ)という語は「豆類」を意味する。なお、イタリア語での野菜の総称は
  ortaggi (オルタッジ、複数形)。}}{乾燥豆}}\label{cuisson-des-legumes-secs}}

\begin{frsecbenv}

Cuisson des légumes secs

\end{frsecbenv}

乾燥豆を水に漬けてもどすのはよろしくない。その年に穫れた良質のものなら、水から弱火でゆっくり沸かして茹でるだけでいい。あく取りをして香味野菜
\footnote{一般的にはクローヴを刺した玉ねぎと縦に四つ割りにしたにんじん、ブーケガルニ。}を加え、蓋をしてごく弱火で茹でる。

あまりにも古い豆や品質が劣るものはあらかじめ水でもどしてもいい。ただし、豆が膨れるのに必要な時間きっかり、つまり1時間半から2時間程度に留めること。

何時間も水につけておくと、発酵が始まってしまう。そうなる美味さもほとんどなくなってしまうし、豆の組織が損なわれて使いものにならなくなってしまうことさえある。

\hypertarget{braisage-des-legumes}{%
\subsection[野菜のブレゼ]{\texorpdfstring{野菜のブレゼ\footnote{よく誤解されがちなものなので注意したい。}}{野菜のブレゼ}}\label{braisage-des-legumes}}

\begin{frsecbenv}

Braisage des légumes

\end{frsecbenv}

野菜は事前に湯がいて冷水にとり、その後に余分な葉などを切り落して成形する。\protect\hyperlink{braisage}{肉料理のブレゼ}と同様に、鍋の底と周囲に豚背脂のシートを張り、野菜を入れる。上面を背脂のシートで覆う。鍋に蓋をして弱火でかるく汗をかかせるように蒸し煮\footnote{suer
  (スュエ)日本ではむしろシュエと呼ぶ現場が多い。}した後、材料がかぶる高さまで\protect\hyperlink{fonds-blanc-ordinaire}{白いフォン}を注ぐ。鍋に蓋をし、中温のオーヴンに入れて火を通す。

火が通ったら、野菜を取り出して水気をきり、料理での用途に合わせて成形する。その後すぐに使う場合は、煮汁の浮き脂を取り除いて\footnote{dégraisser
  (デグレセ)。}から煮詰め、野菜とともにソテー鍋で保温する。事前に仕込んでおく場合には、鍋から皿あるいは専用の陶製の器に広げる。煮汁は浮き脂を取り除かずにそのまま加える。バターを塗った紙で覆ってストックしておく。

\hypertarget{sauce-des-legumes-braises}{%
\subsection{野菜のブレゼのソース}\label{sauce-des-legumes-braises}}

\begin{frsecbenv}

Sauce des légumes braisés

\end{frsecbenv}

ブレゼの煮汁を煮詰め、浮き脂を丁寧に取り除いて使う。場合によっては\protect\hyperlink{glace-de-viande}{グラスドヴィアンド}、あるいは相応量の\protect\hyperlink{sauce-demi-glace}{ドゥミグラス}を加える。どちらの場合も、バターを加えてソースをまろやかに仕上げる。必要ならレモン果汁をほんの少量加える。

\hypertarget{liaison-des-legumes-vert-au-beurre}{%
\subsection{青物野菜をバターであえる}\label{liaison-des-legumes-vert-au-beurre}}

\begin{frsecbenv}

Liaison des légumes verts au beurre

\end{frsecbenv}

茹でた野菜はしっかりと水をきっておく。味付けをしてバターを加え、鍋をあおるようにしてバターが野菜全体にまわるようにする。バターを加えるのは\textbf{
火から外した状態}で行なうこと。そうすればバターの風味が失なわれずに仕上がる。

\hypertarget{liaison-des-legumes-a-la-creme}{%
\subsection{生クリームであえる}\label{liaison-des-legumes-a-la-creme}}

\begin{frsecbenv}

Liaison des légumes à la crème

\end{frsecbenv}

この方法で調理する場合は、野菜をやや固めの状態になるよう下茹でしておくこと。よく水気をきってから野菜を鍋に入れる。沸かした生クリームを野菜が上に顔を出す程度に加える。時々、ヘラでゆっくりかきまぜ\footnote{vanner
  (ヴァネ)。}ながら火入れを仕上げる。

クリームがほぼすっかり煮詰まったら、バターとレモン果汁少々を加える。必要なら、生クリームに\protect\hyperlink{sauce-creme}{ソース・クレーム}を少量加えてもいい。

\hypertarget{cremes-et-puree-de-legumes}{%
\subsection{野菜のクレームとピュレ}\label{cremes-et-puree-de-legumes}}

\begin{frsecbenv}

Crèmes et Purées de légumes

\end{frsecbenv}

乾燥豆とでんぷん質の野菜は裏漉ししてピュレにする。次にバター1片を加えて火にかけ、水気をとばす。牛乳か生クリームを加えて濃さを調節して仕上げる。

アリコヴェール、カリフラワー等のように水分の多い野菜をピュレにする場合は、濃度を出すため、その野菜との相性のいいでんぷん質の野菜のピュレを加えること。

野菜の\textbf{クレーム}にする場合は、でんぷん質の野菜のピュレではなく、しっかりした味で濃く仕上げた\protect\hyperlink{sauce-bechamel}{ベシャメルソース}を加える。

\end{main}




\appendix

%%% 索引ページ出力

\backmatter

\renewcommand{\indexname}{ソース名から料理を探す}

\printindex[src][ソース名から料理を探す]

\renewcommand{\indexname}{総索引}
\printindex





\end{document}



%% Local Variables:
%% TeX-engine: luatex
%% End:
