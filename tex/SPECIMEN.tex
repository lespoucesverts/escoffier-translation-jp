%%% Preamble読み込み
\documentclass[twoside,14Q,a4paper,openany]{ltjsbook}

\usepackage{amsmath}
  \let\equation\gather
  \let\endequation\endgather
\usepackage{amssymb}
\usepackage[no-math]{fontspec}
\usepackage{geometry}
\usepackage{luaotfload}
\usepackage{graphicx}

\usepackage{setspace}


%%%%%%%%% hyperref %%%%%%%%%%%%%
\usepackage{refcount}
\usepackage[unicode=true,hyperfootnotes=false,pageanchor]{hyperref}
\hypersetup{hyperindex=false,%
             breaklinks=true,%
             bookmarks=true,%
             pdfauthor={五島 学},%
             pdftitle={エスコフィエ『料理の手引き』全注解},%
             colorlinks=false%true,%
             %colorlinks=true,%
             citecolor=blue,%
             urlcolor=cyan,%
             linkcolor=magenta,%
             bookmarksdepth=subsubsection,%
             pdfborder={0 0 0},%
             hyperfootnotes=false,%
             plainpages=false,
             }
\urlstyle{same}




%% 欧文フォント設定
% Libertine/Biolinum
\setmainfont[Ligatures=Historic,Scale=1.0]{Linux Libertine O}
\setsansfont[Ligatures=TeX, Scale=MatchLowercase]{Linux Biolinum O} 
%\usepackage{libertine}
\usepackage{unicode-math}
\setmathfont[Scale=1.2]{libertinusmath-regular.otf}
%\unimathsetup{math-style=ISO,bold-style=ISO}
%\setmathfont{xits-math.otf}
%\setmathfont{xits-math.otf}[range={cal,bfcal},StylisticSet=1]




%% Garamond
%\usepackage{ebgaramond-maths}
%\setmainfont[Ligatures=Historic,Scale=1.1]{EB Garamond}%fontspecによるフォント設定

%\usepackage{qpalatin}%palatino

%\setmainfont[Ligatures=Historic,Scale=MatchLowercase]{Tex Gyre Schola}
%\setmainfont[Ligatures=Historic,Scale=MatchLowercase]{Tex Gyre Pagella}
%\setsansfont[Scale=MatchLowercase]{TeX Gyre Heros}  % \sffamily のフォント
%\setsansfont[Scale=MatchLowercase]{TeX Gyre Adventor}  % \sffamily のフォント

%\setmonofont[Scale=MatchLowercase]{Inconsolata}       % \ttfamily のフォント

%\usepackage[cmintegrals,cmbraces]{newtxmath}%数式フォント

\usepackage{luatexja}
\usepackage{luatexja-fontspec}
%\ltjdefcharrange{8}{"2000-"2013, "2015-"2025, "2027-"203A, "203C-"206F}
%\ltjsetparameter{jacharrange={-2, +8}}
\usepackage{luatexja-ruby}

%%%%和文フォント設定
%\usepackage[sourcehan,bold,jis2004,expert,deluxe]{luatexja-preset}%Adobe源ノ明朝、ゴチ
%\usepackage[hiragino-pron,bold,jis2004,expert,deluxe]{luatexja-preset}
%\usepackage[yu-osx,expert,jis2004,bold]{luatexja-preset}
%\usepackage[moga-mogo-ex,bold]{luatexja-preset}

\newopentypefeature{PKana}{On}{pkna} % "PKana" and "On" can be arbitrary string
%%%%明朝にIPAexMincho、ゴチ(太字)にMoboGoBを使う設定。和文カナプロプーショナル使用可能だが読みづらくなる。
\setmainjfont[%
     %YokoFeatures={JFM=prop,PKana=On},%
     %CharacterWidth=AlternateProportional,%
%    CharacterWidth=Proportional,%Mogo, IPAExMinchoには不可
     %Kerning=On,%
     BoldFont={ MoboGoB },%
     ItalicFont={ MoboGoB },%
     BoldItalicFont={ MoboGoExB }%
     % ]{ MogaHMin }
     ]{ IPAExMincho }
     % ]{ IPAmjMincho }
\setsansjfont[%
     %YokoFeatures={JFM=prop,PKana=On},%
     %CharacterWidth=AlternateProportional,%
     % CharacterWidth=Proportional,%Mobo, IPAExGOthicには不可
     %Kerning=On,
     BoldFont={ MoboGoB },%
     ItalicFont={ MoboGoB },%
     BoldItalicFont={ MoboGoExB }%
     % ]{ MoboGo}
     ]{ IPAExGothic }
% %  %%%% 和文仮名プロプーショナルここまで
% %\ltjsetparameter{jacharrange={-2}}%キリル文字%引数に-3を付けるとギリシア文字も可能になるが、%三点リーダーも欧文化されてしまうので注意%


\renewcommand{\bfdefault}{bx}%和文ボールドを有効にする
\renewcommand{\headfont}{\gtfamily\sffamily\bfseries}%和文ボールドを有効にする
%\addfontfeature{Fractions=On}


\defaultfontfeatures[\rmfamily]{Scale=1.2}%効いていない様子
\defaultjfontfeatures{Scale=0.92487}%和文フォントのサイズ調整。デフォルトは 0.962212 倍%ltjsclassesでは不要?
%\defaultjfontfeatures{Scale=0.962212}
%\usepackage{libertineotf}%linux libertine font %ギリシア語含む
%\usepackage[T1]{fontenc}
%\usepackage[full]{textcomp}
%\usepackage[osfI,scaled=1.0]{garamondx}
%\usepackage{tgheros,tgcursor}
%\usepackage[garamondx]{newtxmath}

\usepackage{layout}

%% レイアウト調整(A4Paper,14Q,twoside,ltjsbook.cls) 
%%
\setlength{\hoffset}{0\zw}
\setlength{\oddsidemargin}{0\zw}%タブレット前提の中央配置
\setlength{\evensidemargin}{\oddsidemargin}
% \setlength{\oddsidemargin}{1\zw}%製本時に右ページのみをオフセット
%\setlength{\evensidemargin}{0pt}%
\setlength{\fullwidth}{45\zw}
\setlength{\textwidth}{45\zw}%%ltjsclassesのみ有効
%\setlength{\fullwidth}{159mm}
%\setlength{\textwidth}{159mm}
\setlength{\marginparsep}{0pt}
\setlength{\marginparwidth}{0pt}
\setlength{\footskip}{0pt}
\setlength{\voffset}{-17mm}
\setlength{\textheight}{265mm}
\setlength{\parskip}{0pt}
\setlength{\parindent}{0pt}

%%%% b5j%%%%%%
% \setlength{\hoffset}{-10mm}
% \setlength{\oddsidemargin}{0mm}
% \setlength{\evensidemargin}{0mm}
% %\setlength{\textwidth}{\fullwidth}%%ltjsclassesのみ有効
% \setlength{\fullwidth}{145mm}
% \setlength{\textwidth}{145mm}
% \setlength{\marginparsep}{0pt}
% \setlength{\marginparwidth}{0pt}
% \setlength{\footskip}{0pt}
% \setlength{\voffset}{-10mm}
% \setlength{\textheight}{225mm}
% \setlength{\parskip}{0pt}
%%%ベースライン調整
%\ltjsetparameter{yjabaselineshift=0pt,yalbaselineshift=-.75pt}


%\usepackage{fancyhdr}






%文字サイズ、見出しなどの再定義
\makeatletter
%\renewcommand{\large}{\jsc@setfontsize\large\@xipt{14}}
%\renewcommand{\Large}{\jsc@setfontsize\Large{13}{15}}

\newcommand{\medlarge}{\fontsize{11}{13}\selectfont}
\newcommand{\medsmall}{\fontsize{9.23}{9.5}\selectfont}
\newcommand{\twelveq}{\jsc@setfontsize\twelveq{9.230769}{9.75}\selectfont}
\newcommand{\fourteenq}{\jsc@setfontsize\fourteenq{10.7692}{13}\selectfont}
\newcommand{\fifteenq}{\jsc@setfontsize\fifteenq{11.53846}{14}\selectfont}

\renewcommand{\chapter}{%
  \if@openleft\cleardoublepage\else
  \if@openright\cleardoublepage\else\clearpage\fi\fi
  \plainifnotempty % 元: \thispagestyle{plain}
  \global\@topnum\z@
  \if@english \@afterindentfalse \else \@afterindenttrue \fi
  \secdef
    {\@omit@numberfalse\@chapter}%
    {\@omit@numbertrue\@schapter}}
\def\@chapter[#1]#2{%
  \ifnum \c@secnumdepth >\m@ne
    \if@mainmatter
      \refstepcounter{chapter}%
      \typeout{\@chapapp\thechapter\@chappos}%
      \addcontentsline{toc}{chapter}%
        {\protect\numberline
        % {\if@english\thechapter\else\@chapapp\thechapter\@chappos\fi}%
        {\@chapapp\thechapter\@chappos}%
        #1}%
    \else\addcontentsline{toc}{chapter}{#1}\fi
  \else
    \addcontentsline{toc}{chapter}{#1}%
  \fi
  \chaptermark{#1}%
  \addtocontents{lof}{\protect\addvspace{10\jsc@mpt}}%
  \addtocontents{lot}{\protect\addvspace{10\jsc@mpt}}%
  \if@twocolumn
    \@topnewpage[\@makechapterhead{#2}]%
  \else
    \@makechapterhead{#2}%
    \@afterheading
  \fi}
\def\@makechapterhead#1{%
  \vspace*{0\Cvs}% 欧文は50pt
  {\parindent \z@ \centering \normalfont
    \ifnum \c@secnumdepth >\m@ne
      \if@mainmatter
        \huge\headfont \@chapapp\thechapter\@chappos%変更
        \par\nobreak
        \vskip \Cvs % 欧文は20pt
      \fi
    \fi
    \interlinepenalty\@M
    \huge \headfont #1\par\nobreak
    \vskip 1\Cvs}} % 欧文は40pt%変更

\renewcommand{\section}{%
    \if@slide\clearpage\fi
    \@startsection{section}{1}{\z@}%
    {\Cvs \@plus.5\Cdp \@minus.2\Cdp}% 前アキ
    % {.5\Cvs \@plus.3\Cdp}% 後アキ
    {.5\Cvs}
    {\normalfont\Large\headfont\bfseries\centering}}%変更

\renewcommand{\subsection}{\@startsection{subsection}{2}{\z@}%
    {\Cvs \@plus.5\Cdp \@minus.2\Cdp}% 前アキ
    % {.5\Cvs \@plus.3\Cdp}% 後アキ
    {.5\Cvs}
  %  {\normalfont\large\headfont\bfseries\centering}} %変更
    {\normalfont\large\headfont\centering}} %変更

\renewcommand{\subsubsection}{\@startsection{subsubsection}{3}{\z@}%
  % {0\Cvs \@plus.8\Cdp \@minus.6\Cdp}%変更
    {1sp \@plus.5\Cdp \@minus.5\Cdp}%変更
    {\if@slide .5\Cvs \@plus.3\Cdp \else \z@ \fi}%
    % {\normalfont\medlarge\headfont\leftskip -1\zw}}
    {\normalfont\medlarge\headfont\leftskip -1\zw}}

\renewcommand{\paragraph}{\@startsection{paragraph}{4}{\z@}%
    {0.5\Cvs \@plus.5\Cdp \@minus.2\Cdp}%
    % {\if@slide .5\Cvs \@plus.3\Cdp \else -1\zw\fi}% 改行せず 1\zw のアキ
    {1sp}%後アキ
    {\normalfont\normalsize\headfont}}
\renewcommand{\subparagraph}{\@startsection{subparagraph}{5}{\z@}%
    {\z@}{\if@slide .5\Cvs \@plus.3\Cdp \else -.5\zw\fi}%
    {\normalfont\normalsize\headfont\hskip-.5\zw\noindent}}  

\newcommand{\frchap}[1]{\vspace*{-2ex}%
 \begin{center}\normalfont\headfont\LARGE\setstretch{0.8}
 \scshape#1\normalfont\normalsize
\end{center}\vspace{0.5\zw}\setstretch{1.0}}

\newcommand{\frsec}[1]{\vspace*{-2ex}%
 \begin{center}\normalfont\headfont\large\setstretch{0.8}
 \scshape#1\normalfont\normalsize
\end{center}\vspace{0.5\zw}\setstretch{1.0}}
  
\newcommand{\frsecb}[1]{\vspace*{-2ex}%
\begin{center}\normalfont\headfont\medlarge\setstretch{0.8}%
  \hspace{1em}\scshape#1\normalfont\normalsize%
\end{center}\vspace{0.5\zw}\setstretch{1.0}}

%\newcounter{frsub}[subsubsection]
%\newcommand{\frsub}{\@startsection{frsub}{6}{\z@}%
%  {1sp}{1sp}%
%  {\normalfont\normalsize\bfseries\baselineskip-.8ex\leftskip-1\zw}}
%\let\frsubmark\@gobble
%\newcommand*{\l@frsub}{%
%          \@tempdima\jsc@tocl@width \advance\@tempdima 16.183\zw
%          \@dottedtocline{7}{\@tempdima}{6.5\zw}}
%\renewcommand{\thefrsub}{6}
%\let\frsub\paragraph

\newenvironment{frsubenv}{\begin{spacing}{0.2}\setlength{\leftskip}{-1\zw}\bfseries}{\end{spacing}\normalfont\normalsize\setlength{\leftskip}{0pt}}
\newcommand{\frsub}[1]{\begin{frsubenv}#1\end{frsubenv}\par\vspace{1.1\zw}}

%\newcommand{\frsub}[1]{\vskip -.8ex\hskip -1\zw\textbf{#1}\leftskip0pt}
%\newcommand{\frsub}{\@startsection{frsub}{6}{\z@}%
%   {-1\zw}% 改行せず 1\zw のアキ
%   {-1\zw}%後アキ     
%   {\normalfont\normalsize\bfseries\leftskip -1\zw\baselineskip -.5ex}}%normalsizeから変更
%\newcommand*{\l@frsub}{%
%          \@tempdima\jsc@tocl@width \advance\@tempdima 16.183\zw
%          \@dottedtocline{5}{\@tempdima}{6.5\zw}}


%%%%%%%%%レシピと本文%%%%%%%%%%%%
\usepackage{multicol}
\setlength{\columnsep}{3\zw}

%%% 本文
\newenvironment{Main}{}{}
%%% レシピ
% \setlength{\columnwidth}{24\zw}
%本文ヨリ小%\small
%\newenvironment{recette}{\setlength{\parindent}{0pt}\begin{small}\begin{spaceing}{0.8}\begin{multicols}{2}}{\end{multicols}\end{spacing}\end{small}}
%本文やや小%\medsmall
%\newenvironment{recette}{\setlength{\parindent}{0pt}\begin{medsmall}\begin{spacing}{0.75}\begin{multicols}{2}}{\end{multicols}\end{spacing}\end{medsmall}}
%本文ナミ(無指定)
\newenvironment{recette}{\setlength{\parindent}{0pt}\begin{spacing}{0.8}\begin{multicols}{2}\setlength\topskip{.8\baselineskip}}{\end{multicols}\end{spacing}}


\makeatother


%%% 脚注番号のページ毎のリセットと脚注位置の調整
%\renewcommand{\footnotesize}{\small}

\makeatletter

\usepackage[bottom,perpage,stable]{footmisc}%
%\setlength{\skip\footins}{4mm plus 4mm}
%\usepackage{footnpag}
\renewcommand\@makefntext[1]{%
  \advance\leftskip 0\zw
  \parindent 1\zw
  \noindent
  \llap{\@thefnmark\hskip0.5\zw}#1}


\let\footnotes@ve=\footnote
\def\footnote{\inhibitglue\footnotes@ve}
\let\footnotemarks@ve=\footnotemark
%\def\footnotemark{\inhibitglue\footnotemarks@ve}
\renewcommand{\footnotemark}{\footnotemarks@ve}%変更
% %\def\thefootnote{\ifnum\c@footnote>\z@\leavevmode\lower.5ex\hbox{(}\@arabic\c@footnote\hbox{)}\fi}
\renewcommand{\thefootnote}{\ifnum\c@footnote>\z@\leavevmode\hbox{}\@arabic\c@footnote\hbox{)}\fi}
%\makeatletter
% \@addtoreset{footnote}{page}
% \makeatother
%\usepackage{dblfnote}
%\usepackage[bottom,perpage]{footmisc}

\makeatother

%subsubsectionに連番をつける
%\usepackage{remreset}

\renewcommand{\thechapter}{}
\renewcommand{\thesection}{\hskip-1\zw}
\renewcommand{\thesubsection}{}
\renewcommand{\thesubsubsection}{}
\renewcommand{\theparagraph}{}


%\makeatletter
%\@removefromreset{subsubsection}{subsection}
%\def\thesubsubsection{\arabic{subsubsection}.}
%\newcounter{rnumber}
%\renewcommand{\thernumber}{\refstepcounter{rnumber} }

\renewcommand{\prepartname}{\if@english Part~\else {}\fi}
\renewcommand{\postpartname}{\if@english\else {}\fi}
\renewcommand{\prechaptername}{\if@english Chapter~\else {}\fi}
\renewcommand{\postchaptername}{\if@english\else {}\fi}
\renewcommand{\presectionname}{}%  第
\renewcommand{\postsectionname}{}% 節

%リスト環境
\def\tightlist{\itemsep1pt\parskip0pt\parsep0pt}%pandoc対策

\makeatletter
  \parsep   = 0pt
  \labelsep = .5\zw
  \def\@listi{%
     \leftmargin = 0pt \rightmargin = 0pt
     \labelwidth\leftmargin \advance\labelwidth-\labelsep
     \topsep     = 0pt%\baselineskip
     %\topsep -0.1\baselineskip \@plus 0\baselineskip \@minus 0.1 \baselineskip
     \partopsep  = 0pt \itemsep       = 0pt
     \itemindent = -.5\zw \listparindent = 0\zw}
  \let\@listI\@listi
  \@listi
  \def\@listii{%
     \leftmargin = 1.8\zw \rightmargin = 0pt
     \labelwidth\leftmargin \advance\labelwidth-\labelsep
     \topsep     = 0pt \partopsep     = 0pt \itemsep   = 0pt
     \itemindent = 0pt \listparindent = 1\zw}
  \let\@listiii\@listii
  \let\@listiv\@listii
  \let\@listv\@listii
  \let\@listvi\@listii
\makeatother








%% %%%%%%行取りマクロ
% \makeatletter
% \ifx\Cht\undefined
%  \newdimen\Cht\newdimen\Cdp
%  \setbox0\hbox{\char\jis"2121}\Cht=\ht0\Cdp=\dp0\fi
% \catcode`@=11
% \long\def\linespace#1#2{\par\noindent
%   \dimen@=\baselineskip
%   \multiply\dimen@ #1\advance\dimen@-\baselineskip
%   \advance\dimen@-\Cht\advance\dimen@\Cdp
%   \setbox0\vbox{\noindent #2}%
%   \advance\dimen@\ht0\advance\dimen@-\dp0%
%   \vtop to\z@{\hbox{\vrule width\z@ height\Cht depth\z@
%    \raise-.5\dimen@\hbox{\box0}}\vss}%
%   \dimen@=\baselineskip
%   \multiply\dimen@ #1\advance\dimen@-2\baselineskip
%   \par\nobreak\vskip\dimen@
%   \hbox{\vrule width\z@ height\Cht depth\z@}\vskip\z@}
% \catcode`@=12
% \setlength{\parskip}{0pt}
% \setlength{\topskip}{\Cht}
% \setlength{\textheight}{43\baselineskip}
% \addtolength{\textheight}{1\zh}
% \makeatother
 
%%%%%%%%%%%%失敗%%%%%%%%%%%%
%\let\formule\subsubsection
%\renewcommand{\subsubsection}[1]{\linespace{1}{\formule#1}}
%%%%%%%%%%%%失敗%%%%%%%%%%%%






% PDF/X-1a
% \usepackage[x-1a]{pdfx}
% \Keywords{pdfTeX\sep PDF/X-1a\sep PDF/A-b}
% \Title{Sample LaTeX input file}
% \Author{LaTeX project team}
% \Org{TeX Users Group}
% \pdfcompresslevel=0
%\usepackage[cmyk]{xcolor}

%biblatex
%\usepackage[notes,strict,backend=biber,autolang=other,%
%                   bibencoding=inputenc,autocite=footnote]{biblatex-chicago}
%\addbibresource{hist-agri.bib}
\let\cite=\autocite

% % % % 
\date{}



%%%%インデックス準備

\usepackage{makeidx}
\usepackage{index}
\makeindex

\newindex{src}{idx1}{ind1}{ソース別料理索引} 


\makeatletter
\renewenvironment{theindex}{% 索引を3段組で出力する環境
  \if@twocolumn
      \onecolumn\@restonecolfalse
    \else
      \clearpage\@restonecoltrue
    \fi
    \columnseprule.4pt \columnsep 2\zw
    \ifx\multicols\@undefined
      \twocolumn[\@makeschapterhead{\indexname}%
      \addcontentsline{toc}{chapter}{\indexname}]%変更点
    \else
      \ifdim\textwidth<\fullwidth
        \setlength{\evensidemargin}{\oddsidemargin}
        \setlength{\textwidth}{\fullwidth}
        \setlength{\linewidth}{\fullwidth}
        \begin{multicols}{3}[\chapter*{\indexname}
	\addcontentsline{toc}{chapter}{\indexname}]%変更点%
      \else
        \begin{multicols}{3}[\chapter*{\indexname}
	\addcontentsline{toc}{chapter}{\indexname}]%変更点%
      \fi
    \fi
    \@mkboth{\indexname}{\indexname}%
    \plainifnotempty % \thispagestyle{plain}
    \parindent\z@
    \parskip\z@ \@plus .3\p@\relax
    \let\item\@idxitem
    \raggedright
    \footnotesize\narrowbaselines
  }{
    \ifx\multicols\@undefined
      \if@restonecol\onecolumn\fi
    \else
      \end{multicols}
      \fi
      \clearpage
    }
\makeatother


%%%% 本文中の参照ページ番号表示 %%%%%%%

\makeatletter

%\AtBeginDocument{%
%  \DeclareRobustCommand\ref{\@ifstar\@refstar\@refstar}%
%  \DeclareRobustCommand\pageref{\@ifstar\@pagerefstar\@pagerefstar}}
\let\orig@Hy@EveryPageAnchor\Hy@EveryPageAnchor
\def\Hy@EveryPageAnchor{%
    \begingroup
    \hypersetup{pdfview=Fit}%
    \orig@Hy@EveryPageAnchor
    \endgroup
  }
  \usepackage{etoolbox}
\if@mainmatter{\let\myhyperlink\hyperlink%
\renewcommand{\hyperlink}[2]{\myhyperlink{#1}{#2} [p.\getpagerefnumber{#1}{}] }}
  \AtBeginEnvironment{recette}{%
\let\myhyperlink\hyperlink%
\renewcommand{\hyperlink}[2]{\myhyperlink{#1}{#2} [p.\getpagerefnumber{#1}{}] }}
  \AtBeginEnvironment{Main}{%
\let\myhyperlink\hyperlink%
\renewcommand{\hyperlink}[2]{\myhyperlink{#1}{#2} [p.\getpagerefnumber{#1}{}] }}
% \if@mainmatter{\let\myhyperlink\hyperlink%
% \renewcommand{\hyperlink}[2]{\myhyperlink{#1}{#2} {\ltjsetparameter{yjabaselineshift=0pt,yalbaselineshift=-.75pt}\footnotesize [p.\getpagerefnumber{#1}{}]}}}
%   \AtBeginEnvironment{recette}{%
% \let\myhyperlink\hyperlink%
% \renewcommand{\hyperlink}[2]{\myhyperlink{#1}{#2} {\ltjsetparameter{yjabaselineshift=0pt,yalbaselineshift=-.75pt}\footnotesize [p.\getpagerefnumber{#1}{}]}}}
%   \AtBeginEnvironment{Main}{%
% \let\myhyperlink\hyperlink%
% \renewcommand{\hyperlink}[2]{\myhyperlink{#1}{#2} {\ltjsetparameter{yjabaselineshift=0pt,yalbaselineshift=-.75pt}\footnotesize [p.\getpagerefnumber{#1}{}]}}}


% \def\@@wrindex#1|#2|#3\\{%
%  \if@filesw
%  \ifx\\#2\\%
%   \protected@write\@indexfile{}{% 
%     %\string\indexentry{#1}{\thepage}%%%改変部分。もとは{#1|hyperpage}{\thepage}%
%     \string\indexentry{#1|hyperpage}{\thepage}%%オリジナル%
%         }%
%         \else
%           \HyInd@@@wrindex{#1}#2\\%
%         \fi
%       \fi
%       \endgroup
%       \@esphack
%     }

\makeatother



%%%%% Obsolete Reference Page Numbers 

%\newcommand{\pref}[1]{[p.\pageref{#1}]}
%\newcommand{\pref}[1]{}







%%%% pandoc が三点リーダーを勝手に変える対策
\renewcommand{\ldots}{\noindent…}
%%%%%下線
\usepackage{umoline}
\setlength{\UnderlineDepth}{2pt}
\let\ul\Underline

\newcommand{\maeaki}{}%使用しないので無効化
\newcommand{\atoaki}{\vspace{1.25mm}}
%%分数の表記Obsolete
\usepackage{xfrac}
\let\frac\sfrac
\newcommand{\undemi}{\hspace{.25\zw}$\sfrac{1}{2}$}
\newcommand{\untiers}{\hspace{.25\zw}$\sfrac{1}{3}$}
\newcommand{\deuxtiers}{\hspace{.25\zw}$\sfrac{2}{3}$}
\newcommand{\unquart}{\hspace{.25\zw}$\sfrac{1}{4}$}
\newcommand{\troisquarts}{\hspace{.25\zw}$\sfrac{3}{4}$}
\newcommand{\quatrequatrieme}{\hspace{.25\zw}$\sfrac{4}{4$}}
\newcommand{\uncinquieme}{\hspace{.25\zw}$\sfrac{1}{5}$}
\newcommand{\deuxcinquiemes}{\hspace{.25\zw}$\sfrac{2}{5}$}
\newcommand{\troiscinquiemes}{\hspace{.25\zw}$\sfrac{3}{5}$}
\newcommand{\quatrecinquiemes}{\hspace{.25\zw}$\sfrac{4}{5}$}
\newcommand{\unsixieme}{\hspace{.25\zw}$\sfrac{1}{6}$}
\newcommand{\cinqsixiemes}{\hspace{.25\zw}$\sfrac{5}{6}$}
\newcommand{\quatrequart}{\hspace{.25\zw}$\sfrac{4}{4}$}

\makeatletter
\def\ps@headings{%
  \let\@oddfoot\@empty
  \let\@evenfoot\@empty
  \def\@evenhead{%
    \if@mparswitch \hss \fi
    \underline{\hbox to \fullwidth{\ltjsetparameter{autoxspacing={true}}
%      \textbf{\thepage}\hfil\leftmark}}%
       \normalfont\thepage\hfill\scshape\small\leftmark\normalfont}}%
    \if@mparswitch\else \hss \fi}%
  \def\@oddhead{\underline{\hbox to \fullwidth{\ltjsetparameter{autoxspacing={true}}
        {\if@twoside\scshape\small\rightmark\else\scshape\small\leftmark\fi}\hfil\thepage\normalfont}}\hss}%
  \let\@mkboth\markboth
  \def\chaptermark##1{\markboth{%
    \ifnum \c@secnumdepth >\m@ne
      \if@mainmatter
        \if@omit@number\else
          \@chapapp\thechapter\@chappos\hskip1\zw
        \fi
      \fi
    \fi
    ##1}{}}%
  \def\sectionmark##1{\markright{%
%    \ifnum \c@secnumdepth >\z@ \thesection \hskip1\zw\fi
    \ifnum \c@secnumdepth >\z@ \thesection \hskip-1\zw\fi
    ##1}}}%
\makeatother

\makeatletter
%%%%%%%% Lua GC
\patchcmd\@outputpage{\stepcounter{page}}{%
  \directlua{%
	if jit then
      local k = collectgarbage("count")
      if k>900000 then 
        collectgarbage("collect")
        texio.write_nl("term and log", "GC: ", math.floor(k), math.floor(collectgarbage("count")))
      end
	end
  }%
  \stepcounter{page}%
}{}{}
\makeatother


%\usepackage{vgrid}% here only to help visualize the problem

%\renewcommand{\footnote}[1]{}%単に注釈なしバージョンを作るにはこの行頭の%を消す
%
%%%%以下、TeXにくわしくない方は触らないことをおすすめします。
% \documentclass[twoside,10Q,octavo,openany]{escoffierltjsbook}
\usepackage{amsmath}%数式
\usepackage{amssymb}
\usepackage[no-math]{fontspec}
\usepackage{geometry}
\usepackage{unicode-math}
\usepackage{xfrac}
\usepackage{luaotfload}
\usepackage{graphicx}

%%欧文フォント設定
\setmainfont[Ligatures=Historic,Scale=1.0]{Linux Libertine O}

%%Garamond
%\usepackage{ebgaramond-maths}
%\setmainfont[Ligatures=TeX,Scale=1.0]{EB Garamond}%fontspecによるフォント設定

%\usepackage{qpalatin}%palatino

%\setmainfont[Ligatures=TeX]{TeX Gyre Pagella}%ギリシャ語を用いる場合はこちら
%\setsansfont[Scale=MatchLowercase]{TeX Gyre Heros}  % \sffamily のフォント
\setsansfont[Ligatures=TeX, Scale=1]{Linux Biolinum O}     % Libertine/Biolinum
%\setmonofont[Scale=MatchLowercase]{Inconsolata}       % \ttfamily のフォント
%\unimathsetup{math-style=ISO,bold-style=ISO}
%\setmathfont{xits-math.otf}
%\setmathfont{xits-math.otf}[range={cal,bfcal},StylisticSet=1]

%\index{\usepackage}\usepackage[cmintegrals,cmbraces]{newtxmath}%数式フォント

\usepackage{luatexja}
\usepackage{luatexja-fontspec}
%\ltjdefcharrange{8}{"2000-"2013, "2015-"2025, "2027-"203A, "203C-"206F}
%\ltjsetparameter{jacharrange={-2, +8}}
\usepackage{luatexja-ruby}

%%%%和文仮名プロポーショナル
\usepackage[CharacterWidth=AlternateProportional,sourcehan,bold,jis,jis2004,expert,deluxe]{luatexja-preset}%Adobe源ノ明朝、ゴチ
%\usepackage[hiragino-pron,jis2004,expert,deluxe]{luatexja-preset}
%\usepackage[ipaex,jis2004,expert,deluxe]{luatexja-preset}
\newopentypefeature{PKana}{On}{pkna} % "PKana" and "On" can be arbitrary string
%  \setmainjfont[
%      YokoFeatures={JFM=prop},
%      CharacterWidth=AlternateProportional,
%      BoldFont={SourceHanSans-Medium},
%      ItalicFont={SourceHanSans-Regular},
%      BoldItalicFont={SourceHanSans-medium}
%  ]{SourceHanSerif-Regular}
%  \setsansjfont[
%      YokoFeatures={JFM=prop},
%      CharacterWidth=AlternateProportional,
%      BoldFont={SourceHanSans-Medium},
%      ItalicFont={SourceHanSans-Normal},
%      BoldItalicFont={SourceHanSans-Medium}
%  ]{SourceHanSans-Normal}
 %%%% 和文仮名プロプーショナルここまで
%\ltjsetparameter{jacharrange={-2}}%キリル文字%引数に-3を付けるとギリシア文字も可能になるが、%三点リーダーも欧文化されてしまうので注意%


\renewcommand{\bfdefault}{bx}%和文ボールドを有効にする
\renewcommand{\headfont}{\gtfamily\sffamily\bfseries}%和文ボールドを有効にする
%\addfontfeature{Fractions=On}


\defaultfontfeatures[\rmfamily]{Scale=1.2}%効いていない様子
\defaultjfontfeatures{Scale=0.92487}%和文フォントのサイズ調整。デフォルトは 0.962212 倍%ltjsclassesでは不要?
%\defaultjfontfeatures{Scale=0.962212}
%\usepackage{libertineotf}%linux libertine font %ギリシア語含む
%\usepackage[T1]{fontenc}
%\usepackage[full]{textcomp}
%\usepackage[osfI,scaled=1.0]{garamondx}
%\usepackage{tgheros,tgcursor}
%\usepackage[garamondx]{newtxmath}
\usepackage{xfrac}

\usepackage{layout}

% レイアウト調整(B5,14Q,onside,escoffierltjsbook.cls)
%\setlength{\voffset}{-1.5cm}
%\setlength{\textwidth}{\fullwidth}
%\setlength{\oddsidemargin}{-3mm}
%\setlength{\evensidemargin}{\oddsidemargin}
%\setlength{\textheight}{23cm}

%レイアウト調整(demy-octavo,escoffierltjsbook.cls)
%
\setlength{\hoffset}{-8mm}
\setlength{\oddsidemargin}{0pt}
\setlength{\evensidemargin}{-5mm}
%\setlength{\textwidth}{\fullwidth}%%ltjsclassesのみ有効
\setlength{\fullwidth}{112.5mm}
\setlength{\textwidth}{112.5mm}
\setlength{\marginparsep}{0pt}
\setlength{\marginparwidth}{0pt}
\setlength{\footskip}{0pt}
\setlength{\voffset}{-10mm}
\setlength{\textheight}{195mm}
\setlength{\parskip}{0pt}
%\setlength{\parindent}{0pt}
%%%ベースライン調整
%\ltjsetparameter{yjabaselineshift=0pt,yalbaselineshift=-.75pt}



\def\tightlist{\itemsep1pt\parskip0pt\parsep0pt}

%リスト環境
\makeatletter
  \parsep   = 0pt
  \labelsep = 1\zw
  \def\@listi{%
     \leftmargin = 0pt \rightmargin = 0pt
     \labelwidth\leftmargin \advance\labelwidth-\labelsep
     \topsep     = 0pt%\baselineskip
     %\topsep -0.1\baselineskip \@plus 0\baselineskip \@minus 0.1 \baselineskip
     \partopsep  = 0pt \itemsep       = 0pt
     \itemindent = 0pt \listparindent = 0\zw}
  \let\@listI\@listi
  \@listi
  \def\@listii{%
     \leftmargin = 2\zw \rightmargin = 0pt
     \labelwidth\leftmargin \advance\labelwidth-\labelsep
     \topsep     = 0pt \partopsep     = 0pt \itemsep   = 0pt
     \itemindent = 0pt \listparindent = 1\zw}
  \let\@listiii\@listii
  \let\@listiv\@listii
  \let\@listv\@listii
  \let\@listvi\@listii
\makeatother


%\usepackage{fancyhdr}

\usepackage{setspace}
\setstretch{1.1}


%% %%%%%%行取りマクロ
% \makeatletter
% \ifx\Cht\undefined
%  \newdimen\Cht\newdimen\Cdp
%  \setbox0\hbox{\char\jis"2121}\Cht=\ht0\Cdp=\dp0\fi
% \catcode`@=11
% \long\def\linespace#1#2{\par\noindent
%   \dimen@=\baselineskip
%   \multiply\dimen@ #1\advance\dimen@-\baselineskip
%   \advance\dimen@-\Cht\advance\dimen@\Cdp
%   \setbox0\vbox{\noindent #2}%
%   \advance\dimen@\ht0\advance\dimen@-\dp0%
%   \vtop to\z@{\hbox{\vrule width\z@ height\Cht depth\z@
%    \raise-.5\dimen@\hbox{\box0}}\vss}%
%   \dimen@=\baselineskip
%   \multiply\dimen@ #1\advance\dimen@-2\baselineskip
%   \par\nobreak\vskip\dimen@
%   \hbox{\vrule width\z@ height\Cht depth\z@}\vskip\z@}
% \catcode`@=12
% \setlength{\parskip}{0pt}
% \setlength{\topskip}{\Cht}
% \setlength{\textheight}{43\baselineskip}
% \addtolength{\textheight}{1\zh}
% \makeatother
 
%%%%%%%%%%%%失敗%%%%%%%%%%%%
%\let\formule\subsubsection
%\renewcommand{\subsubsection}[1]{\linespace{1}{\formule#1}}
%%%%%%%%%%%%失敗%%%%%%%%%%%%



%レシピ本文
\usepackage{multicol}
\setlength{\columnsep}{2.7\zw}
%\setlength{\columnwidth}{24\zw}	
\newenvironment{recette}{\setlength{\parindent}{0pt}\begin{small}\begin{spacing}{1.0}\begin{multicols}{2}}{\end{multicols}\end{spacing}\end{small}}

%\newenvironment{recette}{\setlength{\parindent}{0pt}\begin{normalsize}\begin{spacing}{1.0}\begin{multicols}{2}}{\end{multicols}\end{spacing}\end{normalsize}}

%\newenvironment{recette}{\setlength{\parindent}{0pt}\begin{normalsize}\begin{multicols}{2}}{\end{multicols}\end{normalsize}}



%subsubsectionに連番をつける
%\usepackage{remreset}

\renewcommand{\thechapter}{}
\renewcommand{\thesection}{}
\renewcommand{\thesubsection}{}
\renewcommand{\thesubsubsection}{}
\renewcommand{\theparagraph}{}

%\makeatletter
%\@removefromreset{subsubsection}{subsection}
%\def\thesubsubsection{\arabic{subsubsection}.}
%\newcounter{rnumber}
%\renewcommand{\thernumber}{\refstepcounter{rnumber} }

\renewcommand{\prepartname}{\if@english Part~\else {}\fi}
\renewcommand{\postpartname}{\if@english\else {}\fi}
\renewcommand{\prechaptername}{\if@english Chapter~\else {}\fi}
\renewcommand{\postchaptername}{\if@english\else {}\fi}
\renewcommand{\presectionname}{}%  第
\renewcommand{\postsectionname}{}% 節

\makeatother



% PDF/X-1a
% \usepackage[x-1a]{pdfx}
% \Keywords{pdfTeX\sep PDF/X-1a\sep PDF/A-b}
% \Title{Sample LaTeX input file}
% \Author{LaTeX project team}
% \Org{TeX Users Group}
% \pdfcompresslevel=0
%\usepackage[cmyk]{xcolor}

%biblatex
%\usepackage[notes,strict,backend=biber,autolang=other,%
%                   bibencoding=inputenc,autocite=footnote]{biblatex-chicago}
%\addbibresource{hist-agri.bib}
\let\cite=\autocite

% % % % 
\date{}

%%% 脚注番号のページ毎のリセットと脚注位置の調整
%\makeatletter
%  \@addtoreset{footnote}{page}
%  \makeatother
%\usepackage{dblfnote}
%\usepackage[bottom,perpage]{footmisc}
\usepackage[bottom,perpage,stable]{footmisc}%
%\setlength{\skip\footins}{4mm plus 2mm}
%\usepackage{footnpag}
\makeatletter
\renewcommand\@makefntext[1]{%
  \advance\leftskip 1.5\zw
  \parindent 1\zw
  \noindent
  \llap{\@thefnmark\hskip0.5\zw}#1}


\renewenvironment{theindex}{% 索引を3段組で出力する環境
    \if@twocolumn
      \onecolumn\@restonecolfalse
    \else
      \clearpage\@restonecoltrue
    \fi
    \columnseprule.4pt \columnsep 2\zw
    \ifx\multicols\@undefined
      \twocolumn[\@makeschapterhead{\indexname}%
      \addcontentsline{toc}{chapter}{\indexname}]%変更点
    \else
      \ifdim\textwidth<\fullwidth
        \setlength{\evensidemargin}{\oddsidemargin}
        \setlength{\textwidth}{\fullwidth}
        \setlength{\linewidth}{\fullwidth}
        \begin{multicols}{3}[\chapter*{\indexname}
	\addcontentsline{toc}{chapter}{\indexname}]%変更点%
      \else
        \begin{multicols}{3}[\chapter*{\indexname}
	\addcontentsline{toc}{chapter}{\indexname}]%変更点%
      \fi
    \fi
    \@mkboth{\indexname}{\indexname}%
    \plainifnotempty % \thispagestyle{plain}
    \parindent\z@
    \parskip\z@ \@plus .3\p@\relax
    \let\item\@idxitem
    \raggedright
    \footnotesize\narrowbaselines
  }{
    \ifx\multicols\@undefined
      \if@restonecol\onecolumn\fi
    \else
      \end{multicols}
    \fi
    \clearpage
  }
\makeatother



\renewcommand{\ldots}{…}
\usepackage{makeidx}
\makeindex


\usepackage[unicode=true]{hyperref}
%\usepackage{pxjahyper}
\hypersetup{breaklinks=true,%
             bookmarks=true,%
             pdfauthor={},%
             pdftitle={},%
             colorlinks=true,%
             citecolor=blue,%
             urlcolor=cyan,%
             linkcolor=magenta,%
             pdfborder={0 0 0}}


% \hypersetup{
%     pdfborderstyle={/S/U/W 1}, % underline links instead of boxes
%     linkbordercolor=red,       % color of internal links
%     citebordercolor=green,     % color of links to bibliography
%     filebordercolor=magenta,   % color of file links
%     urlbordercolor=cyan        % color of external links
% }

           \urlstyle{same}
%\renewcommand*{\label}[1]{\hypertarget{#1}{}}
%\renewcommand{\hyperlink}[2]{\hyperref[#1]{#2}}

\renewcommand{\ldots}{\noindent…}

\newcommand{\maeaki}{}
%\newcommand{\maeaki}{\vspace*{0.125\zw}}
%\newcommand{\maeaki}{\vspace*{-0.75\zw}}
%\newcommand{\maeaki}{\vspace*{1.0\zw}}
%\newcommand{\maeaki}{\vspace*{1.1\zw}}
%\newcommand{\maeaki}{\vspace*{1.5\zw}}
%\newcommand{\maeaki}{\vspace*{1.75\zw}}
%\newcommand{\maeaki}{\vspace*{1.0mm}}                     
%\newcommand{\maeaki}{\vspace*{2.2\zw}}
%\newcommand{\maeaki}{\vspace*{-.25mm}}

%%分数の表記

\newcommand{\undemi}{$\sfrac{1}{2}$}
\newcommand{\untiers}{$\sfrac{1}{3}$}
\newcommand{\deuxtiers}{$\sfrac{2}{3}$}
\newcommand{\unquart}{$\sfrac{1}{4}$}
\newcommand{\troisquarts}{$\sfrac{3}{4}$}
\newcommand{\quatrequatrieme}{$\sfrac{4}{4$}}
\newcommand{\uncinquieme}{$\sfrac{1}{5}$}
\newcommand{\deuxcinquiemes}{$\sfrac{2}{5}$}
\newcommand{\troiscinquiemes}{$\sfrac{3}{5}$}
\newcommand{\quatrecinquiemes}{$\sfrac{4}{5}$}
\newcommand{\unsixieme}{$\sfrac{1}{6}$}
\newcommand{\cinqsixiemes}{$\sfrac{5}{6}$}
%初版、第二版原書とほぼ同じ判型。ただし調整が必要
% \documentclass[twoside,14Q,a4paper,openany]{ltjsbook}

\usepackage{amsmath}
\usepackage{amssymb}
\usepackage[no-math]{fontspec}
\usepackage{geometry}
\usepackage{xfrac}
\usepackage{luaotfload}
\usepackage{graphicx}

%% 欧文フォント設定
% Libertine/Biolinum
\setmainfont[Ligatures=Historic,Scale=1.0]{Linux Libertine O}
\setsansfont[Ligatures=TeX, Scale=MatchLowercase]{Linux Biolinum O} 
%\usepackage{libertine}
\usepackage{unicode-math}
\setmathfont[Scale=1.2]{libertinusmath-regular.otf}
%\unimathsetup{math-style=ISO,bold-style=ISO}
%\setmathfont{xits-math.otf}
%\setmathfont{xits-math.otf}[range={cal,bfcal},StylisticSet=1]




%% Garamond
%\usepackage{ebgaramond-maths}
%\setmainfont[Ligatures=Historic,Scale=1.1]{EB Garamond}%fontspecによるフォント設定

%\usepackage{qpalatin}%palatino

%\setmainfont[Ligatures=Historic,Scale=MatchLowercase]{Tex Gyre Schola}
%\setmainfont[Ligatures=Historic,Scale=MatchLowercase]{Tex Gyre Pagella}
%\setsansfont[Scale=MatchLowercase]{TeX Gyre Heros}  % \sffamily のフォント
%\setsansfont[Scale=MatchLowercase]{TeX Gyre Adventor}  % \sffamily のフォント

%\setmonofont[Scale=MatchLowercase]{Inconsolata}       % \ttfamily のフォント

%\usepackage[cmintegrals,cmbraces]{newtxmath}%数式フォント

\usepackage{luatexja}
\usepackage{luatexja-fontspec}
%\ltjdefcharrange{8}{"2000-"2013, "2015-"2025, "2027-"203A, "203C-"206F}
%\ltjsetparameter{jacharrange={-2, +8}}
\usepackage{luatexja-ruby}

%%%%和文フォント設定
%\usepackage[CharacterWidth=AlternateProportional,sourcehan,bold,jis,jis2004,expert,deluxe]{luatexja-preset}%Adobe源ノ明朝、ゴチ
%\usepackage[hiragino-pron,jis,bold,jis2004,expert,deluxe]{luatexja-preset}
%\usepackage[YokoFeatures={JFM=prop,PKana=On},ipaex,bold,jis,jis2004,expert,deluxe]{luatexja-preset}
%\usepackage[YokoFeatures={JFM=prop,PKana=On},ipaex,bold,jis,jis2004,expert,deluxe]{luatexja-preset}
%\usepackage[yu-osx,bold]{myluatexja-preset}
%\usepackage[moga-mogo-ex,bold]{myluatexja-preset}

\newopentypefeature{PKana}{On}{pkna} % "PKana" and "On" can be arbitrary string
   \setmainjfont[%
%   YokoFeatures={JFM=prop,PKana=On},
% %   CharacterWidth=AlternateProportional,
% % %       Kerning=On,
         BoldFont={ MoboGoB },%
         ItalicFont={ MoboGoB },%
         BoldItalicFont={ MoboGoExB }%
%         ]{ MogaHMin }
          ]{ IPAExMincho }
     \setsansjfont[%
%        YokoFeatures={JFM=prop,PKana=On},
% %        CharacterWidth=AlternateProportional,
% % %       Kerning=On,
         BoldFont={ MoboGoB },%
         ItalicFont={ MoboGoB },%
         BoldItalicFont={ MoboGoExB }%
         % ]{ MoboGo}
          ]{ IPAExGothic }
%  %%%% 和文仮名プロプーショナルここまで
% %\ltjsetparameter{jacharrange={-2}}%キリル文字%引数に-3を付けるとギリシア文字も可能になるが、%三点リーダーも欧文化されてしまうので注意%


\renewcommand{\bfdefault}{bx}%和文ボールドを有効にする
\renewcommand{\headfont}{\gtfamily\sffamily\bfseries}%和文ボールドを有効にする
%\addfontfeature{Fractions=On}


\defaultfontfeatures[\rmfamily]{Scale=1.2}%効いていない様子
\defaultjfontfeatures{Scale=0.92487}%和文フォントのサイズ調整。デフォルトは 0.962212 倍%ltjsclassesでは不要?
%\defaultjfontfeatures{Scale=0.962212}
%\usepackage{libertineotf}%linux libertine font %ギリシア語含む
%\usepackage[T1]{fontenc}
%\usepackage[full]{textcomp}
%\usepackage[osfI,scaled=1.0]{garamondx}
%\usepackage{tgheros,tgcursor}
%\usepackage[garamondx]{newtxmath}
\usepackage{xfrac}

\usepackage{layout}

%% レイアウト調整(A4Paper,13Q,onside,escoffierltjsbook.cls) 
%%
\setlength{\hoffset}{0\zw}
\setlength{\oddsidemargin}{0\zw}
\setlength{\evensidemargin}{\oddsidemargin}
\setlength{\fullwidth}{45\zw}
\setlength{\textwidth}{45\zw}%%ltjsclassesのみ有効
%\setlength{\fullwidth}{159mm}
%\setlength{\textwidth}{159mm}
\setlength{\marginparsep}{0pt}
\setlength{\marginparwidth}{0pt}
\setlength{\footskip}{0pt}
\setlength{\voffset}{-17mm}
\setlength{\textheight}{265mm}
\setlength{\parskip}{0pt}
%\setlength{\parindent}{0pt}
%%%ベースライン調整
%\ltjsetparameter{yjabaselineshift=0pt,yalbaselineshift=-.75pt}


%\usepackage{fancyhdr}

\usepackage{setspace}
\setstretch{1.0}




%文字サイズ、見出しなどの再定義
\makeatletter
%\renewcommand{\large}{\jsc@setfontsize\large\@xipt{14}}
%\renewcommand{\Large}{\jsc@setfontsize\Large{13}{15}}

\newcommand{\medlarge}{\fontsize{11}{13}\selectfont}
\newcommand{\medsmall}{\fontsize{9.23}{9.5}\selectfont}
\newcommand{\twelveq}{\jsc@setfontsize\twelveq{9.230769}{9.75}\selectfont}
\newcommand{\fourteenq}{\jsc@setfontsize\fourteenq{10.7692}{13}\selectfont}
\newcommand{\fifteenq}{\jsc@setfontsize\fifteenq{11.53846}{14}\selectfont}

\renewcommand{\chapter}{%
  \if@openleft\cleardoublepage\else
  \if@openright\cleardoublepage\else\clearpage\fi\fi
  \plainifnotempty % 元: \thispagestyle{plain}
  \global\@topnum\z@
  \if@english \@afterindentfalse \else \@afterindenttrue \fi
  \secdef
    {\@omit@numberfalse\@chapter}%
    {\@omit@numbertrue\@schapter}}
\def\@chapter[#1]#2{%
  \ifnum \c@secnumdepth >\m@ne
    \if@mainmatter
      \refstepcounter{chapter}%
      \typeout{\@chapapp\thechapter\@chappos}%
      \addcontentsline{toc}{chapter}%
        {\protect\numberline
        % {\if@english\thechapter\else\@chapapp\thechapter\@chappos\fi}%
        {\@chapapp\thechapter\@chappos}%
        #1}%
    \else\addcontentsline{toc}{chapter}{#1}\fi
  \else
    \addcontentsline{toc}{chapter}{#1}%
  \fi
  \chaptermark{#1}%
  \addtocontents{lof}{\protect\addvspace{10\jsc@mpt}}%
  \addtocontents{lot}{\protect\addvspace{10\jsc@mpt}}%
  \if@twocolumn
    \@topnewpage[\@makechapterhead{#2}]%
  \else
    \@makechapterhead{#2}%
    \@afterheading
  \fi}
\def\@makechapterhead#1{%
  \vspace*{0\Cvs}% 欧文は50pt
  {\parindent \z@ \centering \normalfont
    \ifnum \c@secnumdepth >\m@ne
      \if@mainmatter
        \huge\headfont \@chapapp\thechapter\@chappos%変更
        \par\nobreak
        \vskip \Cvs % 欧文は20pt
      \fi
    \fi
    \interlinepenalty\@M
    \huge \headfont #1\par\nobreak
    \vskip 1\Cvs}} % 欧文は40pt%変更

\renewcommand{\section}{%
    \if@slide\clearpage\fi
    \@startsection{section}{1}{\z@}%
    {\Cvs \@plus.5\Cdp \@minus.2\Cdp}% 前アキ
    % {.5\Cvs \@plus.3\Cdp}% 後アキ
    {.5\Cvs}
    {\normalfont\Large\headfont\bfseries\centering}}%変更

\renewcommand{\subsection}{\@startsection{subsection}{2}{\z@}%
    {\Cvs \@plus.5\Cdp \@minus.2\Cdp}% 前アキ
    % {.5\Cvs \@plus.3\Cdp}% 後アキ
    {.5\Cvs}
    {\normalfont\large\headfont\bfseries\centering}} %変更


\renewcommand{\subsubsection}{\@startsection{subsubsection}{3}{\z@}%
    {.25\Cvs \@plus.5\Cdp \@minus.5\Cdp}%変更
    {\if@slide .5\Cvs \@plus.3\Cdp \else \z@ \fi}%
    {\normalfont\medlarge\headfont\leftskip -1\zw}}

\renewcommand{\paragraph}{\@startsection{paragraph}{4}{\z@}%
    {0.5\Cvs \@plus.5\Cdp \@minus.2\Cdp}%
    % {\if@slide .5\Cvs \@plus.3\Cdp \else -1\zw\fi}% 改行せず 1\zw のアキ
    {1sp}%後アキ
    {\normalfont\normalsize\headfont}}
\renewcommand{\subparagraph}{\@startsection{subparagraph}{5}{\z@}%
    {\z@}{\if@slide .5\Cvs \@plus.3\Cdp \else -.5\zw\fi}%
    {\normalfont\normalsize\headfont\hskip-.5\zw\noindent}}  



\newcommand{\frsec}[1]{\vspace*{-1\zw}\begin{center}\normalfont\hspace*{1\zw}\headfont\Large\scshape#1\normalfont\normalsize\end{center}\vspace{0.5\zw}}

\newcommand{\frsecb}[1]{\vspace*{-1\zw}\begin{center}\hspace{1\zw}\normalfont\headfont\large\scshape#1\normalfont\normalsize\end{center}\vspace{0.5\zw}}

%\newcounter{frsub}[subsubsection]
%\newcommand{\frsub}{\@startsection{frsub}{6}{\z@}%
%  {1sp}{1sp}%
%  {\normalfont\normalsize\bfseries\baselineskip-.8ex\leftskip-1\zw}}
%\let\frsubmark\@gobble
%\newcommand*{\l@frsub}{%
%          \@tempdima\jsc@tocl@width \advance\@tempdima 16.183\zw
%          \@dottedtocline{7}{\@tempdima}{6.5\zw}}
%\renewcommand{\thefrsub}{6}
%\let\frsub\paragraph

\newenvironment{frsubenv}{\begin{spacing}{0.2}\setlength{\leftskip}{-1\zw}\bfseries}{\end{spacing}\normalfont\normalsize\setlength{\leftskip}{0pt}}
\newcommand{\frsub}[1]{\begin{frsubenv}#1\end{frsubenv}\par\vspace{1.1\zw}}

%\newcommand{\frsub}[1]{\vskip -.8ex\hskip -1\zw\textbf{#1}\leftskip0pt}
%\newcommand{\frsub}{\@startsection{frsub}{6}{\z@}%
%   {-1\zw}% 改行せず 1\zw のアキ
%   {-1\zw}%後アキ     
%   {\normalfont\normalsize\bfseries\leftskip -1\zw\baselineskip -.5ex}}%normalsizeから変更
%\newcommand*{\l@frsub}{%
%          \@tempdima\jsc@tocl@width \advance\@tempdima 16.183\zw
%          \@dottedtocline{5}{\@tempdima}{6.5\zw}}

\makeatother

%%% 脚注番号のページ毎のリセットと脚注位置の調整
\makeatletter

\usepackage[bottom,perpage,stable]{footmisc}%
%\setlength{\skip\footins}{4mm plus 2mm}
%\usepackage{footnpag}
\renewcommand\@makefntext[1]{%
  \advance\leftskip 1.5\zw
  \parindent 1\zw
  \noindent
  \llap{\@thefnmark\hskip0.5\zw}#1}


\let\footnotes@ve=\footnote
\def\footnote{\inhibitglue\footnotes@ve}
\let\footnotemarks@ve=\footnotemark
%\def\footnotemark{\inhibitglue\footnotemarks@ve}
\renewcommand{\footnotemark}{\footnotemarks@ve}%変更
% %\def\thefootnote{\ifnum\c@footnote>\z@\leavevmode\lower.5ex\hbox{(}\@arabic\c@footnote\hbox{)}\fi}
\renewcommand{\thefootnote}{\ifnum\c@footnote>\z@\leavevmode\hbox{}\@arabic\c@footnote\hbox{)}\fi}
%\makeatletter
% \@addtoreset{footnote}{page}
% \makeatother
%\usepackage{dblfnote}
%\usepackage[bottom,perpage]{footmisc}
\renewcommand{\footnote}[1]{}

\makeatother

%subsubsectionに連番をつける
%\usepackage{remreset}

\renewcommand{\thechapter}{}
\renewcommand{\thesection}{}
\renewcommand{\thesubsection}{}
\renewcommand{\thesubsubsection}{}
\renewcommand{\theparagraph}{}


%\makeatletter
%\@removefromreset{subsubsection}{subsection}
%\def\thesubsubsection{\arabic{subsubsection}.}
%\newcounter{rnumber}
%\renewcommand{\thernumber}{\refstepcounter{rnumber} }

\renewcommand{\prepartname}{\if@english Part~\else {}\fi}
\renewcommand{\postpartname}{\if@english\else {}\fi}
\renewcommand{\prechaptername}{\if@english Chapter~\else {}\fi}
\renewcommand{\postchaptername}{\if@english\else {}\fi}
\renewcommand{\presectionname}{}%  第
\renewcommand{\postsectionname}{}% 節





%レシピ本文
\usepackage{multicol}
\setlength{\columnsep}{3\zw}
%\setlength{\columnwidth}{24\zw}	
%\newenvironment{recette}{\setlength{\parindent}{0pt}\begin{medsmall}\begin{spacing}{0.8}\begin{multicols}{2}}{\end{multicols}\end{spacing}\end{medsmall}}

\newenvironment{recette}{\setlength{\parindent}{0pt}\begin{normalsize}\begin{spacing}{0.8}\begin{multicols}{2}}{\end{multicols}\end{spacing}\end{normalsize}}

%\newenvironment{recette}{\setlength{\parindent}{0pt}\begin{normalsize}\begin{multicols}{2}}{\end{multicols}\end{normalsize}}


%リスト環境
\def\tightlist{\itemsep1pt\parskip0pt\parsep0pt}%pandoc対策

\makeatletter
  \parsep   = 0pt
  \labelsep = .5\zw
  \def\@listi{%
     \leftmargin = 0pt \rightmargin = 0pt
     \labelwidth\leftmargin \advance\labelwidth-\labelsep
     \topsep     = 0pt%\baselineskip
     %\topsep -0.1\baselineskip \@plus 0\baselineskip \@minus 0.1 \baselineskip
     \partopsep  = 0pt \itemsep       = 0pt
     \itemindent = -.5\zw \listparindent = 0\zw}
  \let\@listI\@listi
  \@listi
  \def\@listii{%
     \leftmargin = 1.8\zw \rightmargin = 0pt
     \labelwidth\leftmargin \advance\labelwidth-\labelsep
     \topsep     = 0pt \partopsep     = 0pt \itemsep   = 0pt
     \itemindent = 0pt \listparindent = 1\zw}
  \let\@listiii\@listii
  \let\@listiv\@listii
  \let\@listv\@listii
  \let\@listvi\@listii
\makeatother
%% %%%%%%行取りマクロ
% \makeatletter
% \ifx\Cht\undefined
%  \newdimen\Cht\newdimen\Cdp
%  \setbox0\hbox{\char\jis"2121}\Cht=\ht0\Cdp=\dp0\fi
% \catcode`@=11
% \long\def\linespace#1#2{\par\noindent
%   \dimen@=\baselineskip
%   \multiply\dimen@ #1\advance\dimen@-\baselineskip
%   \advance\dimen@-\Cht\advance\dimen@\Cdp
%   \setbox0\vbox{\noindent #2}%
%   \advance\dimen@\ht0\advance\dimen@-\dp0%
%   \vtop to\z@{\hbox{\vrule width\z@ height\Cht depth\z@
%    \raise-.5\dimen@\hbox{\box0}}\vss}%
%   \dimen@=\baselineskip
%   \multiply\dimen@ #1\advance\dimen@-2\baselineskip
%   \par\nobreak\vskip\dimen@
%   \hbox{\vrule width\z@ height\Cht depth\z@}\vskip\z@}
% \catcode`@=12
% \setlength{\parskip}{0pt}
% \setlength{\topskip}{\Cht}
% \setlength{\textheight}{43\baselineskip}
% \addtolength{\textheight}{1\zh}
% \makeatother
 
%%%%%%%%%%%%失敗%%%%%%%%%%%%
%\let\formule\subsubsection
%\renewcommand{\subsubsection}[1]{\linespace{1}{\formule#1}}
%%%%%%%%%%%%失敗%%%%%%%%%%%%






% PDF/X-1a
% \usepackage[x-1a]{pdfx}
% \Keywords{pdfTeX\sep PDF/X-1a\sep PDF/A-b}
% \Title{Sample LaTeX input file}
% \Author{LaTeX project team}
% \Org{TeX Users Group}
% \pdfcompresslevel=0
%\usepackage[cmyk]{xcolor}

%biblatex
%\usepackage[notes,strict,backend=biber,autolang=other,%
%                   bibencoding=inputenc,autocite=footnote]{biblatex-chicago}
%\addbibresource{hist-agri.bib}
\let\cite=\autocite

% % % % 
\date{}



\makeatletter
\renewenvironment{theindex}{% 索引を3段組で出力する環境
    \if@twocolumn
      \onecolumn\@restonecolfalse
    \else
      \clearpage\@restonecoltrue
    \fi
    \columnseprule.4pt \columnsep 2\zw
    \ifx\multicols\@undefined
      \twocolumn[\@makeschapterhead{\indexname}%
      \addcontentsline{toc}{chapter}{\indexname}]%変更点
    \else
      \ifdim\textwidth<\fullwidth
        \setlength{\evensidemargin}{\oddsidemargin}
        \setlength{\textwidth}{\fullwidth}
        \setlength{\linewidth}{\fullwidth}
        \begin{multicols}{3}[\chapter*{\indexname}
	\addcontentsline{toc}{chapter}{\indexname}]%変更点%
      \else
        \begin{multicols}{3}[\chapter*{\indexname}
	\addcontentsline{toc}{chapter}{\indexname}]%変更点%
      \fi
    \fi
    \@mkboth{\indexname}{\indexname}%
    \plainifnotempty % \thispagestyle{plain}
    \parindent\z@
    \parskip\z@ \@plus .3\p@\relax
    \let\item\@idxitem
    \raggedright
    \footnotesize\narrowbaselines
  }{
    \ifx\multicols\@undefined
      \if@restonecol\onecolumn\fi
    \else
      \end{multicols}
    \fi
    \clearpage
  }
\makeatother



%\renewcommand{\ldots}{…}
\usepackage{makeidx}
\makeindex


\usepackage[unicode=true]{hyperref}
%\usepackage{pxjahyper}
\hypersetup{breaklinks=true,%
             bookmarks=true,%
             pdfauthor={五島 学},%
             pdftitle={エスコフィエ『料理の手引き』全注解},%
             colorlinks=true,%
             citecolor=blue,%
             urlcolor=cyan,%
             linkcolor=magenta,%
             bookmarksdepth=subsubsection,%
             pdfborder={0 0 0}}


% \hypersetup{
%     pdfborderstyle={/S/U/W 1}, % underline links instead of boxes
%     linkbordercolor=red,       % color of internal links
%     citebordercolor=green,     % color of links to bibliography
%     filebordercolor=magenta,   % color of file links
%     urlbordercolor=cyan        % color of external links
% }

\urlstyle{same}
%\renewcommand*{\label}[1]{\hypertarget{#1}{}}
%\renewcommand{\hyperlink}[2]{\hyperref[#1]{#2}}

\renewcommand{\ldots}{\noindent…}
%\usepackage{udline}
% \usepackage{ulem}
\usepackage{umoline}
\setlength{\UnderlineDepth}{2pt}
\let\ul\Underline

\newcommand{\maeaki}{}
%\newcommand{\maeaki}{\vspace{0.125\zw}}
%\newcommand{\maeaki}{\vspace{0.7\zw}}
%\newcommand{\maeaki}{\vspace{2.0\zw}}
%\newcommand{\maeaki}{\vspace{1.1\zw}}
%\newcommand{\maeaki}{\vspace{1.5\zw}}
%\newcommand{\maeaki}{\vspace{1.75\zw}}
%\newcommand{\maeaki}{\vspace{1.0mm}}                     
%\newcommand{\maeaki}{\vspace{2.2\zw}}
%\newcommand{\maeaki}{\vspace{-.25mm}}

%%分数の表記

\newcommand{\undemi}{\hspace{.25\zw}$\sfrac{1}{2}$}
\newcommand{\untiers}{\hspace{.25\zw}$\sfrac{1}{3}$}
\newcommand{\deuxtiers}{\hspace{.25\zw}$\sfrac{2}{3}$}
\newcommand{\unquart}{\hspace{.25\zw}$\sfrac{1}{4}$}
\newcommand{\troisquarts}{\hspace{.25\zw}$\sfrac{3}{4}$}
\newcommand{\quatrequatrieme}{\hspace{.25\zw}$\sfrac{4}{4$}}
\newcommand{\uncinquieme}{\hspace{.25\zw}$\sfrac{1}{5}$}
\newcommand{\deuxcinquiemes}{\hspace{.25\zw}$\sfrac{2}{5}$}
\newcommand{\troiscinquiemes}{\hspace{.25\zw}$\sfrac{3}{5}$}
\newcommand{\quatrecinquiemes}{\hspace{.25\zw}$\sfrac{4}{5}$}
\newcommand{\unsixieme}{\hspace{.25\zw}$\sfrac{1}{6}$}
\newcommand{\cinqsixiemes}{\hspace{.25\zw}$\sfrac{5}{6}$}
\newcommand{\quatresurquatre}{\hspace{.25\zw}$\sfrac{4}{4}$}
%脚注なしバージョン、ただし調整が必要
% \documentclass[twoside,14Q,a4paper,openany]{ltjsbook}

\usepackage{amsmath}
\usepackage{amssymb}
\usepackage[no-math]{fontspec}
\usepackage{geometry}
\usepackage{unicode-math}
\usepackage{xfrac}
\usepackage{luaotfload}
\usepackage{graphicx}

%%欧文フォント設定
\setmainfont[Ligatures=Historic,Scale=1.0]{Linux Libertine O}

%%Garamond
%\usepackage{ebgaramond-maths}
%\setmainfont[Ligatures=Historic,Scale=1.1]{EB Garamond}%fontspecによるフォント設定

%\usepackage{qpalatin}%palatino

%%%%%\setmainfont[Ligatures=Historic,Scale=MatchLowercase]{Tex Gyre Schola}
%\setmainfont[Ligatures=Historic,Scale=MatchLowercase]{Tex Gyre Pagella}
%\setsansfont[Scale=MatchLowercase]{TeX Gyre Heros}  % \sffamily のフォント
%\setsansfont[Scale=MatchLowercase]{TeX Gyre Adventor}  % \sffamily のフォント
\setsansfont[Ligatures=TeX, Scale=MatchLowercase]{Linux Biolinum O}     % Libertine/Biolinum
%\setmonofont[Scale=MatchLowercase]{Inconsolata}       % \ttfamily のフォント
%\unimathsetup{math-style=ISO,bold-style=ISO}
%\setmathfont{xits-math.otf}
%\setmathfont{xits-math.otf}[range={cal,bfcal},StylisticSet=1]

%\index{\usepackage}\usepackage[cmintegrals,cmbraces]{newtxmath}%数式フォント

\usepackage{luatexja}
\usepackage{luatexja-fontspec}
%\ltjdefcharrange{8}{"2000-"2013, "2015-"2025, "2027-"203A, "203C-"206F}
%\ltjsetparameter{jacharrange={-2, +8}}
\usepackage{luatexja-ruby}

%%%%和文フォント設定
%\usepackage[CharacterWidth=AlternateProportional,sourcehan,bold,jis,jis2004,expert,deluxe]{luatexja-preset}%Adobe源ノ明朝、ゴチ
%\usepackage[hiragino-pron,jis,bold,jis2004,expert,deluxe]{luatexja-preset}
%\usepackage[YokoFeatures={JFM=prop,PKana=On},ipaex,bold,jis,jis2004,expert,deluxe]{luatexja-preset}
%\usepackage[YokoFeatures={JFM=prop,PKana=On},ipaex,bold,jis,jis2004,expert,deluxe]{luatexja-preset}
%\usepackage[yu-osx,bold]{myluatexja-preset}
%\usepackage[moga-mogo-ex,bold]{myluatexja-preset}

\newopentypefeature{PKana}{On}{pkna} % "PKana" and "On" can be arbitrary string
   \setmainjfont[%
%   YokoFeatures={JFM=prop,PKana=On},
% %   CharacterWidth=AlternateProportional,
% % %       Kerning=On,
         BoldFont={ MoboGoB },%
         ItalicFont={ MoboGoB },%
         BoldItalicFont={ MoboGoExB }%
%         ]{ MogaHMin }
          ]{ IPAExMincho }
     \setsansjfont[%
%        YokoFeatures={JFM=prop,PKana=On},
% %        CharacterWidth=AlternateProportional,
% % %       Kerning=On,
         BoldFont={ MoboGoB },%
         ItalicFont={ MoboGoB },%
         BoldItalicFont={ MoboGoExB }%
         % ]{ MoboGo}
          ]{ IPAExGothic }
%  %%%% 和文仮名プロプーショナルここまで
% %\ltjsetparameter{jacharrange={-2}}%キリル文字%引数に-3を付けるとギリシア文字も可能になるが、%三点リーダーも欧文化されてしまうので注意%


\renewcommand{\bfdefault}{bx}%和文ボールドを有効にする
\renewcommand{\headfont}{\gtfamily\sffamily\bfseries}%和文ボールドを有効にする
%\addfontfeature{Fractions=On}


\defaultfontfeatures[\rmfamily]{Scale=1.2}%効いていない様子
\defaultjfontfeatures{Scale=0.92487}%和文フォントのサイズ調整。デフォルトは 0.962212 倍%ltjsclassesでは不要?
%\defaultjfontfeatures{Scale=0.962212}
%\usepackage{libertineotf}%linux libertine font %ギリシア語含む
%\usepackage[T1]{fontenc}
%\usepackage[full]{textcomp}
%\usepackage[osfI,scaled=1.0]{garamondx}
%\usepackage{tgheros,tgcursor}
%\usepackage[garamondx]{newtxmath}
\usepackage{xfrac}

\usepackage{layout}

%% レイアウト調整(A4Paper,13Q,onside,escoffierltjsbook.cls) 
%%
\setlength{\hoffset}{0\zw}
\setlength{\oddsidemargin}{0\zw}
\setlength{\evensidemargin}{\oddsidemargin}
\setlength{\fullwidth}{45\zw}
\setlength{\textwidth}{45\zw}%%ltjsclassesのみ有効
%\setlength{\fullwidth}{159mm}
%\setlength{\textwidth}{159mm}
\setlength{\marginparsep}{0pt}
\setlength{\marginparwidth}{0pt}
\setlength{\footskip}{0pt}
\setlength{\voffset}{-17mm}
\setlength{\textheight}{265mm}
\setlength{\parskip}{0pt}
%\setlength{\parindent}{0pt}
%%%ベースライン調整
%\ltjsetparameter{yjabaselineshift=0pt,yalbaselineshift=-.75pt}


%\usepackage{fancyhdr}

\usepackage{setspace}
\setstretch{1.0}




%文字サイズ、見出しなどの再定義
\makeatletter
%\renewcommand{\large}{\jsc@setfontsize\large\@xipt{14}}
%\renewcommand{\Large}{\jsc@setfontsize\Large{13}{15}}

\newcommand{\medlarge}{\fontsize{11}{13}\selectfont}
\newcommand{\medsmall}{\fontsize{9.23}{9.5}\selectfont}
\newcommand{\twelveq}{\jsc@setfontsize\twelveq{9.230769}{9.75}\selectfont}
\newcommand{\fourteenq}{\jsc@setfontsize\fourteenq{10.7692}{13}\selectfont}
\newcommand{\fifteenq}{\jsc@setfontsize\fifteenq{11.53846}{14}\selectfont}

\renewcommand{\chapter}{%
  \if@openleft\cleardoublepage\else
  \if@openright\cleardoublepage\else\clearpage\fi\fi
  \plainifnotempty % 元: \thispagestyle{plain}
  \global\@topnum\z@
  \if@english \@afterindentfalse \else \@afterindenttrue \fi
  \secdef
    {\@omit@numberfalse\@chapter}%
    {\@omit@numbertrue\@schapter}}
\def\@chapter[#1]#2{%
  \ifnum \c@secnumdepth >\m@ne
    \if@mainmatter
      \refstepcounter{chapter}%
      \typeout{\@chapapp\thechapter\@chappos}%
      \addcontentsline{toc}{chapter}%
        {\protect\numberline
        % {\if@english\thechapter\else\@chapapp\thechapter\@chappos\fi}%
        {\@chapapp\thechapter\@chappos}%
        #1}%
    \else\addcontentsline{toc}{chapter}{#1}\fi
  \else
    \addcontentsline{toc}{chapter}{#1}%
  \fi
  \chaptermark{#1}%
  \addtocontents{lof}{\protect\addvspace{10\jsc@mpt}}%
  \addtocontents{lot}{\protect\addvspace{10\jsc@mpt}}%
  \if@twocolumn
    \@topnewpage[\@makechapterhead{#2}]%
  \else
    \@makechapterhead{#2}%
    \@afterheading
  \fi}
\def\@makechapterhead#1{%
  \vspace*{0\Cvs}% 欧文は50pt
  {\parindent \z@ \centering \normalfont
    \ifnum \c@secnumdepth >\m@ne
      \if@mainmatter
        \huge\headfont \@chapapp\thechapter\@chappos%変更
        \par\nobreak
        \vskip \Cvs % 欧文は20pt
      \fi
    \fi
    \interlinepenalty\@M
    \huge \headfont #1\par\nobreak
    \vskip 1\Cvs}} % 欧文は40pt%変更

\renewcommand{\section}{%
    \if@slide\clearpage\fi
    \@startsection{section}{1}{\z@}%
    {\Cvs \@plus.5\Cdp \@minus.2\Cdp}% 前アキ
    % {.5\Cvs \@plus.3\Cdp}% 後アキ
    {.5\Cvs}
    {\normalfont\Large\headfont\bfseries\centering}}%変更

\renewcommand{\subsection}{\@startsection{subsection}{2}{\z@}%
    {\Cvs \@plus.5\Cdp \@minus.2\Cdp}% 前アキ
    % {.5\Cvs \@plus.3\Cdp}% 後アキ
    {.5\Cvs}
    {\normalfont\large\headfont\bfseries\centering}} %変更


\renewcommand{\subsubsection}{\@startsection{subsubsection}{3}{\z@}%
    {.25\Cvs \@plus.5\Cdp \@minus.5\Cdp}%変更
    {\if@slide .5\Cvs \@plus.3\Cdp \else \z@ \fi}%
    {\normalfont\medlarge\headfont\leftskip -1\zw}}

\renewcommand{\paragraph}{\@startsection{paragraph}{4}{\z@}%
    {0.5\Cvs \@plus.5\Cdp \@minus.2\Cdp}%
    % {\if@slide .5\Cvs \@plus.3\Cdp \else -1\zw\fi}% 改行せず 1\zw のアキ
    {1sp}%後アキ
    {\normalfont\normalsize\headfont}}
\renewcommand{\subparagraph}{\@startsection{subparagraph}{5}{\z@}%
    {\z@}{\if@slide .5\Cvs \@plus.3\Cdp \else -.5\zw\fi}%
    {\normalfont\normalsize\headfont\hskip-.5\zw\noindent}}  



\newcommand{\frsec}[1]{\vspace*{-1\zw}\begin{center}\normalfont\hspace*{1\zw}\headfont\Large\scshape#1\normalfont\normalsize\end{center}\vspace{0.5\zw}}

\newcommand{\frsecb}[1]{\vspace*{-1\zw}\begin{center}\hspace{1\zw}\normalfont\headfont\large\scshape#1\normalfont\normalsize\end{center}\vspace{0.5\zw}}

\newcommand{\frsub}[1]{\vskip -.8ex\hskip -1\zw\textbf{#1}\leftskip0pt}
%\newcommand{\frsub}{\@startsection{frsub}{6}{\z@}%
%   {-1\zw}% 改行せず 1\zw のアキ
%   {-1\zw}%後アキ     
%   {\normalfont\normalsize\bfseries\leftskip -1\zw\baselineskip -.5ex}}%normalsizeから変更
%\newcommand*{\l@frsub}{%
%          \@tempdima\jsc@tocl@width \advance\@tempdima 16.183\zw
%          \@dottedtocline{5}{\@tempdima}{6.5\zw}}

\makeatother

%%% 脚注番号のページ毎のリセットと脚注位置の調整
\makeatletter

\usepackage[bottom,perpage,stable]{footmisc}%
%\setlength{\skip\footins}{4mm plus 2mm}
%\usepackage{footnpag}
\renewcommand\@makefntext[1]{%
  \advance\leftskip 1.5\zw
  \parindent 1\zw
  \noindent
  \llap{\@thefnmark\hskip0.5\zw}#1}


\let\footnotes@ve=\footnote
\def\footnote{\inhibitglue\footnotes@ve}
\let\footnotemarks@ve=\footnotemark
%\def\footnotemark{\inhibitglue\footnotemarks@ve}
\renewcommand{\footnotemark}{\footnotemarks@ve}%変更
% %\def\thefootnote{\ifnum\c@footnote>\z@\leavevmode\lower.5ex\hbox{(}\@arabic\c@footnote\hbox{)}\fi}
\renewcommand{\thefootnote}{\ifnum\c@footnote>\z@\leavevmode\hbox{}\@arabic\c@footnote\hbox{)}\fi}
%\makeatletter
% \@addtoreset{footnote}{page}
% \makeatother
%\usepackage{dblfnote}
%\usepackage[bottom,perpage]{footmisc}


\makeatother

%subsubsectionに連番をつける
%\usepackage{remreset}

\renewcommand{\thechapter}{}
\renewcommand{\thesection}{}
\renewcommand{\thesubsection}{}
\renewcommand{\thesubsubsection}{}
\renewcommand{\theparagraph}{}

%\makeatletter
%\@removefromreset{subsubsection}{subsection}
%\def\thesubsubsection{\arabic{subsubsection}.}
%\newcounter{rnumber}
%\renewcommand{\thernumber}{\refstepcounter{rnumber} }

\renewcommand{\prepartname}{\if@english Part~\else {}\fi}
\renewcommand{\postpartname}{\if@english\else {}\fi}
\renewcommand{\prechaptername}{\if@english Chapter~\else {}\fi}
\renewcommand{\postchaptername}{\if@english\else {}\fi}
\renewcommand{\presectionname}{}%  第
\renewcommand{\postsectionname}{}% 節





%レシピ本文
\usepackage{multicol}
\setlength{\columnsep}{3\zw}
%\setlength{\columnwidth}{24\zw}	
%\newenvironment{recette}{\setlength{\parindent}{0pt}\begin{medsmall}\begin{spacing}{0.8}\begin{multicols}{2}}{\end{multicols}\end{spacing}\end{medsmall}}

%\newenvironment{recette}{\setlength{\parindent}{0pt}\begin{normalsize}\begin{spacing}{0.8}\begin{multicols}{2}}{\end{multicols}\end{spacing}\end{normalsize}}

%\newenvironment{recette}{\setlength{\parindent}{0pt}\begin{normalsize}\begin{multicols}{2}}{\end{multicols}\end{normalsize}}

\newenvironment{recette}{}{}

%リスト環境
\def\tightlist{\itemsep1pt\parskip0pt\parsep0pt}%pandoc対策

\makeatletter
  \parsep   = 0pt
  \labelsep = .5\zw
  \def\@listi{%
     \leftmargin = 0pt \rightmargin = 0pt
     \labelwidth\leftmargin \advance\labelwidth-\labelsep
     \topsep     = 0pt%\baselineskip
     %\topsep -0.1\baselineskip \@plus 0\baselineskip \@minus 0.1 \baselineskip
     \partopsep  = 0pt \itemsep       = 0pt
     \itemindent = -.5\zw \listparindent = 0\zw}
  \let\@listI\@listi
  \@listi
  \def\@listii{%
     \leftmargin = 1.8\zw \rightmargin = 0pt
     \labelwidth\leftmargin \advance\labelwidth-\labelsep
     \topsep     = 0pt \partopsep     = 0pt \itemsep   = 0pt
     \itemindent = 0pt \listparindent = 1\zw}
  \let\@listiii\@listii
  \let\@listiv\@listii
  \let\@listv\@listii
  \let\@listvi\@listii
\makeatother
%% %%%%%%行取りマクロ
% \makeatletter
% \ifx\Cht\undefined
%  \newdimen\Cht\newdimen\Cdp
%  \setbox0\hbox{\char\jis"2121}\Cht=\ht0\Cdp=\dp0\fi
% \catcode`@=11
% \long\def\linespace#1#2{\par\noindent
%   \dimen@=\baselineskip
%   \multiply\dimen@ #1\advance\dimen@-\baselineskip
%   \advance\dimen@-\Cht\advance\dimen@\Cdp
%   \setbox0\vbox{\noindent #2}%
%   \advance\dimen@\ht0\advance\dimen@-\dp0%
%   \vtop to\z@{\hbox{\vrule width\z@ height\Cht depth\z@
%    \raise-.5\dimen@\hbox{\box0}}\vss}%
%   \dimen@=\baselineskip
%   \multiply\dimen@ #1\advance\dimen@-2\baselineskip
%   \par\nobreak\vskip\dimen@
%   \hbox{\vrule width\z@ height\Cht depth\z@}\vskip\z@}
% \catcode`@=12
% \setlength{\parskip}{0pt}
% \setlength{\topskip}{\Cht}
% \setlength{\textheight}{43\baselineskip}
% \addtolength{\textheight}{1\zh}
% \makeatother
 
%%%%%%%%%%%%失敗%%%%%%%%%%%%
%\let\formule\subsubsection
%\renewcommand{\subsubsection}[1]{\linespace{1}{\formule#1}}
%%%%%%%%%%%%失敗%%%%%%%%%%%%






% PDF/X-1a
% \usepackage[x-1a]{pdfx}
% \Keywords{pdfTeX\sep PDF/X-1a\sep PDF/A-b}
% \Title{Sample LaTeX input file}
% \Author{LaTeX project team}
% \Org{TeX Users Group}
% \pdfcompresslevel=0
%\usepackage[cmyk]{xcolor}

%biblatex
%\usepackage[notes,strict,backend=biber,autolang=other,%
%                   bibencoding=inputenc,autocite=footnote]{biblatex-chicago}
%\addbibresource{hist-agri.bib}
\let\cite=\autocite

% % % % 
\date{}



\makeatletter
\renewenvironment{theindex}{% 索引を3段組で出力する環境
    \if@twocolumn
      \onecolumn\@restonecolfalse
    \else
      \clearpage\@restonecoltrue
    \fi
    \columnseprule.4pt \columnsep 2\zw
    \ifx\multicols\@undefined
      \twocolumn[\@makeschapterhead{\indexname}%
      \addcontentsline{toc}{chapter}{\indexname}]%変更点
    \else
      \ifdim\textwidth<\fullwidth
        \setlength{\evensidemargin}{\oddsidemargin}
        \setlength{\textwidth}{\fullwidth}
        \setlength{\linewidth}{\fullwidth}
        \begin{multicols}{3}[\chapter*{\indexname}
	\addcontentsline{toc}{chapter}{\indexname}]%変更点%
      \else
        \begin{multicols}{3}[\chapter*{\indexname}
	\addcontentsline{toc}{chapter}{\indexname}]%変更点%
      \fi
    \fi
    \@mkboth{\indexname}{\indexname}%
    \plainifnotempty % \thispagestyle{plain}
    \parindent\z@
    \parskip\z@ \@plus .3\p@\relax
    \let\item\@idxitem
    \raggedright
    \footnotesize\narrowbaselines
  }{
    \ifx\multicols\@undefined
      \if@restonecol\onecolumn\fi
    \else
      \end{multicols}
    \fi
    \clearpage
  }
\makeatother



%\renewcommand{\ldots}{…}
\usepackage{makeidx}
\makeindex


\usepackage[unicode=true]{hyperref}
%\usepackage{pxjahyper}
\hypersetup{breaklinks=true,%
             bookmarks=true,%
             pdfauthor={五島 学},%
             pdftitle={エスコフィエ『料理の手引き』全注解},%
             colorlinks=true,%
             citecolor=blue,%
             urlcolor=cyan,%
             linkcolor=magenta,%
             bookmarksdepth=subsubsection,%
             pdfborder={0 0 0}}


% \hypersetup{
%     pdfborderstyle={/S/U/W 1}, % underline links instead of boxes
%     linkbordercolor=red,       % color of internal links
%     citebordercolor=green,     % color of links to bibliography
%     filebordercolor=magenta,   % color of file links
%     urlbordercolor=cyan        % color of external links
% }

\urlstyle{same}
%\renewcommand*{\label}[1]{\hypertarget{#1}{}}
%\renewcommand{\hyperlink}[2]{\hyperref[#1]{#2}}

\renewcommand{\ldots}{\noindent…}
%\usepackage{udline}
% \usepackage{ulem}
%\usepackage{umoline}
%\setlength{\UnderlineDepth}{2pt}
\let\ul\underline

\newcommand{\maeaki}{}
%\newcommand{\maeaki}{\vspace{0.125\zw}}
%\newcommand{\maeaki}{\vspace{0.7\zw}}
%\newcommand{\maeaki}{\vspace{2.0\zw}}
%\newcommand{\maeaki}{\vspace{1.1\zw}}
%\newcommand{\maeaki}{\vspace{1.5\zw}}
%\newcommand{\maeaki}{\vspace{1.75\zw}}
%\newcommand{\maeaki}{\vspace{1.0mm}}                     
%\newcommand{\maeaki}{\vspace{2.2\zw}}
%\newcommand{\maeaki}{\vspace{-.25mm}}

%%分数の表記

\newcommand{\undemi}{$\sfrac{1}{2}$}
\newcommand{\untiers}{$\sfrac{1}{3}$}
\newcommand{\deuxtiers}{$\sfrac{2}{3}$}
\newcommand{\unquart}{$\sfrac{1}{4}$}
\newcommand{\troisquarts}{$\sfrac{3}{4}$}
\newcommand{\quatrequatrieme}{$\sfrac{4}{4$}}
\newcommand{\uncinquieme}{$\sfrac{1}{5}$}
\newcommand{\deuxcinquiemes}{$\sfrac{2}{5}$}
\newcommand{\troiscinquiemes}{$\sfrac{3}{5}$}
\newcommand{\quatrecinquiemes}{$\sfrac{4}{5}$}
\newcommand{\unsixieme}{$\sfrac{1}{6}$}
\newcommand{\cinqsixiemes}{$\sfrac{5}{6}$}%1段組みバージョン、ただし調整が必要
%
%
%
%
%%% Important!%%%%%%% 文書開始%%%

%%%info%%%
\title{\Huge{オーギュスト・エスコフィエ}\\\HUGE{『料理の手引き』全注解}}
\author{\huge{五 島 学}}
%%%%
%%% Important! 文書開始%
\begin{document}
\flushbottom

%%% 扉 %%%
\maketitle
% 空白ページ

% % 原稿ファイル読み込み
\newpage
\hypertarget{benefactors}{%
\section{後援者の皆様}\label{benefactors}}

\thispagestyle{empty}

\small

\textbf{この「エスコフィエ『料理の手引き』全注解」の翻訳および注釈は、以下の方々の善意ある資金提供により行なわれています(ご支援順、敬称略)。}

\normalsize
\vspace{1\zw}

Sin Masui、\href{20180524-23h,2x,novelsoundsmail@gmail.com}{}
河井健司、\href{20180525-0h14,10x,kwibeng@gmail.com}{} Yuya
HASHIMOTO、\href{20180525-1h40,2x,hashimo0910@gmail.com}{} 善塔一幸(La
Maison
Courtine)、\href{20180525-8h56,10x,kazuyukizento120@docomo.ne.jp}{}
藤井智大(フランス料理店 ミルエテ)、\href{20180525-10h07,1x,apple19761019@yahoo.co.jp}{}
kojima mio、\href{20180525-12h23,1x,teeeeshow@yahoo.co.jp}{} 春野
裕征、\href{20180528-2h41,1x,amanojack.v-o-v@i.softbank.jp}{}

\newpage

\newpage
\thispagestyle{empty}
 %←全角スペース
\newpage
\newpage
\thispagestyle{empty}
 %←全角スペース
\newpage


%%% 序文開始
%\layout%レイアウト数値確認用


% 企画本参考例
% \input{echantillon/echantillon-01}
%\begin{Main}
\hypertarget{morue}{%
\subsection[モリュ]{\texorpdfstring{モリュ\footnote{鱈の近縁種。真鱈によく似ている。通常、morue
  (モリュ)の名で売られているのは干し鱈、塩鱈で、フレッシュのものは
  cabillaud
  (カビヨー)と呼ばれることが多いが、魚類の名称としてはモリュ。}}{モリュ}}\label{morue}}

\frsecb{Morue}

\index{morue@morue|(} \index{もりゆ@モリュ|(}

(約10人分\ldots{}\ldots{}1.25 kg)

\hypertarget{morue-au-beurre-noir-ou-au-beurre-noisette}{%
\subsubsection{モリュ・焦がしバター}\label{morue-au-beurre-noir-ou-au-beurre-noisette}}

モリュを沸騰しない程度の温度で茹でる\footnote{pocher (ポシェ)。}。取り出して湯をきり、皮を剥ぐ\footnote{魚を少ない煮汁(クールブイヨン)で茹でるための細長い魚用鍋
  poissonière
  (ポワソニエール)を使い、切り身ではなく丸ごと調理する前提になっていることに注意。}。そのまま皿にのせて、少しの時間、乾かす。上から粗みじん切りのパセリを振りかける。レモン果汁を身にかけてやり、10人分あたり200
gの黒バター\footnote{黒バター beurre noir
  という名称は19世紀、とりわけカレームが好んで使ったもので、要するに焦がしバターに他ならない。焦がす程度は料理人の判断や食べ手の嗜好にも左右されていいが、文字通り「黒」にする必要はまったくないので注意。こんにちでは、焦がしバターは
  beurre noisette
  (ブールノワゼット)と表現することの方が覆いだろう。なお、合わせバターに
  \protect\hyperlink{beurre-de-noisette}{beurre de noisette}
  (ブールドノワゼット)というのがあり、これは\protect\hyperlink{beurre-d-aveline}{ヘーゼルナッツバター}
  beurre d'aveline のことであり、混同しないよう注意。}すなわち焦がしバターを覆いかけてやる。

\hypertarget{raie}{%
\subsection{エイ}\label{raie}}

\frsecb{Raie}

\index{raie@raie|(} \index{えい@エイ|(}

(約10人分は掃除をしていない状態で2 kg)

エイにはいろんな種類があるが、レ・ブクレ\footnote{raie bouclée
  和名ループえい。体長70〜100
  cmで褐色の背にたくさんの輪のような模様がある。}がいちばんいい。

イギリス、ベルギー、オランダでは、エイの皮を剥ぎ掃除した状態で売られており、すぐに調理出来る。

\textbf{基本的な下拵え}\ldots{}\ldots{}掃除をしていない状態のエイは、ブラシで洗い、切り分ける。

これを水1 Lあたり12 gの塩とヴィネガー2
dLを加えた鍋に入れて沸騰させないよう加熱する。火が通ったらすぐに取り出して水気を切り、皮を剥く。すぐに使わない場合は、茹で汁を布で漉してそこに戻し入れておくこと。

\hypertarget{raie-au-beurre-noir-ou-au-beurre-noisette}{%
\subsubsection{エイ・焦がしバター}\label{raie-au-beurre-noir-ou-au-beurre-noisette}}

\protect\hyperlink{morue-au-beurre-noir-ou-au-beurre-noisette}{モリュ・焦がしバター}とまったく同様に調理する。バターの分量も同じ。

\begin{center}\rule{0.5\linewidth}{\linethickness}\end{center}

\hypertarget{boeuf}{%
\section{赤身肉(1)\ldots{}\ldots{}牛}\label{boeuf}}

\frsec{Boeuf}

\index{boeuf@boeuf|(} \index{うし@牛|(}

\hypertarget{cervelles}{%
\subsection{脳}\label{cervelles}}

\frsecb{Cervelles}

\hypertarget{cervelle-a-la-bourguignonne}{%
\subsubsection{牛の脳・ブルゴーニュ風}\label{cervelle-a-la-bourguignonne}}

\frsub{Cervelle à la Bourguignonne}

\index{cervelle@cervelle!bourguignonne@--- à la Bourguignonne}
\index{bourguignon@bourguignon(ne)!cervelle@Cervelle à la ---ne}
\index{うしののう@牛の脳!ふるこーにゆふう@---・ブルゴーニュ風}
\index{ふるこーにゆふう@ブルゴーニュ風!うしののう@牛の脳・---}
\index{のううし@脳(牛)!ふるこーにゆふう@---・ブルゴーニュ風}

牛の脳は厚さ1〜2 cm程度にスライスする\footnote{明記されていないが、下拵えとして、冷水にさらしてよく血抜きをしておくこと(dégorger
  デゴルジェ)。}。ガルニチュールにするマッシュルームをバターで色艶よく炒めた小玉ねぎ\footnote{petits
  oignons glacés (プチゾニョングラセ)\textless{} glacer
  (グラセ)。白系の小玉ねぎで若どりのものであれば下茹では不要。日本に多い、黄色系品種であれば下茹でしておいたほうが望ましいことが多い。}を加える。赤ワインソースいわゆる「\protect\hyperlink{sauce-bourguignonne}{ブルゴーニュ風}\footnote{\protect\hyperlink{sauce-au-vin-rouge}{赤ワインソース}も参照。}」を覆いかけ、ごく弱火で7〜
8分煮る\footnote{ごく弱火で煮込むことを mijoter (ミジョテ)と言う。}。

やや深い皿に盛り、周囲を、ハート形に切ったパンを澄ましバターで揚げたクルトンで飾る。

\hypertarget{matelote-de-cervelle}{%
\subsubsection{牛の脳のマトロット}\label{matelote-de-cervelle}}

\frsub{Matelote de Cervelle}

\index{matelote@matelote!cervelle@--- de Cervelle}
\index{cervelle@cervelle!matelote@Matelote de ---}
\index{まとろつと@マトロット!うしののう@牛の脳の---}
\index{うしののう@牛の脳!まとろつと@---のマトロット}
\index{のううし@脳(牛)!まとろつと@---のマトロット}

水にさらしてよく血抜きをした脳を、あらかじめたっぷり香味を効かせて用意しておいた\protect\hyperlink{cour-bouillon-c}{赤ワイン入りのクールブイヨン}で茹でる。その後、厚さ1〜2
cmにスライスし\footnote{escaloper (エスカロペ)。}、あらかじめ茹でておいた小さなマッシュルームとバターで色艶よく炒めた小玉ねぎを加える。

茹で汁\footnote{すなわちクールブイヨン。}を布で漉し、\protect\hyperlink{beurre-manie}{ブールマニエ}を加えてとろみを付ける。これを脳の上にかけ、やや深さのある皿に盛り付ける。ハート形に切って澄ましバターで揚げた小さなクルトンを添える。

\hypertarget{ux539fux6ce8}{%
\subparagraph{【原注】}\label{ux539fux6ce8}}

火入れのプロセス以外、このマトロットは上記のレシピ「\protect\hyperlink{cervelle-a-la-bourguignonne}{牛の脳・ブルゴーニュ風}」とまったく同じものだ。

\begin{center}\rule{0.5\linewidth}{\linethickness}\end{center}

\hypertarget{pigeonneaux-aux-petits-pois}{%
\paragraph{仔鳩とプチポワ}\label{pigeonneaux-aux-petits-pois}}

\frsub{Pigeonneaux aux petits pois}

\index{pigeonneau@pigeonneau(x)!petits pois@---x aux petits pois}
\index{petits pois@petits pois!pigeonneaux@Pigeonneaux aux ---}
\index{こはと@仔鳩!ふちほわ@---とプチポワ}
\index{ふちほわ@プチポワ!こはと@仔鳩と---}

塩漬豚ばら肉60
gはさいの目に切って下茹でしてから、仔鳩1羽あたり6個の小玉ねぎとともにバターでこんがり炒める。これらを鍋から取り出して油をきり、その鍋で仔鳩の表面に焼き色を付ける。鍋の脂は取り除き、少量の\protect\hyperlink{fonds-brun}{フォン}を注いで鍋の底に貼り付いた肉汁を溶かし出す\footnote{déglacer
  デグラセ。}。\protect\hyperlink{sauce-demi-glace}{ソース・ドゥミグラス}適量と鳩1羽あたり
1 \(\frac{1}{2}\)
dLのプチポワを加える。鍋に塩漬豚ばら肉と小玉ねぎを戻し入れ、ブーケガルニを加える。そのままごく弱火で火を通す。

\begin{center}\rule{0.5\linewidth}{\linethickness}\end{center}

\hypertarget{pates-pour-pates-moules}{%
\subsubsection{型に詰めて焼くパテ用の生地}\label{pates-pour-pates-moules}}

これは2種ある。(1)標準的な生地、(2)ラードを用いた生地。

\textbf{標準的な生地の材料}\ldots{}\ldots{}ふるった小麦粉1 kg、バター250
g、塩30 g、全卵2個、水およそ4
dL。水の量は使う粉の性質に合わせて加減すること。良質の小麦粉であればそれだけ水をよく吸い込む。

小麦粉を台の上に山にして中央に窪みを作る。そこに塩、水、卵、バターを加えてデトランプ\footnote{détrempe
  (デトロンプ)小麦粉が水分を吸って軽くまとまった状態のこと。}にする。

これを手の平の手首に近いあたりを使って伸ばすようにして捏ねる\footnote{fraiser
  (フレゼ)。}。これを2回行ない、滑らかで均質にまとまった生地にする。延し棒で延してから、布で包み、使うまで冷所に置いておく。

\textbf{ラードを使う生地の材料}\ldots{}\ldots{}ふるった小麦粉1
kg、微温い温度にもどして柔らかくしたラード250 g、全卵2個、塩30
g、ぬるま湯4 dL。

標準的な生地と同様に、デトランプを作り、手の平で捏ねる。

\hypertarget{ux539fux6ce8-1}{%
\subparagraph{【原注】}\label{ux539fux6ce8-1}}

生地は出来るだけ24時間前に仕込んでおくこと。そうすれば粘りは出にくい。休ませた生地の方が、捏ねたばかりの生地より圧倒的に扱いやすいし、よりきれいな焼き色に仕上げられる。

\begin{center}\rule{0.5\linewidth}{\linethickness}\end{center}

\hypertarget{croute-aux-rognons}{%
\subsubsection[仔牛腎臓のクルート]{\texorpdfstring{仔牛腎臓のクルート\footnote{パンやパイの固く焼けた皮、殻のこと。}}{仔牛腎臓のクルート}}\label{croute-aux-rognons}}

\frsub{Croûte aux Rognons}

\index{croute@croûte!rognons@--- aux Rognons}
\index{rognon@rognon!croute@Croûte aux ---s}
\index{くるーと@クルート!こうししんぞう@仔牛腎臓の---}
\index{こうししんそう@仔牛腎臓!くるーと@---のクルート}

パン・ジョコ\footnote{pain Joko
  パンの形状の名称のひとつで、バタールより太く短かく、ブール(boule
  ボール形)ほど丸くない形状のもの。もっとも、あまり厳密な定義はないため、バタールとほぼ同等と考えていいだろう。}あるいは見た目に面白い形状のパンを2
\(\frac{1}{2}\) cm
厚にスライスする。形状を整え、底がごく薄くなるようにして、内部をくり抜く。内側にバターを塗り、オーブンに入れてパリっとさせる。

仔牛の腎臓をマッシュルームとともにソテーし、マデイラ酒やシャブリなどで風味付けし、上記のクルートに盛り込む。

\hypertarget{ux539fux6ce8-2}{%
\subparagraph{【原注】}\label{ux539fux6ce8-2}}

このクルートは食パンを正方形や長方形にして作ってもいい。その場合は澄ましバターで揚げること。

\begin{center}\rule{0.5\linewidth}{\linethickness}\end{center}

\hypertarget{peches-imperatrice}{%
\subsubsection[ペッシュ・アンペラトリス
]{\texorpdfstring{ペッシュ・アンペラトリス \footnote{pêche(s)
  (ペッシュ)桃。impératrice
  (アンペラトリス)皇后の、の意。フランスが「帝国」を名乗ったのはナポレオンによる第一帝政(1804〜1814)およびナポレオン・ボナパルトの甥ナポレオン3世ルイ・ナポレオンによる第二帝政(1852〜1870)の期間だけであり、皇后
  impératrice
  はナポレオン1世の后ジョゼフィーヌおよびマリ=ルイーズ、ナポレオン3世の妻ユジェニーの3のみ。料理名としては、à
  la reine
  (アラレーヌ)王妃風、と同様に「豪華な」程度の意味しか持たないが、その料理が創案された時代の反映として見ることは出来るだろう。なお、桃
  pêche には同音同綴で「釣り」の意があり、よく煮た綴りで péché
  (ペシェ)宗教的な意味での「罪」の意。「ペシェアンペラトリス」と読みまちがえないよう注意。「皇后の背教的罪」という意味になってしまうので、洒落の通じる相手にしか許されないことを覚えておきたい。よくあるフランス語の日本語化で宗教的におかしな意味を持つものとしては他に、
  bûche de Noël
  (ビュッシュドゥノエール)がある。これを「ブッシュドノエル」と日本語的に発音すると、bouche
  de Noël となってしまうが、 bouche (ブッシュ)は amuse-bouche
  のブッシュ、すなわち「口」の意であるから「降誕祭の口」という意味になってしまう。現代日本人は宗教にやや無頓着な傾向が多いが、異文化における宗教はしばしば戦争の直接的原因となるなど、デリケートな部分が食のいたるところに潜んでいるので注意したい。}}{ペッシュ・アンペラトリス }}\label{peches-imperatrice}}

\frsub{Pêches Impératrice}

縁の高さのないグラタン皿\footnote{timbale(タンバル)円筒形の比較的浅い型および野菜料理用の深皿のこと。}の底に、キルシュかマラスキーノで香り付けした\protect\hyperlink{riz-pour-entremets}{製菓用ライス}を敷き詰める。

その上に、バニラ風味のシロップで煮た半割りの桃を配する。その上を薄く覆うようにライスの層を重ねる。さらにその上に、\protect\hyperlink{}{アプリコットソース}の層を作る。砕いたマカロン\footnote{ここではマカロン・クラクレに代表される固いタイプのマカロンのこと。詳しくは\protect\hyperlink{panade-frangipane}{パナード・フランジパーヌ}訳注参照。}を散りばめ、オーブンの入口の方に入れて
\footnote{すなわち低めの温度で。}10〜12分加熱する。

表面を焦がさないように注意すること。

\hypertarget{riz-pour-entremets}{%
\subsubsection{製菓用ライス}\label{riz-pour-entremets}}

\begin{itemize}
\item
  材料\ldots{}\ldots{}カロライナ米\footnote{いわゆる短粒種。}500
  g、砂糖300 g、塩1つまみ、牛乳2 L、卵黄
  12個、バニラ1本、レモンかオレンジの硬外皮を削ったもの\footnote{zeste
    (ゼスト)。}適量、バター 100 g。
\item
  作業手順\ldots{}\ldots{}米を洗う。下茹でして湯ぎりをし、さらに微温湯で洗う。再度水気をきり、沸かした牛乳で煮込む。牛乳にはあらかじめバニラの香りを煮出しておき、砂糖、塩、バターを加えておくこと。
\end{itemize}

沸騰し始めたら鍋に蓋をし、弱火のオーブンに入れて25〜30分加熱する。この間、液体の対流現象によって米が鍋底に絶対に触れないよう注意すること。そうでないと、鍋底に米が貼り付いてしまう。

オーブンから出したら、卵黄を加え、泡立て器で丁寧に混ぜる。米粒が完全に形を保っているよう、混ぜる際に米粒を崩さないこと。

\hypertarget{sauce-a-l-abricot}{%
\subsubsection{アプリコットソース}\label{sauce-a-l-abricot}}

よく熟したアプリコットを目の細かい網で裏漉しする。またはアプリコットジャムを用いてもいい。これを28°Béのシロップでゆるめる。火にかけて沸騰させ、丁寧にアクを引く。ソースがスプーンの表面をコーティング出来るくらいに煮詰まったら火から外し、アーモンドミルクかマデイラ酒、キルシュまたはマラスキーノなど好みで香り付けする。

\hypertarget{ux539fux6ce8-3}{%
\subparagraph{【原注】}\label{ux539fux6ce8-3}}

果物のクルート用にこのソースを作る場合は、上等なバター少々を加えてもいい。
\end{Main}
%\hypertarget{poularde-polignac}{%
\subsubsection[肥鶏・ポリニャック]{\texorpdfstring{肥鶏・ポリニャック\footnote{ポリニャック家はフランス有数の貴族の名家のひとつ。マリ=アントワネットの取り巻きのひとりとして知られたポリニャック公爵夫人ヨランド・ド・ポラストロン(通称ポリニャック伯爵夫人、1749〜1793)およびその次男で王政復古期に極端な反動政治家として知られるジュール・ド・ポリニャック(1780〜1847)のいずれかに捧げられた料理名と考えられている。}}{肥鶏・ポリニャック}}\label{poularde-polignac}}

\frsub{Poularde Polignac}

肥鶏を\protect\hyperlink{les-poeles}{ポワレ}する。

胸肉を切り出し、腹の部分の骨を取り除く。\protect\hyperlink{farce-c}{ファルス・ムスリーヌ}
400 gにマッシュルーム100 gとトリュフ100 gを幅1〜2 mm の千切り\footnote{julienne
  (ジュリエーヌ。}にして混ぜ込み、鶏の腹の部分に詰める。

胸肉は線維に垂直に、厚さ1〜2 cmにスライスする\footnote{escaloper
  (エスカロペ)。}。これにファルスを塗り、トリュフのスライスを挟むのを繰り返して元の形状にする。こうして元の形にした肥鶏をオーブンの入口付近に入れて\footnote{すなわちやや低温のオーブンで。}ファルスに火を通す。胸肉に火が通り過ぎないよう注意すること。

皿に盛り付ける。マッシュルームのピュレ
\(\frac{1}{4}\)、ごく細い千切りにしたトリュフとマッシュルーム大さじ2杯を加えた\protect\hyperlink{sauce-supreme}{ソース・シュプレーム}
を全体を覆うように塗る。

\hypertarget{poularde-talleyrand}{%
\subsubsection{肥鶏・タレーラン}\label{poularde-talleyrand}}

\frsub{Poularde Talleyrand}

肥鶏をポワレする。次に胸肉を切り出し、大きめのさいの目に切る。

マカロニ\footnote{本書の定義によれば、\protect\hyperlink{macaroni}{マカロニ}とは円柱あるいは円筒状のパスタ全般を意味し、スパゲッティのような細いものからカンネローニのように太いものまで含まれる。ここではどの程度の太さのものを使うか指示はないが、必ずしも円筒形に穴が空いたものである必然性もあまり考えられず、また、当時は円筒形のものもブカティーニのように長いまま流通することが多かったようで、マカロニを用いるレシピのほとんどで、茹でてから短かく切り揃える、という指示が頻繁に見られる。なお、マカロニのイタリア語である
  maccheroni
  が最初にフランス語に訳されたのはボッカッチョ『デカメロン』のフランス語訳であり、その写本の多くはそのまま
  maccheroni
  と訳さずに転記するか、ややフランス風の綴りに直す程度であるなかで、macaron
  と訳した写本がある。これが macaron
  というフランス語の文献上最古の記録となっている(\protect\hyperlink{panade-frangipane}{パナード・フランジパーヌ}訳注参照)。なお、このボッカッチョの時代の
  maccheroni
  は、こんにちのニョッキのようなものだったというのが定説。さて、本書においてタレーランの名を冠するレシピのほとんどにマカロニが用いられているが、カレーム『19世紀フランス料理』の未完部分をプリュムレがまとめた第4巻、第5巻にタレーランの名を冠したレシピが見られる。鶉のフィレ・タレーラン(t.4,
  p216)とトリュフのタンバル・タレーラン (t.5;
  p.522)の2つだが、マカロニはまったく用いられていない。やや時代が下って、デュボワ\&ベルナール『古典料理』では鶏のピュレ・タレーラン
  (t.1, p.~199)および小さなタンバル・タレーラン (\emph{Ibid}., p.~205),
  七面鳥のフィレ・タレーラン (\emph{Ibid}.,
  p.~178)の3つのレシピも同様で、マカロニはおろかパスタの類はまったく使わない。グフェ『料理の本』(1867年)にある、トリュフのタンバル・タレーラン
  (pp.656-657)は、生地を敷きつめたタンバル型に、厚さ0.5cmにスライスしてバターで炒めたのをひたすら詰めて焼くというもので、ルイ・エブラという料理人の創案と記されている(pp.656-657)。『料理の手引き』とほぼ同時代の著書であるファーヴルの『事典』もこれをそのまま踏襲している(pp.1861-1862)。ファーヴルは他に、ガトー・タレーラン、肥鶏胸肉のエスカロップ・タレーランのレシピを記しており、後者は「古典料理」と注釈がある(pp.1844-1855)。つまり、『料理の手引き』以前の主要な料理書においてタレーランとマカロニを結びつける記述は見つかっていないわけだ。いっぽう、カレーム『パリ風パティスリの本』(1815)にはマカロニ(乾燥品)を茹でて他のラグー(フィナンシエールなど)とともにタンバルに詰めるレシピがpp.89-98まで4種あるが、これらのページにおいてタレーランには触れられていない。『ラルース・ガストロノミック』初版(1938)にはタレーラン自身の項は立てられているが、その項にもガルニチュールにも、プレ、プーラルド、タンバルにもタレーランの名を冠した料理は見あたらない。そのいっぽうでソース・タレーランというレシピが掲載されており、「鶏のヴルテ2
  dLと白いフォン2
  dLを半量に煮詰める。最後にひと煮立ちさせたら生クリーム大さじ4杯とマデイラ酒½
  dLを加える。火から外してバター50 gを加えてよく混ぜ、布で漉す。

  \par

  野菜のミルポワ大さじ1と細かいさいの目に切ったトリュフ大さじ1、赤く漬けた舌肉大さじ1を加える」(p.965)というもの。タレーランは正式にはタレーラン・ペリゴールが家名であるから、結び付けるとしたらむしろトリュフであり、その点では他のタレーランの名を冠したレシピ群の多くは一致している。また、フォワグラも関連性を認められると考えていいだろう。この『料理の手引き』初版および第二版には、鶏のタンバル・タレーランが図版付きで収録されており、これは、ドーム状の型にマカロニの端をファルスを糊でつなぐようにして型の周囲に敷き詰め、ファルスで塗り固めてから、\protect\hyperlink{garniture-talleyrand}{ガルニチュール・タレーラン}を詰める。おそらくはこの第三版以降消えてしまった「タンバル」仕立てとともにガルニチュール・タレーランが考案され、他の料理にも影響しているのだと考えられる。なお、初版から現行版までオードブルの章にある\protect\hyperlink{timbales-talleyrand-petites}{タンバル・タレーラン(小)}は一貫してマカロニを用いておらず、トリュフのピュレが主素材となっていることから、これがタレーランの名を料理名に冠する根拠あるいは由縁としてもっとも重要なものと考えていいだろう。}を茹でて短かく切り、おろしたパルメザンチーズと\protect\hyperlink{sauce-a-la-creme}{ソース・クレーム}であえたものを鶏胸肉と同量ずつ混ぜあわせる。

腹の骨を取り除き、ガラの内部に上で作ったガルニチュールを詰める。肥鶏を元の形にし、最後に\protect\hyperlink{farce-c}{ファルス・ムスリーヌ}で全体を覆う。

表面にトリュフのスライスを王冠状に貼り付けて飾り、バターを塗った紙で包み、中温のオーブンに入れ、(1)
ファルスに適温で火を通し、(2)
中に詰めたガルニチュールが充分に温まるようにする。

皿に盛り付ける。トリュフエッセンスとトリュフの千切りを加えた\protect\hyperlink{sauce-demi-glace}{ソース・ドゥミグラス}を皿の底に大さじ数杯流し入れる。

ソース入れで同じソースを添えて供する。

\hypertarget{timbale-de-pigeonneaux-la-fayette}{%
\subsubsection[仔鳩のタンバル・ラファイエット]{\texorpdfstring{仔鳩のタンバル・ラファイエット\footnote{ラファイエット侯爵(1757〜1834)。アメリカ独立革命とフランス革命の両方で活躍し、フランス人権宣言を起草したことで知られる。「両大陸の英雄」とも称される。原注にあるように、エスコフィエは、フランスとアメリカ合衆国との親善の意味を込めてこの名を冠したのだろう。}}{仔鳩のタンバル・ラファイエット}}\label{timbale-de-pigeonneaux-la-fayette}}

\frsub{Timbale de pigeonneaux La Fayette}

\begin{enumerate}
\def\labelenumi{\arabic{enumi}.}
\item
  高さが低めの\footnote{原文 une moule plus large que haut
    直訳すると「高さより幅の広い型」。}型でタンバルの殻を焼く\footnote{本書にはpâte
    brisée「ブリゼ生地」は出ておらず、本書により忠実に作るのであれば\protect\hyperlink{pate-a-foncer}{フォンセ生地}を用いることになるだろう。}。
\item
  \protect\hyperlink{mirepoix-fine}{ボルドー風ミルポワ}をバターで蒸し煮し\footnote{étuver
    (エチュヴェ)。}、その鍋でガルニチュール用のエクルヴィス60尾を、白ワイン
  3 dLとコニャック 1
  dL、塩、白こしょう、ピンクペッパーを加えて火を通す。火が通ったらすぐに尾の殻を剥き、煮汁は目の細かいシノワで漉してトリュフのスライス
  100
  gを加え、そこにエクルヴィスの尾の身を浸して温めておく。エクルヴィスの殻などはバター50
  gを加えて細かくすり潰す。これを生クリーム入りの\protect\hyperlink{sauce-bechamel}{ベシャメルソース}
  \(\frac{1}{2}\)
  Lに混ぜ込み、軽くひと煮立ちさせてから布で漉して、保温しておく。
\item
  上の作業と平行して、小さな仔鳩10羽をごく薄い豚背脂のシートで包んで焼く。火が通ったら、背脂のシートを外し、胸肉の部分の皮を剥いて、胸肉を切り出す。仔鳩を焼いた鍋に白ワインを注いでデグラセし、溶かした\protect\hyperlink{glace-de-viande}{グラスドヴィアンド}大さじ数杯とトリュフのスライス100
  gを加え、そこに胸肉を浸して保温しておく。
\item
  中位の太さのマカロニ400
  gを塩湯でやや固めに茹で、よく湯ぎりをしてから、バター100
  g、おろしたてのパルメザンチーズ150
  g、挽きたてのこしょう1つまみ、上で作ったエクルヴィスバター入りのベシャメルソースの
  \(\frac{1}{3}\) を加えてあえる。また、ベシャメルソースの別の
  \(\frac{1}{3}\) はエクルヴィスの尾をあえる。残りは仔鳩の胸肉とあえる。
\end{enumerate}

\textbf{盛り付け}\ldots{}\ldots{}タンバルの殻に、まずマカロニの
\(\frac{2}{3}\)
量を詰め、その上にエクルヴィスとトリュフの半量を詰める。再度マカロニを詰め、その上に仔鳩の胸肉を環状に並べる。その中央にエクルヴィスの尾の身の残りを詰める。最後に残ったマカロニで上面を覆い、大きなトリュフのスライスを何枚か飾る。

\hypertarget{ux539fux6ce8}{%
\subparagraph{【原注】}\label{ux539fux6ce8}}

これは、筆者がはじめてニューヨークを訪れた際に友人たちが催してくれた昼食会の際に、筆者自らが作った料理。

\hypertarget{homard-clarence}{%
\subsubsection[オマール・クラレンス]{\texorpdfstring{オマール・クラレンス\footnote{イギリス王族の公爵位として知られる。料理としては、本書以前のレシピの例は見あたらず、本書、\protect\hyperlink{filets-de-sole-clarence}{舌びらめのフィレ・クラレンス}ス{]}(\#filets-de-sole-clarence)の原注において、必ずこうすべきという共通理解があるものではない、と述べられていることからも、明確な定義は得難い。この名称を冠するのはもっぱら魚および甲殻類の料理で、そのほとんどが\protect\hyperlink{sauce-mornay}{ソース・モルネー}または\protect\hyperlink{sauce-new-burg}{ソース・ニューバーグ}にカレー風味を足したもの。本書では舌びらめのフィレ・クラレンスがレシピ本文ではソース・モルネーだけで作る指示だが、原注において、カレー粉を加えた\protect\hyperlink{sauce-americaine}{ソース・アメリケーヌ}に代えるケースを示唆されている。}}{オマール・クラレンス}}\label{homard-clarence}}

\frsub{Homard Clarence}

オマールは\protect\hyperlink{cout-bouillon-e}{クールブイヨン}で茹で、火が通ったらすぐに取り出して湯をきる。

微温くなるまで冷めたら、縦2つに切る。尾の身を取り出し、やや斜めに厚さ1〜
2
cmの輪切りにして、\protect\hyperlink{fumet-de-poisson}{魚のフュメ}またはマッシュルームの茹で汁少々を加えて保温しておく。

胴の身とクリーム状の部分を取り出し、これを鉢に入れて、生クリーム大さじ
2杯を加えてすり潰し、目の細かい網で漉す。これを、カレー風味の\protect\hyperlink{sauce-bechamel}{ベシャメルソース}
2 \(\frac{1}{2}\) dLに加える。

2つに割った胴のそれぞれに\protect\hyperlink{riz-a-l-indienne}{インド風ライス}を
\(\frac{2}{3}\)
程詰め、その上に輪切りにて保温しておいたオマールの尾の身を、トリュフのスライスを挟んで交互になるように盛り付ける。

用意しておいたカレー風味のベシャメルソースの一部をオマールに軽く塗る。温めておいた長い皿に盛り付ける。

\ldots{}\ldots{}ソースの残りを別添で供する。

\hypertarget{sole-grillee-aux-huuxeetres-a-l-americaine}{%
\subsubsection{舌びらめのグリル焼き、牡蠣添え・アメリカ風}\label{sole-grillee-aux-huuxeetres-a-l-americaine}}

\frsub{Sole grillée aux huîtres à l'Américaine}

舌びらめはグリル焼きにしてもいいし、バターをレモン果汁を加えてほとんど水気のない状態でやや低めの温度で火入れをしてもいい。同様の方法はフィレにおろした舌びらめにも合う。が、いちばん多いのはグリル焼き。火入れの方法がいずれであっても、舌びらめは温めた皿に盛り、提供直前に、少量のダービーソース\footnote{\protect\hyperlink{devilled-sauce}{デビルソース}
  訳注参照。}を沸かして軽く火を入れた牡蠣6個を周囲に飾る。

すぐに、揚げたてのパン粉にパセリのみじん切り1つまみを加え、舌びらめに覆いかける。

\hypertarget{laitues-a-la-moelle}{%
\subsubsection{レチュ・骨髄添え}\label{laitues-a-la-moelle}}

\frsub{Laitues à la Moelle}

レチュを\protect\hyperlink{braisage-des-legumes}{ブレゼ}し、皿に盛り付ける。その上に、やや低温で加熱した牛骨髄の大きなスライスを環状に飾る。軽くバターを加えた\protect\hyperlink{jus-de-veau-lie}{とろみを付けたジュ}をかけて供する。

\hypertarget{peches-imperatrice}{%
\subsubsection[ペッシュ・アンペラトリス
]{\texorpdfstring{ペッシュ・アンペラトリス \footnote{pêche(s)
  (ペッシュ)桃。impératrice
  (アンペラトリス)皇后の、の意。フランスが「帝国」を名乗ったのはナポレオンによる第一帝政(1804〜1814)およびナポレオン・ボナパルトの甥ナポレオン3世ルイ・ナポレオンによる第二帝政(1852〜1870)の期間だけであり、皇后
  impératrice
  はナポレオン1世の后ジョゼフィーヌおよびマリ=ルイーズ、ナポレオン3世の妻ユジェニーの3のみ。料理名としては、à
  la reine
  (アラレーヌ)王妃風、と同様に「豪華な」程度の意味しか持たないが、その料理が創案された時代の反映として見ることは出来るだろう。なお、桃
  pêche には同音同綴で「釣り」の意があり、よく煮た綴りで péché
  (ペシェ)宗教的な意味での「罪」の意。「ペシェアンペラトリス」と読みまちがえないよう注意。「皇后の背教的罪」という意味になってしまうので、洒落の通じる相手にしか許されないことを覚えておきたい。よくあるフランス語の日本語化で宗教的におかしな意味を持つものとしては他に、
  bûche de Noël
  (ビュッシュドゥノエール)がある。これを「ブッシュドノエル」と日本語的に発音すると、bouche
  de Noël となってしまうが、 bouche (ブッシュ)は amuse-bouche
  のブッシュ、すなわち「口」の意であるから「降誕祭の口」という意味になってしまう。現代日本人は宗教にやや無頓着な傾向が多いが、異文化における宗教はしばしば戦争の直接的原因となるなど、デリケートな部分が食のいたるところに潜んでいるので注意したい。}}{ペッシュ・アンペラトリス }}\label{peches-imperatrice}}

\frsub{Pêches Impératrice}

縁の高さのないグラタン皿\footnote{timbale(タンバル)円筒形の比較的浅い型および野菜料理用の深皿のこと。}の底に、キルシュかマラスキーノで香り付けした\protect\hyperlink{riz-pour-entremets}{製菓用ライス}を敷き詰める。

その上に、バニラ風味のシロップで煮た半割りの桃を配する。その上を薄く覆うようにライスの層を重ねる。さらにその上に、\protect\hyperlink{sauce-a-l-abricot}{アプリコットソース}の層を作る。砕いたマカロン\footnote{ここではマカロン・クラクレに代表される固いタイプのマカロンのこと。詳しくは\protect\hyperlink{panade-frangipane}{パナード・フランジパーヌ}訳注参照。}を散りばめ、オーブンの入口の方に入れて\footnote{すなわち低めの温度で。}10〜12分加熱する。

表面を焦がさないように注意すること。

\hypertarget{riz-pour-entremets}{%
\subsubsection{製菓用ライス}\label{riz-pour-entremets}}

\begin{itemize}
\item
  材料\ldots{}\ldots{}カロライナ米\footnote{いわゆる短粒種。}500
  g、砂糖300 g、塩1つまみ、牛乳2 L、卵黄
  12個、バニラ1本、レモンかオレンジの硬外皮を削ったもの\footnote{zeste
    (ゼスト)。}適量、バター 100 g。
\item
  作業手順\ldots{}\ldots{}米を洗う。下茹でして湯ぎりをし、さらに微温湯で洗う。再度水気をきり、沸かした牛乳で煮込む。牛乳にはあらかじめバニラの香りを煮出しておき、砂糖、塩、バターを加えておくこと。
\end{itemize}

沸騰し始めたら鍋に蓋をし、弱火のオーブンに入れて25〜30分加熱する。この間、液体の対流現象によって米が鍋底に絶対に触れないよう注意すること。そうでないと、鍋底に米が貼り付いてしまう。

オーブンから出したら、卵黄を加え、泡立て器で丁寧に混ぜる。米粒が完全に形を保っているよう、混ぜる際に米粒を崩さないこと。

\hypertarget{sauce-a-l-abricot}{%
\subsubsection{アプリコットソース}\label{sauce-a-l-abricot}}

よく熟したアプリコットを目の細かい網で裏漉しする。またはアプリコットジャムを用いてもいい。これを28°Béのシロップでゆるめる。火にかけて沸騰させ、丁寧にアクを引く。ソースがスプーンの表面をコーティング出来るくらいに煮詰まったら火から外し、アーモンドミルクかマデイラ酒、キルシュまたはマラスキーノなど好みで香り付けする。

\hypertarget{ux539fux6ce8-1}{%
\subparagraph{【原注】}\label{ux539fux6ce8-1}}

果物のクルート用にこのソースを作る場合は、上等なバター少々を加えてもいい。

\hypertarget{peches-eugenie}{%
\subsubsection[ペッシュ・ユジェニー]{\texorpdfstring{ペッシュ・ユジェニー\footnote{このレシピは第四版から。索引には
  Pêches Impératrice Eugénie
  とあるため、ナポレオン3世妃ユジェニーを指すと解釈可能。ただし、索引のミスの可能性もある。その場合には具体的に誰を指しているのかは不明ということになる。}}{ペッシュ・ユジェニー}}\label{peches-eugenie}}

いい具合に熟した桃を選ぶこと。丁寧に種を取り除き、皮を剥く。これをやや深い皿に、フレーズ・デ・ボワと交互になるように盛り込む。キルシュとマラスキーノをスプーン数杯ずつ上からかけて、蓋をして氷の上に1時間置いて冷やす。

提供直前に、よく冷えたシャンパーニュ風味のサバイヨンソースを桃に覆いかける。

\hypertarget{peches-imperatrice-froides}{%
\subsubsection{ペッシュ・アンペラトリス}\label{peches-imperatrice-froides}}

桃は半割りにし、バニラ風味のシロップで煮て、そのまま冷ます。桃をシロップから取り出して、よく水気を取り除く。半割りにした桃の種を抜いた穴にたっぷりとバニラアイスを詰め込んでいき、半割りにする前の桃の形状と大きさになるようにする。桃の側面にはよく煮詰めたアプリコットソースを塗り、シロップで炒りつけたアーモンドを細かくして、その上を桃を転がしてまぶす。

フランボワーズのジャムを塗って乾かしたタルトの台にジェノワーズを敷き、キルシュとマラスキーノを浸み込ませた上に桃を盛り付ける。

糸状にした飴を上から覆いかぶせる。



\frontmatter

%\setlength{\parindent}{1\zw}

\begin{Main}

\hypertarget{preface-heston-blumenthal}{%
\section{ヘストン・ブルメンタールによる英訳第5版への序文}\label{preface-heston-blumenthal}}

\vspace{1\zw}

僕は独学でシェフになった。16才の頃、フランスを旅してミシュランの星付きレストランを訪れ、そのときの体験にすっかり魅せられた\ldots{}\ldots{}それは美味しい料理だけじゃなく、レストランの情景やいろんな音、匂い、それに料理をまるで芝居の情景のごとくプレゼンテーションする様子にもだ。僕は一瞬にして、自分がやりたいのはこういうことなんだと理解した。

そんなわけで、学校を辞めて、僕独自の奇妙ともいえる料理修業をはじめた。ミシュランのガイドブックやゴミヨの本を夢中になって読み、年に一度はフランスに行って、それらの本で賞賛されている店に行って食事をし、『フランスグルメツアー』や『ラルース・ガストロノミック』に出ているクラシカルなフランス料理を何度も何度もくりかえし作ってみた。

そしてもちろん、このエスコフィエ『料理の手引き』も。

オーギュスト・エスコフィエが19世紀末から20世紀初頭にかけて、もっとも偉大かつ急進派のシェフだったのは疑うべくもない。オマール・アメリケーヌやピーチメルバのような新しい料理を無数に創作すると同時に、オートキュイジーヌの原則全体を見直して再解釈し、決まりごとのやかましいガルニチュールや重たいソース、やたらと仕事の手間のかかる派手な盛り付けなどは廃したり正していったわけだ。(エスコフィエの口癖のひとつに「シンプルに作れ」というのがあった。そう、まさにその通りの意味にすべきなんだ)。エスコフィエの関心が厨房の中にとどまらずすごい広がりを持っていたことに、僕はいつもインスパイアされていると言っていい。エスコフィエは食品科学、とりわけ食品保存について関心を持っていたんだ。トマト缶詰製造やマギーブイヨンの開発にも関わり、自らのブランドで瓶詰めのソースやピクルスも開発した。

エスコフィエという人物は長い影響を及ぼしている。エスコフィエが最初に確立したやりかたで、現代のレストランはいまも運営されている。ブリガードと呼ばれる明確な命令系統のシステムを仕上げたことで、調理場での料理の作り方はまるっきり変わった。料理長の指示は部門シェフに伝わり、彼らから各料理人に伝えられる。それぞれの料理人は自分の担当している作業に責任を持つ、というわけだ。エスコフィエはまた、料理を食卓にお出しする方法も変えた。全部の料理をずらりと一度に食卓に並べる「フランス式サービス」をやめて、ロシア式サービスつまりコース料理が順を追って提供される方式を支持した。このやりかたはこんにちまでずっと続けられているんだ。僕たちは、オーギュストのおかげで食事をしているみたいなものなんだ。

エスコフィエの正確な指示と創意工夫は彼の『料理の手引き』のすべてのページに溢れている。エスコフィエは『料理の手引き』が「ただのレシピ集である以上に役に立つ本」となることを期待していると書いた。実際、まさしくその通りになっているわけだ。エスコフィエの細部にいたるまでの注意深さは伝説になっている。自分の顧客それぞれの食の好き嫌いを書き留め続けていたという。この『料理の手引き』の制作にあたって、エスコフィエ自身が助手に、わざわざこのために用意させた計量秤を送り付け、それぞれの材料の分量を実際に計らせたという。こんなにもレシピに対して綿密であることにこだわったということは、それらのレシピが実際に作ってとても楽しいものになった(これこそがフランス料理の基礎教育の凄いところだ)のはもちろんだが、そのおかげで、世界の超一流シェフのひとりに創造力=想像力についての深い洞察力を与えてくれたのは事実だ。

シェフたちのなかには、クラシカルな伝統なんてすっとばして、エスコフィエみたいな保守的で時代遅れのくせに偉そうな人物なんて否定したいという思いに駆られる者もいる。それはとんでもない間違いだ。エスコフィエの科学、テクノロジー、マーケティング、厨房の改革への関心、それに料理のプレゼンテーションをシンプルにすること、どれも僕としては、非常に現代的な問題だと思う。知れば知るほど、素晴しい料理というものは伝統をぶち壊すことによってではなく、伝統を新たな方向へ導くことによって生み出されるのだと分かったんだ。\ruby{革命}{レボリューション}よりもむしろ\ruby{進化}{エボリュー
ション}すべきなんだ。

そういう意味で、エスコフィエのこの『料理の手引き』よりもいいスターティングポイントはないと思う。

\begin{flushright}
ヘストン・ブルメンタール、2011年
\end{flushright}

\newpage

\hypertarget{avant-propos}{%
\chapter{序}\label{avant-propos}}

\fifteenq
\setstretch{1.3}

もう20年も前のことだ。本書の着想を我が尊敬する師、今は亡きユルバン・デュボワ\footnote{Urbain
  Dubois (1818〜1901)。19世紀後半を代表する料理人。}先生に話したのは。先生は\ruby{是非}{ぜひ}とも実現させなさいと強く勧めてくださった。けれども忙しさにかまけてしまい、\ruby{漸}{ようや}く
1898年になって、フィレアス・ジルベール\footnote{Philéas Gilbert
  (1857〜1942)。19世紀末から20世紀初頭に活躍した料理人。料理雑誌「ポトフ」を主宰した。}君と話し合い協力をとりつけることが出来た。ところがまもなく、カールトンホテル開業のために私はロンドンに呼び戻され、その厨房の準備や運営に忙殺されることとなった\footnote{エスコフィエはセザール・リッツの経営するホテルグループにおいて料理に関わる重要な役割を一手に担っていた。1890年〜1897年にかけてロンドンのサヴォイホテルの総料理長を勤めた後、1898年にはパリのオテル・リッツの、1899年にはロンドンのカールトンホテルの開業に携わり、1920
  年までカールトンホテルで総料理長を務めた。}。本書の計画を実現させるために落ち着いた時間を取り戻さねばならなくなってしまった。

1898年から放置したままだった本書に再び着手出来たのは、多くの同僚たる料理人諸君の助力と、友人でもあるフィレアス・ジルベール君とエミール・フェチュ\footnote{Emile
  Fétu 生没年不詳。}君の献身的な協力を得られたからに他ならない。この一大事業を完成させることが出来たのは、ひとえに皆の励ましと、とりわけ辛抱強く、粘り強く仕事を手伝ってくれた二人の共著者\footnote{ジルベールとフェチュを指しているが、初版にはこの二人の他にも共著者として4人の名が挙げられている。第二版以降は共著者としてジルベールとフェチュの名しかクレジットされていない。第二版は初版から構成も含め大幅な改訂が行なわれた。その作業を実際に行なったのがジルベールとフェチュだったために、他の共著者のクレジットが抹消されたと考えられる。なお、現行の第四版にはエスコフィエの名しかクレジットされていない。}のおかげだ。

私が作りたいと思ったのは立派な書物というよりはむしろ実用的な本だ。だから、執筆協力者の皆には、作業手順を各自の考えにもとづいて自由にレシピを書いてもらい、私自身は、40年にわたる現場経験に即して、少なくとも原理原則、料理における伝統的基礎を明確に説明するのに専念した\footnote{このとおりであれば、具体的なレシピの執筆者は上記のように複数おり、エスコフィエ自身は各章、各節における「概説」に相当する部分と「原注」を担当したことになる。とはいえ、第二版以降については、口頭によるコメントの「聞き書き」的なものが含まれている可能性が。原書の文体における「ゆらぎ」から推測されよう。つまり、エスコフィエは本書の制作にあたっても、やはり「総料理長」であった、と考えていい。そしてそのことはエスコフィエの偉業である本書の価値をいっかな減ずるものではない。}。

本書は、かつて私が構想したとおりとは言い難い出来だが、いずれはそうなるべく努めねばなるまい。それでもなお、現状でも料理人諸君にとって大いに役立つものと信じている。だからこそ、本書を誰にでも、とりわけ若い料理人にも買える価格にした\footnote{1903年の初版の売価は、\href{http://gallica.bnf.fr/ark:/12148/bpt6k65768837}{フランス国立図書館蔵}のものの表紙には、フランス国内で12フランと記したシールが貼られている。また、\href{https://archive.org/details/b21525912}{リーズ大学図書館蔵の第二版}にも同様に国内売価12フランのシールが貼られている。1912年の第三版も同じく12フランだった(\href{http://gallica.bnf.fr/ark:/12148/bpt6k96923116}{フランス国立図書館蔵}のものに価格を示すシールはないが、訳者個人蔵のものには12フランと記されたシールが貼られている)。なお、辻静雄は「1903年の初版発売当時は、800ページでたった8フラン、全く破格の値段だった」(「エスコフィエ 偉大なる料理人の生涯」、『辻静雄著作集』、新潮社、1995年、729〜730頁)と記しているが、その数字の典拠は示されていない。現在と当時の通貨価値、物価の違いが分りにくいため、この「破格に安い」という言葉にはやや疑問が残るだろう。1900年当時の書籍広告において『料理の手引き』初版と同様の八折り版800ページの料理書が、フランス装10フラン、厚紙の表紙のものが11フランとあるため、初版の12フランという価格は、むしろ料理書としては一般的だったと考えられる。つまり、豪華本ではなく、普通に利用できる料理書だということを強調しているに過ぎないと解釈すべきところだろう。なお、八折り判というのは書籍の大きさを表す用語で、概ね縦20〜25
  cm、横12〜16
  cm程度。この序文でことさらに「実用性」や入手しやすい価格であることが強調されているのは、何度も言及されているデュボワとベルナールの名著『古典料理』が四折り判(概ね縦45
  cm、横30 cm)の豪華本であったことを意識していたためとも推測されよう。}。そもそも若い料理人諸君にこそこの本を読んで
\ruby{貰}{もら}いたい。今はまだ初心者であったとしても、20年後には組織のトップに立つべき人材なのだから。

私はこの本を豪華な装丁の\footnote{かつてフランスでは、大判の紙の両面に印刷して折ったものを糸で綴じただけの状態(いわゆる「フランス装」)で販売された本を、書店で買い求めた者が別途、業者に製本、装丁の依頼をして自分の好みの想定にさせることが一般的に行なわれていた。}、書棚の飾りのごときにはして欲しくない。そうではなく、いつでも、どんな時でも手元に置いて、分からないことを常に明らかにしてくれる\ruby{盟友}{めいゆう}として欲しい。

本書には五千を越える\footnote{初版、第二版は「五千近い」。第三版になってようやくこの表現になった。}レシピが掲載されているが、それでも私は、この教本が完全だとは思っていない。たとえ今この瞬間に完璧であったとしても、明日にはそうではないかも知れぬ。料理は進化し、新しいレシピが日々創案されている。まことに困ったことだが、版を重ねる毎に新しい料理を採り入れ、古くなってしまったものは改善せねばなるまい。

ユルバン・デュボワ、エミール・ベルナール\footnote{Emile Bernard
  (1827〜1897)。クラシンスキ将軍の料理人を務めた。}両氏の著作\footnote{デュボワとベルナールの共著は他にもあるが、ここでは『古典料理』(1856年)を指している。}に昔から慣れ親しみ、その巨大な影がなおも料理の地平を覆い尽している現在、私としては本書がその後継になって欲しいと思っている。カレーム以後、最高の料理の高みに逹した二人に対し、ここであらためて心から敬意を表させていただきいと思う。

調理現場を取り巻く諸事情により、私は、デュボワ、ベルナール両氏がもたらしたサービス(給仕)面での革新\footnote{\protect\hypertarget{service-russe}{19世紀後半に一般的となった
  「ロシア式サービス」のこと。中世以来、格式の高い宴席では、卓上に大
  皿の料理が一度に何種も並べられ、食べる者がそれぞれ好きなように取り
  分けていた。そして卓上の料理がほぼなくなると、また何種類もの皿が卓
  上に並べられる、というのが数回繰り返された。19世紀中頃から、献立を
  食べる順に1種ずつ、大皿料理の場合は食べ手に見せて回ってから、給仕
  が取り分けて供する方式に変えたものがロシア式サービスである。これと
  対比するかたちで旧来の方式をフランス式サービスと呼ぶようになった。
  ロシア式サービスでは、食卓に大皿を並べない代わりに、花を飾りナフキ
  ンを美しく折るなどの工夫により卓上も洗練されたものとなっていった。
  19世紀パリに駐在していたロシア帝国の外交官クラーキンが提唱したと言
  われている。デュボワとベルナールの『古典料理』序文において詳述され
  ている。}}に対し、こんにちのようなとりわけスピードが重視される目まぐるしい生活リズムに合わせて、より大きな変更を加えざるを得なかった。そもそも物理的理由から、料理を載せる飾り台\footnote{\protect\hypertarget{socle}{socle ソークル。パンや米、ジュレな
  どで作った、料理を盛り付けるために銀の盆の上に据える飾り台。カレー
  ムの時代、つまり19世紀前半にはその装飾に凝ることが多かった。食べも
  ので作られてはいるが、料理の一部ではなく、あくまで装飾的要素でしか
  なかった。この飾り台はロシア式サービスの時代になっても豪華絢爛たる
  宴席においては重要なものとして扱われており、デュボワとベ}ルナール『古典料理』でも相応のページ数を割いて説明がなされている。}を廃止して、シンプルな盛り付けにする新たなメソッドと新たな道具を考案する必要があったのだ。デュボワ、ベルナール両氏が推奨した壮麗な盛り付けを私自身が行なっていた頃はもちろん、今なお二方の思想にはまったく共感している。冗談でこんなことを言っているのではない。しかし、カレームを信奉する者たちは、装飾の才があるが\ruby{故}{ゆえ}に、時代にもはや\ruby{似}{そぐ}わなくなってしまった作品に対して改良を加えようとはしなかった。時代に合わせて改良することこそまさに重要ななのに、だ。本書で奨励している盛り付けは、少なくともそれなりの期間、有用であり続けると思う。全ては変化する。姿を変える。それなのに、装飾芸術の役割が変化しないとなどと主張するとは
\ruby{蒙昧}{もうまい}ではないか。芸術は流行によって栄えるものだし、流行のように移ろいやすいものだ。

だが、カレームの時代にはこんにちと同じく\ruby{既}{すで}にあり、料理が続く限りなくならないだろうものがある。それが\ruby{料理の基礎}{フォン
ド キュイジーヌ}だ。そもそも、料理が見た目にシンプルになっても料理そのものの価値は失なわれないが、その逆はどうだろう?
人々の味覚は絶え間なく洗練され続け、それを満足させるために料理そのものも洗練されることになる。こんにちの余剰活動が精神におよぼす悪影響に打ち\ruby{克}{か}つためには、料理そのものがいっそう科学的な、正確なものとなるべきなのだ。

その意味で料理が進歩すればする程、我々料理人たちにとって、19世紀、料理の行く末に大きく影響を与えた三人の料理人の存在は大きなものとなるだろう。カレームとデュボワ、ベルナールはともすれば技術的側面ばかり評価されるが、料理芸術の基礎において何よりも優れているのだ。

既に物故した名だけ挙げるが、確かにグフェ\footnote{Jules Gouffé
  (1807〜1877)。著書多数。主著『料理の本(1867年)は前半が家庭料理、後半が高級料理(オート・キュイジーヌ)の2部構成になっており、レシピもまず材料表を掲げた後に調理手順を説明するという現代の書き方に近く、挿絵も多く分りやすい。この『料理の手引き』とともに19世紀後半のフランス食文化史における名著のひとつ。19世紀前半からのヴィアールやオドが版を内容を増補しながら版を重ねたのに対して、この本は再版の際もほとんど異同がない点もまた特徴のひとつ。}、ファーヴル\footnote{Joséph
  Favre
  (1849〜1903)。スイス生まれの料理人で、パリ、ドイツ、イギリス、ベルギー等において活躍した。著書『料理および食品衛生事典』
  (1884〜1895年)。この『事典』に収録されているレシピの数は5,531であり(番号が振られている)、エスコフィエがレシピ数5千という表現にややこだわりを示しているように思われるのも、ほぼ同時代の出版物であるファーヴルの『事典』を意識していた可能性はある。}、エルーイ\footnote{Edouard
  Hélouis(生没年不詳)。イギリスのアルバート王配(ヴィクトリア女王の夫)(1819〜1861)やイタリアのヴィットーリオ・エマヌエーレ二世(1820〜1878)に仕えたという。著書『王室の晩餐』(1878年)。}、ルキュレ\footnote{『実践的料理』(1859年)の著者C.
  Reculetのこと。}はとても素晴らしい著作を残した。だが、『古典料理』という\ruby{稀代}{きたい}の名著に\ruby{比肩}{ひけん}し得るものはひとつとしてない\footnote{デュボワ、ベルナールの『古典料理』を称揚するあまりにこのような表現になっていると思われるが、そもそもファーヴルの著書は『事典』であって教本ではない。また、グフェ『料理の本』については、掲載レシピ数は少ないものの、記述の明快さと、技術の習熟度合いに合わせて家庭料理と高級料理(haute
  cuisine
  オートキィジーヌ)の二部構成にするなどの配慮から、こんにちでも充分に料理の入門〜中級教本として高く評価出来るものだろう。}。

料理人諸君に、新たに本書を使っていただくにあたり、言うべきことがある。いろいろな料理書、雑誌を読み散らかすのもいいが、偉大な先達の不朽の名著はしっかり読み込むように、と。\ruby{諺}{ことわざ}にあるように「知り過ぎることなはい」のだ。学べば学ぶ程、さらに学ぶべきことは増えていく。そうすれば、柔軟な思考が出来るようになり、料理が上達するためのより効果的な方法を知ることも出来るだろう。

本書を\ruby{上梓}{じょうし}するにあたって\ruby{唯}{ただ}ひとつ望むこと、切に願う\ruby{唯一}{ゆいいつ}のことは、上記の点において、本書の対象たる読者諸君が我が\ruby{言}{げん}に耳を傾け、実践するさまを見ることに尽きる。\nopagebreak

\begin{flushright}
A. エスコフィエ \nopagebreak
\end{flushright}

1902年11月1日

\newpage

\hypertarget{introduction-deuxieme-edition}{%
\section[第二版序文]{\texorpdfstring{第二版序文\footnote{この第二版序文は文体が初版序文と異なり、とりわけ前半部分については、いわゆる「悪文」と見なさざるを得ないものとなっている。また、前半と後半でも文体の「ゆらぎ」のようなものが認められる。内容から判断するかぎり、エスコフィエ自身の言葉であることは確かだが、末尾に署名がなく日付のみ記されていることも含めて考えると、ジルベールとフェチュによる「聞き書き」によって作成された可能性も完全には否定できないと思われる。}}{第二版序文}}\label{introduction-deuxieme-edition}}

\normalsize
\setstretch{1.1}
\vspace*{1\zw}

ここに第二版を上梓するに至ったわけだが、二人の共著者による熱意あふれる仕事のおかげで、私の強い期待をさらに越える本書の成功が約束されたも同然だろう。だからこそ、共著者両君および本書の読者諸君に心からの謝辞を申しあげる次第だ。また、ありがたいことに、称賛の言葉を寄せてくださった方々と、貴重な批判をくださった方々にも御礼申しあげる。批判については、それが正当なものと思われる場合については、本書に反映させるべく努めさせていただいた。

かくも多くの人々に本書を受け入れていただけたことへの謝意を表するには、本書における技術的な価値を高め、初版ではロジカルにレシピを分類しようとしたが故に生じた欠点を解消する他ないだろう。それは、調理理論とレシピを損なうことなしに、本書の計画段階において簡単に済まさざるを得ないと思われたテーマについて\ruby{能}{あた}う限り肉薄することでもある。私たちは本文の見直しをするとともに、多くのレシピを追加した。そのほとんどは調理法と盛り付けにおいて、こんにちの顧客のニーズを\ruby{鑑}{かんが}みて着想したものであり、そのニーズが正当かつ実現可能な範囲において、顧客への給仕のペースが日増しに加速していく傾向をも考慮に入れたものだ。こういった傾向は数年来まさしく際立ってきているが\ruby{故}{ゆえ}に、我々としても常に気を配っておかねばならぬ。

「料理芸術」というものは、その表現形態において、社会心理に左右されるものだ。社会から受ける衝撃に逆らわぬことも必要であり、\ruby{抗}{あらが}
えぬことでもある。快適で安楽な生活がいかなる心配事にも乱されることのないような社会であれば、未来が保証され、財をなす機会もいろいろあるような社会であれば、料理芸術はたゆまぬことなく驚異的な進歩を遂げるだろう。料理芸術とは、ひとが得られる悦びのうちでもっとも快適なもののひとつに寄与しているのだから。

反対に、安穏とした生活の出来ぬ、商工業からもたらされる\ruby{数多}{あま
た}の不安で頭がいっぱいになるような社会において、料理芸術は心配事でいっぱいの人々の心のごく限られた部分にしか美味しさを届けられない。ほとんどの場合、諸事という渦巻きに巻き込まれた人々にとって、食事をするという必要な行為はもはや悦びではなく、辛い義務でしかないのだ。

\ruby{斯}{か}くのごとき生活習慣は\ruby{嘆}{なげ}いていい、\ruby{否}{い
な}、嘆くべきことなのだ。食べ手の健康という観点からも、食べたものを胃が受け付けないという結果になるとしたら、それは絶対に生活習慣が悪いのだ。そういう結果を抑える力は私に出来る範囲を越えている。そういう場合に調理科学が出来ることといえば、軽率な人々に\ruby{能}{あた}うかぎり最良の食べものを与えるという対症療法だけなのだ。

顧客は料理を早く出せと言う。それに対して私たち料理人としては、ご満足いただけるようにするか、失望させてしまうことのどちらかしか出来ない。料理を早く出せという顧客の要求を拒む方法があるとするなら、それ以上の方法で顧客にご満足いただけるようにすることしかない。だから、私たちは顧客の気まぐれの前に折れざるを得ないのだ。これまで私たちが慣れ親しんできた仕事のやり方では、これまでの給仕のスタイルでは、顧客の気まぐれに応えることが出来ぬ。意を決して仕事の方法を改革すべきなのだ。だがひとつだけ、変えてはならぬ、手をつけてはならぬ領域がある。料理ひとつひとつのクオリティだ。それは、料理人にとって仕事のベースとなるフォンや事前に仕込んでおいたストック類がもたらすゆたかな風味に他ならぬ。私たちは既に、盛り付けの領域においては改革に着手した。足手まといにしかならぬ多くのものは既に姿を消したか、いままさに消え去らんとしている。料理の飾り台\footnote{socle
  (ソークル)、\protect\hyperlink{socle}{「序」p.ii、訳注6}参照。}、料理の周囲の装飾\footnote{bordure
  ボルデュール。本書においてもガルニチュールの扱いにおいてこの指示はあるが、19世紀のものと比較するとかなりシンプルな内容になっている。}、飾り串\footnote{hâtelet
  アトレ。一方の端に動物などの姿の装飾の施された銀製の串に、トリュフやクルヴェット(海老)などを事前に別の串(ブロシェット)で焼いてからこの飾り串に刺し直し、それを大きな塊肉や丸鶏、大型の魚
  1尾の料理に刺した。19世紀初頭、カレームの時代に全盛となり、その著書『パリ風料理』において詳述されている。19世紀末まではこの装飾がなされることが多かった。また、その飾り串そのものが美麗な装飾品であるためにコレクションの対象になっていた。}などのことだ。この方向性は推し進められると思う。これについては後述しよう。私たちはシンプルであるということを極限まで追究したい。それと同時に、料理の風味や栄養面での価値を増すことも目指している。料理はより軽い、弱った胃にも優しいものにしたいと考えている。私たちはこの点にのみ尽力したい。料理において役をなさない大部分はすっかり剥ぎ取ってしまいたいと考えているのだ。一言でまとめると、料理は芸術であり続けつつも、より科学的なものとなるだろうし、その作り方はいまだ経験則に基づいただけのものばかりであるが、ひとつのメソッド、偶然などに左右されない正確なものになっていくことだろう。

こんにちは料理の過渡期にある。古典料理メソッドの愛好者はいまなお多く、私たちもそれを理解し、その思想に心から共感するところもある。だが、食事というものがセレモニーであり、かつパーティであった時代を懐しんでどうするというのだ?
古典料理がこんにちの美食家に至福の時を与えるために力を発揮出来る場がどこにあるというのだ?
いったいどうすれば、美食と宴の神コモス\footnote{フランス語 Comus
  (コミュス)。ラテン語では同じ綴りでコムスと読む。ギリシア、ローマ神話における、悦びと美食の神。18世紀の料理本作家マランの主著は『コモス神の贈り物』がタイトル。}に捧げ物を供えるという幸せな機会を毎回得られるのだろうか?だから私たちは本書において、個人的な創作よりむしろ伝統的なフランス料理のレシピ集として、こんにちの料理のレパートリーから姿を消してしまったものも残すことに固執した。その名に値する料理人なら、機会さえ与えられたら王侯貴族も近代の大ブルジョワもひとしく満足させるためには、知っておくべきものなのだ。時間のことなんぞ気にもせぬ穏かな美食家の方々にも、時こそ全てと言わんばかりの金融家やビジネスマンたちにも満足していただくために。だから、本書が新しいメソッドに偏ったものだという非難にはあたらない。私はただ単に、料理芸術の進化の歩みをたどり、いまの時代に即しつつ、食べ手すなわち食事会の主催者と招待客の皆様の意向を絶対的なものとして、それに従いたいと願っているだけなのだ。食べ手の意向に対して私たち料理人は
\ruby{頭}{こうべ}を垂れて従うことしか出来ぬのだから。

私たちは、料理の美味しさを損なうことなくより早く料理を提供できるような方法を、料理人各人が自らの嗜好を犠牲にすることなしに探求すべく
\ruby{誘}{いざな}うことこそが、料理人諸君にとって有益と信じている。全体として、私たちのメソッドはまだまだ日々のルーチンワークに依存し過ぎているものだ。顧客の求めに応えるため、私たちは既に仕事のやり方をシンプルなものにせざるを得なかった。だが、残念ながらいまだ\ruby{途}{み
ち}\ruby{半}{なか}ばに過ぎぬと感じている。私たちは自己の信念をしっかり堅持しており、どうしようもない場合にのみ自説を曲げることもある。だから、装飾に満ちた飾り台を廃止した一方で、盛り付けに時間のかかる厄介で複雑なガルニチュールは残してある。こういったガルニチュールを濫用することはガストロノミーの観点から言って、常に間違っているのは事実だが、残しておくべきものと思われる。それを求める顧客あるいは食事会主催者に絶対に従う必要のある場合はとりわけそうだ。ごく稀にとはいえ、料理の美味しさを損なうことなくそれらを実現可能なこともあるからだ。時間と金銭、広くてスタッフの充実した会場、という3つの本質的要素を最大限活用可能な場合のことだが。

通常の厨房業務においては、ガルニチュールをかなりシンプルな、せいぜい3〜
4種の構成要素からなるものに減らさざるを得なくなっている。そのガルニチュールを添える料理がアントレであれルルヴェ\footnote{19世紀前半まで主流であった「フランス式サービス」つまり、一度に多くの料理の皿を食卓に並べるという給仕方式において、ポタージュを入れた大きな深皿が空くと、それを給仕が下げて、豪華な装飾を施した大きな塊肉の料理がポタージュを置いてあった場所に据えられた。これを
  \protect\hypertarget{releve}{relevé}ルルヴェ(交代したもの、の意)と呼んだ。エスコフィエの時代にはフランス式サービスではなくロシア式サービスに移っており、大きな塊肉の料理や大型の魚1尾まるごとを大皿で出し、給仕が切り分けて配膳するようになっていたが、名称はそのまま残った。Entréeアントレ(もとは「入口」の意)は現代において「前菜」の意味で用いられているが、食卓に大皿で並べられた肉料理(場合によっては魚料理も含む)の総称としてこの語が用いられていた。本書はそれを踏襲している。本書においてルルヴェおよびアントレに分類されている料理の多くは現代においてコース料理の「メイン」に相当するものが多く、実際、英語ではコース料理のメインのことを現在でもこの語で表わすことが多い(前菜はappetizerアペタイザーと呼ぶ)。}であれ、牛・羊肉料理であれ、家禽であれ魚料理であれ、そうせざるを得ない。そのようにして構成要素を減らしたガルニチュールは、素早い皿出しが要求される場合には必ず、ソースと同様に別添で供するのがいい。その場合、盛り付けは奇抜というくらいシンプルなものとなるが\ldots{}\ldots{}メインの料理はより冷めない状態で、より早く、よりきれいに供することが可能になる。給仕が料理を取り皿に分けてお客様に出すにせよ、お客様が大皿を自分たちで受け渡して取り分けるにせよ、サービス担当者は安心して仕事が出来るし、そのほうが容易だ。メインの大皿が山盛りになることはないし、その上に盛り付けられたいろいろな素材のガルニチュールも簡単に取ることが出来るからだ。

こんにちの他のシステムだと、料理を載せるための台や装飾のための飾り串を作り、さらに料理の周囲にガルニチュールを配置するのに、看過出来ぬ程の時間を要していた。こういう盛り付けというのは、料理そのものがさして大きくないものであっても、食べ手の人数が少ない場合であっても、大面積の皿を用いる必要があった。だから、お客様が料理を自分たちで受け渡して取り分ける必要がある場合などは、お客様にとっても、サービス担当者にとってもまことに窮屈なものであった。これは、複雑な構成のガルニチュールの持つ大きな欠点のひとつとして無視できないことだ。他の欠点というのは、あらかじめ盛り付けを行なうことによって美味しさが減じてしまうこと、食べ手が少人数の場合には必然的に、料理を見せて周る間に冷めてしまうこと、などがある。こういう愉快とは言えぬことの結果は何とも情けないことになる。つまり、お客様に大皿に盛り付けた料理をお見せするのはほんの一瞬だけ、お客様は多少なりとも豪華で精密に盛り付けられた料理をちらりと見る暇があるかないか、ということだ。昔日のごとき豪華壮麗な料理を供することの可能な場所もこんにちでは少なくなってきたが、それ以外のところでもこういった悪習が頑固なまでに続けられているというのは、それが昔からの習慣だということでしか説明がつかぬ。

給仕のスピードを容易に上げるために、大きな塊肉の料理でない場合には毎回、下の図のごとき四角形の深皿を出来るだけ用いるよう是非ともお勧めしたい。温かい料理でも、冷製の料理でも、この皿は非常に優れたものであるから、その目的において厨房に備えておくべきものとして他の追随を許さないと言える
\footnote{この段落は、初版の序文の後にある「盛り付け方法をシンプルにすることについて」という挿絵付きの節の内容を短かく縮めたために、ややわかりにくいものになっている。ただし、第二版および第三版においては序文の最後に皿の挿絵が添えられている。}。

繰り返しになるが、本書が新しい方法を勧めているからといって、偏見で古典的なものを悪いと断じているのでは決してない。私たちは、料理人諸君に、顧客たちの生活習慣や味の好みを研究し、自らの仕事をそれらに適合させるよう
\ruby{誘}{いざな}いたいと思っているだけなのだ。我々料理人にとって高名な師とも呼ぶべきカレームは、ある日、同業たる料理人のひとりとおしゃべりをしていた際に、その料理人が仕えている主人の洗練さに欠けた食事の習慣や下卑た味覚を苦々しげに語るのを聞かされたという。その食事の習慣と味覚に憤慨して、自分が人生をかけて追究してきた知的な料理の原則を曲げてまで仕え続けるくらいなら、いっそ辞めてしまいたいと思っている、と。カレームはこう答えた。「そんなことをするのは君のほうが間違っているよ。料理において原則なんていくつも存在しないんだ。あるのはひとつだけ、仕えているお方に満足していただけるか、ということだけなんだよ」と。

今度は我々がその答を考える番だ。自分たちの習慣やこだわりを、料理を出す相手に押しつけるなどと言い張るとしたら、まったくもって馬鹿げたことだ。我々料理人は食べ手の味覚に合わせて料理することこそが第一でありもっとも本質的なことなのだと、私たちは確信している。

私たちがかくも安易に顧客の気まぐれにおもねったり、過度なまでに盛り付けをシンプルにするせいで、料理芸術の価値を下げ、単なる仕事のひとつにしてしまっている、と非難する向きもあるだろう。\ldots{}\ldots{}だがそれは間違いだ。シンプルであることは美しさを排するものではない\footnote{この序文における名句のひとつ。ただし、エスコフィエの時代における「シンプル」とポストモダン以降の時代であるこんにちの「シンプル」はもはや具体的な意味がまったく違うことに留意。もちろん、理念として普遍的な価値を持つ名句であることは確かだろう。}。

ここで、本書の初版において盛り付けについて述べた部分を繰り返すことをお許しいただきたい。

「どんなにささやかな作品にも自らの最高の印をつけられる才というのは、その作品をエレガントで歪みのないものに見せられるわけで、技術というものに不可欠だと私は信じている。

だが、職人が美しい盛り付けを行なうことで自らに課すべき目的とは、食材を他に類のない方法で節度をもって用いつつ大胆に配置することによってのみ、実現されるのだ。未来の盛り付けにおいて絶対に守るべきこととして、食べられないものを使わないこと、シンプルな趣味のよささこそが未来の盛り付けに特徴的な原則となるだろうことを、認めるべきなのだ。

そのような仕事を成し遂げるために、能力ある職人にはいくつもの手段がある。トリュフ、マッシュルーム、固茹で卵の白身、野菜、舌肉などの食べられるものだけを用いて、素晴らしい装飾を組み合わせ、無限に展開できるのだ。

王政復古期\footnote{1814年ナポレオンが退位して国外へ亡命、ルイ18世を戴く王政へ回帰した時期。1830年まで続いたが7月革命でブルボン家は断絶し、その後オルレアン朝による七月王政が1848年まで続いた。}に料理人たちによって流行した複雑な盛り付けの時代は終わった。だが、特殊な例になるが、古い方法で盛り付けをしなければならない場合もあり、そういう時は何よりもまず、盛り付けにかかる時間と利用できる手段を見積らなくてはならない。土台の形状を犠牲にしなくても、装飾の繊細さを忘れなくても、風味ゆたかな素材を軽んじたり劣化させてしまっては、価値のないものにしかならないのだ」。

以上の見解はずっと変わっていない。料理は進歩する(社会がそうであるように)。だが常に芸術であり続けるのだ。

例えば、1850年から人々の生活習慣、習俗が変化したことを皆が認めるにやぶさかでないように、料理もまた変化するのだ。デュボワとベルナールの素晴しい業績は当時のニーズに応えたものだ。だが、たとえ二人がその著書と同じく永遠の存在であったとしても、彼らが称揚した形態は、料理の知識として、我々の時代の要求に応えうるものではない。

私たちは二人の名著を尊重し、敬愛し、研究しなくてはならない。それはカレームとともに、料理人の仕事の\ruby{礎}{いしずえ}たるものだ。だが、書いてあることを盲目的に真似るのではなく、私たち自身で新たな道を切り
\ruby{拓}{ひら}き、私たちもまたこの時代の習俗や慣習に合わせた教本を残すべきと考える次第だ。

\begin{flushright}
1907年2月1日
\end{flushright}

\newpage

\hypertarget{introduction-troisieme-edition}{%
\section{第三版序文}\label{introduction-troisieme-edition}}

\vspace*{1\zw}

『料理の手引き』第三版を同業たる料理人諸賢に向けて上梓するにあたり、絶えず本書を好意的に支持してくださったことと、多くの方々から著者一同にお寄せくださった励ましのお言葉に対し、あらためて深く御礼申しあげる次第だ。

第二版序文の内容につけ加えるべきことは何もない。というのも、第二版序文で料理という仕事について申しあげたことは、1907年当時も今も変わっていない事実だからだし、今後も長くそうであり続けるだろう。とはいえ、この第三版は内容を精査し、かなりの部分を改訂してある。かつては予測でしかなかったことを実証し、この『料理の手引き』初版の序文においてエスコフィエ氏\footnote{この表現から、第三版序文がエスコフィエ自身ではなく、フィレアス・ジルベールかエミール・フェチュのいずれか、あるいは二人によって書かれたと判断される。}が以下のように書かれた約束も果せたと思う。「本書には五千近くもの\footnote{初版および第二版では「五千近い」となっており、第三版で「五千以上」と表現が変更された。}レシピが掲載されているが、それでも私は、この教本が完全だとは思っていない。たとえ今この瞬間に完璧であったとしても、明日にはそうではないかも知れぬ。料理は進化し、新しいレシピが日々創案されているのだ。まことにもって不都合なことだが、版を重ねる毎に新しい料理を採り入れ、古くなってしまったものは改良を加えねばなるまい。」

この言葉が、前回の第二版から300ページを増やしたことの説明となっているわけで、この新版でいくつかの変更を我々が必要と考えた理由でもある。

\begin{enumerate}
\def\labelenumi{\arabic{enumi}.}
\item
  判型の変更\ldots{}\ldots{}あえて判型を大きくすることで、より扱いやすいものとしたこと\footnote{初版および第二版はいわゆる「八折り版」約21.5
    cm×13.5 cmであったのに対し、第三版は約24 cm×16
    cm、つまり現代のB5版よりほんの少し小さめの判型。}
\item
  巻末の目次の組みなおし\ldots{}\ldots{}当初は料理の種類別であったが、本書全体の項目をアルファベット順にまとめたこと\footnote{原文ではTable
    des
    Matière「目次」とあるが、これは巻頭の章を示す目次のことではなく、巻末の「索引」のこと。}
\item
  時代遅れになったと思われるレシピを相当数削除し、その代わりとしてこの数年の間に創案され好評を博したレシピを追加したこと
\end{enumerate}

既に大著であって本書にこれらの変更を加えるために、我々は第二版の巻末に付されていた献立のページを削除せざるを得なかった。

献立についても内容を一新し、多くの献立例を追加して、『メニューの本』という独立した書籍として、この第三版と同時に刊行する予定となっている。この『メニューの本』において我々は献立とその説明文はもちろんのこと、大規模な厨房における日々の業務配分を示す表を入れておいた。

このように別冊とすることで、献立の作成という非常に重要な問題を適切に展開し、ゆとりを持って論じることが可能となったわけだ。

この新刊『メニューの本』は料理人諸賢だけではなくメートルドテル、食事施設の責任者に必携のものとなった。さらには必要なものを奇抜なまでに単純化してしまう家庭の主婦にとっても必携となろう。我々は上記の改良点が、これまで多くの好意的見解をお寄せくださった料理関わる皆様方に、好意的に受け容れていただけると信じている。また、料理芸術の栄光のもと未来に続くモニュメントを建てるべく努めた我々のささやかなる尽力が、料理芸術に利をもたらさんことを信じる次第だ。

\begin{flushright}
1912年5月1日
\end{flushright}

\hypertarget{introduction-quatrieme-edition}{%
\section{第四版序文}\label{introduction-quatrieme-edition}}

\vspace*{1\zw}

『料理の手引き』第三版刊行当時(1912年5月)から後、他の職業、産業と同様に料理界もまた大いなる危機に見舞われた\footnote{第一次世界大戦(1914〜1918)による社会的影響を指している。フランスは戦中から戦後にかけて激しいインフレに見舞われた。なお、この第四版から出版社がそれまでのラール・キュリネールからフラマリオン社に変わった。}。こんにちもなお料理は厳しい試練にさらされている。しかしながら、料理界はその試練に耐えてきたし、戦後のこの辛い時期に終止符を打ち、料理界がさらに前進し始めるのもさして遠いことではないと信じている。だが、目下のところ、あらゆる食材の異常なまでの高騰により、料理長諸賢が責務を果すことがひどく難しくなっている。料理長がその責務を果すということの困難さを経験上よく知っているからこそ、今回の版において我々は、多くのレシピ、とりわけガルニチュールについて、その本質的なところを曲げることなしに、よりシンプルなものにすることにこだわった。

さらに、もはやあまり興味を持たれないであろうレシピは全て削除して、その代わりに近年創案されたレシピを収録することとした。

したがって、料理人諸賢および料理に関心を持つ皆様方に向けてこの『料理の手引き』第四版を上梓するにあたり、旧版同様、皆様に温かく受け容れていただけると信じる次第である\footnote{原書の文体から、この序文も第三版序文と同様に、ジルベールとフェチュによって書かれた可能性も考えられる。}。

\begin{flushright}
1921年1月
\end{flushright}

\newpage
\small
\setstretch{1.0}

\hypertarget{remarque-sur-la-simplification-des-procedes-de-dressage}{%
\section{【参考】盛り付けをシンプルにするということ(初版のみ)}\label{remarque-sur-la-simplification-des-procedes-de-dressage}}

本書では、かつては料理の盛り付けによく用いられた飾り串\footnote{hâtelet
  アトレ。}、縁飾り \footnote{bordure ボルデュール。}、クルトン\footnote{菱形やハート形にしたパンを揚げたもの。}、チョップ花\footnote{papillote
  パピヨット。紙製で、骨付き肉の先端を飾るもの。}などを使う指示がほとんど出てこない。著者としては、盛り付け方法を近代化すると同時に、ほぼ完全に上記のものどもを削除しなくしてしまいたいとさえ考えたくらいだ。

我らが先達が考えていたような盛り付けには、長所がたったひとつしかない。皿を荘厳に、魅力的な姿にすることで、料理を味わう前に、食べ手の目を楽しませ、喜んでいただくということだ。

だが、そうした盛り付けの作業は複雑で難しいものであり、かなりの時間を必要とする。比較的少人数の宴席でないかぎりは、こうした盛り付けは事前に用意しておく必要がある。そのようにして作られた料理は、それを置いておく場所のことを考えに入れないとしても、必ずといっていい程、冷めてしまっている。また、料理を載せる台や縁飾り、飾り串に費す時間も考えなくてはならないし、そういった装飾にかかる費用も考えなくてはならない。忘れてはならないことだが、そのように装飾した皿の見た目の調和がとれている時間というのは、その皿をお客様にお見せする間だけなのだ。メートルドテルのスプーンが料理に触れるやいなや、かくも無惨な姿となりお客様の目には不快なものとなってしまう。こういう不都合はなんとしても改善しなければならなかったのだ。

ここで図に示すような四角形の皿を採用したことで、上記のような問題は解決したと考えている。この皿はパリのリッツホテルで初めて用いられ、ロンドンのカールトンホテルにおいて正式に採用されることとなったものだ。この皿を用いることの利点は絶大で、これを用いない盛り付けなどもはや考えられない程だ。この皿は場所をとらず、皿の内側に盛り付けられた料理は冷めることがない。蓋との距離が近いから保温されているわけだ。魚や肉の切り身は上に重ねて盛るのではなく、ガルニチュールとともに並べて盛り付けることが出来る。そうすることで、最初に給仕されるお客様から最後に給仕される方まで、料理は美味しそうな見た目を保つことが出来るのだ。その結果、クルトンやチョップ花、皿の上にしつらえる料理を載せる台や縁飾り、飾り串、昔の給仕で用いられた面倒なクロッシュ\footnote{cloche
  主に金属製で半球形の保温を目的としたディッシュカバー。}は不要なものとなる。

この皿は冷製料理にもまた便利に使うことが出来る。周囲に氷を積み重ねて囲うか、薄い氷のブロックの上に盛り付ければ、飾りには、ごく繊細なジュレだけていい。そのような繊細なジュレを使うのは昔の方法では不可能だった。かくして、邪魔にさえ思える飾り台も、皿の底の飾りも、アトレも必要なくなった。ショフロワは1切れずつ並べて、周囲を琥珀色のとろけるようなジュレで満たしてやればいい。ムースはもはや「つなぎ」をまったく、あるいはほとんど必要としない。こういうことが、冷製料理の芸術的な見た目を、豪華さや美しさという点でいっかな失なうことなく可能となるのだ。

この新式の什器とそれによって実現可能となる料理に習熟することについて料理人諸君にお報せすることは我々の義務であると考える。利点がとても大きいので、あえて申しあげるが、これを使うことが、給仕を素早く、きれいに、経済的に、そして文句ないまでに実践的なものにする唯一の方法である。

\hypertarget{avertissement-premier-edition}{%
\section{【参考】初版はしがき}\label{avertissement-premier-edition}}

本書はある特定の階層の料理人を対象としているものではなく、全ての料理人が対象であるため、本書のレシピは、経済的観点や料理人が実際に利用可能な手段に応じて、改変できるものだということを述べておきたい。

本書に収められたレシピはすべて、グランドメゾンでの仕事における原則にもとづいて組み立てて調整してある。だから、より格下の店舗などでも、必然的に量を減らせば作れるだろうし、適価で提供出来るようにもなるだろう。

ひとつひとつの項目において、いろいろな飲食を提供する形態を網羅するようにレシピを書くことが不可能だったということは理解されよう。料理人自身が自主性をもって本書の内容を補えるし、そうすべきなのだ。ある者たちにとって非常に大切なことが、大多数の者にとってはそこそこの興味しか引かず、一般的に見たら無益で幼稚に思われることだってあるのだ。

だから、本書に収録したレシピは最大の分量でまとめられたものを考えるべきであり、必要に応じて、各人の判断および物理的に出来る範囲に合わせて、量を減らして作るといい。
\normalsize \setstretch{1.0}

\end{Main}

\setlength{\parindent}{0pt}


%%% 本文開始
\mainmatter


%%%%% 本文開始
% \layout%レイアウト数値確認用
%\makeatletter
%\set\@mainmatter=\true
%\makeatother
%%%原稿ファイル読み込み
%

%%% Chapitre I. Saucesa
%%%%I. Sauces 
\href{未、オスマゾームなどについての補足、カレームとの比較}{}
\href{未、原文対照チェック}{} \href{未、日本語表現校正}{}
\href{未、その他修正}{} \href{未、原稿最終校正}{}

\hypertarget{sauces}{%
\chapter{I. ソース Sauces}\label{sauces}}

\hypertarget{les-fonds-de-cuisine}{%
\section{フォン、その他のストック}\label{les-fonds-de-cuisine}}

\frsec{Les Fonds de Cuisine}

\index{fonds@fonds} \index{ふおん@フォン}

\normalsize
\setstretch{1.0}

本書は実際に厨房で働く料理人を対象としたものだが、まず最初に料理のベースとして仕込んでストックしておくもの\footnote{本書での
  fonds の語は fond (基礎、土台)、fonds
  (資産、資本)、そして料理用語として一般に用いられているフォン、のトリプルミーニングになっている。そのまま「フォン」と訳したいところだが、日本語の場合「出汁」としての意味合いが強いため、本文中では分りやすさを重視してやや冗長に「料理のベースとして仕込んでストックしておくもの」のように訳している。}について少々述べておきたい\footnote{この部分は経営者に向けて書かれているようにも読めるが、エスコフィエの時代以降、料理人がオーナーシェフとして経営に携わるケースが激増したことを考えると、その先見の明に驚かざるを得ない。}。我々料理人にとって重要なものだからだ。

ここで述べる料理のベースとして仕込んでストックしておくものは、実際、料理の土台そのものであり、それなしでは美味しい料理を作ることの出来ない、まず最初に必要なものだ。だからこそ、料理のベースとして仕込んでおくストックはとても重要であり、いい仕事をしたいと努めている料理人ほどこれらを重視している。

これらは、料理において常に立ち戻るべき出発点となるものだが、料理人がいい仕事をしたいと望んでも、才能があっても、それだけでいいものを作ることは出来ない。料理のベースを作るにも材料が必要なのだ。だから、必要な材料は良質のものを自由に使えるようにしなければならない。

筆者としては、むやみな贅沢には反対だが、それと同じくらい、食材コストを抑え過ぎるのも良くないと考えている。そんなことをしていては、伸びる筈の才能の芽を摘んでしまうばかりか、意識の高い料理人ならモチベーションの維持すら出来ないだろう。

どんなに優秀な料理人だって、無から何かを作り出すことは不可能だ。期待される結果に対して、素材の質が劣っていたり量が足りないことがあれば、それでも料理人にいい仕事をしろと要求するなど言語道断である。

料理のベースとして仕込んでおくストックに関する重要ポイントは、必要な材料は質、量ともに充分に、惜しげもなく使えるようにすることだ。

ある調理現場で可能なことが、別の調理現場では不可能な場合があるのは言うまでもない。料理人の仕事内容は顧客層によっても変わる。到達すべき目標によって手段も変わるということだ。

そういう意味で、何事も相対的なものであるとはいえ、こと料理のベースとして仕込んでストックすべきものに関しては絶対に外してはならないポイントがあるわけだ。組織のトップがこの点で出費を惜しんだり、コスト面で過度に目くじらを立てるようでは、美味しい料理なんて出来るわけがないのだから、現実に厨房を仕切っている料理長を批判する資格もない。そんなのが根拠のない言い掛かりなのは明らかだ。素材の質が悪かったり、量が足りないのであれば、料理長が素晴しい料理を出せないのは言うまでもあるまい。ぶどうの搾りかすに水を加えて醗酵させた安ワインを立派な瓶に詰めてしまえば高級ワインになると思う程に馬鹿げたことはないのだ。

料理人は、必要なものを何でも使っていいなら、料理のベースとして仕込んでおくストックにとりわけ力を入れるべきであり、文句のつけようのない出来になるよう気を使うべきだ。そこに手間隙かけていればそれだけ厨房全体の仕事がきちんと進むのだから、注文を受けた料理をきちんと作れるかどうかは、結局のところ、料理のベースとなる仕込み類にどれだけ手間\ruby{隙}{ひま}をかけるかということなのだ。

\newpage

\hypertarget{principaux-fonds-de-cuisine}{%
\section{主要なフォンとストック}\label{principaux-fonds-de-cuisine}}

\frsec{Principaux Fonds de Cuisine}

料理のベースとして仕込んでおくべきものは主として\ldots{}\ldots{}

\begin{itemize}
\tightlist
\item
  \textbf{コンソメ・サンプルとコンソメ・ドゥーブル}
\item
  \textbf{茶色いフォン、白いフォン、鶏のフォン、ジビエのフォン、魚のフォン}\ldots{}\ldots{}これらはとろみを付けたジュ、基本ソースのベースになる
\item
  \textbf{フュメ、エッセンス}\ldots{}\ldots{}派生ソースに用いる
\item
  \textbf{グラスドヴィアンド、鶏のグラス、ジビエのグラス}
\item
  \textbf{茶色いルー、ブロンドのルー、白いルー}
\item
  \textbf{基本ソース}\ldots{}\ldots{}エスパニョル、ヴルテ、ベシャメル、トマト
\item
  \textbf{肉料理用ジュレ、魚料理用ジュレ}
\end{itemize}

\vspace{1\zw}

以下も日常的に使う料理のベースとして仕込んでおくものとして扱う。

\begin{itemize}
\tightlist
\item
  \textbf{ミルポワ、マティニョン}
\item
  \textbf{クールブイヨン、肉および野菜用のブラン}
\item
  \textbf{マリナード、ソミュール}
\item
  \textbf{肉料理用ファルス、魚料理用ファルス}
\item
  \textbf{ガルニチュールに用いるアパレイユ}、など\ldots{}\ldots{}
\end{itemize}

\vspace{1\zw}

本書は上記を順に説明していく構成にはなっていない。グリル、ロースト、グラタン等の調理技法についても順を追っていくわけではない。料理の種類ごとに一定の位置、つまりは関連の深い料理の章の冒頭において説明していくことになる。

\vspace{1\zw}

そのようなわけで、本書においては以下のようになる\ldots{}\ldots{}

\begin{itemize}
\tightlist
\item
  フォン、フュメ、エッセンス、グラス、マリナード、ジュレの説明\ldots{}\ldots{}
  \textbf{ 第1章 ソース}
\item
  コンソメおよびそのクラリフィエ、ポタージュの浮き実についての説明\ldots{}\ldots{}\textbf{第3章
  ポタージュ}
\item
  ファルスとガルニチュール用アパレイユの作り方\ldots{}\ldots{}\textbf{第2章
  ガルニチュール}
\item
  クールブイヨン、魚料理用ファルス等\ldots{}\ldots{}\textbf{第6章
  魚料理}
\item
  グリル、ブレゼ、ポワレの調理理論\ldots{}\ldots{}\textbf{第7章 肉料理}
\end{itemize}

\newpage

\hypertarget{section-grandes-sauces-de-base}{%
\section{基本ソース}\label{section-grandes-sauces-de-base}}

\frsec{Grandes Sauces de Base}

\index{そーす@ソース!きほん@基本---}
\index{sauce@sauce!grandes@Grandes ---s de Base}

\begin{itemize}
\item
  \textbf{およびそれらを組み合せたり煮詰めるなどの方法で作る派生ソース}
\item
  \textbf{イギリス風ソース(温製および冷製)}
\item
  \textbf{いろいろな冷製ソース}
\item
  \textbf{ブール・コンポゼ(ミックスバター)}
\item
  \textbf{マリナード}
\item
  \textbf{ジュレ}
\end{itemize}

\hypertarget{osbservation-sur-la-sauce}{%
\section{概説}\label{osbservation-sur-la-sauce}}

ソースは料理においてもっとも主要な位置にある。フランス料理が世界に冠たるものであるのもひとえにソースの存在によるのだ。だから、ソースは出来るかぎり手間をかけ、細心の注意を払って作るようにしなければならない。

ソースを作るうえでその基礎となるのが何らかの「ジュ」である\footnote{ここではジュといわゆるフォンが同じ意味で使われている。}。すなわち、茶色いソースは「茶色いジュ」(エストゥファード)から作る。ヴルテには「澄んだジュ(白いフォン\footnote{日本の調理現場で「白いフォン」を意味する「フォン・ブラン」は主として鶏のフォンを指すことが多いが、本書で扱われている白いフォンのうち標準的なものは仔牛肉、家禽類をベースとしており、鶏のフォンは別途説明されている。})を使う。ソースを担当する料理人はまず第一に、完璧なジュを作るところから始めなければならない。キュシー侯爵
\footnote{1767-1841。19世紀の著名な美食家。
  著書に『食卓の古典』(1843)がある。料理名にキュシーの名を冠したものも多い。}が言うように、ソース担当の料理人は「頭脳明晰な化学者\footnote{原文
  chimiste。現代は分子ガストロノミーが盛んだが、料理を作る過程で起きる現象や結果を「化学」で説明しようとする試みは少なくともカレームまで遡ることが出来る。\protect\hyperlink{fonds-brun}{茶色いフォン}のレシピにおいて言及されるオスマゾームという想像上の物質もその範疇に含まれるだろう。また、化学の前身たる「錬金術」的概念は中世以来いくつかの料理書において散見される。}でありかつ天才的なクリエイターで、卓越した料理という建造物のいわば大黒柱たる存在」なのだ。

昔のフランス料理\footnote{本書において「昔の料理」と表現される場合は概ね17〜18世紀末と考えていい。}では、素材に串を刺してあぶり焼きするローストを別にすれば、どんな料理も「ブレゼ」か「エチュヴェ」のようなものばかりだった。だが、その時代には既に、フォンが料理という大建築の丸天井の\ruby{要}{か
なめ}だったし、材料コストが重視されるこんにちの我々と比べたら想像も出来ないくらい贅沢に材料を使ってフォンをとっていたのだ。実際、アンヌ・ドートリッシュ\footnote{17世紀に絶対王政を確立したルイ14世の母。}がスペインからルイ13世に嫁いだ際に随行してきたスペインの料理人たちによってフランス料理にルーを用いる方法が伝えられたが\footnote{ルーがスペインからもたらされたというのは逸話、伝承の域を出ない。}、当時はほとんど看過された。ジュそれ自体で充分だったからだ。ところが時代が下り、料理におけるコストの問題が重視されるようになった。ジュはその結果、貧相なものになってしまった。その美味しさを補うものとして、ルーを用いて作るソース・エスパニョルが欠くべからざる存在となった。

ソース・エスパニョルはその完成度の高さゆえに成功をおさめたわけだ。だが、すぐに当初の目的を越えた使い方をされるようになった。19世紀末には本当にこのソースが必要な場合以外にも使われたわけだ。ソース・エスパニョルの濫用によって、どんな料理も固有の香りのない、全部の風味の混ざりあったのっぺりとした調子のものばかりになってしまった。

ようやく近年になって、料理の風味がどれも同じようなものであることに批判が集まってきて、その結果として激しい揺り戻しが起きたのだった。グランドキュイジーヌでは、透き通ったような薄い色合いでしかも風味のしっかりした仔牛のフォンが見直されつつある。そのようなわけで、ソース・エスパニョルそれ自体の重要性はだんだん減っていくだろうと思われる。

ソース・エスパニョルが基本ソースとして扱われるべき理由は何か? ソース・エスパニョルそれ自体に固有の色合いや風味というものはなく、これらはどんなフォンを用いて作るかで決まる。まさにこの点にソース・エスパニョルの長所が存するのだ。補助材料としてルーを加えるが、ルーにはとろみを付けるという意味しかなく、風味にはまったく寄与しない。そもそも、ソースを完璧に仕上げるためには、とろみ以外のルーに含まれる成分はソースからほぼ完全に取り除いてしまっても差し支えはない。不純物を丁寧に取り除いたソースにはルーに含まれていたでんぷん質だけが残っているわけだ。だから、ソースの口あたりを滑らかなものにするために必要なのがでんぷん質だけなら、純粋なでんぷんだけを用いる方がずっと簡単で、作業時間も大幅に短縮されるし、その結果として、ソースを火にかけ過ぎてしまうようなミスも防げる。将来的には、小麦粉ではなく純粋なでんぷんでルーを作るようになるかも知れない。

料理界の現状を\ruby{鑑}{かんが}みるに、\ul{ソース・エスパニョル}と
\ul{とろみを付けたジュ}をそれぞれ使い分けざるを得ない。これにはさまざまな理由があるが、大きな仕立てのブレゼや、羊や仔羊以外を材料にしたラグーでは、肉汁が煮汁に染み出してきて美味しくなるわけだから、トマトを加えたソース・エスパニョルを用いるのがいい。なお、ソース・エスパニョルをさらに丁寧に仕上げるとソース・ドゥミグラスとなる。これはいろいろなソテーに不可欠なもので、今後も変わることはないだろう。

一方、牛や羊、家禽を使った繊細で軽い仕立ての料理にはとろみを付けたジュの方が好まれる。デグラセの際に少量だけ、料理の主素材と同じものからとったジュを用いる。

こんにちのフランス料理においては、肉とソースの調和がとれているべきという、まことに理に適った厳守すべき決まりがある。

だから、ジビエ料理にはジビエのフォンを用いるか、とりたてて際立った個性を持たないフォンを用いて作ったソースを添える。牛や羊のフォンは用いない。ジビエのフォンというのは、さほど濃厚なものを作ることは出来ないが、素材の個性的な風味を表現するには最適だ。こういった事情は魚料理にも当て
\ruby{嵌}{はま}る。ソースそれ自体が際だった風味を持たないものの場合には必ず魚のフュメを加えてやるのだ。このようにしてそれぞれの料理に個性的な風味を実現させることになる。

もちろん、ここまで述べた原則を実現しようにも、コストの問題がしばしば起こることは承知している。けれども、熱意のある、他者の評価を意識している料理人なら問題点を熟考して、完璧とは言わぬまでも満足のいく結果を得ることが出来るだろう。\newpage

\normalsize
\setstretch{1.0}

\hypertarget{traitement-des-elements-de-base}{%
\section{ソースのベース作り}\label{traitement-des-elements-de-base}}

\frsec{Traitement des Éléments de Base dans le Travail des Sauces}

\index{そーす@ソース!そーすつくりのべーす@---のベース作り}
\index{sauce@sauce!Traitement des elements de base dans le travail des sauces@Traitement des Éléments de Base dans le Travail des ---s}
\begin{recette}
\hypertarget{fonds-brun}{%
\subsubsection{茶色いフォン(エストゥファード)}\label{fonds-brun}}

\frsub{Fonds brun ou Estouffade}

\index{ふおん@フォン!ちやいろいふおん@茶色い---}
\index{えすとうふあーと@エストゥファード}
\index{fonds@fonds!brun@--- brun}
\index{fonds@fonds!estouffade@estouffade (fonds brun)}
\index{estouffade@estouffade!fonds brun@ --- (fonds brun)}

(仕上がり10 L分)

\begin{itemize}
\item
  主素材\ldots{}\ldots{}牛すね6
  kg、仔牛のすね6kgまたは仔牛の端肉で脂身を含まないもの6
  kg、骨付きハムのすねの部分1本(前もって下茹でしておくこと)、塩漬けしていない豚皮を下茹でしたもの650
  g。
\item
  香味素材\ldots{}\ldots{}にんじん650 g、玉ねぎ650
  g、ブーケガルニ(パセリの枝100 g、タイム10 g、ローリエ5
  g、にんにく1片)。
\item
  作業手順\ldots{}\ldots{}肉を骨から外す。
\end{itemize}

骨は細かく砕き、オーブンに入れて軽く焼き色を付ける。野菜は焼き色が付くまで炒める。これらを鍋に入れて14
Lの水を注ぎ、ゆっくりと、最低12時間煮込む。水位が下がらぬように、適宜沸騰した湯を足すこと。

大きめのさいの目に切った牛すね肉を別鍋で焼き色が付くまで炒める。先に煮込んでいたフォンを少量加えて煮詰める。この作業を2〜3回行ない、フォンの残りを注ぐ。

鍋を沸騰させて、浮いてくる泡を取り除く。浮き脂も丁寧に取り除く。蓋をして弱火で完全に火が通るまで煮込んだら、布で漉してストックしておく。

\hypertarget{nota-fonds-brun}{%
\subparagraph{【原注】}\label{nota-fonds-brun}}

フォンの材料に牛の骨などが含まれている場合には、事前にその骨だけで12〜
15時間かけてとろ火でフォンをとるといい。

フォンの材料を鍋に焦げ付くくらいまで強く焼き色を付ける\footnote{パンセ
  pincer
  と呼ばれる手法。原義は「抓む」。材料が鍋底に張り付いて、トングなどでしっかり「抓ま」ないと取れないくらい強く焼き付けることからそう呼ばれるようになった。古い料理書では推奨するものも多かった。}のはよろしくない。経験からいって、丁度いい色合いのフォンに仕上げるには、肉に含まれているオスマゾーム\footnote{19世紀頃、赤身肉の美味しさの本質であると考えられていた想像上の物質。赤褐色をした窒素化合物の一種で水に溶ける性質があるとされた。なお、当時のヨーロッパではグルタミン酸はもとよりイノシン酸が「うま味」の要素であるという概念すらなく、「コクがある」corsé
  とか「肉汁たっぷり」onctueux (オンクチュー)や succulent
  (スュキロン)などの表現で肉料理やソースの美味しさが表現された。}の働きだけで充分だ。

\hypertarget{fonds-blanc}{%
\subsubsection{白いフォン}\label{fonds-blanc}}

\frsub{Fonds blanc ordinaire}

\index{ふおん@フォン!しろい@白い---}
\index{fonds@fonds!blanc ordinaire@--- blanc ordinaire}

(仕上がり10 L分)

\begin{itemize}
\item
  主素材\ldots{}\ldots{}仔牛のすね、および端肉10k
  g、鶏の手羽やとさか、足など、または鶏がら4羽分、
\item
  香味素材\ldots{}\ldots{}にんじん800 g、玉ねぎ400 g、ポワロー300
  g、セロリ100 g、ブーケガルニ(パセリの枝100
  g、タイム1枝、ローリエの葉1枚、クローブ4本)。
\item
  使用する液体と味付け\ldots{}\ldots{}水12 L、塩60 g。
\item
  作業手順\ldots{}\ldots{}肉は骨を外し、紐で縛る。骨は細かく砕く。鍋に肉と骨を入れ、水を注ぎ塩を加える。火にかけ、浮いてくるアクを取り除き香味素材を加える。
\item
  加熱時間\ldots{}\ldots{}弱火で3時間。
\end{itemize}

\hypertarget{nota-fonds-blanc}{%
\subparagraph{【原注】}\label{nota-fonds-blanc}}

このフォンは火加減を抑えて、出来るだけ澄んだ仕上がりにすること。アクや浮き脂は丁寧に取り除くこと。

茶色いフォンの場合と同様に、始めに細かく砕いた骨だけを煮てから指定量の水を注ぎ、弱火で5時間煮る方法もある。

この骨を煮た汁で肉を煮るわけだ。その作業内容は上記茶色いフォンの場合と同様。この方法は、骨からゼラチン質を完全に抽出出来るという利点がある。当然のことだが、煮ている間に蒸発して失なわれてしまった分は湯を足してやり、全体量を12
Lにしてから肉を煮ること。

\hypertarget{fonds-de-volaille}{%
\subsubsection{鶏のフォン(フォンドヴォライユ)}\label{fonds-de-volaille}}

\frsub{Fonds de volaille}

\index{ふおん@フォン!とりのふおん@鶏の---}
\index{fonds@fonds!volaille@--- de volaille}
\index{とり@鶏!ふおん@鶏のフォン}
\index{かきん@家禽!とりのふおん@鶏のフォン}
\index{うおらいゆ@ヴォライユ!ふおんとうおらいゆ@フォンドヴォライユ}

白いフォンと同じ主素材、香味素材、水の量で、さらに鶏のとさかや手羽、ガラを適宜増量し、廃鶏3羽を加えて作る。

\hypertarget{jus-de-veau-brun}{%
\subsubsection{仔牛の茶色いフォン(仔牛の茶色いジュ)}\label{jus-de-veau-brun}}

\frsub{Fonds, ou Jus de veau brun}

\index{ふおん@フォン!こうしのちやいろい@仔牛の茶色い---}
\index{しゆ@ジュ!こうしのちやいろいしゆ@仔牛の茶色い---}
\index{fonds@fonds!fonds de veau brun@--- de veau brun}
\index{jus@jus!jus de veau brun@--- de veau brun}
\index{こうし@仔牛!こうしのちやいろいふおん@---の茶色いフォン(ジュ)}
\index{veau@veau!fonds brun@fonds ou jus de --- brun}

(仕上がり10 L分)

\begin{itemize}
\item
  主素材\ldots{}\ldots{}骨を取り除いた仔牛のすね肉と肩肉(紐で縛っておく)6kg、細かく砕いた仔牛の骨5
  kg。
\item
  香味素材\ldots{}\ldots{}にんじん600 g、玉ねぎ400 g、パセリの枝100
  g、ローリエの葉 2枚、タイム2枝。
\item
  使用する液体\ldots{}\ldots{}白いフォンまたは水12 L。水を用いる場合は1
  Lあたり3 gの塩を加える。
\item
  作業手順\ldots{}\ldots{}厚手の片手鍋または寸胴鍋の底に輪切りにしたにんじんと玉ねぎを敷きつめる。その他の香味素材と、あらかじめオーブンで焼き色を付けておいた骨と肉を鍋に加える。
\end{itemize}

蓋をして約10分間、蓋をして弱火にかけた野菜から水分が汗をかくように出るイメージで蒸し焼き状態にし、素材の味を引き出す\footnote{suer
  (スュエ)シュエ。}。フォンまたは水少量を加え、煮詰める。この作業をさらに1〜2回行なう。残りのフォンまたは水を注ぎ、蓋をし、沸騰させる。アクを丁寧に取る。微沸騰の状態で6時間煮る。

布で漉し、ストックしておく。使用目的や必要に応じて、さらに煮詰めてからストックしてもいい。

\hypertarget{fonds-de-gibier}{%
\subsubsection{ジビエのフォン}\label{fonds-de-gibier}}

\frsub{Fonds de gibier}

\index{ふおん@フォン!しひえ@ジビエの---}
\index{fonds@fonds!fonds de gibier@--- de gibier}
\index{しひえ@ジビエ!ふおん@---のフォン}
\index{gibier@gibier!fonds@fonds de ---}

(仕上がり5 L分)

\begin{itemize}
\item
  主素材\ldots{}\ldots{}ノロ鹿の頸、胸肉および端肉3
  kg(老いたノロ鹿がいいが、新鮮なものを使うこと)、野うさぎ\footnote{lièvre
    (リエーヴル)。}の端肉1 kg、老うさぎ2羽、山うずら2羽、老きじ1羽。
\item
  香味素材\ldots{}\ldots{}にんじん250 g、玉ねぎ250
  g、セージ1枝、ジュニパーベリー \footnote{セイヨウネズの樹の実。}15粒、標準的なブーケガルニ。
\end{itemize}

\begin{itemize}
\item
  使用する液体\ldots{}\ldots{}水6 Lおよび白ワイン1瓶。
\item
  加熱時間\ldots{}\ldots{}3時間。
\item
  作業手順\ldots{}\ldots{}ジビエは事前にオーブンで焼き色を付けておき、野菜と香草を敷き詰めた鍋に入れる。野菜類も事前に焼き色を付けておくこと。ジビエを焼くのに用いた天板を白ワインでデグラセし、これを鍋に注ぐ。同量の水も加え、ほぼ水分がなくなるまで煮詰める。
\end{itemize}

この作業の後で、残りの水全量を注ぎ、沸騰させる。丁寧にアクを引きながらごく弱火で煮る\footnote{最後に布で漉す必要があるが、当然のこととして明記されていないので注意。}。

\hypertarget{fumet-de-poisson}{%
\subsubsection[魚のフュメ(フュメドポワソン)]{\texorpdfstring{魚のフュメ(フュメドポワソン)\footnote{本質的には前出の「フォン」と同様のものだが、魚(およびジビエ)を素材としたフォンは香りがポイントとなるため、フュメ
  fumet (香気、良い香りの意)の名称のほうが一般的に使われている。}}{魚のフュメ(フュメドポワソン)}}\label{fumet-de-poisson}}

\frsub{Fonds, ou Fumet de poisson}

\index{ふおん@フォン!さかな@魚の---}
\index{ふゆめ@フュメ!さかな@魚の---}
\index{ふゆめ@フュメ!ほわそん@フュメドポワソン}
\index{fumet@fumet!fumet de poisson@--- de poisson}
\index{fonds@fonds!fumet de poisson@fumet de poisson}

(仕上がり10L分)

\begin{itemize}
\item
  主素材\ldots{}\ldots{}舌びらめ、メルラン\footnote{タラの近縁種。}やバルビュ\footnote{ヒラメの近縁種。}のあら10
  kg。
\item
  香味素材\ldots{}\ldots{}薄切りにした玉ねぎ500 g、パセリの根\footnote{パセリには根がにんじん形に肥大する品種もある(persil
    tubéreux 根パセリ。葉は平らでイタリアンパセリのように使う)。}と茎100
  g、マッシュルームの切りくず\footnote{champignons de Paris
    (シャンピニョンドパリ)いわゆるマッシュルームは、ガルニチュールなど料理の一部として提供する際に、トゥルネ
    tourner
    といって螺旋(らせん)状の切れ込みを入れて装飾したものを使う。その際に少なくない量、具体的にはマッシュルームの重量で15〜20%程度が「切りくず」として発生するのでこれを利用する。この場合だと、少なくとも650〜750
    g程度のマッシュルームの下処理(トゥルネ)をする必要があるが、大きな調理現場以外で毎日それほどのマッシュルームを消費するケースは少ないと思われるので、このレシピのとおりに作るには、切りくずを数日かけて冷蔵庫などで貯めておくなどの工夫が必要だろう。本書のレシピ、とりわけフォンやフュメ、ソースにおいてマッシュルームの切りくずを用いる指示が少なくないので留意されたい。なお、
    tourner(トゥルネ)の原義は「回す」であり、包丁を持った側の手は動かさずに、材料のほうを回すようにして切れ目を入れたり、アーティチョークや果物などの皮を剥くことを意味する。}250
  g、レモンの搾り汁1個分、粒こしょう15
  g(これはフュメを漉す10分前に投入する)。
\item
  使用する液体と調味料\ldots{}\ldots{}水10 L、白ワイン1瓶。液体1
  Lあたり3〜4 gの塩。
\item
  加熱時間\ldots{}\ldots{}30分。
\item
  作業手順\ldots{}\ldots{}鍋底に香味野菜を敷き詰め、魚のあらを入れる。水と白ワインを注ぎ、強火にかける。丁寧にアクを引き、微沸騰の状態を保つようにする。
  30分煮たら目の細かい網で漉す。
\end{itemize}

\hypertarget{nota-fumet-de-poisson}{%
\subparagraph{【原注】}\label{nota-fumet-de-poisson}}

質の悪い白ワインを使うと灰色がかったフュメになってしまう。品質の疑わしいワインは使わないほうがいい。

このフュメはソースを作る際に加える液体として用いる。魚料理用ソース・エスパニョルを作ることを想定する場合には、魚のあらをバターでエチュベしてから水と白ワインを注いで煮るといい。

\hypertarget{fonds-de-poisson-au-vin-rouge}{%
\subsubsection{赤ワインを用いた魚のフォン}\label{fonds-de-poisson-au-vin-rouge}}

\frsub{Fonds de poisson au vin rouge}

\index{ふおん@フォン!あかわいんをもちいたさかなのふおん@赤ワインを用いた魚の---}
\index{fonds@fonds!fonds de poisson au vin rouge@--- de poisson au vin rouge}

このフォンそれ自体を用意することは滅多にない。というのも、例えばマトロットのような料理の魚の煮汁そのものだからだ。

とはいえ、こんにちでは魚のアラをすっかり取り除いた状態で料理を提供する必要がますます高まってきているので、ここでそのレシピを記しておくべきだろう。このフォンの必要性と有用さはどんどん高まっていくと思われる。

原則として、このフォンの仕込みには、料理として提供するのと同じ種類の魚のアラを用いて、その香りの特徴を生かす必要がある。だが、どんな種類の魚を使う場合でも作り方は同じだ。

(仕上がり5 L分)

\begin{itemize}
\item
  主素材\ldots{}\ldots{}料理に用いるのと同じ魚種の頭とアラ2.5 kg。
\item
  香味素材\ldots{}\ldots{}薄切りにして下茹でした玉ねぎ300
  g、パセリの枝100
  g、タイムの小枝1本、小さめのローリエの葉2枚、にんにく5片、マッシュルームの切りくず100
  g。
\item
  使用する液体と調味料\ldots{}\ldots{}水3.5 L、良質の赤ワイン2 L、塩15
  g。
\item
  加熱時間\ldots{}\ldots{}30分。
\item
  作業手順\ldots{}\ldots{}「魚の白いフォン\footnote{前項のフュメドポワソンのこと。}」と同様にする。
\end{itemize}

\hypertarget{nota-fonds-de-poisson-au-vin-rouge}{%
\subparagraph{【原注】}\label{nota-fonds-de-poisson-au-vin-rouge}}

このフォンは魚の白いフォンよりも濃く煮詰めることが可能。とはいえ、保存のために煮詰めないでいいように、その都度、必要な量だけ仕込むことを勧める。

\hypertarget{essence-de-poisson}{%
\subsubsection{魚のエッセンス}\label{essence-de-poisson}}

\frsub{Essence de poisson}

\index{えつせんす@エッセンス!さかな@魚の---}
\index{essence@essence!poisson@--- de poisson}

\begin{itemize}
\item
  主素材\ldots{}\ldots{}メルラン\footnote{タラの近縁種。}および舌びらめの頭、アラ2
  kg。
\item
  香味素材\ldots{}\ldots{}薄切りにした玉ねぎ125
  g、マッシュルームの切りくず300 g、パセリの枝50
  g、レモンの搾り汁1個分。
\item
  使用する液体\ldots{}\ldots{}煮詰めていないフュメドポワソン1
  \(\frac{1}{2}\) L、良質の白ワイン3 dL。
\item
  所要時間\ldots{}\ldots{}45分。
\item
  作業手順\ldots{}\ldots{}鍋にバター100
  gと玉ねぎ、パセリの枝、マッシュルームの切りくずを入れ、強火で色づかないようさっと炒める。アラと端肉を加える。蓋をして約15分弱火で蒸し煮する\footnote{素材を入れた鍋に蓋をして弱火にかけ、少量の水分で蒸し煮状態にすることを
    étuver
    エチュベという。このフランス語をそのまま用いている調理現場も少なくない。}。その間、小まめに混ぜてやること。白ワインを注ぎ、半量になるまで煮詰める。最後にフュメドポワソンを注ぎ、レモン汁と塩2
  gを加える。
\end{itemize}

再び火にかけて、とろ火で15分程煮込んだら、布で漉す。

\hypertarget{nota-essence-de-poisson}{%
\subparagraph{【原注】}\label{nota-essence-de-poisson}}

魚のエッセンスは、舌びらめやチュルボ、チュルボタン、バルビュ\footnote{いずれも鰈、ひらめの近縁種。チュルボタンはチュルボの小さいものを言う。}
などのフィレ\footnote{3枚おろし、または5枚おろしにして、頭とアラを取り除いた状態。}をポシェする際に用いる。

さらに、このエッセンスを煮詰めて、上記でポシェした魚のソースに加えて風味を強くするのに使う。

\hypertarget{essences-diverses}{%
\subsubsection{エッセンスについて}\label{essences-diverses}}

\frsub{Essences diverses}

\index{えつせんす@エッセンス!えつせんすについて@---について(フォン)}
\index{essence@essence!01 diverses@---s diverses (fonds)}

その名のとおり、エッセンスとはごく少量になるまで煮詰めて非常に強い風味を持たせたフォンのこと。

エッセンスは普通のフォンと本質的には同じものだが、素材の風味をしっかり出すために、使用する液体の量はずっと少ない。したがって、仕上げにエッセンスを加える指示がある料理の場合でも、そもそも充分に風味ゆたかなフォンを用いていれば、エッセンスは必要ないことが分かるだろう。

まず最初に、美味しく風味ゆたかなフォンを用いるほうが、あまり出来のよくないフォンで調理し、後からエッセンスで欠点を補うよりもずっと簡単なのだ。その方がいい結果が得られるし、時間と材料の節約にもなる。

セロリ、マッシュルーム、モリーユ\footnote{morille
  キノコの一種。和名アミガサタケ。}、トリュフなど、とりわけ明確な風味の素材のエッセンスを、必要に応じて用いるにとどめるのがいい。

また、十中八九、フォンを仕込む際に素材そのものを加えた方が、エッセンスを仕込むよりもいい結果が得られることは頭に入れておくこと。

そのようなわけで、エッセンスについてこれ以上長々と述べる必要もないと思われる。ベースとなるフォンがコクと風味がゆたかなものならであるなら、エッセンスはまったく無用の長物と言える。

\hypertarget{glaces-diverses}{%
\subsubsection{グラスについて}\label{glaces-diverses}}

\frsub{Glaces diverses}

\index{くらす@グラス!くらすについて@---について}
\index{glace@glace!diverses@---s diverses}

グラスドヴィアンド、鶏のグラス(グラスドヴォライユ)、ジビエのグラス、魚のグラスの用途は多岐にわたる。これらは、上記いずれかの素材でとったフォンをシロップ状になるまで煮詰めたもののことだ。

これらの使い途は、料理の仕上げに表面に塗ってしっとりとした艶を出させるのに用いる場合もあれば、ソースの味や色合いを濃くするために用いたり、あるいは、あまりに出来のよくないフォンで作った料理の場合にはコクを与えるために使うこともある。また、料理によっては適量のバターやクリームを加えてグラスそのものをソースとして用いることもある。

グラスとエッセンスの違いだが、エッセンスが料理の風味そのものを強くすることだけが目的であるのに対して、グラスは素材の持つコクと風味をごく少量にまで濃縮したものだ。

だからほとんどの場合、エッセンスよりもグラスを使うほうがいい。

とはいえ昔の料理長たちの中には、グラスの使用を絶対に認めない者もいた。その理由は、料理を作る度に毎回その料理のためのフォンをとるべきであり、それだけで料理として充分なものにすべき、ということだった。

確かに時間と費用の点で制限がなければその理屈は正しい。だが、こんにちでは、そのようなことの出来る調理現場はほとんどない。そもそもグラスは、正しく適量を用いるのであれば、そのグラスが丁寧に作られたものであるならな、素晴しい結果が得られる。
だから多くの場合、グラスはまことに有用なものと言える。

\hypertarget{glace-de-viande}{%
\subsubsection{グラスドヴィアンド}\label{glace-de-viande}}

\frsub{Glace de viande}

\index{くらす@グラス!ういあんと@---ドヴィアンド}
\index{glace@glace!viande@--- de viande}

茶色いフォン(エストゥファード)を煮詰めて作る。

煮詰めて濃くなっていく途中、何度か布で漉して、より小さな鍋に移しかえていく。煮詰めている際に、丁寧にアクを引くことが、澄んだグラスを作るポイント。

煮詰めている際には、フォンの濃縮具合に応じて、火加減を弱めていくこと。最初は強火でいいが、作業の最後の方は弱火にしてゆっくり煮詰めてやること。

スプーンを入れてみて、引き上げた際に、艶のあるグラスの層でスプーンが覆われ、しっかり張り付いているくらいが丁度いい。要するに、スプーンがグラスでコーティングされた状態になればいいということだ。

\hypertarget{nota-glace-de-viande}{%
\subparagraph{【原注】}\label{nota-glace-de-viande}}

色が薄くて軽い仕上がりのグラスが必要な場合には、茶色いフォンではなく、標準的な仔牛のフォンを用いる。

\hypertarget{glace-de-volaille}{%
\subsubsection{鶏のグラス(グラスドヴォライユ)}\label{glace-de-volaille}}

\frsub{Glace de volaille}

\index{くらす@グラス!とり@鶏の---(---ドヴォライユ)}
\index{くらす@グラス!うおらいゆ@---ドヴォライユ}
\index{glace@glace!volaille@--- de volaille}

鶏のフォン(フォンドヴォライユ)を用いて、グラスドヴィアンドと同様にして作る。

\hypertarget{glace-de-gibier}{%
\subsubsection{ジビエのグラス}\label{glace-de-gibier}}

\frsub{Glace de gibier}

\index{くらす@グラス!しひえ@ジビエの---}
\index{glace@glace!glace de gibier@--- de gibier}
\index{しひえ@ジビエ!くらす@---のグラス}
\index{gibier@gibier!gibier@glace de ---}

ジビエのフォンを煮詰めて作る。ある特定のジビエの風味を生かしたグラスを作るには、そのジビエだけでとったフォンを用いること。

\hypertarget{glace-de-poisson}{%
\subsubsection{魚のグラス}\label{glace-de-poisson}}

\frsub{Glace de poisson}

\index{くらす@グラス!さかな@魚の---}
\index{glace@glace!poisson@--- de poisson}

このグラスを用いることはあまり多くない。日常的な業務においては「魚のエッセンス」を用いることが好まれる。そのほうが魚の風味も繊細になる。魚のエッセンスで魚をポシェした後に煮詰めてソースに加える。
\end{recette}
\hypertarget{roux}{%
\section{ルー}\label{roux}}

\frsec{Roux}

\index{るー@ルー} \index{roux@roux}

ルーはいろいろな派生ソースのベースとなる基本ソースにとろみを付ける役目を持つ。ルーの仕込みは、一見したところさほど重要に思われぬだろうが、実際には正反対だ。丁寧に注意深く作業すること。

茶色いルーは加熱に時間がかかるので、大規模な調理現場では前もって仕込んでおく。ブロンドのルーと白いルーはその都度用意すればいい。
\begin{recette}
\hypertarget{roux-brun}{%
\subsubsection{茶色いルー}\label{roux-brun}}

\frsub{Roux brun}

\index{るー@ルー!ちやいろ@茶色い---} \index{roux@roux!brun@--- brun}

(仕上がり1 kg分)

\begin{enumerate}
\def\labelenumi{\arabic{enumi}.}
\tightlist
\item
  澄ましバター\ldots{}\ldots{}500 g
\item
  ふるった小麦粉\ldots{}\ldots{}600 g
\end{enumerate}

\hypertarget{cuisson-des-roux}{%
\subsubsection{ルーの火入れについて}\label{cuisson-des-roux}}

\index{るー@ルー!ひいれについて@---の火入れについて}
\index{roux@roux!cuisson@cuisson du ---}

加熱時間は使用する熱源の強さで変わってくる。だから数字で何分とは言えない。ただし、火力が強過ぎるよりは弱いくらいの方がいい。というのも、温度が高すぎると小麦粉の細胞が硬化して中身を閉じ込めてしまい、そうなると後でフォンなどの液体を加えた際に上手く混ざらず、滑らかなとろみの付いたソースにならない。乾燥豆をいきなり熱湯で茹でるのと同じようなことが起きるわけだ。低い温度から始めてだんだんと熱くしていけば、小麦粉の細胞壁がゆるんで細胞中のでんぷんが膨張し、熱によって発酵状態の初期のようになる。このようにして、でんぷんをデキストリンに変化させる\footnote{現代の科学的見地からすると必ずしも正確な記述ではないので注意。}。デキストリンは水溶性の物質で、これが「とろみ」の主な要素なのだ。茶色いルーは淡褐色の美しい色合いで滑らかな仕上がりにする。だまがあってはいけない。

ルーを作る際には必ず、澄ましバターを使うこと\footnote{初版〜第三版では「澄ましバターまたは充分に澄ましたグレスドマルミット」となっている。グレスドマルミットとは、コンソメなどを作る際に、浮いてくる油脂を取り除く必要があるが、それを捨てずにまとめてから漉して澄ませたもののこと。基本的に獣脂と考えていい。なお、同時代の料理書
  --- 例えばペラプラ『近代料理技術』(1935年)---
  には、ルーを作るのにバターを使う必要はなく、グレスドマルミットで充分、としているものもある。}。
生のバターには相当量のカゼインが含まれている。カゼインがあると火を均質に通すことが出来なくなってしまう。とはいえ、以下を覚えておくといい。ソースとして仕上げた段階で、ルーで使ったバターは風味という点ではほとんど意味が失なわれている。そもそもソースの仕上げに不純物を取り除く\footnote{dépouiller
  デプイエ。ソースや煮込み料理を仕上げる際に、浮き上がってくる不純物を徹底的に取り除き、目の細かい布などで漉すこと。原義は動物などの皮を剥ぐ、剥くことの意で、野うさぎの皮を剥ぐ、うなぎの皮を剥く、という意味で現代の厨房でも用いられているる。ソースの場合は表面に凝固した蛋白質や油脂の膜が出来、それを「剥ぐように」取り除くことから、あるいは表面に浮いてくる不純物を徹底的に取り除いてきれいなソースに仕上げることを、動物の皮を剥いてきれいな身だけにすることになぞらえて、この用語が用いられるようになったようだ。なお、本書においてécumer(エキュメ)が単に浮いてくる泡やアクを取る、という作業であるのに対して、dépouiller(デプイエ)は「徹底的に不純物を取り除いて美しく仕上げる」という意味合いが込められている。現代では品種改良や農法の変化によって野菜のアクも少なくなり、小麦粉も精製度の高いものを利用出来るなど、食材および調味料の多くで純度の高いものを使用する場合がほとんどであり、このデプイエという作業は20世紀後半にはほとんど行なわれなくなり、écumer(エキュメ)という用語だけで済ませることがほとんど(cf.辻静雄監訳『オリヴェ
  ソースの本』柴田書店、 1970年、27〜28頁)。}段階でバターも完全に取り除かれてしまうわけだ。だからルーに用いるバターは小麦粉に熱を通すためだけのものと考えていい。

ルーはソース作りの出発点だ。だから次の点も記憶に\ruby{留}{とど}めること。小麦粉にでんぷんが含まれているからこそソースに「とろみ」が付く。だから純粋なでんぷん(特性が小麦のでんぷんと同じでも異なったものでも)でルーを作っても、小麦粉の場合と同様の結果が得られるだろう。ただしその場合は小麦粉でルーを作る場合より注意して作業する必要がある。また、小麦粉と違って余計な物質が含まれていないために、全体の分量比率を考え直すことになる。

\hypertarget{nota-roux}{%
\subparagraph{【原注】}\label{nota-roux}}

本文で述べたように、茶色いルーを作る際には澄ましバターを用いる。他の動物性油脂はよほど経済的事情が逼迫していない限り使わないこと。材料コストが問題になる場合でも、ソースの仕上げに不純物を取り除く際に多少の注意を払えば、ルーに用いたバターを回収するのはさして難しいことではない\footnote{既に述べたように初版〜第三版まではバターまたはグレスドマルミットという指示であったことに留意する必要はあるだろう。実際のところ、良質のバターを用いてルーを作ったほうが、軽やかな仕上りのソースになる傾向があることは言うまでもない。}。それを後で他の用途で使えばいいだろう。

\hypertarget{roux-blond}{%
\subsubsection{ブロンドのルー}\label{roux-blond}}

\frsub{Roux blond}

\index{るー@ルー!ふろんと@ブロンドの---}
\index{roux@roux!blond@--- blond}

(仕上がり1 kg分)

材料の比率は茶色いルーと同じ。すなわちバター500 gと、ふるった小麦粉600
g。

火入れは、ルーがほんのりブロンド色になるまで、ごく弱火で行なう。

\hypertarget{roux-blanc}{%
\subsubsection{白いルー}\label{roux-blanc}}

\frsub{Roux blanc}

\index{るー@ルー!しろい@白い---} \index{roux@roux!blanc@--- blanc}

500 gのバターと、ふるった小麦粉600 g。

このルーの火入れは数分、つまり粉っぽさがなくなるまでの時間でいい。
\end{recette}\newpage
\PassOptionsToPackage{unicode=true}{hyperref} % options for packages loaded elsewhere
\PassOptionsToPackage{hyphens}{url}
%
\documentclass[14Q,a4paperpaper,]{ltjsbook}
\usepackage{lmodern}
\usepackage{amssymb,amsmath}
\usepackage{ifxetex,ifluatex}
\usepackage{fixltx2e} % provides \textsubscript
\ifnum 0\ifxetex 1\fi\ifluatex 1\fi=0 % if pdftex
  \usepackage[T1]{fontenc}
  \usepackage[utf8]{inputenc}
  \usepackage{textcomp} % provides euro and other symbols
\else % if luatex or xelatex
  \usepackage{unicode-math}
  \defaultfontfeatures{Ligatures=TeX,Scale=MatchLowercase}
\fi
% use upquote if available, for straight quotes in verbatim environments
\IfFileExists{upquote.sty}{\usepackage{upquote}}{}
% use microtype if available
\IfFileExists{microtype.sty}{%
\usepackage[]{microtype}
\UseMicrotypeSet[protrusion]{basicmath} % disable protrusion for tt fonts
}{}
\IfFileExists{parskip.sty}{%
\usepackage{parskip}
}{% else
\setlength{\parindent}{0pt}
\setlength{\parskip}{6pt plus 2pt minus 1pt}
}
\usepackage{hyperref}
\hypersetup{
            pdftitle={エスコフィエ『料理の手引き』全注解},
            pdfauthor={五 島 学},
            pdfborder={0 0 0},
            breaklinks=true}
\urlstyle{same}  % don't use monospace font for urls
\setlength{\emergencystretch}{3em}  % prevent overfull lines
\providecommand{\tightlist}{%
  \setlength{\itemsep}{0pt}\setlength{\parskip}{0pt}}
\setcounter{secnumdepth}{0}
% Redefines (sub)paragraphs to behave more like sections
\ifx\paragraph\undefined\else
\let\oldparagraph\paragraph
\renewcommand{\paragraph}[1]{\oldparagraph{#1}\mbox{}}
\fi
\ifx\subparagraph\undefined\else
\let\oldsubparagraph\subparagraph
\renewcommand{\subparagraph}[1]{\oldsubparagraph{#1}\mbox{}}
\fi

% set default figure placement to htbp
\makeatletter
\def\fps@figure{htbp}
\makeatother


\title{エスコフィエ『料理の手引き』全注解}
\author{五 島 学}
\date{}
%%%%%%%%%%%%%%%%%%%%%%% added by mgoto
\usepackage{setspace}


% %%%%%%%%% hyperref %%%%%%%%%%%%%
% \usepackage{refcount}
% \usepackage[unicode=true,hyperfootnotes=false,pageanchor]{hyperref}
% \hypersetup{hyperindex=false,%
%              breaklinks=true,%
%              bookmarks=true,%
%              pdfauthor={五島 学},%
%              pdftitle={エスコフィエ『料理の手引き』全注解},%
%              colorlinks=false%true,%
%              %colorlinks=true,%
%              citecolor=blue,%
%              urlcolor=cyan,%
%              linkcolor=magenta,%
%              bookmarksdepth=subsubsection,%
%              pdfborder={0 0 0},%
%              hyperfootnotes=false,%
%              plainpages=false,
%              }
% \urlstyle{same}





%% 欧文フォント設定
% Libertine/Biolinum
\setmainfont[Ligatures=Historic,Scale=1.0]{Linux Libertine O}
\setsansfont[Ligatures=TeX, Scale=MatchLowercase]{Linux Biolinum O} 
%\usepackage{libertine}
\usepackage{unicode-math}
\setmathfont[Scale=1.2]{libertinusmath-regular.otf}
\usepackage{luatexja}
\usepackage{luatexja-fontspec}
%\ltjdefcharrange{8}{"2000-"2013, "2015-"2025, "2027-"203A, "203C-"206F}
%\ltjsetparameter{jacharrange={-2, +8}}
\usepackage{luatexja-ruby}

\newopentypefeature{PKana}{On}{pkna} % "PKana" and "On" can be arbitrary string
%%%%明朝にIPAexMincho、ゴチ(太字)にMoboGoBを使う設定。和文カナプロプーショナル使用可能だが読みづらくなる。
\setmainjfont[%
     %YokoFeatures={JFM=prop,PKana=On},%
     %CharacterWidth=AlternateProportional,%
%    CharacterWidth=Proportional,%Mogo, IPAExMinchoには不可
     %Kerning=On,%
     BoldFont={ MoboGoB },%
     ItalicFont={ MoboGoB },%
     BoldItalicFont={ MoboGoExB }%
     % ]{ MogaHMin }
     ]{ IPAExMincho }
     % ]{ IPAmjMincho }
\setsansjfont[%
     %YokoFeatures={JFM=prop,PKana=On},%
     %CharacterWidth=AlternateProportional,%
     % CharacterWidth=Proportional,%Mobo, IPAExGOthicには不可
     %Kerning=On,
     BoldFont={ MoboGoB },%
     ItalicFont={ MoboGoB },%
     BoldItalicFont={ MoboGoExB }%
     % ]{ MoboGo}
     ]{ IPAExGothic }
     % %  %%%% 和文仮名プロプーショナルここまで

     
\renewcommand{\bfdefault}{bx}%和文ボールドを有効にする
\renewcommand{\headfont}{\gtfamily\sffamily\bfseries}%和文ボールドを有効にする

%リスト環境
\def\tightlist{\itemsep1pt\parskip0pt\parsep0pt}%pandoc対策

\makeatletter
  \parsep   = 0pt
  \labelsep = .5\zw
  \def\@listi{%
     \leftmargin = 0pt \rightmargin = 0pt
     \labelwidth\leftmargin \advance\labelwidth-\labelsep
     \topsep     = 0pt%\baselineskip
     %\topsep -0.1\baselineskip \@plus 0\baselineskip \@minus 0.1 \baselineskip
     \partopsep  = 0pt \itemsep       = 0pt
     \itemindent = -.5\zw \listparindent = 0\zw}
  \let\@listI\@listi
  \@listi
  \def\@listii{%
     \leftmargin = 1.8\zw \rightmargin = 0pt
     \labelwidth\leftmargin \advance\labelwidth-\labelsep
     \topsep     = 0pt \partopsep     = 0pt \itemsep   = 0pt
     \itemindent = 0pt \listparindent = 1\zw}
  \let\@listiii\@listii
  \let\@listiv\@listii
  \let\@listv\@listii
  \let\@listvi\@listii
\makeatother

%%%%インデックス準備
%\usepackage{makeidx}
\usepackage{index}
%\usepackage[useindex]{splitidx}
\newindex{src}{idx1}{ind1}{ソース名から料理を探す}
\makeindex
 
 \makeatletter
\renewenvironment{theindex}{% 索引を3段組で出力する環境
    \if@twocolumn
      \onecolumn\@restonecolfalse
    \else
      \clearpage\@restonecoltrue
    \fi
    \columnseprule.4pt \columnsep 2\zw
    \ifx\multicols\@undefined
      \twocolumn[\@makeschapterhead{\indexname}%
      \addcontentsline{toc}{chapter}{\indexname}]%
    \else
      \ifdim\textwidth=\fullwidth
        \setlength{\evensidemargin}{\oddsidemargin}
        \setlength{\textwidth}{\fullwidth}
        \setlength{\linewidth}{\fullwidth}
        \begin{multicols}{3}[\chapter*{\indexname}%
        \addcontentsline{toc}{chapter}{\indexname}]%
      \else
        \begin{multicols}{2}[\chapter*{\indexname}%
        \addcontentsline{toc}{chapter}{\indexname}]%
      \fi
    \fi
    \@mkboth{\indexname}{}%
    \plainifnotempty % \thispagestyle{plain}
    \parindent\z@
    \parskip\z@ \@plus .3\jsc@mpt\relax
    \let\item\@idxitem
    \raggedright
    \footnotesize\narrowbaselines
  }{
    \ifx\multicols\@undefined
      \if@restonecol\onecolumn\fi
    \else
      \end{multicols}
    \fi
    \clearpage
  }
 \makeatother


%%%% 本文中の参照ページ番号表示 %%%%%%%

\makeatletter

%\AtBeginDocument{%
%  \DeclareRobustCommand\ref{\@ifstar\@refstar\@refstar}%
%  \DeclareRobustCommand\pageref{\@ifstar\@pagerefstar\@pagerefstar}}
\let\orig@Hy@EveryPageAnchor\Hy@EveryPageAnchor
\def\Hy@EveryPageAnchor{%
    \begingroup
    \hypersetup{pdfview=Fit}%
    \orig@Hy@EveryPageAnchor
    \endgroup
  }
  \usepackage{etoolbox}
\if@mainmatter{\let\myhyperlink\hyperlink%
\renewcommand{\hyperlink}[2]{\myhyperlink{#1}{#2} [p.\getpagerefnumber{#1}{}] }}
  \AtBeginEnvironment{recette}{%
\let\myhyperlink\hyperlink%
\renewcommand{\hyperlink}[2]{\myhyperlink{#1}{#2} [p.\getpagerefnumber{#1}{}] }}
  \AtBeginEnvironment{Main}{%
\let\myhyperlink\hyperlink%
\renewcommand{\hyperlink}[2]{\myhyperlink{#1}{#2} [p.\getpagerefnumber{#1}{}] }}


%%%% pandoc が三点リーダーを勝手に変える対策
\renewcommand{\ldots}{\noindent…}

%%% 脚注番号のページ毎のリセットと脚注位置の調整
%\renewcommand{\footnotesize}{\small}

\makeatletter

\usepackage[bottom,perpage,stable]{footmisc}%
%\setlength{\skip\footins}{4mm plus 4mm}
%\usepackage{footnpag}
\renewcommand\@makefntext[1]{%
  \advance\leftskip 0\zw
  \parindent 1\zw
  \noindent
  \llap{\@thefnmark\hskip0.5\zw}#1}


\let\footnotes@ve=\footnote
\def\footnote{\inhibitglue\footnotes@ve}
\let\footnotemarks@ve=\footnotemark
%\def\footnotemark{\inhibitglue\footnotemarks@ve}
\renewcommand{\footnotemark}{\footnotemarks@ve}%変更
% %\def\thefootnote{\ifnum\c@footnote>\z@\leavevmode\lower.5ex\hbox{(}\@arabic\c@footnote\hbox{)}\fi}
\renewcommand{\thefootnote}{\ifnum\c@footnote>\z@\leavevmode\hbox{}\@arabic\c@footnote\hbox{)}\fi}

%%%%%%%%%レシピと本文%%%%%%%%%%%%
\usepackage{multicol}
\setlength{\columnsep}{3\zw}

%%% 本文
\newenvironment{Main}{}{}
%%% レシピ
% \setlength{\columnwidth}{24\zw}
%本文ヨリ小%\small
%\newenvironment{recette}{\setlength{\parindent}{0pt}\begin{small}\begin{spaceing}{0.8}\begin{multicols}{2}}{\end{multicols}\end{spacing}\end{small}}
%本文やや小%\medsmall
%\newenvironment{recette}{\setlength{\parindent}{0pt}\begin{medsmall}\begin{spacing}{0.75}\begin{multicols}{2}}{\end{multicols}\end{spacing}\end{medsmall}}
%本文ナミ(無指定)
\newenvironment{recette}{\setlength{\parindent}{0pt}\begin{spacing}{0.8}\begin{multicols}{2}\setlength\topskip{.8\baselineskip}}{\end{multicols}\end{spacing}}

%文字サイズ、見出しなどの再定義
\makeatletter
%\renewcommand{\large}{\jsc@setfontsize\large\@xipt{14}}
%\renewcommand{\Large}{\jsc@setfontsize\Large{13}{15}}

\newcommand{\medlarge}{\fontsize{11}{13}\selectfont}
\newcommand{\medsmall}{\fontsize{9.23}{9.5}\selectfont}
\newcommand{\twelveq}{\jsc@setfontsize\twelveq{9.230769}{9.75}\selectfont}
\newcommand{\thirteenq}{\jsc@setfontsize\fourteenq{10}{11}\selectfont}
\newcommand{\fourteenq}{\jsc@setfontsize\fourteenq{10.7692}{13}\selectfont}
\newcommand{\fifteenq}{\jsc@setfontsize\fifteenq{11.53846}{14}\selectfont}
\makeatletter
\renewcommand{\chapter}{%
  \if@openleft\cleardoublepage\else
  \if@openright\cleardoublepage\else\clearpage\fi\fi
  \plainifnotempty % 元: \thispagestyle{plain}
  \global\@topnum\z@
  \if@english \@afterindentfalse \else \@afterindenttrue \fi
  \secdef
    {\@omit@numberfalse\@chapter}%
    {\@omit@numbertrue\@schapter}}
\def\@chapter[#1]#2{%
  \ifnum \c@secnumdepth >\m@ne
    \if@mainmatter
      \refstepcounter{chapter}%
      \typeout{\@chapapp\thechapter\@chappos}%
      \addcontentsline{toc}{chapter}%
        {\protect\numberline
        % {\if@english\thechapter\else\@chapapp\thechapter\@chappos\fi}%
        {\@chapapp\thechapter\@chappos}%
        #1}%
    \else\addcontentsline{toc}{chapter}{#1}\fi
  \else
    \addcontentsline{toc}{chapter}{#1}%
  \fi
  \chaptermark{#1}%
  \addtocontents{lof}{\protect\addvspace{10\jsc@mpt}}%
  \addtocontents{lot}{\protect\addvspace{10\jsc@mpt}}%
  \if@twocolumn
    \@topnewpage[\@makechapterhead{#2}]%
  \else
    \@makechapterhead{#2}%
    \@afterheading
  \fi}
\def\@makechapterhead#1{%
  \vspace*{0\Cvs}% 欧文は50pt
  {\parindent \z@ \centering \normalfont
    \ifnum \c@secnumdepth >\m@ne
      \if@mainmatter
        \huge\headfont \@chapapp\thechapter\@chappos%変更
        \par\nobreak
        \vskip \Cvs % 欧文は20pt
      \fi
    \fi
    \interlinepenalty\@M
    \huge \headfont #1\par\nobreak
    \vskip 1\Cvs}} % 欧文は40pt%変更

\renewcommand{\section}{%
    \if@slide\clearpage\fi
    \@startsection{section}{1}{\z@}%
    {\Cvs \@plus.5\Cdp \@minus.2\Cdp}% 前アキ
    % {.5\Cvs \@plus.3\Cdp}% 後アキ
    {.5\Cvs}
    {\normalfont\Large\headfont\bfseries\centering}}%変更

\renewcommand{\subsection}{\@startsection{subsection}{2}{\z@}%
    {\Cvs \@plus.5\Cdp \@minus.2\Cdp}% 前アキ
    % {.5\Cvs \@plus.3\Cdp}% 後アキ
    {.5\Cvs}
  %  {\normalfont\large\headfont\bfseries\centering}} %変更
    {\normalfont\large\headfont\centering}} %変更

\renewcommand{\subsubsection}{\@startsection{subsubsection}{3}{\z@}%
  % {0\Cvs \@plus.8\Cdp \@minus.6\Cdp}%変更
    {1sp \@plus.5\Cdp \@minus.5\Cdp}%変更
    {\if@slide .5\Cvs \@plus.3\Cdp \else \z@ \fi}%
    % {\normalfont\medlarge\headfont\leftskip -1\zw}}
    {\normalfont\medlarge\headfont\leftskip -1\zw}}

\renewcommand{\paragraph}{\@startsection{paragraph}{4}{\z@}%
    {0.5\Cvs \@plus.5\Cdp \@minus.2\Cdp}%
    % {\if@slide .5\Cvs \@plus.3\Cdp \else -1\zw\fi}% 改行せず 1\zw のアキ
    {1sp}%後アキ
    {\normalfont\normalsize\headfont}}
\renewcommand{\subparagraph}{\@startsection{subparagraph}{5}{\z@}%
    {\z@}{\if@slide .5\Cvs \@plus.3\Cdp \else -.5\zw\fi}%
    {\normalfont\normalsize\headfont\hskip-.5\zw\noindent}}  

\newcommand{\frchap}[1]{\vspace*{-2ex}%
 \begin{center}\normalfont\headfont\LARGE\setstretch{0.8}
 \scshape#1\normalfont\normalsize
\end{center}\vspace{0.5\zw}\setstretch{1.0}}

\newcommand{\frsec}[1]{\vspace*{-2ex}%
 \begin{center}\normalfont\headfont\large\setstretch{0.8}
 \scshape#1\normalfont\normalsize
\end{center}\vspace{0.5\zw}\setstretch{1.0}}
  
\newcommand{\frsecb}[1]{\vspace*{-2ex}%
\begin{center}\normalfont\headfont\medlarge\setstretch{0.8}%
  \hspace{1em}\scshape#1\normalfont\normalsize%
\end{center}\vspace{0.5\zw}\setstretch{1.0}}
\makeatother

\newenvironment{frchapenv}{\vspace*{-2ex}\begin{center}\normalfont\headfont%
    \LARGE\setstretch{0.8}\normalfont\normalsize\scshape%
    }{\end{center}\vspace{0.5\zw}\setstretch{1.0}}

\newenvironment{frsecenv}{\vspace*{-2ex}%
\begin{center}\normalfont\headfont\medlarge\setstretch{0.8}%
  \hspace{1em}\normalfont\normalsize\scshape%
}{\end{center}\vspace{0.5\zw}\setstretch{1.0}}

\newenvironment{frsecbenv}{\vspace*{-2ex}%
\begin{center}\normalfont\headfont\medlarge\setstretch{0.8}%
  \hspace{1em}\normalfont\normalsize\scshape%
}{\end{center}\vspace{0.5\zw}\setstretch{1.0}}

\newenvironment{frsecenv}{\vspace*{-2ex}%
  \begin{center}\normalfont\headfont\large\setstretch{0.8}\normalfont\normalsize\scshape}%
  {\end{center}\vspace{0.5\zw}\setstretch{1.0}}

\newenvironment{frsubenv}{\begin{spacing}{0.2}\setlength{\leftskip}{-1\zw}\bfseries}{\end{spacing}\normalfont\normalsize\setlength{\leftskip}{0pt}\par\vspace{1.1\zw}}

\newcommand{\frsub}[1]{\begin{frsubenv}#1\end{frsubenv}\par\vspace{1.1\zw}}


\renewcommand{\thechapter}{}
\renewcommand{\thesection}{\hskip-1\zw}
\renewcommand{\thesubsection}{}
\renewcommand{\thesubsubsection}{}
\renewcommand{\theparagraph}{}


% \makeatletter
% \@removefromreset{subsubsection}{subsection}
% \def\thesubsubsection{\arabic{subsubsection}.}
% \newcounter{rnumber}
% \renewcommand{\thernumber}{\refstepcounter{rnumber} }
\makeatother
\renewcommand{\prepartname}{\if@english Part~\else {}\fi}
\renewcommand{\postpartname}{\if@english\else {}\fi}
\renewcommand{\prechaptername}{\if@english Chapter~\else {}\fi}
\renewcommand{\postchaptername}{\if@english\else {}\fi}
\renewcommand{\presectionname}{}%  第
\renewcommand{\postsectionname}{}% 節

%\makeatletter
\def\ps@headings{%
  \let\@oddfoot\@empty
  \let\@evenfoot\@empty
  \def\@evenhead{%
    \if@mparswitch \hss \fi
    \underline{\hbox to \fullwidth{\ltjsetparameter{autoxspacing={true}}
%      \textbf{\thepage}\hfil\leftmark}}%
       \normalfont\thepage\hfill\scshape\small\leftmark\normalfont}}%
    \if@mparswitch\else \hss \fi}%
  \def\@oddhead{\underline{\hbox to \fullwidth{\ltjsetparameter{autoxspacing={true}}
        {\if@twoside\scshape\small\rightmark\else\scshape\small\leftmark\fi}\hfil\thepage\normalfont}}\hss}%
  \let\@mkboth\markboth
  \def\chaptermark##1{\markboth{%
    \ifnum \c@secnumdepth >\m@ne
      \if@mainmatter
        \if@omit@number\else
          \@chapapp\thechapter\@chappos\hskip1\zw
        \fi
      \fi
    \fi
    ##1}{}}%
  \def\sectionmark##1{\markright{%
%    \ifnum \c@secnumdepth >\z@ \thesection \hskip1\zw\fi
    \ifnum \c@secnumdepth >\z@ \thesection \hskip-1\zw\fi
    ##1}}}%
\makeatother

\makeatletter
%%%%%%%% Lua GC
\patchcmd\@outputpage{\stepcounter{page}}{%
  \directlua{%
	if jit then
      local k = collectgarbage("count")
      if k>900000 then 
        collectgarbage("collect")
        texio.write_nl("term and log", "GC: ", math.floor(k), math.floor(collectgarbage("count")))
      end
	end
  }%
  \stepcounter{page}%
}{}{}
\makeatother

\makeatletter

\def\srcGlaceDeViande#1#2#3#4{%
  \index[src]{glace de viande@Glace de viande!{#1}@{#2}}%
  \index[src]{くらすとういあんと@グラスドヴィアンド!{#3}@{#4}}}

%%%%%%基本ソース

\def\srcEspagnole#1#2#3#4{%
  \index[src]{espagnole@Espagnole!{#1}@{#2}}
  \index[src]{えすはによる@エスパニョル!{#3}@{#4}}}
 
\def\srcEspagnoleMaigre#1#2#3#4{%
  \index[src]{espagnole maigre@Espagnole maigre!{#1}@{#2}}%
  \index[src]{えすはによるさかな@エスパニョル(魚料理用)!{#3}@{#4}}}

\def\srcDemiGlace#1#2#3#4{%
  \index[src]{demi-glace@Demi-glace!{#1}@{#2}}%
  \index[src]{とうみくらす@ドゥミグラス!{#3}@{#4}}}

\def\srcJusDeVeauLie#1#2#3#4{%
  \index[src]{jus veau lie@Jus de veau lié!{#1}@{#2}}%
  \index[src]{とろみをつけたこうしのしゆ@とろみを付けた仔牛のジュ!{#3}@{#4}}}

\def\srcVeoute#1#2#3#4{%
  \index[src]{veloute@Velouté!{#1}@{#2}}%
  \index[src]{うるて@ヴルテ!{#3}@{#4}}}

\def\srcVeouteDeVolaille#1#2#3#4{%
  \index[src]{veloute de volaille@Velouté de volaille!{#1}@{#2}}%
  \index[src]{とりのうるて@鶏のヴルテ!{#3}@{#4}}}

\def\srcVeouteDePoisson#1#2#3#4{%
  \index[src]{veloute de poisson@Velouté de poisson!{#1}@{#2}}%
  \index[src]{さかなのうるて@魚のヴルテ!{#3}@{#4}}}

\def\srcAllemande#1#2#3#4{%
  \index[src]{allemande@Allemande!{#1}@{#2}}%
  \index[src]{あるまんと@アルマンド!{#3}@{#4}}}

\def\srcSupreme#1#2#3#4{%
  \index[src]{supreme@Suprême!{#1}@{#2}}%
  \index[src]{しゆふれーむ@シュプレーム!{#3}@{#4}}}

\def\srcBechamel#1#2#3#4{%
  \index[src]{bechamel@Bechamel!{#1}@{#2}}%
  \index[src]{へしやめる@ベシャメル!{#3}@{#4}}}

\def\srcTomate#1#2#3#4{%
  \index[src]{tomate@Tomate!{#1}@{#2}}%
  \index[src]{とまと@トマト!{#3}@{#4}}}

%%%%%%%%ブラウン系の派生ソース

\def\srcBigarade#1#2#3#4{%
  \index[src]{bigarade@Bigarade!{#1}@{#2}}%
  \index[src]{ひからーと@ビガラード!{#3}@{#4}}}

\def\srcBordelaise#1#2#3#4{%
  \index[src]{bordelaise@Bordelaise!{#1}@{#2}}%
  \index[src]{ほるとーふう@ボルドー風!{#1}@{#2}}}

\def\srcBourguignonne#1#2#3#4{%
  \index[src]{bourguignonne@Bourguignonne!{#1}@{#2}}%
  \index[src]{ふるこーにゆふう@ブルゴーニュ風!{#3}@{#4}}}

\def\srcBretonne#1#2#3#4{%
  \index[src]{bretonne@Bretonne!{#1}@{#2}}%
  \index[src]{ふるたーにゆふう@ブルターニュ風!{#3}@{#4}}}

\def\srcCerises#1#2#3#4{%
  \index[src]{cerises@Cerises (aux)!{#1}@{#2}}%
  \index[src]{すりーす@スリーズ!{#3}@{#4}}}

\def\srcChampignons#1#2#3#4{%
  \index[src]{champignons@Champignons (aux)!{#1}@{#2}}%
  \index[src]{しやんひによん@シャンピニョン!{#3}@{#4}}}

\def\srcChampignons#1#2#3#4{%
  \index[src]{charcutiere@Charcutière!{#1}@{#2}}%
  \index[src]{しやるきゆていえーる@シャルキュティエール!{#3}@{#4}}}

\def\srcChasseur#1#2#3#4{%
  \index[src]{chasseur@Chasseur!{#1}@{#2}}%
  \index[src]{しやすーる@シャスール!{#3}@{#4}}}

\def\srcChasseurEscoffier#1#2#3#4{%
  \index[src]{しやすーるえすこふいえ@シャスール(エスコフィエ)!{#1}@{#2}}%
  \index[src]{chasseur escoffier@Chasseur (Escoffier)!{#3}@{#4}}}

\def\srcChaudFroidBrune#1#2#3#4{%
  \index[src]{chaud-froid brune@Chaud-froid brune!{#1}@{#2}}%
  \index[src]{しよふろわしやいろ@ショフロワ(茶色)!{#3}@{#4}}}

\def\srcChaudFroidBruneCanard#1#2#3#4{%
  \index[src]{chaud-froid brune canards@Chaud-froid brune pour Canards!{#1}@{#2}}%
  \index[src]{しよふろわちやいろかも@ショフロワ(茶色、鴨用)!{#3}@{#4}}}

\def\srcChaudFroidBruneGibier#1#2#3#4{%
  \index[src]{chaud-froid brune gibier@Chaud-froid brune pour Gibier!{#1}@{#2}}%
  \index[src]{しよふろわちやいろしひえ@ショフロワ(茶色、ジビエ用)!{#3}@{#4}}}

\def\srcChaudFroidTomatee#1#2#3#4{%
  \index[src]{chaud-froid tometee@Chaud-froid tomatée!{#1}@{#2}}%
  \index[src]{しよふろわとまて@トマト入りショフロワ!{#3}@{#4}}}

\def\srcChevreuil#1#2#3#4{%
  \index[src]{chevreuil@Chevreuil!{#1}@{#2}}%
  \index[src]{しゆうるいゆ@シュヴルイユ!{#3}@{#4}}}

\def\srcColbert#1#2#3#4{%
  \index[src]{colbert@Colbert!{#1}@{#2}}%
  \index[src]{こるへーる@コルベール!{#3}@{#4}}}

\def\srcDiable#1#2#3#4{%
  \index[src]{diable@Diable!{#1}@{#2}}%
  \index[src]{ていあーふる@ディアーブル!{#3}@{#4}}}

\def\srcDiableEscoffier#1#2#3#4{%
  \index[src]{diable escoffier@Diable Escoffier!{#1}@{#2}}%
  \index[src]{ていあーふるえすこふいえ@ディアーブル・エスコフィエ!{#3}@{#4}}}

\def\srcDiane#1#2#3#4{%
  \index[src]{diane@Diane!{#1}@{#2}}%
  \index[src]{ていあーぬ@ディアーヌ!{#3}@{#4}}}

\def\srcDuxelles#1#2#3#4{%
  \index[src]{duxelles@Duxelles!{#1}@{#2}}%
  \index[src]{てゆくせる@デュクセル!{#3}@{#4}}}

\def\srcEstragon#1#2#3#4{%
  \index[src]{estragon@Estragon!{#1}@{#2}}%
  \index[src]{えすとらこん@エストラゴン!{#3}@{#4}}}

\def\srcFinanciere#1#2#3#4{%
  \index[src]{financierel@Financière!{#1}@{#2}}%
  \index[src]{ふいなんしえーる@フィナンシエール!{#3}@{#4}}}

\def\srcFinesHerbes#1#2#3#4{%
  \index[src]{fines herbes@Fines herbes (aux)!{#1}@{#2}}%
  \index[src]{こうそう@香草!{#3}@{#4}}}

\def\srcGenevoise#1#2#3#4{%
  \index[src]{genevoise@Genevoise!{#1}@{#2}}%
  \index[src]{しゆねーうふう@ジュネーヴ風!{#3}@{#4}}}

\def\srcGodard#1#2#3#4{%
  \index[src]{godard@Godard!{#1}@{#2}}%
  \index[src]{こたーる@ゴダール!{#3}@{#4}}}

\def\srcGrandVeneur#1#2#3#4{%
  \index[src]{grand-veneur@Grand-Veneur!{#1}@{#2}}%
  \index[src]{くらんうぬーる@グランヴヌール!{#3}@{#4}}}

\def\srcGrandVeneurEscoffier#1#2#3#4{%
  \index[src]{grand-veneur escoffier@Grand-Veneur Escoffier!{#1}@{#2}}%
  \index[src]{くらんうぬーるえすこふぃえ@グランヴヌール(エスコフィエ)!{#3}@{#4}}}

\def\srcGratin#1#2#3#4{%
  \index[src]{gratin@Gratin!{#1}@{#2}}%
  \index[src]{くらたん@グラタン!{#3}@{#4}}}

\def\srcHachee#1#2#3#4{%
  \index[src]{hachee@Hachée!{#1}@{#2}}%
  \index[src]{あしえ@アシェ!{#3}@{#4}}}

\def\srcHacheeMaigre#1#2#3#4{%
  \index[src]{hachee maigrel@Hachée maigre!{#1}@{#2}}%
  \index[src]{あしえさかな@アシェ(魚料理用)!{#3}@{#4}}}

\def\srcHussarde#1#2#3#4{%
  \index[src]{hussarde@Hussarde!{#1}@{#2}}%
  \index[src]{ゆさると@ユサルド!{#3}@{#4}}}

\def\srcItalienne#1#2#3#4{%
  \index[src]{italienne@Italienne!{#1}@{#2}}%
  \index[src]{いたりあふう@イタリア風!{#3}@{#4}}}

\def\srcJusLieEstragon#1#2#3#4{%
  \index[src]{jus lie a l'estragon@Jus lié à l'Estragon!{#1}@{#2}}%
  \index[src]{とろみをつけたしゆえすとらこん@とろみを付けたジュ・エストラゴン風味!{#3}@{#4}}}

\def\srcJusLieTomate#1#2#3#4{%
  \index[src]{jus lie tomate@Jus lié tomaté!{#1}@{#2}}%
  \index[src]{とろみをつけたしゆとまと@とろみを付けたジュ・トマト風味!{#3}@{#4}}}

\def\srcLyonnaise#1#2#3#4{%
  \index[src]{lyonnaise@Lyonnaise!{#1}@{#2}}%
  \index[src]{りよんふう@リヨン風!{#3}@{#4}}}

\def\srcMadere#1#2#3#4{%
  \index[src]{madere@Madère!{#1}@{#2}}%
  \index[src]{まてーる@マデール!{#3}@{#4}}}

\def\srcMatelote#1#2#3#4{%
  \index[src]{matelote@Matelote!{#1}@{#2}}%
  \index[src]{まとろつと@マトロット!{#3}@{#4}}}

\def\srcMoelle#1#2#3#4{%
  \index[src]{moelle@Moelle!{#1}@{#2}}%
  \index[src]{もわる@モワル!{#3}@{#4}}}

\def\srcMoscovite#1#2#3#4{%
  \index[src]{moscovite@Moscovite!{#1}@{#2}}%
  \index[src]{もすくわふう@モスクワ風!{#3}@{#4}}}

\def\srcPerigueux#1#2#3#4{%
  \index[src]{perigueux@Périgueux!{#1}@{#2}}%
  \index[src]{へりくー@ペリグー!{#3}@{#4}}}

\def\srcPerigourdine#1#2#3#4{%
  \index[src]{perigourdine@Périgourdine!{#1}@{#2}}%
  \index[src]{へりくるていーぬ@ペリグルディーヌ!{#3}@{#4}}}

\def\srcPiquante#1#2#3#4{%
  \index[src]{piquante@Piquante!{#1}@{#2}}%
  \index[src]{ひかんと@ピカント!{#3}@{#4}}}

\def\srcPoivradeOrdinaire#1#2#3#4{%
  \index[src]{poivrade ordinaire@Poivrade ordinaire!{#1}@{#2}}%
  \index[src]{ほわふらーとひようしゆん@ポワヴラード(標準)!{#3}@{#4}}}

\def\srcPoivradeGibier#1#2#3#4{%
  \index[src]{poivrade gibier@Poivrade pour Gibier!{#1}@{#2}}%
  \index[src]{ほわふらーとしひえ@ポワヴラード(ジビエ用)!{#3}@{#4}}}

\def\srcPorto#1#2#3#4{%
  \index[src]{porto@Porto!{#1}@{#2}}%
  \index[src]{ほると@ポルト!{#3}@{#4}}}

\def\srcPortugaise#1#2#3#4{%
  \index[src]{portugaise@Portugaise!{#1}@{#2}}
  \index[src]{ほるとかるふう@ポルトガル風!{#3}@{#4}}}

\def\srcProvencale#1#2#3#4{%
  \index[src]{provencale@Provençale!{#1}@{#2}}
  \index[src]{ふろうあんすふう@プロヴァンス風!{#3}@{#4}}}

\def\srcRegence#1#2#3#4{%
  \index[src]{regence@Régence!{#1}@{#2}}
  \index[src]{れしやんす@レジャンス!{#3}@{#4}}}

\def\srcRobert#1#2#3#4{%
  \index[src]{robert@Robert!{#1}@{#2}}
  \index[src]{ろへーる@ロベール!{#3}@{#4}}}

\def\srcRobertEscoffier#1#2#3#4{%
  \index[src]{robert escoffier@Robert Escoffier!{#1}@{#2}}
  \index[src]{ろへーるえすこふいえ@ロベール・エスコフィエ!{#3}@{#4}}}

\def\srcRomaine#1#2#3#4{%
  \index[src]{romaine@Romaine!{#1}@{#2}}
  \index[src]{ろーまふう@ローマ風!{#3}@{#4}}}

\def\srcRouennaise#1#2#3#4{%
  \index[src]{rouennaise@Rouennaise!{#1}@{#2}}
  \index[src]{るーあんふう@ルーアン風!{#3}@{#4}}}

\def\srcSalmis#1#2#3#4{%
  \index[src]{salmis@Salmis!{#1}@{#2}}
  \index[src]{さるみ@サルミ!{#3}@{#4}}}

\def\srcTortue#1#2#3#4{%
  \index[src]{tortue@Tortue!{#1}@{#2}}
  \index[src]{とるちゆ@トルチュ!{#3}@{#4}}}

\def\srcVenaison#1#2#3#4{%
  \index[src]{venaison@Venaison!{#1}@{#2}}
  \index[src]{うねそん@ヴネゾン!{#3}@{#4}}}

\def\srcVinRouge#1#2#3#4{%
  \index[src]{vin rouge@Vin rouge (au)!{#1}@{#2}}
  \index[src]{あかわいん@赤ワイン!{#3}@{#4}}}


\def\srcZingaraA#1#2#3#4{%
  \index[src]{zingara a@Zingara A!{#1}@{#2}}
  \index[src]{さんからa@ザンガラ A!{#3}@{#4}}}

\def\srcZingaraB#1#2#3#4{%
  \index[src]{zingara b@Zingara B!{#1}@{#2}}
  \index[src]{さんからb@ザンガラ B!{#3}@{#4}}}


%%%%%%% ホワイト系の派生ソース

\def\srcAlbufera#1#2#3#4{%
  \index[src]{albufera@Albuféra!{#1}@{#2}}
  \index[src]{あるひゆふえら@アルビュフェラ!{#3}@{#4}}}

\def\srcAmericaine#1#2#3#4{%
  \index[src]{americaine@Américaien!{#1}@{#2}}
  \index[src]{あめりけーぬ@アメリケーヌ!{#3}@{#4}}}

\def\srcAnchois#1#2#3#4{%
  \index[src]{anchois@Anchois!{#1}@{#2}}%
  \index[src]{あんちよひ@アンチョビ!{#3}@{#4}}}

\def\srcAurore#1#2#3#4{%
  \index[src]{aurore@Aurore!{#1}@{#2}}%
  \index[src]{おーろーる@オーロール!{#3}@{#4}}}

\def\srcAuroreMaigre#1#2#3#4{%
  \index[src]{aurore maigre@Aurore maigre!{#1}@{#2}}%
  \index[src]{おーろーるさかな@オーロール(魚料理用)!{#3}@{#4}}}

\def\srcBavaroise#1#2#3#4{%
  \index[src]{bavaroise@Bavaroise!{#1}@{#2}}%
  \index[src]{はいえるんふう@バイエルン風!{#3}@{#4}}}

\def\srcBearnaise#1#2#3#4{%
  \index[src]{bearnaise@Béarnaise!{#1}@{#2}}%
  \index[src]{へあるねーす@ベアルネーズ!{#3}@{#4}}}

\def\srcBearnaiseTomatee#1#2#3#4{%
  \index[src]{bearnaise tomatee@Béarnaise tomatée!{#1}@{#2}}%
  \index[src]{へあるねーすとまと@ベアルネース(トマト入り)!{#3}@{#4}}}

\def\srcChoron#1#2#3#4{%
  \index[src]{choron@Choron!{#1}@{#2}}%
  \index[src]{しよろん@ショロン!{#3}@{#4}}}

\def\srcBearnaiseGlaceDeViande#1#2#3#4{%
  \index[src]{bearnaise glace de viande@Bearnaise à la glace de viande!{#1}@{#2}}%
  \index[src]{へあるねーすくらすとういあんと@ベアルネーズ(グラスドヴィアンド入り)!{#3}@{#4}}}

\def\srcFoyot#1#2#3#4{%
  \index[src]{foyot@Foyot!{#1}@{#2}}%
  \index[src]{ふおいよ@フォイヨ!{#3}@{#4}}}

\def\srcValois#1#2#3#4{%
  \index[src]{valois@Valois!{#1}@{#2}}%
  \index[src]{うあろわ@ヴァロワ!{#3}@{#4}}}

\def\srcBercy#1#2#3#4{%
  \index[src]{bercy@Bercy!{#1}@{#2}}%
  \index[src]{へるしー@ベルシー!{#3}@{#4}}}

\def\srcBeurre#1#2#3#4{%
  \index[src]{beurre@Beurre (au)!{#1}@{#2}}%
  \index[src]{ふーる@オ・ブール!{#3}@{#4}}}

\def\srcBatarde#1#2#3#4{%
  \index[src]{batarde@Batarde!{#1}@{#2}}%
  \index[src]{はたると@バタルド!{#3}@{#4}}}

\def\srcBonnefoy#1#2#3#4{%
  \index[src]{bonnefoy@Bonnefoy!{#1}@{#2}}%
  \index[src]{ほぬふおわ@ボヌフォワ!{#3}@{#4}}}

\def\srcBordelaiseVinBlanc#1#2#3#4{%
  \index[src]{bordelaise vin blanc@Bordelaise au vin blanc!{#1}@{#2}}%
  \index[src]{ほるとーふうしろわいん@ボルドー風(白ワイン)!{#3}@{#4}}}

\def\srcBretonneBlanche#1#2#3#4{%
  \index[src]{bretonne blanche@Bretonne (blanche)!{#1}@{#2}}%
  \index[src]{ふるたーにゆふうしろ@ブルターニュ風(ホワイト系)!{#3}@{#4}}}

\def\srcCanotiere#1#2#3#4{%
  \index[src]{canotiere@Canotière!{#1}@{#2}}%
  \index[src]{かのていえーる@カノティエール!{#3}@{#4}}}

\def\srcCapres#1#2#3#4{%
  \index[src]{capres@Câpres (aux)!{#1}@{#2}}%
  \index[src]{けいはー@ケイパー!{#3}@{#4}}}

\def\srcCardinal#1#2#3#4{%
  \index[src]{cardinl@Cardinal!{#1}@{#2}}%
  \index[src]{かるていなる@カルディナル!{#3}@{#4}}}

\def\src#1#2#3#4{%
  \index[src]{champignons blanche@Champignons (aux)(blanche)!{#1}@{#2}}%
  \index[src]{まつしゆるーむしろ@マッシュルーム(ホワイト系)!{#3}@{#4}}}

\def\srcChantilly#1#2#3#4{%
  \index[src]{chantilly@Chantilly!{#1}@{#2}}%
  \index[src]{しやんていい@シャンティイ!{#3}@{#4}}}

\def\srcChateaubriand#1#2#3#4{%
  \index[src]{chateaubriand@Chateaubriand!{#1}@{#2}}
  \index[src]{しやとーふりやん@シャトーブリヤン!{#3}@{#4}}}

\def\srcChaudFroidBlancheOrdinaire#1#2#3#4{%
  \index[src]{choud-froid blanche ordinaire@Chaud-froid blanche ordinaire!{#1}@{#2}}
  \index[src]{しよふろわしろひようしゆん@ショフロワ(白)(標準)!{#3}@{#4}}}

\def\srcChaudFroidBlonde#1#2#3#4{%
  \index[src]{choud-froid blonde@Chaud-froid blonde!{#1}@{#2}}%
  \index[src]{しよふろわふろんと@ショフロワ(ブロンド)!{#3}@{#4}}}

\def\srcChaudFroidAurore#1#2#3#4{%
  \index[src]{chaud-froid aurore@Chaud-froid Aurore!{#1}@{#2}}%
  \index[src]{しよふろわおーろーる@ショフロワ・オーロール!{#3}@{#4}}}

\def\srcChaudFroidVertPre#1#2#3#4{%
  \index{chaud-froid vert-pre@Chaud-froid Vert-pré!{#1}@{#2}}%
  \index[src]{しよふろわうえーるふれ@ショフロワ・ヴェールプレ!{#3}@{#4}}}

\def\srcChaudFroidMaigre#1#2#3#4{%
  \index[src]{chaud-froid maigre@Chaud-froid maigre!{#1}@{#2}}%
  \index[src]{しよふろわさかな@ショフロワ(魚料理用)!{#3}@{#4}}}

\def\srcChivry#1#2#3#4{%
  \index[src]{chivry@Chivry!{#1}@{#2}}%
  \index[src]{しうり@シヴリ!{#3}@{#4}}}

\def\srcCreme#1#2#3#4{%
  \index[src]{creme@Crème (à la)!{#1}@{#2}}%
  \index[src]{くれーむ@クレーム!{#3}@{#4}}}

\def\srcCrevettes#1#2#3#4{%
  \index[src]{crevettes@Crevettes (aux)!{#1}@{#2}}%
  \index[src]{くるうえつと@クルヴェット!{#3}@{#4}}}

\def\srcCurrie#1#2#3#4{%
  \index[src]{currie@Currie!{#1}@{#2}}%
  \index[src]{かれー@カレー!{#3}@{#4}}}

\def\srcCurrieIndienne#1#2#3#4{%
  \index[src]{currie indienne@Currie à l'Indienne!{#1}@{#2}}%
  \index[src]{いんとふうかれー@インド風カレー!{#3}@{#4}}}

\def\srcDiplomate#1#2#3#4{%
  \index[src]{diplomate@Diplomate!{#1}@{#2}}%
  \index[src]{ていふろまつと@ディプロマット!{#3}@{#4}}}

\def\srcEcossaise#1#2#3#4{%
  \index[src]{ecossaise@Ecossaise!{#1}@{#2}}%
  \index[src]{すこつとらんとふう@スコットランド風!{#3}@{#4}}}

\def\srcEstragon#1#2#3#4{%
  \index[src]{estragon@Estragon!{#1}@{#2}}%
  \index[src]{えすとらこん@エストラゴン!{#3}@{#4}}}

\def\srcFinesHerbes#1#2#3#4{%
  \index[src]{fines herbes blanche@Fines herbes blanche (aux)!{#1}@{#2}}%
  \index[src]{こうそうしろ@香草(ホワイト系)!{#3}@{#4}}}

\def\srcGroseilles#1#2#3#4{%
  \index[src]{groseilles@Groseilles!{#1}@{#2}}%
  \index[src]{くろせいゆ@グロゼイユ!{#3}@{#4}}}

\def\srcHollandaise#1#2#3#4{%
  \index[src]{hollandaise@Hollandaise!{#1}@{#2}}%
  \index[src]{おらんてーす@オランデーズ!{#3}@{#4}}}

\def\srcHomard#1#2#3#4{%
  \index[src]{homard@Homard!{#1}@{#2}}%
  \index[src]{おまーる@オマール!{#3}@{#4}}}

\def\srcHongroise#1#2#3#4{%
  \index[src]{hongroise@Hongroise!{#1}@{#2}}%
  \index[src]{はんかりーふう@ハンガリー風!{#3}@{#4}}}

\def\srcHuitres#1#2#3#4{%
  \index[src]{huitres@Huîtres (aux)!{#1}@{#2}}%
  \index[src]{かきいり@牡蠣入り!{#3}@{#4}}}

\def\srcIndienne#1#2#3#4{%
  \index[src]{indienne@Indienne!{#1}@{#2}}%
  \index[src]{いんとふう@インド風!{#3}@{#4}}}

\def\srcIvoire#1#2#3#4{%
  \index[src]{ivoire@Ivoire!{#1}@{#2}}%
  \index[src]{いうおわーる@イヴォワール!{#3}@{#4}}}

\def\srcJoinville#1#2#3#4{%
  \index[src]{joinville@Joinville!{#1}@{#2}}%
  \index[src]{しよわんういる@ジョワンヴィル!{#3}@{#4}}}

\def\srcLaguipiere#1#2#3#4{%
  \index[src]{laguipiere@Laguipière!{#1}@{#2}}%
  \index[src]{らきひえーる@ラギピエール!{#3}@{#4}}}

\def\srcLivonienne#1#2#3#4{%
  \index[src]{livonienne@Livonienne!{#1}@{#2}}%
  \index[src]{りうおにあふう@リヴォニア風!{#3}@{#4}}}

\def\srcMaltaise#1#2#3#4{%
  \index[src]{maltaise@maltaise!{#1}@{#2}}%
  \index[src]{まるたふう@マルタ風!{#3}@{#4}}}

\def\srcMariniere#1#2#3#4{%
  \index[src]{mariniere@Marinière!{#1}@{#2}}%
  \index[src]{まりにえーる@マリニエール!{#3}@{#4}}}

\def\srcMateloteBlanche#1#2#3#4{%
  \index[src]{matelote blanche@Matelote blanche!{#1}@{#2}}%
  \index[src]{まとろつとしろ@マトロット(白)!{#3}@{#4}}}

\def\srcMornay#1#2#3#4{%
  \index[src]{mornay@Mornay!{#1}@{#2}}%
  \index[src]{もるねー@モルネー!{#3}@{#4}}}

\def\srcMousseline#1#2#3#4{%
  \index[src]{mousseline@Mousseline!{#1}@{#2}}%
  \index[src]{むすりーぬ@ムスリーヌ!{#3}@{#4}}}

\def\srcMousseuse#1#2#3#4{%
  \index[src]{mousseuse@Mousseuse!{#1}@{#2}}%
  \index[src]{むすーす@ムスーズ!{#3}@{#4}}}

\def\srcMoutarde#1#2#3#4{%
  \index[src]{moutarde@Moutarde!{#1}@{#2}}%
  \index[src]{むたると@ムタルド!{#3}@{#4}}}

\def\srcNantua#1#2#3#4{%
  \index[src]{nantua@Nantua!{#1}@{#2}}%
  \index[src]{なんちゆあ@ナンチュア!{#3}@{#4}}}

\def\srcNewBurgCru#1#2#3#4{%
  \index[src]{new-burg cru@New-burg avec le homard cru!{#1}@{#2}}%
  \index[src]{にゆーはーくいけ@ニューバーグ(活けオマール)!{#3}@{#4}}}

\def\srcNewBurgCuit#1#2#3#4{%
  \index[src]{new-burg cuit@New-burg avec le homard cuit!{#1}@{#2}}%
  \index[src]{にゆーはーくゆて@ニューバーグ(茹でオマール)!{#3}@{#4}}}

\def\srcNoisette#1#2#3#4{%
  \index[src]{noisette@Noisette!{#1}@{#2}}%
  \index[src]{のわせつと@ノワゼット!{#3}@{#4}}}

\def\srcNormande#1#2#3#4{%
  \index[src]{normande@Normande!{#1}@{#2}}%
  \index[src]{のるまんていふう@ノルマンディ風!{#3}@{#4}}}

\def\srcOrientale#1#2#3#4{%
  \index[src]{orientale@Orientale!{#1}@{#2}}%
  \index[src]{おりえんとふう@オリエント風!{#3}@{#4}}}

\def\srcPaloise#1#2#3#4{%
  \index[src]{paloise@Paloise!{#1}@{#2}}%
  \index[src]{ほーふう@ポー風!{#3}@{#4}}}  

\def\srcPoulette#1#2#3#4{%
  \index[src]{poulette@Poulette!{#1}@{#2}}%
  \index[src]{ふれつと@プレット!{#3}@{#4}}}

\def\srcRavigote#1#2#3#4{%
  \index[src]{ravigote@Ravigote!{#1}@{#2}}%
  \index[src]{らういこつと@ラヴィゴット!{#3}@{#4}}}

\def\srcRegencePoisson#1#2#3#4{%
  \index[src]{regence poisson@Régence pour Poissons!{#1}@{#2}}%
  \index[src]{れしやんすさかなよう@レジャンス(魚料理用)!{#3}@{#4}}}

\def\srcRegenceGarnituresVolaille#1#2#3#4{%
  \index[src]{regence garnitures volaille@Régence pour garnitures de Volaille!{#1}@{#2}}%
  \index[src]{れしやんすとりりようりのかるにちゆーるよう@レジャンス(鶏料理のガルニチュール用)!{#3}@{#4}}}

\def\srcRiche#1#2#3#4{%
  \index[src]{riche@Riche!{#1}@{#2}}%
  \index[src]{りつしゆ@リッシュ!{#3}@{#4}}}

\def\srcRubens#1#2#3#4{%
  \index[src]{rubens@Rubens!{#1}@{#2}}%
  \index[src]{るーへんす@ルーベンス!{#3}@{#4}}}

\def\srcSaintMalo#1#2#3#4{%
  \index[src]{saint-malo@Saint-Malo!{#1}@{#2}}%
  \index[src]{さんまろふう@サンマロ風!{#3}@{#4}}}

\def\srcSmitane#1#2#3#4{%
  \index[src]{smitane@Smitane!{#1}@{#2}}%
  \index[src]{すみたーぬ@スミターヌ!{#3}@{#4}}}

\def\srcSolferino#1#2#3#4{%
  \index[src]{solferino@Solférino!{#1}@{#2}}%
  \index[src]{そるふえりの@ソルフェリノ!{#3}@{#4}}}

\def\srcSoubise#1#2#3#4{%
  \index[src]{soubise@Soubise!{#1}@{#2}}
  \index[src]{すひーす@スビーズ!{#3}@{#4}}}

\def\srcSoubiseTomatee#1#2#3#4{%
  \index[src]{soubise tomatee@Soubise tomatée!{#1}@{#2}}%
  \index[src]{すひーすとまといり@スビース(トマト入り)!{#3}@{#4}}}

\def\srcSouchet#1#2#3#4{%
  \index[src]{souchet@Souchet!{#1}@{#2}}%
  \index[src]{すーしえ@スーシェ!{#3}@{#4}}}

\def\srcTyrolienne#1#2#3#4{%
  \index[src]{tyrolienne@Tyrolienne!{#1}@{#2}}%
  \index[src]{ちろるふう@チロル風!{#3}@{#4}}}

\def\srcTyrolienneAncienne#1#2#3#4{%
  \index[src]{tyroienne ancienne@!Tyrolienne à l'ancienne!{#1}@{#2}}%
  \index[src]{ちろるふうくらしつく@チロル風・クラシック!{#3}@{#4}}}

\def\srcValois#1#2#3#4{%
  \index[src]{valois@Valois!{#1}@{#2}}%
  \index[src]{うあろわ@ヴァロワ!{#3}@{#4}}}

\def\srcVenitienne#1#2#3#4{%
  \index[src]{venitienne@Vénitienne!{#1}@{#2}}%
  \index[src]{うえねついあふう@ヴェネツィア風!{#3}@{#4}}}

\def\srcVeron#1#2#3#4{%
  \index[src]{veron@Véron!{#1}@{#2}}%
  \index[src]{うえろん@ヴェロン!{#3}@{#4}}}

\def\srcVillageoise#1#2#3#4{%
  \index[src]{villageoise@Villageoise!{#1}@{#2}}%
  \index[src]{むらひとふう@村人風!{#3}@{#4}}}

\def\srcVilleroy#1#2#3#4{%
  \index[src]{villeroy@Villeroy!{#1}@{#2}}%
  \index[src]{ういるろわ@ヴィルロワ!{#3}@{#4}}}

\def\srcVilleroySoubisee#1#2#3#4{%
  \index[src]{villeroy soubisee@Villeroy soubisée!{#1}@{#2}}%
  \index[src]{ういるろわすひーすいり@ヴィルロワ(スビーズ入り)!{#3}@{#4}}}

\def\srcVilleroyTomatee#1#2#3#4{%
  \index[src]{villeroy tomatee@Villeroy tomatée!{#1}@{#2}}%
  \index[src]{ういるろわとまといり@ヴィルロワ(トマト入り)!{#3}@{#4}}}

\def\srcVinBlanc#1#2#3#4{%
  \index[src]{vin blanc@Vin blanc!{#1}@{#2}}%
  \index[src]{しろわいん@白ワイン!{#3}@{#4}}}

%%%%イギリス風ソース(温製)

\def\srcCranberriesSauce#1#2#3#4{%
  \index[src]{airelles@Airelles (aux)(Cramberries-Sauce)!{#1}@{#2}}%
  \index[src]{くらんへりー@クランベリー!{#3}@{#4}}}

\def\srcAlbert#1#2#3#4{%
  \index[src]{albert@Albert (Albert-Sauce)!{#1}@{#2}}%
  \index[src]{あるはーと@アルバート!{#3}@{#4}}}

\def\srcAromatic#1#2#3#4{%
  \index[src]{aromates@Aromates (aux)(Aromatic-Sauce)!{#1}@{#2}}%
  \index[src]{あろまていつく@アロマティック!{#3}@{#4}}}

\def\srcButter#1#2#3#4{%
  \index[src]{Beurre anglaise@Beurre (au)(Butter-Sauce)!{#1}@{#2}}%
  \index[src]{はたーそーすいきりす@バターソース(イギリス風)!{#3}@{#4}}}

\def\srcCaper#1#2#3#4{%
  \index[src]{capres anglaise@Câpres (aux)(Capers-Sauce)!{#1}@{#2}}%
  \index[src]{けいはーいきりす@ケイパー(イギリス風)!{#3}@{#4}}}

\def\srcCelery#1#2#3#4{%
  \index[src]{celeri anglaise@Céleri (au)(Celery-Sauce)!{#1}@{#2}}%
  \index[src]{せろりいきりす@セロリ(イギリス風)!{#3}@{#4}}}

\def\srcRoeBuck#1#2#3#4{%
  \index[src]{chevreuil anglaise@Chevreuil (Roe-buck Sauce)!{#1}@{#2}}%
  \index[src]{ろーはつく@ローバック(イギリス風)!{#3}@{#4}}}

\def\srcCream#1#2#3#4{%
  \index[src]{creme anglaise@Crème à l'anglaise (Cream-Sauce)!{#1}@{#2}}%
  \index[src]{くりーむ@クリーム(イギリス風)!{#3}@{#4}}}

\def\srcShrimps#1#2#3#4{%
  \index[src]{crevettes anglaise@Crevettes (aux)(Shrimps-Sauce)!{#1}@{#2}}%
  \index[src]{しゆりんふ@シュリンプ(イギリス風)!{#3}@{#4}}}

\def\srcDevilled#1#2#3#4{%
  \index[src]{diable anglaise@Diable à l'anglaise (Devilled-Sauce)!{#1}@{#2}}%
  \index[src]{てひる@デビル(イギリス風)!{#3}@{#4}}}

\def\srcEcossaise#1#2#3#4{%
  \index[src]{ecossaise@Ecossaise (Scotch eggs Sauce)!{#1}@{#2}}%
  \index[src]{すこつちえつく@スコッチエッグ!{#3}@{#4}}}

\def\srcFennel#1#2#3#4{%
  \index[src]{fenouil anglaise@Fenouil (Fennel Sauce)!{#1}@{#2}}%
  \index[src]{ふえんねるいきりす@フェンネル!{#3}@{#4}}}

\def\srcGooseberry#1#2#3#4{%
  \index[src]{groseilles anglaise@Groseilles (aux)(Gooseberry Sauce)!{#1}@{#2}}%
  \index[src]{くーすへりー@グーズベリー!{#3}@{#4}}}

\def\srcLobster#1#2#3#4{%
  \index[src]{homard anglaise@Homard à l'anglaise!{#1}@{#2}}%
  \index[src]{ろふすたー@ロブスター!{#3}@{#4}}}

\def\srcOyster#1#2#3#4{%
  \index[src]{huitres anglaise@Huîtres (aux)(Oyster Sauce)!{#1}@{#2}}%
  \index[src]{かきいきりす@牡蠣入り(イギリス風)!{#3}@{#4}}}

\def\srcBrownOyster#1#2#3#4{%
  \index[src]{huitres brune anglaise@Brune aux huîtres (Brown Oyster Sauce)!{#1}@{#2}}%
  \index[src]{かきいりふらうんいきりす@牡蠣入りブラウン(イギリス風)!{#3}@{#4}}}

\def\srcBrownGraby#1#2#3#4{%
  \index[src]{jus colore anglaise@Jus coloré (Brown Gravy)!{#1}@{#2}}%
  \index[src]{ふらうんくれいういー@ブラウングレイヴィー!{#3}@{#4}}}

\def\srcEggs#1#2#3#4{%
  \index[src]{oeufs anglaise@OEufs à l'anglaise (aux)(Egg Sauce)!{#1}@{#2}}%
  \index[src]{えつくそーす@エッグ(イギリス風)!{#3}@{#4}}}

\def\srcEggsAndButter#1#2#3#4{%
  \index[src]{oeufs beurre fondu anglaise@OEufs au Beurre fondu (aux)(Eggs and Butter Sauce)!{#1}@{#2}}%
  \index[src]{えつくあんとはたーそーす@エッグアンドバター(イギリス風)!{#3}@{#4}}}

\def\srcOnions#1#2#3#4{%
  \index[src]{oignons anglaise@Oignons (aux)(anglaise)!{#1}@{#2}}%
  \index[src]{おにおんそーす@オニオン(イギリス風)!{#3}@{#4}}}

\def\srcBread#1#2#3#4{%
  \index[src]{pain anglaise@Pain (au)(Bread Sauce)!{#1}@{#2}}%
  \index[src]{ふれつとそーす@ブレッド(イギリス風)!{#3}@{#4}}}

\def\srcFriedBread#1#2#3#4{%
  \index[src]{pain frit anglaise@Pain frit (au)(Fried Bread Sauce)!{#1}@{#2}}%
  \index[src]{ふらいとふれつとそーす@フライドブレッド!{#3}@{#4}}}

\def\srcPersley#1#2#3#4{%
  \index[src]{persil anglaise@Persil (Persley Sauce)!{#1}@{#2}}%
  \index[src]{はせりそーす@パセリ(イギリス風)!{#3}@{#4}}}

\def\srcPersilPoissons#1#2#3#4{%
  \index[src]{persil poissons anglaise@Persil pour Poissons (anglaise)!{#1}@{#2}}%
  \index[src]{さかなりようりようはせりそーす@魚料理用パセリ!{#3}@{#4}}}

\def\srcApple#1#2#3#4{%
  \index[src]{pomme anglaise@Pomme (aux)(Apple Sauce)!{#1}@{#2}}%
  \index[src]{あつふるそーす@アップル(イギリス風)!{#3}@{#4}}}

\def\srcPortWine#1#2#3#4{%
  \index[src]{porto@!Porto (au)(Port Wine Sauce)!{#1}@{#2}}%
  \index[src]{ほーとわいんそーす@ポートワイン(イギリス風)!{#3}@{#4}}}

\def\srcHorseRadhish#1#2#3#4{%
  \index[src]{raifort chaude@Raifort chaude (Horese radhish Sauce)!{#1}@{#2}}%
  \index[src]{ほーすらていしゆ@ホースラディッシュ(イギリス風)!{#3}@{#4}}}

\def\srcReform#1#2#3#4{%
  \index[src]{reforme anglais@Rérorme (Reform Sauce)!{#1}@{#2}}%
  \index[src]{りふおーむ@リフォーム(イギリス風)!{#3}@{#4}}}

\def\srcSageAndOnions#1#2#3#4{%
  \index[src]{sauge oignons anglaise@Sauge et Oignons (Sage and onions Sauce)!{#1}@{#2}}%
  \index[src]{せーしとたまねき@セージと玉ねぎ(イギリス風)!{#3}@{#4}}}

\def\srcYorkshire#1#2#3#4{%
  \index[src]{yorkshire@Yorkshire!{#1}@{#2}}%
  \index[src]{よーくしやー@ヨークシャー(イギリス風)!{#3}@{#4}}}


%%%%% 冷製ソース

\def\srcAioli#1#2#3#4{%
  \index[src]{aioli@aioli, ou beurre de provence!{#1}@{#2}}%
  \index[src]{あいより@アイヨリ/プロヴァンスバター!{#3}@{#4}}}

\def\srcAndalouse#1#2#3#4{%
  \index[src]{andalouse@Andalouse!{#1}@{#2}}%
  \index[src]{あんたるしあふう@アンダルシア風!{#3}@{#4}}}

\def\srcBohemienne#1#2#3#4{%
  \index[src]{bohemienne@Bohémienne!{#1}@{#2}}%
  \index[src]{ほへにあのむすめ@ボヘミアの娘!{#3}@{#4}}}

\def\srcChantillyFroide#1#2#3#4{%
  \index[src]{chantilly froide@Chantilly (froide)!{#1}@{#2}}%
  \index[src]{しやんていいれいせい@シャンティイ(冷製)!{#3}@{#4}}}

\def\srcGenoise#1#2#3#4{%
  \index[src]{genoise@Génoise (froide)!{#1}@{#2}}%
  \index[src]{しえのあうふう@ジェノヴァ風(冷製)!{#3}@{#4}}}

\def\srcGribiche#1#2#3#4{%
  \index[src]{gribiche@Gribiche!{#1}@{#2}}%
  \index[src]{くりひつしゆ@グリビッシュ(冷製)!{#3}@{#4}}}

\def\srcGroseillesRaifort#1#2#3#4{%
  \index[src]{groseilles raifort@Groseilles au Raifort (froide)!{#1}@{#2}}%
  \index[src]{くろせいゆれふおーる@グロゼイユ・レフォール風味!{#3}@{#4}}}

\def\srcItalienneFroide#1#2#3#4{%
  \index[src]{italienne froide@Italienne (froide)!{#1}@{#2}}%
  \index[src]{いたりあふうれいせい@イタリア風(冷製)!{#3}@{#4}}}

\def\srcMayonnaise#1#2#3#4{%
  \index[src]{mayonnaise@Mayonnaise!{#1}@{#2}}%
  \index[src]{まよねーす@マヨネーズ!{#3}@{#4}}}

\def\srcMayonnaiseFouette#1#2#3#4{%
  \index[src]{mayonnaise fouette@!Mayonnaise fouettée, à la russe{#1}@{#2}}%
  \index[src]{ろしあふうほいつふまよねーす@ロシア風ホイップマヨネーズ!{#3}@{#4}}}

\def\srcMayonnaisesDiverses#1#2#3#4{%
  \index[src]{mayonnaises diverses@Mayonnaises diverses!{#1}@{#2}}%
  \index[src]{まよねーすのはりえーしよん@マヨネーズのバリエーション!{#3}@{#4}}}

\def\srcMousquetqire#1#2#3#4{%
  \index[src]{mousquetaire@Mousquetaire (froide)!{#1}@{#2}}%
  \index[src]{むすくてーる@ムルクテール(冷製)!{#3}@{#4}}}

\def\srcMoutardeCreme#1#2#3#4{%
  \index[src]{moutarde creme@Moutarde à la Crème (froide)!{#1}@{#2}}%
  \index[src]{むたるとなまくりーむ@ムタルド・生クリーム入り!{#3}@{#4}}}

\def\srcRavigote#1#2#3#4{%
  \index[src]{ravigote@Ravigote!{#1}@{#2}}%
  \index[src]{らういこつと@ラヴィゴット!{#3}@{#4}}}

\def\srcVinaigrette#1#2#3#4{%
  \index[src]{vinaigrette@Vinaigrette!{#1}@{#2}}%
  \index[src]{ういねくれつと@ヴィネグレット!{#3}@{#4}}}

\def\srcRemoulade#1#2#3#4{%
  \index[src]{remoulade@Rémoulade!{#1}@{#2}}%
  \index[src]{れむらーと@レムラード!{#3}@{#4}}}

\def\srcRusse#1#2#3#4{%
  \index[src]{russe@Russe!{#1}@{#2}}%
  \index[src]{ろしあふう@ロシア風!{#3}@{#4}}}

\def\srcTartare#1#2#3#4{%
  \index[src]{tartare@Tartare!{#1}@{#2}}%
  \index[src]{たるたる@たるたる!{#3}@{#4}}}

\def\srcVerte#1#2#3#4{%
  \index[src]{verte@verte!{#1}@{#2}}%
  \index[src]{うえると@ヴェルト(冷製)!{#3}@{#4}}}

\def\srcVincent#1#2#3#4{%
  \index[src]{vincent@Vincent!{#1}@{#2}}%
  \index[src]{うあんさん@ヴァンサン(冷製)!{#3}@{#4}}}

\def\srcSuedoise#1#2#3#4{%
  \index[src]{suedoise@Suédoise!{#1}@{#2}}%
  \index[src]{すうえーてんふうれいせい@スウェーデン風(冷製)!{#3}@{#4}}}

\def\srcCambridge#1#2#3#4{%
  \index[src]{cambridge@Cambridge!{#1}@{#2}}%
  \index[src]{けんふりつし@ケンブリッジ!{#3}@{#4}}}

\def\srcCumberland#1#2#3#4{%
  \index[src]{cumberland@Cumberland!{#1}@{#2}}%
  \index[src]{かんはーらんと@カンバーランド!{#3}@{#4}}}

\def\srcGloucester#1#2#3#4{%
  \index[src]{gloucester@Gloucester!{#1}@{#2}}%
  \index[src]{くろすたー@グロスター!{#3}@{#4}}}

\def\srcMint#1#2#3#4{%
  \index[src]{menthe@Menthe (Mint Sauce)!{#1}@{#2}}%
  \index[src]{みんと@ミント!{#3}@{#4}}}

\def\srcOxford#1#2#3#4{%
  \index[src]{oxford@Oxford!{#1}@{#2}}%
  \index[src]{おつくすふおーと@オックスフォード!{#3}@{#4}}}

\def\srcColdHorsradish#1#2#3#4{%
  \index[src]{coldhorseradish@Cold Hordradich!{#1}@{#2}}%
  \index[src]{ほーすらていつしゆれいせい@ホースラディッシュ(冷製)!{#3}@{#4}}}




\def\srcBeurreEcrevisse#1#2#3#4{%
  \index[src]{beurre ecrevisse@Beurre d'Ecrevisse!{#1}@{#2}}%
  \index[src]{えくるういすはたー@エクルヴィスバター!{#3}@{#4}}}















\makeatother


%% Local Variables:
%% TeX-engine: luatex
%% End:

%% レイアウト調整(A4Paper,14Q,twoside,ltjsbook.cls) 
%%
\setlength{\hoffset}{0\zw}
\setlength{\oddsidemargin}{0\zw}%タブレット前提の中央配置
\setlength{\evensidemargin}{\oddsidemargin}
% \setlength{\oddsidemargin}{1\zw}%製本時に右ページのみをオフセット
%\setlength{\evensidemargin}{0pt}%
\setlength{\fullwidth}{45\zw}
\setlength{\textwidth}{45\zw}%%ltjsclassesのみ有効
%\setlength{\fullwidth}{159mm}
%\setlength{\textwidth}{159mm}
\setlength{\marginparsep}{0pt}
\setlength{\marginparwidth}{0pt}
\setlength{\footskip}{0pt}
\setlength{\voffset}{-17mm}
\setlength{\textheight}{265mm}
\setlength{\parskip}{0pt}
\setlength{\parindent}{0pt}

\newcommand{\atoaki}{\vspace{1.25mm}}

%%分数の表記Obsolete
\usepackage{xfrac}
\let\frac\sfrac





\begin{document}
\maketitle

\begin{Main}

\hypertarget{grandes-sauces-de-base}{%
\section{基本ソース}\label{grandes-sauces-de-base}}

\begin{frsecenv}

Grandes Sauces de Base

\end{frsecenv}

\index{そーす@ソース!きほん@\textbf{基本---}|(}
\index{sauce@sauce!grandes@\textbf{Grandes ---s de Base}|(}

\end{Main}

\begin{recette}

\hypertarget{sauce-espagnole}{%
\subsubsection[ソース・エスパニョル]{\texorpdfstring{ソース・エスパニョル\footnote{本節冒頭では、ルーがスペインの料理人によってもたらされ、その結果としてソース・エスパニョルが作られるようになったと読める記述があるが、これはむしろ誤りと考えるべき。エスパニョル
  espagnol(e)は「スペイン(風)の」意だが、スペイン料理起源というわけでもない。スペインを想起させるトマトを使うから、あるいは、ソースが茶褐色なのがムーア系スペイン人を想起させるから、など定説はない。カレーム『19世紀フランス料理』第3巻に収められたソース・エスパニョルの作り方は、フォンをとるところから始まり4ページにわたって詳細なものとなっている(pp.8-11)。その中で、肉を入れた鍋に少量のブイヨンを注いで煮詰めることを繰り返す。ここまでは18世紀の料理書で一般的な手法であるが、その後に大量のブイヨンを注いだ後、いきなり強火にかけるのではなく、弱火で加熱していくやり方を「スペイン式の方法」と述べている。カレームにおいては、これがソースの名称の根拠のひとつになっていると考えていいだろう。もちろん、ソース・エスパニョルという名称のソースはカレーム以前からあり、1806年刊のヴィアール『帝国料理の本』にもカレームのレシピより簡単だが、ほぼ同様のものが基本ソースとして収録されている。また、それ以前にもソース・エスパニョルに類する名称のソースはあったが、たとえば1739年刊ムノン『新料理研究』第2巻にある「スペイン風ソース」はかなり趣きが異なる(コリアンダーひと把みを加えるのが特徴的)。同じ料理名でも時代や料理書の著者によってまったく違う料理になっていることは、食文化史において珍しいことではない。また、とりわけ料理名に地名、国名が冠されているものの中には根拠や由来のはっきりしないものも多い。いずれにしても、本書のソース・エスパニョルの源流は19世紀初頭のヴィアールあたりからと考えられる。ソース・エスパニョルは19世紀を通して普及し、茶色いソースの代表的な存在となった。こんにちでもフォンドヴォーをベースとしたソースは、ルーでとろみ付けこそしないが、仔牛の骨などから出るコラーゲンによるとろみを利用したもので、仕上がりの色合いや、ごく標準的ともいえる風味付けの方法などが引継がれ続けている調理現場も少なくない。もっとも、上述のように本書では「茶色いルー」を使うところに「エスパニョル」であることの理由を見い出そうとしていると解釈される。}}{ソース・エスパニョル}}\label{sauce-espagnole}}

\begin{frsubenv}

Sauce espagnole

\end{frsubenv}

\index{そーす@ソース!えすはによる@---・エスパニョル}
\index{えすはによる@エスパニョル!そーす@ソース・---}
\index{すへいんふう@スペイン風(エスパニョル)!そーすえすはによる@ソース・エスパニョル}
\index{sauce@sauce!espagnole@--- Espagnole}
\index{espagnol@espagnol!sauce@Sauce ---e}
\index[src]{えすはによる@エスパニョル} \index[src]{espagnole@Espagnole}

(仕上がり5 L分)

\begin{itemize}
\item
  とろみ付けのための\protect\hyperlink{roux-brun}{ルー}\ldots{}\ldots{}625
  g。
\item
  \protect\hyperlink{fonds-brun}{茶色いフォン}(ソースを仕上げるのに必要な全量)\ldots{}\ldots{}12
  L。
\item
  \protect\hyperlink{mirepoix}{ミルポワ}\footnote{mirepoix
    (ミルポワ)。ソースやフォンにコクを与える目的で、細かいさいの目に切った香味野菜や塩漬け豚ばら肉を合わせたもの。18世紀にミルポワ公爵の料理人が考案したといわれているが真偽は不明。同様のものにmatignon(マティニョン)がある。ミルポワより大きめのさいの目に切るのが一般的とされるが、調理現場によってはあまり区別せずミルポワとのみ呼称するケースも多い。第2章ガルニチュール、\protect\hyperlink{mirepoix}{ミルポワ}訳注参照。}(香味素材)\ldots{}\ldots{}小さなさいの目に切った塩漬け豚ばら肉150
  g、2 mm程度のさいの目\footnote{brunoise (ブリュノワーズ)。1〜2 mm
    のさいの目に切ること。 couper en
    mirepoix(クゥペオンミルポワ)ミルポワに切るとも言う。}に切ったにんじん250
  gと玉ねぎ 150
  g、タイム2枝、ローリエの葉2枚。\index{みるほわ@ミルポワ}\index{mirepoix}
\item
  作業手順
\end{itemize}

\begin{enumerate}
\def\labelenumi{\arabic{enumi}.}
\item
  フォン8
  Lを鍋で沸かす。あらかじめ柔らかくしておいたルーを加え、木杓子か泡立て器で混ぜながら沸騰させる。

  弱火にして\footnote{原文から直訳すると「鍋を火の脇に置く」だが、現代の調理環境では単純に「弱火にする」と解釈していい。}微沸騰の状態を保つ。
\item
  以下のようにしてあらかじめ用意しておいたミルポワを投入する。ソテー鍋に塩漬け豚ばら肉を入れて火にかけて脂を溶かす。そこに、細かく刻んだにんじんと玉ねぎ、タイム、ローリエの葉を加える。野菜が軽く色づくまで強火で炒める。丁寧に、余分な脂を捨てる。これをソースに加える。野菜を炒めたソテー鍋に白ワイン約100
  mL\footnote{原文 un verre de vin blanc
    (アンヴェールドヴァンブロン)。直訳すると「グラス1杯の白ワイン」だが、本書において
    un verre de 〜は「約1 dL=100 mL」と覚えておくといいだろう。}を加えてデグラセ
  \footnote{dégrasser
    鍋に粘液状になって付着している肉汁を酒類あるいは水で溶かし出してソースなどに利用すること。}し、それを半量まで煮詰める。これも同様にソースの鍋に加える。こまめに浮いてくる夾雑物を徹底的に取り除き\footnote{dépouiller
    デプイエ。前節「ルーの火入れについて」訳注参照。}ながら弱火で約1
  時間煮込む。
\item
  ソースをシノワ\footnote{小さな穴が多く空けられた円錐形で、取っ手の付いた漉し器の一種。金属製のものが主流。}で、ミルポワ野菜を軽く押しながら漉し、別の片手鍋に移す。フォン2
  Lを注ぎ足す。さらに2時間、微沸騰の状態を保ちながら煮込む。その後、陶製の鍋に移し、ゆっくり混ぜながら冷ます。
\item
  翌日、再び厚手の片手鍋に移してから、フォン2 Lとトマトピュレ1
  Lまたは同等の生のトマトつまり2
  kgを加える。トマトピュレを用いる場合は、あらかじめオーブンでほとんど茶色になるまで焼いておくといい。そうするとトマトピュレの酸味を抜くことが出来る。そうすればソースを澄ませる作業が楽になるし、ソースの色合いも温かそうで美しいものになる。ソースをヘラか泡立て器で混ぜながら強火で沸騰させる。弱火にして1時間微沸騰の状態を保つ。最後に、表面に浮いている不純物を、細心の注意を払いながら徹底的に取り除く。布で漉し、完全に冷めるまで、ゆっくり混ぜ続けること\footnote{この、ヘラなどでゆっくり混ぜながら冷ます作業を
    vanner
    (ヴァネ)すると呼ぶが、日本の調理現場ではあまり用いられていない。}。
\end{enumerate}

\hypertarget{nota-sauce-espagnole}{%
\subparagraph{【原注】}\label{nota-sauce-espagnole}}

ソース・エスパニョルで仕上げに不純物を取り除くのにかかる時間はいちがいには言えない。これは、ソースに用いるフォンの質次第で変わるからだ。

ソースにするフォンが上質なものであればある程、仕上げに不純物を取り除く作業は早く済む。そういう場合には、ソース・エスパニョルを5時間で作ることも無理ではない。

\atoaki{}

\hypertarget{sauce-espagnole-maigre}{%
\subsubsection[魚料理用ソース・エスパニョル]{\texorpdfstring{魚料理用ソース・エスパニョル\footnote{フランス語のソース名にあるmaigre(メーグル)はこの場合、一般的に「魚用、魚料理用」と訳すが、厳密には「小斉の際の料理用」となろう。小斉とは、カトリックで古くから特定の期間、曜日に肉類を断つ食事をする宗教的食習慣。日本の「お精進」とニュアンスは近いが、小斉においては忌避されるのは鳥獣肉のみであり、魚介や乳製品はいいとされた。こじつけのように、水鳥は水のものだから魚介扱いであり、またイルカも魚類として扱われていた。小斉が行なわれるのは復活祭の前46日間(四旬節、逆に言えばカーニバルの最終日マルディグラの翌日から46日)と、週に一度(多くの場合は金曜)であった。合計すると小斉が行なわれるのは年間100日近くもあり、中世から18世紀の料理人たちは小斉の宴席に供する料理に工夫を凝らしていた。この習慣は19世紀になるとだんだん廃れていき、エスコフィエの時代には、料理人に対して小斉のための料理を要求することは少なくなっていった。}}{魚料理用ソース・エスパニョル}}\label{sauce-espagnole-maigre}}

\begin{frsubenv}

Sauce espagnole maigre

\end{frsubenv}

\index{そーす@ソース!えすはによるるさかな@---・エスパニョル (魚料理用)}
\index{えすはによる@エスパニョル!そーすさかなよう@ソース・--- (魚料理用)}
\index{すへいんふう@スペイン風(エスパニョル)!そーすえすはによるさかな@ソース・エスパニョル(魚料理用)}
\index{sauce@sauce!espagnole maigre@--- Espagnole maigre}
\index{espagnol@espagnol!sauce maigre@Sauce Espagnole maigre}
\index[src]{さかなりようりようえすはによる@魚料理用エスパニョル}
\index[src]{espagnole maigre@Espagnole maigre}

(仕上がり5 L分)

\begin{itemize}
\item
  バターを用いて\footnote{初版〜第三版にかけては、茶色いルーを作るのに「バターまたは、きれいなグレスドマルミット(コンソメなどを作る際に表面に浮いてくる脂をすくい取って、不純物を漉し取ったものであり、基本的に獣脂)」を用いる、とある。上述のように、カトリックにおける「小斉」の場合、獣脂は忌避されたがバターなどの乳製品は許容された。そのため特に「バターを用いて作ったルー」という指定がなされ、第四版では茶色いルーに澄ましバターのみを使う旨が強調されたが、ここでは初版以来の記述がそのまま残っているために、やや冗長に思われる表現となっている。}作った\protect\hyperlink{roux-brun}{ルー}\ldots{}\ldots{}500
  g。
\item
  \protect\hyperlink{fumet-de-poisson}{魚のフュメ(フュメドポワソン)}(ソースを仕上げるために必要な全量)\ldots{}\ldots{}10
  L。
\item
  ミルポワ\ldots{}\ldots{}標準的なソース・エスパニョルと同じ\protect\hyperlink{mirepoix}{ミルポワ}野菜を同量と、塩漬け豚ばら肉の代わりにバターを用い、マッシュルームまたはマッシュルームの切りくず\footnote{champignons
    de Paris
    (シャンピニョンドパリ)いわゆるマッシュルームは、ガルニチュールなど料理の一部として提供する際に、トゥルネ
    tourner
    といって螺旋(らせん)状の切れ込みを入れて装飾したものを使う。その際に少なくない量、具体的には重量で15〜20%程度が「切りくず」として発生するのでこれを利用する。なお、tourner(トゥルネ)の原義は「回す」であり、包丁を持った側の手は動かさずに、材料のほうを回すようにして切れ目を入れたり、アーティチョークや果物などの皮を剥くことを意味する。}250
  gを加える。
\item
  作業手順\ldots{}\ldots{}標準的なソース・エスパニョルとまったく同様に作る。
\item
  加熱時間と不純物を取り除くのに必要な時間\ldots{}\ldots{}5時間。
\end{itemize}

仕上げに漉してから、標準的なソース・エスパニョルとまったく同様に、完全に冷めるまでゆっくり混ぜ続けること。

\atoaki{}

\hypertarget{observation-sauce-espagnole-maigre}{%
\subsubsection{魚料理用ソース・エスパニョル補足}\label{observation-sauce-espagnole-maigre}}

このソースを日常的な料理のベースとなる仕込みに含めるかどうかについては意見が分れるところだ。

普通のソース・エスパニョルは、つまるところ風味の点ではほとんどニュートラルなものだから、それに魚のフュメを加えれば、魚料理用ソース・エスパニョルとして充分に通用するだろう。どうしても上で挙げた魚料理用ソース・エスパニョルが必要になるのは、宗教的に厳格に小斉の決まりを守って料理を作る場合のみで、さすがにその場合は代用品などない。

\atoaki{}

\hypertarget{sauce-demi-glace}{%
\subsubsection[ソース・ドゥミグラス]{\texorpdfstring{ソース・ドゥミグラス\footnote{日本の洋食などで一般的な「デミグラス」あるいは「ドミグラス」」とはかなり異なった仕上りのソースであることに注意。ソース・エスパニョルの仕上げにあたって、徹底的に不純物を取り除くことを何度も強調しているのは、透き通った茶色がかった色合いの、なめらかなソースを目指すからであり、それをさらに徹底させるということは、透明度、なめらかさの面でさらに上を目指すということを意味するからだ。ちなみに、アメリカに本社のあるメーカーの「デミグラスソース」の缶詰はもっぱら日本で販売されている製品であり、ヨーロッパおよびアメリカでは同一ブランドに該当する商品は存在しないようだ。}}{ソース・ドゥミグラス}}\label{sauce-demi-glace}}

\begin{frsubenv}

Sauce demi-glace

\end{frsubenv}

\index{そーす@ソース!とうみくらす@---・ドゥミグラス}
\index{とうみくらす@ドゥミグラス!そーす@ソース・---}
\index{sauce@sauce!demi-glace@--- Demi-glace}
\index{demi-glace@demi-glace!sauce@sauce ---}
\index[src]{demi-glace@Demi-glace}
\index[src]{とうみくらす@ドゥミグラス}

一般に「ドゥミグラス」と呼ばれているものは、いったん仕上がった\protect\hyperlink{sauce-espagnole}{ソース・エスパニョル}をさらに、もうこれ以上は無理という位に徹底的に不純物を取り除いたもののことだ。

最後の仕上げに\protect\hyperlink{glace-de-viande}{グラスドヴィアンド}などを加える。風味付けに何らかの酒類\footnote{本書ではマデイラ酒(マデイラワイン、ポルトガルの酒精強化ワイン、すなわちブドウ果汁が酵母により醗酵している途中で蒸留酒を加えて醗酵を止める製法のもので、甘口のものが多い)が用いられることが多い。}を加えれば、当然ながらソースの性格も変わるので、最終的な使い途に応じて決めること。

\hypertarget{nota-sauce-demi-glace}{%
\subsubsection{【原注】}\label{nota-sauce-demi-glace}}

ソースの色合いを決めるワインを仕上げに加える際には、「火から外して」行なうこと。沸騰しているとワインの香りがとんでしまうからだ。

\atoaki{}

\hypertarget{jus-de-veau-lie}{%
\subsubsection{とろみを付けた仔牛のジュ}\label{jus-de-veau-lie}}

\begin{frsubenv}

Jus de veau lié

\end{frsubenv}

\index{しゆ@ジュ!こうしのしゆ@仔牛の---(とろみを付けた)}
\index{そーす@ソース!とろみをつけたこうしのしゆ@とろみを付けた仔牛のジュ}
\index{こうし@仔牛!とろみをつけたこうしのしゆ@とろみを付けた---のジュ}
\index{jus@jus!jus veau lie@--- de veau lié}
\index{veau@veau!jus lie@jus de --- lié}
\index{jus de veau lie@Jus de veau lié}
\index[src]{とろみをつけたこうしのしゆ@とろみを付けた仔牛のジュ}

(仕上がり1 L分)

\begin{itemize}
\item
  仔牛のフォン\ldots{}\ldots{}\protect\hyperlink{jus-de-veau-brun}{仔牛の茶色いフォン}
  4 L。
\item
  とろみ付け材料\ldots{}\ldots{}アロールート\footnote{allow-root
    南米産のクズウコンを原料とした良質のでんぷん。日本では入手が難しいこともあり、コーンスターチが用いられることがほとんど。}30
  g。
\item
  作業手順\ldots{}\ldots{}よく澄んだ仔牛のフォン4
  Lを強火にかけ、\(\frac{1}{4}\) 量つまり1 Lになるまで煮詰める。
\end{itemize}

大さじ数杯分の冷たいフォンでアロールートを溶く。これを沸騰している鍋に加える。1分程度だけ火にかけ続けたら、布で漉す。

\hypertarget{nota-jus-de-veau-lie}{%
\subparagraph{【原注】}\label{nota-jus-de-veau-lie}}

この、とろみを付けた仔牛のジュは、本書では頻繁に使う指示をしているが、必ず、しっかりした味で透き通った、きれいな薄茶色に仕上げること。

\atoaki{}

\hypertarget{veloute}{%
\subsubsection[ヴルテ(標準的な白いソース)]{\texorpdfstring{ヴルテ\footnote{velouté
  (ヴルテ)原義は「ビロードのように柔らかな、なめらかな」。日本ではベシャメルソースと混同されやすいが、内容がまったく異なるソースなので注意。}(標準的な白いソース)}{ヴルテ(標準的な白いソース)}}\label{veloute}}

\begin{frsubenv}

Velouté, ou sauce blanche graisse

\end{frsubenv}

\index{うるて@ヴルテ!ひようひゆんてきなそーすうるて@標準的なソース・---}
\index{そーす@ソース!うるてひようひゆん@ヴルテ(標準的な)}
\index{ふるーて@ブルーテ ⇒ ヴルテ} \index{veloute@velouté}
\index{veloute@velouté!sauce blanche grasse@sauce blanche grasse}
\index[src]{veloute@Velouté} \index[src]{うるて@ヴルテ}

(仕上がり5 L分)

\begin{itemize}
\item
  とろみ付けの材料\ldots{}\ldots{}バターを用いて作った\footnote{\protect\hyperlink{sauce-espagnole-maigre}{魚料理用ソース・エスパニョル}、訳注参照。}\protect\hyperlink{roux-blond}{ブロンドのルー}
  625 g。
\item
  よく澄んだ\protect\hyperlink{fonds-blanc-ordinaire}{仔牛の白いフォン}\ldots{}\ldots{}5
  L。
\item
  作業手順\ldots{}\ldots{}ルーをフォンに溶かし込む。フォンは冷たくても熱くてもいいが、フォンが熱い場合にはソースが充分なめらかになるよう注意して溶かすこと。混ぜながら沸騰させる。微沸騰の状態を保ちながら、浮いてくる不純物を完全に取り除いていく\footnote{dépouiller
    (デプイエ)。\protect\hyperlink{sauce-espagnole}{ソース・エスパニョル}、訳注参照。}。この作業はとりわけ細心の注意を払って行なうこと。
\item
  加熱時間と不純物を取り除く作業に必要な時間\ldots{}\ldots{}1時間半。
\end{itemize}

その後、ヴルテを布で漉す\footnote{ある程度濃度のある液体やピュレを布で漉す場合、昔は「二人がかりで行なう必要があり、それぞれが巻いた布の端を左手に持ち、右手に持った木杓子を使って圧し搾る」(『ラルース・ガストロノミーク』初版、
  1938年)という方法が一般的だった。}。陶製の鍋に移してゆっくり混ぜながら完全に冷ます。

\atoaki{}

\hypertarget{veloute-de-volaille}{%
\subsubsection{鶏のヴルテ}\label{veloute-de-volaille}}

\begin{frsubenv}

Velouté de volaille

\end{frsubenv}

\index{うるて@ヴルテ!とりのうるて@鶏の---(ヴルテドヴォライユ)}
\index{そーす@ソース!うるてとり@ヴルテ(鶏)}
\index{ふるーて@ブルーテ ⇒ ヴルテ}
\index{うおらいゆ@ヴォライユ!うるてとうおらいゆ@ヴルテドヴォライユ(鶏のヴルテ)}
\index{かきん@家禽!とりのうるて@鶏のヴルテ}
\index{veloute@velouté!volaille@--- de Volaille}
\index{sauce@sauce!veloute volaille@Velout\'e de Volaille}
\index[src]{veloute de volaille@Velouté de volaille}
\index[src]{とりのうるて@鶏のヴルテ}

このヴルテの作り方だが、上述の\protect\hyperlink{veloute}{標準的なヴルテ}と、材料比率と作業はまったく同じ。使用する液体として\protect\hyperlink{fonds-de-volaille}{鶏の白いフォン(フォンドヴォライユ)}を使う。

\atoaki{}

\hypertarget{veloute-de-poisson}{%
\subsubsection{魚料理用ヴルテ}\label{veloute-de-poisson}}

\begin{frsubenv}

Velouté de poisson

\end{frsubenv}

\index{うるて@ヴルテ!さかなうるて@魚料理用---}
\index{そーす@ソース!うるてさかな@ヴルテ(魚料理用)}
\index{veloute@velouté!poisson@--- de Poisson}
\index{sauce@sauce!veloute poisson@Velouté de Poisson}
\index[src]{veloute de poisson@Velouté de poisson}
\index[src]{さかなのうるて@魚のヴルテ}

ルーと液体の分量は標準的なヴルテとまったく同じだが、仔牛のフォンではなく\protect\hyperlink{fumet-de-poisson}{魚のフュメ}を用いて作る。

ただし、魚を素材として用いるストックはどれもそうだが、手早く作業すること。不純物を取り除く作業も20分程度にとどめること。その後、布で漉し、陶製の鍋に移してゆっくり混ぜながら完全に冷ます。

\atoaki{}

\hypertarget{sauce-allemande}{%
\subsubsection[ソース・アルマンド(パリ風ソース)]{\texorpdfstring{ソース・アルマンド(パリ風ソース\footnote{原書では「パリ風ソース(元ソース・アルマンド)」となっているが、後述のように、こんにちでもソース・アルマンドの名称のほうが一般的であるため、ここではSauce
  Parisienneの「訳語」としてソース・アルマンドをあてることとした。})}{ソース・アルマンド(パリ風ソース)}}\label{sauce-allemande}}

\begin{frsubenv}

Sauce parisienne (ex-Allemande)

\end{frsubenv}

\index{そーす@ソース!ぱりふう@パリ風--- ⇒ ---・アルマンド}
\index{はりふう@パリ風!そーす@---ソース ⇒ ---・アルマンド}
\index{といつふう@ドイツ風!そーす@ソース・アルマンド}
\index{あるまん@アルマン(ド)!そーす@ソース・アルマンド}
\index{sauce@sauce!parisienne@--- parisienne (ex-Allemande)}
\index{parisien@parisien!sauce@Sauce Parisienne = Sauce Allemande}
\index{allemand@allemand!sauce@Sauce allemande (--- Parisienne)}
\index[src]{allemande@Allemande} \index[src]{あるまんと@アルマンド}

(仕上がり1 L分)

これは、\protect\hyperlink{veloute}{標準的なヴルテ}に卵黄でとろみを付けたソース。

\begin{itemize}
\item
  標準的なヴルテ\ldots{}\ldots{}1 L。
\item
  追加素材\ldots{}\ldots{}卵黄5個、白いフォン(冷たいもの)
  \(\frac{1}{2}\)
  L、粗く砕いたこしょう1ひとつまみ、すりおろしたナツメグ少々、マッシュルームの煮汁2
  dL、レモン汁少々。
\item
  作業手順\ldots{}\ldots{}厚手のソテー鍋にマッシュルームの茹で汁と白いフォン、卵黄、粗く砕いたこしょう、ナツメグ、レモン汁を入れる。泡立て器でよく混ぜ、そこにヴルテを加える。火にかけて沸騰させ、強火で
  \$\frac{2}{3} 量になるまで、ヘラで混ぜながら煮詰める。
\end{itemize}

ヘラの表面がソースでコーティングされる状態になるまで煮詰めたら、布で漉す。

膜が張らないよう、表面にバターのかけらをいくつか載せてやり、湯煎にかけておく。

\begin{itemize}
\tightlist
\item
  仕上げ\ldots{}\ldots{}提供直前に、バター100 gを加えて仕上げる。
\end{itemize}

\hypertarget{nota-sauce-allemande}{%
\subparagraph{【原注】}\label{nota-sauce-allemande}}

ソース・アルマンド(ドイツ風)とも呼ばれるが、本書では「パリ風」の名称を採用した。そもそも「アルマンド」というの名称に正当性がないからだ。習慣としてそう呼ばれてきただけであって、明らかに理屈に合わない名称だ
\footnote{エスコフィエは普仏戦争に従軍した経歴があり、ドイツ嫌いとして知られていた。}。1883年に雑誌「料理技術」に某タヴェルネ氏が寄せた記事には、当時ある優秀な料理人がアルマンドなどという理屈に合わない名称を使うのはやめたという話が出ている。

こんにち既に「パリ風ソース」の名称を採用している料理長もいる。そう呼んだほうが好ましいわけだが、残念なことにまだ一般的にはなっていない
\footnote{エスコフィエの願いもむなしく、現代においてもソース・アルマンドの名称で定着している。この「全注解」においても以後は「ソース・アルマンド」と訳しているので注意されたい。なお、「ドイツ風」というソース名の由来について、ソースの淡い黄色がドイツ人に多い金髪を想起させるからだとカレームは述べている。}。

\atoaki{}

\hypertarget{sauce-supreme}{%
\subsubsection[ソース・シュプレーム]{\texorpdfstring{ソース・シュプレーム\footnote{suprême
  原義は「至高の」だが、料理においてはしばしば鶏や鴨の胸肉、白身魚のフィレなどを意味する。また、このソースのように、とくに意味もなくこの名を料理につけられているケースも多い。}}{ソース・シュプレーム}}\label{sauce-supreme}}

\begin{frsubenv}

Sauce supême

\end{frsubenv}

\index{そーす@ソース!そーすしゆふれーむ@---・シュプレーム}
\index{しゆふれーむ@シュプレーム!そーす@ソース・---}
\index{sauce@sauce!supreme@--- Suprême}
\index{supreme@suprême!sauce@Sauce ---} \index[src]{supreme@Suprême}
\index[src]{しゆふれーむ@シュプレーム}

\protect\hyperlink{veloute-de-volaille}{鶏のヴルテ}に生クリーム\footnote{フランスの生クリームのうち、料理でよく使われるのは、日本の生クリームにやや近い「クレーム・フレッシュ・パストゥリゼ」(低温殺菌した生クリームで乳脂肪分30〜38%)のほか、「クレーム・フレッシュ・エペス」(低温殺菌後に乳酸醗酵させたもので日本で一般的な生クリームより濃度がある)、「クレーム・ドゥーブル」(殺菌後に乳酸醗酵させたもので乳脂肪分40%程度でかなり濃度がある)などがある。}を加えてなめらかに仕上げ\footnote{monter
  モンテ。原義は「上げる、ホイップする」だが、ソースの仕上げの際などに、バターや生クリームを加えて、なめらかに仕上げることも「モンテ」の語を使用する場合が多い。}たもの。ソース・シュプレームは、正しく作った場合「白さの\ruby{際}{きわ}だったとても繊細な」仕上がりのものでなくてはいけない。

(仕上がり1 L分)

\begin{itemize}
\item
  鶏のヴルテ\ldots{}\ldots{}1 L。
\item
  追加素材\ldots{}\ldots{}\protect\hyperlink{fonds-de-volaille}{鶏の白いフォン}
  1 L、マッシュルームの茹で汁1 dL、良質な生クリーム2 \(\frac{1}{2}\)
  dL。
\item
  作業手順\ldots{}\ldots{}鍋に鶏のフォンとマッシュルームの茹で汁、鶏のヴルテを入れて強火にかけ、ヘラで混ぜながら、生クリームを少しずつ加え、煮詰めていく。このヴルテと生クリームを煮詰めたものの分量は、上で示した仕上がり
  1 Lのソース・シュプレームを作るには、 \(\frac{1}{3}\)
  量まで煮詰まっていなくてはならない。
\end{itemize}

布で漉し、仕上げに1 dLの生クリームとバター80
gを加えてゆっくり混ぜながら冷ますと、丁度最初のヴルテと同量になる。

\atoaki{}

\hypertarget{sauce-bechamel}{%
\subsubsection[ベシャメルソース]{\texorpdfstring{ベシャメルソース\footnote{17世紀にルイ14世のメートルドテルを務めたこともあるルイ・ベシャメイユLouis
  Béchameil(1630〜1703)の名が冠されているこのソースは、彼自身の創案あるいは彼に仕えていた料理人によるものという説もあったが真偽は疑わしい。17世紀頃の成立であることは確かだが、おそらくは古くからあったソースを改良したものに過ぎず、また、19世紀前半のカレームのレシピはヴルテを煮詰め、卵黄と煮詰めた生クリームでとろみを付けるというものだった。同様に1867年刊グフェ『料理の本』のレシピも、炒めた仔牛肉と野菜に小麦粉を振りかけてからブイヨン注ぎ、これを煮詰め、漉してから生クリームを加えるというものだった。}}{ベシャメルソース}}\label{sauce-bechamel}}

\begin{frsubenv}

Sauce Béchamel

\end{frsubenv}

\index{そーす@ソース!へしやめる@ベシャメル---}
\index{へしやめる@ベシャメル!そーす@---ソース}
\index{sauce@sauce!bechamel@--- Béchamel}
\index{bechamel@Béchamel (sauce)} \index[src]{bechamel@Bechamel}
\index[src]{へしやめる@ベシャメル}

(仕上がり 5 L分)

\begin{itemize}
\item
  \protect\hyperlink{roux-blanc}{白いルー}\ldots{}\ldots{}650 g。
\item
  使用する液体\ldots{}\ldots{}沸かした牛乳5 L。
\item
  追加素材\ldots{}\ldots{}白身で脂肪のない仔牛肉300
  gをさいの目に切り、みじん切りにした玉ねぎ(小)2個分とタイム1枝、粗く砕いたこしょう1つまみ、塩25
  g とバターを鍋に入れて蓋をし、色付かないように弱火で蒸し煮したもの。
\item
  作業手順\ldots{}\ldots{}沸かした牛乳でルーを溶く。混ぜながら沸騰させる。ここに、先に蒸し煮しておいた野菜と調味料、仔牛肉を加える。弱火で1時間煮込む。布で漉し\footnote{\protect\hyperlink{veloute}{ヴルテ}訳注参照。}、表面にバターのかけらをいくつか載せて膜が張らないようにする。肉類を絶対に使わない\footnote{小斉のこと。\protect\hyperlink{sauce-espagnole-maigre}{魚料理用ソース・エスパニョル}訳注参照。}で調理する必要がある場合は、仔牛肉を省き、香味野菜などは上記のとおりに作ること。
\end{itemize}

このソースは次のようなやり方をすると手早く作ることも出来る。沸かした牛乳に塩、薄切りにした玉ねぎ、タイム、粗く砕いたこしょう、ナツメグを加える。蓋をして弱火で10分煮る。これを漉してルーを入れた鍋の中に入れ、強火にかけて沸騰させる。その後15〜20分だけ煮込めばいい。

\atoaki{}

\hypertarget{sauce-tomate}{%
\subsubsection{トマトソース}\label{sauce-tomate}}

\begin{frsubenv}

Sauce tomate

\end{frsubenv}

\index{そーす@ソース!とまとそーす@トマト---}
\index{とまと@トマト!ソース@---ソース}
\index{sauce@sauce!tomate@--- tomate}
\index{tomate@tomate!sauce@Sauce ---} \index[src]{tomate@Tomate}
\index[src]{とまと@トマト}

(仕上がり5 L分)

\begin{itemize}
\item
  主素材\ldots{}\ldots{}トマトピュレ4 L、または生のトマト6 kg。
\item
  \protect\hyperlink{mirepoix}{ミルポワ}\ldots{}\ldots{}さいの目に切って下茹でしておいた塩漬け豚ばら肉140
  g 、1〜2 mm 角のさいの目に刻んだにんじん200 gと玉ねぎ150
  g、ローリエの葉 1枚、タイム1枝、バター100 g。
\item
  追加素材\ldots{}\ldots{}小麦粉150 g、白いフォン2 L、にんにく2片。
\item
  調味料\ldots{}\ldots{}塩20 g、砂糖30 g、こしょう1つまみ。
\item
  作業手順\ldots{}\ldots{}厚手の片手鍋で、塩漬け豚ばら肉を軽く色付くまで炒める。ミルポワの野菜を加え、野菜も色よく炒める。小麦粉を振りかける。ブロンド色になるまで炒めてから、トマトピュレまたは潰した生トマトと白いフォン、砕いたにんにく、塩、砂糖、こしょうを加える。
\end{itemize}

火にかけて混ぜながら沸騰させる。鍋に蓋をして弱火のオーブンに入れ1時間半〜2時間加熱する。

目の細かい漉し器または布で漉す。再度、火にかけて数分間沸騰させる。保存用の器に移し、ソースが空気に触れて表面に膜が張らないよう、バターのかけらを載せてやる。

\hypertarget{nota-sauce-tomate}{%
\subparagraph{【原注】}\label{nota-sauce-tomate}}

トマトピュレを使い、小麦粉は使わず、その他は上記のとおりに作ってもいい。漉し器か布で漉してから、充分な濃度になるまでしっかり煮詰めてやること。

\index{そーす@ソース!きほん@\textbf{基本---}|)}
\index{sauce@sauce!grandes@\textbf{Grandes ---s de Base}|)}

\end{recette}

\end{document}
\newpage
\href{未、原文対照チェック}{} \href{未、日本語表現校正}{}
\href{未、その他修正}{} \href{未、原稿最終校正}{}

\hypertarget{petites-sauces-brunes-composees}{%
\section{ブラウン系の派生ソース}\label{petites-sauces-brunes-composees}}

\frsec{Petites Sauces Brunes Composées}

\index{そーす@ソース!ふらうんはせい@ブラウン系の派生---}
\index{sauce@sauce!petites brunes composees@Petites ---s Brunes Composées}
\begin{recette}
\hypertarget{sauce-bigarade}{%
\subsubsection[ソース・ビガラード]{\texorpdfstring{ソース・ビガラード\footnote{ビガラードは本来、南フランスで栽培されるビターオレンジの一種。}}{ソース・ビガラード}}\label{sauce-bigarade}}

\frsub{Sauce Bigarade}

\index{そーす@ソース!ひからーと@---・ビガラード}
\index{ひからーと@ビガラード!そーす@ソース・---}
\index{sauce@sauce!bigarade@--- Bigarade}
\index{bigarade@bigarade!sauce@Sauce ---}

\hypertarget{sauce-bigarade-pour-caneton-braise}{%
\subparagraph{仔鴨のブレゼ用}\label{sauce-bigarade-pour-caneton-braise}}

\ldots{}\ldots{}仔鴨をブレゼ\footnote{料理の仕立てとしてのブレゼはたんに「蒸し煮」することではない。原則的な手順をごく簡単に述べておく。厚めに輪切りにしたにんじんと玉ねぎをバターまたはラードで炒め、ブーケガルニとともに鍋に入れる。表面を色よく焼き固めた肉を、脂身の少ない肉の場合には豚背脂のシートで包んで素材がぴったり入る大きさ鍋に入れ、\protect\hyperlink{fonds-brun}{茶色いフォン}を注ぎ、蓋をしてオーブンに入れ、微沸騰の状態を保つようにして煮込む。火が通ったら肉を取り出し、鍋に残った煮汁でソースを作る。詳細については\protect\hyperlink{}{第7章 肉料理}参照。}した際の煮汁を漉してから浮き脂を取り除き\footnote{dégraisser
  デグレセ。}、煮詰める。煮詰まったらさらに目の細かい布で漉し、ソース1
Lあたりオレンジ4個とレモン1個の搾り汁でのばす。

\hypertarget{sauce-bigarade-pour-caneton-poele}{%
\subparagraph{仔鴨のポワレ用}\label{sauce-bigarade-pour-caneton-poele}}

\ldots{}\ldots{}仔鴨をポワレ\footnote{ポワレについても簡単に述べておく。本書においてポワレは「フライパンで焼く」という意味で用いられることは決してない(フライパンで魚などを焼くことをポワレと呼ぶようになったのは20世紀後半のこと)。本書では「ローストの一種」と定義されており(この点がカレームとはまったく異なる)、3〜4mm角に切った香味野菜(マティニョン)を生のまま鍋の底に入れ、その上に味付けをした肉を置く。溶かしバターをかけてから、蓋をして中火のオーブンに入れて蒸し焼きにする。時折様子を見て溶かしバターをかけてやること。肉に火が通ったら鍋から取り出し、\protect\hyperlink{jus-de-veau-brun}{茶色い仔牛のフォン}を注いで弱火にかけて10分程煮込み、マティニョンとして用いた野菜から風味を引き出してソースにする。これがレシピにある「ポワレのフォン」となる。}のフォンから浮き脂を取り除き、でんぷんで軽くとろみ付けする。砂糖20
gに大さじ \(\frac{1}{2}\)
杯のヴィネガーを加えて火にかけカラメル状にしたものを加える。ブレゼ用と同様に、オレンジとレモンの搾り汁でのばす。

仔鴨のブレゼ用、ポワレ用いずれの場合も、細かい千切りにしてよく下茹でしておいたオレンジの皮大さじ2とレモンの皮\footnote{柑橘類の表皮を薄く剥いてごく細い千切りにしたり、器具を用いておろしたものをzeste(ゼスト)と呼ぶ。千切りにしたものは苦味を取り除くために下茹ですることが多い。}大さじ1を加えて仕上げる。

\hypertarget{sauce-bordelaise}{%
\subsubsection{ボルドー風ソース}\label{sauce-bordelaise}}

\frsub{Sauce Bordelaise}

\index{そーす@ソース!ほるとーふう@ボルドー風---}
\index{ほるとーふう@ボルドー風!そーす@---ソース}
\index{sauce@sauce!bordelaise@--- Bordelaise}
\index{bordelais@bordelais(e)!sauce@sauce ---e}

赤ワイン3
dLにエシャロットのみじん切り大さじ2、粗く砕いたこしょう、タイム、ローリエの葉
\(\frac{1}{2}\)
枚を加えて火にかけ、\(\frac{1}{4}\)量になるまで煮詰める。ソース・エスパニョル1
dLを加えて火にかけ、浮いてくる夾雑物を丁寧に取り除きながら弱火で15分間煮る。目の細かい布で漉す。

溶かした\protect\hyperlink{glace-de-viande}{グラスドヴィアンド}大さじ1杯とレモン汁
\(\frac{1}{4}\)
個分、細かいさいの目か輪切りにしてポシェしておいた牛骨髄を加えて仕上げる。

\ldots{}\ldots{}牛、羊の赤身肉\footnote{原文 viande noire de boucherie
  (ヴィヨンドノワールドブシュリ)逐語訳すれば「肉屋の赤身肉」(noir(e)は黒の意だが肉の場合は赤身肉を指す)だが、一般的に
  viande de boucherie
  (ヴィヨンドドブシュリ)とだけ言えば、\ul{伝統的に肉屋で扱かわ\\れてきた、白身肉を除く畜産精
  肉のことで、具体的には牛、羊、場合によっては馬も含まれる}(馬肉は食材としてあまり一般的ではないが、専門店がある)。副生物(内臓や足、耳、舌肉など)は含まれない。この場合の「白身肉」とは一般的に乳呑仔牛、乳呑仔羊のことであり、鶏(およびその他の家禽)や豚は別扱いになる。ここでわざわざviande
  noire
  赤身肉と表現しているのは、19世紀後半以降、上記のような区分がやや曖昧になったことによるものだろう。以下、本書の訳ではviande
  de boucherieの訳語として「牛、羊肉」をあてることにする。}のグリル用

\hypertarget{nota-sauce-bordelaise}{%
\subparagraph{【原注】}\label{nota-sauce-bordelaise}}

こんにちではボルドー風ソースをこのように赤ワインを用いて作るが、本来的には誤りである。元来は白ワインが用いられていた。これは\protect\hyperlink{sauce-bonnefoy}{ボルドー風ソース・ボヌフォワ}として後述。

\hypertarget{sauce-bourguignonne}{%
\subsubsection{ブルゴーニュ風ソース}\label{sauce-bourguignonne}}

\frsub{Sauce Bourguignonne}

\index{そーす@ソース!ふるこーにゆふう@ブルゴーニュ風---}
\index{ふるこーにゆふう@ブルゴーニュ風!そーす@---ソース}
\index{sauce@sauce!bourguignonne@--- Bourguignonne}
\index{bourguignon@bourguignon(ne)!sauce@Sauce Bourguignonne}

上質の赤ワイン1 \(\frac{1}{2}\) L
に、エシャロット5個の薄切りとパセリの枝、タイム、ローリエの葉
\(\frac{1}{2}\) 枚、マッシュルームの切りくず\footnote{料理に使うマッシュルームは通常、トゥルネ(tourner
  包丁を持った側の手は動かさずに材料を回して切ることからついた用語)すなわち螺旋状に装飾して供するが、その際に少なくない量(具体的には重量で15〜20%)の切りくずが出るのでこれを使う。}25
gを加えて、半量になるまで煮詰める。布で漉し、ブールマニエ80 g(バター45
gと小麦粉 35 g)を加えてとろみを付ける。提供直前にバター150
gを溶かし込み、カイエンヌ\footnote{赤唐辛子の粉末だがカイエンヌは本来、品種名。日本のタカノツメと比べると辛さもややマイルドで、風味も異なる。}ごく少量で加えて風味よく仕上げる。

\ldots{}\ldots{}いろいろな卵料理や、家庭料理に好適なソース。

\hypertarget{sauce-bretonne}{%
\subsubsection{ブルターニュ風ソース}\label{sauce-bretonne}}

\frsub{Sauce Bretonne}

\index{そーす@ソース!ふるたーにゆふうふらうんけい@ブルターニュ風---(ブラウン系)}
\index{ふるたーにゆふう@ブルターニュ風!そーすふらうんけい@---ソース(ブラウン系)}
\index{sauce@sauce!bretonne brune@--- Bretonne (brune)}
\index{breton@breton(ne)!sauce brune@Sauce Bretonne (brune)}

中位の玉ねぎ2個をみじん切りにして、バターでブロンド色になるまで炒める。白ワイン2
\(\frac{1}{2}\)
dLを注ぎ、半量になるまで煮詰める。ここにソース・エスパニョル3
\(\frac{1}{2}\)
dLおよびトマトソース同量を加える。7〜8分間煮立ててから、刻んだパセリを加えて仕上げる。

\hypertarget{nota-sauce-bretonne}{%
\subparagraph{【原注】}\label{nota-sauce-bretonne}}

このソースは\protect\hyperlink{haricots-blancs-bretonne}{白いんげん豆のブルターニュ風}以外にはほとんど使われない。

\hypertarget{sauce-aux-cerises}{%
\subsubsection[ソース・スリーズ]{\texorpdfstring{ソース・スリーズ\footnote{cerise
  (スリーズ)
  さくらんぼの意。このレシピではグロゼイユ(すぐり)のジュレを用いるが、古くはさくらんぼを用いていたことからこの名称となったと言われている。}}{ソース・スリーズ}}\label{sauce-aux-cerises}}

\frsub{Sauce aux Cerises}

\index{そーす@ソース!すりーす@---・スリーズ}
\index{くろせいゆ@グロゼイユ!そーす@ソース!すりーす@---・スリーズ}
\index{さくらんほ@サクランボ!そーす@ソース!すりーす@---・スリーズ}
\index{sauce@sauce!cerise@--- aux Cerises}
\index{cerise@cerise!sauce@sauce aux ---s}

ポルト酒2 dLにイギリス風ミックススパイス\footnote{Mixed
  spiceのこと。Pudding
  spiceとも呼ばれる。シナモン、ナツメグ、オールスパイスの組み合わせが典型的。これにクローブ、生姜、コリアンダーシード、キャラウェイシードなどが加わっていることも多い。}1つまみと、すりおろしたオレンジの皮を大さじ
\(\frac{1}{2}\)
杯加えて\(\frac{2}{3}\)量になるまで煮詰める。\protect\hyperlink{gelee-de-groseilles-a}{グロゼイユのジュレ}
2 \(\frac{1}{2}\) dLを加え、仕上げにオレンジ果汁を加える。

\ldots{}\ldots{}大型ジビエの料理用だが、鴨のポワレやブレゼにも用いられる。

\hypertarget{sauce-aux-champignons}{%
\subsubsection[ソース・シャンピニョン]{\texorpdfstring{ソース・シャンピニョン\footnote{champignons
  キノコ全般を意味する語だが、単独で用いられる場合は champignons de
  Paris(シャンピニョンドパリ)いわゆるマッシュルームを指す。}}{ソース・シャンピニョン}}\label{sauce-aux-champignons}}

\frsub{Sauce aux Champignons}

\index{そーす@ソース!まつしゆるーむ@マッシュルーム ⇒ ---・シャンピニョン}
\index{まっしゅるーむ@マッシュルーム ⇒ シャンピニョン}
\index{しやんひによん@シャンピニョン!そーすふらうんけいはせい@ソース・---(ブラウン系)}
\index{sauce@sauce!champignons brune@--- aux Champignons (brune)}
\index{champignon@champignon!sauce brune@Sauce aux Champignons (brune)}

マッシュルームの茹で汁2 \(\frac{1}{2}\) dL を半量になるまで煮詰める。
\protect\hyperlink{sauce-demi-glace}{ソース・ドゥミグラス}8
dLを加えて数分間煮立てる。布で漉し、バター50
gを投入して味を調え、あらかじめ下茹でしておいた小さめのマッシュルームの笠100
gを加えて仕上げる。

\hypertarget{sauce-charcutiere}{%
\subsubsection[ソース・シャルキュティエール]{\texorpdfstring{ソース・シャルキュティエール\footnote{シャルキュトリ(豚肉加工業)風、の意。Charcutrieの語源はchar(肉)
  +cuite(調理された)+rie(業)。ハムやソーセージなどと定番の組合せであるマスタードを使う\protect\hyperlink{sauce-robert}{ソース・ロベール}と、おなじく定番のつけ合わせであるコルニション(小さいうちに収穫してヴィネガー漬けにしたきゅうり。専用品種がある)を使うことに由来。}}{ソース・シャルキュティエール}}\label{sauce-charcutiere}}

\frsub{Sauce Charcutière}

\index{そーす@ソース!しやるききゆとりふう@シャルキュトリ風 ⇒ ---・シャルキュティエール}
\index{しやるきゆとりふう@シャルキュトリ風!そーす@---ソース ⇒ ソース・シャルキュティエール}
\index{sauce@sauce!charcutiere@--- Charcutière}
\index{charcutier@charcutier(ère)!sauce@Sauce Charcutière}

提供直前に、\protect\hyperlink{sauce-robert}{ソース・ロベール}1 Lに細さ2
mm程度、短かめの千切り\footnote{julenne
  (ジュリエーヌ)1〜2mm程度の細さの千切りにした野菜などのこと。調理現場によって「ジュリエンヌ」「ジュリアン」(なぜか男性名)と呼ぶところもある。}
にしたコルニション\footnote{シャルキュトリ(豚肉加工業)風、の意。Charcutrieの語源はchar(肉)
  +cuite(調理された)+rie(業)。ハムやソーセージなどと定番の組合せであるマスタードを使う\protect\hyperlink{sauce-robert}{ソース・ロベール}と、おなじく定番のつけ合わせであるコルニション(小さいうちに収穫してヴィネガー漬けにしたきゅうり。専用品種がある)を使うことに由来。}
100
gを加える(\protect\hyperlink{sauce-robert}{ソース・ロベール}参照)。

\hypertarget{sauce-chasseur}{%
\subsubsection[ソース・シャスール]{\texorpdfstring{ソース・シャスール\footnote{狩人風、の意。古くは猟獣肉をすり潰したものを使った料理を指したという説もある。マッシュルームとエシャロット、白ワインを使うのが特徴であり、このソースを使った料理にも「シャスール」の名が付けられる。}}{ソース・シャスール}}\label{sauce-chasseur}}

\frsub{Sauce Chasseur}

\index{そーす@ソース!しやすーる@---・シャスール}
\index{しやすーる@シャスール!そーす@ソース・---}
\index{かりうとふう@狩人風 ⇒ ソース・シャスール}
\index{sauce@sauce!chasseur@--- Chasseur}
\index{chasseur@chasseur!sauce@Sauce ---}

生のマッシュルームを薄切りにしたもの150 gをバターで炒める。エシャロット
\footnote{échalote
  玉ねぎによく似ているが、小ぶりで水分が少なく、香味野菜としてよく用いられる。伝統的な品種は種子ではなく種球を植えて栽培する。なお、日本でしばしば「エシャレット」の名称で流通しているものはラッキョウの若どりであり、フランス料理で用いるエシャロットとはまったく異なる。}のみじん切り大さじ2
\(\frac{1}{2}\) 杯を加えてさらに軽く炒め、白ワイン3 dL
を注ぎ、半量になるまで煮詰める。\protect\hyperlink{sauce-tomate}{トマトソース}3
dL と\protect\hyperlink{sauce-demi-glace}{ソース・ドゥミグラス}2
dLを加える。数分間沸騰させたら、バター150 gと、セルフイユ\footnote{cerfeuil
  日本ではチャービルとも呼ばれるセリ科のハーブ。}とエストラゴン\footnote{estragon
  日本ではタラゴンとも呼ばれるヨモギ科のハーブ。フランス料理ではとても好まれる重要なハーブのひとつ。フレンチタラゴンとロシアンタラゴンの2種がある。料理に用いるのはフレンチタラゴンであり、この品種は種子ではなく株分けや挿し芽で殖やして栽培される。寒さには比較的強いが、日本の梅雨の湿度や夏の暑さには弱い。}をみじん切りにしたもの大さじ1
\(\frac{1}{2}\) 杯を加えて仕上げる。

\hypertarget{sauce-chasseur-procede-escoffier}{%
\subsubsection{ソース・シャスール(エスコフィエ流)}\label{sauce-chasseur-procede-escoffier}}

\frsub{Sauce Chasseur (Procédé Escoffier)}

\index{そーす@ソース!しやすーるえすこふいえ@---・シャスール(エスコフィエ流)}
\index{しやすーる@シャスール!そーすしゃすーるえすこふぃえ@ソース・---(エスコフィエ流)}
\index{かりうとふう@狩人風 ⇒ ソース・シャスール(エスコフィエ流)}
\index{sauce@sauce!chasseur escoffier@--- Chasseur (Procédé Escoffier)}
\index{chasseur@chasseur!sauce escoffier@Sauce --- (Procédé Escoffier)}

生のマッシュルームを薄切りにしたもの150
gを、バターと植物油で軽く色付くまで炒める。みじん切りにしたエシャロット大さじ1杯を加え、なるべくすぐに余分な油をきる。白ワイン2
dL とコニャック約50 mL
を注ぎ、半量になるまで煮詰める。\protect\hyperlink{sauce-demi-glace}{ソース・ドゥミグラス}4
dLと\protect\hyperlink{sauce-tomate}{トマトソース}2
dL、\protect\hyperlink{glace-de-viande}{グラスドヴィアンド}大さじ
\(\frac{1}{2}\) 杯を加える。

5分間沸騰させたら、仕上げにパセリのみじん切り少々を加える。

\hypertarget{sauce-chaud-froid-brune}{%
\subsubsection[茶色いソース・ショフロワ]{\texorpdfstring{茶色いソース・ショフロワ\footnote{choud-froid(ショフロワ)はchaudショ「熱い、温かい」とfroidフロワ「冷たい」の合成語で、火を通した肉や魚を冷まし、表面にこのソース・ショフロワを覆うように塗り付け、さらにジュレを覆いかけた料理。料理の発祥については諸説あり、なかでもルイ15世に仕えていた料理長ショフロワChaufroixが考案したという説を支持してなのか、英語ではこの料理をChaufroixと綴ることも多い。Chaud-froidの表記は19世紀後半には文献に見られる。なお、複数形はchauds-froidsと綴る。トリュフの薄切りやエストラゴンなどのハーブその他で表面に華麗な装飾を施すことが19世紀には盛んに行なわれていた。現代でも装飾に凝った仕立てにするケースは多い。}}{茶色いソース・ショフロワ}}\label{sauce-chaud-froid-brune}}

\frsub{Sauce Chaud-froid brune}

\index{そーす@ソース!しよふろわちやいろ@茶色い---・ショフロワ}
\index{しよふろわ@ショフロワ!そーすふらうんけい@茶色いソース・---}
\index{sauce@sauce!chaud-froid brune@--- Chaud-froid brune}
\index{chaud-froid@chaud-froid!sauce brune@Sauce --- brune}

(仕上がり1 L分)

\protect\hyperlink{sauce-demi-glace}{ソース・ドゥミグラス}\(\frac{3}{4}\)
Lとトリュフエッセンス1 dL、ジュレ6〜7 dLを用意する。

ソース・ドゥミグラスにトリュフエッセンスを加えて、強火で煮詰めるが、この時に鍋から離れないこと。煮詰めながらジュレを少量ずつ加えていく。最終的に\(\frac{2}{3}\)量程度まで煮詰める。

味見をして、ソースがショフロワに使うのに丁度いい濃さになっているか確認すること。

マデイラ酒またはポルト酒 \(\frac{1}{2}\)
dLを加える。布で漉し、ショフロワの主素材の表面に塗り付けるのに丁度いい固さになるまで、丁寧にゆっくり混ぜながら冷ます。

\hypertarget{sauce-chaud-froid-brune-pour-canards}{%
\subsubsection{茶色いソース・ショフロワ(鴨用)}\label{sauce-chaud-froid-brune-pour-canards}}

\frsub{Sauce Chaud-froid brune pour Canards}

\index{そーす@ソース!しよふろわちやいと@茶色い---・ショフロワ(鴨用)}
\index{しよふろわ@ショフロワ!ちやいろいそーすかもよう@茶色いソース・---(鴨用)}
\index{sauce@sauce!chaud-froid brune pour canards@--- Chaud-froid brune pour Canards}
\index{chaud-froid@chaud-froid!sauce brune pour Canards@Sauce --- brune pour Canards}

作り方は上記、\protect\hyperlink{sauce-chaud-froid-brune}{茶色いソース・ショフロワ}と同様だが、トリュフエッセンスではなく、鴨のガラでとったフュメ1
\(\frac{1}{2}\)
dLを用いること。また、上記のレシピよりややしっかり煮詰めること。

ソースを布で漉したら、オレンジ3個分の搾り汁、とオレンジの皮をごく薄く剥いて細かい千切りにしたもの\footnote{zeste
  ゼスト。オレンジやレモンの皮の表面を器具を用いてすりおろすか、ナイフでごく薄く表皮を向き、細かい千切りにしたもの。ここでは後者を使う指定になっている。}大さじ2杯を加える。オレンジの皮の千切りはしっかりと下茹でしてよく水気をきっておくこと。

\hypertarget{sauce-chaud-froid-brune-pour-gibier}{%
\subsubsection{茶色いソース・ショフロワ(ジビエ用)}\label{sauce-chaud-froid-brune-pour-gibier}}

\frsub{Sauce Chaud-froid brune pour Gibier}

\index{そーす@ソース!しよふろわちやいろじびえよう@茶色い---・ショフロワ(ジビエ用)}
\index{しよふろわ@ショフロワ!そーすしよふろわちやいろじびえよう@茶色いソース・---(ジビエ用)}
\index{sauce@sauce!chaud-froid brune pour Gibier@--- Chaud-froid brune pour Gibier}
\index{chaud-froid@chaud-froid!sauce brune pour Gibier@Sauce --- brune pour Gibier}

作り方は上記\protect\hyperlink{sauce-chaud-froid-brune}{標準的なソース・ショフロワ}と同じだが、トリュフエッセンスではなく、ショフロワとして供するジビエのガラでとったフュメ\footnote{\protect\hyperlink{fonds-de-gibier}{ジビエのフォン}参照。}2
dLを用いること。

\hypertarget{sauce-chaud-froid-tomatee}{%
\subsubsection{トマト入りソース・ショフロワ}\label{sauce-chaud-froid-tomatee}}

\frsub{Sauce Chaud-froid tomatée}

\index{そーす@ソース!しよふろわとまといり@トマト入り---・ショフロワ}
\index{しよふろわ@ショフロワ!そーすちやいろとまといり@トマト入りソース・---}
\index{sauce@sauce!chaud-froid tomatée@--- Chaud-froid tomatée}
\index{chaud-froid@chaud-froid!sauce tomatée@Sauce --- tomatée}

良質で、既によく煮詰めてあるトマトピュレ1 Lを、さらに煮詰めながら7〜8
dLのジュレを少しずつ加えていく。全体量が1 L以下になるまで煮詰めること。

布で漉し、使いやすい固さになるまで、ゆっくり混ぜながら冷ます。

\hypertarget{sauce-chevreuil}{%
\subsubsection{ソース・シュヴルイユ}\label{sauce-chevreuil}}

\frsub{Sauce Chevreuil}

\index{しゆうるいゆ@シュヴルイユ!そーす@ソース・---}
\index{そーす@ソース!しゅうるいゆ@---・シュヴルイユ}
\index{のろしか@ノロ鹿 ⇒ シュヴルイユ!そーす@ソース!しゆうるいゆ@ソース・シュヴルイユ}
\index{sauce@sauce!chevreuil@--- Chevreuil}
\index{chevreuil@chevreuil!sauce@Sauce ---}

\protect\hyperlink{sauce-poivrade}{標準的なソース・ポワヴラード}と同様に作るが、

\begin{enumerate}
\def\labelenumi{\arabic{enumi}.}
\item
  マリネした牛・羊肉の料理に添える場合\footnote{chevreuil
    シュヴルイユはノロ鹿のことだが、このように事前にマリネした牛・羊肉を用いた料理にもこのソースを使い「シュヴルイユ(風)(仕立て)」と\ruby{謳}{うた}う。1806年刊ヴィアール『帝国料理の本』においてノロ鹿のフィレは香辛料を加えたワインヴィネガーで48時間マリネしてから調理すると書かれている。オド『女性料理人のための本』では、確認出来た1834年の第4版から1900年の第78版に至るまで、ノロ鹿の項において「一週間もヴィネガーたっぷりの漬け汁でマリネするのはやりすぎだが、強い味が好みなら1〜4日間」香辛料と赤ワインあるいはヴィネガーでマリネするといい、と説明されている。つまり、ノロ鹿とは必ずマリネしてから調理するものという一種のコンセンサスがあったために、マリネした牛・羊肉の料理にも「シュヴルイユ(風)」の名称が謳われるようになったと考えられる。}は、ハム入りの\protect\hyperlink{mirepoix}{ミルポワ}を加える。
\item
  ジビエ料理に添える場合は、そのジビエの端肉を加える。
\end{enumerate}

素材をヘラなどで強く押し付けるようにして漉す\footnote{シノワ(\protect\hyperlink{sauce-espagnole}{ソース・エスパニョル}訳注参照)などを用いる。}。良質の赤ワイン
1 \(\frac{1}{2}\)
dLをスプーン1杯ずつ加えながら煮て、浮き上がってくる不純物を丁寧に取り除いていく\footnote{dépouiller
  デプイエ ≒ écumer エキュメ。}。

最後に、カイエンヌごく少量と砂糖1つまみを加えて味を\ruby{調}{とと
の}え、布で漉す。

\hypertarget{sauce-colbert}{%
\subsubsection[ソース・コルベール]{\texorpdfstring{ソース・コルベール\footnote{17世紀の政治家、ジャン・バティスト・コルベール(1619〜1683)の名を冠したもの。}}{ソース・コルベール}}\label{sauce-colbert}}

\frsub{Sauce Colbert}

\index{そーす@ソース!こるへーる@---・コルベール}
\index{こるへーる@コルベール!そーす@ソース・---}
\index{sauce@sauce!colbert@--- Colbert}
\index{colbert@Colbert!sauce@Sauce ---}

\protect\hyperlink{beurre-maitre-d-hotel}{メートルドテルバター}に\protect\hyperlink{glace-de-viande}{グラスドヴィアンド}を加えたもののことだが、正しくは「\protect\hyperlink{beurre-colbert}{ブール・コルベール}」と呼ぶべきものだ\footnote{具体的なレシピは\protect\hyperlink{beurre-colbert}{ブール・コルベール}参照のこと。}。

また、ブール・コルベールと\protect\hyperlink{sauce-chateaubriand}{ソース・シャトーブリアン}との違いを明確にさせようとして、メートルドテルバターにエストラゴンを加える者もいる。だが、必ずそうすべきということではない。実際、ブール・コルベールとソース・シャトーブリアンは明らかに違うものだからだ。ソース・シャトーブリアンは軽く仕上げたグラスドヴィアントにバターとパセリのみじん切りを加えたものである。一方、ブール・コルベールあるいはソース・コルベールと呼ばれているものはあくまでもバターが主であって、グラスドヴィアンドは補助的なものに過ぎない。

\hypertarget{sauce-diable}{%
\subsubsection[ソース・ディアーブル]{\texorpdfstring{ソース・ディアーブル\footnote{diable
  (ディアーブル)悪魔の意。}}{ソース・ディアーブル}}\label{sauce-diable}}

\frsub{Sauce Diable}

\index{そーす@ソース!ていあーふる@---・ディアーブル}
\index{ていあーふる@ディアーブル!そーす@ソース・---}
\index{あくま@悪魔 ⇒ ディアーブル!そーす@ソース!そーすていあーふる@ソース・ディアーブル}
\index{sauce@sauce!diable@--- Diable}
\index{diable@diable!sauce@Sauce ---}

このソースはごく少量ずつ作るのが一般的だが、ここではそれを守らずに、仕上り2
\(\frac{1}{2}\) dLとして説明する

白ワイン3
dLにエシャロット3個分のみじん切りを加え、\(\frac{1}{3}\)量以下になるまで煮詰める。

\protect\hyperlink{sauce-demi-glace}{ソース・ドゥミグラス}2
dLを加えて数分間煮立たせ、仕上げにカイエンヌの粉末をたっぷり効かせる\footnote{「たっぷり」という表現に惑わされないよう注意。}。

\ldots{}\ldots{}鶏と鳩のグリルに合わせる。

\hypertarget{nota-sauce-diable}{%
\subparagraph{【原注】}\label{nota-sauce-diable}}

白ワインではなくヴィネガーを煮詰め、仕上げにハーブを加えて作る調理現場もあるが、著者としては本書で示しているの作り方がいいと思う。

\hypertarget{sauce-diable-escoffier}{%
\subsubsection{ソース・ディアーブル・エスコフィエ}\label{sauce-diable-escoffier}}

\frsub{Sauce Diable Escoffier}

\index{そーす@ソース!ていあーふるえすこふいえ@---・ディアーブル・エスコフィエ}
\index{ていあーふる@ディアーブル!そーす@ソース!えすこふいえ@ソース・---・エスコフィエ}
\index{あくま@悪魔 ⇒ ディアーブル!そーす@ソース!エスコフイエ@ソース・ディアーブル・エスコフィエ}
\index{sauce@sauce!diable escoffier@--- Diable Escoffier}
\index{diable@diable!sauce escoffier@Sauce --- Escoffier}

このソースは完成品が市販\footnote{現在は市販されていないと思われる。フランスにおいては未確認だが、
  1980年代までアメリカ合衆国ではナビスコがソース・ロベール・エスコフィエとともに瓶詰めを生産、販売していた。初版ではこれら2つの製品への言及がなく、第二版で追加されたことから、1903年〜1907年の間に製品化された可能性もある。また、第二版(1907年)と同年の英訳版、第三版(1912年)にはソース・スリーズ・エスコフィエの記述が見られるが、これは第四版で削除されており、生産中止になったと思われる。エスコフィエ・ブランドの既製品ソースはさらに他にもあったようだが詳細は不明。なお、エスコフィエは1922年頃、ジュリユス・マジがブイヨンキューブ(日本では「マギーブイヨン」の商品名)を開発する際にも協力した。}されている。同量の柔くしたバターを混ぜ合わせるだけでいい。

\hypertarget{sauce-diane}{%
\subsubsection[ソース・ディアーヌ]{\texorpdfstring{ソース・ディアーヌ\footnote{ローマ神話の女神ディアーナのこと。ギリシア神話のアルテミスに相当し、狩猟、貞潔の女神。また月の女神ルーナ(セレーネー)と同一視された。ここでは大型ジビエ料理用のソースであるから、狩猟の女神という意味合いが強い。}}{ソース・ディアーヌ}}\label{sauce-diane}}

\frsub{Sauce Diane}

\index{そーす@ソース!ていあーぬ@---・ディアーヌ}
\index{ていあーぬ@ディアーヌ!そーす@ソース・---}
\index{sauce@sauce!diane@--- Diane} \index{diane@Diane!sauce@Sauce ---}

不純物を充分に取り除き、コクと風味ゆたかな\protect\hyperlink{sauce-poivrade}{ソース・ポワヴラード}5
dLを用意する。提供直前に、泡立てた生クリーム4 dL (生クリーム2
dLを泡立てて倍量にする)と、小さな三日月の形にしたトリュフのスライスと固茹で卵の白身を加える。

\ldots{}\ldots{}大型ジビエの骨付き背肉および、その中心部を円筒形に切り出したもの
\footnote{noisette ノワゼット。}、フィレ料理用。

\hypertarget{sauce-duxelles}{%
\subsubsection[ソース・デュクセル]{\texorpdfstring{ソース・デュクセル\footnote{\protect\hyperlink{duxelles-seche}{デュクセル・セッシュ}(第2章ガルニチュール)を用いることからこの名称が用いられている。}}{ソース・デュクセル}}\label{sauce-duxelles}}

\frsub{Sauce Duxelles}

\index{そーす@ソース!てゆくせる@---・デュクセル}
\index{てゆくせる@デュクセル!そーす@ソース・---}
\index{sauce@sauce!duxelles@--- Duxelles}
\index{duxelles@duxelles!sauce@Sauce ---}

白ワイン2 dLとマッシュルームの茹で汁2
dLにエシャロットのみじん切り大さじ2
杯を加えて、\(\frac{1}{3}\)量まで煮詰める。\protect\hyperlink{sauce-demi-glace}{ソース・ドゥミグラス}
\(\frac{1}{2}\) Lとトマトピュレ1 \(\frac{1}{2}\)
dL、\protect\hyperlink{duxelles-seche}{デュクセル・セッシュ}大さじ4杯を加える。5分間煮立たせ、パセリのみじん切り大さじ
\(\frac{1}{2}\) を加える。

\ldots{}\ldots{}グラタンの他、いろいろな料理に用いられる。

\hypertarget{nota-sauce-duxelles}{%
\subparagraph{【原注】}\label{nota-sauce-duxelles}}

ソース・デュクセルはイタリア風ソースと混同されることが多いが、ソース・デュクセルにはハムも、赤く漬けた舌肉も入れないので、まったく別のものだ。

\hypertarget{sauce-estragon}{%
\subsubsection[ソース・エストラゴン]{\texorpdfstring{ソース・エストラゴン\footnote{ヨモギ科のハーブ。\protect\hyperlink{sauce-chasseur}{ソース・シャスール}訳注参照。}}{ソース・エストラゴン}}\label{sauce-estragon}}

\frsub{Sauce Estragon}

\index{そーす@ソース!えすとらこんちゃいろ@---・エストラゴン(ブラウン系)}
\index{えすとらこん@エストラゴン!そーすふらうんけい@ソース・---(ブラウン系)}
\index{sauce@sauce!estragon brune@--- Estragon (brune)}
\index{estragon@estragon!sauce brune@Sauce --- (brune)}

(仕上がり2 \(\frac{1}{2}\) dL分)

白ワイン2 dLを沸かし、エストラゴンの枝20
gを投入する。蓋をして10分間、煎じる\footnote{infuser(アンフュゼ)煮出す、煎じる、の意。なおハーブティはこの派生語infusion(アンフュジオン)と呼ぶ。}。2
\(\frac{1}{2}\)
dLの\protect\hyperlink{sauce-demi-glace}{ソース・ドゥミグラス}または、\protect\hyperlink{jus-de-veau-lie}{とろみを付けた仔牛のジュ}を加え、約\(\frac{2}{3}\)
量になるまで煮詰める。布で漉し、みじん切りにしたエストラゴン小さじ1杯を加えて仕上げる。

\ldots{}\ldots{}仔牛や仔羊の背肉の中心を円筒形に切り出した料理や家禽料理用。

\hypertarget{sauce-financiere}{%
\subsubsection[ソース・フィナンシエール]{\texorpdfstring{ソース・フィナンシエール\footnote{Financier徴税官(財務官)風の意。フランス革命以前の徴税官は、王に代わって徴税を行なう大貴族が就く役職であり、膨大な利権によりきわめて裕福であったという。このソースと組み合わせる\protect\hyperlink{garniture-a-la-financiere}{ガルニチュール・フィナンシエール}が、雄鶏のとさかと睾丸、仔羊の胸腺肉、トリュフなどの比較的入手困難あるいは高級とされる食材で構成されていることが名称の由来と思われる。ブリヤ=サヴァランは『美味礼讃』(味覚の生理学)において、徴税官たちは旬のはしりの食材を真っ先に食べられる、いわば特権階級だと述べている。なお、カレーム『19世紀フランス料理』においては、ソースとガルニチュールを分離せず、「ラグー・アラ・フィナンシエール」として採りあげられているが、全ての素材を別々に加熱調理してソースと合わせるものであり、いわゆる「煮込み」とは呼びがたいものとなっている。フランス料理の影響が比較的強かった北イタリアにこの原型に近いと思われるラグー「ピエモンテ風フィナンツィエラ」がある。鶏のとさか、肉垂、睾丸、鶏レバーおよび仔牛の胸腺肉などを煮込んだものだが、レシピを読む限りにおいては比較的庶民的あるいは農民的料理に変化したものと思われる
  (cf.~Anna Gosetti della Salda, \emph{Le Ricette Regionali Italiane},
  Milano, Solares, 1967,
  p.57.)。ちなみに焼き菓子のフィナンシエfinancierも同語源だが、何故その名称になったかは不明。}}{ソース・フィナンシエール}}\label{sauce-financiere}}

\frsub{Sauce Financière}

\index{そーす@ソース!ふいなんしえーる@---・フィナンシエール}
\index{ふいなんしえ@フィナンシエ/フィナンシエール!そーす@ソース・フィナンシエール}
\index{ちょうせいかんふう@徴税官風 ⇒ フィナンシエール!そーすふぃなんしえーる@ソース・フィナンシエール}
\index{sauce@sauce!financiere@--- Financière}
\index{financier@financier(ère)!sauce@Sauce Financière}

\protect\hyperlink{sauce-madere}{ソース・マデール}1 \(\frac{1}{4}\)
Lを\(\frac{3}{4}\)量以下になるまで煮詰め、火から外してトリュフエッセンス1
dLを加える。布で漉して仕上げる。

\ldots{}\ldots{}\protect\hyperlink{garniture-a-la-financiere}{ガルニチュール・フィナンシエール}用だが、その他の肉料理にも用いられる。

\hypertarget{sauce-aux-fines-herbes}{%
\subsubsection[香草ソース]{\texorpdfstring{香草ソース\footnote{料理名では、いわゆる「ハーブ」についてかつてfines
  herbesの表現が多く用いられた。だが、こんにちでは特定のハーブ名をソースや料理名に添えて言うことが多い。例えばCôtelette
  de veau au
  thymコトレットドヴォオタン(仔牛の骨付き背肉、タイム風味)、やFilet de
  bar poêlé, compote de tomate au basilicフィレドバールポワレ
  コンポットートドトマトバジリック(スズキのフィレとトマトのコンポート、バジル風味)など。また、栽培レベルで「香草、ハーブ」の総称としては
  herbes aromatiques
  (エルブザロマティック)、あるいはたんにaromatiques(アロマティック)が一般的。}}{香草ソース}}\label{sauce-aux-fines-herbes}}

\frsub{Sauce aux Fines Herbes}

\index{そーす@ソース!こうそうふらうんけい@香草---(ブラウン系)}
\index{こうそう@香草!そーすふらうんけい@---ソース(ブラウン系)}
\index{はーぶ@ハーブ ⇒ 香草!こうそうそーすふらうんけい@香草ソース(ブラウン系)}
\index{sauce@sauce!fines herbes@--- aux Fines Herbes}
\index{fines herbes@fines herbes!sauce@Sauce aux ---}

白ワイン3
dLを沸かし、パセリの葉、セルフイユ、エストラゴン、シブレットを各1つまみ強、投入する。約20分間煎じる。布で漉し、\protect\hyperlink{sauce-demi-glace}{ソース・ドゥミグラス}または\protect\hyperlink{jus-de-veau-lie}{とろみを付けた仔牛のジュ}6
dLを加える。仕上げに、煎じるのに使ったのと同じ香草を細かく刻んだもの計、大さじ2
\(\frac{1}{2}\) 杯とレモンの搾り汁少々を加える。

\hypertarget{nota-sauce-aux-fines-herbes}{%
\subparagraph{【原注】}\label{nota-sauce-aux-fines-herbes}}

古典料理ではこの「香草ソース」と\protect\hyperlink{sauce-duxelles}{ソース・デュクセル}が混同されることもあったが、こんにちではまったく違うものとして扱われている。

\hypertarget{sauce-genevoise}{%
\subsubsection{ジュネーヴ風ソース}\label{sauce-genevoise}}

\frsub{Sauce Genevoise}

\index{そーす@ソース!しゆねーうふう@ジュネーヴ風---}
\index{しゆねーうふう@ジュネーヴ風!そーす@---ソース}
\index{sauce@sauce!genevoise@--- Genevoise}
\index{genevois@genevois(e)!sauce@Sauce Genevoise}

鍋にバターを熱し、細かく刻んだミルポワを色付かないよう強火でさっと炒める。ミルポワの材料は、にんじん100
g、玉ねぎ80 g、タイムとローリエ少々、パセリの枝20 g。そこにサーモンの頭1
kgと粗く砕いたこしょう1つまみを入れ、蓋をして弱火で15分程蒸し煮する。

鍋に残ったバターを捨て、赤ワイン1
Lを注ぐ。半量になるまで煮詰める。そこに\protect\hyperlink{sauce-espagnole-maigre}{魚料理用ソース・エスパニョル}
\(\frac{1}{2}\)
Lを加える。弱火で1時間煮込む。漉し器を使い、材料を押しつけながら漉す。しばらく休ませてから、表面に浮いた油脂を取り除く\footnote{dégraisser
  デグレセ。レードルなどを用いて浮いてきた余計な油脂を取り除く作業。}

さらに赤ワイン \(\frac{1}{2}\) Lと、魚のフュメ \(\frac{1}{2}\)
Lを加える。ソースの表面に浮いてくる不純物を徹底的に取り除き\footnote{dépouiller
  デプイエ ≒ écumer エキュメ。}、丁度いい濃さになるまで煮詰める。

これを布で漉し、静かに混ぜながら、アンチョヴィのエッセンス大さじ1杯とバター150
gを加えて仕上げる。

\ldots{}\ldots{}サーモン、鱒料理用。

\hypertarget{nota-sauce-genevoise}{%
\subparagraph{【原注】}\label{nota-sauce-genevoise}}

このソースはもともとカレームが「ジェノヴァ風」\footnote{Sauce à la
  génoise au vin de Bordeaux
  ボルドー産ワインを用いたジェノヴァ風ソース(『19世紀フランス料理』第3巻、80頁)。本書のこのレシピと同様に魚料理用ソースだ。ボルドーの赤ワインにみじん切りにして下茹でしたマッシュルーム、トリュフ、エシャロットを加えてオールスパイスとこしょう少々を入れ、適度に煮詰める。ソース・エスパニョルと赤ワインを加え、湯煎にかけておく。提供直前にバター少量を加えて仕上げる、というもの。本書においてこのソースを「原型」とするのには疑問が残るところだろう。}と名付けたものだが、その後ルキュレ、グフェ\footnote{グフェ『料理の本』(1867年)の420ページにあるジュネーヴ風ソースは、薄切りにした玉ねぎ、エシャロット、粗挽きこしょう、にんにく、バターを鍋に入れて色付くまで炒め、そこにブルゴーニュ産赤ワインを注ぐ。弱火で玉ねぎに火が通るまで煮る。ソース・エスパニョルと仔牛のブロンドのジュを加えて煮詰め、布で漉す。提供直前にマデイラ酒の風味を加えて茹でたトリュフのみじん切りとアンチョビバターを加える、というもの。赤ワインと玉ねぎ、仕上げにアンチョビを加える点は共通しているが、グフェのが肉料理用であるのに対して、本書のこのソースは明らかに魚料理用であり、まったく同じソースと呼べるとは言い難い。}が立て続けに「ジュネーヴ風」の名称を用いた。だが、ジュネーヴは赤ワインの産地ではないから理屈としてはおかしい\footnote{料理名に冠された地名は、由来が明確にあるものがある一方で、まったく意味不明か、あるいはいい加減な思い付きで付けられたのではないかとさえ思われるものも少なくない。(à
  la) russe「ロシア風」や (à la)
  moscovite「モスクワ風」などはロシア料理起源か、あるいは18世紀末〜
  19世紀前半にかけてロシア帝国の宮廷や貴族がこぞってフランスから料理人を招聘し、帰国した彼らが創案した料理などはある程度しっかりとした由来がわかるものも多い。一方で、(à
  l')espagnole「スペイン風」(à l')italienne「イタリア風」(à la)
  romaine「ローマ風」(à la grecque) 「ギリシア風」(à
  l')allemande「ドイツ風」(à
  l')hollandaise「オランダ風」などは由来の不明なケースが非常に多い。\protect\hyperlink{sauce-espagnole}{ソース・エスパニョル}などはその典型例とも言うべきものだろう。この原注では由来に非常にこだわっているが、そもそもカレームのレシピは上述のように「ボルドー産ワインを用いたジェノバ風ソース」であるから、赤ワインの産地かどうかということは実はさしたる問題にはならない。重要なのは後半の、赤ワインを用いることがこのソースのポイントということ。}。

間違っているとはいえ、ジュネーヴ風という名称で定着してしまっているので、本書でもそのままにしている。だが、ジュネーヴ風であれジェノヴァ風であれ、カレーム、ルキュレ、デュボワ、グフェはいずれもこのソースに赤ワインを用いるよう指示している。つまり赤ワインを用いることがこのソースのポイント。

\hypertarget{sauce-godard}{%
\subsubsection[ソース・ゴダール]{\texorpdfstring{ソース・ゴダール\footnote{ガルニチュール・ゴダールの構成要素がガルニチュール・フィナンシエールとよく似ている点などから、おそらくは18世紀の徴税官(つまりフィナンシエ)であり作家としても活動したクロード・ゴダール・ドクール
  Claude Godard
  d'Aucour(1716〜1795)の名を冠したものと考えられる。なお、底本とした現行版(第四版)では最後がdではなくtとなっているが、初版から第三版にいたるまでdとなっており、現行版は明らかな誤植。}}{ソース・ゴダール}}\label{sauce-godard}}

\frsub{Sauce Godard}

\index{そーす@ソース!こたーる@---・ゴダール}
\index{こたーる@ゴダール!そーす@ソース・---}
\index{sauce@sauce!godard@--- Godard}
\index{godard@Godard!sauce@Sauce ---}

シャンパーニュまたは辛口の白ワイン4
dLにハム入りの細かく刻んだ\protect\hyperlink{mirepoix}{ミルポワ}、\protect\hyperlink{sauce-demi-glace}{ソース・ドゥミグラス}1
Lとマッシュルームのエッセンス2
dLを加える。弱火に10分かけ、シノワ\footnote{\protect\hyperlink{sauce-espagnole}{ソース・エスパニョル}訳注参照。}で漉す。

\(\frac{2}{3}\)量になるまで煮詰め、布で漉す。

\ldots{}\ldots{}\protect\hyperlink{garniture-godard}{ガルニチュール ゴタール}用。

\hypertarget{sauce-grand-veneur}{%
\subsubsection[ソース・グランヴヌール]{\texorpdfstring{ソース・グランヴヌール\footnote{王家や貴族に仕える狩猟長のことをgrand-veneur(グランヴヌール)と呼ぶ。}}{ソース・グランヴヌール}}\label{sauce-grand-veneur}}

\frsub{Sauce Grand-Veneur}

\index{そーす@ソース!くらんうぬーる@---・グランヴヌール}
\index{くらんうぬーる@グランヴヌール!そーす@ソース・---}
\index{sauce@sauce!grand-veneur@--- Grand-Veneur}
\index{grand-veneur@grand-veneur!sauce@Sauce ---}

\protect\hyperlink{fonds-de-gibier}{大型ジビエのフュメ}で澄んだ色合いに作った\protect\hyperlink{sauce-poivrade}{ソース・ポワヴラード}に、ソース
1 Lあたり野うさぎの血1 dLをマリネ液1 dLで薄めたものを加える。

火をごく弱くして、血が沸騰しないよう気をつけながら数分間煮る。布で漉す。

\hypertarget{sauce-grand-veneur-procede-escoffier}{%
\subsubsection{ソース・グランヴヌール(エスコフィエ流)}\label{sauce-grand-veneur-procede-escoffier}}

\frsub{Sauce Grand-Veneur (Procédé Escoffier)}

\index{そーす@ソース!くらんうぬーるえすこふいえ@---・グランヴヌール(エスコフィエ流)}
\index{くらんうぬーる@グランヴヌール!そーすえすこふいえ@ソース・---(エスコフィエ流)}
\index{sauce@sauce!grand-veneur escoffier@--- Grand-Veneur (Procédé Escoffier)}
\index{grand-veneur@grand-veneur!sauce escoffier@Sauce --- (Procédé Escoffier)}

軽く仕上げた\protect\hyperlink{sauce-poivrade}{ソース・ポワヴラード}1
Lあたり\protect\hyperlink{gelee-de-groseilles-a}{グロゼイユのジュレ}大さじ2杯と生クリーム2
\(\frac{1}{2}\) dLを加える。

\ldots{}\ldots{}上記2つのソースは鹿、猪などの大きな塊肉の料理に用いる。

\hypertarget{sauce-gratin}{%
\subsubsection[ソース・グラタン]{\texorpdfstring{ソース・グラタン\footnote{魚のグラタン用ソースだが、グラタンの技術的ポイントについては\protect\hyperlink{gratins}{第
  7章「肉料理」のグタランの項目}参照。}}{ソース・グラタン}}\label{sauce-gratin}}

\frsub{Sauce Gratin}

\index{そーす@ソース!くらたん@---・グラタン}
\index{くらたん@グラタン!そーす@ソース・---}
\index{sauce@sauce!gratin@--- Gratin}
\index{gratin@gratin!sauce@Sauce ---}

白ワインと、このソースを合わせる魚のアラなどでとった\protect\hyperlink{fumet-de-poisson}{魚のフュメ}各3
dLにエシャロットのみじん切り大さじ1 \(\frac{1}{2}\)
杯を加え、半量以下になるまで煮詰める。

\protect\hyperlink{duxelles-seche}{デュクセル・セッシュ}大さじ3杯と、\protect\hyperlink{sauce-espagnole-maigre}{魚料理用ソース・エスパニョル}または\protect\hyperlink{sauce-demi-glace}{ソース・ドゥミグラス}5
dLを加える。5〜6分間煮立たせる。提供直前に、パセリのみじん切り大さじ
\(\frac{1}{2}\) を加えて仕上げる。

\ldots{}\ldots{}舌びらめ、メルラン\footnote{タラの近縁種。}、バルビュ\footnote{鰈の近縁種。この場合のフィレはいわゆる「五枚おろし」にしたもの。}のフィレなどのグラタン用。

\hypertarget{sauce-hachee}{%
\subsubsection[ソース・アシェ]{\texorpdfstring{ソース・アシェ\footnote{細かく刻んだもの、の意。}}{ソース・アシェ}}\label{sauce-hachee}}

\frsub{Sauce Hachée}

\index{そーす@ソース!あしえ@---・アシェ}
\index{あしえ@アシェ!そーす@ソース・---}
\index{sauce@sauce!hachee@--- Hachée}
\index{hache@haché(e)!sauce@Sauce Hachée}

玉ねぎの細かいみじん切り100 gと、エシャロットの細かいみじん切り大さじ 1
\(\frac{1}{2}\) 杯をバターで色付かないよう炒める。ヴィネガー3
dLを注ぎ、半量まで煮詰める。\protect\hyperlink{sauce-espagnole}{ソース・エスパニョル}4
dLと\protect\hyperlink{sauce-tomate}{トマトソース}1 \(\frac{1}{2}\) dL
を加える。5〜6分煮立たせる。

ハムの脂身のない部分を細かく刻んだもの大さじ1 \(\frac{1}{2}\)
杯と小ぶりのケイパー大さじ1 \(\frac{1}{2}\)
杯、{[}デュクセル・セッシュ{]}大さじ1 \(\frac{1}{2}\)
杯、パセリのみじん切り大さじ \(\frac{1}{2}\) 杯を加えて仕上げる

\ldots{}\ldots{}このソースは\protect\hyperlink{sauce-piquante}{ソース・ピカント}と等価のものと考えていい。用途も同じ。

\hypertarget{sauce-hachee-maigre}{%
\subsubsection{魚料理用ソース・アシェ}\label{sauce-hachee-maigre}}

\frsub{Sauce Hachée maigre}

\index{そーす@ソース!あしえ@魚料理用---・アシェ}
\index{あしえさかな@アシェ(魚料理用)!そーす@ソース・---}
\index{sauce@sauce!hachee maigre@--- Hachée maigre}
\index{hache@haché(e)!sauce maigre@Sauce Hachée maigre}

上記と同様に、玉ねぎとエシャロットを色付かないようバターで炒め、ヴィネガーを注いで煮詰める。

魚の\protect\hyperlink{courts-bouillons-de-poisson}{クールブイヨン}5
dLを注ぎ、\protect\hyperlink{roux-brun}{茶色いルー}45
gまたはブールマニエ50 gでとろみを付ける。弱火で8〜10分間煮込む。

提供直前に、細かく刻んだハーブミックス大さじ1杯と\protect\hyperlink{duxelles-seche}{デュクセル・セッシュ}大さじ1
\(\frac{1}{2}\) 杯、小粒のケイパー大さじ1 \(\frac{1}{2}\)
杯、アンチョヴィソース大さじ \(\frac{1}{2}\) 杯とバター60
g、または80〜100 gのアンチョヴィバターを加えて仕上げる。

\ldots{}\ldots{}エイのような、あまり高級ではない茹でた魚\footnote{原文
  poissons
  bouillis。このフランス語の表現だと加熱する際に沸騰させているニュアンスがあるが、本書の「魚料理」の章において、魚を塩を加えて茹でる、あるいはクールブイヨンで煮る際に、沸騰しない程度の温度で加熱(ポシェ
  pocher)すべきと強調されている。この表現は初版からのものであり、恐らくはこのソースの部分を実際に執筆した者と、魚料理の説明部分を執筆した者が異なることによるわかりにくさ、という可能性も排除出来ない。いずれにしても、このソースの場合は、合わせる魚をクールブイヨンで沸騰しない程度の温度で加熱(ポシェ)し、そのクールブイヨンの一部をソースに加えていることから、単に「茹でた魚」と言っても、本書における魚の加熱方法に則った調理をすべきと解されよう。}用。

\hypertarget{sauce-hussarde}{%
\subsubsection[ソース・ユサルド]{\texorpdfstring{ソース・ユサルド\footnote{もとはハンガリーで農家20戸につき1人の割合で招集された騎兵
  hussard
  を指す。この語は16世紀まで遡ることが出来るが、のちに「乱暴者」といったニュアンスでも使われるようになった。à
  la hussarde
  は「乱暴に、粗野に」の意味でも用いられるが、料理においてはレフォールを使ったものに名付けられることが多い。}}{ソース・ユサルド}}\label{sauce-hussarde}}

\frsub{Sauce Hussarde}

\index{そーす@ソース!ゆさると@---・ユサルド}
\index{ゆさると@ユサルド!そーす@ソース・---}
\index{sauce@sauce!hussarde@--- Hussarde}
\index{hussard@Hussard(e)!sauce@Sauce Hussarde}

玉ねぎ2個とエシャロット2個を細かくみじん切りにして、バターで色よく炒める。白ワイン4
dLを注ぎ、半量になるまで煮詰める。\protect\hyperlink{sauce-demi-glace}{ソース・ドゥミグラス}4
dLとトマトピュレ大さじ2杯、\protect\hyperlink{fonds-blanc}{白いフォン}2
dL、生ハムの脂身のないところ80
g、潰したにんにく1片、ブーケガルニを加える。弱火で25〜30分煮込む。

ハムを取り出して、ソースをスプーンで押すようにして布で漉す。

火にかけて温め、小さなさいの目\footnote{brunoise ブリュノワーズ。}に刻んだハムと、おろしたレフォール
\footnote{raifort (レフォール)いわゆる西洋わさび、ホースラディッシュ。}少々、パセリのみじん切りをたっぷり1つまみ加えて仕上げる。

\ldots{}\ldots{}牛、羊肉のグリルまたは串を刺してローストしてアントレ\footnote{通常、ローストは料理区分としてアントレに含められることはないが、牛フィレは牛の部位のなかでも比較的小さいものとして、まるごと1本のローストであっても原則的にはアントレに分類される。このソースを用いる「牛フィレ ユサルド」は牛フィレの塊に串を刺してローストし、ポム・デュシェスとマッシュルームを合わせる。}として供する際に用いる。

\hypertarget{sauce-italienne}{%
\subsubsection[イタリア風ソース]{\texorpdfstring{イタリア風ソース\footnote{この「イタリア風」には根拠も由来も見出すことが出来ない。地名、国名を料理名に冠した代表例のひとつ。}}{イタリア風ソース}}\label{sauce-italienne}}

\frsub{Sauce Italienne}

\index{そーす@ソース!いたりあふう@イタリア風---}
\index{いたりあふう@イタリア風!そーす@---ソース}
\index{sauce@sauce!Italienne@--- Italienne}
\index{italien@italien(ne)!sauce@Sauce Italienne}

トマトの風味の効いた\protect\hyperlink{sauce-demi-glace}{ソース・ドゥミグラス}\(\frac{3}{4}\)
Lに、\protect\hyperlink{duxelles-seche}{デュクセル・セッシュ}大さじ4杯と、加熱ハムの脂身のないところを小さなさいの目に切ったもの125
gを加える。5〜6分間煮る。提供直前に、パセリとセルフイユ、エスゴラゴンのみじん切り大さじ1杯を加えて仕上げる。

\ldots{}\ldots{}いろいろな肉料理に合わせる。

\hypertarget{nota-sauce-italienne}{%
\subparagraph{【原注】}\label{nota-sauce-italienne}}

このソースを魚料理に合わせる場合、ハムは使わずに\protect\hyperlink{fumet-de-poisson}{魚のフュメ}を煮詰めて加える。

\hypertarget{jus-lie-a-lestragon}{%
\subsubsection{とろみを付けたジュ・エストラゴン風味}\label{jus-lie-a-lestragon}}

\frsub{Jus lié à l'Estragon}

\index{そーす@ソース!とろみをつけたしゆえすとらこん@とろみを付けたジュエストラゴン風味}
\index{しゆ@ジュ!とろみをつけたえすとらごん@とろみを付けた---・エストラゴン風味}
\index{えすとらこん@エスゴラゴン!とろみをつけたしゆ@とろみを付けたジュ・---風味}
\index{sauce@sauce!jus lie estragon@Jus lié à l'Estragon}
\index{estragon@estragon!jus lie estragon@Jus lié à l'Estragon}
\index{jus@jus!lie estragon@--- lié à l'Estragon}

\protect\hyperlink{jus-de-veau-brun}{仔牛のフォン}または\protect\hyperlink{fonds-de-volaille}{鶏のフォン}に、エストラゴン50
gを加えて香りを煮出し\footnote{imfuser アンフュゼ。}たもの。

布で漉してから、アロールート\footnote{コーンスターチで代用する。}または、でんぷん30
gでとろみを付ける。

\ldots{}\ldots{}白身肉のノワゼットや家禽のフィレなどに添える。

\hypertarget{jus-lie-tomate}{%
\subsubsection{とろみを付けたジュ・トマト風味}\label{jus-lie-tomate}}

\frsub{Jus lié tomaté}

\index{そーす@ソース!とろみをつけたしゆとまと@とろみを付けたジュ・トマト風味}
\index{しゆ@ジュ!とろみをつけたとまとふうみ@とろみを付けた---・トマト風味}
\index{とまと@トマト!とろみをつけたしゆ@とろみを付けたジュ・---風味}
\index{sauce@sauce!jus lie tomatee@Jus lié tomaté}
\index{tomate@tomate!jus lie tomate@Jus lié tomaté}
\index{jus@jus!lie tomate@--- lié tomaté}

\protect\hyperlink{jus-de-veau-brun}{仔牛のフォン}1
Lあたり\protect\hyperlink{essence-de-tomate}{トマトエッセンス}3
dLを加え、 \(\sfrac{4}{5}\)量まで煮詰める。

\ldots{}\ldots{}牛、羊肉料理用。

\hypertarget{sauce-lyonnaise}{%
\subsubsection{リヨン風ソース}\label{sauce-lyonnaise}}

\frsub{Sauce Lyonnaise}

\index{そーす@ソース!りよんふう@リヨン風---}
\index{りよんふう@リヨン風!そーす@---ソース}
\index{sauce@sauce!lyonnaise@--- Lyonnaise}
\index{liyonnais@lyonnais(e)!sauce lyonnaise@Sauce ---e}

中位の大きさの玉ねぎ3個をみじん切りにし、バターでじっくり、ごく弱火でブロンド色になるまで炒める。白ワイン2
dLとヴィネガー2 dLを注ぐ。
\(\frac{1}{3}\)量まで煮詰め、\protect\hyperlink{sauce-demi-glace}{ソース・ドゥミグラス}\(\frac{3}{4}\)
Lを加える。5〜6分かけて表面に浮いてくる不純物を丁寧に取り除き\footnote{dépouiller
  デプイエ。現代ではエキュメと呼ぶ現場が多い。}、布で漉す。

\hypertarget{nota-sauce-lyonnaise}{%
\subparagraph{【原注】}\label{nota-sauce-lyonnaise}}

このソースを合わせる料理によっては、ソースを布で漉さずに玉ねぎを残してもいい。

\hypertarget{sauce-madere}{%
\subsubsection{ソース・マデール}\label{sauce-madere}}

\frsub{Sauce Madère}

\index{そーす@ソース!までーる@---・マデール}
\index{まてらしゆ@マデイラ酒 ⇒ マデール!そーす@ソース・マデール}
\index{sauce@sauce!madere@--- Madère}
\index{madere@madère!sauce@Sauce ---}

\protect\hyperlink{sauce-demi-glace}{ソース・ドゥミグラス}を煮詰め\footnote{ソース・ドゥミグラスは既に煮詰めて仕上がった状態のものなので、9
  割程度にまでしか煮詰めないことに注意。}、火から外して、ソース1
Lあたりマデイラ酒1 dLの割合で加え、普通の濃度にする。

\hypertarget{sauce-matelote}{%
\subsubsection[ソース・マトロット]{\texorpdfstring{ソース・マトロット\footnote{水夫風、船員風、の意。トゥーレーヌ地方の郷土料理Matelote
  d'anguille(マトロットダンギーユ)うなぎの赤ワイン煮込み、が有名。とはいえ本書にも数種のレシピが収録されているように、赤ワイン煮込みにとどまらず、マトロットの名称を持つ料理は他にも複数存在する。}}{ソース・マトロット}}\label{sauce-matelote}}

\frsub{Sauce Matelote}

\index{そーす@ソース!まとろつと@---・マトロット}
\index{まとろつと@マトロット!そーす@ソース・---}
\index{sauce@sauce!matelote@--- Matelote}
\index{matelote@matelote!sauce@Sauce ---}

魚をポシェするのに使った\protect\hyperlink{court-bouillon-c}{赤ワイン入りの魚用クールブイヨン}3
dLにマッシュルームの切りくず25
gを加え、\(\frac{1}{3}\)量になるまで煮詰める。

煮詰めたら\protect\hyperlink{sauce-espagnole-maigre}{魚料理用ソース・エスパニョル}8
dL を加えてひと煮立ちさせる。布で漉し、バター150
gとごく少量のカイエンヌの粉末を加えて仕上げる。

\hypertarget{sauce-moelle}{%
\subsubsection[ソース・モワル]{\texorpdfstring{ソース・モワル\footnote{moelle
  骨髄のこと。}}{ソース・モワル}}\label{sauce-moelle}}

\frsub{Sauce Moelle}

\index{そーす@ソース!もわる@---・モワル}
\index{こつずい@骨髄 ⇒ モワル!そーすもわる@ソース・モワル}
\index{sauce@sauce!moelle@--- Moelle}
\index{moelle@moelle!sauce@Sauce ---}

ソースの作り方は\protect\hyperlink{sauce-bordelaise}{ボルドー風ソース}とまったく同じだが、バターを加えるのは何らかの野菜料理に添える場合のみであり、その場合のバターの量は通常どおりとするこ。

どんな場合にせよ、仕上げに、小さなさいの目に切ってポシェしておいた骨髄をソース1
Lあたり150〜180 gおよび刻んで下茹でしたパセリの葉小さじ1杯を加える。

\hypertarget{sauce-moscovite}{%
\subsubsection[モスクワ風ソース]{\texorpdfstring{モスクワ風ソース\footnote{moscovite(モスコヴィット)すなわちモスクワ風の名称を持つ料理や菓子は多い。
  18世紀後半から19世紀前半にかけて、ロシアの宮廷や貴族らの間でフランスの食文化が流行し、多くのフランス人料理人が招聘され、彼らはロシア料理のレシピをフランスに持ち帰った。クーリビヤックなどが代表的な例だろう。また、19世紀後半になると、とりわけフランス料理においてもロシア料理からの影響が多く見られるようになる。キャビアとウォトカを食前に愉しむのが流行したのもその時代からである。フランスとロシアの食文化は相互に影響関係にあったと言えよう。}}{モスクワ風ソース}}\label{sauce-moscovite}}

\frsub{Sauce Moscovite}

\index{そーす@ソース!もすくわふう@モスクワ風---}
\index{もすくわふう@モスクワ風!そーす@---ソース}
\index{sauce@sauce!moscovite@--- Moscovite}
\index{moscovite@moscovite!sauce@Sauce ---}

\protect\hyperlink{fonds-de-gibier}{大型ジビエのフュメ}で作った\protect\hyperlink{sauce-poivrade}{ソース・ポワヴラード}を\(\frac{3}{4}\)
L用意する。提供直前にマラガ酒1 dL とジェニパーベリーを煎じた汁7
cL\footnote{1 cL(センチリットル) = 10 mL、つまりこの場合は70 mL。}、焼いた松の実かスライスして焼いたアーモンド40
g、大きさを揃えてぬるま湯でもどしておいたコリント産干しぶどう \footnote{小粒で黒いギリシア産干しぶどう。}40
gを加えて仕上げる。

\ldots{}\ldots{}大型ジビエ\footnote{venaison
  ヴネゾン。ジビエのうち大型のものを指す。実際はノロ鹿や猪を指すことがほとんど。}の塊肉の料理用。

\hypertarget{sauce-perigueux}{%
\subsubsection[ソース・ペリグー]{\texorpdfstring{ソース・ペリグー\footnote{トリュフの産地として有名なペリゴール地方の町の名。}}{ソース・ペリグー}}\label{sauce-perigueux}}

\frsub{Sauce Périgueux}

\index{そーす@ソース!へりくー@---・ペリグー}
\index{へりくー@ペリグー!そーす@ソース・---}
\index{sauce@sauce!perigueux@--- Périgueux}
\index{perigueux@Périgueux!sauce@Sauce ---}

やや濃いめに煮詰めた\protect\hyperlink{sauce-demi-glace}{ソース・ドゥミグラス}\(\frac{3}{4}\)
Lに、トリュフエッセンス1 \(\frac{1}{2}\) dLと細かく刻んだトリュフ100
gを加える。

\ldots{}\ldots{}いろいろな肉料理、\protect\hyperlink{}{タンバル}、\protect\hyperlink{}{温製パテ}に合わせる。

\hypertarget{sauce-perigourdine}{%
\subsubsection[ソース・ペリグルディーヌ]{\texorpdfstring{ソース・ペリグルディーヌ\footnote{périgourdin(e)
  (ペリグルダン/ペリグルディーヌ)ペリゴール地方風の意。}}{ソース・ペリグルディーヌ}}\label{sauce-perigourdine}}

\frsub{Sauce Périgourdine}

\index{そーす@ソース!へりくるていーぬ@---・ペリグゥルディーヌ}
\index{へりこーるふう@ペリゴール風 ⇒ ペリグルダン/ペリグルディーヌ!そーす@ソース・ペリグルディーヌ}
\index{sauce@sauce!perigourdine@--- Périgourdine}
\index{perigourdin@périgourdin(e)!sauce@Sauce Périgourdine}

ソース・ペリグーのバリエーション。トリュフを細かく刻むのではなく、オリーブ形か小さな真珠のような形状にナイフで成形\footnote{tourner
  トゥルネ。包丁を持っている側の手は動かさずに材料を回すようにして形を整えること。}したものを加える。トリュフを厚めにスライスして加える場合もある。

\hypertarget{sauce-piquante}{%
\subsubsection[ソース・ピカント]{\texorpdfstring{ソース・ピカント\footnote{piquant(e)
  (ピカン、ピカント)
  一般的には唐辛子などが「辛い」の意だが、このソースでは唐辛子の類は使われておらず、むしろ酸味の効いたソースと言えよう。古くからのソース名。}}{ソース・ピカント}}\label{sauce-piquante}}

\frsub{Sauce Piquante}

\index{そーす@ソース!ひかんと@---・ピカント}
\index{ひかんと@ピカント!そーす@ソース・---}
\index{sauce@sauce!piquante@--- Piquante}
\index{piquant@piquant(e)!sauce@Sauce Piquante}

白ワイン3 dLと良質のヴィネガー3 dLにエシャロットのみじん切り大さじ2
\(\frac{1}{2}\) 杯を合わせて半量に煮詰める。

\protect\hyperlink{sauce-espagnole}{ソース・エスパニョル}6
dLを加え、浮いてくる不純物を取り除きながら\footnote{dépouiller
  デプイエ。エキュメécumerと呼ぶ現場も多い。}10分間煮る。

火から外し、コルニション\footnote{専用品種のきゅうりを小さなうちに収穫して酢漬けにしたもの。同様のピクルス用きゅうりとしてガーキンスという品種系統があるがもっぱらアメリカのハンバーガーに挟まれるようなサイズで収穫して漬けたものであり、フランス料理では用いない。}、パセリ、セルフイユ、エストラゴンを細かく刻んだもの大さじ2杯を加えて仕上げる。

\ldots{}\ldots{}豚肉のグリル焼き、ブイイ\footnote{bouilli
  茹で肉。もとはブイヨンをとった後の茹で肉のことを指した。単純に「茹でた肉」としてもいいのだが、17世紀にはこの食べ方が流行したという歴史もあり、野菜などと共に、あるいは他の素材なしに茹でた肉はたんに「ブイイ」bouilli
  と呼ばれる。}、ローストによく合わせるソース。牛肉のブイイや牛や羊の\protect\hyperlink{}{エマンセ}にも合わせることが出来る。

\hypertarget{sauce-poivrade}{%
\subsubsection[ソース・ポワヴラード
(標準)]{\texorpdfstring{ソース・ポワヴラード\footnote{このソースは遅くとも16世紀まで遡ることが出来る。1505年に出版された\href{http://gallica.bnf.fr/ark:/12148/bpt6k792720}{『フランス語版プラティナ』}がpoivradeというフランス語の初出。この本において「ジビエ用こしょうのソース、ポワヴラード」Saulce
  de poyvre ou poyvrade pour
  saulvagieとしてレシピが見られる。パンをよく焼いてヴィネガーに浸してすり潰す。水でもどした干しぶどうと獣の血を加えて混ぜ、玉ねぎと未熟ぶどう果汁、パンを浸した残りのヴィネガーを加えて漉し器か布で漉す。これを鍋に入れ、こしょう、生姜、シナモンを入れて炭火の上で30分程煮込む。獣の肉を獣脂を熱したフライパンで焼き、皿に盛る。上からポワヴラードをかけて供する、という内容(f.LXII)。またこの本には、魚料理用のポワヴラードも掲載されている。ただし、これが現代まで続くソース・ポワヴラードの原型と捉えるのは早計に過ぎる。ここで注目すべきは、最終的に肉あるいは魚のような主素材とソースが一体化したものは中世〜ルネサンス期にはポタージュと呼ばれていたのに対し、ここではソースを別のものと捉えている点である。ポワヴラードという語そのものは「こしょうを効かせたもの」という意味に過ぎず、1660年刊ピエール・ド・リュヌPierre
  de Lune『新フランス料理』におけるPoivrade de pigeonneaux
  若鳩のポワヴラードは、背開きにした若鳩を平たくのばし、塩、こしょう
  をして弱火でグリルする。薔薇の香りもしくはにんにく風味のヴィネガーを添えて供する、というもの(p.190)。ピエール・ド・リュヌのレシピにおいてソースに相当するものはヴィネガーであり、むしろ味付けでこしょうを効かせているということが料理名の根拠となっているに過ぎない。ちなみに、生食可能な小さなサイズのアーティチョークも古くからポワヴラードと呼ばれている。}
(標準)}{ソース・ポワヴラード (標準)}}\label{sauce-poivrade}}

\frsub{Sauce Poivrade ordinaire}

\index{そーす@ソース!ほわうらーと@---・ポワヴラード(標準)}
\index{ほわうらーと@ポワヴラード!そーす@ソース・---(標準)}
\index{sauce@sauce!poivrade ordinaire@--- Poivrade ordinaire}
\index{poivrade@poivrade!sauce ordinaire@Sauce --- ordinaire}

細かいさいの目に切ったにんじん100 gと玉ねぎ80
g、刻んだパセリの茎、タイム少々、ローリエの葉少々からなる\protect\hyperlink{mirepoix}{ミルポワ}を油で色付くまで炒める。

ヴィネガー1 dLとマリナード2
dLを注ぎ、\(\frac{1}{3}\)量になるまで煮詰める。\protect\hyperlink{sauce-espagnole}{ソース・エスパニョル}1
Lを注ぎ、約45分間煮込む。

ソースを漉す10分前に、大粒のこしょう8個を叩きつぶして加える。ソースにこしょうを入れてからの時間がこれ以上少しでも長いと、こしょうの風味が支配的になり過ぎることになるので注意。

漉し器で香味素材を軽く押すようにして漉す。\protect\hyperlink{marinades-et-saumures}{マリナード}\footnote{ヴィネガーやワイン、香味素材、塩などを合わせて肉を漬け込む液体。マリネ液と呼ぶこともある。}2
dLでソースをのばす。火にかけて35分間、所定の量\footnote{明記されていないが、ここでは約1
  L。}になるまで煮詰めながら、表面に浮いてくる不純物を徹底的に取り除く\footnote{dépouiller
  デプイエ。現代ではécumerエキュメの語を使う現場が多い。}。

さらに布で漉し、バター50 gを加えて仕上げる\footnote{現代では、バターでモンテするmonter
  au beurreという表現を用いる現場も多い。}。

\hypertarget{sauce-poivrade-pour-gibier}{%
\subsubsection{ソース・ポワヴラード(ジビエ用)}\label{sauce-poivrade-pour-gibier}}

\frsub{Sauce Poivrade pour Gibier}

\index{そーす@ソース!ほわうらーとしひえ@---・ポワヴラード(ジビエ用)}
\index{ほわうらーと@ポワヴラード!そーすしひえ@ソース・---(ジビエ用)}
\index{sauce@sauce!poivrade gibier@--- Poivrade pour Gibier}
\index{poivrade@poivrade!sauce  gibier@Sauce --- pour Gibier}

細かいさいの目に切ったにんじん125 gと玉ねぎ125
g、タイムの枝と鳥類ではないジビエ\footnote{gibier à poil
  逐語訳すると「毛の生えているジビエ」すなわち」鹿、猪、野うさぎなどを指す。}の端肉1
kgからなる\protect\hyperlink{mirepoix}{ミルポワ}を油で色よく炒める。

ミルポワが色付いてきたら、鍋の油を捨てる。ヴィネガー3 dLと白ワイン2 dL
を注ぎ、完全に煮詰める。

\protect\hyperlink{sauce-espagnole}{ソース・エスパニョル}1
Lと\protect\hyperlink{fonds-de-gibier}{ジビエの茶色いフォン}2
L、\protect\hyperlink{marinades-et-saumures}{マリナード}1 Lを加える。

鍋に蓋をして弱火にかける。可能ならオーブンがいい。3時間半〜4時間加熱する。

ソースを漉す8分前に、大粒のこしょう12個を叩きつぶして加える。

漉し器で材料を押すようにして漉す。

これをジビエのフォン\(\frac{1}{4}\) Lとマリナード\(\frac{1}{4}\)
Lでのばし、再び火にかけて40分間、表面に浮いてくる不純物を丁寧に取り除きながら、1
Lになるまで煮詰める。

これを布で漉し、バター75 gを加えて仕上げる。

\hypertarget{nota-sauce-poivrade-pour-gibier}{%
\subparagraph{【原注】}\label{nota-sauce-poivrade-pour-gibier}}

一般的にはジビエ料理のソースにはバターを加えないことになっているが、本書では軽くバターを加えることを推奨する。そうすると、ソースの色の赤みは薄まるが、繊細で滑らかな口あたりに仕上がる。

\hypertarget{sauce-au-porto}{%
\subsubsection{ソース・ポルト}\label{sauce-au-porto}}

\frsub{Sauce au Porto}

\index{そーす@ソース!ほると@---・ポルト}
\index{ほるとしゆ@ポルト酒 ⇒ ポルト!そーす@ソース・---}
\index{sauce@sauce!porto@--- au Porto}
\index{porto@Porto!sauce@Sauce au ---}

マデイラ酒ではなくポルト酒を用いて、\protect\hyperlink{sauce-madere}{ソース・マデール}と同様に作る。

\hypertarget{sauce-portugaise}{%
\subsubsection[ポルトガル風ソース]{\texorpdfstring{ポルトガル風\footnote{日本でもフランス語のままソース・ポルチュゲーズと呼ばれることは多い。フランス料理においてポルトガル風の名称を付けた料理はトマトをベースとしたものがほとんど。ただし、トマトを使うからといってポルトガル風の名が必ず付くというわけではない。}ソース}{ポルトガル風ソース}}\label{sauce-portugaise}}

\frsub{Sauce Portugaise}

\index{そーす@ソース!ほるとかるふう@ポルトガル風---}
\index{ほるとかるふう@ポルトガル風!そーす@---ソース}
\index{sauce@sauce!porugaise@--- Portugaise}
\index{portugais@portugais(e)!sauce@Sauce Portugaise}

(仕上がり1 L分)

大きめの玉ねぎ1個を細かくみじん切りにする。鍋に油を熱し、強火で玉ねぎを炒める。玉ねぎがブロンド色になったら、皮を剥いて種子を取り除き、粗みじん切りにしたトマト750
gと、つぶしたにんにく1片、塩、こしょうを加える。トマトの酸味が強い場合は砂糖少々も加える。鍋に蓋をして、弱火で煮る。\protect\hyperlink{essence-de-tomate}{トマトエッセンス}少々と、薄めに作ったトマトソースを適量\footnote{仕上がりの全体量が1
  Lなので、トマトソースを加える量は、グラスドヴィアンドを加える前の段階で0.9
  L程度になるよう調整する。}、温めて溶かした\protect\hyperlink{glace-de-viande}{グラスドヴィアンド}1
dL、新鮮なパセリの葉のみじん切り大さじ1杯を加えて仕上げる。

\hypertarget{sauce-provencale}{%
\subsubsection{プロヴァンス風ソース}\label{sauce-provencale}}

\frsub{Sauce Provençale}

\index{そーす@ソース!ふろうあんすふう@プロヴァンス風---}
\index{ふろうあんすふう@プロヴァンス風!そーす@---ソース}
\index{sauce@sauce!provencale@--- Provençale}
\index{provencal@provençal(e)!sauce@Sauce Provençale}

大ぶりのトマト12個の皮を剥き、つぶして種子は取り除いて、粗く刻む\footnote{concasser
  コンカセ。}。ソテー鍋に2 \(\frac{1}{2}\)
dLの油を熱し、そこにトマトを入れる。塩、こしょう、粉砂糖1つまみで味を調える。しっかりつぶしたにんにく(小)1片と細かく刻んだパセリ小さじ1杯を加える。

蓋をして弱火で30分間程、煮溶かす。

\hypertarget{nota-sauce-provencale}{%
\subparagraph{【原注】}\label{nota-sauce-provencale}}

このソースについてはさまざまな解釈があるが、本書ではブルジョワ料理における本物の「プロヴァンス風ソース」のレシピ、つまりはトマトの「フォンデュ」\footnote{加熱によって溶かしたもの、の意。このレシピはあくまでも「ソース」であり、料理を作る際のアパレイユ≒パーツとしてのいわゆる\protect\hyperlink{portugaise}{トマトフォンデュ}については第2章ガルニチュール、温製ガルニチュール用のアパレイユなど、の項を参照。}、を収録した。

\hypertarget{sauce-regence}{%
\subsubsection[ソース・レジャンス]{\texorpdfstring{ソース・レジャンス\footnote{Régence(レジョンス)はこの場合固有名詞としての「摂政時代」を指す。すなわちオルレアン公フィリップがルイ15世の幼少期に摂政を務めていたた時代(1715〜1723年)のこと。オルレアン公は美食家として有名で、とりわけシャンパーニュを好んだという。この時期はフランス宮廷料理の絶頂期でもあった。}}{ソース・レジャンス}}\label{sauce-regence}}

\frsub{Sauce Régence}

\index{そーす@ソース!れしやんす@---・レジャンス}
\index{れしやんす@レジャンス!そーす@ソース・---}
\index{sauce@sauce!regence@--- Régence}
\index{regence@Régence!sauce@Sauce ---}

ライン産ワイン3
dLに、細かく刻んであらかじめ火を通しておいた\protect\hyperlink{mirepoix}{ミルポワ}1
dLと生トリュフの切りくず25
gを加え、半量になるまで煮詰める。トリュフのシーズンでない時季はトリュフエッセンスを使う。\protect\hyperlink{sauce-demi-glace}{ソース・ドゥミグラス}8
dLを加え、数分間弱火にかけて浮いてくる不純物を丁寧に取り除き\footnote{dépouiller
  デプイエ ≒ écumer エキュメ。}、布で漉す。

\ldots{}\ldots{}牛、羊の大きな塊肉の料理用。

\hypertarget{sauce-robert}{%
\subsubsection[ソース・ロベール]{\texorpdfstring{ソース・ロベール\footnote{この名称のソースは古くからある。文献で初めて出てくるのは16世紀フランソワ・ラブレーの小説『ガルガンチュアとパンタグリュエル』。その「第四の書」で料理人の名が大量に列挙される章がある。そのうちの多くは架空の人名だが、その中のロベールという料理人がこのソースを考案したと書いている。ただし、具体的にどのようなソースかまでは描写されておらず「うさぎのロースト、鴨、加工していない豚肉、卵のポシェ、塩漬けのメルラン{[}鱈の近縁種{]}、その他まことに多くの料理に欠かせないソース」と書いてあるのみ(第40章)。どんな料理にも合うと書かれてしまうとむしろ特徴を捉え難くなってしまう。いずれにせよ、遅くとも16世紀には「ソース」として成立していたと考えられる。また、17世紀のシャルル・ペロー著『物語集』の「眠れる森の美女」においても、このソース名が登場する一節がある。このように16世紀以降多くの文学作品をはじめとする文献にこのソース名は見られる。レシピとしては、1651年刊ラ・ヴァレーヌ『フランス料理の本』における「豚腰肉 ソース・ロベール添え」がもっとも古いもののひとつだろう。概略は、豚腰肉を、ヴェルジュ{[}未熟ぶどう果汁、中世料理においてよく用いられた{]}とヴィネガー、セージを振り掛けながらローストする。下に置いた脂受け皿に焼いた豚肉から流れ落ちた脂がたまるので、これを使って玉ねぎをこんがり炒める。炒めた玉ねぎの上に豚後ろ身を載せ、豚腰肉をローストする際にかけたのと同じソースをかける。このソースはソースロベールと呼ばれている(p.51)。また、干鱈のソース・ロベール添えの場合は、バターとヴェルジュ少々、マスタードで作るが、ケイパーやシブール{[}葱{]}を加えてもいい(p.202)とあり、同じ名称のソースとは見做しがたい。18世紀以降のソース・ロベールは多かれ少なかれいずれもマスタードを加える点が共通しているので、名称が先にあり、内容が時代とともにはっきりしたものになっていたのだろう。}}{ソース・ロベール}}\label{sauce-robert}}

\frsub{Sauce Robert}

\index{そーす@ソース!ろへーる@---・ロベール}
\index{ろへーる@ロベール!そーす@ソース・---}
\index{sauce@sauce!robert@--- Robert}
\index{robert@Robert!sauce@Sauce ---}

(仕上がり5 dL分)

大きめの玉ねぎを細かくみじん切りにし、バターで色付かないよう強火でさっと炒める。

白ワイン2
dLを注ぎ、\(\frac{1}{3}\)量になるまで煮詰める。\protect\hyperlink{sauce-demi-glace}{ソース・ドゥミグラス}3
dLを加え、弱火で10分間煮る。

シノワ\footnote{主として金属製で円錐形に取っ手の付いた漉し器。清朝の高級役人がかぶっていた帽子の形状から「中国の」を意味するchinoisの名称となったと言われている。}で漉し(これは任意。漉さなくてもいい)、火から外して、粉砂糖1つまみとマスタード大さじ1杯を加えて仕上げる。

\hypertarget{sauce-robert-escoffier}{%
\subsubsection{ソース・ロベール・エスコフィエ}\label{sauce-robert-escoffier}}

\frsub{Sauce Robert Escoffier}

\index{そーす@ソース!ろへーるえすこふいえ@---・ロベール・エスコフィエ}
\index{ろへーる@ロベール!そーすえすこふいえ@ソース・---・エスコフィエ}
\index{sauce@sauce!robert escoffier@--- Robert Escoffier}
\index{robert@Robert!sauce escoffier@Sauce --- Escoffier}

このソースは完成品が市販されている\footnote{\protect\hyperlink{sauce-diable-escoffier}{ソース・ディアーブル・エスコフィエ}訳注参照。}。

温かい料理にも冷たい料理にもよく合う。温かい料理に合わせる場合は、同量の\protect\hyperlink{jus-de-veau-brun}{仔牛の茶色いフォン}と混ぜること。

\ldots{}\ldots{}豚、仔牛、鶏、魚のグリル焼きに特によく合う。

\hypertarget{sauce-romaine}{%
\subsubsection[ローマ風ソース]{\texorpdfstring{ローマ風\footnote{フランス料理における「ローマ風」の名称は「イタリア風」と同様にとくに根拠や由来が見出せないものが多い。このソースの場合は松の実を使うところから、20世紀前半に活躍したイタリアの作曲家レスピーギのローマ三部作のうちの「ローマの松」を想起させるが、残念ながらこの曲が作曲されたのは1924年、つまり本書より後なので関係はない。だが、松の実を採るイタリアカサマツは、アッピア街道の並木などで有名なように、イタリアとりわけローマ近辺において多く見られる(だからこそレスピーギが曲の題材にしたわけだが)。その意味においては、松の実を使っているということがこのソース名の根拠と見ることも不可能ではないだろう。しかしながら、それを証明する文献、史料があるかは不明。}ソース}{ローマ風ソース}}\label{sauce-romaine}}

\frsub{Sauce Romaine}

\index{そーす@ソース!ろーまふう@ローマ風---}
\index{ろーまふう@ローマ風!そーす@---ソース}
\index{sauce@sauce!romain@--- Romaine}
\index{romain@romain(e)!Sauce Romaine}

砂糖50 gを火にかけてブロンド色にカラメリゼ\footnote{焦がさないように弱火で混ぜながら熱で砂糖を溶かしていく。}する。これをヴィネガー
1 \(\frac{1}{2}\)
dLでのばす。砂糖を完全に溶かし込めたら、\protect\hyperlink{sauce-espagnole}{ソース・エスパニョル}6
dLと\protect\hyperlink{fonds-de-gibier}{ジビエのフォン}3
dLを加える。これを\(\frac{3}{4}\)量弱まで煮詰める。布で漉し、松の実20
gをローストしたものと、大きさが揃るよう選別したスミヌル干しぶどう\footnote{トルコ産の白い干しぶどう。}20
gおよびコリント干しぶとう\footnote{ギリシア産の黒い小粒の干しぶどう(\protect\hyperlink{sauce-moscovite}{モスクワ風ソース}参照)。}20
gを温湯でもどしたものを加えて仕上げる。

\hypertarget{nota-sauce-romaine}{%
\subparagraph{【原注】}\label{nota-sauce-romaine}}

上記のとおり作る場合、このソースは大型ジビエ料理用だが、ジビエのフォンではなく通常の\protect\hyperlink{fonds-brun}{茶色いフォン}を使えば、マリネした牛、羊肉の料理に合わせることも可能。

\hypertarget{sauce-rouennaise}{%
\subsubsection[ルーアン風ソース]{\texorpdfstring{ルーアン風\footnote{ルーアンは野生のcolvertコルヴェール、いわゆる青首鴨を家禽化したルーアン鴨の産地として有名。}ソース}{ルーアン風ソース}}\label{sauce-rouennaise}}

\frsub{Sauce Rouennaise}

\index{そーす@ソース!るーあんふう@ルーアン風---}
\index{るーあんふう@ルーアン風!そーす@---ソース}
\index{sauce@sauce!rouannaise@--- Rouannaise}
\index{rouannais@rouannais(e)!sauce@Sauce Rouannaise}

(仕上がり5 dL分)

\protect\hyperlink{sauce-bordelaise}{ボルドー風ソース}4 dL
を用意する。ただし、良質な赤ワインを使って作ること。(\protect\hyperlink{sauce-bordelaise}{ボルドー風ソース}参照)。

中位の大きさの鴨のレバー3個を裏漉しする。こうして出来たレバーのピュレをソースに加え、沸騰させない程度の温度で火を通す\footnote{pocher
  ポシェする。}。絶対に沸騰させないこと。沸騰させてしまうと途端にレバーのピュレが粒状になってしまう。

布で漉し、塩こしょうを効かせる。

このソースの特質\ldots{}\ldots{}エシャロットを加えた赤ワインを煮詰めたものに鴨の生レバーのピュレを加えたもの。

\ldots{}\ldots{}ルーアン産鴨のローストには、いわば必須といってもいいソース。

\hypertarget{sauce-salmis}{%
\subsubsection[ソース・サルミ]{\texorpdfstring{ソース・サルミ\footnote{語源は「ごった煮」を意味する
  salmigondis
  とするのが定説のようだが、salmigondisがその意味で用いられるようになったのは19世紀以降と考えられ、それ以前はragoûtラグーと同義と見なされていた。ラグーはその語源的意味が「食欲をそそるもの」であり、17世紀に、それまでポタージュと呼ばれていた煮込み料理についてラグーの名称をつけることが流行した。また、salmigondisの古い語形のひとつsalmigondinは16世紀の小説家フランソワ・ラブレー『ガルガンチュアとパンタグリュエル』の「第四の書」において用いられているが、日本語の「ごった煮」のニュアンスとはかなり違う意味で、美味な料理のひとつとして挙げられている。いずれにしても、salmigondin,
  salmigondisというラグーの別称が、ある時期から鳥類を材料にしたものに限定されるようになったことは確かで、カレームの『19世紀フランス料理』ではsalmisの語で、野鳥などのラグーを呼んでいる。例えば「ベカスのサルミ」「ペルドローのサルミ」など。}}{ソース・サルミ}}\label{sauce-salmis}}

\frsub{Sauce Salmis}

\index{そーす@ソース!さるみ@---・サルミ}
\index{さるみ@サルミ!そーす@ソース・---}
\index{sauce@sauce!salmis@--- Salmis}
\index{salmis@salmis!sauce@Sauce ---}

ソースというよりはむしろクリ\footnote{coulis \textless{} couler
  クレ「流れる」から派生した語だが、料理用語としては、やや水分の多いピュレと理解するといい。日本では「クーリ」と呼ぶことも多い。ここでは二つの解釈が可能で、ひとつは\protect\hyperlink{}{ポタージュ・クリ}に近いという意味。もうひとつは「昔ながらのソース」の意。後者の場合、エスコフィエが「古典料理」と呼ぶ17、18世紀においてソースのことをクリと呼んでいたのを踏まえていると考えられる。}と呼んだほうがいいこのソースの作り方はどんな場合も一点を除いて変わることがない。それは、このソースを合わせるジビエ(鳥)の種類によって、つまり普通に肉料理として扱えるジビエか、肉断ち\footnote{小斉のこと。カトリックの習慣として(厳密な教義ではない)四旬節(復活祭までの46日間)や毎週金曜などに行なわれる、肉食を断つ行為のこと。}の際の食材として扱えるもの\footnote{ある種の水鳥はイルカと同様に魚と同等のものと見做され、小斉の場合にも食材として認められていた。具体的にはハシヒロ鴨、オナガ鴨、サルセル鴨など。もっとも、水鳥を肉断ちの際の食材として扱うというのは一種の詭弁ともいえなくないわけで、このソースを作る際に\protect\hyperlink{sauce-espagnole-maigre}{魚料理用ソース・エスパニョル}をベースとした\protect\hyperlink{sauce-demi-glace}{ソース・ドゥミグラス}を使うとは考え難く、本文にあるようにフォンの代用としてマッシュルームの茹で汁を用いるという指示を守るだけで、厳密に小斉の料理として成立するレシピと言えるかは疑問の残るところだ。}かで、どんな液体を用いるかということだけだ。

細かく刻んだ\protect\hyperlink{mirepoix}{ミルポワ}150
gをバターでじっくり色付くまで炒める。そこに、その料理で用いているジビエの手羽と腿の皮、ガラを細かく刻んで加える。

白ワイン3
dLを注ぎ、\(\frac{1}{3}\)量まで煮詰める。\protect\hyperlink{sauce-demi-glace}{ソース・ドゥミグラス}8
dLを加えて、約45分間弱火で煮込む。漉し器で漉すが、その際に香味野菜とガラのエキス\footnote{原文quintessence(カンテソンス)。本来の意味は錬金術でいう「第五元素」。16世紀の作家フランソワ・ラブレーは存命当時、自著を筆名「カンテサンス抽出をなし遂げたアルコフリバス師」で出版していた時期がある。もっとも、このカンテサンスという語自体は中世以来、料理において「エキス」「美味しさの本質」程度の意味でよく用いられた。}が得られるよう、強く押し絞ってやること。こうして出来たクリを、このソースを合わせる鳥と同種のものでとったフォン4
dLで薄める。

ジビエが肉断ちの食材と見做されるもので、なおかつそれを厳格に守って作らなければならない場合は、このときフォンの代わりにマッシュルームの茹で汁を用いればいい。

約45分〜1時間、弱火にかけて浮いてくる不純物を丁寧に取り除いてやる\footnote{dépouiller
  デプイエ。現代ではécumerエキュメの語を用いる現場が多い。}。さらにソースを\(\frac{2}{3}\)以下の量になるまで煮詰める。これにマッシュルームの茹で汁とトリュフエッセンスを適量加えて丁度いい濃度になるよう調製する。

布で漉し、軽くバターを加えて仕上げる\footnote{原文は légèrement
  beurrerでありそのまま訳したが、現代の調理現場ではmonter au beurre
  バターでモンテする、という表現がよく使われる。}。

\hypertarget{nota-sauce-salmis}{%
\subparagraph{【原注】}\label{nota-sauce-salmis}}

仕上げの際に、ソース1 Lあたりバター約50 gを加えるが、これは任意。

\hypertarget{sauce-tortue}{%
\subsubsection[ソース・トルチュ]{\texorpdfstring{ソース・トルチュ\footnote{tortue
  (トルチュ)は海亀のこと。古くは海亀料理用のソースだったが、19世紀以降は仔牛の頭肉料理に合わせるのが一般的になった。なお、
  tortu(e)という形容詞があり「曲がりくねった、(性格が)ひねくれた」という同音異義語があるが、このソースの由来とは無関係。}}{ソース・トルチュ}}\label{sauce-tortue}}

\frsub{Sauce Tortue}

\index{そーす@ソース!とるちゅ@---・トルチュ}
\index{とるちゅ@トルチュ!そーす@ソース・---}
\index{うみかめ@海亀 ⇒ トルチュ!そーすとるちゆ@ソース・トルチュ}
\index{sauce@sauce!tortue@--- Tortue}
\index{tortue@tortue!sauce@Sauce ---}

2 \(\frac{1}{2}\)
Lの\protect\hyperlink{jus-de-veau-brun}{仔牛のフォン}を鍋で沸かし、セージ3
g、マジョラム1 g、ローズマリー1 g、バジル2 g、タイム1 g、ローリエの葉1
g、パセリの葉1つまみ、マッシュルームの切りくず25 gを投入する。蓋をして
25分間煎じる。こうして煎じた液体を漉す2分前に大粒のこしょう4個を加える。

布で漉し、\protect\hyperlink{sauce-demi-glace}{ソース・ドゥミグラス}7
dLに\protect\hyperlink{sauce-tomate}{トマトソース}3
dLを合わせたものに、上記で煎じた液体を、風味が際立つ程度に適量加える。\(\frac{3}{4}\)量まで煮詰め、布で漉す。仕上げにマデラ酒1
dLとトリュフエッセンス少々を加え、さらにカイエンヌで風味を引き締める。

\hypertarget{nota-sauce-tortue}{%
\subparagraph{【原注】}\label{nota-sauce-tortue}}

このソースはある程度まとまった量で作る必要がある。カイエンヌを使う指示があるからだ。それでも、カイエンヌはとても気をつけて量を加減する必要がある\footnote{フランス料理において(というよりも伝統的かつ一般的なフランス人にとって)、唐辛子の辛さは嫌われる傾向が非常に強い。}。

\hypertarget{sauce-venaison}{%
\subsubsection[ソース・ヴネゾン]{\texorpdfstring{ソース・ヴネゾン\footnote{Venaison(ヴネゾン)とはノロ鹿chevreuilや猪sanglierなどの大型ジビエのこと。なおニホンジカやエゾジカはcerf(セール)に分類され、フランス料理の食材としてはあまり高く評価されない傾向がある。}}{ソース・ヴネゾン}}\label{sauce-venaison}}

\frsub{Sauce Venaison}

\index{そーす@ソース!うねそん@---・ヴネゾン}
\index{うねそん@ヴネゾン!そーす@ソース・---}
\index{おおかたしひえ@大型ジビエ ⇒ ヴネゾン!そーす@ソース・ヴネゾン}
\index{sauce@sauce!venaison@--- Venaison}
\index{venaison@venaison!sauce@Sauce ---}

完全に仕上げた「\protect\hyperlink{sauce-poivrade-pour-gibier}{ジビエ用ソース・ポワヴラード}」\(\frac{3}{4}\)
Lに、\protect\hyperlink{gelee-de-groseilles-a}{グロゼイユのジュレ}大さじ3杯強を生クリーム1
dLで溶いてから加える。

グロゼイユのジュレと生クリームを加えるのは、鍋を火から外して、提供直前にすること。

\ldots{}\ldots{}大型ジビエ料理用。

\hypertarget{sauce-vin-rouge}{%
\subsubsection{赤ワインソース}\label{sauce-vin-rouge}}

\frsub{Sauce au Vin rouge}

\index{そーす@ソース!あかわいん@赤ワイン---}
\index{あかわいん@赤ワイン!そーす@---ソース}
\index{sauce@sauce!vin rouge@--- au Vin rouge}
\index{vin@vin!sauce rouge@Sauce au --- rouge}

「赤ワインソース」という場合、煮詰めてからブールマニエでとろみを付けるブルゴーニュ風の仕立てか、魚を煮るのに用いた赤ワインを使うことが特徴である「ソース・マトロット」のいずれかから派生したものなのは言うまでもない。もっとも、後者の場合はワインの風味は失われてしまっていてソースの水気と味付けの意味しか持っていないと言える。

両者どちらもまさしく「赤ワインソース」だが、\protect\hyperlink{sauce-bourguignonne}{ブルゴーニュ風ソース}と\protect\hyperlink{sauce-matelote}{ソース・マトロット}はそれぞれ作り方も用途も違うから別々の名称として、この「茶色い派生ソース」の節で説明した。

筆者としては、本当の「赤ワインソース」は以下のように作るものと考えている。

ごく細かく刻んだ標準的な\protect\hyperlink{mirepoix}{ミルポワ}125
gをバターで炒める。良質の赤ワイン \(\frac{1}{2}\)
Lを注ぐ。半量になるまで煮詰める。つぶしたにんにく1片、\protect\hyperlink{sauce-espagnole}{ソース・エスパニョル}7
\(\frac{1}{2}\) dLを加え、12〜
15分、火にかけて浮いてくる不純物を丁寧に取り除く\footnote{dépouiller
  デプイエ ≒ écumer エキュメ。}。

布で漉し、バター100
gとアンチョビエッセンス小さじ1杯、カイエンヌ1つまみを加えて仕上げる。

\ldots{}\ldots{}魚料理用ソース。

\hypertarget{sauce-zingara-a}{%
\subsubsection[ソース・ザンガラ
A]{\texorpdfstring{ソース・ザンガラ\footnote{もとの語形はzingaro
  ザンガロ、またはヂンガロ。ジプシー、ボヘミアンの意。料理ではパプリカ粉末やカイエンヌを用いたものに命名されることが多い。}
A}{ソース・ザンガラ A}}\label{sauce-zingara-a}}

\frsub{Sauce Zingara A}

\index{そーす@ソース!さんからa@---・ザンガラ A}
\index{さんから@ザンガラ!そーすa@ソース・--- A}
\index{しふしーふう@ジプシー風!そーすa@ソース・ザンガラ A}
\index{sauce@sauce!zingara a@--- Zingara A}
\index{zingara@Zingara!sauce a@Sauce --- A}

このソースは古典料理の\protect\hyperlink{garniture-zingara}{ガルニチュール・ザンガラ}とはまったく関係がない。むしろイギリス料理に由来し、本書でもイギリス風ソースの節において似たようなものをいくつか採り上げている。

ヴィネガー2 \(\frac{1}{2}\)
dLにエシャロットのみじん切り大さじ1杯を加えて半量になるまで煮詰める。\protect\hyperlink{jus-de-veau-lie}{茶色いジュ}7
dLを注ぎ、バターで揚げたパンの身160
gを加える。弱火で5〜6分間煮る。パセリのみじん切り大さじ1杯とレモン
\(\frac{1}{2}\) 個分の搾り汁を加えて仕上げる。

\hypertarget{sauce-zingara-b}{%
\subsubsection{ソース・ザンガラ B}\label{sauce-zingara-b}}

\frsub{Sauce Zingara B}

\index{そーす@ソース!さんからb@---・ザンガラ B}
\index{さんから@ザンガラ!そーすb@ソース・--- B}
\index{しふしーふう@ジプシー風!そーすb@ソース・ザンガラ B}
\index{sauce@sauce!zingara b@--- Zingara B}
\index{zingara@Zingara!sauce b@Sauce --- B}

白ワイン3 dLとマッシュルームの茹で汁3
dLを合わせて\(\frac{1}{3}\)量になるまで煮詰める。

\protect\hyperlink{sauce-demi-glace}{ソース・ドゥミグラス}4
dLと\protect\hyperlink{sauce-tomate}{トマトソース}2 \(\frac{1}{2}\)
dL、\protect\hyperlink{fonds-blanc}{白いフォン}1
dLを注ぐ。浮いてくる不純物を徹底的に取り除きながら5〜6分火にかける。

仕上げに、カイエンヌ1つまみで風味を引き締め、太さ1〜2 mmの千切りにした
\footnote{julienne
  (ジュリエーヌ)。日本語では「ジュリエンヌ」と言うことが多いが、「ジュリヤン」のように言う調理現場もある。}ハム(脂身のないところ)と赤く漬けた舌肉70
gおよびマッシュルーム 50 g、トリュフ 30 gを加える。

\ldots{}\ldots{}仔牛料理、鶏料理用。
\end{recette}\newpage
\href{未、原文対照チェック}{} \href{未、日本語表現校正}{}
\href{未、その他修正}{} \href{未、原稿最終校正}{}

\hypertarget{petites-sauces-blanches}{%
\section{ホワイト系の派生ソース}\label{petites-sauces-blanches}}

\frsec{Petites Sauces Blanches, Composées et de Réductions}

\index{そーす@ソース!ほわいとはせい@ホワイト系の派生---|(}
\index{sauce@sauce!petites blanches composees@Petites ---s Blanches Composées|(}
\begin{recette}
\hypertarget{sauce-albufera}{%
\subsubsection[ソース・アルビュフェラ]{\texorpdfstring{ソース・アルビュフェラ\footnote{ナポレオン軍の元帥、ルイ・ガブリエル・スーシェ
  Louis-Gabriel Suchet, duc d'Albufera
  (1770〜1826)のこと。スペイン戦役の際にそれまでの軍功を称えられ、ナポレオンが1812年にアルビュフェラ公爵位を新設して授けた。帝政期の英雄のひとりであり、アルビュフェラおよびスーシェの名を冠した料理がいくつかある。1814年に帝政が崩壊した後も軍務、政務に携わり、最終的にフランス貴族院議員の地位を得た。アルビュフェラ公爵位については、1815年7月24日の勅令においてに正式に抹消されている。このソースの特徴は赤ピーマン(パプリカ)を加熱してなめらかにすり潰し、バターに練り込んだものを使う点にあるが、どのような経緯でこのソースに赤ピーマンを用いるようになったのかは不明。ただし、このソースを合わせる\protect\hyperlink{poularde-albufera}{「肥鶏 アルビュフェラ」}は詰め物(ファルス)に米を用いるが、アルビュフェラは湖の周辺の湿地帯で米の生産がおこなわれているという点では一応の関連性が認められよう。なお、アルビュフェラはバレンシアの湖とそこに形成された潟であり、現在はバレンシア州のアルブフェーラ自然公園となっている。}}{ソース・アルビュフェラ}}\label{sauce-albufera}}

\frsub{Sauce Albuféra}

\index{そーす@ソース!あるひゆふえら@---・アルビュフェラ}
\index{あるひゆふえら@アルビュフェラ!そーす@ソース・---}
\index{sauce@sauce!albufera@--- Albuféra}
\index{albufera@Albuféra!sauce@Sauce ---}

\protect\hyperlink{sauce-supreme}{ソース・シュプレーム}1
Lあたりに、溶かしたブロンド色の\protect\hyperlink{glace-de-viande}{グラスドヴィアンド}2
dLと、標準的な分量比率で作った\protect\hyperlink{beurre-de-pimentos}{赤ピーマンバター}50
gを加える。

\ldots{}\ldots{}鶏など家禽のポシェ\footnote{ポシェは通常、沸騰させない程度の温度で茹でること、だが、ここで想定しているのは丸鶏をポシェしたもの。つまりは「仕立て」であり、前の注で触れた\protect\hyperlink{poularde-albufera}{「肥鶏 アルビュフェラ」}がこれに相当する。「仕立て」としての鶏のポシェは通常、中抜きした部分に詰め物(ファルス)をして手羽を脚を畳むようにしてまとめて糸で縫い(brider
  ブリデ)、さらに豚背脂のシートで包んでちょうどいい大きさの鍋に入れて、あらかじめ用意しておいた\protect\hyperlink{fonds-blanc}{白いフォン})
  が鶏にかぶる程度まで注ぐ。鍋を火にかけていったん沸騰したら、火を弱めるかオーブンに入れて、蓋をしてポシェの温度すなわち微沸騰を保つようにして加熱する。詳しくは\protect\hyperlink{les-poches}{第7章肉料理「ポシェ」の項}参照。}またはブレゼ\footnote{ポシェと同様に丸鶏をブレゼという「仕立て」に調理したものを想定しているので注意。肉料理の「仕立て」としてのブレゼについては\protect\hyperlink{les-braisages-de-viandes-blanches}{第7章肉料理「ブレゼ」の白身肉のブレゼ}参照。}にソースとして添える。

\atoaki{}

\hypertarget{sauce-americaine}{%
\subsubsection[ソース・アメリケーヌ]{\texorpdfstring{ソース・アメリケーヌ\footnote{オマール・アメリケーヌという料理の由来は諸説あるが、19世紀フランスの料理人ピエール・フレス
  Pierre Fraysse
  がアメリカで働いた後にパリで1853年に開いたレストラン「シェ・ピーターズ」でこの料理名で提供したというのが定説。ただし、1853年以前にレストラン「ボヌフォワ」に「ラングドック産オマール ソース・アメリケーヌ添え」というメニューあり、フレスはその料理に改変を加えたか、名前だけをシンプルに「アメリケーヌ」とした程度という説もある。かつては、オマールの主産地のひとつブルターニュ地方を意味する古い形容詞
  armoricain(e)
  アルモリカン、アルモリケーヌの音が変化した料理名だと主張されることもあったが、
  19世紀には南仏産が中心であったトマトを用いる点で矛盾が生じてしまう。いずれにしても、この料理名がフレスの店シェ・ピーターズを基点として広く知られるようになったことは事実。1867年のグフェ『料理の本』にはソース・エスパニョルをベースに白ワインとトマトで作るオマール・アメリケーヌのレシピが掲載されているので、比較的短期間で広まった料理なのは確か。また、オマールではないが、その20年程前に遡ってカレーム『19世紀フランス料理』には、「海亀のポタージュ アメリカ風」Potage
  de tortue à
  l'américaineおよびその派生型「海亀のソース アメリカ風」のレシピが掲載されている。レシピに先立って、カレームは「アメリカでも海亀のポタージュは本書のイギリス風海亀のポタージュと同様に、つまりロンドン風に調理するという。ところが、ボストンとニューヨークで暮したことのある人々から、アメリカ人は海亀のポタージュにうなぎのフィレを加えることを伝え聞いた。当然ながらイギリス風の海亀のポタージュとは異なる味わいのものとなる(t.1,
  pp.289-290」と述べている。ここではソースのほうの概要を見ておこう。皮を剥いた小さめのうなぎを筒切りにする。これをラグー鍋に入れてシャンパーニュを注ぐ。洗ったアンチョビのフィレとにんんく、玉ねぎ、薄切りにしたマッシュルーム、タイム、バジル、ローリエの葉、ローズマリー、マジョラム、サリエット、メース少々、粗く砕いたこしょう、カイエンヌ少々を加える。弱火にかけて煮込み、煮詰めていく。これを布で絞り漉す。ここにコンソメとソース・エスパニョルを加え、再度火にかけていい具合になるまで煮詰める。シャンパーニュをグラス\(\frac{1}{2}\)杯加えて布で漉す。提供直前にバター少々と鶏のグラス、レモン果汁を加える(t.3,
  pp.81-82)、というもの。}}{ソース・アメリケーヌ}}\label{sauce-americaine}}

\frsub{Sauce Américaine}

\index{そーす@ソース!あめりけーぬ@---・アメリケーヌ}
\index{あめりかん@アメリカン/アメリケーヌ!そーす@ソース・アメリケーヌ}
\index{sauce@sauce!americaine@--- américaine}
\index{americain@américain(e)!sauce americaine@sauce ---e}

このソースは\protect\hyperlink{homard-americaine}{オマール・アメリケーヌ}という料理そのものと言っていい(「魚料理」の章、甲殻類、\protect\hyperlink{homard-americaine}{オマール・アメリケーヌ}参照)。

このソースは通常、オマール\footnote{homard
  ロブスターのこと。なお高級料理では800〜900
  g程度の大きなものが好んで使用される。。}の身をガルニチュールとした魚料理に添えられる。オマールの身をやや斜めになるよう厚さ1
cm程度の輪切りにし\footnote{escalopper
  (エスカロペ)。エスカロップ、すなわち厚さ1〜2
  cm程度の円形に切ることだが、オマールの場合はやや斜めに切るようにして面積を大きくすることが一般的。ここで使用するオマールは900
  g〜1 kg程度のものを想定していることに注意。}、魚料理のガルニチュールとして供するわけだ。

\atoaki{}

\hypertarget{sauce-anchois}{%
\subsubsection{アンチョビソース}\label{sauce-anchois}}

\frsub{Sauce Anchois}

\index{そーす@ソース!あんちよひ@アンチョビ---}
\index{あんちよひ@アンチョビ!そーす@---ソース}
\index{sauce@sauce!anchois@--- Anchois}
\index{anchois@anchois!sauce anchois@Sauce ---}

\href{}{ノルマンディー風ソース}8
dLを、バターを加える前の段階まで作る。\href{}{アンチョビバター}125
gを混ぜ込む。アンチョビのフィレ50
gを洗い、よく水気を絞ってから小さなさいの目に切ったのを加えて仕上げる。

\ldots{}\ldots{}魚料理用。

\atoaki{}

\hypertarget{sauce-aurore}{%
\subsubsection[ソース・オーロール]{\texorpdfstring{ソース・オーロール\footnote{夜明けの光、曙光のこと。オーロラの意味もあるため、日本では「オーロラソース」と呼ばれることもあるが、マヨネーズとトマトケチャップを同量で混ぜ合わせたものもそう呼ばれることが多いので注意。なお、
  Sauce à l'aurore
  というほぼ同じ名称のものが1806年刊ヴィエアール『帝国料理の本』に掲載されているが、これはヴルテにレモン果汁とこしょう、ナツメグを加えたものを用意し、別に茹で卵の黄身を用意する。茹で卵の黄身を漉し器に圧し付けるようにして麺状に引き出す。提供直前に、ソースにこの黄身の麺を加える。ここからは決して沸騰させないこと、というもの(p.59)。麺状にした卵黄を朝の光の筋に見立てたもので、鍋で加えるか、ソース入れにソースを入れた上に載せるなどの方法も考えられるが、いずれにしてもヴィアールの時代(19世紀初頭)はフランス式サーヴィスつまり大きな食卓に何種もの料理を一度に並べるという方式だったために、このソースの見た目の美しさをある程度じっくりと食べ手は楽しむことが出来ただろう。その後の文献ではオドもカレームもこの名称のソースには触れておらず、デュボワとベルナールの『古典料理』(1867年)において、Sauce
  à
  l'Auroreとして、ベシャメルソースに煮詰めた仔牛のブロンドとマッシュルームの茹で汁、トマトソースを添加して、スライスしたマッシュルームを加えるというレシピが掲載されている(p.57)。初期のロシア式サービスにおいては、客に料理を最初に見せてまわり、その後に切り分けて供するという方式であったために、おそらくヴィアールの「ソース・アローロール」では一瞬で失なわれてしまったであろう美しさのポイントが、このようにソース色合いそのものに代えたことで、最後の食べ手の分を取り分けるまで美しさを維持できるようになった、つまりは初期のロシア式サービスの欠点を補うものとなったと考えられよう。なお、19世紀はトマトが食材として急激に普及、流行した時代であったこともこのソースの変化と関係があると思われる。}}{ソース・オーロール}}\label{sauce-aurore}}

\frsub{Sauce Aurore}

\index{そーす@ソース!おーろーる@---・オーロール}
\index{おーろーる@オーロール!そーす@ソース・---}
\index{sauce@sauce!aurore@--- Aurore}
\index{aurore@aurore!sauce@Sauce ---}

\protect\hyperlink{veloute}{ヴルテ}に真っ赤なトマトピュレを加えたもの。分量は、ヴルテが
\(\frac{3}{4}\)に対し、トマトピュレ
\(\frac{1}{4}\)とする。仕上げに、ソース1 Lあたり100 gのバターを加える。

\ldots{}\ldots{}卵料理、仔牛、仔羊肉の料理、鶏料理用。

\atoaki{}

\hypertarget{sauce-aurore-maigre}{%
\subsubsection{魚料理用ソース・オーロール}\label{sauce-aurore-maigre}}

\frsub{Sauce Aurore maigre}

\index{そーす@ソース!おーろーるさかなよう@魚料理用---・オーロール}
\index{おーろーる@オーロール!そーすさかな@魚料理用ソース・---}
\index{sauce@sauce!aurore maigre@--- Aurore maigre}
\index{aurore@aurore!sauce maigre@Sauce --- maigre}

\protect\hyperlink{veloute-de-poisson}{魚料理用ヴルテ}に、上記と同じ割合でトマトピュレを加える。ソース1
Lあたりバター125 gを加えて仕上げる。

\ldots{}\ldots{}魚料理用

\atoaki{}

\hypertarget{sauce-bavaroise}{%
\subsubsection{バイエルン風ソース}\label{sauce-bavaroise}}

\frsub{Sauce Bavaroise}

\index{そーす@ソース!はいえるんふう@バイエルン風---}
\index{はいえるんふう@バイエルン風!そーす@---ソース}
\index{sauce@sauce!bavaroise@--- Bavaroise}
\index{bavarois@bavarois(e)!sauce bavaroise@Sauce ---(e)}

ヴィネガー5
dLにタイムとローリエの葉少々とパセリの枝4本、大粒のこしょう7〜8個と、おろした\footnote{原文
  râpé \textless{} râpe
  ラープと呼ばれる器具を用いておろすが、日本のおろし金と目の大きさが違うので注意。多くの場合、マンドリーヌ
  mandrine と呼ばれる野菜用スライサーにこの機能が付属している。}レフォール\footnote{raifort
  西洋わさび、ホースラディッシュ。}大さじ2杯を加え、半量になるまで煮詰める。

この煮詰めた汁に卵黄6個を加え\footnote{卵黄を加える前に一度漉しておいたほうがいいだろう。}、\protect\hyperlink{sauce-hollandaise}{オランデーズソース}を作る要領で、バター400
gと大さじ1
\(\frac{1}{2}\)杯の水を少しずつ加えながら、ソースがしっかり乳化するまで混ぜていく。布で漉す。

\protect\hyperlink{beurre-d-ecrevisse}{エクルヴィスバター}100
gと泡立てた生クリーム大さじ2杯、さいの目に切ったエクルヴィス\footnote{ざりがにのこと。通常はヨーロッパザリガニécrevisse
  à pattes
  rougesエクルヴィスアパットルージュを指す。高級食材としてとても好まれている。現在は代用としてécrevisse
  de
  Californieエクルヴィスドカリフォルニ(ウチダザリガニ)が用いられることもある。日本在来のニホンザリガニや、外来種だが多く生息しているアメリカザリガニは通常、フランス料理には用いられない。いずれもジストマ(寄生虫)のリスクがあるため、生食は厳禁。}の尾の身を加えて仕上げる。

\ldots{}\ldots{}魚料理用のこのソースは、ムースのような仕上がりにすること。

\atoaki{}

\hypertarget{sauce-bearnaise}{%
\subsubsection[ソース・ベアルネーズ]{\texorpdfstring{ソース・ベアルネーズ\footnote{ベアルヌは旧地方名で、フランス南西部、現在のピレネー・アトランティック県のことを指すが、このソースはその地方とまったく関係がない。
  19世紀パリ郊外のレストランPavillon Henri
  IV(日本語に訳すと「アンリ4世亭」となろうか)が店名に掲げているアンリ4世がベアルヌのポー生まれであることにちなんで命名したソース名というのが定説。}}{ソース・ベアルネーズ}}\label{sauce-bearnaise}}

\frsub{Sauce Béarnaise}

\index{そーす@ソース!へあるねーす@---・ベアルネーズ}
\index{へあるぬふう@ベアルヌ風!そーす@ソース・---}
\index{へあるねーす@ベアルネーズ!そーす@ソース・---}
\index{sauce@sauce!bearnaise@--- Béarnaise}
\index{bearnais@béarnais(e)!sauce bearnaise@Sauce ---e}

白ワイン2 dLとエストラゴンヴィネガー2
dLに、エシャロットのみじん切り大さじ4杯、枝のままの粗く刻んだエストラゴン20
g、セルフイユ10 g、粗挽きこしょう5
g、塩1つまみを加えて、\(\frac{1}{3}\)量になるまで煮詰める。

煮詰まったら、数分間放置して温度を下げる。ここに卵黄6個を加え、弱火にかけて、生のバター(あるいはあらかじめ溶かしておいてもいい)500
gを加えて軽くホイップしながらなめらかになるよう混ぜる。

卵黄に徐々に火が通っていくことでソースにとろみが付くので、絶対に弱火で作業をすること。

バターを混ぜ込んだら、布で漉して味を調える。カイエンヌごく少量を加えて風味を引き締める。仕上げに、刻んだエストラゴン大さじ杯とセルフイユ大さじ\(\frac{1}{2}\)杯を加える。

\ldots{}\ldots{}牛、羊肉のグリル用。

\hypertarget{nota-sauce-bearnaise}{%
\subparagraph{【原注】}\label{nota-sauce-bearnaise}}

このソースを熱々で提供しようとは考えないこと。このソースは要するにバターで作ったマヨネーズなのだ。ほの温い程度で充分であり、もし熱くし過ぎてしまうと、ソースが分離してしまう。

そうなってしまったら、冷水少々を加えて泡立て器でホイップして元のあるべき状態に戻してやること。

\atoaki{}

\hypertarget{sauce-bearnaise-tomatee}{%
\subsubsection[トマト入りソース・ベアルネーズ/ソース・ショロン]{\texorpdfstring{トマト入りソース・ベアルネーズ/ソース・ショロン\footnote{19世紀後半、パリで有名レストラン「ヴォワザン」の料理長を務めたアレクサンドル・ショロン
  Alexandre Choron (1837〜1924)。自ら考案し、命名したという。}}{トマト入りソース・ベアルネーズ/ソース・ショロン}}\label{sauce-bearnaise-tomatee}}

\frsub{Sauce Béarnaise tomatée, dite Sauce Choron}

\index{そーす@ソース!へあるねーすとまといり@トマト入り---・ベアルネーズ}
\index{へあるぬふう@ベアルヌ風!そーすとまといり@トマト入りソース・ベアルネーズ}
\index{そーす@ソース!しよろん@---・ショロン}
\index{しよろん@ショロン!そーす@ソース・---}
\index{sauce@sauce!bearnaise tomatee@--- Béarnaise tomatée}
\index{bearnais@b\'earnais!sauce bearnaise tomatee@Sauce Béarnaise tomatée}
\index{sauce@sauce!choron@--- Choron}
\index{choron@Choron!sauce@Sauce ---}

ソース・ベアルネーズを上記のとおりに作るが、最後にセルフイユとエストラゴンのみじん切りは加えない。充分固めに作っておき、ソースの
\(\frac{1}{4}\)量の、充分に煮詰めたトマトピュレを加える。ソースの濃度が丁度いい具合になるよう注意すること。

\ldots{}\ldots{}\protect\hyperlink{tournedos-choron}{トゥルヌド・ショロン}、および他のさまざまな料理に添える。

\atoaki{}

\hypertarget{sauce-bearnaise-a-la-glace-de-viande}{%
\subsubsection[グラスドヴィアンド入りソース・ベアルネーズ/ソース・フォイヨ/ソース・ヴァロワ]{\texorpdfstring{グラスドヴィアンド入りソース・ベアルネーズ/ソース・フォイヨ/ソース・ヴァロワ\footnote{ソース・フォイヨの名称は、19世紀〜20世紀初頭にパリにあったレストランおよびそのオーナーシェフの名によるもの。このソースを使った「仔牛の背肉・フォイヨ」がスペシャリテだったという。ソース・ヴァロワについては、ヴァロワ王家およびヴァロワ公爵であったルイ・フィリップ(7月王政期のフランス国王。在位1830〜1848)にちなんだ名称。前出のフォイヨはレストランを開く以前、ルイ・フィリップに仕えていた。}}{グラスドヴィアンド入りソース・ベアルネーズ/ソース・フォイヨ/ソース・ヴァロワ}}\label{sauce-bearnaise-a-la-glace-de-viande}}

\frsub{Sauce Béarnaise à la glace de viande, dite Foyot, ou Valois}

\index{へあるぬふう@ベアルヌ風!そーすくらすとういあんといり@グラスドヴィアンド入りソース・ベアルネーズ}
\index{そーす@ソース!へあるねーすくらすとういあんといり@---・ベアルネーズ(グラス・ド・ヴィアンド入り)}
\index{そーす@ソース!ふおいよ@---・フォイヨ}
\index{ふおいよ@フォイヨ!そーす@ソース・---}
\index{そーす@ソース!うあろわ@---・ヴァロワ}
\index{うあろわ@ヴァロワ!そーす@ソース・---}
\index{sauce@sauce!bearnaise a la glace de viande@--- Béarnaise à la glace de viande}
\index{bearnais@b\'earnais!sauce bearnaise a la glace de viande@Sauce Béarnaise à la glace de viande}
\index{sauce@sauce!foyot@--- Foyot} \index{foyot@Foyot!sauce@Sauce ---}
\index{sauce@sauce!valois@--- Valois}
\index{valois@Valois!sauce@Sauce ---}

標準的な\protect\hyperlink{sauce-bearnaise}{ソース・ベアルネーズ}を上記の分量で、固めに作る。溶かした\protect\hyperlink{glace-de-viande}{グラスドヴィアンド}1
dLを少しずつ加えて仕上げる。

\ldots{}\ldots{}牛、羊肉のグリル用。

\atoaki{}

\hypertarget{sauce-bercy}{%
\subsubsection[ソース・ベルシー]{\texorpdfstring{ソース・ベルシー\footnote{パリ東部、セーヌ川左岸にある地名。かつては荷揚げ港があり、19世紀には小さなレストランが多く店を構えていたという。}}{ソース・ベルシー}}\label{sauce-bercy}}

\frsub{Sauce Bercy}

\index{そーす@ソース!へるしー@---・ベルシー}
\index{へるしー@ベルシー!そーす@ソース・---}
\index{sauce@sauce!bercy@--- Bercy} \index{bercy@Bercy!sauce@Sauce ---}

細かくみじん切りにしたエシャロット大さじ2杯をバターでさっと色付かないよう炒める。白ワイン2
\(\frac{1}{2}\)
dLと\protect\hyperlink{fumet-de-poisson}{魚のフュメ}か、このソースを合わせる魚の茹で汁2
\(\frac{1}{2}\) dLを注ぐ。

\(\frac{2}{3}\)量弱まで煮詰めたら、\protect\hyperlink{veloute-de-poisson}{ヴルテ}\(\frac{3}{4}\)
Lを加える。ひと煮立ちさせてから、鍋を火から外し、バター100
gとパセリのみじん切り大さじ1杯を加えて仕上げる。

\atoaki{}

\hypertarget{sauce-au-beurre}{%
\subsubsection[ソース・オ・ブール
/ソース・バタルド]{\texorpdfstring{ソース・オ・ブール
/ソース・バタルド\footnote{バタルドは「雑種の、中間の」の意。卵黄とバターだけでとろみを付ける\protect\hyperlink{sauce-hollandaise}{ソース・オランデーズ}と似てはいるが小麦粉も使うことからこの名が付いたと言われている。なお、パンのバタール
  bâtard
  も同じ語だが、細いバゲットと太いドゥーリーヴルの「中間」の太さとだからというのが通説。}}{ソース・オ・ブール /ソース・バタルド}}\label{sauce-au-beurre}}

\frsub{Sauce au Beurre, dite Sauce Bâtarde}

\index{そーす@ソース!ふーる@---・オ・ブール}
\index{はたー@バター!そーす@ソース・オ・ブール}
\index{そーす@ソース!はたると@---・バタルド}
\index{はたると@バタルド!そーす@ソース・---}
\index{sauce@sauce!beurre@--- au Beurre}
\index{beurre@beurre!sauce@Sauce au ---}
\index{sauce@sauce!batarde@--- Bâtarde}
\index{batard@bâtard!sauce@Sauce Bâtarde}

小麦粉45 gと溶かしバター45 gをよく混ぜ合わせ粘土状にする。そこに、7
gの塩を加えた熱湯7 \(\frac{1}{2}\)
dLを一気に注ぎ、泡立て器で勢いよく混ぜ合わせる。とろみ付け用の卵黄5個を生クリーム大さじ1
\(\frac{1}{2}\)杯でゆるめたものと、レモン汁少々を加える。

布で漉し、鍋を火から外して、良質なバター300 gを加えて仕上げる。

\ldots{}\ldots{}アスパラガスや、さまざまな茹でた魚\footnote{\protect\hyperlink{sauce-hachee-maigre}{魚料理用ソース・アシェ}訳注参照。}

\hypertarget{nota-sauce-au-beurre}{%
\subparagraph{【原注】}\label{nota-sauce-au-beurre}}

このソースはとろみを付けた後、湯煎にかけておき、提供直前にバターを加えるようにするといい。\footnote{本書には、日本でもかつて有名だった、エシャロットのみじん切りを加えたヴィネガーを煮詰めてバターを溶かし込んだ魚料理用ソース「ソース・ブールブラン」Sauce
  (au) Beurre blanc
  は収録されていない。このソース・ブールブランはナント地方やアンジュー地方で淡水魚アローズやブロシェに合わせる伝統的なソース。1890年頃にナント地方の女性料理人クレマンス・ルフーヴルが、ソース・ベアルネーズを作るつもりが誤って卵を加えるのを忘れてしまった結果として出来たものだとも言われている。}

\atoaki{}

\hypertarget{sauce-bonnefoy}{%
\subsubsection[ソース・ボヌフォワ/白ワインで作るボルドー風ソース]{\texorpdfstring{ソース・ボヌフォワ/白ワインで作るボルドー風ソース\footnote{ソース・ボヌフォワの名称は、19世紀中頃にあったレストランの名による。このレストランで考案されたソースだという説もある。}}{ソース・ボヌフォワ/白ワインで作るボルドー風ソース}}\label{sauce-bonnefoy}}

\frsub{Sauce Bonnefoy, ou Sauce Bordelaise au vin blanc}

\index{ほぬふおわ@ボヌフォワ!そーす@ソース・---}
\index{そーす@ソース!ほぬふおわ@---・ボヌフォワ}
\index{そーす@ソース!ほるとーふうしろわいん@ボルドー風--- (白)}
\index{ほるとーふう@ボルドー風!そーすしろ@---ソース(白)}
\index{sauce@sauce!bonnefoy@--- Bonnefoy}
\index{bonnefoy@Bonnefoy!sauce@Sauce ---}
\index{sauce@sauce!bordelaise vin blanc@--- Bordelaise au vin blanc}
\index{bordelais@bordelais!sauce vin blanc@Sauce Bordelaise au vin blanc}

ブラウン系の派生ソースの節で採り上げた、赤ワインを用いて作る\protect\hyperlink{sauce-bordelaise}{ボルドー風ソース}とまったく同じ作り方だが、赤ワインではなく、グラーヴかソテルヌの白ワインを用いる。また\protect\hyperlink{sauce-espagnole}{ソース・エスパニョル}ではなく\protect\hyperlink{veloute}{標準的なヴルテ}を使うこと。

このソースは仕上げに、みじん切りにしたエストラゴンを加える。

\ldots{}\ldots{}魚のグリル、白身肉のグリル用。

\atoaki{}

\hypertarget{sauce-bretonne-blanche}{%
\subsubsection{ブルターニュ風ソース}\label{sauce-bretonne-blanche}}

\frsub{Sauce Bretonne}

\index{そーす@ソース!ふるたーにゆふうしろ@ブルターニュ風---(ホワイト系)}
\index{ふるたーにゆふう@ブルターニュ風!そーすしろ@---ソース(ホワイト系)}
\index{sauce@sauce!bretonne blanche@--- Bretonne (blanche)}
\index{breton@breton!sauce blanche@Sauce Bretonne (blanche)}

長さ3〜5 cm位の、ごく細い千切り\footnote{julienne (ジュリエーヌ)。}にしたポワローの白い部分30
gとセロリの白い部分30 g、玉ねぎ30 g、マッシュルーム30
gをバターで完全に火が通るまで鍋に蓋をして弱火で蒸し煮する\footnote{étuver
  エチュヴェ。本来は油脂とごく少量の水分を加えて弱火で蒸し煮することだが、野菜については、バターだけを使う場合も多い。ほぼ同様の加熱方法に
  étouffer (エトゥフェ)がある。後者の原義は「窒息させる」。}。

\protect\hyperlink{veloute-de-poisson}{魚のヴルテ}\(\frac{3}{4}\)
Lを加え、しばらく弱火にかけて浮いてくる不純物を丁寧に取り除く\footnote{dépouiller
  デプイエ ≒ écumer エキュメ。}。生クリーム大さじ3杯とバター50
gを加えて仕上げる。

\atoaki{}

\hypertarget{sauce-canotiere}{%
\subsubsection[ソース・カノティエール]{\texorpdfstring{ソース・カノティエール\footnote{小舟の漕ぎ手、の意。}}{ソース・カノティエール}}\label{sauce-canotiere}}

\frsub{Sauce Canotière}

\index{そーす@ソース!かのていえーる@---・カノティエール}
\index{かのていえーる@カノティエール!そーす@ソース・---}
\index{sauce@sauce!canotiere@--- Canotière}
\index{canotier@canotier(ère)!sauce@Sauce Canotière}

淡水魚を煮るのに用いた、\protect\hyperlink{court-bouillon-b}{白ワイン入りクールブイヨン}を
\(\frac{1}{3}\)量に煮詰める。クールブイヨンにはしっかり香り付けしてあり塩はごく少量しか入っていないこと。

1 Lあたり80
gのブールマニエを加えてとろみを付ける。軽く煮立たせたら、鍋を火から外してバター150
gとカイエンヌごく少量を加えて仕上げる。

\ldots{}\ldots{}淡水魚のクールブイヨン煮用。

\hypertarget{nota-sauce-canotiere}{%
\subparagraph{【原注】}\label{nota-sauce-canotiere}}

バターでグラセした小玉ねぎと小ぶりのマッシュルームを加えると、「\protect\hyperlink{sauce-matelote-blanche}{白いソース・マトロット}」の代用となる。

\atoaki{}

\hypertarget{sauce-aux-capres}{%
\subsubsection{ケイパー入りソース}\label{sauce-aux-capres}}

\frsub{Sauce aux Câpres}

\index{そーす@ソース!けいはー@ケイパー入り---}
\index{けいはー@ケイパー!そーす@---入りソース}
\index{sauce@sauce!capres@--- aux Câpres}
\index{capre@câpre!sauce capres@Sauce aux Câpres}

上記の\protect\hyperlink{sauce-au-beurre}{ソース・オ・ブール}に、ソース1
Lあたり大さじ4 杯のケイパーを提供直前に加える。

\ldots{}\ldots{}いろいろな種類の魚を煮た料理に用いる。

\atoaki{}

\hypertarget{sauce-cardinal}{%
\subsubsection[ソース・カルディナル]{\texorpdfstring{ソース・カルディナル\footnote{カトリックの\ruby{枢機卿}{すうききよう}(カルディナル)の衣が伝統的に赤いものであること、およびオマールが「海の枢機卿」と呼ばれることに由来。}}{ソース・カルディナル}}\label{sauce-cardinal}}

\frsub{Sauce Cardinal}

\index{そーす@ソース!かるていなる@---・カルディナル}
\index{かるていなる@カルディナル!そーす@ソース・---}
\index{sauce@sauce!cardinal@--- Cardinal}
\index{cardinal@cardinal!sauce@Sauce ---}

\protect\hyperlink{sauce-bechamel}{ベシャメルソース}\(\frac{3}{4}\)
Lに、(1)\protect\hyperlink{fumet-de-poisson}{魚のフュメ}とトリュフエッセンスを同量ずつ合わせて
\(\frac{3}{4}\)量まで煮詰めたものを1 \(\frac{1}{2}\)
dL加える。(2)生クリーム 1 \(\frac{1}{2}\) dLを加える。

鍋を火から外し、真っ赤に作った\protect\hyperlink{beurre-de-homard}{オマールバター}を加え、カイエンヌごく少量で風味を引き締める。

\ldots{}\ldots{}魚料理用。

\atoaki{}

\hypertarget{sauce-aux-champignons-blanche}{%
\subsubsection{マッシュルーム入りソース}\label{sauce-aux-champignons-blanche}}

\frsub{Sauce aux Champignons}

\index{そーす@ソース!まつしゆるーむしろ@マッシュルーム---(ホワイト系)}
\index{まつしゆるーむ@マッシュルーム!そーすしろ@---ソース(ホワイト系)}
\index{sauce@sauce!champignonsblanche@--- aux Champignons (blanches)}
\index{champignon@champignon!sauce blanche@Sauce aux Champignons (blanche)}

マッシュルームを茹でた汁3 dLを
\(\frac{1}{3}\)量まで煮詰める。\protect\hyperlink{sauce-allemande}{ソース・アルマンド}\(\frac{3}{4}\)
Lを加え、数分間沸騰させる。あらかじめ\ruby{螺旋}{らせん}状に刻みを入れて整形\footnote{tourner
  トゥルネ。原義は「回す」。包丁を動かさずに材料の方を回すようにして切る、刻み目を入れることがこの用語の由来。マッシュルームの場合はその際に大量の切りくず(具体的には重量で15〜20%)が発生するので、それをソースなどの風味付けに利用する。}してから茹でておいた真っ白で小さなマッシュルーム100
gを加えて仕上げる。

\ldots{}\ldots{}鶏料理用。魚料理に添えることもある。魚料理に合わせる場合は、ソース・アルマンドではなく\protect\hyperlink{veloute-de-poisson}{魚料理用ヴルテ}を用いること。

\atoaki{}

\hypertarget{sauce-chantilly}{%
\subsubsection[ソース・シャンティイ]{\texorpdfstring{ソース・シャンティイ\footnote{料理および製菓では生クリームをホイップしたクレーム・シャンティイが有名だが、元来はパリ北方に位置する町の名。ここのシャンティイ城で17世紀に、歴史上主要なメートルドテルのひとりヴァテルが自害した事件は有名で小説化、映画化もされた。}}{ソース・シャンティイ}}\label{sauce-chantilly}}

\frsub{Sauce Chantilly}

\index{そーす@ソース!しやんていい@---・シャンティイ}
\index{しやんていい@シャンティイ!そーす@ソース・---}
\index{sauce@sauce!chantilly@--- Chantilly}
\index{chantilly@Chantilly!sauce@Sauce ---}

まれにこれを「ソース・シャンティイ」と呼ばれることもあるが、これは後述の「\protect\hyperlink{sauce-mousseline}{ソース・ムスリーヌ}」に他ならない\footnote{むしろ、初版から掲載されている冷製ソースの\protect\hyperlink{sauce-chantilly-froide}{ソース・シャンティイ}と混同しないよう留意すべきだろう。}。

\atoaki{}

\hypertarget{sauce-chateaubriand}{%
\subsubsection[ソース・シャトーブリヤン]{\texorpdfstring{ソース・シャトーブリヤン\footnote{料理において通常、シャトーブリヤンは牛フィレの中心部分を3
  cm程度の厚さに切ったものを指す。この名称の由来には主に2説あり、ひとつはフランスロマン主義文学の父と言われる小説家フランソワ・ルネ・シャトーブリヤン
  François René Chateaubriand
  (1768〜1848)の名を冠したというもの。ちなみにフランスロマン主義文学の母と呼ばれているのはスタール夫人Anne
  Louise Germaine de
  Staël(1766〜1817)。料理におけるシャトーブリヤンという名の由来のもうひとつの説は、ブルターニュ地方で畜産物の集積地であったシャトーブリヤン
  Châteaubriant
  という地名に由来するというもの。なお、本書の初版および第四版では
  Chateaubriandの綴り、第二版はChâteaubriantであり、第三版は
  Châteaubrian\textbf{d}という奇妙な綴りとなっている。}}{ソース・シャトーブリヤン}}\label{sauce-chateaubriand}}

\frsub{Sauce Chateaubriand}

\index{そーす@ソース!しゃとーふりやん@---・シャトーブリヤン}
\index{しゃとーふりやん@シャトーブリヤン! そーす@ソース・---}
\index{sauce@sauce!chateaubriand@--- Chateaubriand}
\index{chateaubriand@Chateaubriand!sauce@Sauce ---}

(仕上がり5 dL分)

白ワイン4
dLに、みじん切りにしたエシャロット4個分とタイム少々、ローリエの葉少々、マッシュルームの切りくず40
gを加え、\(\frac{1}{3}\)量になるまで煮詰める。

\protect\hyperlink{jus-de-veau-brun}{仔牛のジュ}\footnote{本書では「仔牛の茶色いジュ」のレシピは掲載されているが、仔牛の「白い」ジュについての言及はない。ここでは通常の仔牛の茶色いジュを用いればいい。また、\protect\hyperlink{sauce-colbert}{ソース・コルベール}の項(第二版で加えられた)で、\protect\hyperlink{beurre-colbert}{ブール・コルベール}とこのソースを比較するにあたり、このソースを「軽く仕上げたグラスドヴィアンドにバターとパセリのみじん切りを加えたもの」と述べている(\protect\hyperlink{sauce-colbert}{ソース・コルベール}本文参照)。このため、なぜこのソース・シャトーブリヤンが「ブラウン系の派生ソース」の節ではなく「ホワイト系の派生ソース」に分類されているのか疑問が残るところ。}4
dLを加え、半量になるまで煮詰める。布で漉し、鍋を火から外して、メートルドテルバター250
gと細かく刻んだエストラゴン小さじ \(\frac{1}{2}\)杯を加えて仕上げる。

\ldots{}\ldots{}牛、羊の赤身肉のグリル用。

\atoaki{}

\hypertarget{sauce-chaud-froid-blanche-ordinaire}{%
\subsubsection{白いソース・ショフロワ(標準)}\label{sauce-chaud-froid-blanche-ordinaire}}

\frsub{Sauce Chaud-froid blanche ordinaire}

\index{そーす@ソース!しよふろわしろ@白い---・ショフロワ(標準)}
\index{しよふろわ@ショフロワ!そーすしろ@白いソース---(標準)}
\index{sauce@sauce!chaud-froid blanche ordinaire@--- Chaud-froid blanche ordinaire}
\index{chaud-froid@chaud-froid!sauce blanche ordinaire@Sauce --- blanche ordinaire}

(仕上がり1
L分)\ldots{}\ldots{}\protect\hyperlink{veloute}{標準的なヴルテ}\(\frac{3}{4}\)
L、\protect\hyperlink{gelee-de-volaille}{鶏でとった白いジュレ}6〜7
dL、生クリーム\footnote{フランスの生クリームについては\protect\hyperlink{sauce-supreme}{ソース・シュプレーム}訳注参照。}3
dL。

厚手のソテー鍋にヴルテを入れる。強火にかけ、ヘラで混ぜながらジュレと用意した生クリーム
\(\frac{1}{3}\)量を少しずつ加えていく。

所定の分量にするには、\(\frac{2}{3}\)量くらいまで煮詰めることになる。

味見をして、固さを確認する。これを布で漉す\footnote{粘度の高いソースなどを布で漉す方法については、\protect\hyperlink{veloute}{ヴルテ}訳注参照。}。生クリームの残りを少しずつ加え、ゆっくり混ぜながら、ショフロワに仕立てる食材を覆うのにいい固さになるまで冷ましてやる。

\atoaki{}

\hypertarget{sauce-chaud-froid-blonde}{%
\subsubsection{ブロンドのソース・ショフロワ}\label{sauce-chaud-froid-blonde}}

\frsub{Sauce Chaud-froid blonde}

\index{そーす@ソース!しよふろわふろんと@ブロンドの---・ショフロワ}
\index{しよふろわ@ショフロワ!そーすふろんと@ブロンドのソース---}
\index{sauce@sauce!chaud-froid blonde@--- Chaud-froid blonde}
\index{chaud-froid@chaud-froid!sauce blonde@Sauce --- blonde}

上記と同様に作るが、ヴルテではなく\protect\hyperlink{sauce-allemande}{ソース・アルマンド}を用いる。また、生クリームの量は半分に減らすこと。

\atoaki{}

\hypertarget{sauce-chaud-froid-aurore}{%
\subsubsection[ソース・ショフロワ・オーロール]{\texorpdfstring{ソース・ショフロワ・オーロール\footnote{夜明け、曙光の意。}}{ソース・ショフロワ・オーロール}}\label{sauce-chaud-froid-aurore}}

\frsub{Sauce Chaud-froid Aurore}

\index{そーす@ソース!しよふろわおーろーる@---・ショフロワ・オーロール}
\index{しよふろわ@ショフロワ!そーすおーろーる@ソース・---・オーロール}
\index{おーろーる@オーロール!そーすしよふろわおーろーる@ソース・ショフロワ・---}
\index{sauce@sauce!chaud-froid aurore@--- Chaud-froid Aurore}
\index{chaud-froid@chaud-froid!sauce aurore@Sauce --- Aurore}
\index{aurore@aurore!sauce chaud-froid aurore@Sauce Chaud-froid ---}

標準的な\protect\hyperlink{sauce-chaud-froid-blanche-ordinaire}{白いソース・ショフロワ}を上記のとおり作る。そこに、真っ赤なトマトピュレを布で漉したもの
1 \(\frac{1}{2}\) dLとパプリカ粉末0.25
gを少量のコンソメで煎じた\footnote{infuser
  アンフュゼ。煮出す、煎じる、の意。}ものを加える。

\ldots{}\ldots{}鶏のショフロワ用。

\hypertarget{nota-sauce-chaud-froid-aurore}{%
\subparagraph{【原注】}\label{nota-sauce-chaud-froid-aurore}}

あまり鮮かな色にしたくない場合は、パプリカを煎じた汁は数滴だけ加えるにとどめるといい。

\atoaki{}

\hypertarget{sauce-choud-froid-vert-pre}{%
\subsubsection[ソース・ショフロワ・ヴェールプレ]{\texorpdfstring{ソース・ショフロワ・ヴェールプレ\footnote{緑の野原、草原、の意。}}{ソース・ショフロワ・ヴェールプレ}}\label{sauce-choud-froid-vert-pre}}

\frsub{Sauce Chaud-froid au Vert-pré}

\index{そーす@ソース!しよふろわうえーるふれ@---・ショフロワ・ヴェールプレ}
\index{しよふろわ@ショフロワ!そーすうえーるふれ@ソース・---・ヴェールプレ}
\index{うえーるふれ@ヴェールプレ!そーすしよふろわうえーるふれ@ソース・ショフロワ・---}
\index{sauce@sauce!chaud-froid vert-pre@--- Chaud-froid au Vert-pré}
\index{chaud-froid@chaud-froid!sauce vert-pre@Sauce --- au Vert-pré}
\index{vert-pre@vert-pré!sauce chaud-froid vert-pre@Sauce Chaud-froid au ---}

鍋に白ワイン2
dLを沸かし、セルフイユとエストラゴン、刻んだシブレット、刻んだパセリの葉を各1つまみずつ投入する。蓋をして火から外し、10分間煎じてから布で漉す。

最初に示したとおりの分量で\protect\hyperlink{sauce-chaud-froid-blanche-ordinaire}{標準的なソース・ショフロワ}を作り、煮詰めながら、上記の香草を煎じた液体を少しずつ混ぜ込む。この段階で1
Lになるまで煮詰めておくこと。

\protect\hyperlink{}{ほうれんそうから採った緑の色素}をソースに加え、\ul{ほんのり薄い緑色}にする。

この色素を加える際にはよく注意して、上で示したとおりの色合いになるよう少しずつ投入すること。

このソースは各種の鶏\footnote{日本語では鶏と一言で済ませるが、フランス語では
  poussin プサン(ひよこ、ひな鶏)、poulette
  プレット(若い雌鶏)、poulet プレ(若鶏)、poule
  プール(雌鶏)、poulet de grain プレドグラン(50〜70日の若鶏)、poulet
  reine
  プレレーヌ(若鶏と肥鶏の中間のサイズでソテーやローストにする)、poulet
  quatre quarts プレカトルカール(45日程で食用にする)、poularde
  プラルド(肥鶏、1.8 kg以上のものが多く、
  AOCを取得している産地もある)、chapon シャポン(去勢鶏、最大で 6kg
  程になるというが、肉質は雌鶏に近く、高級品とされている)、coq
  コック(雄鶏)などに細かく分類されている。}のショフロワ、とりわけ「\protect\hyperlink{}{ショフロワ・プランタニエ}」に用いる。

\atoaki{}

\hypertarget{sauce-chaud-froid-maigre}{%
\subsubsection{魚料理用ソース・ショフロワ}\label{sauce-chaud-froid-maigre}}

\frsub{Sauce Chaud-froid maigre}

\index{そーす@ソース!さかなりようりようしよふろわ@魚料理用---・ショフロワ}
\index{しよふろわ@ショフロワ!さかなりようりようそーす@魚料理用ソース---}
\index{sauce@sauce!chaud-froid maigre@--- Chaud-froid maigre}
\index{chaud-froid@chaud-froid!sauce maigre@Sauce --- maigre}

作り方の手順と分量は\protect\hyperlink{sauce-chaud-froid-blanche-ordinaire}{標準的なソース・ショフロワ}とまったく同じだが、以下の点を変更する。(1)通常の\protect\hyperlink{veloute}{ヴルテ}ではなく\protect\hyperlink{veloute-de-poisson}{魚料理用ヴルテ}を用いる。(2)\protect\hyperlink{}{鶏のジュレ}ではなく\protect\hyperlink{gelee-de-poisson-blanche}{白い魚のジュレ}を用いること。

\hypertarget{nota-sauce-chaud-froid-maigre}{%
\subparagraph{【原注】}\label{nota-sauce-chaud-froid-maigre}}

一般的に、このソースは魚のフィレやエスカロップ、甲殻類に\protect\hyperlink{mayonnaise-collee}{コーティング用マヨネーズ}の代わりとして用いることをお勧めする。コーティング用マヨネーズにはいろいろ不都合な点があり、そのうちの最大のものは、ゼラチンが溶けるにつれて油が浸み出してきてしまうことだ。こういう不都合はこの魚料理用ソース・ショフロワを使う場合には出てこない。このソースは風味も明確ですっきりしているからコーティング用マヨネーズよりも好ましいだろう。

\atoaki{}

\hypertarget{sacue-chivry}{%
\subsubsection[ソース・シヴリ]{\texorpdfstring{ソース・シヴリ\footnote{19世紀フランスの作家フレデリック・スリエ
  Frédéric Soulié (1800〜
  1847)の劇『ディアーヌ・ド・シヴリ』\emph{Diane de Chivry}
  (1838年)あるいは1897年に新聞「フィガロ」に掲載されたエルネスト・カペンデュの小説『ビビタパン』の登場人物名Chivryにちなんだか、あるいはまったく別の人物の名を冠したものかは不明。}}{ソース・シヴリ}}\label{sacue-chivry}}

\frsub{Sauce Chivry}

\index{そーす@ソース!しうり@---・シヴリ}
\index{しうり@シヴリ!そーす@ソース・---}
\index{sauce@sauce!chivry@--- Chivry}
\index{chivry@Chivry!sauce@Sauce ---}

白ワイン1 \(\frac{1}{2}\) dLに以下を各1つまみずつ投入する\footnote{明記されていないが、この時点で白ワインは沸かしておく。}\ldots{}\ldots{}セルフイユ、パセリ、エストラゴン、シブレット、時季が合えばサラダバーネット
\footnote{pumprenelle パンプルネル、和名ワレモコウ。}の若い葉。蓋をして鍋を火から外し、10分間煎じる\footnote{infuser
  アンフュゼ。}。布で絞るようにして漉す。

こうしてハーブ類を煎じた液体を、あらかじめ沸かしておいた\protect\hyperlink{veloute}{ヴルテ}\(\frac{3}{4}\)
Lに加える。火から外し、\protect\hyperlink{beurre-chivry}{ブール・シヴリ}100
gを加えて仕上げる(\protect\hyperlink{beurres-composes}{合わせバターの節}参照)。

\ldots{}\ldots{}ポシェ\footnote{pocher
  原則的には、沸騰しない程度の温度で加熱調理すること。この場合は、下処理した鶏一羽まるごとをぎりぎり入るくらいの大きさの鍋に入れて水あるいはクールブイヨンを用いてゆっくり火を通す調理を意味している(温度管理が難しい場合はオーブンを用いることもある)。}あるいは茹でた鶏の料理用。

\hypertarget{nota-sauce-chivry}{%
\subparagraph{【原注】}\label{nota-sauce-chivry}}

サラダバーネットは生育するにつれて苦味が強くなるの、必ず若いものを使うこと。

\atoaki{}

\hypertarget{sauce-choron}{%
\subsubsection{ソース・ショロン}\label{sauce-choron}}

\frsub{Sauce Choron}

\protect\hyperlink{sauce-bearnaise-tomatee}{トマト入りソース・ベアルネーズ}参照。

\atoaki{}

\hypertarget{sauce-creme}{%
\subsubsection{ソース・クレーム}\label{sauce-creme}}

\frsub{Sauce à la Crème}

\index{そーす@ソース!くれーむ@---・クレーム}
\index{くりーむ@クリーム!そーす@ソース・クレーム}
\index{sauce@sauce!creme@--- à la Crème}
\index{creme@crème!sauce@Sauce à la ---}

\protect\hyperlink{sauce-bechamel}{ベシャメルソース}1 Lに生クリーム2
dLを加えて、ヘラで混ぜながら強火で、全体量の\(\frac{3}{4}\)になるまで煮詰める。

布で漉す\footnote{粘度や濃度の高いソースを漉す方法については\protect\hyperlink{veloute}{ヴルテ}訳注参照。}。フレッシュなクレーム・ドゥーブル\footnote{乳酸醗酵させた濃度の高い生クリーム。詳しくは\protect\hyperlink{sauce-supreme}{ソース・シュプレーム}訳注参照。}2
\(\frac{1}{2}\) dLとレモン果汁半個分を少しずつ加えて仕上げる。

\ldots{}\ldots{}茹でた魚、野菜料理、鶏、卵料理用。

\atoaki{}

\hypertarget{sauce-aux-crevettes}{%
\subsubsection[ソース・クルヴェット]{\texorpdfstring{ソース・クルヴェット\footnote{小海老のこと。フランスでよく料理に用いられるのは生の状態で甲殻が灰色がかった小さめのcrevettes
  grises(クルヴェット・グリーズ)と、やや大きめでピンク色のcrevettes
  roses(クルヴェット・ローズ)。美味しい。ちなみに日本でよく食べられているブラックタイガーはフランス語にするとcrevette
  géante tigréeと言う。}}{ソース・クルヴェット}}\label{sauce-aux-crevettes}}

\frsub{Sauce aux Crevettes}

\index{そーす@ソース!くるうえつと@---・クルヴェット}
\index{くるうえつと@クルヴェット!そーす@ソース・---}
\index{sauce@sauce!crevette@--- aux Crevettes}
\index{crevette@crevette!sauce@Sauce aux Crevettes}

\protect\hyperlink{veloute-de-poisson}{魚料理用ヴルテ}または\protect\hyperlink{sauce-bechamel}{ベシャメルソース}1
Lに、生クリーム1 \(\frac{1}{2}\)
dLと\protect\hyperlink{fumet-de-poisson}{魚のフュメ}1 \(\frac{1}{2}\)
dLを加える。

火にかけて9
dLになるまで煮詰める。鍋を火から外し、\protect\hyperlink{}{ブール・ルージュ}25
g(ソース全体に淡いピンクの色合いを付けるのが目的)を足した\protect\hyperlink{}{クルヴェットバター}100
gを加える。殻を剥いたクルヴェットの尾の身大さじ3杯を加え、カイエンヌ1つまみで風味を引き締めて仕上げる。

\ldots{}\ldots{}魚料理およびある種の卵料理用。

\atoaki{}

\hypertarget{sauce-currie}{%
\subsubsection{カレーソース}\label{sauce-currie}}

\frsub{Sauce Currie}

\index{そーす@ソース!かれー@カレー---}
\index{かれー@カレー!そーす@---ソース}
\index{sauce@sauce!currie@---  Currie}
\index{currie@currie!sauce@Sauce ---}

以下の材料をバターで軽く色付くまで炒める\ldots{}\ldots{}玉ねぎ250
g、セロリ100 g、パセリの根\footnote{パセリには根パセリpersil
  tubéreux(ペルスィチュベルー)といって根が肥大する品種系統もある。平葉で、葉の香りはフランスで一般的なモスカールドタイプ(葉の縮れるタイプ)とやや異なる。イタリアンパセリのように用いることが可能。}30
g、これらはすべてやや厚めにスライスする。タイム1枝とローリエの葉少々、メース少々を加える。小麦粉50
gとカレー粉\footnote{カレーは植民地インドの料理としてイギリスに伝わり、18世紀にはC\&B
  社によって混合スパイスであるカレー粉が開発された。フランスはあまりインドやその他のカレーの食文化と接することもなかったために、こんにちでも「珍しい料理」の範疇にとどまっている。とはいえ、19世紀にインドからアンティル諸島のうちの英領地域に連れて来られたインド人たちがカレーを伝え、それが広まってフランス領アンティーユにおいてコロンボ
  colomboというカレーのバリエーションが成立した。コロンボはこんにちのフランスでも(インドのカレーとは別のものとして)比較的よく知られたものとなっている(少なくともcurry,
  currieという語よりは一般的認知度が高いと言えるだろう)。}小さじ1
杯弱を振り入れる。小麦粉が色付かない程度に炒めて火を通したら、\protect\hyperlink{}{白いコンソメ}
\(\frac{3}{4}\)
Lを注ぐ。沸騰したら、弱火にして約45分煮る。軽く押し絞るように布で漉す。ソースを温めて、浮いてきた油脂は取り除き
\footnote{dégraisser (デグレセ)。}、湯煎にかけておく。

\ldots{}\ldots{}魚料理、甲殻類、鶏、さまざまな卵料理に合わせる。

\hypertarget{nota-sauce-currie}{%
\subparagraph{【原注】}\label{nota-sauce-currie}}

ココナツミルクをソースに加えることもある。その場合、白いコンソメの
\(\frac{1}{4}\)量をココナツミルクに代えること。

\atoaki{}

\hypertarget{sauce-currie-indienne}{%
\subsubsection{インド風カレーソース}\label{sauce-currie-indienne}}

\frsub{Sauce Currie à l'Indienne}

\index{そーす@ソース!いんとふうかれー@インドカレー---}
\index{かれー@カレー!そーすいんとふう@インド---ソース}
\index{sauce@sauce!currie indienne@---  Currie à l'Indienne}
\index{currie@currie!sauce indienne@Sauce --- à l'Indienne}

みじん切り\footnote{原文ciseler
  シズレ。鋭利な刃物でみじん切りにすること、スライスすること。原義は「ハサミで切る」。なお、日本語でみじん切りに相当する用語にはhacherアシェもある(hache斧から派生した語)。後者は野菜の他、肉類を細かく刻む際にも用いられる。ミートチョッパーをフランス語ではhachoirアショワールと呼ぶ。}にした玉ねぎ1個と、パセリ、タイム、ローリエ、メース、シナモン各少々のブーケガルニを、バターとともに弱火にかけて色付かないよう蒸し煮する。

カレー粉3 gを振り入れ、ココナツミルク \(\frac{1}{2}\) Lを注ぐ。ヴルテ
\(\frac{1}{2}\)
Lを加える(ソースを肉料理に合わせるか、魚料理に合わせるかで、ヴルテも標準的なものを使うか、魚料理用を使うか決めること)。弱火で15分程煮る。布で漉し、生クリーム1
dLとレモン果汁少々を加えて仕上げる。

\hypertarget{nota-sauce-currie-indienne}{%
\subparagraph{【原注】}\label{nota-sauce-currie-indienne}}

ここで示した量のココナツミルクは、生のココヤシの実700 gをおろして、 4
\(\frac{1}{2}\)
dLの温めた牛乳で溶いて作る。それを布で強く絞って漉してから使うこと。

ココナツミルクがない場合には、同量のアーモンドミルクを用いてもいい。

インドの料理人によるこのソースの作り方はさまざまで、基本だけが同じというものだ。

だが、本来のレシピがあったところで、使い物にはならないだろう。インドのカレーは我が国の大多数にとっては我慢ならぬものだろうから。ここで記した作り方は、ヨーロッパ人の味覚を勘案したものなので、本来のものよりいい筈だ。

\atoaki{}

\hypertarget{sauce-diplomate}{%
\subsubsection[ソース・ディプロマット]{\texorpdfstring{ソース・ディプロマット\footnote{外交官風、の意。繊細で豪華な仕立ての料理に付けられる名称。}}{ソース・ディプロマット}}\label{sauce-diplomate}}

\frsub{Sauce Diplomate}

\index{そーす@ソース!ていふろまつと@---・ディプロマット}
\index{ていふろまつと@ディプロマット!そーす@ソース・---}
\index{sauce@sauce!diplomate@--- Diplomate}
\index{diplomat@diplomat(e)!sauce@Sauce ---e }

\ruby{既}{すで}に仕上げでおいた\protect\hyperlink{sauce-normande}{ノルマンディ風ソース}1
Lに、\protect\hyperlink{beurre-de-homard}{オマールバター}75 gを加える。

さいの目に切ったオマールの尾の身大さじ2杯と同様にさいの目に切ったトリュフ大さじ1杯を加えて仕上げる。

\ldots{}\ldots{}大きな魚一尾まるごとの\footnote{relevé
  (ルルヴェ)。17世紀〜19世紀前半ににスタイルとして完成したフランス式サービスでは、豪華な装飾を施した飾り台(socleソークル)に載せられ、皿の周囲を飾るようにガルニチュールが配され(borduresボルデュール)、主役である大きな塊肉や魚まるごと1尾の料理にはしばしば飾り串(hâteletアトレ)が刺してある、きわめて壮麗な大皿料理が置かれた。なお『料理の手引き』ではこうした仕立てについては時代に
  \ruby{似}{そぐ}わないものとして、ごく簡潔にしか説明されていないが、初版、第二版に付属している献立表、および第三版以降独立して出版された『メニューの本』にはルルヴェの語はしばしば見られる。}料理用。

\atoaki{}

\hypertarget{sauce-ecossaise}{%
\subsubsection{スコットランド風ソース}\label{sauce-ecossaise}}

\frsub{Sauce Ecossaise}

\index{そーす@ソース!すこつとらんとふう@スコットランド風---}
\index{すこつとらんとふう@スコットランド風!そーす@---ソース}
\index{sauce@sauce!ecossaise@--- Ecossaise}
\index{ecossais@écossais(e)!sauce@Sauce ---e}

上記の分量どおりに作った\protect\hyperlink{sauce-creme}{ソース・クレーム}9
dLに以下を加えて作る。1〜2
mmの細さに千切りにしたにんじん、セロリ、さやいんげんをバターを加えて鍋に蓋をして弱火で蒸し煮し\footnote{étuver
  エチュヴェ。}、\protect\hyperlink{}{白いコンソメ}に完全に浸したものを1
dL。

\noindent\ldots{}\ldots{}卵料理、鶏料理に添える。

\atoaki{}

\hypertarget{sauce-estragon-blanche}{%
\subsubsection[ソース・エストラゴン]{\texorpdfstring{ソース・エストラゴン\footnote{ヨモギ科のハーブ。詳しくは茶色い派生ソースの\protect\hyperlink{sauce-chasseur}{ソース・シャスール}訳注参照。}}{ソース・エストラゴン}}\label{sauce-estragon-blanche}}

\frsub{Sauce Estragon}

\index{そーす@ソース!えすとらこんしろ@---・エストラゴン(ホワイト系)}
\index{えすとらこん@エストラゴン!そーすしろ@ソース・--- (ホワイト系)}
\index{sauce@sauce!estragon blanche@--- Estragon (blanche)}
\index{estragon@estragon!sauce blanche@Sauce --- (blanche)}

エストラゴンの枝30 gを粗く刻み\footnote{concasser コンカセ。}、強火で下茹でする\footnote{blanchir
  ブランシール。}。水気をしっかりときり、エストラゴンをスプーンですり潰し、あらかじめ用意しておいた\protect\hyperlink{veloute}{ヴルテ}を大さじ4杯加える。これを布で漉す。こうして作ったエストラゴンのピュレを\protect\hyperlink{veloute-de-volaille}{鶏のヴルテ}または\protect\hyperlink{veloute-de-poisson}{魚料理用ヴルテ}1
Lに混ぜ込む。どちらのヴルテを使うから、合わせる料理によって決めること。味を調え、みじん切りにしたエストラゴン大さじ
\(\frac{1}{2}\)杯を加えて仕上げる。

\ldots{}\ldots{}卵料理、鶏肉料理、魚料理に合わせる。

\atoaki{}

\hypertarget{sauce-aux-fines-herbes-blanche}{%
\subsubsection{香草ソース}\label{sauce-aux-fines-herbes-blanche}}

\frsub{Sauce aux Fines Herbes}

\index{そーす@ソース!こうそうしろ@香草---(ホワイト系)}
\index{こうそう@香草!そーすしろ@---ソース(ホワイト系)}
\index{はーぶ@ハーブ ⇒ 香草!こうそうそーすしろ@香草ソース(ホワイト系)}
\index{sauce@sauce!fines herbes blanche@--- aux Fines Herbes (blanche)}
\index{fines herbes@fines herbes!sauce blanche@Sauce aux --- (blanche)}

(仕上がり5 dL分)

あらかじめ2種のうちどちらかの方法(\protect\hyperlink{sauce-vin-blanc}{白ワインソース}参照)で作っておいた\protect\hyperlink{sauce-vin-blanc}{白ワインソース}
\(\frac{1}{2}\)
Lに、\protect\hyperlink{beurre-d-echalote}{エシャロットバター}40
gと、パセリ、セルフイユ、エストラゴンのみじん切りを大さじ1
\(\frac{1}{2}\)杯加える。

\ldots{}\ldots{}魚料理用。

\atoaki{}

\hypertarget{sauce-foyot}{%
\subsubsection{ソース・フォイヨ}\label{sauce-foyot}}

\frsub{Sauce Foyot}

⇒
\protect\hyperlink{sauce-bearnaise-a-la-glace-de-viande}{グラスドヴィアンド入りソース・ベアルネーズ}参照。

\atoaki{}

\hypertarget{sauce-groseilles}{%
\subsubsection[ソース・グロゼイユ]{\texorpdfstring{ソース・グロゼイユ\footnote{日本語で「すぐりの実」のことだが、こんにちでは「黒すぐり」の方が一般的かも知れない。黒すぐりはフランス語では
  cassis カシスと呼ばれる。一般的なグロゼイユにはフサスグリと呼ばれる
  groseille rouge グロゼイユ・ルージュ(赤すぐり)とgroseille
  blancheグロゼイユ・ブランシュ(白すぐり)の2種があり、どちらもブドウのように房なりする。上記とは別に、このソースで用いられるgroseille
  à
  maquereauグロゼイヤマクロー(maquereauは鯖の意。日本では英語経由のグーズベリーまたはグースベリーの名称でも呼ばれることが多い。単に西洋すぐりとも呼ぶ)という比較的大粒で薄く縞模様の入る種類もある。これは通常は緑色だが、まれに紫色になる変種もあるという。いずれもフランスでは料理や菓子作りによく用いられる。}}{ソース・グロゼイユ}}\label{sauce-groseilles}}

\frsub{Sauce Groseilles}

\index{そーす@ソース!くろせいゆ@---・グロゼイユ}
\index{くろせいゆ@グロゼイユ!そーす@ソース・---}
\index{sauce@sauce!groseilles@--- Groseilles}
\index{groseille@groseille!sauce@Sauce ---s}

緑色の濃いグーズベリー500 gを銅の片手鍋で下茹でする。

5分間煮立てたら、水気をきって、粉砂糖大さじ3杯と白ワイン大さじ2〜3杯を加えて、完全に火をとおす。布で漉す。

こうして出来たピュレに、\protect\hyperlink{sauce-au-beurre}{ソース・オ・ブール}5
dLを加え、よく混ぜる。

\ldots{}\ldots{}このソースはグリルあるいはイギリス風\footnote{à
  l'anglaise
  (アラングレーズ)。通常は塩適量を加えた湯でボイルすることを指す。}に茹でた鯖によく合う。とはいえ、他の魚料理にも合わせてもいい。

\hypertarget{nota-sauce-groseiles}{%
\subparagraph{【原注】}\label{nota-sauce-groseiles}}

このソースは緑色の房なりのグロゼイユ\footnote{一般的なフサスグリであれば白系統の「未熟果」を用いるということと解釈される。}でも作ることが可能。

\atoaki{}

\hypertarget{sauce-hollandaise}{%
\subsubsection[オランデーズソース]{\texorpdfstring{オランデーズソース\footnote{ニューヨーク発祥の朝食メニューとして知られるエッグ・ベネディクト\emph{Egg
  Benedict}に必ず用いられることで有名なうえ、一般的には「バターで作るマヨネーズ」のイメージが強いかも知れない。実際のところは、ラ・ヴァレーヌ『フランス料理の本』(1651年)において「アスパラガスの白いソース添え」Asperges
  à la sauce
  blancheというレシピにおいて、このオランデーズソースの原型ともいうべきものが示されている。アスパラガスは固めに塩茹でする。「新鮮なバター、卵黄、塩、ナツメグ、ヴィネガー少々をよくかき混ぜる。ソースが滑らかになったら、アスパラガスに添えて供する(p.238)」。簡潔な記述だが、これがオランデーズソースの原型であることは間違いないだろう。おそらくはラ・ヴァレーヌ以前から存在していた可能性も否定できない。なお植物油を用いたマヨネーズが文献上で確認されるのが18世紀以降で、19世紀初頭から爆発的に流行し、広まったもの。また、マヨネーズについては、現代ヨーロッパにおいても卵黄ではなく全卵を用いて作るほうが多数を占めている点が異なることに注意。なお、オランデーズとは「オランダ風」の意だが、なぜこの名称となったのかについては不明な点が多い。また、2007年版の『ラルース・ガストロノミック』では、オランデーズソースを作る際には温度に注意することと、よくメッキされた銅鍋かステンレス製の鍋を用いる必要があり、アルミ製の鍋だと緑色に変色する可能性があることに注意を促している
  (p.455)。}}{オランデーズソース}}\label{sauce-hollandaise}}

\frsub{Sauce Hollandaise}

\index{そーす@ソース!おらんてーす@オランデーズ---}
\index{おらんてーす@オランデーズ!そーす@---ソース}
\index{おらんたふう@オランダ風!そーす@オランデーズソース}
\index{sauce@sauce!hollandaise@--- Hollandaise}
\index{hollandais@hollandais(e)!sauce@Sauce ---e}

大さじ4杯の水とヴィネガー大さじ2杯に、粗挽きこしょう1つまみと肌理の細かい塩1つまみを加えて、\(\frac{1}{3}\)量まで煮詰める。この鍋を熱源のそばか、湯煎にかける。

大さじ5杯の水と卵黄5個を加える。生のまま、あるいは溶かしたバター500 g
を加えながらしっかりホイップする。ホイップしている途中で、水を大さじ3〜
4杯、少量ずつ足してやる。水を足すのは、軽やかな仕上がりにするため。

レモンの搾り汁少々と必要なら塩を足して味を調え、布で漉す。

湯煎にかけておくが、ソースが分離しないように、温度は微温くしておく。

\ldots{}\ldots{}魚料理、野菜料理用。

\hypertarget{nota-sauce-hollandaise}{%
\subparagraph{【原注】}\label{nota-sauce-hollandaise}}

ヴィネガーを煮詰めて使うのは、いつも最高品質のものが使えるとはかぎらないからで、水は
\(\frac{1}{3}\)量まで減らしたほうがいい。ただし、煮詰める作業を完全に省いてしまわないこと。

\atoaki{}

\hypertarget{sauce-homard}{%
\subsubsection{ソース・オマール}\label{sauce-homard}}

\frsub{Sauce Homard}

\index{そーす@ソース!おまーる@---・オマール}
\index{おまーる@オマール!そーす@ソース・---}
\index{sauce@sauce!homard@--- Homard}
\index{homard@homard!sauce@Sauce ---}

\protect\hyperlink{veloute-de-poisson}{魚料理用ヴルテ}\(\frac{3}{4}\)
Lに、生クリーム1 \(\frac{1}{2}\)
dLと\protect\hyperlink{beurre-de-homard}{オマールバター}80
g、\protect\hyperlink{beurre-colorant-rouge}{赤いバター}40
gを加えて仕上げる。

\ldots{}\ldots{}魚料理用。

\hypertarget{nota-sauce-homard}{%
\subparagraph{【原注】}\label{nota-sauce-homard}}

このソースを魚1尾まるごとの料理に添える場合には、さいの目に切ったオマールの尾の身を大さじ3杯加える。

\atoaki{}

\hypertarget{sauce-hongroise}{%
\subsubsection[ハンガリー風ソース]{\texorpdfstring{ハンガリー風ソース\footnote{原書でも用いられている語paprikaパプリカはハンガリー語。唐辛子、ピーマンの仲間であり、16世紀以降17世紀にヨーロッパ全土に広まり、その土地ごとの風土に合わせて品種が多様化した。パプリカはとりわけ辛味成分をほとんど含んでいないのが特徴。ただし、ハンガリーの食文化において大きな役割を果すようになったのは19世紀以降になってからと言われている。}}{ハンガリー風ソース}}\label{sauce-hongroise}}

\frsub{Sauce Hongroise}

\index{そーす@ソース!はんかりーふう@ハンガリー風---}
\index{はんかりーふう@ハンガリー風!そーす@---ソース}
\index{sauce@sauce!hongroise@---  Hongroise}
\index{hongrois@hongrois(e)!sauce@Sauce ---e}

大きめの玉ねぎ1個のみじん切りをバターで色付かないよう強火で炒める。塩1
つまみとパプリカ粉末1 gで味付けする。

このソースを添える料理に合わせて\protect\hyperlink{veloute}{標準的なヴルテ}あるいは\protect\hyperlink{veloute-de-poisson}{魚料理用ヴルテ}
1 Lを加え、数分間軽く煮立てる。

布で漉し、バター100 gを加えて仕上げる。

このソースは淡いピンク色に仕上げるべきであり、その色を出しているのがパプリカ粉末だけによるものだということに注意。

\ldots{}\ldots{}仔羊や仔牛のノワゼット\footnote{noisette
  ロースの中心部分を円筒形に切り出して調理したもの。}にとりわけよく合う。卵料理、鶏料理、魚料理にも。

\atoaki{}

\hypertarget{sauce-aux-huitres}{%
\subsubsection{牡蠣入りソース}\label{sauce-aux-huitres}}

\frsub{Sauce aux Huîtres}

\index{そーす@ソース!かきいり@牡蠣入り---}
\index{かき@牡蠣!そーす@牡蠣入りソース}
\index{sauce@sauce!huitres@---  aux Huîtres}
\index{huitre@huître!sauce@Sauce aux Huîtres}

後述の\protect\hyperlink{sauce-normande}{ノルマンディ風ソース}に、ポシェ\footnote{pocher
  \textless{} poche
  ポシュ(ポケット)、からの派生語。ポーチドエッグを作る際に、ポケット状になるところからこの用語が定着した。沸騰しない程度の温度で加熱調理すること。}して周囲をきれいにした牡蠣の身を加えたもの。

\atoaki{}

\hypertarget{sauceindienne}{%
\subsubsection{インド風ソース}\label{sauceindienne}}

\frsub{Sauce Indienne}

\index{そーす@ソース!いんとふう@インド風---}
\index{いんとふう@インド風!そーす@---ソース}
\index{sauce@sauce!indienne@---  Indienne}
\index{indien@indien(ne)!sauce@Sauce ---ne}

⇒ \protect\hyperlink{sauce-currie-indienne}{インド風カレーソース}参照。

\atoaki{}

\hypertarget{sauce-ivoire}{%
\subsubsection[ソース・イヴォワール]{\texorpdfstring{ソース・イヴォワール\footnote{象牙、の意。}}{ソース・イヴォワール}}\label{sauce-ivoire}}

\frsub{Sauce Ivoire}

\index{そーす@ソース!いうおわーる@---・イヴォワール}
\index{いうおわーる@イヴォワール!そーす@ソース・---}
\index{sauce@sauce!ivoire@--- Ivoire}
\index{ivoire@ivoire!sauce@Sauce ---}

\protect\hyperlink{sauce-supreme}{ソース・シュプレーム}1
Lに、ブロンド色の\protect\hyperlink{glace-de-viande}{グラスドヴィアンド}大さじ3杯を加え、象牙のようなくすんだ色合いにする。

\ldots{}\ldots{}低めの温度でしっとり仕上がるよう茹でた\footnote{pocher
  (ポシェ)。}鶏に添える。

\atoaki{}

\hypertarget{sauce-joinville}{%
\subsubsection[ソース・ジョワンヴィル]{\texorpdfstring{ソース・ジョワンヴィル\footnote{19世紀、7月王政期の国王ルイ・フィリップの第3子、フランソワ・ドルレアン・ジョワンヴィル海軍大将(1818〜1900)のこと。エクルヴィスとクルヴェットを用いた料理に彼の名が冠されたものがいくつかある。}}{ソース・ジョワンヴィル}}\label{sauce-joinville}}

\frsub{Sauce Joinville}

\index{しよわんういる@ジョワンヴィル!そーす@ソース・---}
\index{そーす@ソース!しよわんういる@---・ジョワンヴィル}
\index{sauce@sauce!joinville@--- Joinville}
\index{joinville@Joinville!sauce@Sauce ---}

\protect\hyperlink{sauce-normande}{ノルマンディ風ソース}1
Lを、仕上げる直前の段階まで作る\footnote{すなわち、布で漉すところまで。}。\protect\hyperlink{beurre-d-ecrevisse}{エクルヴィスバター}60
gと\protect\hyperlink{}{クルヴェットバター}60 gを加えて仕上げる。

このソースを添える魚料理にガルニチュールが既にある場合は、これ以上は何も加えない。

ガルニチュールを伴なわない大きな茹でた魚\footnote{魚の場合は、クールブイヨンを用いてやや低めの温度で煮たもの。\protect\hyperlink{sauce-hachee-maigre}{魚料理用ソース・アシェ}訳注参照。}に添える場合には、細さ1〜
2 mmの千切りにした真黒なトリュフを大さじ2杯加えること。

\hypertarget{nota-sauce-joinville}{%
\subparagraph{【原注】}\label{nota-sauce-joinville}}

同様のソースはいろいろあるが、最後の仕上げにエクルヴィスバターとクルヴェットバターを組み合わせて加える点がソース・ジョワンビルが他のものと違うポイント。

\atoaki{}

\hypertarget{sauce-laguipiere}{%
\subsubsection[ソース・ラギピエール]{\texorpdfstring{ソース・ラギピエール\footnote{18世紀末〜19世紀初頭にかけて活躍したフランスを代表する料理人の名(?〜1812)。はじめコンデ公に仕え、革命時にコンデ公の亡命にも随行したが、後にフランスに帰国し、ナポレオン\ruby{麾下}{きか}に入った。ナポレオン自身は食に無頓着であったが、直接的にはミュラ元帥のもとで料理長として活躍した。タレーランに仕えていたアントナン・カレームは2年程の期間であったが、ラギピエールとともに宴席の仕事に携わり、生涯を通して師と仰ぐ程に尊敬してやまなかった。当然だが料理においてカレームはラギピエールから大きく影響を受け、そのことを後年、数冊の自著で明記している。ラギピエール自身はミュラ元帥に従ってロシア戦線に赴き、その撤退の途中、極寒の地で凍死した。カレームは1828年刊『パリ風の料理』の冒頭2ページを「ラギピエールの想い出に」と題し、とても力強い文体でその死を悼んだ。}}{ソース・ラギピエール}}\label{sauce-laguipiere}}

\frsub{Sauce Laguipière}

\index{らきひえーる@ラギピエール!そーす@ソース・---}
\index{そーす@ソース!らきひえーる@---・ラギピエール}
\index{sauce@sauce!laguipiere@--- Laguipière}
\index{laguipiere@Laguipière!sauce@Sauce ---}

上述のとおりに作った\protect\hyperlink{sauce-au-beurre}{ソース・オ・ブール}
1
Lに、レモン1個の搾り汁と\protect\hyperlink{glace-de-poisson}{魚のグラス}またはそれと同等に煮詰めた\protect\hyperlink{fumet-de-poisson}{魚のフュメ}大さじ4杯を加える。

このソースは茹でた魚に添える。

\hypertarget{nota-sauce-laguipiere}{%
\subparagraph{【原注】}\label{nota-sauce-laguipiere}}

カレームが考案したこのソースのレシピに、本書で加えた変更点はただ1箇所のみ、\protect\hyperlink{glace-de-volaille}{鶏のグラス}ではなく魚のグラスに代えたことだけだ。さらに言うと、このソースはカレームによって「ソース・オ・ブール ラギピエール風」と名付けられたものだ\footnote{カレームの未完の大著『19世紀フランス料理』第3巻に、このソースのレシピが掲載されている。少し長くなるが引用すると「ラグー用片手鍋に、
  \ul{魚料理用グランドソース}の章で示したソース・オ・ブールをレードル
  1杯入れる。ここに上等のコンソメ大さじ1杯か鶏のグラス少々を加える。塩1つまみ、ナツメグ少々、良質のヴィネガーまたはレモン果汁適量を加える。数秒間煮立たせ、上等なバターをたっぷり加えてから供する。(中略)ソースに火を通してからバターを加えるというこの方法によって、なめらかな口あたりで、油っぽくならない仕上がりになる。だからこそ私はこのソース・オ・ブールをグランドソースに分類しなかったのだし、バターを加える派生ソースにおいてこれは重要なことだからだ。それは魚料理用ソースについても同様のことだ(pp.117-118)」。このレシピにおいて、カレームの表現には矛盾がある。「魚用グランドソースの章で示した」とあるのに「グランドソースに分類しなかった」となっていることだ。実際、ソース・オ・ブールそれ自体はこの「ラギピエール風」の直前にある。さて、このソースが「ラギピエール風」であることの理由だが、同じ巻の「魚料理用ソース・エスパニョル」の説明の冒頭において、ラギピエールから聞いた話として、四旬節の期間(小斉=肉断ちをする慣習がカトリックに根強くあった)に、魚料理用のソースにコンソメや仔牛のブロンドのジュを混ぜている修道士料理人がいたの、と述べている。それなら美味しくて当然だろう、とカレームが問うと、ラギピエールは「そうやって作った料理は、通常の肉を食べていい時の料理とは違うものであり、かといって肉断ちの料理でもない、まさに中間のものだ。その判定は天のみぞ知るところだろう。結局のところ、修道士たちは元気に暮していたのだから、それは正しかったのだよ」と煙に巻いたという。カトリックの習慣としての小斉=肉断ちのための魚料理用ソースに、肉由来である鶏のグラスもしくはコンソメを加えるというところが、ラギピエール風と名付けた
  \ruby{所以}{ゆえん}であり、まさにこれこそがソース・ラギピエールの重要なポイントと考えられる。『料理の手引き』においてこのレシピを担当した執筆者はこのエピソードを読んでいなかったのだろうか?
  あるいは何らかの誤解ゆえに改変をしたのか、ラギピエール風の\ruby{所以}{ゆ
  えん}である鶏のグラス、コンソメを用いるべきところを、魚のグラスに代えてしまい、このソース名の由来を換骨奪胎してしまう結果となっている。本書の初版において、原注がその文体から、エスコフィエの手になるものか、あるいは聞き書きしたコメントであることはほぼ明らかなので、なぜエスコフィエがこの点を見逃したか、あるいは許容したのかは非常に興味深い。ところで、カレームが、バターを仕上げの際に加えるということ、いわゆるブールモンテmonter
  au
  beurreによってソースの口あたりをなめらかなものにし、色艶をよくするということをことさらに言及しているの点もまた、注目に値すべきだろう。}。

\atoaki{}

\hypertarget{sauce-livonienne}{%
\subsubsection[リヴォニア風ソース]{\texorpdfstring{リヴォニア風ソース\footnote{現在のラトビア東北部からエストニア南部にかけての古い地域名、いわゆるバルト三国の一地域と捉えていい。本書執筆時にはロシア帝国の一部となっていた。なお、料理名に冠される地名のうちの少からずのものに明確な由来のないのと同様に、このソースについても名称の由来は不明。}}{リヴォニア風ソース}}\label{sauce-livonienne}}

\frsub{Sauce Livonienne}

\index{そーす@ソース!りうおにあふう@リヴォニア風---}
\index{りうおにあふう@リヴォニア風!そーす@---ソース}
\index{sauce@sauce!livonienne@---  Livonienne}
\index{livonien@livonien(ne)!sauce@Sauce ---ne}

バターを加えて仕上げた\footnote{monter au beurre バターでモンテする。}\protect\hyperlink{veloute-de-poisson}{魚のフュメで作ったヴルテ}1
Lに、1〜2 mmの細さで長さ3〜4 cmの千切り\footnote{julienne ジュリエンヌ}にしたにんじん、セロリ、マッシュリューム、玉ねぎをあらかじめバターを加えて弱火で蒸し煮\footnote{étuver
  au beurre バターでエチュヴェする。}しておいたもの100
gを加える。最後に、1〜2
mmの細さのトリュフの千切りと粗く刻んだパセリを加える。\ldots{}\ldots{}その後、味を調えること。

\ldots{}\ldots{}このソースは、トラウト、サーモン、舌びらめ、チュルボタン\footnote{turbotin
  \textless{} turbo チュルボ。鰈の近縁種。}、バルビュ\footnote{barbue
  鰈の近縁種。}のような魚によく合う。

\atoaki{}

\hypertarget{sauce-maltaise}{%
\subsubsection[マルタ風ソース]{\texorpdfstring{マルタ風ソース\footnote{シチリアの南方に位置するマルタ島を中心とした国、マルタはオレンジをはじめとした柑橘類の産地であり、とりわけ19世紀にはマルタ産のブラッドオレンジが人気であった。一例としてバルザックの小説『二人の若妻の手記』において、つわりに苦しむ妻のために夫がマルセイユの街で「マルタ産、ポルトガル産、コルシカ産のオレンジを買い求めた」
  (p.312)と書かれている。}}{マルタ風ソース}}\label{sauce-maltaise}}

\frsub{Sauce Maltaise}

\index{そーす@ソース!まるたふう@マルタ風---}
\index{まるたふう@マルタ風!そーす@---ソース}
\index{maltais@maltais(e)!sauce@Sauce ---e}
\index{sauce@sauce!maltaise@--- Maltaise}

前述のとおりに、\protect\hyperlink{sauce-hollandaise}{ソース・オランデーズ}を作り、提供直前に、\textbf{ブラッドオレンジ}2個の搾り汁を加える。ブラッドオレンジを用いないとこのソースは成立しないので注意。オレンジの皮の表面をおろしたもの\footnote{zeste
  ゼスト。} 1つまみを加えて仕上げる。

\ldots{}\ldots{}アスパラガスに添える。

\atoaki{}

\hypertarget{sauce-mariniere}{%
\subsubsection[ソース・マリニエール]{\texorpdfstring{ソース・マリニエール\footnote{marinier/marinière
  \textless{} mare
  ラテン語「海」から派生した語。貝や魚を白ワインで煮た料理にも付けられる名称。}}{ソース・マリニエール}}\label{sauce-mariniere}}

\frsub{Sauce Marinière}

\index{まりにえーる@マリニエール!そーす@ソース・---}
\index{そーす@ソース!まりにえーる@---・マリニエール}
\index{sauce@sauce!mariniere@--- Marinière}
\index{marinier@marinier(ère)!sauce@Sauce Marinière}

\protect\hyperlink{sauce-bercy}{ソース・ベルシー}を本書で示したとおりの分量で用意する。これにムール貝の茹で汁を詰めたもの大さじ3〜4杯を加え、卵黄6個でとろみを付ける\footnote{卵黄でとろみ付けをする場合、あらかじめ生クリームあるいは茹で汁などで乳化させてからよく混ぜながら加えるのであれば、必ずしも弱火でなくても問題ない。ただし、沸騰状態だと滑かに仕上がらないリスクが残るので、ある程度は弱火にした方がいいだろう。}。

\ldots{}\ldots{}ムール貝の料理専用。

\atoaki{}

\hypertarget{sauce-matelote-blanche}{%
\subsubsection[白いソース・マトロット]{\texorpdfstring{白いソース・マトロット\footnote{水夫風、船員風、の意。}}{白いソース・マトロット}}\label{sauce-matelote-blanche}}

\frsub{Sauce Matelote blanche}

\index{そーす@ソース!まとろつとしろ@白い---・マトロット}
\index{まとろつと@マトロット!そーすしろ@白いソース・---}
\index{sauce@sauce!matelote blanche@--- Matelote blanche}
\index{matelote@matelote!sauce@Sauce --- blanche}

白ワインで作った魚のクールブイヨン3
dLにフレッシュなマッシュルームの切りくず\footnote{料理、ガルニチュールとして供するマッシュルームは、トゥルネといって螺旋(らせん)状に切り込みを入れて装飾するのが一般的。その下ごしらえの際に大量のマッシュルームの切りくず(おおむねに重量比で15〜20%)が出るのを利用する。}25
gを加えて \(\frac{1}{3}\)量まで煮詰める。

\protect\hyperlink{veloute-de-poisson}{魚料理用ヴルテ}8
dLを加える。数分間煮立たせる。布で漉し、バター150 gを加える。

カイエンヌ\footnote{cayenne
  唐辛子の1品種。日本で一般的なカエンペッパーよりは辛さがマイルドで風味も異なる。}ごく少量で風味を引き締める。

ガルニチュールとして、下茹でしてからバターで色艶よく炒めた\footnote{glacer
  au
  beurre(グラセオブール)。バターでグラセする、と表現する調理現場も多い。glace
  グラス(鏡)が語源であるため、本来は「光沢を出させる、照りをつける」の意だが、食材や料理によってその手法はさまざま。にんじんや小玉ねぎの場合にはあらかじめ下茹でしておく必要がある。}小玉ねぎ20個と、あらかじめ茹でておいた小さな白いマッシュルーム\footnote{これを用意している段階で、上述のトゥルネを行なう。常識的なこととして明記されていないことに注意。この作業の結果、ソースを作る際に魚の茹で汁(クールブイヨン)に加えるマッシュルームの切りくずが発生する。}20個を加える。

\atoaki{}

\hypertarget{sauce-mornay}{%
\subsubsection[ソース・モルネー]{\texorpdfstring{ソース・モルネー\footnote{19世紀中頃にパリのレストラン、デュランの料理長ジョゼフ・ヴォワロンが創案したと言われている。モルネーは人名だが、具体的に誰を指しているかについては諸説ある。}}{ソース・モルネー}}\label{sauce-mornay}}

\frsub{Sauce Mornay}

\index{もるねー@モルネー!そーす@---ソース}
\index{そーす@ソース!もるねー@---・モルネー}
\index{sauce@sauce!mornay@--- Mornay}
\index{mornay@Mornay!sauce@Sauce ---}

\protect\hyperlink{sauce-bechamel}{ベシャメルソース}1
Lに、このソースを合わせる魚の茹で汁 2
dLを加え、\(\frac{2}{3}\)量程に煮詰める\footnote{初版ではこの煮詰める作業はなく「固めに作ったベシャメルソース1
  L に対し、魚の茹で汁2 dLを加える」となっている。}。おろした\footnote{râper
  (ラペ) \textless{}
  râpe(ラープ)という器具を用いておろすこと。パルメザン(パルミジャーノ)は硬質チーズなので一般的な半筒形のチーズおろし器でいいが、グリュイエールは比較的軟質なので、より目の粗い器具(例えばマンドリーヌに付属している機能のうち、にんじんをおろす際に使う部分など)を用いるといい。}グリュイエールチーズ50
gとパルメザンチーズ50
gを加える。少しの間、火にかけたままにしてよく混ぜ、チーズを完全に溶かし込む。バター100
gを加えて仕上げる\footnote{monter au beurre
  モンテオブール。バターでモンテする、と表現することも多い。}。

\hypertarget{nota-sauce-mornay}{%
\subparagraph{【原注】}\label{nota-sauce-mornay}}

魚以外の料理に合わせる場合\footnote{例えば茹でた野菜などにかけて、サラマンダー(強力な上火だけのオーブンの一種)に入れて軽く焦げ目を付け、グラタンにするようなケースも多い。}も作り方はまったく同じだが、魚の茹で汁は加えない。

\atoaki{}

\hypertarget{sauce-mousseline}{%
\subsubsection[ソース・ムスリーヌ/ソース・シャンティイ]{\texorpdfstring{ソース・ムスリーヌ/ソース・シャンティイ\footnote{mousseline
  \textless{} mousse ムース。-ine
  は「小さい」を意味する接尾辞。その前にLの文字が入るのは、mousseの語源がメソポタミアの都市Mossoul
  (モスリン布の生産地だった)であることによる。シャンティイの名称で呼ばれることがあるのは、固く立てた生クリームすなわち
  crème
  Chantilly(クレーム・シャンティイ)を加えるところから。\protect\hyperlink{sauce-chantilly}{ソース・シャンティイ}訳注も参照。}}{ソース・ムスリーヌ/ソース・シャンティイ}}\label{sauce-mousseline}}

\frsub{Sauce Mousseline, dite Sauce Chantilly}

\index{むすりーぬ@ムスリーヌ!そーす@ソース・---}
\index{そーす@ソース!むすりーぬ@---・ムスリーヌ}
\index{sauce@sauce!mousseine@--- Mousseline}
\index{mousseline@mousseline!sauce@Sauce ---}

前述のとおりの分量と作り方で\protect\hyperlink{sauce-hollandaise}{オランデーズソース}を用意する(\protect\hyperlink{sauce-hollandaise}{オランデーズソース}参照)。

提供直前に、固く泡立てた生クリーム大さじ4杯\footnote{大さじ1杯 = 15
  ccという考えにとらわれないよう注意。この計量単位は日本で戦後普及したものに過ぎず、本書においては文字通りに「大きなスプーンで4杯」という大雑把な単位として考える必要がある。このソースの場合は「固く泡立てた生クリームを適量」と読み替えてもいいだろう。名称どおりに滑らかでふんわりとした口あたりに仕上げるのがポイント。}をソースに混ぜ込む。

\ldots{}\ldots{}このソースは、茹でた魚や、アスパラガス、カルドン\footnote{cardon
  アーティチョークの近縁種で、アーティチョークが開花前の蕾を食用とするのに対し、カルドンは軟白させた茎葉を食用とする。フランスではトゥーレーヌ地方産が有名。草丈1.5
  m位まで成長させた株を紐で束ねて軟白する。厳冬期は株元から刈り取って小屋などで保管するのが伝統的な手法。イタリア北部ピエモンテでは株を倒してその上に土を被せて軟白するというユニークな方法で栽培するcardo
  gobboカルドゴッボもよく知られている。}、セロリ\footnote{セロリには緑の濃い品種系統と、やや緑が薄く、中心部が自然に軟白されたようになる系統がある。野菜料理として用いられるのは主として後者の芯に近い、自然に軟白された部分。coeur
  de céleri
  クールドセルリと呼ぶ。前者については、もっぱら香味野菜としてフォンやポタージュ、煮込み料理などに用いられる。このタイプは風味に癖があるため、生食にはあまり適していない。}に添える。

\atoaki{}

\hypertarget{sauce-mousseuse}{%
\subsubsection[ソース・ムスーズ]{\texorpdfstring{ソース・ムスーズ\footnote{細かく泡立った、の意。なお、シャンパーニュのようなvin
  mousseux ヴァン・ムスー(発泡ワイン)のムスーは同じ語の男性形.}}{ソース・ムスーズ}}\label{sauce-mousseuse}}

\frsub{Sauce Mousseuse}

\index{むすー@ムスー(ズ)!そーす@ソース・ムスーズ}
\index{そーす@ソース!むすーす@---・ムスーズ}
\index{sauce@sauce!mousseuse@--- Mousseuse}
\index{mousseux@mousseux/mousseuse!sauce@Sauce Mousseuse}

沸騰した湯の中に、小さめのソテー鍋を入れて熱し、水気をよく拭き取る。このソテー鍋に、あらかじめ充分に柔らかくしておいたバター500
gを入れる。塩8 gを加え、泡立て器でしっかり混ぜながら、レモン
\(\frac{1}{4}\)個分の搾り汁と冷水4 dLを少しずつ加える。

最後に、固く泡立てた生クリーム大さじ4杯を混ぜ込む。

このレシピは、ソースに分類してはいるが、むしろ合わせバターというべきものだ。茹でた魚\footnote{クールブイヨンなどを用いてやや低温で煮た魚、の意。\protect\hyperlink{sauce-hachee-maigre}{魚料理用ソース・アシェ}訳注参照。}に合わせる。

茹でた魚から伝わる熱だけでバターは充分に溶けるので、見た目も風味も溶かしバターをソースにするよりずっといいものだ。

\atoaki{}

\hypertarget{sauce-moutarde}{%
\subsubsection[ソース・ムタルド]{\texorpdfstring{ソース・ムタルド\footnote{マスタードのこと。マスタードソースと呼んでもいいが、アメリカ風の印象を与えるかも知れない。}}{ソース・ムタルド}}\label{sauce-moutarde}}

\frsub{Sauce Moutarde}

\index{そーす@ソース!むたると@---・ムタルド}
\index{むたると@ムタルド(マスタード)!そーす@ソース・ムタルド}
\index{ますたーと@マスタード(ムタルド)!そーす@ソース・ムタルド}
\index{sauce@sauce!moutarde@--- Moutarde}
\index{moutarde@moutarde!sauce@Sauce ---}

普通、このソースは提供直前に作ること。

必要の分量の\protect\hyperlink{sauce-au-beurre}{ソース・オ・ブール}を用意する。鍋を火から外し、ソース2
\(\frac{1}{2}\) dLあたり大さじ1杯のマスタードを加える。

このソースを仕上げて、提供するまで時間を空けなくてはならない場合は、湯煎にかけておく。沸騰させないよう注意すること。

\atoaki{}

\hypertarget{sauce-nantua}{%
\subsubsection[ソース・ナンチュア]{\texorpdfstring{ソース・ナンチュア\footnote{ローヌ・アルプ地方にあるナンチュア湖でエクルヴィスが穫れることに由来したソース名。エクルヴィスについて詳しくは\protect\hyperlink{sauce-bavaroise}{バイエルン風ソース}訳注参照。}}{ソース・ナンチュア}}\label{sauce-nantua}}

\frsub{Sauce Nantua}

\index{そーす@ソース!なんちゆあ@---・ナンチュア}
\index{なんちゆあ@ナンチュア!そーす@ソース・---}
\index{えくるういす@エクルヴィス!そーす@ソース・ナンチュア}
\index{sauce@sauce!nantua@--- Nantua}
\index{nantua@Nantua!sauce@Sauce ---}
\index{ecrevisse@ecrevisse!sauce@Sauce Nantua}

\protect\hyperlink{sauce-bechamel}{ベシャメルソース}1 Lに生クリーム2
dLを加え、 \(\frac{2}{3}\)量まで煮詰める。

布で漉し、生クリームをさらに1 \(\frac{1}{2}\)
dL加えて、通常の濃度に戻す。

良質な\protect\hyperlink{beurre-d-ecrevisse}{エクルヴィスバター}125
gと、小さめのエクルヴィスの尾の身\footnote{しっかり下茹でして殻を剥いたものを用いること。}20を加えて仕上げる。

\atoaki{}

\hypertarget{sauce-new-burg-avec-le-homard-cru}{%
\subsubsection[活けオマールで作るソース・ニューバーグ]{\texorpdfstring{活けオマールで作るソース・ニューバーグ\footnote{ここでは英語由来のソース名のため英語風にカタカナ書きしたが、フランスでは「ニュブール」のように発音されることも多い。}}{活けオマールで作るソース・ニューバーグ}}\label{sauce-new-burg-avec-le-homard-cru}}

\frsub{Sauce New-burg avec le homard cru}

\index{そーす@ソース!にゆーはーくいけおまーる@活けオマールを使う---・ニューバーグ}
\index{にゆーはーく@ニューバーグ!そーすいけおまーる@活けオマールを使うソース・---}
\index{sauce@sauce!new-burg homard cru@--- New-burg avec le homard cru}
\index{new-burg@New-burg!sauce homard cru@Sauce --- avec le homard cru}

800〜900 gのオマールを切り分ける。

胴の中のクリーム状の部分をスプーンで取り出し、これをよくすり潰して30 g
のバターを合わせ、別に取り置いておく。

バター40
gと植物油大さじ4杯を鍋に入れて熱し、切り分けたオマールの身を色付くまで焼く。塩とカイエンヌで調味する。殻が真っ赤になったら、鍋の油を完全に捨て、コニャック大さじ2杯と、マルサラ酒もしくはマデイラの古酒2
dLを注いで火を付けてアルコール分を燃やす\footnote{flamber
  (フロンベ)フランベする。}。注いだ酒が
\(\frac{1}{3}\)量になるまで煮詰めたら、生クリーム2
dLと\protect\hyperlink{fumet-de-poisson}{魚のフュメ}2
dLを注ぐ。弱火で25分間煮る。

オマールの身をざるにあげて水気をきる。殻から身を取り出して、さいの目に切る。

取り置いておいたオマールのクリーム状の部分をソースに混ぜ込み、完全に火が通るように軽く煮立たせてやる。さいの目に切ったオマールの身を加えて混ぜる。味見をして、必要なら塩を加えて修正する。

\hypertarget{nota-sauce-new-burg-avec-le-homard-cru}{%
\subparagraph{【原注】}\label{nota-sauce-new-burg-avec-le-homard-cru}}

さいの目に切ったオマールの身をソースに混ぜ込むのは絶対必要というわけではない。薄くやや斜めにスライスして、このソースを合わせる魚料理に添えてもいい。

\atoaki{}

\hypertarget{sauce-new-burg-avec-le-homard-cuit}{%
\subsubsection[茹でたオマールで作るソース・ニューバーグ]{\texorpdfstring{茹でたオマールで作るソース・ニューバーグ\footnote{このソースの元となった料理「オマール・ニューバーグ」は、19世紀後半にニューヨークのレストラン、デルモニコーズで常連客のアイデアをもとにフランス出身の料理長シャルル・ラノフェール(チャールズ・レンフォーファー)が完成させたと言われており、そのレシピがラノフェールの著書\href{https://archive.org/details/epicureancomplet00ranhrich}{『ジ・エピキュリアン』}(英語)に掲載されている(p.411)。現在もデルモニコーズのスペシャリテ。なお、これらソースのレシピは第二版で追加されたが、もとの料理は初版から収録されている。}}{茹でたオマールで作るソース・ニューバーグ}}\label{sauce-new-burg-avec-le-homard-cuit}}

\frsub{Sauce New-burg avec le homard cuit}

\index{そーす@ソース!にゆーはーくゆてたおまーる@茹でたオマールを使う---・ニューバーグ}
\index{にゆーはーく@ニューバーグ!そーすゆてたおまーる@茹でたオマールを使うソース・---}
\index{sauce@sauce!new-burg homard cuit@--- New-burg avec le homard cuit}
\index{new-burg@New-burg!sauce homard cuit@Sauce --- avec le homard cuit}

オマールを\protect\hyperlink{court-bouillon-e}{標準的なクールブイヨン}で茹でる。尾の身を殻から外し、やや斜めに厚さ1
cm程度にスライスする\footnote{détailler en escalopes
  (デタイエオネスカロプ)= escalopper(エスカロペ)
  エスカロップ(厚さ1〜2 cm程度の薄切り)に切る。}。ソテー鍋の内側にたっぷりとバターを塗り、そこに切ったオマールを並べるように入れる。塩とカイエンヌでしっかりと味を付け、表皮が赤く発色するように両面を焼く。上等なマデイラ酒をひたひたの高さまで注ぎ、ほぼ完全になくなるまで煮詰める。

提供直前に、オマールのスライスの上に、生クリーム2
dLと卵黄3個を溶いたものを注ぎ、火から外して、ゆっくり混ぜながら\footnote{vanner
  ヴァネする。}しっかりととろみを付ける。

\hypertarget{nota-sauce-new-burg-b}{%
\subparagraph{【原注】}\label{nota-sauce-new-burg-b}}

\protect\hyperlink{sauce-americaine}{ソース・アメリケーヌ}と同様に、これら2種のソースも元来はオマールを供するための料理だった。ソースとオマールが、要するにひとつの料理を構成していたわけだ。

ところが、そのような料理は午餐(ランチ)でしか提供することが出来ない。多くの人々は胃が弱く、夕食では消化しきれないのだ\footnote{レシピにおいて指示されているオマールが大きなものであることに注意。}。

そうした問題解決のために、我々はこれを、舌びらめのフィレやムスリーヌに添えるオマールのソースとして使うことにしたのだ。オマールの身はガルニチュールとして添えるにとどめることにした。結果は好評であった。

カレー粉やパプリカ粉末を調味料として用いれば、このソースのとてもいいバリエーションが作れる。とりわけ舌びらめや脂身の少ない白身魚によく合う。\ldots{}\ldots{}その場合、魚に少量の\protect\hyperlink{riz-indienne}{インド風ライス}を添えるといい。

\atoaki{}

\hypertarget{sauce-noisette}{%
\subsubsection[ソース・ノワゼット]{\texorpdfstring{ソース・ノワゼット\footnote{ヘーゼルナッツ、はしばみの実。}}{ソース・ノワゼット}}\label{sauce-noisette}}

\frsub{Sauce Noisette}

\index{そーす@ソース!へーせるなっつ@---・ノワゼット}
\index{へーせるなつつ@ヘーゼルナッツ!そーす@ソース・ノワゼット}
\index{のわせつと@ノワゼット!へーぜるなっつそーす@ヘーゼルナッツソース}
\index{sauce@sauce!noisette@--- Noisette}
\index{noisette@noisette!sauce@Sauce ---}

\protect\hyperlink{sauce-hollandaise}{ソース・オランデーズ}を本書のレシピのとおりに作る。提供直前に仕上げとして、上等なバターで作った\protect\hyperlink{beurre-de-noisette}{ブール・ド・ノワゼット}75
gを加える。

\ldots{}\ldots{}ポシェ\footnote{pocher
  沸騰しない程度の温度で茹でること。魚の場合は\protect\hyperlink{court-bouillon-a}{クールブイヨン}を用いてやや低めの温度で火を通すこと。}したサーモン、トラウトにとてもよく合う。

\atoaki{}

\hypertarget{sauce-normande}{%
\subsubsection{ノルマンディ風ソース}\label{sauce-normande}}

\frsub{Sauce Normande}

\index{そーす@ソース!のるまんてふう@ノルマンディ風---}
\index{のるまんていふう@ノルマンディ風!そーす@---ソース}
\index{sauce@sauce!normande@--- Normande}
\index{normande@normande!sauce@Sauce ---}

\protect\hyperlink{veloute-de-poisson}{魚料理用ヴルテ}\(\frac{3}{4}\)
Lに\footnote{原書にはリットルの表記がないが、本書における標準的な仕上がり量が
  1 Lであることと、文脈から訳者が補った。}、マッシュルームの茹で汁1
dLとムール貝の茹で汁1 dL、舌びらめのフュメ\footnote{舌びらめの料理に合わせるソースであるために、舌びらめのアラなどが必然的に出るのを無駄にせず使うということだが、現代のレストランの厨房などではかえって無理が生じることになる。このレシピ通りに作る場合は何らかのオペレーション上の工夫が必要。}
2
dLを加える。レモン果汁少々と、とろみ付け用に卵黄5個を生クリーム2dLで溶いたものを加える。強火で
\(\frac{2}{3}\)量つまり約8 dLまで煮詰める。

布で漉し、クレーム・ドゥーブル\footnote{乳酸醗酵した濃い生クリーム。\protect\hyperlink{sauce-supreme}{ソース・シュプレーム}訳注参照。}1
dLとバター125 gを加える。

\ldots{}\ldots{}このソースは\protect\hyperlink{sole-normande}{舌びらめのノルマンディ風}専用。とはいえ、使い方によっては無限の可能性がある。

\hypertarget{nota-sauce-normande}{%
\subparagraph{【原注】}\label{nota-sauce-normande}}

基本的に本書では、どんなレシピにおいても、牡蠣の茹で汁は使わないことにしている。牡蠣の茹で汁は塩味がするだけで風味がない。だから、可能であればムール貝の茹で汁を大さじ何杯か加えるほうがずっといい\footnote{このレシピは初版からの異同が大きい。初版では「魚料理用ヴルテ1
  Lあたり卵黄6個でとろみを付け、牡蠣の茹で汁2
  dLと魚のエッセンス、生クリーム2
  dLを加えながら煮詰める。仕上げにバター100 gとクレーム・ドゥーブル1
  dLを加える」となっており、用途には触れられていない。第二版、第三版ではやや細かなレシピとなり用途も「舌びらめのノルマンディ風」と指定されて現行版に近いものになるが、牡蠣の茹で汁を使うことは初版と同じ。つまり、第四版で牡蠣の茹で汁からムール貝の茹で汁を使うことに変更し、この原注が付けられた。このソースにおける改変は、前出のソース・ラギピエールのケースとやや似ているところもある。牡蠣を用いることから、牡蠣の産地であるノルマンディ風という名称となったソースであるのに、そこから牡蠣を排除するという、いわば換骨奪胎がなされているからだ。}。

\atoaki{}

\hypertarget{sauce-orientale}{%
\subsubsection[オリエント風ソース]{\texorpdfstring{オリエント風ソース\footnote{フランス語の
  orient
  オリヨン(東方)は、具体的にいうと北アフリカの一部、アラビア半島、西アジアくらいまでを指すのが一般的。その意味では、カレー粉を加えたことで「オリエント風」と称するのは、当時のフランス人にとって、理解できなくもないだろうが実感は伴わなかった可能性がある。フランス人にとっての「オリエント」である北アフリカやトルコといった地域の食文化は19世紀にはかなりフランスに伝わっていたからだ。こういった文化的なイメージのずれは、エスコフィエ本人が料理長としてのキャリアの大半をイギリスで過ごしたこととも関係があると思われる。つまり、フランス人にとっての「オリエント」とインドという植民地を持つイギリス人の「オリエント」は同じ言葉であっても、想起される具体的な内容が違うということである。}}{オリエント風ソース}}\label{sauce-orientale}}

\frsub{Sauce Orientale}

\index{そーす@ソース!おりえんとふう@オリエント風---}
\index{おりえんとふう@オリエント風!そーす@---ソース}
\index{とうほうふう@東方風!おりえんたるそーす@オリエント風ソース}
\index{sauce@sauce!orientale@--- Orientale}
\index{oriental@oriental(e)!sauce@Sauce ---e}

\protect\hyperlink{sauce-americaine}{ソース・アメリケーヌ}
\(\frac{1}{2}\) Lを用意し、カレー粉で風味付けをして
\(\frac{2}{3}\)量まで煮詰める。鍋を火から外し、生クリーム 1
\(\frac{1}{2}\) dLを混ぜ込む。

\ldots{}\ldots{}このソースの用途は\protect\hyperlink{sauce-americaine}{ソース・アメリケーヌ}と同じ。

\atoaki{}

\hypertarget{sauce-paloise}{%
\subsubsection[ポー風ソース]{\texorpdfstring{ポー風ソース\footnote{ポーは15世紀以来、ベアルヌ地方の中心都市。}}{ポー風ソース}}\label{sauce-paloise}}

\frsub{Sauce paloise}

\index{そーす@ソース!ほーふう@ポー風---}
\index{ほーふう@ポー風!そーす@---ソース}
\index{sauce@sauce!paloise@--- Paloise}
\index{palois@palois!sauce@Sauce Paloise}
\index{pau@Pau!sauce paloise@Sauce Paloise}

\protect\hyperlink{sauce-bearnaise}{ソース・ベアルネーズ}を本書に書いてあるとおりの方法と分量で用意する(\protect\hyperlink{sauce-bearnaise}{ソース・ベアルネーズ}参照)が、以下の点を変える。

\begin{enumerate}
\def\labelenumi{\arabic{enumi}.}
\item
  香りの中心となるエストラゴンを同量のミント\footnote{フランス料理よりはむしろイギリス料理でよく使われるミントを用いたこのソースをポー風と呼ぶのは、かつてこの地がイギリス貴族たちに保養地として好まれたことにちなんでいるという説もある。}に変更し、白ワインとヴィネガーを煮詰める際に加える。
\item
  さらに、仕上げの際に、細かく刻んだエストラゴンも使わない。細かく刻んだミントを使う。
\end{enumerate}

\ldots{}\ldots{}このソースの用途はソース・ベアルネーズとまったく同じ。

\atoaki{}

\hypertarget{sauce-poulette}{%
\subsubsection{ソース・プレット}\label{sauce-poulette}}

\frsub{Sauce Poulette}

\index{そーす@ソース!ふれつと@---・プレット}
\index{ふれつと@プレット!そーす@ソース・---}
\index{sauce@sauce!poulette@--- Poulette}
\index{poulette@poulette!sauce@Sauce ---}

マッシュルームの茹で汁2 dLを
\(\frac{1}{3}\)量まで煮詰める。ここに\protect\hyperlink{sauce-allemande}{ソース・アルマンド}1
Lを加え、数分間沸騰させる。鍋を火から外し、レモン果汁少々とバター60
g、パセリのみじん切り大さじ1杯を加えて仕上げる。

\ldots{}\ldots{}このソースは野菜料理に合わせるが、羊の足の料理にもよく合う。

\atoaki{}

\hypertarget{sauce-ravigote}{%
\subsubsection[ソース・ラヴィゴット]{\texorpdfstring{ソース・ラヴィゴット\footnote{ravigote
  \textless{} ravigoter
  身体を丈夫にする、元気にさせる、の派生語。香草を主体として酸味を効かせたソース(および煮込み料理)は中世以来あったが、18世紀以降ravigoteという呼び名が一般的となり、19世紀以降はこの表現がしばしば使われるようになった。ソース・ラヴィゴットは冷製と温製の2種がある。なお、ソース・ラヴィゴットのレシピとして最初期のもののひとつ、1755年刊ムノン『宮廷の晩餐』第1巻に掲載されているソース・ラヴィゴットの作り方は、薄切りにしたにんにく、セルフイユ、サラダバーネット、エストラゴン、クレソンアレノワ(オルレアン芹)、シブレットを洗ってから圧し潰し、コップ1杯のコンソメ(=この当時のコンソメはグラスドヴィアンドに近いものであることに注意)に入れて沸騰させないよう1時間以上かけて煎じる。漉し器で押すようにして漉し、ブールマニエ、塩、こしょうで味付けをして火にかけ、レモンの搾り汁で仕上げる、というもの(p.135)。}}{ソース・ラヴィゴット}}\label{sauce-ravigote}}

\frsub{Sauce Ravigote}

\index{そーす@ソース!らういこつと@---・ラヴィゴット}
\index{らういこつと@ラヴィゴット!そーす@ソース・---}
\index{sauce@sauce!ravigote@--- Ravigote}
\index{ravigote@ravigote!sauce@Sauce ---}

白ワイン1 \(\frac{1}{2}\) dLとヴィネガー1 \(\frac{1}{2}\)
dLを半量になるまで煮詰める。\protect\hyperlink{veloute}{標準的なヴルテ}8
dLを加え、数分間煮立たせる。鍋を火から外し、\protect\hyperlink{beurre-d-echalote}{エシャロットバター}90〜100
gと、セルフイユ\footnote{cerfeuil チャービル。}とエストラゴン\footnote{estragon
  フレンチタラゴン。詳しくは\protect\hyperlink{sauce-chasseur}{ソース・シャスール}訳注参照。}、シブレット\footnote{ciboulette
  日本ではチャイブとも呼ばれる。アサツキと訳されることもあるが、風味がまったく異なるので代用は不可。春に紫色の小さくてきれいな花をたくさん咲かせるので、エディブルフラワーとしてもよく用いられる。}を細かく刻んだものを同量ずつ合わせたもの計大さじ1
\(\frac{1}{2}\)杯を加えて仕上げる。

\ldots{}\ldots{}茹でた鶏に合わせる。白い内臓\footnote{家畜の副生物すなわち正肉以外の部分のうち、内臓をabatsアバと呼ぶ。そのうちの、心臓、レバー、舌などはabats
  rougeアバルージュ(赤い内臓)、耳、尾、胃、腸、足、頭、仔牛および仔羊の胸腺肉(ris
  de veauリドヴォー、ris
  d'agneauリダニョー)や腸間膜(fraiseフレーズ)などはabats
  blancアバブロン(白い内臓、白い副生物)を呼ばれている。こうした副生物の料理は古くから好まれ、16世紀フランソワ・ラブレー『ガルガンチュアとパンタグリュエル』においてもしばしば登場する。とりわけ「ガルガンチュア」の冒頭では、出産間近なお妃が臓物料理を食べ過ぎるなどというエピソードが印象深い。なお、鶏の副生物(とさか、内臓、脚など)はabattisアバティと呼ばれるので混同しないよう注意。}料理にも合わせることがある。

\atoaki{}

\hypertarget{sauce-regence-pour-poissons}{%
\subsubsection[魚料理用ソース・レジャンス]{\texorpdfstring{魚料理用ソース・レジャンス\footnote{ソース・レジャンスという名称については「ブラウン系の派生ソース」の\protect\hyperlink{sauce-regence}{ソース・レジャンス}訳注参照。}}{魚料理用ソース・レジャンス}}\label{sauce-regence-pour-poissons}}

\frsub{Sauce Régence pour Poissons, et garnitures de Poissons}

\index{そーす@ソース!れしやんすさかなよう@魚料理用---・レジャンス}
\index{れしやんす@レジャンス!そーすさかなよう@魚料理用ソース・---}
\index{sauce@sauce!regence poissons@--- Régence pour Poissons}
\index{regence@Régence!sauce poissons@Sauce --- pour Poissons}

ライン産ワイン2 dLと\protect\hyperlink{fumet-de-poisson}{魚のフォン}2
dLに新鮮なマッシュルームの切りくず20
gと生トリュフの切りくず20gを加えて半量になるまで煮詰める。

煮詰まったら布で漉し、仕上げた状態の\protect\hyperlink{sauce-normande}{ノルマンディ風ソース}8
dLを加える。

トリュフエッセンス大さじ1杯を加えて仕上げる。

\atoaki{}

\hypertarget{sauce-regence-pour-garnitures-de-volaille}{%
\subsubsection[鶏料理のガルニチュール用ソース・レジャンス]{\texorpdfstring{鶏料理のガルニチュール用ソース・レジャンス\footnote{わかりやすい例としては、後述の\protect\hyperlink{garniture-regence}{ガルニチュール・レジャンス
  B}参照。}}{鶏料理のガルニチュール用ソース・レジャンス}}\label{sauce-regence-pour-garnitures-de-volaille}}

\frsub{Sauce Régence pour garnitures de Volaille}

\index{そーす@ソース!れしやんすとりりようりのかるにてゆーるよう@鶏料理のガルニチュール用---・レジャンス}
\index{れしやんす@レジャンス!そーすとりりようりののかるにてゅーるよう@鶏料理のガルニチュール用ソース・---}
\index{sauce@sauce!regence garnitures de volaille@--- Régence pour garnitures de Volaille}
\index{regence@Régence!sauce garnitures de volaille@Sauce --- pour garnitures de Volaille}

ライン産白ワイン2 dLとマッシュルームの茹で汁2 dLにトリュフの切りくず 40
gを加え、半量になるまで煮詰める。

\protect\hyperlink{sauce-allemande}{ソース・アルマンド}8
dLを加え、布で漉す。トリュフエッセンス大さじ1杯を加えて仕上げる。

\atoaki{}

\hypertarget{sauce-riche}{%
\subsubsection[ソース・リッシュ]{\texorpdfstring{ソース・リッシュ\footnote{リッチな、裕福な、の意。ソース・ディプロマットがそもそも豪華な料理に合わせるものであり、さらにトリュフを足すことでより一層「リッチ」なものにした、ということ。}}{ソース・リッシュ}}\label{sauce-riche}}

\frsub{Sauce Riche}

\index{そーす@ソース!りつしゆ@---・リッシュ}
\index{りつしゆ@リッシュ!そーす@ソース・---}
\index{sauce@sauce!riche@--- Riche} \index{riche@riche!sauce@Sauce ---}

\protect\hyperlink{sauce-diplomate}{ソース・ディプロマット}を本書で示したとおりの分量と作り方で用意する。

トリュフエッセンス1 dLと、さいの目に切った真黒なトリュフ80
gを加えて仕上げる。

\atoaki{}

\hypertarget{sauce-rubens}{%
\subsubsection[ソース・ルーベンス]{\texorpdfstring{ソース・ルーベンス\footnote{フランドル派の画家、Peter
  Paul Rubens
  ピーテル・パウル・ルーベンス(1577〜1640)のこと。フランスでは古くから
  Pierre Paul Rubens
  ピエール・ポール・リュベンスの表記が定着している、現代フランスでは原語のままの綴り、発音を尊重することが多い。}}{ソース・ルーベンス}}\label{sauce-rubens}}

\frsub{Sauce Rubens}

\index{そーす@ソース!るーへんす@---・ルーベンス}
\index{るーへんす@ルーベンス!そーす@ソース・---}
\index{rubens@Rubens!sauce@Sauce ---}
\index{sauce@sauce!rubens@--- Rubens}

1〜2 mm角の小さなさいの目\footnote{brunoise ブリュノワーズ}に切った\protect\hyperlink{}{標準的なミルポワ}100
gをバターで色付くまで炒める。白ワイン2
dLと\protect\hyperlink{fumet-de-poisson}{魚のフュメ}3
dLを注ぎ、25分間火にかけておく。

目の細かいシノワ\footnote{円錐形で取っ手の付いた漉し器。}で漉す。数分間静かに休ませてから、浮いてきた油脂を丁寧に取り除く\footnote{dégraisser
  デグレセ。}。 \(\frac{1}{2}\) dLになるまで煮詰め、マデイラ酒大さじ1
杯を加える。

ここに卵黄2個を加えてとろみを付け、普通のバター100
gと\protect\hyperlink{beurre-colorant-rouge}{色付け用の赤いバター}30
g、アンチョビエッセンス少々を加えて仕上げる。

\ldots{}\ldots{}茹でた、すなわちポシェ\footnote{魚の場合はクールブイヨンを用いて、沸騰しない程度の温度で加熱調理すること。}した魚にこのソースはとてもよく合う。

\atoaki{}

\hypertarget{sauce-saint-malo}{%
\subsubsection[サンマロ風ソース]{\texorpdfstring{サンマロ風ソース\footnote{ブルターニュ地方の港町。観光地として有名であり、バカンスシーズンには多くの人が訪れる。}}{サンマロ風ソース}}\label{sauce-saint-malo}}

\frsub{Sauce Saint-Malo}

\index{そーす@ソース!さんまろふう@サンマロ風---}
\index{さんまろふう@サンマロ風!そーす@---ソース}
\index{sauce@sauce!saint-malo@--- Saint-Malo}
\index{saint-malo@Saint-Malo!sauce@Sauce ---}

(仕上がり5 dL分)

本書で示したとおりに作った\protect\hyperlink{sauce-vin-blanc}{白ワインソース}
\(\frac{1}{2}\)
Lに細かく刻んで白ワインで茹でたエシャロット大さじ1杯、もしくは、可能なら、\protect\hyperlink{beurre-d-echalote}{エシャロットバター}50
gと、マスタード大さじ
\(\frac{1}{2}\)杯、アンチョビエッセンス少々を加える。

\ldots{}\ldots{}海水魚のグリルに合わせる。

\atoaki{}

\hypertarget{sauce-smitane}{%
\subsubsection[ソース・スミターヌ]{\texorpdfstring{ソース・スミターヌ\footnote{サワークリームを意味するロシア語
  \ltjsetparameter{jacharrange={-2}} Сметана
  \ltjsetparameter{jacharrange={+2}}スメタナが由来。ロシア料理とフランス料理との相互影響関係にいては、\protect\hyperlink{service-russe}{序
  p.II 訳注3}
  および\protect\hyperlink{sauce-moscovite}{モスクワ風ソース}訳注参照。}}{ソース・スミターヌ}}\label{sauce-smitane}}

\frsub{Sauce Smitane}

\index{すみたーぬ@スミターヌ!そーす@ソース・---}
\index{そーす@ソース!すみたーぬ@---・スミターヌ}
\index{さわーくりーむ@サワークリーム!そーすすみたーぬ@ソース・スミターヌ}
\index{sauce@sauce!smitane@--- Smitane}
\index{smitane@smitane!sauce@Sauce ---}
\index{creme aigre@crème aigre!sauce smitane@Sauce Smitane}

中位の大きさの玉ねぎ2個を細かくみじん切りにし、バターで色付くまで炒める。白ワイン2
dLを注ぎ、完全に煮詰める。サワークリーム \(\frac{1}{2}\) Lを加える。
5分間沸騰させたら、布で漉す。サワークリームの風味を生かすために、必要に応じてレモンの搾り汁少々を加える。

\ldots{}\ldots{}ジビエのソテーやカスロール仕立て\footnote{原文は gibiers
  sautés, ou cuits à la casserole
  となっており、ジビエのソテーまたはカスロール(片手鍋)で火を通したもの、というのが逐語訳だが、ここでは
  en casserole に解釈して訳した。雉、ペルドロー
  (山うずらの若鳥)、野生のうずらなどのen casserole
  が本書にも多数収録されているためである。カスロール仕立てen
  casseroleとは、油脂を熱したカスロールで肉を焼いた後に取り出し、フォンなどを加えてソースを作り、肉を鍋に戻し入れて鍋ごと供する仕立てのこと。なお、casserole
  のうちフランスに古くからあるタイプのものは比較的浅い鍋で、ソースパンとも呼ばれる。深いものはcasserole
  russe カスロールリュス(ロシア式片手鍋)と言う。}用。

\atoaki{}

\hypertarget{sauce-solferino}{%
\subsubsection{ソース・ソルフェリノ}\label{sauce-solferino}}

\frsub{Sauce Solférino}

\index{そーす@ソース!そるふえりの@---・ソルフェリノ}
\index{そるふえりの@ソルフェリノ!そーす@ソース・---}
\index{sauce@sauce!solferino@--- Solférino}
\index{solferino@Solférino!sauce@Sauce ---}

よく熟したトマト15個をしっかり搾って、その果汁を器に入れる。これを布で漉し、濃いシロップ状になるまで煮詰める\footnote{第2章ガルニチュール、\protect\hyperlink{essence-de-tomate}{トマトエッセンス}も参照。}。

溶かした\protect\hyperlink{glace-de-viande}{グラスドヴィアンド}大さじ3杯とカイエンヌ1つまみ、レモン\(\frac{1}{2}\)個分の搾り汁を加える。

火から外して、エストラゴン風味の\protect\hyperlink{beurre-maitre-d-hotel}{メートルドテルバター}100
gと\protect\hyperlink{beurre-d-echalote}{エシャロットバター}100
gを加える。

\ldots{}\ldots{}このソースはどんな肉のグリルにもよく合う。

\hypertarget{nota-sauce-solferino}{%
\subparagraph{【原注】}\label{nota-sauce-solferino}}

言い伝えによると、フランス軍がたびたび進軍して戦ったロンバルディア平野で、たくさんの料理が創作された。このソースもそのひとつであり、カプリアナ村においてフランスとサルデーニャの連合軍司令官の昼食に供されたという。その村の近くであの苛烈きわまるソルフェリノの戦い\footnote{1859年に起きたフランス=サルデーニャ連合軍とオーストリア帝国軍の戦闘。戦場視察したナポレオン三世はその光景のあまりの悲惨さにイタリア独立戦争への介入から手を引くことを決意したともいう。}が繰り広げられたのだ。

伝えられているレシピはおそらくは調理担当軍人によるものだろうが、充分に日常的に使えるものだった。このソースは、Sauce
Saint-Cloud ソース・サンクルー\footnote{サンクルーはパリ近郊の地名。普仏戦争時(1870〜1871)にパリ包囲戦の舞台となり、休戦協定の結ばれた2日後に大火に見舞われた。いずれにせよ戦争の悲惨さを蔭に持つソース名ということになるが、エスコフィエ自身が普仏戦争において従軍したために、その名称をこのソースに付けることは許し難かったのだろう。}と呼ばれることもあるが、それは誤りだ。作り方も材料もソース・サンクルーの名を付けるにはまったく値しない程の誤りだ。

\atoaki{}

\hypertarget{sauce-soubise}{%
\subsubsection[ソース・スビーズ/玉ねぎのクリ・スビーズ]{\texorpdfstring{ソース・スビーズ/玉ねぎのクリ・スビーズ\footnote{18世紀の代表的料理人のひとりFrançois
  Marinフランソワ・マラン(生没年不詳)が仕えたシャルル・ド・ロアン・スビーズ元帥のこと。マランは4巻からなる『コモス神の贈り物、あるいは食卓の悦楽』(1739
  年刊)を著した。}}{ソース・スビーズ/玉ねぎのクリ・スビーズ}}\label{sauce-soubise}}

\frsub{Sauce Soubise, ou Coulis d'oignons Soubise}

\index{そーす@ソース!すひーす@---・スビーズ}
\index{すひーす@スビーズ!そーす@ソース・---}
\index{くり@クリ!たまねぎのくりすひーず@玉ねぎのクリ・スビーズ}
\index{sauce@sauce!soubise@--- Soubise}
\index{soubise@Soubise!sauce@Sauce ---}
\index{coulis@coulis!oignons soubise@Coulis d'oignons Soubise}

このソースの作り方には以下の2つがある。

\begin{enumerate}
\def\labelenumi{\arabic{enumi}.}
\tightlist
\item
  玉ねぎ500 gを薄切りにする\footnote{émincer
    (エモンセ)日本ではエマンセと言うことが多い。}。これをしっかり下茹でしておく。
\end{enumerate}

玉ねぎはしっかりと水気をきって、バターを加えて鍋に蓋をして弱火で色付かないよう注意して蒸し煮する\footnote{étuver
  エチュヴェ。}。ここに濃厚に作った\protect\hyperlink{sauce-bechamel}{ベシャメルソース}
\(\frac{1}{2}\)
Lを加える。塩1つまみと白こしょう少々、粉砂糖1つまみ強を加える。

オーブンに入れてじっくり火入れする。布で漉し、鍋に移したソースを熱する。バター80
gと生クリーム1 dLを加えて仕上げる。

\begin{enumerate}
\def\labelenumi{\arabic{enumi}.}
\setcounter{enumi}{1}
\tightlist
\item
  上記と同様に薄切りにした玉ねぎを下茹でし、水気をきる。豚背脂の薄いシート\footnote{barde
    de lard
    (バルドドラール)豚背脂を薄くスライスしたもの。ベーコンと誤解されがちなので注意。エスコフィエ以前の時代のフランス料理ではきわめて多用されるとても重要なものなのでぜひとも覚えておきたい。自作する際には、豚背脂の塊を冷凍した後、適度な固さに戻してから機械などを使用してスライスすると作業が容易になる。}を敷き詰めた丁度いい大きさの深手の片鍋\footnote{casserole
    russe
    \protect\hyperlink{sauce-smitane}{ソース・スミターヌ}訳注参照。}に、下茹でして水気をきった玉ねぎをすぐに入れ、カロライナ米\footnote{長粒種。リゾットなどに適している。}120
  gと\protect\hyperlink{consomme-blanc}{白いコンソメ}7
  dL、塩、こしょう、砂糖は上記と同様に加え、さらにバター25gも加える。
\end{enumerate}

強火にかけて沸騰したら、オーブンに入れてゆっくり加熱する。

鉢に米と玉ねぎを移し入れてすり潰す。これを布で漉し、温める。上記と同様にバターを生クリームを加えて仕上げる。

\hypertarget{nota-sauce-soubise}{%
\subparagraph{【原注】}\label{nota-sauce-soubise}}

スビーズはソースというよりはむしろクリ\footnote{クリcoulisについては、\protect\hyperlink{sauce-salmis}{ソース・サルミ}訳注参照。}であって、真っ白な仕上がりにすべきだ。

ベシャメルを用いた作り方のほうが米を用いるよりもいいだろう。というのも、より滑らかな口あたりのクリになるからだ。その一方、米を使うとよりしっかりした仕上がりになる。

どちらの方法で作るかは、このスビーズを合わせる料理の種類によって決めるべきだ。

\atoaki{}

\hypertarget{sauce-soubise-tomatee}{%
\subsubsection{トマト入りソース・スビーズ}\label{sauce-soubise-tomatee}}

\frsub{Sauce Soubise tomatée}

\index{そーす@ソース!すひーすとまといり@トマト入り---・スゥビーズ}
\index{すひーす@スビーズ!そーすとまといり@トマト入りソース・---}
\index{sauce@sauce!soubise tomatee@--- Soubise tomatée}
\index{soubise@Soubise!sauce tomatee@Sauce --- tomatée}

上記のいずれかの方法で作ったソース・スビーズに
\(\frac{1}{3}\)量の、滑らかで真っ赤なトマトピュレを加える。

\atoaki{}

\hypertarget{sauce-souchet}{%
\subsubsection[ソース・スーシェ]{\texorpdfstring{ソース・スーシェ\footnote{ナポレオン軍の元帥を務めたルイ・スーシェ・アルビュフェラ公爵のこと。
  正しくはSuchetだが、料理名としては Souchet
  とも綴られる。\protect\hyperlink{sauce-albufera}{ソース・アルビュフェラ}訳注参照。}}{ソース・スーシェ}}\label{sauce-souchet}}

\frsub{Sauce Souchet}

\index{そーす@ソース!すーしえ@---・スーシェ}
\index{すーしえ@スーシェ!そーす@ソース・---}
\index{sauce@sauce!souchet@--- Souchet}
\index{souchet@Souchet!sauce@Sauce ---}

オランダおよびフランドル地方の\protect\hyperlink{}{ワーテルゾイ}から派生したソース。

いくらか変化したかたちでイギリス料理に取り入れられ、近代料理の原則に合うようにさらに手を加えたもの。

細さ1〜2 mm角、長さ3〜4 cmの千切り\footnote{julienne ジュリエンヌ。}にした、にんじん、根パセリ、セロリ計150
gを用意する。

これを鍋に入れてバターを加え、蓋をして蒸し煮する\footnote{étuver au
  beurre エチュヴェオブール}。\protect\hyperlink{fumet-de-poisson}{魚のフォン}\(\frac{3}{4}\)
Lと白ワイン2 dLを注ぐ。弱火で煮て、このクールブイヨン\footnote{court-bouillon
  原義は「量の少ないブイヨン」。実際、魚などを茹でる(ポシェする)際には、ぎりぎりの大きさの鍋を用いて茹で汁の量は出来るだけ少なく済むようにする。誤解しやすい用語なので注意。}を漉す。千切りにした野菜は別に取り置いておく。

このクールブイヨンで、切り分けた魚を煮る。

魚に火が通ったら、魚の身を取り出して、クールブイヨンはシノワ\footnote{円錐形に取っ手の付いた漉し器。}漉す。これを約
\(\frac{1}{4}\)量すなわち2 \(\frac{1}{2}\)
dLになるまで煮詰める。\protect\hyperlink{sauce-vin-blanc}{白ワインソース}を加えて適当なとろみが付くようにする。あるいは単純にブールマニエでとろみを付け、軽くバターを加えてもいい。

ソースの中に取り置いていた千切りの野菜を戻し入れる。魚の切り身を覆うようにソースをかけて供する。

\atoaki{}

\hypertarget{sauce-tyrolienne}{%
\subsubsection[チロル風ソース]{\texorpdfstring{チロル風ソース\footnote{そもそも\protect\hyperlink{sauce-choron}{ソース・ショロン}をバターではなく植物油を用いて作るものであるから、オーストリアのチロル地方とはまったく関係がない。1848年のイタリア、チロルでのオーストリアに対する反乱を記念した命名だという説もあるが、真偽は不明。ただし、本書の初版からほぼ異同のない内容で収録されているため、それなりに古くから存在しているソースと思われる。}}{チロル風ソース}}\label{sauce-tyrolienne}}

\frsub{Sauce Tyrolienne}

\index{そーす@ソース!ちろるふう@チロル風---}
\index{ちろるふう@チロル風!そーす@---ソース}
\index{sauce@sauce!tyrolienne@--- Tyrolienne}
\index{tyrolien@tyrolien(ne)!sauce@Sauce ---ne}

\protect\hyperlink{sauce-bearnaise}{ソース・ベアルネーズ}を作る場合とまったく同じ要領で、白ワインとヴィネガー、香草類を煮詰める(\protect\hyperlink{sauce-bearnaise}{ソース・ベアルネーズ})参照。布で漉してきつく絞る。

これに、よく煮詰めた真っ赤なトマトピュレ大さじ2杯と卵黄6個を加える。鍋をごく弱火にかけながら、\protect\hyperlink{mayonnaise}{マヨネーズ}を作る要領で植物油5
dLを加えてしっかりと乳化させる。最後に味を調え、カイエンヌ\footnote{赤唐辛子の一品種だが、日本のカエンペッパーより辛さもマイルドで風味が違うことに注意。}ごく少量で風味を引き締める。

\ldots{}\ldots{}このソースは牛肉、羊肉のグリルや魚のグリル焼きに合う。

\atoaki{}

\hypertarget{sauce-tyrolienne-a-l-ancienne}{%
\subsubsection[チロル風ソース・クラシック]{\texorpdfstring{チロル風ソース・クラシック\footnote{このレシピは第四版のみ。ここでの
  à l'ancienne
  は「昔ながらの」という意味ではないと解釈される。ベースとなっているソース・ポワヴラードが古くからあるソースであることからこの名称を付けたのだろう。}}{チロル風ソース・クラシック}}\label{sauce-tyrolienne-a-l-ancienne}}

\frsub{Sauce Tyrolienne à l'ancienne}

\index{そーす@ソース!ちろるふうくらしつく@チロル風---・クラシック}
\index{ちろるふう@チロル風!そーす@---ソース・クラシック}
\index{sauce@sauce!tyrolienne ancienne@--- Tyrolienne à l'ancienne}
\index{tyrolien@Tyrolien!sauce ancienne@Sauce Tyrolienne à l'ancienne}

大きめの玉ねぎ2個をごく薄くスライス\footnote{émincer エマンセ。}してバターで炒める。トマト3個を押し潰して皮を剥き、種を取り除いてから加える。\protect\hyperlink{sauce-poivrade}{ソース・ポワヴラード}5
dLを加える。7〜8分間煮て仕上げる。

\atoaki{}

\hypertarget{sauce-valois}{%
\subsubsection{ソース・ヴァロワ}\label{sauce-valois}}

\frsub{Sauce Valois}

\index{うあろわ@ヴァロワ!そーす@ソース・---}
\index{そーす@ソース!うあろわ@---・ヴァロワ}
\index{sauce@sauce!valois@--- Valois}
\index{valois@Valois!sauce@Sauce ---}

\protect\hyperlink{sauce-bearnaise-a-la-glace-de-viande}{グラスドヴィアンド入りソース・ベアルネーズ}のこと(\protect\hyperlink{sauce-bearnaise}{ソース・ベアルネーズ}参照)。

\hypertarget{nota-sauce-valois}{%
\subparagraph{【原注】}\label{nota-sauce-valois}}

「ソース・ヴァロワ」はグフェが1863年頃に創案したらしい。少なくともその頃に作られるようになったものであろう。近年では「ソース・フォイヨ」の名称のほうが一般的だが、いかにもあり得そうな異論反論を受けないためにもここでその起源を記しておくのがいいと思われた。

\atoaki{}

\hypertarget{sauce-venitienne}{%
\subsubsection[ヴェネツィア風ソース]{\texorpdfstring{ヴェネツィア風ソース\footnote{ヴェネツィア料理ではさまざまな香草を用いるものがあることから、その影響を受けた、あるいは類似したものにこの名称が付けられることが多い。なお、ヴェネツィアの近く、漁港で有名なキオッジャ近郊は農業がとても盛んで、地場品種の野菜もビーツやカボチャ、ラディッキオにもキオッジャの名が付く品種がある。}}{ヴェネツィア風ソース}}\label{sauce-venitienne}}

\frsub{Sauce Vénitienne}

\index{そーす@ソース!うえねついあふう@ヴェネツィア風---}
\index{うえねついあふう@ヴェネツィア風!そーす@---ソース}
\index{sauce@sauce!venitienne@--- Vénitienne}
\index{venitien@vénitien(ne)!sauce@Sauce ---ne}

エストラゴンヴィネガー4
dLに、エシャロットのみじん切り大さじ2杯とセルフイユ25
gを加え、\(\frac{1}{3}\)量まで煮詰める。煮詰めたら布で漉し、軽く絞ってやる。ここに\protect\hyperlink{sauce-vin-blanc}{白ワインソース}\(\frac{3}{4}\)
L
を加える。\protect\hyperlink{beurre-colorant-vert}{色付け用の緑のバター}125
gと、セルフイユとエストラゴンのみじん切り大さじ1杯を加えて仕上げる。

\ldots{}\ldots{}さまざまな魚料理に添える。

\atoaki{}

\hypertarget{sauce-veron}{%
\subsubsection[ソース・ヴェロン]{\texorpdfstring{ソース・ヴェロン\footnote{Louis
  Véron
  (1798〜1867)。医師であり、文学愛好家、美食家としても有名だった。文芸誌「ルヴュ・ド・パリ」を主宰した後、新聞「ル・コンスティチュショネル」の社主となり、ウージェーヌ・シューの新聞連載小説『彷徨えるユダヤ人』を掲載、大ヒットに導いた。、自宅は文壇サロンのようだったという。主著『パリのとあるブルジョワの回想録』(1853〜
  1955年刊)。}}{ソース・ヴェロン}}\label{sauce-veron}}

\frsub{Sauce Véron}

\index{そーす@ソース!うえろん@---・ヴェロン}
\index{うえろん@ヴェロン!そーす@ソース・---}
\index{veron@Véron!sauce@Sauce ---} \index{sauce@sauce!veron@--- Véron}

仕上げた状態の\protect\hyperlink{sauce-normande}{標準的なノルマンディ風ソース}\(\frac{3}{4}\)
Lに、\protect\hyperlink{sauce-tyrolienne}{チロル風ソース}\(\frac{1}{4}\)
Lを加える。よく混ぜ合わせ、溶かしたブロンド色の\protect\hyperlink{glace-de-viande}{グラスドヴィアンド}大さじ2杯とアンチョビエッセンス大さじ1杯を加えて仕上げる。

\ldots{}\ldots{}魚料理用。

\atoaki{}

\hypertarget{sauce-villageoise}{%
\subsubsection[村人風ソース]{\texorpdfstring{村人風ソース\footnote{文字通り「村人風」の意。このソースの他にもこの名称を冠した料理はあるが由来などは不明。}}{村人風ソース}}\label{sauce-villageoise}}

\frsub{Sauce Villageoise}

\index{そーす@ソース!むらひとふう@村人風---}
\index{むらひとふう@村人風!そーす@---ソース}
\index{sauce@sauce!villageoise@--- Villageoise}
\index{villageois@villageois!sauce@Sauce Villageoise}

\protect\hyperlink{veloute}{標準的なヴルテ}\(\frac{3}{4}\)
Lに、ブロンド色の\protect\hyperlink{jus-de-veau-brun}{仔牛のジュ}\footnote{本書には「仔牛の茶色いジュ」のレシピはあるが、ブロンド色のものについては記述がない。}1
dLとマッシュルームの茹で汁1 dLを加える。
\(\frac{2}{3}\)量くらいまで煮詰め、布で漉す。

\protect\hyperlink{sauce-soubise}{ベシャメルで作ったソース・スビーズ}\footnote{2つある作り方のうちの1の方。}2
dLと、とろみ付けの卵黄4個を加える。沸騰させないよう気をつけて温め、火から外してバター100
gを加えて仕上げる。

\ldots{}\ldots{}仔牛、仔羊などの白身肉に合わせる。

\atoaki{}

\hypertarget{sauce-villeroy}{%
\subsubsection[ソース・ヴィルロワ]{\texorpdfstring{ソース・ヴィルロワ\footnote{ルイ15世の養育係を務めたヴィルロワ元帥
  François de Villeroi の名を冠したものとされる。}}{ソース・ヴィルロワ}}\label{sauce-villeroy}}

\frsub{Sauce Villeroy}

\index{そーす@ソース!ういるろわ@---・ヴィルロワ}
\index{ういるろわ@ヴィルロワ!そーす@ソース・---}
\index{sauce@sauce!villeroy@--- Villeroy}
\index{villeroy@Villeroy!sauce@Sauce ---}

\protect\hyperlink{sauce-allemande}{ソース・アルマンド}1
Lに、トリュフエッセンス大さじ4
杯と\protect\hyperlink{essences-diverses}{ハムのエッセンス}大さじ4杯を加える。

ヘラで混ぜながら強火にかけ、主素材となるものをソースに漬けて取り出したとき際に、全体をソースが覆うようになるような濃さまで煮詰めていく。

\hypertarget{nota-sauce-villeroy}{%
\subparagraph{【原注】}\label{nota-sauce-villeroy}}

このソースの唯一の使い途は、素材をこのソースで包み込んでから、イギリス式パン粉衣を付けて揚げるものだ。この方法で調理したものは常に「ヴィルロワ風」の名称となる。このソースは、古典料理において「隠れたソース」と呼ばれていたもののうちの典型例と言える。

\atoaki{}

\hypertarget{sauce-villeroy-soubisee}{%
\subsubsection{スビーズ入りソース・ヴィルロワ}\label{sauce-villeroy-soubisee}}

\frsub{Sauce Villeroy Soubisée}

\index{そーす@ソース!ういるろわすひーすいり@スビーズ入り---・ヴィルロワ}
\index{ういるろわ@ヴィルロワ!そーすすひーすいり@スビーズ入りソース・---}
\index{sauce@sauce!villeroy soubisee@--- Villeroy Soubisée}
\index{villeroy@Villeroy!sauce soubisee@Sauce --- Soubisée}

\protect\hyperlink{sauce-allemande}{ソース・アルマンド}に
\(\frac{1}{3}\)量の\protect\hyperlink{sauce-soubise}{スビーズのピュレ}\footnote{ソース・スビーズは濃度があるのでピュレと呼んだと考えていいだろう。クリ
  coulis は「やや水分の多いピュレ」と同義だからだ。}を加え、上記と同様に煮詰めて作る。

このソースを付ける素材や仕立てに合わせて、ソース1 Lあたり80〜100
gのトリュフのみじん切りを加えることもある。

\atoaki{}

\hypertarget{sauce-villeroy-tomatee}{%
\subsubsection{トマト入りソース・ヴィルロワ}\label{sauce-villeroy-tomatee}}

\frsub{Sauce Villeroy tomatée}

\index{そーす@ソース!ういるろわとまといり@トマト入り---・ヴィルロワ}
\index{ういるろわ@ヴィルロワ!そーすとまといり@トマト入りソース・---}
\index{sauce@sauce!villeroy tomatee@--- Villeroy tomatée}
\index{villeroy@Villeroy!sauce tomatee@Sauce --- tomatée}

\protect\hyperlink{sauce-villeroy}{標準的なソース・ヴィルロワ}とまったく作り方は同じだが、\protect\hyperlink{sauce-allemande}{ソース・アルマンド}の
\(\frac{1}{3}\)量の上等で真っ赤なトマトピュレを加えて作る。

\atoaki{}

\hypertarget{sauce-vin-blanc}{%
\subsubsection{白ワインソース}\label{sauce-vin-blanc}}

\frsub{Sauce vin blanc}

\index{そーす@ソース!しろわいん@白ワイン---}
\index{しろわいん@白ワイン!そーす@---ソース}
\index{わいん@ワイン!しろわいん@白ワイン!そーす@---ソース}
\index{sauce@sauce!vin blanc@--- vin blanc}
\index{vin@vin!sauce vin blanc@Sauce vin blanc}

このソースには以下の3種類の作り方がある。

\begin{enumerate}
\def\labelenumi{\arabic{enumi}.}
\item
  \protect\hyperlink{veloute-de-poisson}{魚料理用ヴルテ}1
  Lに、ソースを合わせる魚でとった\protect\hyperlink{fumet-de-poisson}{フュメ}2
  dLと、卵黄4個を加える。\(\frac{2}{3}\) 量まで煮詰め、バター150
  gを加える。この「白ワインソース」は、仕上げにオーブンに入れて照りをつける魚料理に合わせる。
\item
  良質の\protect\hyperlink{fumet-de-poisson}{魚のフュメ}1
  dLを半分にまで煮詰める。卵黄
  5個を加え、\protect\hyperlink{sauce-hollandaise}{オランデーズソース}を作る際の要領で、バター500
  gを加えてよく乳化させる。
\item
  卵黄5個を片手鍋\footnote{casserole カスロール。}に入れて溶きほぐし、軽く温めてやる。バター500
  gを加えて乳化させていく途中で、上等な\protect\hyperlink{fumet-de-poisson}{魚のフュメ}1
  dLを少しずつ加えていく\footnote{いずれの作り方にも白ワインが出てこないのは、それぞれで使われている\protect\hyperlink{fumet-de-poisson}{魚のフュメ}において\ruby{既}{すで}に白ワインを用いているから。}。
\end{enumerate}

\index{そーす@ソース!ほわいとはせい@ホワイト系の派生---|)}
\index{sauce@sauce!petites blanches composees@Petites ---s Blanches Composées|)}
\end{recette}\newpage
\href{未、原文対照チェック}{} \href{未、日本語表現校正}{}
\href{未、その他修正}{} \href{未、原稿最終校正}{}

\hypertarget{ux30a4ux30aeux30eaux30b9ux98a8ux30bdux30fcux30b9ux6e29ux88fd24}{%
\section[イギリス風ソース(温製)]{\texorpdfstring{イギリス風ソース(温製)\footnote{この節では初版で31、第二版は33、第三版と第四版で30のレシピが掲載されている。1907年刊の英語版\emph{A
  Guide to Modern Cookery}でこの節に相当する``Hot English
  Sauces''には10のレシピしか掲載されていない。この大きな数の差をどう解釈するかは意見の分かれるところだろうが、対象読者がフランス人であるかイギリス人であるかという違いを意識し、ニーズに応えるかたちをとったと考えるのが妥当だろう。ただし、あくまでもエスコフィエあるいは共同執筆者の解釈を経た「イギリス風」のソースがほとんどであることは、例えば「\protect\hyperlink{roe-buck-sauce}{ローバックソース}」において\protect\hyperlink{sauce-espagnole}{ソース・エスパニョル}を用いていること、つまりはエスコフィエが構築したソースの体系に組み込まれ得るものであることから判断がつく。}}{イギリス風ソース(温製)}}\label{ux30a4ux30aeux30eaux30b9ux98a8ux30bdux30fcux30b9ux6e29ux88fd24}}

\frsec{Sauces Anglaises Chaudes}

\index{そーす@ソース!いきりすふうおんせい@イギリス風温製---}
\index{いきりすふう@イギリス風!そーすおんせい@---ソース(温製)}
\index{sauce@sauce!anglaises chaudes@---s anglaises chaudes}
\index{anglais@anglais(e)!sauces chaudes@Sauces ---es chaudes}
\begin{recette}
\hypertarget{cranberries-sauce}{%
\subsubsection[クランベリーソース]{\texorpdfstring{クランベリーソース\footnote{英語のcranberryはツルコケモモ(学名Vaccinium
  oxycoccos)であり、フランス語airelles rougesはコケモモ(学名Vaccinium
  vitis-idaea
  L.)で、非常によく似た近縁種であり、しばしば混同される。本書でもとくに区別されていない。}}{クランベリーソース}}\label{cranberries-sauce}}

\frsub{Sauce aux Airelles\hspace{1em}\normalfont(\textit{Cranberries-Sauce})}

\index{そーす@---ソース!くらんへりー@クランベリー---}
\index{くらんへりー@クランベリー!そーす@---ソース}
\index{いきりすふう@イギリス風!くらんへりーそーす@クランベリーソース}
\index{sauce@sauce!airelles@--- aux Airelles (Cranberries-Sauce)}
\index{airelle@airelle!sauce airelles@Sauce aux Airelles (Cranberries-Sauce)}
\index{anglais@anglais(e)!sauce airelles@Sauce aux Airelles (Cranberries-Sauce)}
\index{cranberry!Cranberries-Sauce}

クランベリー500 gを1
Lの湯で、鍋に蓋をして茹でる。果肉に火が通ったら、湯をきって、目の細かい網で裏漉しする。

こうして出来たピュレに茹で汁を適量加えてやや濃度のあるソースの状態にする。好みに応じて砂糖を加える。

このソースは市販品があり\footnote{\protect\hyperlink{sauce-robert-escoffier}{ソース・ロベール・エスコフィエ}などのようなエスコフィエブランドの商品というわけではないと思われる。}、水少々を加えて温めるだけで使える。

\ldots{}\ldots{}七面鳥のロースト用。

\hypertarget{albert-sauce}{%
\subsubsection[アルバートソース]{\texorpdfstring{アルバートソース\footnote{ザクセン=コーブルク=ゴータ公アルバート王配(ヴィクトリア女王の夫)(1819〜1861)のこと。女王エリザベス二世の高祖父。本書序文p.ii
  において触れられている料理人エルーイがアルバート王配に仕えていたことがある。なお、本書に掲載されていないが、Sole
  Albert
  「舌びらめ アルベール」という料理がある。しかしながら、これはパリのレストラン、マキシムズMaxim'sでメートルドテルを務めたアルベール・ブラゼール
  Albert
  Blazerの名を冠したもので1930年代に創案されたもの。このソースとはまったく関係がないことに注意。}}{アルバートソース}}\label{albert-sauce}}

\frsub{Sauce Albert\hspace{1em}\normalfont(\textit{Albert-Sauce})}

\index{いきりすふう@イギリス風!あるはーとそーす@アルバートソース}
\index{そーす@ソース!あるはーと@アルバート---}
\index{あるはーと@アルバート!そーす@---ソース}
\index{sauce@sauce!albert@--- Albert (Albert-Sauce)}
\index{albert@Albert!sauce@Sauce --- (Albert-Sauce)}
\index{anglais@anglais(e)!sauce albert@Sauce Albert (Albert-Sauce)}

すりおろしたレフォール\footnote{raifort ホースラディッシュ、西洋わさび。}150
gに\protect\hyperlink{}{白いコンソメ}2 dLを注ぎ、弱火で20分間煮る。

\protect\hyperlink{butter-sauce}{イギリス式バターソース}3
dLと生クリーム2 \(\frac{1}{2}\) dL、パンの白い身の部分40
gを加える。強火にかけて煮詰め、木ヘラで圧し絞るようにしながら布で漉す\footnote{二人で作業すると容易。\protect\hyperlink{veloute}{ヴルテ}訳注参照。}。卵黄2個を加えてとろみを付け\footnote{このソースの特徴として、イギリスのローストビーフに欠かせないものとされるレフォール(ホースラディッシュ)を用いていることの他に、とろみ付けにパンと卵黄を使っている点にも注目すべきだろう。とろみ付けの要素としてはきわめて中世料理風と言ってもいい。ただし、中世の料理では、パンはこんがりと焼いてからヴィネガーなどでふやかしてよくすり潰し、さらに布で漉してとろみ付けに用いるのが一般的だった。パンの白い身の部分をそのまま使えるということは、それだけ小麦の精白度合いが高いということでもある。}、塩1つまみとこしょう少々で味を調える。

仕上げに、マスタード小さじ1杯をヴィネガー大さじ1杯で溶いてから加える。

\ldots{}\ldots{}牛肉、主としてフィレ肉のブレゼに添える。

\hypertarget{aromatic-sauce}{%
\subsubsection{アロマティックソース}\label{aromatic-sauce}}

\frsub{Sauce aux Aromates\hspace{1em}\normalfont(\textit{Aromatic-Sauce})}

\index{そーす@ソース!あろまていつく@アロマティック---}
\index{いきりすふう@イギリス風!あろまていつくそーす@アロマティックソース}
\index{こうそう@香草!あろまていつく@アロマティックソース}

\index{sauce@sauce!aromates@--- aux Aromates (Aromatic-Sauce}
\index{aromate@aromate!sauce@Sauce aux Aromates}
\index{anglais@anglais(e)!sauce aromates@Sauce aux Aromates (Aromatic-Sauce)}

\protect\hyperlink{consomme-blanc}{コンソメ} \(\frac{1}{2}\)
Lに、タイム1枝、バジル4 g、サリエット\footnote{シソ科の香草。サマーセイヴォリー。和名キダチハッカ。}1
g、マジョラム1 g、セージ1 g、シブレット\footnote{ciboulette
  チャイヴ。アサツキと訳されることもあるが、日本のアサツキとは風味が違うので注意。}1を刻んだもの1つまみ、エシャロット\footnote{玉ねぎによく似ているが小さくて水分量の少ない香味野菜。英語由来のシャロットと呼ばれることも。日本の青果マーケットに見られる「エシャレット」はらっきょうの若どりであってまったく別のもの。}2個のみじん切り、ナツメグ少々、大粒のこしょう4個を入れて、10分間煎じる\footnote{infuser
  アンフュゼ。}。

シノワ\footnote{円錐形で取っ手の付いた漉し器。}で漉し、バターで作った\footnote{本書第四版ではルーは必ずバターを用いる指示がなされているが、初版から第三版までは、バターもしくはグレスドマルミット(コンソメなどを作る際に浮いてきた油脂をすくい取って漉したもの)を使うという指示だっため、「バターで作った」という記述がこのように残っているレシピが散見される。}ブロンドのルー50
gを入れてとろみを付ける。数分間沸かしてから、レモン
\(\frac{1}{2}\)個分の搾り汁と、みじん切りにして下茹でしておいたセルフイユ\footnote{cerfeuil
  チャービル。}とエストラゴン\footnote{estragon フレンチタラゴン。}計大さじ1杯を加えて仕上げる\footnote{このソースで用いられている香草類の種類の多さは特筆に値するだろう。ブラウン系の派生ソースにある\protect\hyperlink{sauce-aux-fines-herbes}{香草ソース}およびホワイト系派生ソースの\protect\hyperlink{sauce-aux-fines-herbes-blanche}{香草ソース}と比較されたい。}。

\ldots{}\ldots{}大きな魚まるごと1尾のポシェあるいは牛、羊肉の大掛かりな仕立て(ルルヴェ\footnote{relevé
  \protect\hyperlink{sauce-diplomate}{ソース・ディプロマット}訳注参照。})に添える。

\hypertarget{butter-sauce}{%
\subsubsection{バターソース}\label{butter-sauce}}

\frsub{Sauce au Beurre à l'anglaise\hspace{1em}\normalfont(\textit{Butter Sauce})}

\index{いきりすふう@イギリス風!はたーそーす@バターソース}
\index{そーす@ソース!はたー@バター--}
\index{はたー@バター!そーすいきりすふう@---ソース(イギリス風)}
\index{sauce@sauce!beurre anglais@--- au Beurre à l'anglaise (Butter Sauce)}
\index{beurre@beurre!sauce anglaise@Sauce au Beurre à l'anglaise (Butter Sauce)}
\index{anglais@anglais(e)!sauce beurre@Sauce au Beurre à l'anglaise (Butter Sauce)}

フランスの\protect\hyperlink{sauce-au-beurre}{ソース・オ・ブール}と同様に作るが、より濃度の高い仕上がりにする点が違う。分量は、バター60
g、小麦粉60 g、1 Lあたり塩7 gを加えて沸かした湯 \(\frac{3}{4}\)
L。レモンの搾り汁5〜6滴、バター200 g。とろみ付け用の卵黄は用いない。

\hypertarget{capers-sauce}{%
\subsubsection{ケイパーソース}\label{capers-sauce}}

\frsub{Sauce aux Câpres\hspace{1em}\normalfont(\textit{Capers-Sauce})}

\index{いきりすふう@イギリス風!けいはー@ケイパーソース}
\index{そーす@ソース!けいはー@ケイパー---}
\index{けいはー@ケイパー!そーすいきりすふう@---ソース(イギリス風)}
\index{sauce@sauce!capres anglais@--- aux Câpres (Capers-Sauce)}
\index{capre@câpre!sauce capres anglaise@Sauce aux Câpres (Capers-Sauce)}
\index{anglais@anglais(e)!sauce capres anglais@Sauce aux Câpres (Capers-Sauce)}

上記の\protect\hyperlink{butter-sauce}{バターソース}1
Lあたり大さじ4杯のケイパーを加えたもの。

\ldots{}\ldots{}茹でた魚に添える。また、イギリス風\footnote{à l'anglaise
  アラングレーズ。茹でる(下茹でも含む)場合には、塩を加えた湯で茹でることを指す。なお、パン粉衣
  pané à l'anglaise
  という場合には、現代の日本でもなじみのある、小麦粉、溶きほぐした卵、パン粉の順で衣を付けて揚げることを言う。調理法全体を通しての規則性はなく、あくまでも「イギリス風に由来する」または「イギリス風」を意味するものなので注意。}に茹でた仔羊腿肉には欠かせない。

\hypertarget{celery-sauce}{%
\subsubsection{セロリソース}\label{celery-sauce}}

\frsub{Sauce au Céleri\hspace{1em}\normalfont(\textit{Celery-Sauce})}

\index{いきりすふう@イギリス風!せろり@セロリソース}
\index{そーす@ソース!せろり@セロリ---}
\index{せろり@セロリ!そーすいきりすふう@---ソース(イギリス風)}
\index{sauce@sauce!celeri anglais@--- au Céleri (Celery-Sauce)}
\index{celeri@céleri!sauce anglaise@Sauce au Céleri (Celery-Sauce)}
\index{anglais@anglais(e)!sauce celeri@Sauce au Céleri (Celery-Sauce)}

セロリ6株を掃除して、芯のところだけを使う\footnote{緑色が薄いタイプのセロリは中心部が自然に軟白され、柔らかいので、フランス料理でも非常に好まれる。}。これをソテー鍋に並べ、\protect\hyperlink{}{白いコンソメ}をセロリがかぶるまで注ぐ。ブーケガルニとクローブを刺した玉ねぎ1個を入れ、弱火で加熱する。

セロリの水気をきり、鉢に入れてすり潰す。これを布で漉す。こうして出来たセロリのピュレと同量の\protect\hyperlink{cream-sauce}{クリームソース}を加える。セロリの茹で汁を煮詰めたものを大さじ2〜3杯加える。

沸騰しない程度に温め、すぐに提供しない場合は湯煎にかけておく。

\ldots{}\ldots{}茹でた鶏または鶏のブレゼに添える。

\hypertarget{roe-buck-sauce}{%
\subsubsection[ローバックソース]{\texorpdfstring{ローバックソース\footnote{roebuck
  英語でノロ鹿のこと。}}{ローバックソース}}\label{roe-buck-sauce}}

\frsub{Sauce Chevreuil\hspace{1em}\normalfont(\textit{Roe-buck Sauce})}

\index{いきりすふう@イギリス風!ろーはつく@ローバックソース}
\index{そーす@ソース!ろーはっく@ローバック---}
\index{ろーはつく@ローバック!そーすいきりすふう@---ソース(イギリス風)}
\index{のろしか@ノロ鹿 ⇒ シュヴルイユ!そーす@ソース!ろーはつくそーす@ローバックソース(イギリス風)}
\index{しゆうるいゆ@シュヴルイユ!ろーはつくそーす@ローバックソース(イギリス風)}
\index{sauce@sauce!chevreuil anglaise@--- Chevreuil (Roe-buck Sauce)}
\index{chevreuil@chevreuil!sauce anglaise@Sauce Chevreuil (Roe-buck Sauce)}
\index{anglais@anglais(e)!sauce chevreuil@Sauce Chevreuil (Roe-buck Sauce)}

中位の大きさの玉ねぎを1 cm角くらいの粗みじん切\footnote{paysanne
  ペイザンヌに切る、と言う。主として野菜について言うが、1 cm角で厚さ1〜2
  mm程度。}りにし、生ハム80 g
も同様に刻む。これをバターで軽く色付くまで炒める。ブーケガルニを入れ、ヴィネガー1
\(\frac{1}{2}\) dLを注ぎ、ほとんど完全に煮詰める。

\protect\hyperlink{sauce-espagnole}{ソース・エスパニョル}3
dLを注ぎ、15分程弱火にかけて、浮いてくる不純物を取り除く\footnote{dépouiller
  デプイエ ≒ écumer エキュメ。}。

15分経ったら、ブーケガルニを取り出し、ポルト酒コップ1杯\footnote{約1
  dL。}と\protect\hyperlink{}{グロゼイユのジュレ}大さじ1杯強を加えて仕上げる。

\ldots{}\ldots{}大型ジビエ肉\footnote{この場合は当然、ノロ鹿の料理だが、フランス料理でノロ鹿は時間をかけてマリネしてから調理し、そのマリナード(漬け汁)もソースに用いるのと比べると非常にシンプルなソースになっている点が興味深い。}の料理に添える。

\hypertarget{cream-sauce}{%
\subsubsection{クリームソース}\label{cream-sauce}}

\frsub{Sauce Crème à l'anglaise\hspace{1em}\normalfont(\textit{Cream-Sauce})}

\index{いきりすふう@イギリス風!くりーむ@クリームソース}
\index{そーす@ソース!くりーむ@クリーム---}
\index{くりーむ@クリーム!そーすいきりすふう@---ソース(イギリス風)}
\index{sauce@sauce!creme anglaise@--- Crème à l'anglaise (Cream-Sauce)}
\index{creme@crème!sauce anglaise@Sauce Crème à l'anglaise (Cream-Sauce)}
\index{anglais@anglais(e)!sauce creme@Sauce Crème à l'anglaise (Cream-Sauce)}

バター100 gと小麦粉60
gで\protect\hyperlink{roux-blanc}{白いルー}を作る。

\protect\hyperlink{consomme-blanc}{白いコンソメ}7
dLでルーをのばし、マッシュルームのエッセンス1 dLと生クリーム2
dLを加える。

火にかけて沸騰させる。小玉ねぎ1個とパセリ1束を加え、弱火で15分程煮込む。提供直前に小玉ねぎとパセリは取り出す。

\ldots{}\ldots{}仔牛の骨付き背肉の塊\footnote{carré
  カレ。もとは「四角形」の意。料理では、肋骨ごとに切り分けていない仔牛および仔羊の骨付き背肉の塊を指す。}のローストに合わせる。

\hypertarget{shrimps-sauce}{%
\subsubsection{シュリンプソース}\label{shrimps-sauce}}

\frsub{Sauce Crevettes à l'anglaise\hspace{1em}\normalfont(\textit{Shrimps-Sauce})}

\index{いきりすふう@イギリス風!しゆりんふ@シュリンプソース}
\index{そーす@ソース!しゆりんふ@シュリンプ---}
\index{くるうえつと@クルヴェット!そーすいきりすふう@シュリンプソース(イギリス風)}
\index{sauce@sauce!crevettes anglaise@--- Crevettes à l'---e (Shrimps-Sauce)}
\index{crevette@crevette!sauce anglaise@Sauce Crevettes à l'---e (Shrimps-Sauce)}
\index{anglais@anglais(e)!sauce crevettes@Sauce Crevettes à l'---e (Shrimps-Sauce)}

カイエンヌ少量を加えて風味を引き締めた\protect\hyperlink{butter-sauce}{イギリス風バターソース}1
Lに、アンチョビエッセンス小さじ1杯と殻を剥いた小海老\footnote{フランス語は
  crevette(s)
  クルヴェット。\protect\hyperlink{sauce-aux-crevettes}{ソース・クルヴェット}訳注参照。}の尾の身125
gを加える。

\ldots{}\ldots{}魚料理用。

\hypertarget{devilled-sauce}{%
\subsubsection{デビルソース}\label{devilled-sauce}}

\frsub{Sauce Diable\hspace{1em}\normalfont(\textit{Devilled Sauce})}

\index{いきりすふう@イギリス風!てひるそーす@デビルソース}
\index{そーす@ソース!てひる@デビル---}
\index{あくま@悪魔 ⇒ ディアーブル!そーすいぎりすふう@デビルソース(イギリス風)}
\index{ていあーふる@ディアーブル!てひるそーす@デビルソース(イギリス風)}
\index{sauce@sauce!diable anglaise@--- Diable (Devilled Sauce)}
\index{diable@diable!sauce anglaise@Sauce --- (Devilled Sauce)}
\index{anglais@anglais(e)!sauce diable@Sauce Diable (Devilled Sauce)}

1 \(\frac{1}{2}\)
dLのヴィネガーにエシャロットのみじん切り大さじ1杯強を加えて、半量になるまで煮詰める。\protect\hyperlink{sauce-espagnole}{ソース・エスパニョル}2
\(\frac{1}{2}\) dLとトマトピュレ大さじ2杯を加え、5分間程煮る。

仕上げに、ダービーソース\footnote{原文Derby-sauce、1940年代にアメリカで市販されていたのは確認されているが、ここで言及されているのとまったく同じかは不明。なお、初版および第二版でこの部分は「ハーヴェイソースとウスターシャーソース各大さじ1杯」、第三版では「ハーヴェイソースとエスコフィエソース各大さじ1」となっている。「ダービーソース」が当初「エスコフィエソース」として商品化された後に何らかの事情により名称変更がなされたという可能性も否定できないが、第二版および英語版において\protect\hyperlink{sauce-diable-escoffier}{ソース・ディアーブル・エスコフィエ}および\protect\hyperlink{sauce-robert-escoffier}{ソース・ロベール・エスコフィエ}、さらに第二版と同年刊の英語版のみに掲載されているSauce
  aux Cerises
  Escoffierソース・オ・スリーズ・エスコフィエのように既にエスコフィエブランドの既製品ソースがあるために、矛盾が生じてしまう。第三版の記述が\protect\hyperlink{sauce-diable-escoffier}{ソース・ディアーブル・エスコフィエ}を意味していると解釈すれば矛盾は生じないだろう。ハーヴェイソースについては\protect\hyperlink{brown-gravy}{ブラウングレイヴィー}訳注参照。}大さじ1杯とカイエンヌ1つまみ強を加え、シノワ\footnote{円錐形で取っ手の付いた漉し器。}か布で漉す。

\hypertarget{scotch-eggs-sauce}{%
\subsubsection{スコッチエッグソース}\label{scotch-eggs-sauce}}

\frsub{Sauce Ecossaise\hspace{1em}\normalfont(\textit{Scotch eggs Sauce})}

\index{いきりすふう@イギリス風!すこつちえつくそーす@スコッチエッグソース}
\index{そーす@ソース!すこつとらんといきりすふう@スコッチエッグ---}
\index{すこつとらんと@スコットランド!すこつちえつくそーす@スコッチエッグソース(イギリス風)}
\index{すこつちえつく@スコッチエッグ!そーす@---ソース(イリギス風)}
\index{sauce@sauce!sance ecossaise@--- Ecossaise (Scotch eggs Sauce)}
\index{scotland@Scotland!sauce ecossaise@Sauce Ecossaise (Scotch eggs Sauce)}
\index{anglais@anglais(e)!sauce eccossaise@Sauce Ecossaise (Scotch eggs Sauce)}

バター60 gと小麦粉30 g、沸かした牛乳4
dLで\protect\hyperlink{sauce-bechamel}{ベシャメルソース}を用意する。味付けは通常どおりにすること。ソースが沸騰したらすぐに、固茹で卵の白身4個を薄切りにした\footnote{émincer
  エマンセ、薄切りにすること。}ものを加える。

提供直前に、茹で卵の卵黄を目の粗い漉し器で漉したものを混ぜ込む。

\ldots{}\ldots{}\ruby{鱈}{たら}には欠かせないソース。

\hypertarget{fennel-sauce}{%
\subsubsection[フェンネルソース]{\texorpdfstring{フェンネルソース\footnote{日本語でフェンネルと呼ばれるものは、(a)主に香草として葉を利用するタイプfenouil
  sauvage(フヌイユソヴァージュ)と、(b)白く肥大した株元を食用とするフローレンス・フェンネルfenouil
  de florence(フヌイユ・ド・フロロンス)またはfenouil
  bulbeux(フヌイユビュルブー)と呼ばれる2種がある。本書ではどちらを用いるのか明記されていないことが多いが、一般に、葉を利用するタイプは香りが非常に強く、フローレンスフェンネルの葉も食用可能だが、香りは比較的おとなしい。}}{フェンネルソース}}\label{fennel-sauce}}

\frsub{Sauce au Fenouil\hspace{1em}\normalfont(\textit{Fennel Sauce})}

\index{いきりすふう@イギリス風!ふえんねる@フェンネルソース}
\index{そーす@ソース!ふえんねるいきりすふう@フェンネル---(イギリス風)}
\index{ふえんねる@フェンネル!そーすいきりすふう@フェンネルソース(イギリス風)}
\index{sauce@sauce!fenouil anglaise@--- au Fenouil (Fennel Sauce)}
\index{fenouil@fenouil!sauce anglaise@Sauce au Fenouil (Fennel Sauce)}
\index{anglais@anglais(e)!sauce fenouil@Sauce au Fenouil (Fennel Sauce)}

普通に作った\protect\hyperlink{butter-sauce}{バターソース}2
\(\frac{1}{2}\)
dLあたり、細かく刻んで下茹でしたフェンネル大さじ1杯を加える。

\ldots{}\ldots{}このソースは主として、グリルあるいは茹でた鯖に合わせる。

\hypertarget{gooseberry-sauce}{%
\subsubsection{グーズベリーソース}\label{gooseberry-sauce}}

\frsub{Sauce aux Groseilles\hspace{1em}\normalfont(\textit{Gooseberry Sauce})}

\index{いきりすふう@イギリス風!くーすへりーそーす@グーズベリーソース}
\index{そーす@ソース!くーすへりーいきりすふう@グーズベリー---}
\index{くーすへりー@グーズベリー!そーすいきりすふう@グーズベリーソース(イギリス風)}
\index{すくり@すぐり!くーすへりーそーすいきりすふう@グーズベリーソース(イギリス風)}
\index{くろせいゆ@グロゼイユ!くーすへりーそーすいきりすふう@グーズベリーソース(イギリス風)}
\index{sauce@sauce!groseilles anglaise@--- aux Groseilles (Gooseberry Sauce)}
\index{groseille@groseille!sauce anglaise@Sauce aux Groseilles (Gooseberry Sauce)}
\index{anglais@anglais(e)!sauce groseilles@Sauce aux Groseilles (Gooseberry Sauce)}

グーズベリー1 Lの皮を剥いて洗い、砂糖125 gと水1
dLを加えて火にかける。目の細かい漉し器で裏漉しする。

\ldots{}\ldots{}このピュレはグリルした鯖に合わせる。

\hypertarget{lobster-sauce}{%
\subsubsection{ロブスターソース}\label{lobster-sauce}}

\frsub{Sauce Homard à l'anglaise\hspace{1em}\normalfont(\textit{Lobster Sauce})}

\index{いきりすふう@イギリス風!ろふすたーそーす@ロブスターソース}
\index{そーす@ソース!ろふすたーいきりすふう@ロブスター---}
\index{ろふすたー@ロブスター!そーすいきりすふう@ロブスターソース(イギリス風)}
\index{おまーる@オマール!ろふすたーそーす@ロブスターソース(イギリス風)}
\index{sauce@sauce!homard anglaise@--- Homard à l'anglaise (Lobster Sauce)}
\index{homard@homard!sauce anglaise@Sauce --- à l'anglaise (Lobster Sauce)}
\index{anglais@anglais(e)!sauce homard@Sauce Homard à l'---e (Lobster Sauce)}

カイエンヌを加えて風味を引き締めた\protect\hyperlink{sauce-bechamel}{ベシャメルソース}1
Lに、アンチョビエッセンス大さじ1杯と、さいの目に切ったオマールの尾の身100
gを加える\footnote{ホワイト系派生ソースの節にある\protect\hyperlink{sauce-homard}{ソース・オマール}を比較すると、このソースのシンプルさが際立って見えるが、ベシャメルを基本ソースにしている点で、やはり「ソースの体系」に組込まれたものであり、純粋にイギリス料理由来というわけでもないと思われる。なお、このレシピは初版からほぼ異同がなく、1907年の英語版には含まれていない。}。

\ldots{}\ldots{}魚料理用。

\hypertarget{oyster-sauce}{%
\subsubsection{牡蠣入りソース}\label{oyster-sauce}}

\frsub{Sauce aux Huîtres\hspace{1em}\normalfont(\textit{Oyster Sauce})}

\index{いきりすふう@イギリス風!かきいりそーす@牡蠣入りソース}
\index{そーす@ソース!かきいりいきりすふう@牡蠣入り---(イギリス風)}
\index{かき@牡蠣!そーすいきりすふう@牡蠣入りソース(イギリス風)}
\index{sauce@sauce!huitres anglaise@--- aux huitres (Oyster Sauce)}
\index{huitre@huître!sauce anglaise@Sauce aux ---s (Oyster Sauce)}
\index{anglais@anglais(e)!sauce huitre@Sauce aux Huîtres (Oyster Sauce)}

バター20 gと小麦粉15 gでブロンドのルーを作る。

このルーを、牛乳1 dLと生クリーム1
dLで溶く。塩1つまみを加えて調味し、火にかけて沸騰させたら弱火にして10分間煮る。

布で漉し、カイエンヌを加えて風味を引き締める。沸騰しない程度の温度で火を通して周囲をきれいに掃除した牡蠣の身12個を1
cm程度の厚さに切って、ソースに加える。

\ldots{}\ldots{}もっぱら茹でた魚\footnote{初版および第二版では「もっぱら茹でた生鱈に合わせる」とある。このレシピも1907年の英語版には掲載されていない。}に添える。

\hypertarget{brown-oyster-sauce}{%
\subsubsection{牡蠣入りブラウンソース}\label{brown-oyster-sauce}}

\frsub{Sauce brune aux Huîtres\hspace{1em}\normalfont(\textit{Brown Oyster Sauce})}

\index{いきりすふう@イギリス風!かきいりふらうんそーす@牡蠣入りブラウンソース}
\index{そーす@ソース!かきいりふらうんいきりすふう@牡蠣入りブラウン---(イギリス風)}
\index{かき@牡蠣!そーすふらうんいきりすふう@牡蠣入りブラウンソース(イギリス風)}
\index{sauce@sauce!brune huitres anglaise@--- brune aux huitres (Brown Oyster Sauce)}
\index{huitre@huître!sauce brune anglaise@Sauce brune aux ---s (Brown Oyster Sauce)}
\index{anglais@anglais(e)!sauce brune huitre@Sauce brune aux Huîtres (Brown Oyster Sauce)}

上記の牡蠣入りソースと作り方はまったく同じだが、牛乳と生クリームではなく、\protect\hyperlink{fonds-brun}{茶色いフォン}2
dLを使うこと。

\ldots{}\ldots{}このソースは、グリル焼きした肉や、肉のプディング\footnote{本書にはイギリス風の肉料理としてのプディングのレシピも掲載されている。\protect\hyperlink{beefteak-pudding}{ビーフステークのプディング}、\protect\hyperlink{beefsteak-and-kidney-pudding}{ビーフステークとキドニーのプディング}、\protect\hyperlink{beefsteak-and-oysters-pudding}{ビーフステークと牡蠣のプディング}。なお、本書でのbeefsteakビーフステークとは肉の切り方のことを意味しており、グリル焼きあるいはソテーしたもののことではない。ここでは厚さ1
  cm程度にスライスした牛肉のことを指している。}、生鱈のグリル焼きに合わせる。

\hypertarget{brown-gravy}{%
\subsubsection{ブラウングレイヴィー}\label{brown-gravy}}

\frsub{Jus coloré\hspace{1em}\normalfont(\textit{Brown Gravy})}

\index{いきりすふう@イギリス風!ふらうんくれいういー@ブラウングレイビヴィー}
\index{そーす@ソース!ふらうんくれいういー@ブラウングレイヴィー(イギリス風)}
\index{くれいういー@グレイヴィー!そーすふらうんいきりすふう@ブラウングレイヴィー(イギリス風ソース)}
\index{sauce@sauce!jus colore anglaise@Jus coloré (Brown Gravy)}
\index{gravy@gravy!jus colore anglaise@Jus coloré (Brown Gravy)}
\index{anglais@anglais(e)!jus colore@Jus coloré (Brown Gravy)}

\protect\hyperlink{butter-sauce}{イギリス風バターソース}4
dLに、ローストの肉汁2 dLとケチャップ\footnote{ここではマッシュルームケチャップのこと。マッシュルームの薄切りを塩、こしょう、香辛料で5〜6日漬け込み、その絞り汁を沸かして香辛料とトマトを加えて味を調え、漉してから保存する
  (『ラルース・ガストロノミック』初版)。なお、ketchupは語源が、中国福建省アモイの方言で、香辛料を加えて醗酵させた魚醤の一種を意味するのkôe-chiapまたは
  kê-chiap(鮭汁)だとされている。これがマレー語に伝播し、kecap(発音はケーチャプ)と変化し、17世紀頃、現在のシンガポールおよびマレーシアを植民地支配していたイギリス人の知るところとなった。イギリスにも古くから魚醤の類はあり、そのバリエーションのひとつとして、マッシュルームとエシャロットを添加した魚醤をketchupと呼ぶようになった。やがて魚醤文化の衰退とともに、ケチャップと呼ばれるものはマッシュルームが主原料となり、いわゆるマッシュルームケチャップが18世紀頃に成立したとされる。これは、塩漬けにして醗酵させたマッシュルームの搾り汁にメース、ナツメグ、こしょうなどの香辛料を加えて煮詰め、漉したもの。これにトマトを添加するようになった時期は判然としないが、おそらくは
  19世紀初頭だったと思われる。フランスの料理書では1814年刊ボヴィリエ『調理技術』第1巻に作り方が詳述されているが(p.72)、トマトは用いないマッシュルームケチャップのバリエーション。トマトを主原料としたケチャップは、アメリカのハインツHeinzが1876年にハインツ・トマトケチャップを製品化して以降、徐々に広まっていった。このため、英語圏で成立、普及したトマトケチャップがフランスにおいて知られるようになるのは、少なくとも上記『ラルース・ガストロノミック』初版(1938年)よりも後のことであり、おそらくは第二次大戦後だろうと思われる。}大さじ
\(\frac{1}{2}\)杯、ハーヴェイソース\footnote{Herwey Sauce
  19世紀〜20世紀前半にかけて既製品が流通していた。現在は商品としては存在していないと思われる。原料はアンチョビ、ヴィネガー、マッシュルームケチャップ、にんにく、大豆由来原料(詳細不明、おそらくは大豆レシチンすなわち大豆油かと思われる)、カイエンヌ、コチニール色素などであったという。}大さじ
\(\frac{1}{2}\)杯を加える。

\ldots{}\ldots{}もっぱら仔牛のローストに添える。

\hypertarget{eggs-sauce}{%
\subsubsection{エッグソース}\label{eggs-sauce}}

\frsub{Sauce aux OEufs à l'anglaise\hspace{1em}\normalfont(\textit{Eggs Sauce})}

\index{いきりすふう@イギリス風!えつくそーす@エッグソース}
\index{そーす@ソース!えつくそーす@エッグ---}
\index{たまこ@卵!そーすうーいきりすふう@エッグソース(イギリス風)}
\index{sauce@sauce!oeufs anglaise@--- aux OEufs à l'anglaise (Eggs Sauce)}
\index{oeuf@oeuf!sauce anglaise@Sauce aux ---s à l'anglaise (Eggs Sauce)}
\index{anglais@anglais(e)!sauce oeufs@Sauce aux OEufs à l'---e (Eggs Sauce)}

小麦粉60 gとバター30
gで\protect\hyperlink{roux-blanc}{白いルー}を作る。あらかじめ沸かしておいた牛乳
\(\frac{1}{2}\)
Lで溶く。塩、白こしょう、ナツメグ少々で味を調える。火にかけて沸騰したら弱火にして5〜6分間煮る。

固茹で卵2個を白身、黄身ともに、さいの目に刻んでソースに加える。

\ldots{}\ldots{}ハドック\footnote{Haddock
  鱈の一種。フランス語では同じ綴りでアドックまたは églefin,
  aiglefinエーグルファンと呼ばれる。イギリスでは主に塩漬けを燻製にしたものを指す。}やモリュ\footnote{morue
  モリュ。干し鱈、塩鱈のこと。生のものはcabillaudカビヨと呼ばれる。}の料理に合わせるのが一般的。

\hypertarget{eggs-and-butter-sauce}{%
\subsubsection{エッグアンドバターソース}\label{eggs-and-butter-sauce}}

\frsub{Sauce aux OEufs au beurre à l'anglaise\hspace{1em}\normalfont(\textit{Eggs and Butter Sauce})}

\index{いきりすふう@イギリス風!えつくあんとはたーそーす@エッグアンドバターソース}
\index{そーす@ソース!えつくあんとはたーそーす@エッグアンドバターソース}
\index{たまこ@卵!そーすうーるいきりすふう@エッグアンドバターソース(イギリス風)}
\index{はたー@バター!えつくあんとはたーそーす@エッグアンドバターソース(イギリス風)}
\index{sauce@sauce!oeufs beurre fondu anglaise@--- aux OEufs au Beurre fondu (Eggs and butter Sauce)}
\index{oeuf@oeuf!sauce oeufs beurre fondu@Sauce aux OEufs au beurre fondu (Eggs and butter Sauce)}
\index{anglais@anglais(e)!sauce oeufs beurre fondu@Sauce aux OEufs au beurre fondue (Eggs and butter Sauce)}

バター250 gを溶かし、塩適量、こしょう少々、レモン
\(\frac{1}{2}\)個分の搾り汁、固茹で卵3個を熱いうちに大きめのさいの目に刻んだもの、みじん切りにして下茹でしたパセリ小さじ1杯を加える。

\ldots{}\ldots{}茹でた魚の大きな仕立ての料理\footnote{relevé
  ルルヴェ。\protect\hyperlink{releve}{第二版序文訳注2}、および\protect\hyperlink{sauce-diplomate}{ソース・ディプロマット}訳注参照。}に添える。

\hypertarget{onions-sauce}{%
\subsubsection{オニオンソース}\label{onions-sauce}}

\frsub{Sauce aux Oignons\hspace{1em}\normalfont(\textit{Onions Sauce})}

\index{いきりすふう@イギリス風!おにおんそーす@オニオンソース}
\index{そーす@ソース!おにおんそーす@オニオンソース(イギリス風)}
\index{たまねき@玉ねぎ!そーすおにおんいきりすふう@オニオンソース(イギリス風)}
\index{sauce@sauce!oignons anglaise@--- aux Oignons (Onions Sauce)}
\index{oignon@oignon!sauce anglaise@Sauce aux Oignons (Onions Sauce)}
\index{anglais@anglais(e)!sauce oignons@Sauce aux Oignons (Onions Sauce)}

玉ねぎ200 gを薄切りにする\footnote{émincer エマンセ。}。牛乳6
dLに塩、こしょう、ナツメグを加えて玉ねぎを茹でる。

火が通ったらすぐに、玉ねぎの水気をしっかりきって、みじん切りにする。

バター40 gと小麦粉40
gで\protect\hyperlink{roux-blanc}{白いルー}を作る。これを玉ねぎを茹でた牛乳でのばす。火にかけて沸騰させ、みじん切りにした玉ねぎを加える。ソースはとても濃い状態になっていること。そのまま7〜8分煮る。

\ldots{}\ldots{}このソースは何にでも合わせられる。うさぎ、鶏、牛などの胃や腸の料理
\footnote{tripes
  トリップ。主として反芻動物(すなわち牛)の胃腸の食材としての総称。日本ではTripes
  à la mode de
  Caen(トリップアラモードドコン)「カン風トリップ煮込み」が有名。}、茹でたマトン、ジビエのブレゼなど\ldots{}\ldots{}このソースは必ず合わせる肉の上にかけてやること\footnote{本書におけるソースは特に指示がない場合はソース入れ(saucière
  ソシエール)で料理本体と別添して供すると考えておくといい。}。

\hypertarget{bread-sauce}{%
\subsubsection{ブレッドソース}\label{bread-sauce}}

\frsub{Sauce au Pain\hspace{1em}\normalfont(\textit{Bread Sauce})}

\index{いきりすふう@イギリス風!ふれつとそーす@ブレッドソース}
\index{そーす@ソース!ふれつとそーす@ブレッドソース}
\index{はん@パン!そーすふれつといきりすふう@ブレッドソース(イギリス風)}
\index{sauce@sauce!pain anglaise@--- au Pain (Bread Sauce)}
\index{pain@pain!sauce anglaise@Sauce au Pain (Bread Sauce)}
\index{anglais@anglais(e)!sauce pain@Sauce au Pain (Bread Sauce)}

牛乳 \(\frac{1}{2}\) Lを沸かし、フレッシュなパンの白い身80
gを投入する。塩1つまみ強、クローブ1本を刺した小玉ねぎ1個、バター30
gを加える。

弱火で15分程煮る。玉ねぎを取り出し、泡立て器でソースが滑かになるまでよく混ぜる。生クリーム約1
dLを加えて仕上げる。

\ldots{}\ldots{}鶏やジビエ(鳥類)のローストに合わせる。

\hypertarget{nota-bread-sauce}{%
\subparagraph{【原注】}\label{nota-bread-sauce}}

このブレッドソースを鶏のローストに添える場合は、ローストの肉汁もソース入れで添えること。ジビエの場合はさらに、よく乾かしたパンを揚げた「ブレッドクランプス」をソース入れに入れて添えること。また、フライドポテトの皿も添えること。

\hypertarget{fried-bread-sauce}{%
\subsubsection{フライドブレッドソース}\label{fried-bread-sauce}}

\frsub{Sauce au Pain frit\hspace{1em}\normalfont(\textit{Fried bread Sauce})}

\index{いきりすふう@イギリス風!ふらいとふれつとそーす@フライドブレッドソース}
\index{そーす@ソース!ふらいとふれつとそーす@フライドブレッドソース}
\index{はん@パン!そーすふらいとふれつといきりすふう@フライドブレッドソース(イギリス風)}
\index{sauce@sauce!pain frit anglaise@--- au Pain frit (Fried bread Sauce)}
\index{pain@pain!sauce frit anglaise@Sauce au Pain frit (Fried bread Sauce)}
\index{anglais@anglais(e)!sauce pain frit@Sauce au Pain frit (Fried bread Sauce)}

\protect\hyperlink{}{コンソメ}2
dLに、小さなさいの目に切った脂身のないハム80
gとエシャロット2個のみじん切りを加える。弱火で10分間煮る\footnote{mijoter
  ミジョテ。弱火で煮込むこと。}。

その間に、バター50 gを熱してパンの身50
gを揚げておく。提供直前に、揚げたパンをコンソメに入れる。パセリのみじん切り1つまみとレモンの搾り汁少々で仕上げる。

\ldots{}\ldots{} このレシピは小鳥\footnote{つぐみ(grive
  グリーヴ)など小さな鳥類のローストは、下処理した後に胸肉の部分を豚背脂のシートで一羽ずつ包み、数羽をまとめて串刺しにしてローストするのが一般的だった。}のロースト用。

\hypertarget{perseley-sauce}{%
\subsubsection{パセリソース}\label{perseley-sauce}}

\frsub{Sauce Persil\hspace{1em}\normalfont(\textit{Persley Sauce})}

\index{いきりすふう@イギリス風!はせりそーす@パセリソース}
\index{そーす@ソース!はせりそーす@パセリソース}
\index{はせり@パセリ!そーすはせりいきりすふう@パセリソース(イギリス風)}
\index{sauce@sauce!persil anglaise@--- Persil (Perseley Sauce)}
\index{persil@persil!sauce anglaise@Sauce Persil (Perseley Sauce)}
\index{anglais@anglais(e)!sauce persil@Sauce Persil (Perseley Sauce)}

\protect\hyperlink{bread-sauce}{イギリス風バターソース} \(\frac{1}{2}\)
Lに、パセリの香りを煮出した湯\footnote{infusion アンフュジオン
  \textless{} infuser
  アンフュゼ(煎じる、香りなどを煮出す)。なお、いわゆるハーブティはthéテよりもむしろ、infusion
  と呼ばれるのが一般的。}1
dLを加える。みじん切りして下茹でした\footnote{blanchir
  ブランシール。下茹ですること。モスカールド(葉の縮れるタイプ)のパセリは葉が厚く固くなりやすいためにこの作業の指示が書かれているのだろう。新鮮で柔らかいパセリの葉であれば、細かく刻んでそのまま用いた方がいい結果を得られる。}パセリの葉大さじ1杯強を加えて仕上げる。

\ldots{}\ldots{}仔牛の頭肉、仔牛の足、脳などに合わせる。

\hypertarget{ux9b5aux6599ux7406ux7528ux30d1ux30bbux30eaux30bdux30fcux30b945bis}{%
\subsubsection[魚料理用パセリソース]{\texorpdfstring{魚料理用パセリソース\footnote{このソースは第二版から。英語名は付されていない。}}{魚料理用パセリソース}}\label{ux9b5aux6599ux7406ux7528ux30d1ux30bbux30eaux30bdux30fcux30b945bis}}

\frsub{Sauce Persil pour Poissons}

\index{いきりすふう@イギリス風!はせりそーすさかなりようりよう@パセリソース(魚料理用)}
\index{そーす@ソース!さかなりようりようはせりそーす@魚料理用パセリソース}
\index{はせり@パセリ!そーすはせりいきりすふうさかなりようりよう@パセリソース(魚料理用、イギリス風)}
\index{sauce@sauce!persil poissons anglais@--- Persil pour Poissons}
\index{persil@persil!sauce anglaise poissons@Sauce Persil pour Poissons}
\index{anglais@anglais(e)!sauce persil poissons@Sauce Persil pour Poissons}

\protect\hyperlink{roux-blanc}{白いルー}60
gを、このソースを合わせる魚に火を通すのに使ったクールブイヨン
\(\frac{1}{2}\)
Lでのばす。クールブイヨンはパセリの香りをしっかり効かせたものであること。そうでない場合は、パセリの香りを煮出した湯を加えてこのソースの特徴をきちんと出してやること。

5〜6分間煮て、細かく刻んで下茹でしたパセリの葉大さじ1杯とレモン果汁少々で仕上げる。

\hypertarget{apple-sauce}{%
\subsubsection{アップルソース}\label{apple-sauce}}

\frsub{Sauce aux pommes\hspace{1em}\normalfont(\textit{Apple Sauce})}

\index{いきりすふう@イギリス風!あつふるそーす@アップルソース}
\index{そーす@ソース!あつふるそーす@アップルソース}
\index{りんこ@リンゴ!あつふるそーす@アップルソース(イギリス風)}
\index{sauce@sauce!pommes anglaise@ aux Pommes (Apple Sauce)}
\index{pomme@pomme!sauce anglaise@Sauce aux Pommes (Apple Sauce)}
\index{anglais@anglais(e)!sauce pommes@Sauce aux Pommes (Apple Sauce)}

普通にリンゴのマーマレードを作る。砂糖ごく少なめにし、シナモンの粉末をほんの少量加えること。\ldots{}\ldots{}これを提供直前に泡立て器で滑らかになるまでよく混ぜる。

\ldots{}\ldots{}このマーマレードは微温い温度で供する。鴨、がちょう、豚のローストなど、何にでも合う。

\hypertarget{nota-apple-sauce}{%
\subparagraph{【原注】}\label{nota-apple-sauce}}

ある種のローストにこのマーマレードを添えるというのは、とくにイギリスに限ったものではない。ドイツ、ベルギー、オランダでも同様に行なわれていることだ。

これらの国では、ジビエのローストにはリンゴかコケモモのマーマレード、あるいは果物のコンポート(冷製、温製どちらも)のいずれかを必ず添えるものだ\footnote{果物のコンポートへの言及は第三版から。また、1907年の英語版\emph{A
  Guide to Modern
  Cookery}には原注そのものがない。英語版のレシピは「中位の大きさのリンゴ2ポンド(約900
  g)を四つ割りにして皮を剥き、芯を取り除いて刻む。これをシチュー鍋に入れ、大さじ1杯の砂糖とシナモン少々、水を大さじ2〜3杯加える。蓋をして弱火にかけて煮る。提供直前に泡立て器で滑らかにする。このソースは微温い温度で、鴨、がちょう、うさぎのローストなどに添える」(p.45)となっている。}。

\hypertarget{porto-wine-sauce}{%
\subsubsection{ポートワインソース}\label{porto-wine-sauce}}

\frsub{Sauce au Porto\hspace{1em}\normalfont(\textit{Porto Wine Sauce})}

\index{いきりすふう@イギリス風!ほーとわいんそーす@ポートワインソース}
\index{そーす@ソース!ほーとわいんそーす@ポートワインソース}
\index{ほるとしゆ@ポルト酒!ほーとわいんそーす@ポートワインソース(イギリス風)}
\index{sauce@sauce!porto anglaise@--- au Porto (Port Wine Sauce)}
\index{porto@porto!sauce anglaise@Sauce au --- (Porto Wine Sauce)}
\index{anglais@anglais(e)!sauce porto@Sauce au Porto (Porto Wine Sauce)}

ポルト酒1 \(\frac{1}{2}\)
dLにエシャロットのみじん切り大さじ1杯とタイム1枝を加えて半量になるまで煮詰める。オレンジ2個とレモン
\(\frac{1}{2}\)
個の搾り汁を加える。オレンジの外皮の硬い部分を器具でおろしたもの\footnote{zeste
  ゼスト。オレンジやレモンの外皮の硬い部分(ごく表面の部分だけ)を薄く剥いて千切りにしたり、この場合のようにrâpeラップという器具でおろして風味付けに用いる。}小さじ1杯と塩
1つまみ、カイエンヌごく少量を加える。

これを布で漉し、美味しい\protect\hyperlink{jus-de-veau-lie}{とろみを付けた仔牛のジュ}5
dLを加える。

\ldots{}\ldots{}野生の鴨、その他のジビエ全般に合わせる。

\hypertarget{nota-porto-wine-sauce}{%
\subparagraph{【原注】}\label{nota-porto-wine-sauce}}

このイギリス料理のソースは、フランスの多くの飲食店で使われている。

\hypertarget{horse-radish-sauce}{%
\subsubsection{ホースラディッシュソース}\label{horse-radish-sauce}}

\frsub{Sauce Raifort chaude\hspace{1em}\normalfont(\textit{Horse radish Sauce})}

\index{いきりすふう@イギリス風!ほーすらていつしゆそーす@ホースラディッシュソース}
\index{そーす@ソース!ほーすらていつしゆそーす@ホースラディッシュソース}
\index{れふおーる@レフォール!ほーすらていつしゆそーす@ホースラディッシュソース(イギリス風)}
\index{sauce@sauce!raifort chaude anglaise@--- au Raifort Chaude (Horse radish Sauce)}
\index{raifort@raifort!sauce anglaise chaude@Sauce au Raifort chaude (Horse radish Sauce)}
\index{anglais@anglais(e)!sauce raifort chaude@Sauce au Raifort chaude (Horse radish Sauce)}

⇒ \protect\hyperlink{albert-sauce}{アルバートソース}の別名。

\hypertarget{reform-sauce}{%
\subsubsection[リフォームソース]{\texorpdfstring{リフォームソース\footnote{19世紀ロンドンの会員制クラブ、リフォームでフランス人料理長アレクシス・ソワイエが考案したソース。このような場合、Reformを固有名詞扱いとして英語のままとするのが現代のフランス語における考え方だが、
  20世紀初頭にはまだ、固有名詞さえもフランス語的に言い換えることがごく普通であった。}}{リフォームソース}}\label{reform-sauce}}

\frsub{Sauce Réforme\hspace{1em}\normalfont(\textit{Reform Sauce})}

\index{いきりすふう@イギリス風!りふおーむそーす@リフォームソース}
\index{そーす@ソース!りふおーむそーす@リフォームソース}
\index{りふおーむ@リフォーム!りふおーむそーす@リフォームソース(イギリス風)}
\index{sauce@sauce!reforme anglaise@--- Réforme (Reform Sauce)}
\index{reform@reform!sauce anglaise@Sauce Réforme (Reform Sauce)}
\index{anglais@anglais(e)!sauce reform@Sauce Réforme (Reform Sauce)}

\protect\hyperlink{sauce-poivrade}{ソース・ポワヴラード}と\protect\hyperlink{sauce-demi-glace}{ソース・ドゥミグラス}を合わせ、ガルニチュールとして1〜2
mmの細さで短かめの千切り\footnote{julienne courte ジュリエンヌクルト。}にした中位のサイズのコルニション2個、固茹で卵の白身、中位の大きさのマッシュルーム2個、トリュフ20
gおよび赤く漬けた牛舌肉 \footnote{langue écarlate ラングエカルラット。}を加える。

\ldots{}\ldots{}このソースは「リフォーム風」羊のコトレット\footnote{côtelette
  コトレット。羊、仔牛、仔羊の肋骨付きでカットした背肉のこと。牛の場合はcôteコットと呼ばれるが、côteという語そのものは元来「肋骨」の意。côtelette
  の -ette
  は「縮小辞」といって、より小さいものという意味を付加している。つまり、牛のcôteよりも小さいから
  côteletteとなる。なおこの語が日本語の「カツレツ」の語源だといわれている。}用。

\hypertarget{sage-and-onions-sauce}{%
\subsubsection{セージと玉ねぎのソース}\label{sage-and-onions-sauce}}

\frsub{Sauce Sauge et Oignons\hspace{1em}\normalfont(\textit{Sage and onions Sauce})}

\index{いきりすふう@イギリス風!せーしとたまねきのそーす@セージと玉ねぎのソース}
\index{そーす@ソース!せーしとたまねきのそーす@セージと玉ねぎのソース}
\index{たまねき@玉ねぎ!せーしそーすいきりすふう@セージと玉ねぎのソース(イギリス風)}
\index{せーし@セージ!たまねきのそーすいきりすふう@セージと玉ねぎのソース(イギリス風)}
\index{sauce@sauce!sauge oignons anglaise@--- Sauge et Oignons (Sage and onions Sauce)}
\index{oignon@oignon!sauce sauge anglaise@Sauce Sauge et ---s (Sage and onions Sauce)}
\index{sauge@sauge!sauce oignons anglaise@Sauce --- et Oignons (Sage and onions Sauce)}
\index{anglais@anglais(e)!sauce sauge oignons@Sauce Sauge et Oignons (Sage and onions Sauce)}

大きめの玉ねぎ2個をオーブンで焼く。冷めたら皮を剥き、みじん切りにする
\footnote{hacher アシェ。}。パンの身150
gを牛乳に浸してから圧しつぶして水分を抜く。これを玉ねぎにを混ぜ込む。

セージのみじん切り大さじ2杯と塩、こしょうで調味する。

\ldots{}\ldots{}これは鴨の詰め物にする。

\hypertarget{nota-sage-and-onions-sauce}{%
\subparagraph{【原注】}\label{nota-sage-and-onions-sauce}}

鴨をローストした際のジュを大さじ5〜6杯この詰め物に加えてソース入れで供する。

パンの身と同量の牛の脂身を茹でてみじん切りにしたものを加えることも多い。

\hypertarget{sauce-yorkshire}{%
\subsubsection[ヨークシャーソース]{\texorpdfstring{ヨークシャーソース\footnote{このレシピは初版からほぼ異同がなく(初版ではソース名が
  Yorkshire
  Sauceだったことと、「仔鴨とハムに合わせる」だったのが第二版で現在とまったく同じになることのみ)、原注もない。1907年版の英語版にも掲載されていないが、1903年アメリカ、シカゴで刊行された『\href{https://archive.org/details/stewardshandbook00whitiala}{スチュワードハンドブック}』のソースの項目のなかに、「ヨークシャーソース\ldots{}\ldots{}ハム用のオレンジソース。エスパニョル、カラント(=グロゼイユ)ゼリー、ポートワイン、オレンジジュース、茹でて千切りにしたオレンジの外皮」(p.434)とあり、エスコフィエの『料理の手引き』初版当時には既にアメリカで知られているソースであったことがわかる。ただしイギリスのヨークシャー州とどのような関係あるいはソース名の由来があるのかは不明。}}{ヨークシャーソース}}\label{sauce-yorkshire}}

\frsub{Sauce Yorkshire}

\index{いきりすふう@イギリス風!よーくしやーそーす@ヨークシャーソース}
\index{そーす@ソース!よーくしやーそーす@ヨークシャーソース}
\index{よーくしやー@ヨークシャー!よーくしやーそーす@ヨークシャーソース(イギリス風)}
\index{sauce@sauce!yorkshire@--- Yorkshire}
\index{yorkshire@Yorkshire!sauce anglaise@Sauce ---}
\index{anglais@anglais(e)!sauce yorkshire@Sauce Yorkshire}

オレンジの外皮の硬い表面だけを薄く削って細かい千切りにしたもの大さじ1
杯強を、ポルト酒2 dLでしっかり茹でる。

オレンジの皮の千切りを取り出して水気をきる。ポルト酒の入った鍋に、\protect\hyperlink{sauce-espagnole}{ソース・エスパニョル}大さじ1杯強と、\protect\hyperlink{}{グロゼイユのジュレ}も大さじ1杯強を加える。粉末のシナモン少々と、カイエンヌ少々を加える。

わずかの時間、煮詰める。布で漉し、オレンジ1個の搾り汁と千切りにした皮を加えて仕上げる。

\ldots{}\ldots{}仔鴨のローストやブレゼ、およびハムのブレゼに添える。
\end{recette}\newpage
\hypertarget{ux51b7ux88fdux30bdux30fcux30b9}{%
\section{冷製ソース}\label{ux51b7ux88fdux30bdux30fcux30b9}}

\frsec{Sauces Froides}

\index{sauce@sauce!sauces froides@sauces froides}
\index{そーす@ソース!04れいせい@★冷製---}
\begin{recette}
\hypertarget{sauce-aioli}{%
\subsubsection[アイヨリ/プロヴァンスバター]{\texorpdfstring{アイヨリ/プロヴァンスバター\footnote{aïoli(アイヨリ)はailloliとも綴るが、
  ail(にんにく)+
  oil(油)の合成語。19世紀前半には既にアカデミーフランセージの辞書に収録されており、広く知られていたようだ。茹でた塩鱈やじゃがいも、茹で卵、アーティチョーク、さやいんげんなどに合わせることが多い。Beurre
  de
  Provence(ブールドプロヴォンス)の名称を持つレシピとしてもっとも古いと思われるものは、1758年刊マラン『コモス神の贈り物』の「鳩のプロヴァンスバター添え」だろう(t.2,
  pp.290-230)。ただし、このレシピは卵黄と油の乳化ソースではない。また、オリーブオイルそのものを
  beurre de Provence
  プロヴァンスバターと呼ぶことも多かった。実際、オリーブオイルは品質にもよるが5℃以下でほぼ固形化する。}}{アイヨリ/プロヴァンスバター}}\label{sauce-aioli}}

\frsub{Sauce Aïoli, ou Beurre de Provence}

\index{そーす@ソース!04れいせい@★冷製---!あいより@アイヨリ}
\index{そーす@ソース!04れいせい@★冷製---!ふろふあんすはたー@プロヴァンスバター}
\index{あいより@アイヨリ}
\index{ふろふあんす@プロヴァンス!ふろふあんすはたー@プロヴァンスバター}
\index{はたー@バター!ふろふあんすはたー@プロヴァンスバター}
\index{sauce@sauce!sauce froide@sauce froide!aioli@--- Aïoli}
\index{sauce@sauce!sauce froide@sauce froide!beurre de provence@Beurre de Provence}
\index{aioli@Aïoli!sauce@Sauce ---}
\index{provence@Provence!Beurre de Provence (Aïoli)}
\index{beurre@beurre!beurre de provence@Beurre de Provence (Aïoli)}

にんにく4片(30 g)を鉢\footnote{この種の作業には、大理石製のものが伝統的によく用いられる。。}に入れて細かくすり潰す。ここに生の卵黄1個、塩1つまみを加える。混ぜながら、2
\(\frac{1}{2}\) dLの油\footnote{原書ではとくに言及されていないが、プロヴァンス地方ではオリーブオイルを用いることが一般的。}を初めは1滴ずつ加えていき、ソースがまとまりはじめたら糸を垂らすようにして加える。この作業は鉢に入れたままで、棒をはげしく動かして行なう。

攪拌する作業の途中、レモン1個分の搾り汁と冷水大さじ
\(\frac{1}{2}\)杯を少しずつ加えて、ソースが固くなり過ぎないようにしてやること。

\hypertarget{nota-sauce-aioli}{%
\subparagraph{【原注】}\label{nota-sauce-aioli}}

このアイヨリソースが分離してしまいそうな時は、卵黄をさらに1個足して、マヨネーズと場合と同様に修正すること。

\hypertarget{sauce-andalouse}{%
\subsubsection[アンダルシア風ソース]{\texorpdfstring{アンダルシア風ソース\footnote{いうまでもなくスペインのアンダルシア地方のことだが、トマトやオリーブオイル、チョリソなどこの地方を「想起」させる食材が使われている料理などがこの名称になっている傾向がある。ところが、トマトにしろオリーブオイルにしろアンダルシア地方特有というわけではなく、アンダルシアが産地として有名なチョリソくらいしか、料理名の根拠となり得るものはない。逆に言えば、アンダルシア地方の食文化との関係は、そこに用いられている食材以外にはないものと考えてもいい。料理名に付けられた地方名がとりたてて根拠や由来のないものであることを示す一例。}}{アンダルシア風ソース}}\label{sauce-andalouse}}

\frsub{Sauce Andalouse}

\index{そーす@ソース!04れいせい@★冷製---!あんたるしあふう@アンダルシア風---}
\index{あんたるしあ@アンダルシア!そーす@---風ソース}
\index{そーす@ソース!あんたるしあふう@アンダルシア風---}
\index{sauce@sauce!sauce froide@sauce froide!Andalouse@--- Andalouse}
\index{sauce@sauce!andalouse@--- Andalouse}
\index{andalous@Andalous(e)!sauce@Sauce Andalouse}

ごく固く仕上げた\protect\hyperlink{mayonnaise}{ソース・マヨネーズ}\troisquarts{}
Lに、上等な赤いトマトピュレ2 \(\frac{1}{2}\)
dLを加える。小さなさいの目に切ったポワヴロン\footnote{Poivron
  いわゆる日本で青果として輸入されているパプリカ(肉厚の辛くないピーマン)とほぼ同じものだが、香辛料として用いられる粉末のパプリカと混同を避けるため、あえてフランス語をそのままカタカナに訳した。}75
gを仕上げに加える。

\hypertarget{sauce-bohemienne}{%
\subsubsection[ソース・ボヘミアの娘]{\texorpdfstring{ソース・ボヘミアの娘\footnote{アイルランド出身の作曲家マイケル・ウィリアム・バルフェMichael
  William Balfe (1808〜1870)のオペラ\emph{The Bohemien
  Girl}『ボヘミアの少女』のフランス語版タイトル\href{https://archive.org/details/labohmiennegrand00balf}{\emph{La
  Bohémienne}}『ラボエミエーヌ』にちなんだものと言われている。この作品はロンドンで
  1843年初演、1862年に四幕形式のフランス語版がパリのオペラ=コミック劇場で上演され、大ヒットしたという。この名を冠した料理はいくつかあるが、いずれもチェコのボヘミア地方とは何の関連性も認められないため、オペラの人気作品にあやかった料理名と考えるのが妥当だろう。}}{ソース・ボヘミアの娘}}\label{sauce-bohemienne}}

\frsub{Sauce Bohémienne}

\index{そーす@ソース!04れいせい@★冷製---!ほへみあのむすめ@---ボヘミアの娘}
\index{ほへみあ@ボヘミア!そーす@ソース・---の娘}
\index{そーす@ソース!ほへみあ@---・ボヘミアの娘}
\index{sauce@sauce!sauce froide@sauce froide!bohemienne@--- Bohémienne}
\index{sauce@sauce!bohemienne@--- Bohémienne}
\index{bohemien@bohémien(ne)!sauce@Sauce Bohémienne}

陶製の容器に、濃厚でよく冷やした\protect\hyperlink{sauce-bechamel}{ベシャメルソース}1
\(\frac{1}{2}\) dLと卵黄4個、塩10
g、こしょう少々、ヴィネガー数滴を入れる。

泡立て器で全体をよく混ぜ、標準的なマヨネーズを作るのとまったく同じ要領で、油1
Lとエストラゴンヴィネガー大さじ2杯程を加える。

\ldots{}\ldots{}仕上げに、マスタード大さじ1杯を加える。

\hypertarget{sauce-chantilly-froide}{%
\subsubsection[ソース・シャンティイ]{\texorpdfstring{ソース・シャンティイ\footnote{パリ近郊の地名。詳しくはホワイト系派生ソースの\protect\hyperlink{sauce-chantilly}{ソース・シャンティイ}訳注参照。}}{ソース・シャンティイ}}\label{sauce-chantilly-froide}}

\frsub{Sauce Chantilly}

\index{そーす@ソース!04れいせい@★冷製---!しやんていい@---・シャンティイ}
\index{しやんていい@シャンティイ!そーす@ソース・---(冷製)}
\index{そーす@ソース!しやんていい@---・シャンティイ}
\index{sauce@sauce!sauce froide@sauce froide!chantilly@--- Chantilly}
\index{sauce@sauce!chantilly@--- Chantilly (froide)}
\index{chantilly@Chantilly!sauce@Sauce --- (froide)}

酸味付けにレモンを用いて、固く仕上げた\protect\hyperlink{mayonnaise}{ソース・マヨネーズ}\troisquarts{}
Lを用意しておく。提供直前に、ごく固く泡立てた生クリーム大さじ4杯\footnote{大さじ1杯
  = 15
  ccという概念にとらわれないよう注意。原文は、大きなスプーンで泡立てた生クリームをざっくりと4回加えるイメージで書かれている。本書における通常のソースの仕上がり量が約1
  Lであることを考慮すると、最低でも100 mL以上は加えることになるだろう。}を加える。その後、味を\ruby{調}{ととの}える。

\ldots{}\ldots{}もっぱら、アスパラガスの冷製、温製に添える。

\hypertarget{nota-sauce-chantilly-froide}{%
\subparagraph{【原注】}\label{nota-sauce-chantilly-froide}}

生クリームを加えるのは、このソースを使うまさにその時にすること。前もって加えておくと、ソースが分離してしまう恐れがあるので注意。

\hypertarget{sauce-genoise-froids}{%
\subsubsection[ジェノヴァ風ソース]{\texorpdfstring{ジェノヴァ風ソース\footnote{あまり明確な由来はないが、ジェノヴァが地中海に面した港町であり、このソースが魚料理用であるという点で一応の説明はつくだろう。}}{ジェノヴァ風ソース}}\label{sauce-genoise-froids}}

\frsub{Sauce Génoise}

\index{そーす@ソース!04れいせい@★冷製---!しえのうあふう@ジェノヴァ風---}
\index{しえのうあふう@ジェノヴァ風!そーす@ソース・---(冷製)}
\index{そーす@ソース!しえのうあふう@ジェノヴァ風---}
\index{sauce@sauce!sauce froide@sauce froide!genoise@--- Génoise}
\index{sauce@sauce!genoise@--- Génoise (froide)}
\index{genois@Génois(e)!sauce@Sauce ---e (froide)}

殻と皮を剥いたばかりのピスタチオ40 gと、松の実25
g、松の実がない場合はスイートアーモンド20
gを鉢に入れてよくすり潰し、冷めた\protect\hyperlink{sauce-bechamel}{ベシャメルソース}小さじ1杯程度を加えて練ってペースト状にする。これを目の細かい網で裏漉しする。陶製の容器に卵黄6個、塩1つまみ、こしょう少々を入れる。泡立て器でよく混ぜる。油1
Lと中位の大きさのレモン2個の搾り汁を少しずつ加えてよく混ぜて乳化させていく\footnote{明記されていないが、ソースをしっかりと乳化させるためには\protect\hyperlink{mayonnaise}{マヨネーズ}と同様に作業すること。}。仕上げにハーブのピュレ大さじ3杯を加える。これは、パセリの葉とセルフイユ、エストラゴン、時季が合えばサラダバーネットを同量ずつ用意し、強火で2分間下茹でしてから湯をきり、冷水にさらしてから水気を強く絞り、裏漉しして作っておく。

\ldots{}\ldots{}冷製の魚料理全般に合わせられる。

\hypertarget{sauce-gribiche}{%
\subsubsection[ソース・グリビッシュ]{\texorpdfstring{ソース・グリビッシュ\footnote{由来不明の語。ノルマンディ方言で「子どもを怖がらせるおばさん」の意味で用いられるということが分かっているのみ。19世紀後半以降に創案もしくは一般化したソースと思われる。本書初版には当然のように既に収録されており、その後の大きな異同もない。ただ、本書初版以前に出版された料理書においてこのソースのレシピはまだ見つかっていない。ファーヴルは1905年刊『料理および食品衛生事典』第二版で「ある種のレムラードにレストランで付けられた名称」と定義し、掲載しているレシピは本書初版のものと大差ないが、「ウスターシャソース少々も加える」となっているところが目を引く。また、1913年初版のプルーストの長編小説『失なわれた時を求めて』の「スワン家の方へ」冒頭において「彼(=スワン)を招いていない夕食会のために、ソース・グリビッシュやパイナップルのサラダのレシピが必要になるや、ためらいもなく探しに行かせたりするのだった」(p.18)。もしこの語り手の記述が正確であるなら、19世紀末には広く知られたものであったと考えるべきだが、小説の場合は必ずしも歴史的事実と符号するわけではないので注意が必要。}}{ソース・グリビッシュ}}\label{sauce-gribiche}}

\frsub{Sauce Gribiche}

\index{そーす@ソース!04れいせい@★冷製---!くりひつしゆ@---・グリビッシュ}
\index{くりひつしゆ@グリビッシュ!そーす@ソース・---(冷製)}
\index{そーす@ソース!くりひつしゆ@---・グリビッシュ}
\index{sauce@sauce!sauce froide@sauce froide!gribiche@--- Gribiche}
\index{sauce@sauce!gribiche@--- Gribiche (froide)}
\index{gribiche@gribiche!sauce@Sauce --- (froide)}

茹であがったばかりの固茹で卵の黄身6個を陶製のボウルに入れ、マスタード小さじ1杯、塩1つまみ強、こしょう適量を加えてよく練り、滑らかなペースト状にする。植物油
\(\frac{1}{2}\) Lとヴィネガー大さじ1
\(\frac{1}{2}\)杯を加えながらよく混ぜて乳化させる。仕上げに、コルニションとケイパーのみじん切り計
100
gと、パセリとセルフイユ、エストラゴンのみじん切りのミックスを大さじ1杯、短かめの千切りにした固茹で卵の白身3個分を加える。

\ldots{}\ldots{}冷製の魚料理に添えるのが一般的。

\hypertarget{sauce-groseilles-au-raifort}{%
\subsubsection{レフォール風味のソース・グロゼイユ}\label{sauce-groseilles-au-raifort}}

\frsub{Sauce Groseilles au Raifort}

\index{そーす@ソース!04れいせい@★冷製---!れふおーるふうみくろせいゆ@レフォール風味の---・グロゼイユ}
\index{くろせいゆ@グロゼイユ!そーすれふおーる@レフォール風味のソース・---(冷製)}
\index{れふおーる@レフォール!そーすくろせいゆ@---風味のソース・グロゼイユ}
\index{そーす@ソース!れふおーるふうみくろせいゆ@レフォール風味の---・グロゼイユ}
\index{sauce@sauce!sauce froide@sauce froide!grroseilles raifort@--- Grroseilles au Rifort}
\index{sauce@sauce!groseille@--- Groseilles au Raifort (froide)}
\index{raifort@raifort!sauce@Sauce Groseilles au --- (froide)}
\index{groseille@groseille!sauce@Sauce --- au Raifort (froide)}

ポルト酒1 dLにナツメグ、シナモン、塩、こしょう各1つまみを加え、を
\(\frac{2}{3}\)量まで煮詰める。溶かした\protect\hyperlink{}{グロゼイユのジュレ}4
dLと細かくすりおろしたレフォール大さじ2杯を加える。

(さまざまな用途に使える)

\hypertarget{sauce-italienne-froide}{%
\subsubsection[イタリア風ソース]{\texorpdfstring{イタリア風ソース\footnote{このソースも温製のイタリア風ソースと同様に名称にとくに由来などはないと思われる。}}{イタリア風ソース}}\label{sauce-italienne-froide}}

\frsub{Sauce Italienne}

\index{そーす@ソース!04れいせい@★冷製---!いたりあふう@イタリア風---}
\index{いたりあふう@イタリア風!そーす@---ソース(冷製)}
\index{そーす@ソース!いたりあふうれいせい@イタリア風---(冷製)}
\index{sauce@sauce!sauce froide@sauce froide!italienne@--- Italienne}
\index{sauce@sauce!italienne@--- Italienne (froide)}
\index{italien@italien(ne)!sauce froide@Sauce ---ne (froide)}

仔牛の脳半分を、香草を効かせたクールブイヨンで火を通し、目の細かい網で裏漉しする。同量の牛あるいは羊の脳でもいい。

裏漉ししたピュレを陶製の器に入れ、泡立て器で滑らかになるまで混ぜる。卵黄5個と塩10
g、こしょう1つまみ強、油1
Lとレモン果汁1個分でマヨネーズを作り、そこの脳のピュレを加える。パセリのみじん切り大さじ1杯強を加えて仕上げる。

\ldots{}\ldots{}どんな冷製の肉料理にも合う。

\hypertarget{mayonnaise}{%
\subsubsection[マヨネーズ]{\texorpdfstring{マヨネーズ\footnote{このソース名の語源には諸説あり、未だ定説と呼べるものはない。
  Mayonnaise
  という綴りそのものは1806年のヴィアール『帝国料理の本』が初出で、Saumon
  à la Mayonnaise, Filet de Sole en Mayonnaise, Poulet en Mayonnaise
  の3つのレシピが掲載されている。そのうちのひとつ、サーモンのマヨネーズは、筒切りにしたサーモンを茹でて冷まし、ジュレを混ぜたマヨネーズをかける、という内容であり、ソースについてはマヨネーズの項を参照となっているが、どういうわけかこの本にマヨネーズそのもののレシピはない。また、「鶏のマヨネーズ仕立て」におけるソースはどう見てもこんにち我々が理解しているマヨネーズとまったく違い、鶏のゼラチン質を冷し固める要素として利用したものだ。同じヴィアールの改訂版ともいうべき『王国料理の本』(1822
  年)にはマヨネーズのレシピが掲載されている。興味深いことに「このソースにはいろいろな作り方がある。生の卵黄を使うもの、ジュレを使うもの、仔牛のグラスを使うものや仔牛の脳を使うもの」として、もっとも一般的な方法として生の卵黄を使う方法が示されている。生の卵黄に攪拌しながら少しずつ油を加えていき、固くなってきたらヴィネガー少々を加えてコシをきる、という方法であり、こんにち我々の知るマヨネーズに非常に近いものとなっている。また、1814年刊ボヴィリエ『調理技法』のソース・マヨネーズは、焼き物の器に油大さじ3〜4杯とエストラゴンヴィネガー2杯を入れる。細かく刻んだエストラゴン、エシャロット、サラダバーネットをたっぷり加え、ジュレ大さじ2、
  3杯を加える。ソースがまとまって、ポマード状になったら、味を調える
  (p.66)、というもの。ここでも卵黄と植物油の乳化ソースとはなっていない。綴りについては、カレームはmagner(マニェ)捏ねる、という意味の動詞から派生したものだとして、magnonnaiseもしくはmagnionnaiseと綴るべきだと『パリ風料理の本』で力説している。グリモ・ド・ラ・レニェールは中世フランス語で卵黄を意味するmoyeuの派生語としてmoyeunnaiseという綴りを使っている。そのほかフランス大西洋岸の地名バイヨンヌの形容詞bayonnais(バヨネ)が語源だという説もある。綴りの起源についてある程度有力視されているのは、1756年にリシュリュー公爵が当時イギリスに占領されていたミノルカ島のマオン港
  Mahon を奪取したことにちなんで、mahonnaise
  と名づけられたというもの。ところで、植物油ではなくバターを用いるものとして、\protect\hyperlink{sauce-hollandaise}{オランデーズソース}の原型ともいうべきレシピが1651年のラ・ヴァレーヌ『フランス料理の本』に、Asperges
  à la Sauce blanche
  アスパラガスのホワイトソース添え(p.238)として掲載されていることや、卵黄をポタージュやラグーのとろみ付けに使うことが古くから行なわれていたことなどを総合すると、良質のオリーブオイルやひまわり油を利用しやすい環境にある南フランスの方がどちらかといえば、卵黄と植物油の乳化作用を利用したソースの発達、普及しやすい環境にあったとも想像されよう。なお、この『料理の手引き』では卵黄のみを用いたレシピとなっているが、全卵を用いる場合もある。日本の市販品でも卵黄のみを使うメーカーと全卵を使用しているメーカーが混在している。なお、マヨネーズの仕上がりは、卵黄のみか全卵を用いるかという問題もあるが、どのような植物油を使うかにも大きく左右されるので注意。}}{マヨネーズ}}\label{mayonnaise}}

\frsub{Sauce Mayonnaise}

\index{そーす@ソース!04れいせい@★冷製---!まよねーす@マヨネーズ}
\index{まよねーす@マヨネーズ}
\index{そーす@ソース!まよねーす@マヨネーズ}
\index{sauce@sauce!sauce froide@sauce froide!mayonnaise@--- Mayonnaise}
\index{sauce@sauce!mayonnaise@--- Mayonnaise}
\index{mayonnaise@mayonnaise!sauce@Sauce ---}

冷製ソースのほとんどはマヨネーズの派生ソースだから、\protect\hyperlink{sauce-espagnole}{ソース・エスパニョル}や\protect\hyperlink{veloute}{ヴルテ}と同様に基本ソースと見なされる。マヨネーズの作り方はきわめてシンプルだが、以下に述べるポイントはしっかり頭に入れておく必要がある。

\hypertarget{proportions-mayonnaise}{%
\subparagraph{材料と分量}\label{proportions-mayonnaise}}

\ldots{}\ldots{}卵黄6個、「からざ」は取り除いておくこと。油1 L。塩 10
g、白こしょう1 g、ヴィネガー大さじ1
\(\frac{1}{2}\)杯または、より白い仕上がりを目指す場合にはヴィネガーと同等量のレモン果汁。

\begin{enumerate}
\def\labelenumi{\arabic{enumi}.}
\item
  塩、こしょう、ヴィネガーまたはレモン果汁ほんの少々を加えて、泡立て器で卵黄を溶く。
\item
  油を最初は1滴ずつ加えていき、滑らかにまとまり始めたら、糸を垂らすようにして油を加えていく。
\item
  何回かに分けてヴィネガーもしくはレモン果汁を少量ずつ加え、コシを切ってやること\footnote{原文
    rompre le corps de la sauce
    ソースの粘り気をヴィネガーなどを加えることで「ゆるめる」あるいは「のばす」こと。ここでは「コシをきる」と訳したが、日本の調理用語なので注意。この作業は、一見乳化したように見えてもまだ乳化が不完全であるため、何回かに分けて濃度を下げ、攪拌を続けることで乳化を促進させ安定したものにするのが目的。}。
\item
  最後に熱い湯を大さじ3杯加える。これは乳化をしっかりさせて、作り置きしておく必要がある場合でもソースが分離しないようにするため。
\end{enumerate}

\hypertarget{nota-mayonnaise}{%
\subparagraph{【原注】}\label{nota-mayonnaise}}

\noindent 1.
卵黄だけの段階で塩こしょうをするとソースが分離してしまうのではないかというのは思い込みに過ぎず、実際に調理現場で作業している者はそう考えていない。むしろ、塩を卵黄の水分に溶かし込んでおいた方が、卵黄がまとまりやすくなることは科学的に証明されている\footnote{当時の知見であることに注意。}。

\begin{enumerate}
\def\labelenumi{\arabic{enumi}.}
\setcounter{enumi}{1}
\item
  マヨネーズを作る際に、氷の上に容器を置いて作業するのも間違いだ。事実はまったく逆で、冷気が伝わることがもっとも分離させてしまいやすい原因だ。寒い季節には、油はやや微温めか、せめて厨房の室温くらいにするべきだ\footnote{オリーブオイルのように、飽和温度が高い種類の油ではよく見られる現象。ひまわり油でさえも寒さで濁るので、この指摘は正しい。}。
\item
  マヨネーズが分離してしまう原因としては\ldots{}\ldots{}

  \begin{enumerate}
  \def\labelenumii{\arabic{enumii}.}
  \tightlist
  \item
    最初に油を入れ過ぎてしまうこと。
  \item
    冷え過ぎた油を使うこと
  \item
    卵黄の量に対して油の量が多過ぎること。卵黄1個につき油を乳化させることが出来るのは、作り置きするのには1
    \(\frac{3}{4}\) dL、すぐに使う場合でも2 dLが限度\footnote{卵黄の乳化能力は含まれているレシチンの量で決まるので理論上はもっと大量の油を乳化することが可能。風味や仕上がりを考慮に入れて、この数字はあくまでも目安と考えたほうがいい。}。
  \end{enumerate}
\end{enumerate}

\hypertarget{mayonnaise-collee}{%
\subsubsection{コーティング用マヨネーズ}\label{mayonnaise-collee}}

\frsub{Sauce Mayonnaise collée}

\index{そーす@ソース!04れいせい@★冷製---!こーていんくようまよねーす@コーティング用マヨネーズ}
\index{まよねーす@マヨネーズ!こーていんくよう@コーティング用---}
\index{そーす@ソース!こーていんくようまよねーす@コーティング用マヨネーズ}
\index{sauce@sauce!sauce froide@sauce froide!mayonnaise collee@--- Mayonnaise collée}
\index{sauce@sauce!mayonnaise collee@--- Mayonnaise collée}
\index{mayonnaise@mayonnaise!sauce collée@Sauce Mayonnaise collée}

コーティング用マヨネーズは、マヨネーズ7 dLに溶かしたジュレ3
dLを混ぜ込んだもの。野菜サラダをあえるのに使う他、\protect\hyperlink{}{「ロシア風」ショフロワ}の素材を覆うのにも使う。

\hypertarget{nota-mayonnaise-collee}{%
\subparagraph{【原注】}\label{nota-mayonnaise-collee}}

\protect\hyperlink{sauce-chaud-froid-maigre}{魚料理用ソース・ショフロワ}の項で述べたように、このコーティング用マヨネーズの代わりに魚料理用ソース・ショフロワを使う方がいい。その方がコーティング用マヨネーズを使う場合よりも風味も見た目もよくなる。というのも、コーティング用マヨネーズは、冷気によってゼラチンが固まるとともに収縮し、マヨネーズに圧力がかかるために、ソースで素材を覆った表面に油が浸み出してしまう\footnote{初版における原注は、「コーティング用マヨネーズで覆ったものは、数時間経つと、油の露で覆われたようになってしまうことがある。その原因は、冷気によってゼラチンが固まる際に収縮し、その結果マヨネーズに圧力がかかり、液体である油がソースを覆った表面に浸みだしてくることだ。これを避けるために、コーティング用マヨネーズはこんにちでは使われなくなっており、我々の場合だと、かなり以前から魚料理用ソース・ショフロワを用いている(p.163)」。第二版以降、多少の異同はあるが、ほぼ第四版の記述と同様。いずれにしても、ジュレ(親水性アミノ酸であるコラーゲンが主体)を加えたことで、親水基と疎水基を併せ持つ卵黄レシチンの乳化作用が崩れてマヨネーズが分離した結果だということには気付いていなかったと思われる。}。こういうふうに浸みが出ることを防ぐには、どんな場合でも、このコーティング用マヨネーズではなく魚料理用ソース・ショフロワを用いることをお勧めする\footnote{この『料理の手引き』ではジュレを加えたマヨネーズの使用に否定的だが、カレーム『19世紀フランス料理』ではSauce
  Magnonaiseとして、まず最初にジュレを加えるレシピが掲載されている。概略を示すと、氷の上に置いた陶製の容器に卵黄2個、塩、白こしょう少々、エストラゴンヴィネガー少々を入れる。木のさじで素早くかき混ぜる。まとまってきたら、エクス産の油大さじ1杯とヴィネガー少々を、少しずつ加えていく。容器の壁に叩きつけるようにしてソースを泡立てていく。この作業でマニョネーズの白さが決まるという。また、油をごく少量ずつ加えていくことを強調している。粘度が出て滑らかになったら、最後に油をグラス二杯(≒2
  dL)と\textbf{アスピック用ジュレ}をグラス
  \(\frac{1}{2}\)杯、エストラゴンヴィネガー適量を加えて仕上げる、というもの(t.3,
  p.132.
  強調は引用者による)。また、カレームは卵黄に含まれるレシチンによって乳化作用が起きることを経験的にさえも理解していなかったようであり、卵黄を用いないマニョネーズのレシピも掲載されている。なかでも特徴的なのは、「ジュレ入りの白いマニョネーズ」のレシピで、これは氷の上に鍋を置き、大きなレードル2杯の白いジュレと同量の油、レードル1杯のヴィネガー、塩、こしょうを入れて卵白用の泡立て器でよく混ぜ、途中何回かレモン果汁を少しずつ加えて白く仕上げるようにする、というもの(\emph{ibid}.,
  p.133)。とりわけ舞踏会や格式ある大規模な宴席で魚のフィレや鶏のアスピックを飾るのに適していると述べている。}。少なくとも、そうするのが一般的になりつつある。

\hypertarget{mayonnaise-fouette-a-la-russe}{%
\subsubsection{ロシア風ホイップマヨネーズ}\label{mayonnaise-fouette-a-la-russe}}

\frsub{Sauce Mayonnaise fouettée, à la Russe}

\index{そーす@ソース!04れいせい@★冷製---!ろしあふうほいつふまよねーす@ロシア風ホイップマヨネーズ}
\index{まよねーす@マヨネーズ!ろしあふうほいつふ@ロシア風ホイップ---}
\index{そーす@ソース!ろしあふうほいつふまよねーす@ロシア風ホイップマヨネーズ}
\index{ろしあふう@ロシア風!ほいつふまよねーす@---ホイップマヨネーズ}
\index{sauce@sauce!sauce froide@sauce froide!mayonnaise fouettee russe@--- fouettée à la Russe}
\index{sauce@sauce!mayonnaise fouettee russe@--- Mayonnaise fouettée à la Russe}
\index{mayonnaise@mayonnaise!sauce fouettée russe@Sauce --- fouettée à la Russe}
\index{russe@russe!sauce mayonnaise fouettee@Sauce Mayonnaise fouettée à la ---}

陶製かホーローの容器に、溶かしたジュレ4 dLとマヨネーズ3
dL、エストラゴンヴィネガー大さじ1杯、おろしてさらに細かく刻んだレフォール\footnote{ホースラディッシュ、西洋わさび。}大さじ
1杯を入れる。

全体を混ぜ、容器を氷の上に置いて泡立て器でホイップする。ムース状になり、軽く固まり始めるまで、つまりこのソースを使うのに充分な流動性がある状態のところで作業をやめる\footnote{分量比率を考えると、構造的には前項の注で言及したカレームのジュレを主体としたマニョネーズに近いものと思われる。}。\ldots{}\ldots{}主に、野菜のサラダを型に詰めて固めるのに用いる。

\hypertarget{mayonnaises-divierses}{%
\subsubsection{マヨネーズのバリエーション}\label{mayonnaises-divierses}}

\frsub{Sauce Mayonnaise diverses}

\index{そーす@ソース!04れいせい@★冷製---!まよねーすのはりえーしよん@マヨネーズのバリエーション}
\index{まよねーす@マヨネーズ!はりえーしよん@---のバリエーション}
\index{そーす@ソース!まよねーすのはりえーしよん@マヨネーズのバリエーション}
\index{sauce@sauce!sauce froide@sauce froide!mayonnaises diverses@--- Mayonnaise diverses}
\index{sauce@sauce!mayonnaises dieverses@---s Mayonnaises diverses}
\index{mayonnaise@mayonnaise!sauces diverses@Sauces --- diverses}

オードブルや冷製料理に合わせるのに、大型甲殻類\footnote{homard
  オマール、langouste ラングースト(≒伊勢エビ)など。}およびエクルヴィス\footnote{ざりがにのこと。詳しくは\protect\hyperlink{sauce-bavaroise}{バイエルン風ソース}訳注参照。}の卵やクリーム状の部分を用いたり、クルヴェット\footnote{小海老のこと。詳しくは\protect\hyperlink{sauce-aux-crevettes}{ソース・クルヴェット}訳注参照。}、キャビア、アンチョビなどを加えることでマヨネーズにバリエーションを付けることが出来る。

上記の材料のいずれかをすり潰してから少量のマヨネーズを加えてピュレ状にして布で漉す。これを適量のマヨネーズに混ぜ合わせればよい。

\hypertarget{sauce-mousquetaire}{%
\subsubsection[ソース・ムスクテール]{\texorpdfstring{ソース・ムスクテール\footnote{マスケット銃兵、近衛騎兵、の意。日本でも子どもむけに翻案されたもので有名な19世紀のアレクサンドル・デュマ(ペール)の小説
  \emph{Les Trois Mousquetaires} 『三銃士』の「銃士」がこれに相当する。}}{ソース・ムスクテール}}\label{sauce-mousquetaire}}

\frsub{Sauce Mousquetaire}

\index{そーす@ソース!04れいせい@★冷製---!むすくてーる@---・ムスクテール}
\index{むすくてーる@ムスクテール!そーす@ソース・---}
\index{そーす@ソース!むすくてーる@---・ムスクテール}
\index{sauce@sauce!sauce froide@sauce froide!mousquetaire@--- Mousquetaire}
\index{sauce@sauce!mousquetaire@--- Mousquetaire}
\index{mousquetaire@mousquetaire!sauce@Sauce ---}

マヨネーズ1 Lに以下を加える。ごく細かいみじん切りにしたエシャロット80 g
を白ワイン1 \(\frac{1}{2}\)
dLに加えてほとんど煮詰めたもの。溶かした\protect\hyperlink{glace-de-viande}{グラスドヴィアンド}大さじ3杯、シブレット\footnote{チャイヴ。アサツキとも訳されることがあるが、日本のものとは風味が異なるので注意。}を細かく刻んだもの大さじ1杯強。カイエンヌごく少量かミルで挽いたこしょう少々で風味を引き締める。

\ldots{}\ldots{}羊、牛肉の冷製料理に添える。

\hypertarget{sauce-moutarde-a-la-creme}{%
\subsubsection{生クリーム入りソース・ムタルド}\label{sauce-moutarde-a-la-creme}}

\frsub{Sauce moutarde à la crème}

\index{そーす@ソース!04れいせい@★冷製---!なまくりーむいりむたると@生クリーム入り---・ムタルド}
\index{そーす@ソース!むたるとなまくりーむいり@生クリーム入り---・ムタルド(冷製)}
\index{むたると@ムタルド(マスタード)!そーすなまくりーむいり@生クリーム入りソース・---(冷製)}
\index{ますたーと@マスタード(ムタルド)!そーすなまくりーむいり@生クリーム入りソース・ムタルド}
\index{sauce@sauce!sauce froide@sauce froide!moutarde creme@--- moutarde à la crème}
\index{sauce@sauce!moutarde creme@--- moutarde à la crème (froide)}
\index{moutarde@moutarde!sauce creme@Sauce --- à la crème (froide)}

陶製の容器にマスタード大さじ3杯と塩1つまみ、こしょう少々とレモン果汁少々を入れて混ぜ合わせる。ここに少しずつ、マヨネーズを作る要領で、ごく新鮮なクレーム・エペス\footnote{乳酸醗酵させた、とても濃度のある生クリーム。}約2
dLを加える。

\ldots{}\ldots{}オードブル用。

\hypertarget{sauce-raifort-aux-noix}{%
\subsubsection{くるみ入りソース・レフォール}\label{sauce-raifort-aux-noix}}

\frsub{Sauce Raifort aux noix}

\index{そーす@ソース!04れいせい@★冷製---!くるみいりれふおーる@くるみ入り---・レフォール}
\index{そーす@ソース!くるみいりれふおーる@くるみ入り---・レフォール(冷製)}
\index{れふおーる@レフォール(ホーシュラディッシュ)!くるみいりそーす@くるみ入りソース・---(冷製)}
\index{くるみ@くるみ!くるみいりれふおーる@---入りソース・レフォール}
\index{sauce@sauce!sauce froide@sauce froide!raifort noix@--- Raifort aux noix}
\index{sauce@sauce!raifort noix@--- Raifort aux noix (froide)}
\index{raifort@raifort!sauce noix@Sauce --- aux noix (froide)}
\index{noix@noix!sauce@sauce!raufort@Sauce Raifort aux --- (froide)}

陶製の器に、おろしたレフォール250 gと皮を剥いて刻んだくるみ250 g、塩5
g、砂糖15 g、クレーム・エペス3 dLを入れて混ぜ合わせる。

\ldots{}\ldots{}オンブルシュヴァリエ\footnote{サケ科の淡水魚。体長20〜30
  cmのものが多く、最大で70
  cmを越えるものもいるという。日本の岩魚に近い。フランスではアルプスのドイツおよびイタリアとの国境付近に生息するが、現代では養殖も多いという。}の冷製用。

\hypertarget{sauce-ravigote-froide}{%
\subsubsection[ソース・ラヴィゴット/ヴィネグレット]{\texorpdfstring{ソース・ラヴィゴット/ヴィネグレット\footnote{ラヴィゴットの意味などについてはホワイト系派生ソースの\protect\hyperlink{sauce-ravigote}{ソース・ラヴィゴット}参照。現代フランス語の
  vinaigrette(ヴィネグレット)はいわゆる「ドレッシング」を指す。語源的にはヴィネガーを意味するvinaigre(ヴィネーグル)に縮小辞
  -ette
  を付けたもの。ヴィネグレットという名称のレシピとしてもっとも古いのは14世紀に成立したとされる「タイユヴァン」のもので、いわゆる「ヴァチカン写本」に収録されており、\ul{Potaige Lyans}「とろみを付けた煮物」に分類されている。概要を示すと、menue-hasteムニュアット(豚の脾臓およびレバー半分と腎臓)をローストする。火を通しすぎないよう注意。それを切り分けて、鍋にラード、輪切りにした玉ねぎとともに入れて、炭火にかけ、よく混ぜながら火を通す。全体によく火が通ったら、牛のブイヨンとワインを注いで沸かす。マニゲット、サフランなどを鉢でよくすり潰したらヴィネガーでのばして加え、再度沸騰させる。全体にとろみがあって茶色に仕上げる、というもの(p.222)。これをほぼ書き写したと思われる14世紀末に書かれた『ル・メナジエ・ド・パリ』のレシピでは、肉の下処理としてよく洗ってから湯通しすること、とろみ付けの要素としてこんがり焼いたパンを香辛料とともにすり潰してワインとヴィネガーで溶く、という指示が追加されている。また、こんがり焼いたパンを使わずに茶色に仕上げられるわけがない云々という『ル・メナジエ・ド・パリ』の筆者自身の感想も記されている。15世紀に書かれたシカールの『料理について』でも豚のレバーを焼いてから煮込みヴィネガーを加えるもので、細部は違うが基本的に似たものであり、中世においては豚レバーを煮込んでヴィネガーで味付けしたもの、ということになる。これが変化したと思われるのは17世紀。1693年刊マシアロ『宮廷およびブルジョワ料理の本』にはBoeuf,
  Vinaigretteというレシピがあり、牛肉に背脂を刺して塩茹でして冷まし、ヴィネガーをひと垂らししてレモンのスライスを添えるというとても単純なもの。ところが、1694年のアカデミーフランセーズの辞書には既に「ヴィネガー、油、塩、こしょう、パセリ、シブール{[}葱{]}」で作る冷製ソース」という定義がなされている。こんにち我々がイメージするヴィネグレットの定義にほぼ近い。おおむね17世紀以降、とりわけ後半にヴィネガーと油、塩を合わせた冷製ソースというコンセンサスが形成されたと想像される。}}{ソース・ラヴィゴット/ヴィネグレット}}\label{sauce-ravigote-froide}}

\frsub{Sauce Ravigote, ou Vinaigrette}

\index{そーす@ソース!04れいせい@★冷製---!らういこつと@ラヴィゴット(ヴィネグレット)}
\index{そーす@ソース!らういこつと@---・ラヴィゴット(冷製)}
\index{らういこつと@ラヴィゴット!そーす@ソース・---(冷製)}
\index{ういねくれつと@ヴィネグレット ⇒ソース・ラヴィゴット(冷製)}
\index{sauce@sauce!sauce froide@sauce froide!ravigote@--- ravigote, ou vinaigrette}
\index{sauce@sauce!ravigotte froide@--- Ravigote, ou Vinaigrette (froide)}
\index{ravigote@ravigote!sauce@Sauce --- , ou vinaigrette (froide)}
\index{vinaigrette@vinaigrette ⇒ sauce ravigote (froide)}

\hypertarget{ux6750ux6599}{%
\subparagraph{材料}\label{ux6750ux6599}}

\ldots{}\ldots{}油5 dL、ヴィネガー2
dL、小さめのケイパー小さじ2杯、パセリ50
g、セルフイユとエストラゴン、シブレットを刻んだもの40
g、細かくみじん切りにした玉ねぎ70 g、塩4 g、こしょう1
g。以上をよく混ぜ合わせる。

\ldots{}\ldots{}仔牛の頭や足、羊の足などに合わせる。

\hypertarget{sauce-remoulade}{%
\subsubsection[ソース・レムラード]{\texorpdfstring{ソース・レムラード\footnote{ソース名としての初出はおそらくムノン『ブルジョワ屋敷勤めの女性料理人のための本』(1734)におけるSauce
  à la rémolade
  {[}sic.{]}だろう。レシピの概要は、エシャロット、パセリ、シブール、にんにく1片、アンチョビ、ケイパー、いずれもごく細かく刻んで鍋に入れ、塩、粗挽きこしょうを加え、マスタード少々と油、ヴィネガーでのばす、というもの。つまり、乳化ソースであるマヨネーズをベースにした本書のレムラードと、乳化させないという点が異なるのみで、基本的なところは共通していると見ていい。ヴィアール『帝国料理の本』第7版(1812年)にはRémouladeの綴りで、緑色のレムラード、レムラード、インド風レムラードと3種のレシピが掲載されている(この版にはまだマヨネーズのレシピは掲載されていない)。このうちのレムラードのレシピの概要は、グラス1杯のマスタードを器に入れ、エシャロットのみじん切り少々と香草少々を加える。油を大さじ6〜
  7杯、ヴィネガー大さじ3〜4杯、塩、粗挽きこしょうを加える。これらをよく混ぜ合わせ、生の卵黄2個を加えてさらによく混ぜる。ソースがよくまとまるように気をつけてしっかり綷。やや濃い仕上がりにする、というもの(p.53)。手順的にはやや異なるが、卵黄を用いて乳化させようとしていることがわかる。緑のレムラードも生の卵黄を用いるなど、香草をすり潰すことと、ほうれんそうの緑の色素を用いる以外はレムラードと同様。なお、インド風レムラードの場合は固茹で卵の卵黄10個をよくすり潰して大さじ8杯の油を加えてさらによく混ぜる。唐辛子とターメリックの粉末、塩、こしょう、ヴィネガーを加える。出来るだけ粘りが出るようにする。これを布で漉して供する(id.)。カレームに至るとさらにレシピは洗練されたものとなり、Sauce
  Rémoulade à la
  Ravigote(ソース・レムラード・アラ・ラヴィゴット)では、セルフイユとエストラゴン、サラダバーネット、シブレットを茹がいて水にさらした後に水気を搾り、固茹で卵の卵黄を加えてよくすり潰し、塩、こしょう、ナツメグで調味して、上等のマスタードを加える。ここにエクス産の油とエストラゴンヴィネガーを少しずつ加えていく。最後に布で漉す(t.1,
  p.135)というもの。いずれにしてもマヨネーズを基本ソースとして展開するという『料理の手引き』の発想、体系化にいたるまで100年近くを要したことになる。}}{ソース・レムラード}}\label{sauce-remoulade}}

\frsub{Sauce Rémoulade}

\index{そーす@ソース!04れいせい@★冷製---!れむらーと@---・レムラード}
\index{れむらーと@レムラード!そーす@ソース・---}
\index{そーす@ソース!れむらーと@---・レムラード}
\index{sauce@sauce!sauce froide@sauce froide!remoulade@--- Rémoulade}
\index{sauce@sauce!remoulade@--- Rémoulade}
\index{remoulade@rémoulade!sauce@Sauce ---}

\protect\hyperlink{mayonnaise}{マヨネーズ}1
Lに以下のものを加える。マスタード大さじ 1
\(\frac{1}{2}\)杯。コルニション100とケイパー50gを細かく刻んで、圧して余分な水気を絞ったもの。パセリ、セルフイユ、エストラゴンのみじん切り大さじ1
杯。アンチョビエッセンス大さじ \(\frac{1}{2}\)杯。

\hypertarget{sauce-russe-froide}{%
\subsubsection{ロシア風ソース}\label{sauce-russe-froide}}

\frsub{Sauce Russe}

\index{そーす@ソース!04れいせい@★冷製---!ろしあふう@ロシア風---}
\index{ろしあふう@ロシア風!そーすれいせい@---ソース(冷製)}
\index{そーす@ソース!ろしあふうれいせい@ロシア風---(冷製)}
\index{sauce@sauce!sauce froide@sauce froide!russe@--- Russe}
\index{sauce@sauce!russe@--- Russe (froide)}
\index{russe@russe!sauce froide@Sauce --- (froide)}

鉢に、オマール\footnote{homard ロブスター。}かラングースト\footnote{langouste
  ≒ 伊勢エビ。}の胴のクリーム状の部分100 gとキャビア100 g\footnote{チョウザメの卵の塩蔵品のことだが、「高級」とされる順に、beluga
  (ベルガ)、osciètre, ossetra(オシエートル、オセトラ)、
  sevruga(セヴルガ)の種類がある(ここで示した読みがなはフランス語風のもの)。}、マヨネーズ大さじ2〜3杯を加えてよくすり潰す。これを目の細かい漉し器で裏漉しする。こうして出来たピュレに、マヨネーズ
\(\frac{3}{4}\)
Lを加える。大さじ1杯強のマスタードと、同量のダービーソース\footnote{初版では原注として、風味付けにマスタードを加えることを示唆しているのみ。第二版では「マスタードとウスターシャソースを各大さじ1杯強」、第三版では「マスタードとエスコフィエソースを大さじ1杯強」と変遷している。なお、ダービーソースDerby
  Sauce
  の1946年の広告には、このブランド名でバーベキューソース、ステーキソース、ウスターシャソース、ホットソース、チャプスイソースのラインナップが記されている。現実問題として、もし加えるとするならリー\&ペリンのようなウスターシャソースということになろうか。}を加えて仕上げる。

\ldots{}\ldots{}魚および甲殻類の冷製料理に添える。

\hypertarget{sauce-tartare}{%
\subsubsection[タルタルソース]{\texorpdfstring{タルタルソース\footnote{タルタル(タタール)=フランス人から見て東方の蛮族、というイメージで語られがちだが、カレーム『19世紀フランス料理』にあるSauce
  Rémoulade à la Mogol
  {[}Mongoleの誤植と思われる{]}「モンゴル風ソース・レムラード」およびSauce
  à la
  Tartare「タルタル風ソース」のレシピを見るかぎり、誤解という可能性も感じられる。前者は固茹で卵の卵黄に塩、こしょう、ナツメグ、カイエンヌ、砂糖、油、エストラゴンヴィネガーを合わせてピュレ状にして布で漉し、サフランを煎じた汁で美しい黄色に染め、刻んだシブレットを加えて仕上げるというもの。後者はソース・アルマンドとマスタード同量に生の卵黄2個を加え、塩、こしょう、ナツメグで調味してエクス産の油レードル2杯分とレードル
  \(\frac{1}{2}\)杯のエストラゴンヴィネガーを少しずつ加えながら混ぜていく。みじん切りにして下茹でしたエシャロット少々とにんにく少々、エストラゴンとセルフイユのみじん切りを大さじ1杯加える、というもの(pp.137-138)。少なくともこれらのレシピにおいて、タルタルすなわち野蛮、というニュアンスを見出すことは出来ないだろう。なお、Steak
  tartareタルタルステーキのレシピは本書には掲載されておらず、1938年の『ラルース・ガストロノミック』初版が初出と思われる(p.1019)。}}{タルタルソース}}\label{sauce-tartare}}

\frsub{Sauce Tartare}

\index{そーす@ソース!04れいせい@★冷製---!たるたる@タルタル---}
\index{たるたる@タルタル!そーすれいせい@---ソース(冷製)}
\index{そーす@ソース!たるたるれいせい@タルタル---(冷製)}
\index{sauce@sauce!sauce froide@sauce froide!tartare@--- Tartare}
\index{sauce@sauce!tartare@--- Tartare (froide)}
\index{tartare@tartare!sauce froide@Sauce --- (froide)}

固茹で卵の黄身8個をすり潰して滑らかになるまでよく練る。塩、挽きたてのこしょう各1つまみ強で味付けする。油1
Lとヴィネガー大さじ2杯を加えながらソースを立てていく\footnote{明記されていないが、\protect\hyperlink{mayonnaise}{マヨネーズ}や\protect\hyperlink{sauce-gribiche}{ソース・グリビッシュ}と同様に作業すること。}。若どりの玉ねぎ\footnote{いわゆる「オニオンヌーヴォー」だが、日本でこの名称で流通しているものは黄色系の品種が多いのに対し、フランスでは白系品種(oignon
  blanc オニョンブロン)が多く、風味が異なることに注意。}の葉またはシブレット20
gをすり潰してマヨネーズ大さじ2杯でのばし、目の細かい網で裏漉ししたものを加えて仕上げる。

\ldots{}\ldots{}このソースは、冷製の家禽や肉料理、魚料理、甲殻類いずれにも合う。また、「ディアーブル(悪魔風)」仕立ての肉料理、鶏料理にも用いられる。

\hypertarget{sauce-verte}{%
\subsubsection[ソース・ヴェルト]{\texorpdfstring{ソース・ヴェルト\footnote{緑のソース、の意。この名称のソースは中世からある。このレシピではほうれんそうとクレソンが主体になっているが、時代とともにその材料には変遷がある。中世においては、麦の若葉をすり潰して用いるレシピが多かった。}}{ソース・ヴェルト}}\label{sauce-verte}}

\frsub{Sauce Verte}

\index{そーす@ソース!04れいせい@★冷製---!うえると@---・ヴェルト}
\index{うえーる@ヴェール/ヴェルト!そーす@ソース・ヴェルト}
\index{そーす@ソース!うえると@---・ヴェルト}
\index{sauce@sauce!sauce froide@sauce froide!verte@--- Verte}
\index{sauce@sauce!verte@--- Verte}
\index{vert@vert(e)!sauce@Sauce ---e}

ほうれんそうの葉\footnote{日本では、ほうれんそうを葉のみではなく葉軸とともに利用するのが一般的だが、伝統的なフランス料理において葉軸は使われないのが普通。そもそも日本のほうれんそうは密植して葉が立つように仕立てて比較的若どりするのに対して、ヨーロッパ品種のほうれんそうは株間を充分にとってロゼッタ状に葉が広がるように栽培するのが伝統的な手法。この場合、葉は肉厚に仕上がるが、葉軸は太くて固いため可食部と見なされなかった。昔のフランスの八百屋の店先では軸を切り捨てる作業風景がよく見られたという。現代では機械収穫に適した立性の品種が増えており、専用の大型機械で株元近くから切り取り、自動的に軸をある程度除去して併走する巨大なコンテナに移すという収穫方法が普及しており、量産品のピュレなどに使用されている。}50
gとクレソンの葉50 g、パセリの葉とセルフイユ、エストラゴンを同量ずつ計50
gを、沸騰した湯に投入し、強火で5分間茹でる。水気をきり、手早く冷水にさらす。しっかりと圧し絞って水気をきり、鉢に入れてすり潰す。これをトーション\footnote{綿などの天然素材で出来た調理場及びホール業務に用いられる布。サイズは50〜55
  cm×70〜80 cmのものが多い。}でくるんできつく絞り、葉の濃い汁を1
dL搾りだす。

固く立てて風味付けをした\protect\hyperlink{mayonnaise}{マヨネーズ}9
dLにこの緑の汁を加える。

\ldots{}\ldots{}冷製の魚料理や甲殻類に合わせる。

\hypertarget{sauce-vincent}{%
\subsubsection[ソース・ヴァンサン]{\texorpdfstring{ソース・ヴァンサン\footnote{18世紀フランスを代表する料理人のひとり、Vincent
  La
  Chapelleヴァンサン・ラシャペル(1690または1703〜1745)の名を冠したソース。チェスターフィールド伯フィリップ・スタンホープに仕えていた頃に三巻からなる『近代料理』\emph{Modern
  Cook}英語版を1733年に上梓。そのフランス語版(全4巻)は1835年に\emph{Le
  Cuisinier
  moderne}のタイトルでアムステルダムで刊行。その後、全5巻からなる第二版を1742年に自費で出版した。}}{ソース・ヴァンサン}}\label{sauce-vincent}}

\frsub{Sauce Vencent}

\index{そーす@ソース!04れいせい@★冷製---!うあんさん@---・ヴァンサン}
\index{うあんさん@ヴァンサン!そーす@ソース・---}
\index{そーす@ソース!うあんさん@---・ヴァンサン}
\index{sauce@sauce!sauce froide@sauce froide!vincent@--- Vincent}
\index{sauce@sauce!vincent@--- Vincent}
\index{vincent@Vincent!sauce@Sauce ---}

\hypertarget{sauce-vincent-1}{%
\subparagraph{作り方(1)}\label{sauce-vincent-1}}

\ldots{}\ldots{}オゼイユ\footnote{タデ科の葉菜。日本語ではソレルとも。日本のスカンポに近いが、オゼイユは野菜として品種の選抜育成が長期にわたって行なわれたことに留意。}の葉とパセリの葉、セルフイユ、エストラゴン、シブレット、サラダバーネット\footnote{pimprenelle
  パンプルネル。}のごく若い葉をきっちり同量ずつ、計100 g、クレソンの葉60
gとほうれんそうの葉60 gを沸騰した湯で強火で2〜3分間茹がく。

湯をきって、冷水にさらす。しっかり水分を圧し絞って、鉢\footnote{伝統的には大理石製の鉢がこの種の作業には用いられた。}に入れる。茹であがったばかりの固茹で卵の黄身6個を加えて滑かになるまですり潰す。

これを布で漉し\footnote{このように濃度のあるものを布で漉す方法については\protect\hyperlink{veloute}{ヴルテ}訳注参照。}、陶製の容器に移す。塩1つまみ強とこしょう適量、生の卵黄5個を加える。油8
dLとヴィネガー適量を加えながら混ぜ、滑らかに乳化させる。

風味付けにダービーソース\footnote{\protect\hyperlink{sauce-russe-froide}{ロシア風ソース}訳注参照。}大さじ1杯を加えて仕上げる。

\hypertarget{sauce-vincent-2}{%
\subparagraph{作り方(2)}\label{sauce-vincent-2}}

\ldots{}\ldots{}作り方(1)の香草と葉菜のピュレを作るところまでは同じ。

これに\protect\hyperlink{mayonnaise}{マヨネーズ}を加えて、同様に仕上げる。

\ldots{}\ldots{}冷製の魚料理、甲殻類にとりわけ合う。

\hypertarget{nota-sauce-vincent}{%
\subparagraph{【原注】}\label{nota-sauce-vincent}}

このソースは18世紀の偉大な料理人のひとり、ヴァンサン・ラシャペルが考案したもの\footnote{\protect\hyperlink{mayonnaise}{ソース・マヨネーズ}の訳注において述べたように、卵黄と植物油をベースとした乳化ソースとしてのマヨネーズの起源は判然としないところが多いが、19世紀初頭のヴィアールやカレームの記述を読むかぎりにおいて、卵黄レシチンによる油と水分の乳化作用については経験レベルでさえはっきりとは認識されていなかった。このソースあるいはこれに相当するレシピがヴァンサン・ラシャペルの著書に掲載されていないこと、ヴァンラン・ラシャペルがレストランの店主ではなく貴族に仕えていた料理人、メートルドテルであったことを考慮すると、このソースの考案者が彼である可能性も、自身の名をソース名に冠した可能性もきわめて低い。もっとも、香草の扱いを得意としていたのは事実のようで、Sauce
  en
  Ravigote(ソース・オン・ラヴィゴット)だけでも5種のレシピが掲載されている。香草と葉菜を茹でてすり潰したピュレをこれらのソースで使用していることから、後世にこの名称が付いた、あるいはこのソースの最大のポイントがヴァンサン・ラシャペルを思わせる香草のピュレだと考えるのが妥当だろう。}。

\hypertarget{sauce-suedoise}{%
\subsubsection[スウェーデン風ソース]{\texorpdfstring{スウェーデン風ソース\footnote{基本的にソース名はアルファベット順に掲載されているのだが、このソースだけが後からとって付けたように末尾にある。実際、このレシピは第二版から掲載となっているが、ある程度組版が進んだ段階で急遽追加されたのだろうか。なお、1907年の英語版には掲載されていない。原注の最後「このソースはマスタードで風味付けしてもいい」は第四版で追加されたものだが、他は第二版からまったく異同がなく、掲載順も変化していないのはいささか不思議なところ。}}{スウェーデン風ソース}}\label{sauce-suedoise}}

\frsub{Sauce Suédoise}

\index{そーす@ソース!04れいせい@★冷製---!すうえーてんふう@スウェーデン風---}
\index{すうえーてんふう@スウェーデン風!そーすれいせい@---ソース(冷製)}
\index{そーす@ソース!すうえーてんふうれいせい@スウェーデン風---(冷製)}
\index{sauce@sauce!sauce froide@sauce froide!suedoise@--- Suédoise}
\index{sauce@sauce!suedoise@--- Suédoise (froide)}
\index{suedois@suédois(e)!sauce froide@Sauce ---e (froide)}

酸味のある固いリンゴを薄切りにして鍋にしっかり蓋をして煮る。普通の果肉が甘いリンゴを使う場合にはレモン果汁数滴を加えること。リンゴを煮る際には、白ワインを大さじ数杯だけ加えればいい。リンゴを煮るというよりは蒸気の圧力で溶かすイメージ。

これを目の細かい網で裏漉しする。このリンゴのピュレを2 \(\frac{1}{2}\)
dLになるまで煮詰める。充分に冷ましてから、\protect\hyperlink{mayonnaise}{マヨネーズ}
\(\frac{3}{4}\)
Lを加える。風味付けにおろした(または細かく刻んだ)レフォール大さじ1
\(\frac{1}{2}\)杯を加えて仕上げる。

\ldots{}\ldots{}このソースはとりわけ豚肉の冷製に合う。がちょうのローストの冷製にもよく合う。

\hypertarget{nota-sauce-suedoise}{%
\subparagraph{【原注】}\label{nota-sauce-suedoise}}

リンゴの時季でない場合は、リンゴのピュレの代わりに房なりの緑のグロゼイユ\footnote{すぐり。ここではホワイトカラントの若どりのものを指している。}またはグーズベリー\footnote{groseilles
  à maquereau (グロゼイユザマクロー)。}のピュレ2 \(\frac{1}{2}\)
dLを固く立てたマヨネーズ1
Lに加える。このソースはマスタードで風味付けしてもいい。
\end{recette}
\hypertarget{sauces-froides-anglaises}{%
\subsection[イギリス風ソース(冷製)]{\texorpdfstring{イギリス風ソース(冷製)\footnote{この節に収録されているレシピは初版から第四版まで、表現の異同はあるが、項目に変化はない。興味深いことに、1907年刊の英語版\emph{A
  Guide to Modern Cookery}においても全て掲載されている。}}{イギリス風ソース(冷製)}}\label{sauces-froides-anglaises}}

\frsec{Sauces Froides Anglaises}

\index{そーす@ソース!05いきりすふうれいせい@★イギリス風冷製---}
\index{いきりすふう@イギリス風!そーすれいせい@---ソース(冷製)}
\index{sauce@sauce!froides anglaises@---s froides anglaises}
\index{anglais@anglais(e)!sauces froides@sauces froides ---es}
\begin{recette}
\hypertarget{cambridge-sauce}{%
\subsubsection[ケンブリッジソース]{\texorpdfstring{ケンブリッジソース\footnote{ケンブリッジはイングランド東部のケンブリッジシャーの州都。大学都市として有名。}}{ケンブリッジソース}}\label{cambridge-sauce}}

\frsub{Sauce Cambridge\hspace{1em}\normalfont(\textit{Cambridge-Sauce})}

\index{いきりすふう@イギリス風!そーすれいせい@---ソース(冷製)!けんふりつし@ケンブリッジソース}
\index{そーす@ソース!05いきりすふうれいせい@★イギリス風冷製---!けんふりつし@ケンブリッジ---}
\index{けんふりつし@ケンブリッジ!そーす@---ソース}
\index{sauce@sauce!froides anglaises@---s froides anglaises!cambridge@--- Cambridge (Cambridge-Sauce)}
\index{cambridge@Cambridge!sauce@Sauce --- (Cambridge-Sauce)}
\index{anglais@anglais(e)!sauces froides@sauces ---es froides!cambridge@Sauce Cambridge (Cambridge-Sauce)}

固茹で卵の黄身6個と、よく洗ったアンチョビのフィレ4枚、小さめのケイパー大さじ1杯、セルフイユとエストラゴンとシブレットのみじん切りを同量ずつ計大さじ1杯を鉢に入れてよくすり潰す。マヨネーズを作る際の要領で、マスタード小さじ1杯、油1
\(\frac{1}{2}\) dL\footnote{マヨネーズを作る際の要領で、と表現しているのに対して油の量が少なく思われるが、初版は「油1
  dL」、第二版以降は「1 \(\frac{1}{2}\) dL」となっている。}とヴィネガー大さじ1杯を加える。カイエンヌごく少量で風味を引き締める。ヘラでソースを混ぜながら布で漉し
\footnote{濃度のあるソースを布で漉す方法については\protect\hyperlink{veloute}{ヴルテ}訳注参照。}、ボウルに入れる。泡立て器で軽く混ぜて滑らかにしてやり、パセリのみじん切り小さじ1杯を加えて仕上げる。

\hypertarget{cumberland-sauce}{%
\subsubsection[カンバーランドソース]{\texorpdfstring{カンバーランドソース\footnote{イングランド北部の旧カウンティ(行政区分、ほぼ「州」と考えていい)のひとつ。現在はウェストモーランド、ランカシャー、ヨークシャーの一部と統合され、カンブリアとなっている。}}{カンバーランドソース}}\label{cumberland-sauce}}

\frsub{Sauce Cumberland\hspace{1em}\normalfont(\textit{Cumberland-Sauce})}

\index{いきりすふう@イギリス風!そーすれいせい@---ソース(冷製)!かんはーらんと@カンバーランドソース}
\index{そーす@ソース!05いきりすふうれいせい@★イギリス風冷製---!かんはーらんと@カンバーランド---}
\index{かんはーらんと@カンバーランド!そーす@---ソース}
\index{sauce@sauce!froides anglaises@---s froides anglaises!cumberland@--- Cumberland (Cumberland-Sauce)}
\index{cumberland@Cumberland!sauce@Sauce --- (Cumberland-Sauce)}
\index{anglais@anglais(e)!sauces froides@sauces ---es froides!cumberland@Sauce Cumberland (Cumberland-Sauce)}

鍋に\protect\hyperlink{gelee-de-groseilles-a}{グロゼイユのジュレ}大さじ4杯を入れて溶かし、そこにポルト酒1
dL と細かいみじん切りにして下茹でして水気を絞ったエシャロット大さじ
\(\frac{1}{2}\)杯、オレンジの表皮と\footnote{zeste
  ゼスト。柑橘類の硬い外皮をrâpe(ラプ)と呼ばれる器具を用いておろした場合にもこの語を用いる。}とレモンの表皮を薄く剥いてごく細い千切りにしてしっかり下茹でしてよく水気をきって冷ましたもの各大さじ1杯、オレンジ1個の搾り汁、レモン
\(\frac{1}{2}\)個分の搾り汁、マスタード小さじ1杯、カイエンヌごく少量、粉末の生姜少々を加える。

全体をよく混ぜる。

\ldots{}\ldots{}大型ジビエの冷製に合わせる。

\hypertarget{gloucester-sauce}{%
\subsubsection[グロスターソース]{\texorpdfstring{グロスターソース\footnote{イングランド南部、グロースターシャーの州都。}}{グロスターソース}}\label{gloucester-sauce}}

\frsub{Sauce Gloucester\hspace{1em}\normalfont(\textit{Gloucester-Sauce})}

\index{いきりすふう@イギリス風!そーすれいせい@---ソース(冷製)!くろすたー@グロスターソース}
\index{そーす@ソース!05いきりすふうれいせい@★イギリス風冷製---!くろすたー@グロスター---}
\index{くろすたー@グロスター!そーす@---ソース}
\index{sauce@sauce!froides anglaises@---s froides anglaises!gloucester@--- Gloucester (Gloucester-Sauce)}
\index{gloucester@Gloucester!sauce@Sauce --- (Gloucester-Sauce)}
\index{anglais@anglais(e)!sauces froides@sauces ---es froides!gloucester@Sauce Gloucester (Gloucester-Sauce)}

ごく固く立てた\protect\hyperlink{mayonnaise}{マヨネーズ}1 Lに、レモン
\(\frac{1}{2}\)個分の搾り汁を加えたサワークリーム2
dLと、細かく刻んだフェンネル1つまみ、ダービーソース\footnote{初版と第二版は「ウスターシャソース数滴」、第三版は「エスコフィエソース数滴」となっている。ダービーソースについては\protect\hyperlink{sauce-russe-froide}{ロシア風ソース}訳注も参照のこと。}大さじ2杯を加える。

\ldots{}\ldots{}主として肉の冷製料理に合わせる。

\hypertarget{mint-sauce}{%
\subsubsection{ミントソース}\label{mint-sauce}}

\frsub{Sauce Menthe\hspace{1em}\normalfont(\textit{Mint-Sauce})}

\index{いきりすふう@イギリス風!そーすれいせい@---ソース(冷製)!みんと@ミントソース}
\index{そーす@ソース!05いきりすふうれいせい@★イギリス風冷製---!みんと@ミント---}
\index{みんと@ミント!そーす@---ソース}
\index{sauce@sauce!froides anglaises@---s froides anglaises!menthe@--- Menthe (Mint-Sauce)}
\index{menthe@menthe!sauce@Sauce --- (Mint-Sauce)}
\index{anglais@anglais(e)!sauces froides@sauces ---es froides!menthe@Sauce Menthe (Mint-Sauce)}

ミントの葉50
gをごく細い千切りか、みじん切りにする。これをボウルに入れて、白いカソナード\footnote{通常cassonadeすなわち粗糖は褐色のものが多い。}かパウダーシュガー25
gとヴィネガー1 \(\frac{1}{2}\)
dL、塩1つまみ、水大さじ4杯を加える。全体によく混ぜること。

\ldots{}\ldots{}仔羊\footnote{本書で仔羊agneau(アニョー)と言う場合はほぼ例外なく乳呑仔羊、
  agneau de
  lait(アニョードレ)を意味する。現代は仔羊という語の意味する範囲が広くなり、牧草および飼料によりある程度まで肥育した羊の赤身肉も「仔羊」として扱うが、乳呑仔羊は白身肉なので注意。}の温製、冷製に添える。

\hypertarget{oxford-sauce}{%
\subsubsection[オックスフォードソース]{\texorpdfstring{オックスフォードソース\footnote{イングランド東部、オックスフォードシャーの州都。英語圏では最古の大学であるオックスフォード大学を中心とした学園都市として有名。}}{オックスフォードソース}}\label{oxford-sauce}}

\frsub{Sauce Oxford\hspace{1em}\normalfont(\textit{Oxford Sauce})}

上述の\protect\hyperlink{cumberland-sauce}{カンバーランドソース}と同様に作るが、以下の2点を変更する\footnote{オレンジとレモンの皮の扱いと量を変えただけで別のソースとして扱うことに疑問はあるが、これについては初版から一貫してまったく説明がない。何らかのエピソードがこれらのソース名にはあったと思われるが不明。}。

\begin{enumerate}
\def\labelenumi{\arabic{enumi}.}
\item
  オレンジとレモンの外皮は千切りにするのではなく、器具を用いておろすこと。
\item
  その量は半分にする。つまり、おろした外皮はそれぞれ大さじ
  \(\frac{1}{2}\)杯にすること。
\end{enumerate}

\ldots{}\ldots{}用途はカンバーランドソースと同じ。

\hypertarget{cold-horseradish-sauce}{%
\subsubsection{ホースラディッシュソース}\label{cold-horseradish-sauce}}

\frsub{Sauce Raifort\hspace{1em}\normalfont(\textit{Cold horseradish sauce})}

\index{いきりすふう@イギリス風!そーすれいせい@---ソース(冷製)!ほーすらていつしゆ@ホースラディッシュソース}
\index{そーす@ソース!05いきりすふうれいせい@★イギリス風冷製---!ほーすらていつしゆ@ホースラディッシュ---}
\index{ほーすらていつしゆ@ホースラディッシュ!そーす@---ソース}
\index{sauce@sauce!froides anglaises@---s froides anglaises06!raifort@--- Raifort (Cold horseradish sauce)}
\index{raifort@raifort!sauce@Sauce --- (Cold horseradish sauce)}
\index{anglais@anglais(e)!sauces froides@sauces ---es froides!raifort@Sauce Raifort (Cold horseradish sauce)}

陶製の器に、マスタード大さじ1杯、細かくおろしたレフォール50
g、パウダーシュガー50 g、塩1つまみ、生クリーム5
dL、牛乳に浸してからよく圧したパンの身250
g、ヴィネガー大さじ2杯を入れて混ぜ合わせる。

\ldots{}\ldots{}このソースは茹でた牛肉やローストに合わせる。よく冷やしてから供すること。

\hypertarget{horseradish-sauce}{%
\subparagraph{【原注】}\label{horseradish-sauce}}

ソースにヴィネガーを加えるのは作業の最後とすること。
\end{recette}\newpage
\href{未、原文対照チェック}{} \href{未、日本語表現校正}{}
\href{未、その他修正}{} \href{未、原稿最終校正}{}

\hypertarget{beurres-composes}{%
\section{合わせバター}\label{beurres-composes}}

\vspace{-1\zw}
\begin{center}
\textbf{グリル、ソースの補助材料、オードブル用}
\end{center}
\vspace{1\zw}

\frsec{Beurres Composés pour Adjuvants de Sauces et Hors-d'oeuvre}

\index{あわせはたー@合わせバター} \index{はたー@バター ⇒ 合わせバター}
\index{ふーるこんほせ@ブール・コンポゼ ⇒ 合わせバター}
\index{みつくすはたー@ミックスバター ⇒ 合わせバター}
\index{beurre@beurre!beurres composes@---s Composés}

\hypertarget{observation-sur-les-beurres-composes}{%
\subsection{概説}\label{observation-sur-les-beurres-composes}}

本書においてレシピを掲載している合わせバター\footnote{beurre composé
  ブール・コンポゼ。ミックスバターとも。少なくともバターは中世以来用長く用いられてきた食材だが、中世〜ルネサンスにおいては獣脂(もっぱらラード)のほうが多く用いられる傾向にあった。17
  世紀以降はたとえばラ・ヴァレーヌ『フランス料理の本』におけるアスパラガスの白いソース添え(\protect\hyperlink{sauce-hollandaise}{ソース・オランデーズ}訳注参照)のように、バターを料理に用いることが中世の料理書と比較すると圧倒的に増えたのは事実である。ムノンの1741年刊『ブルジョワ屋敷に勤める女性料理人のための本』のバターの項には「良質のバターを用いることは料理でとても重要なことであり、バターが匂いを放っているようではどんな素晴しい皿も台無しだ。料理担当の女中であればこのことをよく理解しておくことと、良質なバターの価格を手に入れるのに金を惜しんではならないことを肝に銘じておくこと。最良のバターは自然な黄色をしており、白いものは大抵の場合、さして美味しくない。バルボットという植物から採った黄色で着色されたバターもある。こういうバターの色は、自然なバターの黄色よりもくすんだもので、慣れれば簡単に見分けることが出来る(p.320)」
  \textgreater{}\textgreater{}\textgreater{}\textgreater{}\textgreater{}\textgreater{}\textgreater{}
  Stashed changes}のうちのほとんどは、甲殻類の合わせバターを除いて、料理に直接用いられることがとても少ない。だが、合わせバターはさまざまなシチュエーションで役に立つ。ポタージュでは野菜の合わせバターが、その他の合わせバターはソース作りにおいて有用だ。ソースの風味と性格を明確に伝える決め手になるからだ。

だから、読者である料理人諸君には、ここに書いてあることを真剣に読みとっていただきたい。

甲殻類のバターについては、経験上、湯煎にかけながら煮出して\footnote{infuser
  (アンフュゼ)。}から、氷水で冷やした陶製の容器に布で漉し入れるといい。そうすれば、冷たい状態で作るよりも赤みがきれいに出る。だが逆に、熱によって風味の繊細さが失なわれてしまい、雑味さえも出てしまう。

この問題点を解決するために、我々は二種類の違うバターを作るという方式を採ることにした。ひとつは甲殻類の胴のクリーム状の部分と切りくずあるいは身そのものを生のバターとともに鉢ですり潰して、目の細かい網で裏漉しするか、布で漉すというもの。このバターはソースに完璧ともいうべき風味を添えてくれる。とりわけベシャメルソースをベースとしたソースの場合はそうだ。

もうひとつは、甲殻類の殻だけを用いて、熱して作るものだ。これは「色付け」の役割しか持たない。この方式はまことに素晴しい結果を得られるので、ぜひとも実行していただきたい。

場合によっては、我々はバターを同様の上等な生クリームに代えることがある。生クリームのほうがバターよりも、素材の持つ風味や香気をよく吸収する。こうすればソースやポタージュの仕上げに加えるのに文句ないクリ\footnote{coulis
  (クリ)水分のやや多いピュレをイメージするといい。「クーリ」と呼ぶ日本の調理現場は多い。}を作ることが出来るわけだ。

色付け用のバターを使うと、ソースがきれいに色付き、個性的なソースとなる。どんな場合でも、カルミン色素\footnote{コチニール色素ともいう。ラックカイガラムシなどを原料として抽出した色素。ヨーロッパでは古代から中世にかけてケルメスカイガラムシから抽出され利用されてきた、非常に歴史の古い色素。とりわけルネサン期には高級毛織物の染料として需要が高まった。また絵の具にも使用された。その後、ウチワサボテンでエンジムシを大量に養殖していた中南米を支配下に置いたスペインが、これを新大陸産のカルミンとしてヨーロッパ各国に売ることで巨万の富を得たという。かつて食品工業において多用された。
  1838年の『ラルース・ガストロノミック』初版では、「コチニールから抽出される鮮かな赤色色素で毒性はない。多くの食品に着色料として用いられている」とある。現在は使用が減りつつあり、代替品としてビーツから抽出したビートレッドなどが増えてきている。また、この本文でカルミン色素の使用を「くすんだ、情けない色合いを与える」として否定的に扱っているのは、この色素がpHによって色調が変化し、なおかつ蛋白質を多く含む料理に加えると色素自体が紫色に変化する(結果としてソースやポタージュ全体が濁ったような色になる)ことがあるためだろう。}よりもずっといい。カルミン色素はソースやポタージュにくすんだ、なさけない色合いしか与えてはくれないのだ。

合わせバターは一般的に、使う際にその都度作る\footnote{原文 au moment
  (オモモン)その都度、の意。à la minute
  (アラミニュット)と呼ぶ調理現場もある。}ものだが、作り置きしておかなければならない場合は、白い紙で円筒形に包んで冷蔵保管すること。
\begin{recette}
\hypertarget{beurre-d-ail}{%
\subsubsection{にんにくバター}\label{beurre-d-ail}}

\frsub{Beurre d'Ail}

\index{あわせはたー@合わせバター!にんにくはたー@にんにくバター}
\index{にんにく@にんにく!はたー@---バター}
\index{beurre@beurre!--- d'ail@--- d'Ail}
\index{ail@ail!beurre@Beurre d'---}

皮を剥いたにんにく200 gを強火でしっかり茹でる\footnote{生のにんにくには胃腸を刺激する酵素が含まれているが、熱により不活性化するので、よく火を通す必要がある。}。よく湯をきってから、鉢に入れてすり潰し、バター250
gと合わせ、布で漉す。

\hypertarget{beurre-d-anchois}{%
\subsubsection{アンチョビバター}\label{beurre-d-anchois}}

\frsub{Beurre d'Anchois}

\index{はたー@バター!あわせはたー@合わせバター!あんちよひはたー@アンチョビバター}
\index{あわせはたー@合わせバター!あんちよひはたー@アンチョビバター}
\index{あんちよひ@アンチョビ!はたー@---バター}
\index{beurre@beurre!anchois@--- d'Anchois}
\index{anchois@anchois!beurre@Beurre d'---}

アンチョビのフィレ200
gをよく洗い、しっかり水気を絞る。これを鉢に入れて細かくすり潰す。バター250
gを加えて布で漉す。

\hypertarget{beurre-d-amande}{%
\subsubsection{アーモンドバター}\label{beurre-d-amande}}

\frsub{Beurre d'Amande}

\index{あわせはたー@合わせバター!あーもんとはたー@アーモンドバター}
\index{あーもんと@アーモンド!はたー@---バター}
\index{beurre@beurre!amande@--- d'Amande}
\index{amande@amande!beurre@Beurre d'---}

アーモンド\footnote{アーモンドには一般的なスイートアーモンド amandes
  doucesと、苦味のあるビターアーモンドamande
  amèresの二種がある。後者はごく微量の青酸化合物を含むのであまり多く使われることはないが、香りがいいためリキュールなどの香り付けにごく少量が用いられることがある。}150
gを湯むきしてよく洗い、すぐに水数滴を加えてすり潰してペースト状にする。これをバター250
gと混ぜ合わせ、布で漉す。

\hypertarget{ux30d6ux30fcux30ebux30c0ux30f4ux30eaux30fcux30cc8}{%
\subsubsection[ブール・ダヴリーヌ]{\texorpdfstring{ブール・ダヴリーヌ\footnote{Aveline(アヴリーヌ)はヘーゼルナッツの仲間でセイヨウハシバミの大粒な変種。イタリア、ピエモンテ産やシチリア産が有名。}}{ブール・ダヴリーヌ}}\label{ux30d6ux30fcux30ebux30c0ux30f4ux30eaux30fcux30cc8}}

\hypertarget{beurre-d-aveline}{%
\paragraph{Beurre d'Aveline}\label{beurre-d-aveline}}

\index{あわせはたー@合わせバター!ふーるたうりーぬ@ブール・ダヴリーヌ}
\index{あうりーぬ@アヴリーヌ!ふーる@ブール・---}
\index{へーせるなつつ@ヘーゼルナッツ!ふーるたうりーぬ@ブール・ダヴリーヌ}
\index{beurre@beurre!aveline@--- d'Aveline}
\index{aveline@aveline!beurre@Beurre d'---}

アヴリーヌ150
gを焙煎して丁寧に皮を剥く。油が浮いてこないよう水を数滴加えてペースト状にすり潰す。これとバター250
gを混ぜ合わせる。目の細かい網で裏漉しするか、布で漉す。

\hypertarget{beurre-bercy}{%
\subsubsection[ブール・ベルシー]{\texorpdfstring{ブール・ベルシー\footnote{\protect\hyperlink{sauce-bercy}{ソース・ベルシー}訳注参照。}}{ブール・ベルシー}}\label{beurre-bercy}}

\frsub{Beurre Bercy}

\index{あわせはたー@合わせバター!ふーるたへるしー@ブール・ベルシー}
\index{へるしー@ベルシー!ふーる@ブール・---}
\index{beurre@beurre!bercy@--- Bercy}
\index{bercy@Bercy!beurre@Beurre ---}

白ワイン2
dLに細かく刻んだエシャロット大さじ1杯を加えて半量になるまで煮詰める。生温い程度まで冷ましてから、ポマード状に柔らかくした\footnote{ポマードは昔よく用いられた整髪料だが、現代では珍しいものとなってしまった。ただ、この表現は定型句のひとつであるとともに、現代日本の調理現場で現在もこの表現を用いるところがあるため、あえてこの訳語を採用した。意味としては「指がすっと入る程度の柔らかさ」であり、シリコンゴムのヘラなどで混ぜやすいけれども、溶けて液体にはなっているわけではない状態のことを意味している。}バター
200 gを混ぜ込む。牛骨髄500 gをさいの目に切って\footnote{原文 couper en
  dés。フランス語のまま「デにする(切る)」と表現することもある。}、沸騰しない程度の湯で火を通し、よく湯ぎりをして加える。パセリのみじん切り大さじ1杯と塩8
g、挽きたてのこしょう1つまみ強とレモン\(\frac{1}{2}\)個分の果汁を加えて仕上げる。

\hypertarget{beurre-de-caviar}{%
\subsubsection{キャビアバター}\label{beurre-de-caviar}}

\frsub{Beurre de Caviar}

\index{あわせはたー@合わせバター!きやひあはたー@キャビアバター}
\index{きやひあ@キャビア!はたー@---バター}
\index{beurre@beurre!caviar@--- de Caviar}
\index{caviar@caviar!beurre@Beurre de ---}

圧縮キャビア\footnote{もとはロシアで雪の中の樽で保存するために圧縮したもの。キャビアのグレードはベルガ、オセトラ、セヴルガが混ざっているのが多いという。比較的安価に利用できる。}75
gを細かくすり潰す。パター250 gを加えて、布で漉す。

\hypertarget{beurre-chivry}{%
\subsubsection[ブール・シヴリ/ブール・ラヴィゴット]{\texorpdfstring{ブール・シヴリ/ブール・ラヴィゴット\footnote{それぞれの名称などについては、\protect\hyperlink{sacue-chivry}{ソース・シヴリ}および\protect\hyperlink{sauce-ravigote}{ソース・ラヴィゴット}訳注参照。}}{ブール・シヴリ/ブール・ラヴィゴット}}\label{beurre-chivry}}

\frsub{Beurre Chivry, ou Beurre Ravigote}

\index{あわせはたー@合わせバター!しうり@ブール・シヴリ}
\index{あわせはたー@合わせバター!らういこつと@ブール・ラヴィゴット}
\index{しうり@シヴリ!ふーる@ブール・---}
\index{らういこつと@ラヴィゴット!ふーる@ブール・---}
\index{beurre@beurre!chivry@--- Chivrya}
\index{beurre@beurre!ravigote@--- Ravigote}
\index{chivry@Chivry!beurre@Beurre ---}
\index{ravigote@ravitote!beurre@Beurre ---}

パセリの葉とセルフイユ、エストラゴン、シヴレット、若摘みのサラダバーネット100
gを数分間下茹でし、水にさらしてから圧して余分な水気を絞る。エシャロットのみじん切り25
gも下茹でする。これらを鉢に入れてすり潰す。

バター125 gを加え、布で漉す。

\hypertarget{beurre-colbert}{%
\subsubsection[ブール・コルベール]{\texorpdfstring{ブール・コルベール\footnote{\protect\hyperlink{sauce-colbert}{ソース・コルベール}本文および訳注参照。}}{ブール・コルベール}}\label{beurre-colbert}}

\frsub{Beurre Colbert}

\index{あわせはたー@合わせバター!ふーるこるへーる@ブール・コルベール}
\index{こるへーる@コルベール!ふーる@ブール・---}
\index{beurre@beurre!colbert@--- Colbert}
\index{colbert@Colbert!beurre@Beurre ---}

\protect\hyperlink{beurre-maitre-d-hotel}{メートルドテルバター}200
gに、溶かした\protect\hyperlink{glace-de-viande}{グラスドヴィアンド}大さじ2杯と細かく刻んだエストラゴン小さじ2杯を加える。

\hypertarget{beurre-colorant-rouge}{%
\subsubsection{色付け用の赤いバター}\label{beurre-colorant-rouge}}

\frsub{Beurre Colorant rouge}

\index{あか@赤!いろつけようのはたー@色付け用の--いバター}
\index{あわせはたー@合わせバター!いろつけようのあかいはたー@色付け用の赤いバター}
\index{ちやくしよくそざい@着色素材!いろつけようのあかいはたー@色付け用の赤いバター}
\index{beurre@beurre!colorant rouge@--- Colorant rouge}
\index{colorant@colorant!beurre rouge@Beurre --- rouge}
\index{rouge@rouge!beurre colorant@Beurre colorant ---}

出来るだけ沢山の甲殻類の殻などの残りをまとめて用意する。殻の内側、外側に張り付いている膜などをきれいに取り除く。よく乾燥させてから、鉢\footnote{伝統的には大理石製の鉢が用いられることが多かった。}
に入れて細かく粉砕して、同じ重さのバターを加える。これを湯煎にかけてよく混ぜながら溶かす。氷水を入れた陶製の器に、布で漉し入れる。固まったバターをトーション\footnote{\protect\hyperlink{sauce-verte}{ソース・ヴェルト}訳注参照。}で包み、余計な水を絞り出す。

\hypertarget{nota-beurre-colorant-rouge}{%
\subparagraph{【原注】}\label{nota-beurre-colorant-rouge}}

この色付け用のバターを作るのに用いる甲殻類の殻がどうしてもない場合は、\protect\hyperlink{beurre-de-paprika}{パプリカバター}を用いてもいいだろう。だがいずれにせよ、どんなソースであっても、仕上がりの色合いを決めるには、出来るだけ、他の植物由来の赤色着色料の使用は避けることを勧める\footnote{この原注は第三版から。原文le
  rouge colorant
  végétal直訳すると「植物由来の赤色着色料」だが。ここではおそらくカルミン色素(コチニール色素)のことと思われる(\protect\hyperlink{observation-sur-les-beurres-composes}{本節「概説」参照})。他に赤系着色料として、ベニバナ色素、紅麹などもあるが、いずれも中国や日本において発達しことを考慮すると、両大戦間である1920年頃に「避けるべき」というほど普及していたのは、実際には昆虫由来であるコチニール色素と思われる。なお、ベニバナ色素も化学的にはカルミン酸色素。また、甲殻類の殻を茹でると赤くなるが、この色素はアスタキサンチンといい、1938年に物質として「発見」された。もちろんエスコフィエをはじめとする料理人は経験上、甲殻類の殻を適度に加熱することで、タンパク質と結びついていたアスタキサンチンがタンパク質の熱変性によって遊離して取り出せることを経験的によく知っており、それを利用してこの赤いバターを考案したと考えられる。ちなみにサーモン、鮭の身の赤色もおなじアスタキサンチンによるもので、近縁種の鱒と同様に本来は白身。}。

\hypertarget{beurre-colorant-vert}{%
\subsubsection{色付け用の緑のバター}\label{beurre-colorant-vert}}

\frsub{Beurre Colorant vert}

\index{みとり@緑!いろつけようのはたー@色付け用の--のバター}
\index{あわせはたー@合わせバター!いろつけようのみとりのはたー@色付け用の緑のバター}
\index{ちやくしよくそざい@着色素材!いろつけようのみとりのはたー@色付け用の緑のバター}
\index{beurre@beurre!colorant vert@--- Colorant vert}
\index{colorant@colorant!beurre vert@Beurre --- vert}
\index{vert@vert(e)!beurre colorantl@Beurre colorant ---}

ほうれんそうの葉1
kgをよく洗い、しっかり振って水気をきる。これを鉢に入れてすり潰す。トーション\footnote{\protect\hyperlink{sauce-verte}{ソース・ヴェルト}訳注参照。}で包んで緑の汁を絞り出す。これをソテー鍋に入れて湯煎にかけ、水分を蒸発させてペースト状にする\footnote{原文
  coaguler
  凝固させる、の意。ここでは説明的に意訳した。なお、ほうれんそうに限らず、植物の緑色は葉緑素(クロロフィル)によるものであり、葉緑素はマグネシウム(苦土)を核として窒素が周囲に結びついた構造を持つ化学物質。ほうれんそうの緑が濃いのは土壌からのマグネシウム吸収能力が高いため。食品に含まれるマグネシウムはカルシウムの吸収を促す作用があるとされている。ただし、フランスの伝統的なほうれんそうの栽培方法は、夏の終わりから初秋にかけた種を蒔き、11月頃から大きくなった葉を順次かき取って収穫するというもの。露地栽培でも1株で3
  回程度は春になるまでに収穫できるとされた。なお、フランス語で食材および調理したほうれんそうは常に
  épinard\textbf{s}
  と複数形で表される。古くは14世紀の『ル・メナジエ・ド・パリ』にespinarsという綴りで言及があり「ポレ(ブレット)の一種で、葉が長くて茎が細く、緑は普通のポレよりも濃い。espinochesとも言う。四旬節の初め頃に食べる」(t.2,
  p.141)とある。、中世・ルネサンス期のフランスでは、ほうれんそうの普及はり進まなかったようで、16世紀末にオリヴィエ・ド・セールが『農業経営論』(1600年)において「比較的新しい野菜」として栽培方法も含めて紹介している。また、épinardの語源をオリヴィエ・ド・セールは種子が尖っている(épineux)からだと書いているが、実際には、この野菜が西アジア起源のものであり、ペルシャ語のaspanāḫからフランス語に入って現在のépinardという語形に至ったと考えられている。いっぽう、日本のほうれんそうはごく一部の地域を除いては、戦後高度成長期に普及した葉菜のひとつであり、じつのところ歴史は浅い。しばしば言われる東洋系、西洋系の違いにしても、普及当初からその交配品種が使われるようになっていたために、あまり意味はない。日本で青果として流通しているほうれんそうのほとんどは、密植、立性にして比較的若どり(農協などの出荷団体によって違うが、概ね草丈25cm程度で5株から10株で200
  gの規格が平均的)のため、用いている品種がほぼ西洋系のものを交配親としている場合でも、立性に栽培するために、葉の厚みなどはあまり問題とされていない。上述のように、フランスではかつて、葉以外を可食部として見なさず、軸を切り捨てるのが普通だったことと比べると、食文化の違いの大きさがよくわかる一例だろう。}。

これを、ぴんと張ったナフキンの上に移し、さらに水気をきる。

パレットナイフを使って緑の色素を集め、鉢に入れてその倍の重さのバターを加えて練り込む。

\ldots{}\ldots{}布で漉し、冷蔵保存する。

\hypertarget{nota-beurre-colorant-vert}{%
\subparagraph{【原注】}\label{nota-beurre-colorant-vert}}

人工的な色素よりもこの緑の色素を用いたほうが利点が大きい。

\hypertarget{beurre-de-crevettes}{%
\subsubsection{クルヴェットバター}\label{beurre-de-crevettes}}

\frsub{Beurre de crevettes}

\index{あわせはたー@合わせバター!くるうえつとはたー@クルヴェットバター}
\index{くるうえつと@クルヴェット!はたー@---バター}
\index{beurre@beurre!crevette@--- de Crevettes}
\index{crevette@crevette!beurre@Beurre de ---s}

クルヴェット・グリーズ\footnote{フランスで好んで食される小海老の一種。\protect\hyperlink{sauce-aux-crevettes}{ソース・クルヴェット}訳注参照。}150
gを鉢に入れて細かくすり潰す。バター150 gを加えて、布で漉す。

\hypertarget{beurre-d-echalote}{%
\subsubsection{エシャロットバター}\label{beurre-d-echalote}}

\frsub{Beurre d'Echalote}

\index{あわせはたー@合わせバター!えしやろつとはたー@エシャロットバター}
\index{えしやろつと@エシャロット!はたー@---バター}
\index{beurre@beurre!echalote@--- d'Echalote}
\index{echalote@echalote!beurre@Beurre d' ---}

エシャロット125
gを鉢に入れてすり潰し、さっと茹でて湯をきり、トーションに包んで圧すようにして水気を取り除く。バター125gを加えて、布で漉す。

\hypertarget{beurre-d-ecrevisse}{%
\subsubsection{エクルヴィスバター}\label{beurre-d-ecrevisse}}

\frsub{Beurre d'Ecrevisse}

\index{あわせはたー@合わせバター!えくるういすはたー@エクルヴィスバター}
\index{えくるういす@エクルヴィス!はたー@---バター}
\index{beurre@beurre!ecrevisse@--- d'Ecrevisse}
\index{ecrevisse@écrevisse!beurre@Beurre d'---}

\protect\hyperlink{}{ビスク}を作る要領で、\protect\hyperlink{mirepoix}{ミルポワ}とともに茹でたエクルヴィス\footnote{ヨーロッパざりがに。詳しくは\protect\hyperlink{sauce-bavaroise}{バイエルン風ソース}訳注参照。}の胴や殻、尾などを鉢に入れて細かくすり潰す。これと同じ重さのバターを加え、布で漉す。

\hypertarget{beurre-pour-les-escargots}{%
\subsubsection{エスカルゴ用バター}\label{beurre-pour-les-escargots}}

\frsub{Beurre pour les Escargots}

\index{あわせはたー@合わせバター!えすかるこようはたー@エスカルゴ用バター}
\index{えすかるこ@エスカルゴ!はたー@---用バター}
\index{beurre@beurre!escargots@--- pour les escargots}
\index{escargot@escargot!beurre@Beurre pour les ---s}

(エスカルゴ50個分)

バター350 gに、細かいみじん切りにしたエシャロット35
gと、にんにく1片をすり潰してペースト状にしたもの、パセリのみじん切り25
g(大さじ1杯)、塩 12 g、こしょう2
gを加える。捏ねるようにしてよく混ぜ合わせ、冷蔵する。

\hypertarget{beurre-d-estragon}{%
\subsubsection{エストラゴンバター}\label{beurre-d-estragon}}

\frsub{Beurre d'Estragon}

\index{あわせはたー@合わせバター!えすとらこんはたー@エストラゴンバター}
\index{えすとらこん@エストラゴン!はたー@---バター}
\index{beurre@beurre!estragon@--- d'Estragon}
\index{estragon@estragon!beurre@Beurre d'---}

新鮮なエスゴラゴンの葉125
gを2分間茹がいてから湯きりして冷水にさらす。圧して余分な水気を絞る。これを鉢に入れてすり潰す。バター125
gを加えて、布で漉す。

\hypertarget{beurre-de-hereng}{%
\subsubsection{にしんバター}\label{beurre-de-hereng}}

\frsub{Beurre de Hareng}

\index{あわせはたー@合わせバター!にしんはたー@にしんバター}
\index{にしん@にしん!はたー@---バター}
\index{beurre@beurre!hareng@--- de Hareng}
\index{hareng@hareng!beurre@Beurre de ---}

にしんの燻製のフィレ\footnote{原文 hareng
  saur(アロンソール)。タイセイヨウニシンの内臓を抜いて10日程塩漬けにし、塩抜き後に24〜48時間乾燥させてから15時間以上、
  32℃程度で冷燻にしたもの。強い匂いが特徴。日本のにしんとは種が異なること、スモークサーモンと同様に冷燻であることに注意。}3枚の皮を剥いて、さいの目に切り、鉢に入れて細かくすり潰す。バター250
gを加え、布で漉す。

\hypertarget{beurre-de-homard}{%
\subsubsection{オマールバター}\label{beurre-de-homard}}

\frsub{Beurre de Homard}

\index{あわせはたー@合わせバター!おまーるはたー@オマールバター}
\index{おまーる@オマール!はたー@---バター}
\index{beurre@beurre!homard@--- de Homard}
\index{homard@homard!beurre@Beurre de ---}

使える範囲の量のオマールの胴のクリーム状の部分と卵やコライユ\footnote{オマールの胴の背側にある朱色の内子。}を鉢に入れてすり潰す。それと同じ重さのバターを加え、布で漉す。

\hypertarget{beurre-de-laitance}{%
\subsubsection{白子バター}\label{beurre-de-laitance}}

\frsub{Beurre de Laitance}

\index{あわせはたー@合わせバター!しらこはたー@白子バター}
\index{しらこ@白子!はたー@---バター}
\index{beurre@beurre!laitance@--- de Laitance}
\index{laitance@laitance!beurre@Beurre de ---}

沸騰しない程度の温度で茹で、よく冷ました白子\footnote{日本ではスケトウダラの白子が一般的だが、フランスの伝統的高級料理では鯉の白子がもっとも一般的。他に鯖やにしんの白子も用いられる。}125
gを鉢に入れてすり潰す。バター250
gとマスタード小さじ1杯を加えて、布で漉す。

\hypertarget{beurre-maitre-d-hotel}{%
\subsubsection[メートルドテルバター]{\texorpdfstring{メートルドテルバター\footnote{メートルドテル
  maître d'hôtel
  とは直訳すれば「館{[}やかた{]}の主」あるいは「館の指導者」の意だが、時代および王家あるいは貴族やブルジョワの館、近現代のレストランにおいてそれぞれ異なった意味で用いられる職名。(1)王家においては
  grand
  maîtreグランメートルを補佐する仕事として食卓関連の仕事を取り仕切る職のこと。王と親しくすることが出来るために、有力貴族がこの職に就くことを希望することが多かったという。(2)大貴族や大ブルジョワの館において、食材の手配やワインの管理、料理人の選抜などの一切を取り仕切り、とりわけ宴席においてはメニュー作りが重要な仕事のひとつとして課される職。歴史上もっとも有名なメートルドテルのひとり、ヴァテルはこの職に相当する。コンデ公に仕え、シャンティイ城でルイ14世らを招いて数日にわたって開催された千人規模大宴席の一切を取り仕切り、最後に手配した魚が届かないと誤解して自害した。彼が息を引きとってからすぐ後に魚は大量に届けられた、という(\protect\hyperlink{sauce-chantilly}{ソース・シャンティイ}訳注も参照)。(3)近代から20世紀中葉にかけて、とりわけ料理人がオーナーではないレストランの場合はメニューの決定、ソムリエおよび給仕人の指揮、客の応対などを担当し、その店で最高のサービス技術を誇る者のつく職名とされた。なお、現代ではほとんど「給仕長」程度の意味しか持たなくなってしまった職名といえる。上記を総合すると、この
  beurre à la maître d'hôtel
  という名称は「当家(当店)特製のバター」あるいは「当家(当店)自慢のバター」程度の意味ということになる。実際のところ、この名称の由来などは不明だが、たとえばフランソワ・マランFrançois
  Marin(生没年不詳)の著書、『コモス神の贈り物』のタイトルページに記された著者の肩書は「スビーズ元帥のメートルドテル、フランソワ・マラン」となっているように、本来はもっとも料理に精通した者の就く役職であった。このため、maître
  d'hôtel-cuisinier
  という語も18、19世紀には用いられていた。つまり、直接的に包丁を握り鍋を振ることはなくても、献立を組み、料理のレシピを考えるのもまたメートルドテルの重要な仕事であった。それを踏まえてカレームは1822年に、それ以前の主要な宴席の献立を詳細に分析した『フランスのメートルドテル』を出版した。つまり、カレームもまた、食卓外交の裏側でメートルドテル=キュイジニエの役割を果たしていたということになる。カレームをたんなるパティシエや料理人という現代的な狭い職の枠にはめて捉えることの出来ない時代だったとも言えよう。それは、エスコフィエについても言えることであり、初版および第二版の末尾には献立例が掲載され、第三版以降は『メニューの本』として独立させたが、総料理長であるということは即ちかつて貴族の館に仕えたメートルドテルの仕事を勤めるに他ならない、ということを示唆しているし、その点は現代の一流ホテルにおいてもあまり変化していないと思われる。この合わせバターの名称も含めた原型のひとつとして、上述のマラン『コモス神の贈り物』第2巻には、「いんげん豆のメートルドテル風」というレシピがある。これは水から茹でたいんげん豆を湯をきってから鍋に入れ、バター、パセリ、エシャロットの細かいみじん切り、塩、こしょうで味付けし、最後にレモン果汁かヴィネガー少々で仕上げるというもの(p.380)。カレームの未完の名著『19世紀フランス料理』第3巻では、「鯖用のメートルドテルバター」として、イジニー産バター8オンス(約250
  g弱)と大きめのレモン1個分の搾り汁、細かく刻んだパセリ大さじ2杯、塩2つまみ強、細かく挽いたこしょう1つまみ弱を木杓子を使ってよく混ぜ合わせる。食欲がわくような調味を心掛けるべし、とある(pp.128-129)。また、同じくヴィアールの『王国料理の本』(内容は1806年初版の『帝国料理の本』の改訂版であり、毎年のように改版され続けているために歴史的に貴重な史料)1846年版では、冷製メートルドテルとして、鍋に
  \(\frac{1}{4}\)ポンドのバターとパセリ少々、エシャロットのみじん切り少々、塩、粗挽きこしょう、レモン果汁を入れ、木杓子でよく練る。これを肉料理あるいは魚料理の下にでも、中にでも、上にでも流すといい、とある(p.48)。このように、
  19世紀前半にはメートルドテルバターの性格がほぼ定着していたと言えよう。}}{メートルドテルバター}}\label{beurre-maitre-d-hotel}}

\frsub{Beurre à la Maître-d'hôtel}

\index{あわせはたー@合わせバター!めーとるとてるはたー@メートルドテルバター}
\index{めーとるとてる@メートルドテル!はたー@---バター}
\index{beurre@beurre!maitre hotel@--- à la Maître d'hôtel}
\index{maitre hotel@maître d'hôtel!beurre@Beurre à la ---}

バター250
gをポマード状に柔らかくする。パセリのみじん切り大さじ1杯強と塩8
g、こしょう1 g、レモン
\(\frac{1}{4}\)個分の果汁を加えてよく混ぜ合わせる。

\hypertarget{nota-beurre-maitre-d-hotel}{%
\subparagraph{【原注】}\label{nota-beurre-maitre-d-hotel}}

このメートルドテルバターに大さじ1杯のマスタードを加えるのもバリエーションとしてお勧め。とりわけ牛、羊肉や魚のグリル焼きによく合う。

\hypertarget{beurre-manie}{%
\subsubsection{ブールマニエ}\label{beurre-manie}}

\frsub{Beurre Manié}

\index{あわせはたー@合わせバター!ふーるまにえ@ブールマニエ}
\index{ふーるまにえ@ブールマニエ} \index{beurre@beurre!manie@--- Manié}
\index{beurre manie@beurre manié}

これはマトロットの煮汁などに、手早くとろみ付けをするのに用いる。小麦粉75
gにバター100 gの割合が原則\footnote{このバターと小麦粉の割合は絶対というわけではなく、本書でもしばしば異なる割合で作ったブールマニエを用いる指示が見られる。}。

ブールマニエでとろみを付けたソースは、その後は出来るだけ沸騰させないこと。さもないと、生の小麦粉の不快な味が強まる危険性があるからだ。

\hypertarget{beurre-marchand-de-vin}{%
\subsubsection[ブール・マルシャンドヴァン]{\texorpdfstring{ブール・マルシャンドヴァン\footnote{「ワイン商人風」の意。煮詰めた赤ワインをバターを混ぜ込むところからの名称だろう。}}{ブール・マルシャンドヴァン}}\label{beurre-marchand-de-vin}}

\frsub{Beurre Marchand de vin}

\index{あわせはたー@合わせバター!ふーるまるしやんとうあん@ブール・マルシャンドヴァン}
\index{わいんしようにん@ワイン商人 ⇒ マルシャンドヴァン!ふーる@ブール・マルシャンドヴァン}
\index{まるしやんとうあん@マルシャンドヴァン!ふーる@ブール・マルシャンドヴァン}
\index{beurre@beurre!marchand vin@--- Marchand de vin}
\index{marchand vin@marcand de vin!beurre@Beurre ---}

赤ワイン2 dLに細かいみじん切りにしたエシャロット25
gを加えて半量になるまで煮詰める。塩1つまみ、挽きたて\footnote{原文
  poivre de moulin
  (ポワーヴルドムラン)、直訳すると「ミルで挽いたこしょう」だが、その場合は即座に使用するのが一般的なので、あえて「挽きたてのこしょう」と訳している。}(または粗く砕いた\footnote{原文
  mignonette
  (ミニョネット)。ミルを用いずに、包丁の側面などで圧し砕いたこしょうを指す。})こしょう1つまみ、溶かした\protect\hyperlink{glace-de-viande}{グラスドヴィアンド}大さじ1杯、ポマード状に柔らかくしたバター150
g、レモン\(\frac{1}{4}\)個分の果汁とパセリのみじん切り大さじ1杯を加える。全体をよく混ぜ合わせる。

\ldots{}\ldots{}グリル焼きにした牛リブロース\footnote{原文 entrecôte
  grillé(アントルコット グリエ)。}用。

\hypertarget{beurre-a-la-meuniere}{%
\subsubsection{ムニエル用バター}\label{beurre-a-la-meuniere}}

\frsub{Beurre à la Meunière}

\index{あわせはたー@合わせバター!むにえるようはたー@ムニエル用バター}
\index{むにえる@ムニエル!はたー@---用バター}
\index{beurre@beurre!meuniere@--- à la Meunière}
\index{meuniere@meunière (à la)!beurre@Beurre à la ---}

焦がしバターに、提供直前にレモン果汁数滴を加えたもの。

\ldots{}\ldots{}魚の「ムニエル\footnote{小麦粉をまぶして、バターで焼く手法および仕立て。原文にある
  à la meunière
  を直訳すると「粉挽き女風」の意。かつては主に水車を動力として石臼などを用い小麦を挽き、その後「ふるい」にかけていた。粉挽き職人は小麦粉の粉塵をかぶって真っ白になっていることが多かったところから付いた料理名。}」用。

\hypertarget{beurre-de-montpellier}{%
\subsubsection[モンペリエバター]{\texorpdfstring{モンペリエバター\footnote{南フランスの都市。モンプリエのようにも発音される。どちらも正しい。複数の発音が正しいとされる例として有名なもののひとつ。}}{モンペリエバター}}\label{beurre-de-montpellier}}

\frsub{Beurre de Montpellier}

\index{あわせはたー@合わせバター!もんへりえはたー@モンペリエバター}
\index{もんへりえ@モンペリエ!はたー@---バター}
\index{beurre@beurre!monpellier@--- de Montpellier}
\index{montepellier@Montpellier!beurre@Beurre de ---}

銅の鍋に湯を沸かし、クレソンの葉とパセリの葉、セルフイユ、シブレット、エスゴラゴンを同量ずつ計90〜100
gと、ほうれんそうの葉25
gを投入する。これとは別の鍋で同時に、エシャロットの細かいみじん切り40
gを下茹でする。ハーブは湯をきって冷水にさらす。しっかり圧し絞って余計な水気を取り除く。エシャロットも同様にする。これらを鉢に入れてすり潰す。

中くらいのサイズのコルニション3個と、水気を絞ったケイパー大さじ1杯、小さなにんにく1片、アンチョビのフィレ4枚を加える。全体が滑らかなペースト状になったら、バター750
gと固茹で卵の黄身3個、生の卵黄2個を加える。混ぜながら、最後に植物油2
dLを少しずつ加える。目の細かい漉し器か布で漉し、泡立て器で混ぜて滑らかにする。塩味を\ruby{調}{ととの}え、カイエンヌごく少量で風味を引き締める。

\ldots{}\ldots{}魚の冷製料理に添える。ビュッフェの場合には魚に覆いかけて供する。

\hypertarget{beurre-de-montpellier-pour-croutonnage-de-plats}{%
\subsubsection{装飾用モンペリエバター}\label{beurre-de-montpellier-pour-croutonnage-de-plats}}

\frsub{Beurre de Montpellier pour Croûtonnage de plats}

\index{あわせはたー@合わせバター!そうしよくようもんへりえはたー@装飾用モンペリエバター}
\index{もんへりえ@モンペリエ!そうしよくようはたー@装飾用---バター}
\index{beurre@beurre!beurre monpellier cretonnage@--- de Montpellier pour croûtonnage de plats}
\index{montepellier@Montpellier!beurre cretonnage@Beurre de --- pour Croûtonnage de plats}

モンペリエバターを装飾のためだけに作る場合には、植物油と茹で卵の黄身、生の卵黄は用いずに作る。平皿に流し入れて均等な厚みにしてやると細部の装飾作業が容易になる。

\hypertarget{beurre-de-moutarde}{%
\subsubsection{マスタードバター}\label{beurre-de-moutarde}}

\frsub{Beurre de Moutarde}

\index{あわせはたー@合わせバター!ますたーとはたー@マスタードバター}
\index{ますたーと@マスタード!はたー@---バター}
\index{beurre@beurre!moutarde@--- de Moutarde}
\index{moutarde@moutarde!beurre@Beurre de ---}

フランス産マスタード大さじ1
\(\frac{1}{2}\)杯をポマード状に柔らかくしたバター250
gに混ぜ込み、冷蔵する。

\hypertarget{beurre-noir-pour-les-grands-services}{%
\subsubsection[格式ある宴席用焦がしバター]{\texorpdfstring{格式ある宴席用焦がしバター\footnote{直訳すると、「大規模で格式の高い宴席において供する黒バター」。かつてのフランス式サービスによる宴席ではルルヴェと呼ばれる非常に壮麗な装飾を施した肉料理、魚料理がポタージュの後に供された。このバターはそういったケースを想定している。実際、カレームが焦がしバターについてこの「黒バター」beurre
  noir
  という表現を好んで用いていたことからも、『料理の手引き』の時代においてはやや大時代的な、過去の華やかな宴席のためのもの、というイメージだったと考えられる。}}{格式ある宴席用焦がしバター}}\label{beurre-noir-pour-les-grands-services}}

\frsub{Beurre noir pour les grands services}

\index{あわせはたー@合わせバター!かくしきあるえんせきようこかしはたー@格式ある宴席用焦がしバター}
\index{くろ@黒!はたー@---バター} \index{はたー@バター!こかし@焦がし---}
\index{beurre@beurre!noir grands services@--- noir pour les grands services}
\index{noir@noir(e)!beurre@Beurre --- pour les grands services}

(仕上がり10人分\footnote{原文 proportion pour un service
  (プロポルスィオンプーランセルヴィス)、直訳すると「1サーヴィスの分量」。17、18世紀から20世紀初頭にかけて宴席での人数の単位に
  service
  (セルヴィス)という語があてられた。原則として8〜12人分とされたが、ごく大雑把に10人前と捉えていい。舞踏会も含め、大規模で華やかな宴席が頻繁に行なわれていた時代においては、ある程度大まかに料理の単位を決めておくことで、食材の手配から仕込み、調理などを効率化していた。このため『料理の手引き』のレシピのほとんどは仕上がり量が1
  serviceになるよう記されている。})

バター125
gをフライパンに入れて火にかけて溶かし、茶色くなるまで加熱する。布で漉して湯煎にかける。\ruby{微温}{ぬる}くなったら、粗く砕いたこしょう\footnote{mignonette
  (ミニョネット)。こしょうの粒を肉叩きや包丁の側面などで押し潰して砕いたもの。}を加えて煮詰めたヴィネガー小さじ1杯を加える。提供直前に、丁度いい温度になるまで温めなおす。揚げたパセリの葉とケイパー大さじ1杯を料理にのせてから、この焦がしバターをかけてやる。

\hypertarget{beurre-de-noisette}{%
\subsubsection[ブール・ド・ノワゼット]{\texorpdfstring{ブール・ド・ノワゼット\footnote{焦がしバターのことを一般にbeurre
  noisette(ブールノワゼット)と呼ぶので、混同しないよう注意。}}{ブール・ド・ノワゼット}}\label{beurre-de-noisette}}

\frsub{Beurre de noisette}

\index{あわせはたー@合わせバター!ふーるとのわせつと@ブール・ド・ノワゼット}
\index{のわせつと@ノワゼット!ふーるとのわせつと@ブール・ド・---}
\index{beurre@beurre!noisette@--- de noisette}
\index{noisette@noisette!beurre@Beurre de ---}

⇒ \protect\hyperlink{beurre-d-aveline}{ブール・ダヴリーヌ}参照。

\hypertarget{beurre-de-paprika}{%
\subsubsection{パプリカバター}\label{beurre-de-paprika}}

\frsub{Beurre de Paprika}

\index{あわせはたー@合わせバター!はふりかはたー@パプリカバター}
\index{はふりか@パプリカ!はたー@---バター}
\index{beurre@beurre!Paprika@--- de Paprika}
\index{paprika@paprika!beurre@Beurre de ---}

玉ねぎのみじん切り大さじ1杯とパプリカ4
gをバターでいい色合いになるまで炒め、ポマード状に柔らかくしておいたバター250
gに混ぜる。布で漉す。

\hypertarget{beurre-de-pimentos}{%
\subsubsection{赤ピーマンバター}\label{beurre-de-pimentos}}

\frsub{Beurre de Pimentos}

\index{あわせはたー@合わせバター!あかひーまんはたー@赤ピーマンパプリカバター}
\index{あかひーまん@赤ピーマン!はたー@---バター}
\index{ほわうろん@ポワヴロン ⇒ 赤ピーマン}
\index{beurre@beurre!pimentos@--- de Pimentos}
\index{pimentos@pimentos!beurre@Beurre de ---}
\index{poivron@poivron ⇒ pimento!beurre pimentos!Beurre de Pimentos}

ブレゼ\footnote{野菜のブレゼの方法については\protect\hyperlink{}{第13章野菜料理}参照。}した赤いポワヴロン\footnote{原文
  poivron。日本の青果では「パプリカ」と呼ばれる肉厚で苦みの少ない品種。「カリフォルニア・ワンダー」が代表的品種。未熟なものは緑色だが完熟すると真っ赤になる。また、熟すと黄色、紫などになる品種もある。}100
gをバター250 gと合わせて細かくすり潰し、布で漉す。

\hypertarget{beurre-de-pistache}{%
\subsubsection{ピスタチオバター}\label{beurre-de-pistache}}

\frsub{Beurre de Pistache}

\index{あわせはたー@合わせバター!ひすたちおはたー@ピスタチオバター}
\index{ひすたちお@ピスタチオ!はたー@---バター}
\index{beurre@beurre!pistache@--- de Pistache}
\index{pistache@pistache!beurre@Beurre de ---}

殻から剥いて湯剥きしたばかりのピスタチオ150
gを、水数滴を加えながら細かくすり潰す。パター250 gを加え、布で漉す。

\hypertarget{beurre-a-la-polonaise}{%
\subsubsection{ポーランド風バター}\label{beurre-a-la-polonaise}}

\frsub{Beurre à la Polonaise}

\index{あわせはたー@合わせバター!ほーらんとふうはたー@ポーランド風バター}
\index{ポーランド@ポーランド!はたー@---風バター}
\index{beurre@beurre!polonaise@--- à la Polonaise}
\index{polonais@polonais(e)!beurre@Beurre à la ---e}

バター250 gをヘーゼルナッツ色\footnote{原文 cuire à la noisette
  (キュイーラノワゼット)すなわち「茶色く」なるまで火を通すということ。現代では、焦がしバターのことを
  beurre noisette
  (ブールノワゼット)と呼ぶことが多いが、本書においては\protect\hyperlink{beurre-de-noisette}{ヘーゼルナッツバター}という項目を立てているために、混同を避ける意味で、このような表現になっていると思われる。}になるまで火を通す。丁度いい色合いになったら、上等なパンの身60
gを投入する。

\hypertarget{beurre-de-raifort}{%
\subsubsection[レフォールバター]{\texorpdfstring{レフォールバター\footnote{ホースラディッシュ、西洋わさび。}}{レフォールバター}}\label{beurre-de-raifort}}

\frsub{Beurre de Raifort}

\index{あわせはたー@合わせバター!れふおーる@レフォールバター}
\index{れふおーる@レフォール!はたー@---バター}
\index{beurre@beurre!raifort@--- de Raifort}
\index{raifort@raifort!beurre@Beurre de ---}

器具を用いておろしたレフォール50 gを鉢に入れてすり潰す。バター250
gを加え、布で漉す。

\hypertarget{beurre-ravigote}{%
\subsubsection{ブール・ラヴィゴット/ブール・ヴェール}\label{beurre-ravigote}}

\frsub{Beurre Ravigote ou Beurre vert}

\index{あわせはたー@合わせバター!らういこつと@ブール・ラヴィゴット}
\index{あわせはたー@合わせバター!うえーる@ブール・ヴェール}
\index{みどり@緑 ⇒ ヴェール/ヴェルト!ブール・ヴェール}
\index{らういこつと@ラヴィゴット!ふーる@ブール・---}
\index{beurre@beurre!ravigote@--- Ravigote}
\index{beurre@beurre!vert@--- vert}
\index{ravigote@ravigote!beurre@Beurre ---}
\index{vert@vert(e)!beurre@Beurre ---}

⇒ \protect\hyperlink{beurre-chivry}{ブール・シヴリ}参照。

\hypertarget{beurre-de-saumon-fume}{%
\subsubsection{スモークサーモンバター}\label{beurre-de-saumon-fume}}

\frsub{Beurre de Saumon fumé}

\index{あわせはたー@合わせバター!すもーくさーもん@スモークサーモンバター}
\index{すもーくさーもん@スモークサーモン!はたー@---バター}
\index{beurre@beurre!saumon fumet@--- de Saumon fumé}
\index{saumon fume@saumon fumé!beurre@Beurre de ---}

スモークサーモン100 gとバター250 gを鉢に入れてよくすり潰し、布で漉す。

\hypertarget{beurre-de-truffe}{%
\subsubsection{トリュフバター}\label{beurre-de-truffe}}

\frsub{Beurre de Truffe}

\index{あわせはたー@合わせバター!とりゆふ@トリュフバター}
\index{とりゆふ@トリュフ!はたー@---バター}
\index{beurre@beurre!truffe@--- de Truffe}
\index{truffe@truffe!beurre@Beurre de ---}

真黒な黒トリュフ100
gと\protect\hyperlink{sauce-bechamel}{ベシャメルソース}小さじ1杯を鉢に入れてすり潰す。良質なバター200
gを加え、布で漉す。

\hypertarget{beurres-printaniers}{%
\subsubsection[ブール・プランタニエ各種]{\texorpdfstring{ブール・プランタニエ\footnote{printanier
  (プランタニエ)春の、を意味する語で、とりわけ春先の「はしり」の野菜を用いる場合によくこの表現があてられる。}各種}{ブール・プランタニエ各種}}\label{beurres-printaniers}}

\frsub{Beurres Printaniers}

\index{あわせはたー@合わせバター!ふーるとふらんたにえ@ブール・プランタニエ}
\index{ふらんたにえ@プランタニエ!ふーる@ブール・---}
\index{beurre@beurre!printaniers@--- Printaniers}
\index{printanier@printanier(ère)!beurre@Beurres Printaniers}

野菜の合わせバターはポタージュやソースの仕上げによく用いられる。

野菜はまず、それぞれの種類に応じた方法で火を通すこと。例えば、にんじんやナヴェ\footnote{原文
  navet 蕪のことだが、日本の蕪とは調理特性および風味が異なるので注意。}の場合はバターを加えて弱火で蒸し煮してからコンソメで煮る。緑の野菜、例えばプチポワ\footnote{原文
  petits pois
  いわゆるグリンピースのことだが、20世紀以降、だんだん若どりのものが好まれる傾向が強まっており、日本で一般的なグリンピースと比較するとかなり小さめの段階で収穫されるものが多く、火入れに必要な時間もごく短かい傾向にある。直径7〜8mm程度の若どりのものはグリンピース特有の青臭さが少なく、フレッシュであれば生食でも美味しい。フランスあるいはイタリア産の冷凍品が多く出回っているが加熱必須。}、さやいんげん\footnote{原文
  haricots verts (アリコヴェール)。}、アスパラガスの穂先などの場合は、しっかり下茹でして火を通す。

その後、野菜と同じ重さのバターとともに鉢に入れてすり潰し、布で漉す。

\hypertarget{coulis-divers}{%
\subsubsection[クリ各種]{\texorpdfstring{クリ\footnote{coulis
  (クリ)の基本概念としては、ピュレよりは水分の多いもの、と解していいのだが、ここではやや特殊な用法となっていることに注意。また、日本の調理現場では「クーリ」と呼ぶ傾向が根強く残っている。しかし、フランス語として見たとき、この語それ自体のアクセント(フランス語のアクセントは長音)は最後の
  i の音にある。ou (発音記号/ u /
  )は日本語にない音で、多くの日本人の耳には強く感じられるために、このような習慣が付いたのだと思われる。少なくとも日本語的な発音で「クーリ」と覚えても、フランス語としては通じない可能性が高いので注意。}各種}{クリ各種}}\label{coulis-divers}}

\frsub{Coulis divers}

\index{くーり@クーリ} \index{くり@クリ!}
\index{coulis divers@Coulis divers}

\begin{itemize}
\item
  エクルヴィス\footnote{ヨーロッパザリガニ。詳しくは\protect\hyperlink{sauce-bavaroise}{バイエルン風ソース}訳注参照。}の殻、または
\item
  クルヴェット\footnote{小海老。詳しくは\protect\hyperlink{sauce-aux-crevettes}{ソース・クルヴェット}訳注参照。}の胴や殻、または
\item
  オマールやラングスト\footnote{原文 langouste ≒ 伊勢エビ。}の胴にあるクリーム状の部分や卵、コライユ\footnote{朱色がかった内子のこと。}
\end{itemize}

を鉢に入れてすり潰す。

すり潰した材料100 gあたり大さじ4杯の新鮮な生クリームを加え、布で漉す。

これらのクリは提供直前に用いること。使い方はこの「合わせバター」の節冒頭に記しておいたので参照されたい。

\hypertarget{huile-de-crustaces}{%
\subsubsection{甲殻類のオイル}\label{huile-de-crustaces}}

\frsub{Huile de Crustacés}

\index{あふら@油 ⇒ オイル!こうかくるい@甲殻類の---}
\index{おいる@オイル!こうかくるい@甲殻類の---}
\index{こうかくるい@甲殻類!オイル@甲殻類のオイル}
\index{huile@huile!crustaces@--- de Crustacés}
\index{crustace@crustacé!huile@Huile de ---s}

このレシピは、オマールやラングストに添えるマヨネーズの仕上げに加えるため考案されたもので、言ってみればマヨネーズの新しいバリエーションだ。

使えるだけの甲殻類の殻などをすり潰し、バターではなくオイルを同量、つまり甲殻類と同じ重さだけ加える。つまり、オイルの重さが1
dLあたり95
gくらい、あるいは、すり潰した甲殻類の殻が大さじ6杯に対してオイル1
dLということになる。

甲殻類の殻をすり潰してペースト状になってきたら、すりこ木でよく混ぜながら油を少量ずつ加えていく。

まず目の細かい漉し器で漉し、さらに布で漉し、冷やしておく。このオイルは絶対に熱を与えてはいけない。
\end{recette}\newpage
\href{未、原文対照チェック}{} \href{未、日本語表現校正}{}
\href{未、その他修正}{} \href{未、原稿最終校正}{}

\hypertarget{marinades-et-saumures}{%
\section[マリナードとソミュール]{\texorpdfstring{マリナードとソミュール\footnote{マリナードはマリネ液とも言う。marinade
  \textless{} mariner
  (マリネ)語源はラテン語のmare(海)。古フランス語では「海で泳ぐ、海に潜る」の意で使われていたが、16世紀には既に、料理用語として用いられていたようだ。ラブレー『ガルガンチュアとパンタグリュエル』第四の書(1548年)において、lancerons
  marinez (マリネしたブロシェの幼魚)という表現が見られる。なおブロシェ
  brochet
  はノーザンパイク、和名キタカワカマス。川カマス属の淡水、汽水魚。この場面はパンタグリュエルに「小斉」のご馳走として捧げられた料理のリストの一部であり、「塩漬けのメルルーサ、卵料理各種、モリュ(塩漬けにした鱈)、アドック(塩漬け後に燻製にした鱈)」などとともに列挙されており、いずれも塩辛いから、食後の消化をよくするために飲むワインの量が倍になった(p.681)とある。したがって、lancerons
  marinezのマリネとは「海水あるいは塩水に漬けた」の意に解釈されよう。一方、ソミュールについては、11世紀末頃に、「保存のため漬け込む塩水」の意味で
  salmuire
  という語形が使用され、16世紀には「塩水およびその他の液体からなるもの」としてsaumureという現在とおなじ語形が記録されている。マリナードとソミュールが明確に分化したのはおそらく17世紀頃。1651年刊ラ・ヴァレーヌ『フランス料理の本』に見られるマリナードの語には曖昧さが残っているが、例えば
  \emph{Poulets
  marinez}(鶏のマリネ)というレシピは「鶏を開いて叩き、しっかり味付けしたヴィネガーに漬ける。小麦粉をまぶすか、卵と小麦粉で作った衣を付けてラードで揚げる。マリナードに戻し入れて軽く弱火で煮てから供する(p.36)」。また、\emph{Longe
  de
  mouton}(仔羊の腰肉のロースト)は、「棒状に切った豚背脂をラルデ針を使って刺し込み、串を刺してローストする。玉ねぎ、塩、こしょう、ごく少量のオレンジまたはレモンの外皮(ゼスト)とブイヨンとヴィネガーでマリナードを作る。肉に火が通ったら、ソース(マリナード)とともに弱火で煮込む。とろみ付けには小麦粉をラードで茶色くなるまで炒めたもの、すなわち後代のルーの原型といえるものを少々加える(p.80)」とあり、別の項目では「(串を刺した肉の下の受け皿にある)マリナードを小まめにかけながらローストする(p.106)」とある。全体的な印象としてはラ・ヴァレーヌのマリナードとは中世のドディーヌにヴィネガーを効かせたもののようにも受け取れるが、最初に見たように、「漬け込む」ものとしてもヴィネガーを用いている点に注目すべきだろう。これは18、19世紀に引継がれ、1756年マラン『コモス神の贈り物』第1巻において、\emph{Cervelle
  de veau en
  marinade}(仔牛の脳のマリナード仕立て)では、血抜きした仔牛の脳を豚背脂のシートで包みブイヨン少々で茹で、「冷ましてからヴィネガーもしくはレモン果汁に漬け込む。その後、水気をきって溶き卵に浸し、パン粉をつけて揚げる。小麦粉を溶いた揚げ衣に浸して揚げてもいい(p.206)」とある。19世紀初頭のヴィアールも同様で、『帝国料理の本』初版(1806年)において、\emph{Pieds
  d'agneau en marinade} (仔羊の足のマリナード仕立て)などいくつかの
  marinadeを冠するレシピが掲載されている。肝心のマリナードについての記述は欠落しているが、この版においてはよく見られる現象。なお、仔羊の足のマリナード仕立ては、マリナードがない場合は「塩、こしょう、ビネガーに、茹でた仔羊の足を漬けてから、揚げ衣を付けて揚げる(p.214)」となっている。1814年ボヴィリエ『調理技法』では「加熱マリナード」のレシピが掲載されている。これは、卵くらいの大きさのバターを鍋に入れ、輪切りにしたにんじん1、2本、同様にした玉ねぎ、ローリエの葉1枚、にんにく1片、タイム、バジル、枝ごとのパセリ、シブール(≒葱)2〜3本を加えて強火で炒める。野菜が色付きはじめたら、約250
  mLの白ワインヴィネガーと約0.5
  Lの水を注ぎ、塩、こしょうする。そのまま沸かしてから漉し、必要に応じて使う(pp.60-61)、というもの。もっとも、仔牛の脳のマリナード仕立てなどマランのレシピと大差ない揚げものも同書に掲載されている。また、1834年版のオドにおいても鶏のマリナードはラ・ヴァレーヌのものと同工異曲のものに留まっている。1837年版でロースト用マリナードの項が追加され、豚背脂とにんにく1片を細かく刻み、パセリ1つまり、塩、こしょう、ヴィネガー大さじ1杯、油大さじ4杯を合わせてよく混ぜる
  (p.419)。1853年版ではマリネしたうなぎのグリル焼き、というレシピが掲載される。これは、皮を剥いてぶつ切りにし、バターでソテーしたうなぎを深皿に並べ、塩、こしょうハーブ、マッシュルーム、細かく刻んだエシャロットとシブールを被せ、油大さじ1杯をかける。2〜3時間マリネしたら、パン粉をまぶしてグリル焼きする(p.310)というもの。また、
  mariner(マリネ)という動詞は、オドの1834年版でもに、ノロ鹿の腿肉のローストにおいて、「オリーブオイルと塩で5〜6時間マリネする」
  (p.155)という記述が見られる。1867年刊グフェ『料理の本』では、ヴィネガーをベースとしたソースとしてのマリナード(p.404)と仕立てとしてのマリナードがある。後者の例としては
  \emph{Tête de veau en marinade}
  (仔牛の頭 マリナード仕立て)が好例だろう。仔牛の頭肉半分を3
  cm角に切り、下茹でしてから水にさらし、牛脂と小麦粉、香草類を加えた湯で茹でる。これを、塩、こしょう、油、ヴィネガーに1時間漬け込む。水気をきって揚げ衣を付けて油で揚げる(p.156)。ここでは肉を漬け込む液体としてmarinadeの語が用いられている。このように、marinadeという名詞とmariner「漬け込む」という動詞の用法に若干の不統一が見られるため、『料理の手引き』におけるマリナードすなわちマリネ液、という概念は比較的新しいものと思われる。}}{マリナードとソミュール}}\label{marinades-et-saumures}}

\frsec{Marinades et Saumures}

\index{marinade saumures@marinade et saumures} \index{marinade@marinade}
\index{まりなーととそみゆーる@マリナードとソミュール}
\index{まりなーと@マリナード}

\vspace{1\zw}

マリナードとソミュールにはいろいろな種類があるが、最終的な目的は同じで、

\begin{enumerate}
\def\labelenumi{\arabic{enumi}.}
\item
  素材に料理で使う香辛料やハーブの香りを浸み込ませる
\item
  ある種の肉を柔らかくさせる
\item
  場合によっては保存のために用いる。とりわけ温度と湿度で素材が駄目になってしまうような場合。さらに、目指す料理の仕上がりに合わせて素材の状態を調節する
\end{enumerate}
\begin{recette}
\hypertarget{marinade-instantanee}{%
\subsubsection{即席マリナード}\label{marinade-instantanee}}

\frsub{Marinade instantanée}

\index{marinade@marinade!marinade instantanee@marinade instantanée}
\index{まりなーと@マリナード!そくせき@即席---}

このマリナードはすぐに素材を使う場合、例えば赤身肉のグリル焼きや、ガランティーヌ、テリーヌ、パテのような冷製料理の補助材料\footnote{具体的には\protect\hyperlink{farces}{ファルス}のこと。}にする肉に用いる。

\begin{enumerate}
\def\labelenumi{\arabic{enumi}.}
\item
  グリル焼きにする肉の場合\ldots{}\ldots{}ごく薄くスライスしたエシャロットとパセリの枝、タイムの枝、ローリエの葉を肉の上に散らす。量は適宜加減すること。レモン果汁
  \(\frac{1}{2}\)個分に対して油大さじ1杯の割合で、上からかけてやる。
\item
  仔牛、ジビエのフィレ肉、ハム、豚背脂などを細かく切ったもの\footnote{原文
    lardon
    (ラルドン)、通常は拍子木状に切ったものを言うが、ここではファルスとして後で細かく挽くことになるので、形状はあまり問題にならない。}の場合\ldots{}\ldots{}塩こしょうしてから、白ワイン3、コニャック3、油1の割合のマリナードを上からかけてやる。
\end{enumerate}

ここで用いた風味付けの材料は、後でファルスにする際に加えることになる。

いずれの場合でも、マリナードに浸した肉を小まめに裏返してやり、マリナードがよく浸み込むようにしてやること。

\atoaki{}

\hypertarget{marinade-crue-pour-viandes-de-boucherie-ou-venaison}{%
\subsubsection{牛、羊肉および大型ジビエ用の非加熱マリナード}\label{marinade-crue-pour-viandes-de-boucherie-ou-venaison}}

\frsub{Marinade crue pour viandes de boucherie ou venaison}

\index{marinade@marinade!marinade crue viande boucherie venaison@marinade crue pour viande de boucherie ou venaison}
\index{まりなーと@マリナード!うしひつしおおかたしひえようひかねつ@牛、羊肉および大型ジビエ用非加熱---}

(仕上がり2 L分)

\begin{itemize}
\item
  香味素材\ldots{}\ldots{}にんじん100 g、玉ねぎ100 g、エシャロット40
  g、セロリ30 g、にんにく2片、パセリの枝3本、タイム1枝、ローリエの葉
  \(\frac{1}{2}\)枚、大粒のこしょう6個、クローブ2本。
\item
  使用する液体\ldots{}\ldots{}白ワイン1 \(\frac{1}{4}\) L、ヴィネガー5
  dL、油2 \(\frac{1}{2}\) dL。
\item
  作業手順\ldots{}\ldots{}マリネする素材に塩とこしょうを振る。にんじん、玉ねぎ、エシャロットを薄切り\footnote{émincer
    (エマンセ)薄切りにする、スライスする。}にし、半量を容器の底に敷く。容器の大きさは素材とマリナードがぴったり入る程度のものを用いること。素材を入れて、残りの香味野菜で蓋をするようにして、白ワインとヴィネガー、油を注ぎ入れる。
\end{itemize}

冷蔵し、マリネ液に漬かった素材を小まめに裏返してやること。

\atoaki{}

\hypertarget{marinade-cuite-pour-viandes-de-boucherie-ou-venaison}{%
\subsubsection{牛、羊肉および大型ジビエ用の加熱マリナード}\label{marinade-cuite-pour-viandes-de-boucherie-ou-venaison}}

\frsub{Marinade cuite pour viandes de boucherie ou venaison}

\index{marinade@marinade!marinade cuite viande boucherie venaison@marinade cuite pour viande de boucherie ou venaison}
\index{まりなーと@マリナード!うしひつしおおかたしひえようかねつ@牛、羊肉および大型ジビエ用加熱---}

(仕上がり2 L分)

\begin{itemize}
\item
  香味素材\ldots{}\ldots{}非加熱マリナードと同じ材料で同じ分量
\item
  使用する液体\ldots{}\ldots{}白ワイン1 \(\frac{1}{2}\) L、ヴィネガー3
  dL、油2 \(\frac{1}{2}\) dL。
\item
  作業手順\ldots{}\ldots{}鍋に油を熱し、ごく薄くスライスしたにんじん、玉ねぎ、エシャロットおよびその他の香味素材を軽く色付くまで炒める。

  白ワインとヴィネガーを注ぎ、弱火で約30分間火を通す。

  必ず、マリナードが完全に冷めてからマリネする素材にかけること。
\end{itemize}

\atoaki{}

\hypertarget{marinade-crue-ou-cuite-pour-grosse-venaison}{%
\subsubsection[とりわけ大型のジビエ用、非加熱および加熱マリナード]{\texorpdfstring{とりわけ大型のジビエ\footnote{具体的には赤鹿
  cerf(セール)
  や猪、トナカイの成獣など。ニホンジカやエゾジカはcerfに分類されるので、これを参考にするといいだろう。}用、非加熱および加熱マリナード}{とりわけ大型のジビエ用、非加熱および加熱マリナード}}\label{marinade-crue-ou-cuite-pour-grosse-venaison}}

\frsub{Marinade crue ou cuite pour grosse venaison}

\index{marinade@marinade!marinade crue cuite grosse venaison@marinade crue ou cuite pour grosse venaison}
\index{まりなーと@マリナード!とりわけおおかたのしひえようひかねつおよひかねつ@とりわけ大型のジビエ用非加熱および加熱---}

(仕上がり2 L分)

\begin{itemize}
\item
  香味素材\ldots{}\ldots{}牛、羊肉および大型ジビエ用のマリナードと同じだが、ローズマリー12
  gを追加する。
\item
  使用する液体\ldots{}\ldots{}ヴィネガー16 dL、油4 dL。
\item
  作業手順\ldots{}\ldots{}非加熱、加熱ともに作業手順は上記のレシピのとおり。
\end{itemize}

\atoaki{}

\hypertarget{marinade-cuite-pour-le-mouton-en-chevreuil}{%
\subsubsection[羊のシュヴルイユ仕立て用の加熱マリナード]{\texorpdfstring{羊のシュヴルイユ仕立て用の加熱マリナード\footnote{\protect\hyperlink{sauce-chevreuil}{ソース・シュヴルイユ}参照。}}{羊のシュヴルイユ仕立て用の加熱マリナード}}\label{marinade-cuite-pour-le-mouton-en-chevreuil}}

\frsub{Marinade cuite pour le mouton en chevreuil}

\index{marinade@marinade!marinade cuite mouton en chevreuil@marinade cuite pour le mouton en chevreuil}
\index{まりなーと@マリナード!ひつしのしゆうるいゆしたてようのかねつまりなーと@羊のシュヴルイユ仕立て用加熱---}

(仕上がり2 L分)

\begin{itemize}
\item
  香味素材\ldots{}\ldots{}上記のとおりの分量の素材に、ジュニパーベリー\footnote{セイヨウネズの実。ジンの香り付けに用いられている。}10粒とバジル1つまみ、ローズマリー1つまみを足す。
\item
  使用する液体\ldots{}\ldots{}牛、羊および大型ジビエ用の加熱マリナードと同じ。
\item
  作業手順\ldots{}\ldots{}鍋に油を熱し、薄切りにしたにんじん、玉ねぎ、エシャロットおよびその他の香味素材を軽く色付くまで炒める。

  白ワインとヴィネガーを注ぎ、弱火で約30分間火を通す。
\end{itemize}

\atoaki{}

\hypertarget{marinade-cuite-pour-le-mouton-en-chamois}{%
\subsubsection[羊のシャモワ仕立て用の加熱マリナード]{\texorpdfstring{羊のシャモワ仕立て\footnote{オートザルプ県の山岳地帯およびピレネー山脈に生息する野生の山羊。ピレネー山脈のものは
  Isard
  (イザール)と呼ばれる。若い獣の肉は大型ジビエのなかでもとりわけ美味とされる。成獣の肉は固く、しっかりマリネする必要があると言われている。しばしばノロ鹿と比較される。ここでは、羊肉を白ワインベースのマリナードに漬け込む仕立て、すなわちシュヴルイユ仕立てとの対比として、赤ワインでより強い風味のマリナードに漬け込むことで、シャモワ仕立てとしている。なお、本書にシャモワ仕立てのレシピは掲載されていない。シュヴルイユ仕立てと同様と考えていい。}用の加熱マリナード}{羊のシャモワ仕立て用の加熱マリナード}}\label{marinade-cuite-pour-le-mouton-en-chamois}}

\frsub{Marinade cuite pour le mouton en chamois}

\index{marinade@marinade!marinade cuite mouton en chevreuil@marinade cuite pour le mouton en chevreuil}
\index{まりなーと@マリナード!ひつしのしやもわしたてようのかねつまりなーと@羊のシャモワ仕立て用加熱---}

(仕上がり2 L分)

\begin{itemize}
\item
  香味素材\ldots{}\ldots{}非加熱マリナードと同じ分量の素材に、ジュニパーベリー\footnote{セイヨウネズの実。ジンの香り付けに用いられている。}15粒とバジル15
  g、ローズマリー15 gを足す。
\item
  使用する液体\ldots{}\ldots{}良質な赤ワイン1 \(\frac{1}{2}\)
  L、ヴィネガー3 dL、油2 \(\frac{1}{2}\) dL。
\item
  作業手順\ldots{}\ldots{}上記と同じ。

  このマリナードに上等な赤ワインを使える場合には、素材の量を次のように調整すること。赤ワイン12
  dL、ワインヴィネガー6 dL、油は上記の分量とする。

  ワインの酸味の強さによっては、ヴィネガーの量をワインと同量にすることさえ可能。
\end{itemize}

\hypertarget{observation-sur-les-marinades}{%
\subparagraph{マリナードについての注意事項}\label{observation-sur-les-marinades}}

\ldots{}\ldots{} 1.
加熱マリナードを使用するのは、素材へのマリナードの浸透作用を促進するのが目的。素材をマリナードに漬け込む時間は、加熱、非加熱ともに、素材の種類と大きさ、気温、環境の変化を勘案して決めること。

\begin{enumerate}
\def\labelenumi{\arabic{enumi}.}
\setcounter{enumi}{1}
\tightlist
\item
  一般的な牛、羊肉と肉質の柔らかい大型ジビエに使うマリナードに純粋な酢酸を用いるのは絶対にやめておくこと。酢酸の腐食作用によって肉の風味が失なわれてしまうからだ\footnote{この注記は第二版から。内容が当時の知見にもとづいたものであることに注意。ただし、19世紀には木酢液を原料として工業用の氷酢酸が既に製造されていた。また、タンパク質はpHの変化によって分解されるので、マリナードにヴィネガーを加えるのは理にかなっている。なお、肉を柔らかくする効果のあるタンパク質分解酵素(プロテアーゼ)の代表的なひとつであるパパインの発見は1940年代になってからのこと。パイナップルに含まれているブロメラインの効果は経験的に知られていた可能性もあるが、この酵素が60℃で不活性化することが広く知られるようになったのは、少なくとも日本では比較的近年のことに過ぎない。}。猪、赤鹿\footnote{cerf
    (セール)、ニホンジカやエゾジカもフランス語で表現するとこれに含まれるので、これらの料理について
    chevreuil (しゅう゛るいゆ)ノロ鹿の名をつけるは、厳密には誤り。}、トナカイなどの固い肉についても、純粋な酢酸だけを使うのは不可。
\end{enumerate}

\atoaki{}

\hypertarget{conservation-des-marinades}{%
\subsubsection{マリナードの保存方法}\label{conservation-des-marinades}}

\frsub{Conservation des marinades}

\index{marinade@marinade!conservation marinades@conservation des marinades}
\index{まりなーと@マリナード!ほそんほうほう@---の保存方法}

マリナードを長期間保存しておく必要がある場合には、とりわけ夏場は、本書で示した分量に対して2〜3
gのホウ酸を加えるといい。

さらに、夏のあいだは2日に一度、冬季は4〜5日に一度、マリナードを沸騰させ、冷めたら毎回そのマリナードに使っているのと同じワインを
2 dLとヴィネガー1 dLを足してやること。
\end{recette}
\hypertarget{saumures}{%
\subsection[ソミュール]{\texorpdfstring{ソミュール\footnote{この見出しは第四版のみ。初版〜第三版にかけては、マリナードとソミュールのレシピの間に区切りをつけるものは何も挿入されていない。}}{ソミュール}}\label{saumures}}

\frsecb{Saumures}

\index{saumure@saumure} \index{そみゆーる@ソミュール}
\begin{recette}
\hypertarget{saumure-au-sel}{%
\subsubsection{塩漬け用ソミュール}\label{saumure-au-sel}}

\frsub{Saumure au sel}

\index{saumure@saumure!sel@--- au sel}
\index{そみゆーる@ソミュール!しおつけよう@塩漬け用---}

このソミュールは、グレーソルト\footnote{フランス語は sel gris
  (セルグリ)または gros gris (グログリ)。灰色がかった粗塩。}1
kgに対して硝石\footnote{原文 salpêtre
  (サルペートル)硝酸カリウム。殺菌作用と、肉類を赤く発色させる効果を持つ。現代の日本では亜硝酸カリウム、亜硝酸ナトリウムが使われることが多い。いずれも日本では劇物指定されているが、シャルキュトリ(豚肉加工品の製造)においては不可欠とも言われるな薬品であり、とりわけボツリヌス菌対策の効果が大きい。そのため劇物ではあるが、食品添加物として認められており、使用限界量が厳密に定められている(食品添加物は国あるいは地域によって扱いが異なるので注意)。硝酸塩あるいは亜硝酸塩による肉の赤い発色を「着色料によるもの」と誤認する消費者は少なくない。これはかつて「魚肉ソーセージ」がコチニール色素でピンク色に染められていたことから連想される誤認と思われる。また、食品添加物イコール毒という安直な考えから忌避する消費者も少なくないのは事実だろう。こうしたことから、現代日本のレストランでは、製造後すぐに提供可能であるために、これら硝酸塩、亜硝酸塩の類を用いないところもある。}40
gの割合で作る。この硝石入りの塩の総量は、塩漬けにする肉の数と大きさで決まる。素材が完全に覆えて、重しが出来る分量とすること。

\begin{itemize}
\tightlist
\item
  作業手順\ldots{}\ldots{}肉を塩漬けにする前にまず、太い針を充分深く刺して穴を何箇所も空ける。次に硝石の粉末を肉の表面にすり付ける。塩1
  kgあたりタイム1枝、ローリエの葉
  \(\frac{1}{2}\)枚を加えて肉と塩を容器に詰める。
\end{itemize}

\atoaki{}

\hypertarget{saumure-liquide-pour-langues}{%
\subsubsection[舌肉用の液体ソミュール]{\texorpdfstring{舌肉用の液体ソミュール\footnote{このソミュールに舌肉を漬け込むと、硝石の作用で舌肉が赤く発色する。それを拍子木状などに切って鶏やフィレ肉の表面に、同様に切ったトリュフや豚背脂などとともに刺して装飾することが19世紀〜20世紀初頭までよく行なわれた。現代ではほとんど行なわれなくなった装飾方法。この場合はあくまでも料理の装飾を目的としたものであり、牛や豚の舌肉を保存食として利用する場合には塩漬けや燻製などの方法も用いられる。}}{舌肉用の液体ソミュール}}\label{saumure-liquide-pour-langues}}

\frsub{Saumure liquide pour langues}

\index{saumure@saumure!liquide langues@--- liquide pour langues}
\index{そみゆーる@ソミュール!したにくようのえきたい@舌肉用の液体---}

\begin{itemize}
\item
  材料\ldots{}\ldots{}水5 L、グレーソルト2.25 kg、硝石150
  g、茶色いカソナード\footnote{砂糖きびを原料とした粗糖。通常は茶褐色のものが多く「赤糖」とも呼ばれるが、精白したものもある。精製が不完全であるため独特の風味があり、料理および製菓でしばしば用いられる。}
  300 g、こしょう12 g、ジュニパーベリー12粒、タイム1枝、ローリエの葉1
  枚。
\item
  作業手順\ldots{}\ldots{}充分な大きさの鍋に材料を全て入れ、強火で沸騰させる。その後、完全に冷めてから、針で穴を複数空けて硝石をしっかりすり込んだ舌肉を入れた容器に注ぎ込む。平均的な重さの舌肉を漬け込む期間は冬季で8日間、夏季は6日間。
\end{itemize}

\atoaki{}

\hypertarget{grande-saumure}{%
\subsubsection[グランドソミュール]{\texorpdfstring{グランドソミュール\footnote{この項は第二版で追加された。通常はシャルキュティエすなわちシャルキュトリ専門の職人が行なう規模のものであり、料理人の仕事の範疇をやや越えるとも考えられる。}}{グランドソミュール}}\label{grande-saumure}}

\frsub{Grande saumure}

\index{saumure@saumure!grande@grande ---}
\index{そみゆーる@ソミュール!くらんと@グランド---}

(仕上がり50 L分)

\begin{itemize}
\tightlist
\item
  水\ldots{}\ldots{}50 L
\item
  塩\ldots{}\ldots{}25 kg
\item
  硝石\ldots{}\ldots{}2.7 kg
\item
  カソナード\ldots{}\ldots{}1.6 kg
\item
  作業手順\ldots{}\ldots{}メッキされた銅の鍋に材料を全て入れ、強火にかける。沸騰したら、皮を剥いたじゃがいも1個を投入する。じゃがいもが浮いてくるようであれば、じゃがいもが沈みはじめる寸前まで水を足す。逆に、じゃがいもが完全に底まで沈んでしまうようなら、じゃがいもが水面に見えてくるまで煮詰める必要がある。
\end{itemize}

ソミュールがちょうどいい具合になったら、鍋を火から外して、このソミュールで漬け込み槽に注ぎ込む。漬け込み槽の素材は、スレート製、岩製、セメント製、あるいはレンガ製でしっかりエナメル引きしたものを用いること。

漬け込み槽の底に、木製の網を敷き、その上に漬け込む肉を置くといい。肉が槽の底面に直接当たっていると、肉の下側にソミュール液が浸透しない可能性がある。

漬け込む肉は、たとえ小さなものであっても、専用の携行可能な注入器具を使ってソミュールを内部に注入してから、漬け込み槽に入れてやること。この準備作業を怠ると、肉全体が均等に塩漬けにならない可能性がある。肉の中心部がちょうどいい塩加減になる頃には外側は塩が強すぎるということになってしまうのだ。牛のランプ、イチボなどの塊肉で、4〜5
kgの大きさの場合は、ソミュール液を注入してやる方法を使えば8日間で漬かる。

牛舌肉をこの方法で漬ける場合は、出来るだけ新鮮なものを用いる必要がある。軟骨部分をきれいに取り除いてやり、肉叩きか麺棒で丁寧に叩いてやる。ブリデ針\footnote{主として鶏などの手羽や腿をまとめて整形し、その形状を保つよう糸で縫う際に用いる縫い針。}を使って、表面全体に刺し穴をつけてやる。それからソミュールに漬け込むが、何らかの重しをして浮き上がらないようにしてやること。

\hypertarget{observation-grande-saumure}{%
\subparagraph{【原注】}\label{observation-grande-saumure}}

ソミュールはマリナードほどは腐敗しにくいとはいえ、天候が悪い時季などはとりわけ、よく様子を見て、時々は沸騰させてやるのがいい。沸騰させれば多少は濃縮されてしまうから、本文記載の方法でじゃがいもを用いて、毎回少量の水を加える必要がある。
\end{recette}\newpage
\href{未、原文対照チェック}{} \href{未、日本語表現校正}{}
\href{未、その他修正}{} \href{未、原稿最終校正}{}

\hypertarget{gelees-diverses}{%
\section{ジュレ}\label{gelees-diverses}}

\frsec{Gelées diverses}

\index{gelee@gelée} \index{しゆれ@ジュレ}

どんなジュレも、ベースとなっているのはほぼ全てフォンだ。だから、フォンのメインとなっている素材によってジュレの風味が決まるわけだ。その結果としてジュレの用途も\ruby{自}{おの}ずと決まってくる。

人工的な凝固剤を使わずにジュレを確実に固めるためには、フォンのメインとなる素材に、仔牛の足や豚皮のようなゼラチン質の量を計算して加えることになる。仔牛の足や豚の皮を使えば、ジュレを確実に凝固させられるし、しかも柔らかな口あたりに仕上げられる。

そうはいっても、とりわけ夏季には、クラリフィエ\footnote{clarifier
  \textgreater{} clarification 次項参照。}の作業を行なう前に必ず、フォンを氷の上に垂らしてみて、固さと濃度を確認し、必要があれば板ゼラチンを何枚か加えてやること。

追加する板ゼラチンの量は、どんな場合でも、フォン1 Lあたり9
g(6枚)を越えないこと。板ゼラチンは、透き通っていてぱりぱりと割れやすく、
\ruby{膠}{にかわ}っぽい味のしないものを選ぶこと。必ず冷水でもどしてから使うか、せめてよく洗ってから用いること。

標準的なジュレを作る際に人工着色料を使うことはお勧め出来ない。標準的なジュレは充分に色よく仕上がるものだ。さらに、最後にマデイラ酒を加えてやれば充分に、標準的なジュレの特徴ともいえる淡い琥珀色に仕上がる。
\begin{recette}
\hypertarget{fonds-pour-gelee-ordinaire}{%
\subsubsection[標準的なジュレ用のフォン]{\texorpdfstring{標準的なジュレ用のフォン\footnote{この項および次の「白いジュレ用のフォン」は初版と第二版以降の異同が大きい。この「標準的なジュレ用のフォン」は初版では使用する液体が
  8 litres et demi de remouillage
  いわゆる「二番のフォン」であり、加熱時間も6時間と短かい。第二版は「水8.5
  L」になるが、加熱時間は6時間のままで、作業手順が「ソース用の白いフォンと同じ」となっている。第三版で現在の記述となった。}}{標準的なジュレ用のフォン}}\label{fonds-pour-gelee-ordinaire}}

\frsub{Fonds pour gelée ordinaire}

\index{gelee@gelée!fonds ordinaire@Fonds pour gelée ordinaire}
\index{しゆれ@ジュレ!ひようひゆんてきなしゆれようのふおん@標準的なジュレ用のフォン}

(仕上がり5 L分)

\begin{itemize}
\item
  主素材\ldots{}\ldots{}仔牛のすね肉とバモルソー\footnote{原文 bas
    morceaux 煮込みなどに用いる部位の総称。bas
    は「低い」が原義であり、食材として低級な部位というニュアンス。}2
  kg、細かく砕いた仔牛の骨1.5 kg、牛の脚肉1.5
  kg\ldots{}\ldots{}これらの肉と骨はオーブンで軽く色付けておくこと。
\item
  ゼラチン質\ldots{}\ldots{}骨を取り除いて\footnote{原文 désosser
    (デゾセ)骨を取り除く。}下茹でした\footnote{原文 blanchir
    (ブランシール)。下茹ですることがだ、原義は「白くする」。もとは中世において肉を調理する際にはローストであれ煮込みであれ、ほぼ必ず下茹でしていた。赤い肉を茹でると表面が白くなることからこの用語が定着することになったが、現在ではもっぱら野菜の下茹でなどについて言うことがほとんど。「ブランシェ」と言う現場もあるようだが、もとのフランス語からやや離れているので「ブランシール」で覚えるといいだろう。}仔牛の足3本、背脂を付けたままの生の豚皮\footnote{塩漬けなどの加工をしていない、ということ。}250
  g。
\item
  香味素材\ldots{}\ldots{}にんじん200 g、玉ねぎ200 g、ポワロー\footnote{poireau(x)
    ポロねぎ。日本の長葱とは異なり、植物としてはむしろ、にんにくに近い。葱と風味がかなり異なるため、見た目が似ているからと下仁田葱で「代用」するのはあまり好ましいとはいえないだろう。ポワローは古代ローマ時代からヨーロッパで広く親しまれてきた野菜のひとつであり、ローマ皇帝ネロが演説で大きな声を出すために、ポワローの蜂蜜漬けを好んだという逸話もある。伝統的な栽培方法の場合、旬は秋〜冬。播種から収穫まで10ヶ月以上かかる品種も多い。太さ3〜5
    cm、軟白部が20〜 40
    cmくらいのものが多い。フランスの標準的な規格では軟白部20
    cm以上。かつては日本の長葱と同様に成長に応じて「土寄せ」して栽培していたが、その方法では内部に土砂が入りやすい。現代のフランスではサヴォイキャベツやプチポワ同様に、大規模、機械化栽培が進んでいる品目のひとつ。また、太さ1
    cm程のミニ・ポワローも付け合わせ用の高級野菜として人気がある。元来ミニ・ポワローは苗の「間引き」を利用したものだったが、現在ではミニ・ポワローむけの品種も開発されている。日本にも秋〜冬季はヨーロッパ産が、春〜夏季はオーストラリア産が安定的に輸入されている。日本国内での生産も明治以降、試みられてはいるが、需給バランスとコスト的な問題から断念せざるを得ないケースも少なくないようだ。なお、第二次大戦前は八丈島などでこうした西洋野菜の栽培が行なわれ、船便で東京まで運ばれていたという(cf.~大木健二『大木健二の洋菜ものがたり』日本デシマル、1997年)。なお、現代フランス語でブレット(ふだんそう)のことを
    poirée (ポワレ)とも呼ぶが、これは ポワロー poireau
    と同語源。中世の料理書にはしばしば、野菜をペースト状になるまで煮込んだポタージュとして
    porée
    (ポレ)というものが出てくるが、どちらを材料として用いているか判別できないケースもある。ブレットはビーツともとは同じもので、16世紀頃に品種分化されたといわれており、bette(ベット=ビーツ)のrave(ラーヴ=根)がbetterave(ベトラーヴ)つまり現代フランス語でビーツを意味する語となり、betteはbête(獣、愚かな)と同じ発音であることが嫌われてblette(ブレット)と日常的に呼ばれるようになった。}50
  g、セロリ 50 g、充分な香りと量のブーケガルニ。
\item
  使用する液体\ldots{}\ldots{}水 8.5 L。
\item
  加熱時間\ldots{}\ldots{}6時間。
\item
  作業手順\ldots{}\ldots{}ソース用の\protect\hyperlink{fonds-brun}{茶色いフォン}とまったく同じ。ただし、ジュレ用のフォンの色合いはソース用のフォンよりも薄くしておくこと。
\end{itemize}

\hypertarget{fonds-pour-gelee-blanche}{%
\subsubsection[白いジュレ用のフォン]{\texorpdfstring{白いジュレ用のフォン\footnote{初版全文は「主素材、ゼラチン質、香味素材は上記のとおり。注ぐ液体は水(原文
  mouillage à
  blanc)、作業手順は基本の白いフォンと同様」。第二版で現在の記述となっている。この文脈からすると、白いフォンの二番を使うとも解釈され得るが、前項の「標準的なジュレ用のフォン」が最終的に水を用いて作ることになっているのと比較すると、加熱時間および作業手順が何と同様なのか曖昧になってしまうため、ここでは液体、加熱時間、作業手順を\protect\hyperlink{fonds-blanc}{標準的な白いフォン}と同じと解釈した。なお、英訳第5版では、but
  use very white stock instead of
  water「水ではなく白いフォン」を注ぐとなっている。}}{白いジュレ用のフォン}}\label{fonds-pour-gelee-blanche}}

\frsub{Fonds pour gelée blanche}

\index{gelee@gelée!fonds ordinaire@Fonds pour gelée ordinaire}
\index{しゆれ@ジュレ!しろいしゆれようのふおん@白いジュレ用のフォン}

主素材、ゼラチン質、香味素材の種類と分量は前記の\protect\hyperlink{fonds-pour-gelee-ordinaire}{標準的なジュレ用のフォン}を参照。

使用する液体の量は\protect\hyperlink{fonds-blanc}{標準的な白いフォン}とまったく同じにすること。

加熱時間も作業手順も同様。

\hypertarget{fonds-pour-gelee-de-volaille}{%
\subsubsection{鶏のジュレ用のフォン}\label{fonds-pour-gelee-de-volaille}}

\frsub{Fonds pour gelée de volaille}

\index{gelee@gelée!fonds volaille@Fonds pour gelée de volaille}
\index{しゆれ@ジュレ!とりのしゆれようのふおん@鶏のジュレ用のフォン}
\index{しゆれ@ジュレ!うおらいゆのしゆれようのふおん@ヴォライユのジュレ用のフォン ⇒ 鶏のジュレ用のフォン}
\index{とり@鶏!しゆれようのふおん@---のジュレ用のフォン}
\index{うおらいゆ@ヴォライユ!しゆれようのふおん@ヴォライユのジュレ用のフォン ⇒ 鶏のジュレ用のフォン}

(仕上がり5 L分)

\begin{itemize}
\item
  主素材\ldots{}\ldots{}仔牛のすね肉1.5 kg、牛の脚肉1.5
  kg、細かく砕いた仔牛の骨1.5
  kg、鶏ガラ、とさか、手羽先、足など(とりわけ湯通しした手羽と足)、1.5
  kg。
\item
  ゼラチン質\ldots{}\ldots{}骨を取り除いて下茹でした仔牛の足(小)3本。
\item
  香味素材\ldots{}\ldots{}材料の種類は標準的なジュレ用のフォンと同じだが、量はやや少なめにすること。
\item
  使用する液体\ldots{}\ldots{}軽く仕上げた\protect\hyperlink{fonds-blanc}{白いフォン}
  8 L。
\item
  加熱時間\ldots{}\ldots{}4時間半。
\item
  作業手順\ldots{}\ldots{}ソース用の\protect\hyperlink{fonds-de-volaille}{鶏のフォン}とまったく同じ。
\end{itemize}

\hypertarget{fonds-pour-gelee-de-gibier}{%
\subsubsection{ジビエのジュレ用のフォン}\label{fonds-pour-gelee-de-gibier}}

\frsub{Fonds pour gelée de gibier}

\index{gelee@gelée!fonds gibier@Fonds pour gelée de gibier}
\index{gibierl@gibier!gelee@gelée!fonds@Fonds pour gelée de ---}
\index{しゆれ@ジュレ!しひえのしゆれようのふおん@ジビエのジュレ用のフォン}
\index{しひえ@ジビエ!しゆれ@ジュレ!ふおん@---のジュレ用のフォン}

(仕上がり5 L分)

\begin{itemize}
\item
  主素材\ldots{}\ldots{}仔牛のすね肉1 kg、牛の脚肉2 kg、仔牛の骨750
  g、ジビエのガラやバモルソー\footnote{\protect\hyperlink{fonds-pour-gelee-ordinaire}{標準的なジュレ用のフォン}訳注参照。}1.75
  kg。これらはすべてオーブンで焼いて色付けておくこと。
\item
  ゼラチン質\ldots{}\ldots{}\protect\hyperlink{fonds-pour-gelee-de-volaille}{鶏のジュレ用のフォン}と同じ。
\item
  香味素材\ldots{}\ldots{}材料の種類は標準的なジュレ用のフォンと同じだが、セロリとタイムを
  \(\frac{1}{3}\)量多くすること。ジュニパーベリー\footnote{セイヨウネズの実。ジンの香りを特徴付けているもの。}7〜8粒を追加すること。
\item
  使用する液体\ldots{}\ldots{}水8 L。
\item
  加熱時間\ldots{}\ldots{}4時間。
\item
  作業手順\ldots{}\ldots{}ソース用の\protect\hyperlink{fonds-de-gibier}{ジビエのフォン}とまったく同じ。
\end{itemize}

\hypertarget{fonds-de-poisson-pour-gelee-ordinaire}{%
\subsubsection{標準的なジュレ用の魚のフォン}\label{fonds-de-poisson-pour-gelee-ordinaire}}

\frsub{Fonds de poisson pour gelée ordinaire}

\index{gelee@gelée!fonds poisson@Fonds de poisson pour gelée ordinaire}
\index{poisson@poisson!gelee@gelée!fonds@Fonds de --- pour gelée ordinaire}
\index{しゆれ@ジュレ!ひようしゆんてきなしゆれようのさかなのふおん@標準的なジュレ用の魚のフォン}
\index{さかな@魚!しゆれようのふおん@標準的なのジュレ用の---のフォン}

(仕上がり5 L分)

\begin{itemize}
\item
  主素材\ldots{}\ldots{}グロンダン\footnote{ホウボウ科の魚。和名カナガシラ。}、ヴィーヴ\footnote{ハチミシカ科の海水魚の総称。}、メルラン\footnote{鱈の近縁種。}などの安い魚750
  g、舌びらめのアラと端肉750 g。
\item
  香味素材\ldots{}\ldots{}薄切りにした\footnote{émincer エマンセ。}玉ねぎ200
  g、パセリの根2本、フレッシュなマッシュルームの切りくず100 g。
\item
  使用する液体\ldots{}\ldots{}やや薄めで透き通った仕上がりの魚のフュメ6
  L。
\item
  加熱時間\ldots{}\ldots{}45分間。
\item
  作業手順\ldots{}\ldots{}\protect\hyperlink{fumet-de-poisson}{魚のフュメ}と同じ。
\end{itemize}

\hypertarget{fonds-pour-gelee-de-poisson-au-vin-rouge}{%
\subsubsection{赤ワインを用いた魚のジュレ用のフォン}\label{fonds-pour-gelee-de-poisson-au-vin-rouge}}

\frsub{Fonds pour gelée de poisson au vin rouge}

\index{gelee@gelée!fonds poisson rouge@Fonds pour gelée de poisson au vin rouge}
\index{poisson@poisson!gelee@gelée!fonds rouge@Fonds pour gelée de poisson au vin rouge}
\index{しゆれ@ジュレ!あかわいんをもちいたさかなのしゆれようのふおん@赤ワインを用いた魚のジュレ用のフォン}
\index{さかな@魚!しゆれようのふおんあかわいん@赤ワインを用いた---のジュレ用のフォン}

このフォンは通常、鯉やトラウトなどの魚料理に用いられる。

このフォンに使用する液体は、良質なブルゴーニュ産赤ワインと\protect\hyperlink{fumet-de-poisson}{魚のフュメ}を同量ずつにする。魚のフュメは、ジュレが確実に固まるよう、ゼラチン質が多めのものを用いること。

風味付けは、魚に火を通すのに使った香味野菜によるもので充分だ。

\hypertarget{observation-sur-l-emplois-des-fonds-destines-aux-gelees}{%
\paragraph{ジュレ用のフォンについての注意}\label{observation-sur-l-emplois-des-fonds-destines-aux-gelees}}

\ldots{}\ldots{}ジュレ用のフォンは出来るだけ、使用する前日に仕込んでおくこと。いい具合に煮込んだら、浮き脂を取り除き\footnote{dégraisser
  デグレセ。}、漉してから陶製の容器に入れて冷ます。

冷めるとフォンは凝固する。取り除ききれなかったごくわずかな脂が表面に浮いてくるが、板状に固まるので容易に取り除くことが出来る。布あるいは漉し器でフォンを漉した際にすり抜けてしまった堆積物も自重で容器の底に沈むので、フォンを完全に澄ませることが出来る。
\end{recette}
\newpage

\hypertarget{clariication-des-gelees}{%
\subsection[ジュレのクラリフィエ]{\texorpdfstring{ジュレのクラリフィエ\footnote{clarification
  (クラリフィカスィオン)澄ませること、透明にさせること、の意の名詞だが、(1)本文にあるように、ただ単に「澄ませる」だけではなく、風味を補ったり強化し、色合いを調節する作業も兼ねていること、(2)現代日本の調理現場ではフランス語の動詞
  clarifier
  をカタカナにして「クラリフィエ」と呼ぶケースが多いことなどを考慮して、カタカナで動詞形のクラリフィエとした。なお、「クラリフェ」と呼ぶ現場もあるようだが、もとのフランス語がclarif\textbf{i}erとiの音があるのでこれは許容しがたい。}}{ジュレのクラリフィエ}}\label{clariication-des-gelees}}

\frsecb{Clarification des Gelées}

\index{gelee@gelée!clarification@Clarification des ---s}
\index{clarification@clarification!gelee@--- des gelées}
\index{しゆれ@ジュレ!くらりふいえ@---のクラリフィエ}
\index{くらりふいえ@クラリフィエ!しゆれ@ジュレの---}
\begin{recette}
\hypertarget{gelees-ordinaires}{%
\subsubsection[標準的なジュレ]{\texorpdfstring{標準的なジュレ\footnote{このgrasse
  \textless{} gras
  は「脂気のある、太った」の意ではなく、カトリックにおける「小斉」の食事を
  maigre
  と表現することと対になっているもの。すなわち「小斉ではない通常の」の意であることに注意。小斉については\protect\hyperlink{sauce-espagnole-maigre}{魚料理用ソース・エスパニョル}訳注および\protect\hyperlink{sauce-laguipiere}{ソース・ラギピエール}訳注参照。}}{標準的なジュレ}}\label{gelees-ordinaires}}

\frsub{Gelées grasses ordinaires}

\index{gelee@gelée!grasses ordinaire@---s grasses ordinaires}
\index{clarification@clarification!gelees grasses ordinaire@gelées  grasses ordinaires}
\index{しゆれ@ジュレ!ひようしゆんてきな@標準的な---}
\index{くらりふいえ@クラリフィエ!ひようしゆんてきな@標準的なジュレ}

(仕上がり5 L分)

\begin{enumerate}
\def\labelenumi{\arabic{enumi}.}
\item
  まずフォンの濃度を確認する。必要に応じて追加すべきゼラチンの量を調整する。
\item
  ジュレ用のフォンは充分に浮き脂を取り除き\footnote{dégraisser
    デグレセ。}、沈殿物も取り除い\footnote{décanter デカンテ。}てあること。
\item
  厚手で適切な大きさの片手鍋\footnote{casserole カスロール。}に、細挽き\footnote{ミートチョッパーやフードプロセッサが一般化する以前はアショワールhachoirという、両側に柄の付いた刃が湾曲した専用の包丁で細かく刻んでいた。}にした脂身のない\footnote{ここで原文はmaigreを用いているが、これはもちろん「脂気のない」の意。}赤身の牛肉
  500 gとセルフイユとエストラゴン計10 g、卵白3個分を入れる。
\item
  冷たい、あるいは生温い状態のジュレ用のフォンを挽肉の上から入れ、泡立て器かヘラで混ぜる。\\
  ゆっくり混ぜながら、強過ぎない程度の火加減で沸騰させる。卵白に含まれるアルブミンの分子が澄ませる作用を持っているので\footnote{やや大雑把な説明になるが、液体中に浮遊している不純物を抱き込むかたちで卵白が熱変性により凝固する、その結果として液体を「澄ませる」ことになる。ただし、これだけだと液体の味そのものや風味が薄くなってしまうために、それを補うあるいは強化する意味で挽肉や香草、香り付けの酒類を加える、ということ。}、混ぜることで卵白がまんべんなく広がるようにするわけだ。\\
  15分程、微沸騰の状態を保ち、目の詰まった布で漉す。
\end{enumerate}

\hypertarget{nota-gelees-grasses-ordinaires}{%
\subparagraph{【原注】}\label{nota-gelees-grasses-ordinaires}}

ジュレに酒類を添加するのは、ほぼ冷めた状態になってからにするのがいい。クラリフィエの作業中に酒類を加えるのは、沸騰しているために味が悪くなってしまうので、致命的な誤りでさえある。

そうではなく、ほぼ冷めた状態のジュレに酒類を添加すれば、その香気はそのまま保たれることになる。

作業の最後に酒類をジュレに添加すればジュレを薄めてしまう結果になるわけだからそれを考慮して、添加する酒類の量によっては、あらかじめジュレを充分に固めに作っておくのがいい。そうすれば、ジュレが固まるのに充分なゼラチンの濃度を保てるわけだ。

マデイラ酒、マルサラ酒、シェリー酒を加える場合の分量はジュレ1 Lあたり1
dLとすること。

ライン産のワインやシャンパーニュ、銘醸白ワインを加える場合は、ジュレ1
Lあたり2
dLとすること。加える酒類がどんなものであっても、文句ない程に良質のものを用いるべきだ。質の悪い酒類を加えてジュレの仕上がりを台無しにしてしまうくらいなら、加えないほうがまだましと言える。

\hypertarget{gelee-de-volaille}{%
\subsubsection{鶏のジュレ}\label{gelee-de-volaille}}

\frsub{Gelée de volaille}

\index{gelee@gelée!volaille@--- de volaille}
\index{clarification@clarification!gelee volaille@gelée de volaille}
\index{しゆれ@ジュレ!とりのしゆれ@鶏の---}
\index{くらりふいえ@クラリフィエ!とりのしゆれ@鶏のジュレ}
\index{しゆれ@ジュレ!うおらいゆの@ヴォライユの---}
\index{くらりふいえ@クラリフィエ!うおらいゆのしゆれ@ヴォライユのジュレ}

鶏のジュレのクラリフィエは標準的なジュレの場合とまったく同じに行なう。香味素材(セルフイユとエストラゴン)、澄ませるための材料(卵白)も同様にする。

ただし、味の補強に用いる肉については変更すること。すなわち牛の赤身肉を半量にして、残り半量は鶏の首肉にする。つまり、牛肉250
gと鶏の首肉250 g の挽肉を用いる。

\hypertarget{nota-gelee-de-volaile}{%
\subparagraph{【原注】}\label{nota-gelee-de-volaile}}

鶏のローストのガラを粗く砕いてエチューヴ\footnote{食品の乾燥などに主に用いられる低温のオーブンの一種。}でよく乾燥させて脂気を抜いたものを、このクラリフィエの際に加えると、素晴しい結果が得られる。

\hypertarget{gelee-de-gibier}{%
\subsubsection{ジビエのジュレ}\label{gelee-de-gibier}}

\frsub{Gelée de gibier}

\index{gelee@gelée!gibier@--- de gibier}
\index{clarification@clarification!gelee gibier@gelée de gibier}
\index{しゆれ@ジュレ!しひえ@ジビエの---}
\index{くらりふいえ@クラリフィエ!しひえのしゆれ@ジビエのジュレ}

クラリフィエの作業のやり方はまったく同様。ただし、このジュレを作る際には、いくつか留意すべきポイントがある。

標準的なジビエのジュレ、つまり特有の風味を持たせないものの場合は、味の補強には牛の挽肉250
gとジビエの赤身の挽肉250 gを用いること。

ジュレに独特の香りを持たせる必要がある場合には、必ず、肉それ自体に香気のあるジビエの肉、すなわち、ペルドロー、雉、ジェリノット\footnote{gélinotte
  雷鳥の一種。}などをクラリフィエの際に用いること。

どんなジビエのジュレでも仕上げに、ジュレ1
Lあたり大さじ2杯の上等なコニャックを加える。ただし、コニャックは絶対に良質のものでなければいけない。平凡なコニャックしか使えないのなら、これは省いたほうがいい。

この香り付けをしなくても、ジュレは不完全なものとはいえ、一応使えるものになる。いっぽうで、ありきたりのコニャックで香り付けすると、美味しくは仕上がらない。

\hypertarget{gelee-de-poisson-blanche}{%
\subsubsection{魚の白いジュレ}\label{gelee-de-poisson-blanche}}

\frsub{Gelée de poisson blanche}

\index{gelee@gelée!poisson blanche@--- de poisson blanche}
\index{clarification@clarification!gelee poisson blanche@gelée de poisson blanche}
\index{しゆれ@ジュレ!さかなのしろい@魚の白い---}
\index{くらりふいえ@クラリフィエ!さかなのしろいしゆれ@魚の白いジュレ}

魚のジュレのクラリフィエは以下のとおり\footnote{(1)または(2)の方法をとる、と解釈していいだろう。}。

\begin{enumerate}
\def\labelenumi{\arabic{enumi}.}
\item
  卵白を使う場合、ジュレ5
  Lあたり卵白3個分に、クラリフィエによって薄まってしまうのを補うためにメルランの身を細かく刻んだもの250
  gを加える。
\item
  もし可能なら新鮮なキャビア、なければ圧縮キャビア\footnote{\protect\hyperlink{beurre-de-caviar}{キャビアバター}訳注参照。}をジュレ1
  Lあたり50
  g用いる。方法は魚のコンソメのクラリフィエで説明している\footnote{概要は、キャビアをピュレ状にすり潰し、冷たい魚のコンソメでのばして加える。火にかけて絶えず混ぜながら沸かし、微沸騰の状態を20分保った後、布で漉す、という方法。}
  (ポタージュの章を参照)。
\end{enumerate}

魚のジュレの香り付けには、辛口のシャンパーニュもしくはブルゴーニュの銘醸白ワインを用いるといいが、\protect\hyperlink{gelees-ordinaires}{標準的なジュレ}の注において説明した酒類を加える場合の注意事項を勘案すること。

\hypertarget{nota-gelee-de-poisson-blanche}{%
\subparagraph{【原注】}\label{nota-gelee-de-poisson-blanche}}

場合によっては、ジュレ1
Lあたり4尾のエクルヴィスを用いることで、魚のジュレに独特の風味付けをすることも出来る。エクルヴィスをソテーしてビスクを作る要領で煮てから、鉢に入れて細かくすり潰し、最後に漉す作業の10分前に魚のフォンに加える。

\hypertarget{gelee-de-poisson-au-vin-rouge}{%
\subsubsection{赤ワインを用いた魚のジュレ}\label{gelee-de-poisson-au-vin-rouge}}

\frsub{Gelée de poisson au vin rouge}

\index{gelee@gelée!poisson vin rouge@--- de poisson au vin rouge}
\index{clarification@clarification!gelee poisson vin rouge@gelée de poisson au vin rouge}
\index{しゆれ@ジュレ!あかわいんをもちいたさかなの@赤ワインを用いた魚の---}
\index{くらりふいえ@クラリフィエ!あかわいんをもちいたさかなのしゆれ@赤ワインを用いた魚のジュレ}

このジュレのクラリフィエには、ジュレ5 Lあたり卵白4個分を用いる。

赤ワインで魚を煮ている途中や、ジュレのクラリフィエ作業の際に、タンニン由来の色素にすぐ変化してしまうことがしばしば、というかほぼ必ず起こる。ワインが分解してしまうのは魚のフュメに含まれているゼラチン質と接触して反応するためのようだ。こんにちに至るまで、これを避ける方法は見つかっていない。

そのため、色合いの不足を補うには人工色素(液体のカルミン\footnote{コチニール色素。\protect\hyperlink{beurres-composes}{合わせバター}本文および訳注参照。}か別の植物由来の色素)を加える必要がある。ただし、使用量にはごく細心の注意を払い、ジュレがやや深みをおびたバラ色を越えてしまわないようにすること。
\end{recette}\newpage


%%% Chapitre II. Garnitures
%% II. garnitures
\href{未、原文対照チェック}{} \href{未、日本語表現校正}{}
\href{未、その他修正}{} \href{未、原稿最終校正}{}

\begin{Main}

\hypertarget{garnitures}{%
\chapter{II. ガルニチュール}\label{garnitures}}

\frchap{Garnitures}

\index{garniture@garniture|(} \index{かるにちゆーる@ガルニチュール|(}

料理においてガルニチュール\footnote{garniture
  一般的には「付け合せ」と訳すが、本書におけるガルニチュールはたんなる料理の「付け合わせ」にとどまらず、こんにちではそれ自体がひとつの料理として成立し得るものも多い。そのため、あえて片仮名でガルニチュールとした。なお、「付け合わせ」の意味で「ガルニ」または「ガロニ」などというスラングを用いる調理現場もある。}は重要なものだから、料理人は決してガルニチュールの役割を軽視してはいけない。ガルニチュールの構成をどうするかは、添える料理の主素材との関係性で決まる。気まぐれ的なものや不自然なものは絶対にいけない。

ガルニチュールの構成要素は、場合によりけりだが、もっぱらどんな種類の料理に添えるかで決まる。具体的には、野菜料理やパスタ、ファルスでさまざまな形状に作ったクネル\footnote{quenelle
  仔牛肉や鶏肉、豚肉などと獣脂をすり潰して、しばしば「つなぎ」として後述のパナードを加えて練り、スプーンなどを用いて整形し、沸騰しない程度の温度で茹でる{[}ポシェ{]}またはオーブンで焼いたもの。スプーンを2つ使ってラグビーボールに似た形状にしたものが代表的だが、他にもいろいろな形状、大きさにする。}、あるいは雄鶏のとさかとロニョン\footnote{\protect\hyperlink{garniture-a-la-financiere}{ガルニチュール・フィナンシエール}やそのバリエーションともいえる\protect\hyperlink{garniture-godard}{ガルニチュール・ゴダール}で必須の素材。ロニョンrognonは通常なら腎臓を意味するが、この場合のロニョンは
  rognon blanc
  ロニョンブラン(白いロニョン)とも呼ばれるもので、雄鶏の精巣のこと。}、さまざまな種類の茸、オリーブとトリュフ、イカや貝および甲殻類、場合によっては卵、小魚、牛や羊の副生物\footnote{正肉以外の部分。例えば内臓や骨髄など。Ris
  de vea(リドヴォー)仔牛胸腺肉などはこれに含まれる。}など。

その昔、ガルニチュールというのは、マトロットやコンポート、ブルゴーニュ風料理などのように風味付けのために用いた素材がそのまま添えられたものであった。

ガルニチュールにする野菜は、どういう仕立ての皿にするかで役割が決まり、それに合うように切って形状を整え、調理する。ただし、野菜の調理法は「野菜料理」として調理する場合と同じだ。

パスタやイカ、貝類、甲殻類についても同様のことが言える。

この章では、それぞれのガルニチュールを構成する素材とその分量を示すに留めるので、各素材の調理法ついてはその素材に対応する章を参照すること。

\hypertarget{serie-des-farces-diverses}{%
\section[ファルス]{\texorpdfstring{ファルス\footnote{本来は「詰め物」の意で、鶏のローストの内臓を抜いた空洞部分に詰めたり、ガランティーヌやパテアンクルートの内部の詰め物などの用途に用いられる。この意味はこんにちでも変化がないが、本文にあるように、クネルにしてガルニチュールの一部にするなど、用途は多岐にわたる。本書ではファルスとして用いられるもののうち、肉および魚肉をベースにしたものをこの節にまとめて分類、説明している。したがって、ここでファルスとして挙げられていないファルスも料理によっては多い(例えば丸鶏の空洞部分に米などを詰めるのもファルス)ことに注意。}}{ファルス}}\label{serie-des-farces-diverses}}

\frsec{Série des farces diverses}

\index{farce@farce} \index{ふあるす@ファルス}

ガルニチュールの多くは、その構成要素にファルスあるいはファルスで作った「クネル」が含まれている。ファルスはまた、多くの大きな仕立ての料理にも使われる。ここではまずファルスの材料および作り方を示し、使い途については後で述べることにする。

ファルスは大きく5種に分類される。

\begin{enumerate}
\def\labelenumi{\arabic{enumi}.}
\item
  仔牛肉と脂で作るもの。すなわち古典料理における\ul{ゴディヴォ}。
\item
  基本となる材料はさまざまだが、「つなぎ」に主としてパナードを使うもの。
\item
  近代的な手法で、生クリームを用いてふんわり泡立てたファルス。ムース、ムスリーヌに用いる。
\item
  レバーをベースとした「ファルス・\ul{グラタン}」。種類はいろいろだが作り方は常に同じ。
\item
  \ruby{主}{おも}に\protect\hyperlink{}{ガランティーヌ}、\protect\hyperlink{}{パテアンクルート}、\protect\hyperlink{}{テリーヌ}などの冷製料理に用いるシンプルなファルス。
\end{enumerate}

\hypertarget{les-panades-pour-farces}{%
\subsection[ファルス用のパナードについて]{\texorpdfstring{ファルス用のパナードについて\footnote{パナードは本来、パンと水、バターを弱火で時間をかけて煮た粥のようなものを意味した。本書ではその意味を拡大して肉や魚肉をベースとしたファルスを加熱する際に崩れないようにする「つなぎ」として、この語を用いている。そのため、必ずしもパンを材料としていないものが含まれている。}}{ファルス用のパナードについて}}\label{les-panades-pour-farces}}

\frsecb{Les Panades pour Farces}

\index{farce@farce!panade@les panades pour farces}
\index{ふあるす@ファルス!はなーと@---用パナード}

ファルスに用いるパナードにはいくつもの種類がある。ファルスの種類や、そのファルスを添える料理の性質によって使い分けることとなる。

原則として、パナードの分量は、ファルスのベースとする素材が何であれ、その半量を越えないようにすること。

卵とバターを用いるパナードの場合はレシピの分量どおりに作らなければならないから、それを合わせて作るファルスの全体量のほうを調節してやること。

パナードE以外のパナードは使用する際には必ず完全に冷めた状態になっていること。パナードが出来上がったら、バターを塗った平皿か天板に流し広げ、早く冷めるようにする。このとき、バターを塗った紙で蓋をするか、表面にバターのかけらをいくつか置いてやり、パナードが直接空気に触れないようにしてやること。

以下のパナードのレシピは仕上がり重量が正味500
gになるように調整してある。

したがって、必要な量のパナードを作るのに材料を増やしたり減らしたりするのも難しくはないだろう\footnote{原文では、Rien
  de plus simple, donc, que \ldots{}
  となっており、直訳すると「これ以上に簡単なことはない」と言いきっているが、都度計算しなければならないことに変わりはないので、多少ニュアンスを柔らげて訳した。}。

\hypertarget{panades}{%
\subsection{パナード}\label{panades}}

\frsecb{Panades}

\index{panade} \index{はなーと@パナード}

\end{Main}

\begin{recette}

\hypertarget{panade-a}{%
\subsubsection{A. パンのパナード}\label{panade-a}}

\frsub{Panade au pain}

\index{panade!a pain@A. --- au pain}
\index{はなーと@パナード!a@A. パンの---}

(魚を素材にした固めのファルス用)

\begin{itemize}
\item
  材料\ldots{}\ldots{}沸かした牛乳3 dL、固くなった白パン\footnote{ここではいわゆるバゲットのようなパンの外側を削り落した白い部分、あるいは食パンの「耳」を切り落した白い部分を使う、ということ。なお、フランスのパンは使う小麦粉の精白度や種類によって、pain
    complet (パンコンプレ)全粒粉パン、pain de
    sègle(パンドセーグル)ライ麦パン、精白度の高い小麦粉と食塩、塩、パン種だけで作るバゲットなどの
    pain
    と、バターや砂糖を加えて作るヴィエノワズリ(クロワッサンやパンオショコラ、ブリオシュなど)に分けられる。イギリスやアメリカのいわゆる食パン(フランス語
    pain de mie
    パンドミ)は小麦粉、バター、塩、イースト菌、牛乳などで作られている。また、現代フランスでバゲットなどのパンに用いられている小麦粉の精白度は、T-55と呼ばれる灰分
    (小麦粉を燃やした際に残る炭水化物およびタンパク質以外の要素)0.5〜
    0.6%のものが主流であり、いわゆる食パンpain de
    mie(パンドミ)やヴィエノワズリにはT-45(灰分0.5%以下)が多く用いられている。このほか
    T-65(灰分0.62〜0.75%)およびT-80(灰分0.75〜0.9%)、T-110(灰分
    1.0〜1.2%)、T-150(灰分1.4%前後、いわゆる全粒粉)のように種類がある。このうちT-45およびT-55はfarine
    blanche(ファリーヌブロンシュ)と呼ばれ、T-150はfarine
    complète(ファリーヌコンプレット)と通称されている。灰分が高くなればそれだけ不純物が多いわけだから、粉は薄い茶色あるいはグレーがかった色合いになり、パンを焼く場合などはグルテン形成が難しくなりやすい。そのいっぽうで、香りゆたかなパンを実現しやすいという面もある。結果として、例えば全粒粉パンは香りはいいが固い仕上がりになりやすい。かつては精白度の低い(すなわち灰分の多い)粉ほど重量あたりの価格が安く、パンの価格もそれに比例していた。中世においてはパンの価格は基本的に1ドゥニエ(通貨単位)で、精白度の高いものは200〜300
    g、精白度の低いものは700〜800
    g程と大きな差があったという。ところで、本書では基本的に小麦粉を使う場合にその精白度についての指示はないが、概ねT-55またはT-45相当のもの考えていいだろう。なお、日本に輸入されている小麦は北米産のものがほとんどで、硬質小麦を粉にしたものが「強力粉」、軟質小麦の場合は「薄力粉」と呼ばれ、精白度合いによる分類は通常なされていないが、製品としては概ねT-45相当あるいはそれ以上の精白度のものが多い。}の身250
  g、塩5 g。
\item
  作業手順\ldots{}\ldots{}パンの身を牛乳に浸して完全にもどす。強火にかけて、ペースト状になったパンがヘラから簡単に取れるくらいまで水気をとばす。バターを塗った平皿か天板に広げ、冷ます。
\end{itemize}

\vspace*{1\zw}

\hypertarget{panade-b}{%
\subsubsection{B. 小麦粉のパナード}\label{panade-b}}

\frsub{Panade à la farine}

\index{panade!b farine@B. --- à la farine}
\index{はなーと@パナード!b@B. 小麦粉の---}

(肉、魚などあらゆるファルスに用いられる)

\begin{itemize}
\item
  材料\ldots{}\ldots{}水3 dL、塩2 g、バター50 g、篩にかけた小麦粉150 g。
\item
  作業手順\ldots{}\ldots{}片手鍋に水、塩、バターを入れて火にかけ、沸騰させる。火から外して小麦粉を加えて混ぜる。再度火にかけて、\protect\hyperlink{pate-a-chou}{シュー生地}を作る要領で余計な水分をとばす。上記パナードAと同様にして冷ます。
\end{itemize}

\atoaki{}

\hypertarget{panade-c}{%
\subsubsection[C. パナード・フランジパーヌ]{\texorpdfstring{C.
パナード・フランジパーヌ\footnote{フランジパーヌとは製菓で用いられる、小麦粉、砂糖、卵を混ぜて牛乳とバニラを加えて煮、砕いたマカロンmacaronを加えたクリーム。本文にあるように、このパナード・フランジパーヌにはマカロンは加えないので、作り方のプロセスが途中まで似ていることからの命名だろう。なお、本来のクレーム・フランジパーヌに用いられるマカロンは、現代日本でよく知られているタイプとは異なり、すり潰したアーモンドと卵白、砂糖を混ぜた生地を紙の上にクルミ大に絞り出してオーブンで焼いただけのシンプルなもの。Macaron
  craquelé(マカロンクラクレ)はこのタイプの代表的なもので、焼く際に膨らんで割れ目が出来ることからクラクレ(裂け目のある)の名称が付けられた。ところで、日本にマカロンが伝わった時期は不明だが、このタイプのものが太平洋戦争前には、アーモンドを落花生に代え、「まころん」の名称でいくつかの製菓会社で製造されるようになり、現在も生産されている。フランス語
  macaron の初出はボッカッチョ『デカメロン』のフランス語訳で、原文
  maccheroni
  の訳語として現われる。ただ、このフランス語訳は異本も多く、そのうちの写本のひとつに
  macaronという語が見られるに過ぎない点で、フランス語への影響という意味では微妙なところだ。むしろ既にフランス語として存在したmacaron
  と音が似ているからというだけの理由で訳語としてあてた可能性さえある。ボッカッチョの原書におけるマッケーローニはこんにちのそれ(マカロニ)とは違い、ニョッキのようなものだったと解釈されるのが定説であり、「マッケローニやラヴィオリを去勢鶏のブロードで煮る」という文脈で出てくる。次にmacaronという語がフランス語の文献で現われるのは16世紀フランソワ・ラブレーの小説『ガルガンチュアとパンタグリュエル』の「第四の書」であり、長い献立リストの一部として登場する(p.678)。このリストにおいて``Poupelin,
  Macaron. Tartres vingt
  sortes.''「ププラン(パティスリの一種)、マカロン、20種ものタルト」と並んでいることから、ボッカッチョのマッケローニとはまったく違うものであることがわかる。また、17世紀には上述のようなマカロンの存在は知られていたという説があり、さらにフランス革命期にカルメル会修道女たちが隠れて作っていたというmacaron
  des
  soeurs(マカロン・クラクレのタイプで平たい形状)はナンシーの名物としてこんにちも有名。なおこのmacaron
  des soeurs のsoeurs
  は「姉妹たち」の意味ではなく「(修道女である)シスター」のことなので間違えないよう注意。}}{C. パナード・フランジパーヌ}}\label{panade-c}}

\frsub{Panade à la Frangipane}

\index{panade!c frangipane@C. --- à la Frangipane}
\index{はなーと@パナード!c@C. ---・フランジパーヌ}

(鶏のファルス、魚のファルス用)

\begin{itemize}
\item
  材料\ldots{}\ldots{}小麦粉125 g、卵黄4個、溶かしバター90 g、塩2
  g、こしょう1 g、おろしたナツメグの粉ごく少量、牛乳2 \(\frac{1}{2}\)
  dL。
\item
  作業手順\ldots{}\ldots{}片手鍋に小麦粉と卵黄を入れてよく練る。溶かしバター、塩、こしょう、ナツメグを加える。沸かした牛乳で少しずつ溶きのばしていく。
\end{itemize}

\protect\hyperlink{creme-frangipane}{標準的なフランジパーヌ}と同様に、火にかけて5〜6分間、泡立て器で混ぜながら煮る。ちょうどいい漉さになったら、バットに移して\footnote{débarasser
  (デバラセ)バットなどに移す、片付ける、の意。とりわけ前者の意味に注意。}冷ます。

\atoaki{}

\hypertarget{panade-d}{%
\subsubsection{D. 米のパナード}\label{panade-d}}

\frsub{Panade au Riz}

\index{panade!d riz@D. --- au Riz}
\index{はなーと@パナード!d@D. 米の---}

(いろいろなファルスに用いられる)

\begin{itemize}
\item
  材料\ldots{}\ldots{}米200 gすなわち2
  dLあるいは大さじ8杯。\protect\hyperlink{}{白いコンソメ}6 dL、バター20
  g。
\item
  作業手順\ldots{}\ldots{}米を入れた鍋にコンソメを注ぎ、バターを加える。火にかけて沸騰させたら、オーブンに入れて40〜45分間加熱する。この間、米に触れないようにすること。
\end{itemize}

オーブンから出したら、米粒がよく潰れるようにヘラでしっかりと混ぜる。その後、冷ます。

\hypertarget{panade-e}{%
\subsubsection{E. じゃがいものパナード}\label{panade-e}}

\frsub{Panade à la pomme de terre}

\index{panade!e riz@E. --- à la pomme de terre}
\index{はなーと@パナード!e@E. じゃがいもの---}

(仔牛および他の白身肉の、詰め物\footnote{fourrré
  (フレ)詰め物をした。farci
  (ファルシ)も同様に「詰め物をした」の意だが、後者はより一般的で、前者はオムレツやクレープに中身を詰めて「包む」のが本来の意味。すなわち、このパナードを加えたファルスで、何らかの素材を「包む」と解釈してもいい。とりわけこの
  fourréには日本料理の用語「射込む」をあてる場合もある。}をする大きなクネルに用いられる)

\begin{itemize}
\item
  材料\ldots{}\ldots{}茹でて皮を剥いたばかりの中位のサイズのじゃがいも2個、牛乳3
  dL、塩 2 g、白こしょう \(\frac{1}{2}\) g、ナツメグ少々、バター20 g。
\item
  作業手順\ldots{}\ldots{}牛乳を2.5 dLになるまで煮詰める\footnote{原文は
    réduire le lait d'un sixième 直訳すると「牛乳を
    \(\frac{1}{6}\)量だけ煮詰める」すなわち「\$\frac{5}{6}量まで煮詰める」のだが、かえって分かりにくいため、具体的な数字に直して訳した。分量を代えて作る場合には85%まで煮詰めるくらいと考えてもいいだろう。そもそも、じゃがいもの重さが曖昧なのだから、あまり細かい数字にこだわらず臨機応変に考えること。}。バター、調味料、薄く輪切りにしたじゃがいもを加え、15分間程加熱する。
\end{itemize}

このパナードはまだ少し\ruby{温}{ぬる}いくらいで使用すること。完全に冷めてからではいけない。完全に冷めてから練ると粘りが出てしまうからだ。

\end{recette}

\begin{Main}

\hypertarget{farces}{%
\subsection{ファルス}\label{farces}}

\frsecb{Farces}

\index{farce} \index{ふあるす@ファルス}

ベースとなる素材が\ul{仔牛}、\ul{鶏}、\ul{ジビエ}あるいは\ul{甲殻類}であっても、分量と作業手順はどんなファルスでも同じだ。そのベースにする素材を代えればいいのだから、ここでは各種ファルスの典型的なレシピを示せば充分だろう。料理で用いられるファルスひとつひとつを説明するのに一章をあてる必要はないと思われる。

\end{Main}

\begin{recette}

\hypertarget{farce-a}{%
\subsubsection{A. パナードとバターを用いるファルス}\label{farce-a}}

\frsub{Farce à la Panade et au beurre}

\index{farce!a@A. --- à la Panade et au beurre}
\index{ふあるす@ファルス!a@A. パナードとバターを用いる---}

(標準的なクネル、肉料理\footnote{原文 entrée
  (アントレ)、現代では「前菜」の意味で用いられるが、本書では Relevé et
  Entrée
  「ルルヴェとアントレ」すなわち肉料理の章に収録されているレシピ、仕立てのこと。これらのうちとりわけ大掛かりな仕立てのものをルルヴェ、それ以外をアントレと考えていい。本来ルルヴェもアントレも魚を主素材にした仕立てが少なからずあったり、17世紀〜
  19世紀前半にかけての料理書では、いかに魚料理を大掛かりでゴージャスな仕立てでしかも美味なものにするか、が大きなテーマを占めていた。本書ではこれら四旬節の際などの「小斉」すなわち「肉断ちの料理」にあまりこだわらない傾向があるために「魚料理」としてまとめられている。アントレの場合は、概ね10人前を一皿に盛ったものを指し、現代でも立派にメインの料理として通用するものがほとんど。実際、英語での前菜は
  hors-d'oeuvre または appetizer の語を用い、メインデュッシュには entree
  (またはフランス語のまま entrée)の語が現代でもあてられている。}の縁飾り
etc.)

\begin{itemize}
\item
  材料\ldots{}\ldots{}ていねいに筋取りをした肉1
  kg、\protect\hyperlink{panade-b}{パナードB} 500 g、塩12 g、こしょう2
  g、全卵4個、卵黄8個。
\item
  作業手順\ldots{}\ldots{}肉をさいの目に切って鉢に入れ、調味料を加えてすり潰す。いったん肉を取り出して、パナードをよくすり潰しながらバターを加える。肉を戻し入れ、すりこ木\footnote{pilon
    (ピロン)形状は日本のすりこ木をやや異なるのが多い。裏漉し用の漉し器(tamis
    タミ)とともに用いるピロンの場合は、棒の端に円盤状のやや厚い板を付けた形状のものが多かった。現代の手動式のポテトマッシャーのようなイメージだろうか。なお、ここでは大理石の鉢もしくは陶製のボウルを用いて作業していることに注意。現代ではフードプロセッサなどを用いるところだろうが、かつては人力で、力を込めて丁寧に作業していたということは頭に留めておきたい。}で力強く練って全体をまとめる。
\end{itemize}

次に全卵と卵黄を加えて混ぜ合わせる。これは2回に分けても1回でやってもいい。裏漉しして陶製の容器に入れる。さらに泡立て器で滑かになるまで混ぜる。

\hypertarget{nota-farce-a}{%
\subparagraph{【原注】}\label{nota-farce-a}}

どんな種類のファルスを作る場合でも、必ず少量を沸騰しない程度の温度で茹でて\footnote{pocher
  (ポシェ)。}テストしてから、クネルの整形に取りかかること。

\atoaki{}

\hypertarget{farce-b}{%
\subsubsection{B. パナードと生クリームを用いるファルス}\label{farce-b}}

\frsub{Farce à la Panade et à la Crème}

\index{farce!b@B. --- à la Panade et à la crème}
\index{ふあるす@ファルス!b@B. パナードと生クリームを用いる---}

(滑らかな仕上がりのクネル用)

\begin{itemize}
\item
  材料\ldots{}\ldots{}筋取りをした肉1
  kg、\protect\hyperlink{panade-c}{パナードC} 400 g、卵白5 個分、塩15
  g、白こしょう2 g、ナツメグ1 g、クレーム・ドゥーブル \footnote{乳酸発酵させた濃い生クリーム。フランスの生クリームについては\protect\hyperlink{sauce-supreme}{ソース・シュプレーム}訳注参照。}1
  \(\frac{1}{2}\) L。
\item
  作業手順\ldots{}\ldots{}どんな肉を使う場合でも、卵白を少しずつ加えながらしっかりとすり潰すこと。
\end{itemize}

パナードを加え、すりこ木でしっかり練り、二つの素材がよくよく混ざり合うようにする。

目の細かい網で裏漉しし、鍋にファルスを入れる。ヘラで滑らかになるよう混ぜ、鍋を氷の上に置いて一時間ほど休ませる。

生クリームの
\(\frac{1}{3}\)量を少しずつ加えながら、のばしていく。最終的に残りの\(\frac{2}{3}\)の生クリームも加えるが、これは先に泡立て器で軽く立てておくこと。

生クリームを全部加えた時点で、ファルスは真っ白で滑らかでしかも、ふんわりとした仕上がりにならなくてはいけない。

\hypertarget{nota-farce-b}{%
\subparagraph{【原注】}\label{nota-farce-b}}

手に入った生クリームが必ずしも最上級のものでない場合には、パナードC
を用いて\protect\hyperlink{farce-a}{バターを用いたファルス}を作った方がまだいい。

\atoaki{}

\hypertarget{farce-c}{%
\subsubsection{C.
生クリームを用いる滑らかなファルス/ファルス・ムスリーヌ}\label{farce-c}}

\frsub{Farce à la Crème, ou Mousseline}

\index{farce!c@C. --- fine à la crème, ou Mousseline}
\index{mousseline!farce mousseline}
\index{ふあるす@ファルス!c@C. 生クリームを用いる滑らかな---/---・ムスリーヌ}
\index{むすりーぬ@ムスリーヌ!ふあるす@ファルス・---}

(ムース、ムスリーヌ、ポタージュ用クネルなど)

\begin{itemize}
\item
  材料\ldots{}\ldots{}丁寧に掃除をして筋取りをした肉1
  kg、卵白4個分、クレーム・エペス\footnote{crème épaisse fraîche
    低温殺菌の後、乳酸醗酵させたとても濃い生クリーム。前出のクレーム・ドゥーブルよりも濃い。}1
  \(\frac{1}{2}\) L、塩18 g、白こしょう3 g。
\item
  作業手順\ldots{}\ldots{}肉と調味料を鉢に入れて細かくすり潰す。卵白を少量ずつ加えていく。目の細かい網で裏漉しする。
\end{itemize}

これをソテー鍋に入れ、ヘラで滑らかになるまで混ぜたら、たっぷりの氷で鍋を囲むようにして2時間冷やす。

次に、生クリームを少しずつ加えながらファルスをのばしていく。丁寧に練っていくこと。またこの作業は鍋底を常に氷にあてた状態で行なうこと。

\hypertarget{nota-farce-c}{%
\subparagraph{【原注】}\label{nota-farce-c}}

\ldots{}\ldots{} 1.
上で示した生クリームの分量は平均的な数字だ。ファルスのベースとなっている素材つまり肉、魚、甲殻類によってそれぞれタンパク質の特性が違うのだから、素材に吸収される生クリームの量には多少の違いがでてくるわけだ。

\begin{enumerate}
\def\labelenumi{\arabic{enumi}.}
\setcounter{enumi}{1}
\item
  ここで示したファルスの作り方は、滑らかな仕上がりのファルスの典型であって、これを越える繊細さを出せるものはないから、ファルスに出来る材料すべて、つまり各種の肉、ジビエ、鶏、魚、甲殻類などに適用していい。
\item
  卵白の量は、ファルスのベースと素材によって調整する必要がある。鶏や仔牛肉のようにアルブミンが多く含まれていて\footnote{当時の知見であることに注意。卵白が主としてアルブミンで出来ているのは事実だが、肉については現代の知見と大きなズレがある。本書において、赤身肉は「オスマゾーム」という架空の、茶褐色をした美味しさのエキスのようなものが豊富に含まれており、仔牛などの白身肉はアルブミンが主体であるとする考え方が随所に認められる。現代ならイノシン酸の「うま味」とテクスチュア、焼いた場合はメイラード反応による香気成分などが美味しさを感じさせる要素であると考えるところだが、フランス料理は長い歴史においてイノシン酸というアミノ酸の一種が「うま味」成分であるということを知らずに、けれども経験的にイノシン酸の比率が増えるようにブイヨンあるいはフォン、ソースなどの味を追究していった。イノシン酸やグルタミン酸、グアニル酸などのアミノ酸による「うま味」の概念そのものが、20世紀末になってようやく認知されるようになったに過ぎない。あくまでも「経験則」にもとづいて美味しさの探求が行なわれてきたと言える。}新鮮な肉であれば、成獣の固くなった肉を使う場合よりも量は少なくて済む。つまり、捌いたばかりでまだ温かい若鳥の胸肉を使ってこのファルス・ムスーズを作るのであれば、卵白は省略してもいい。
\item
  良質の生クリームが入手できる環境にあるなら、他のファルスを作るよりもこのファルスの方がいいだろう。とりわけ、甲殻類をベースとしたファルスについては重要なことだ。
\end{enumerate}

\end{recette}

\begin{Main}

\hypertarget{godiveau}{%
\subsection[ゴディヴォ/仔牛肉とケンネ脂のファルス]{\texorpdfstring{ゴディヴォ\footnote{ゴディヴォgodiveau
  はフランソワ・ラブレーの小説『ガルガンチュアとパンタグリュエル』の「第三の書」(1546年)が初出。原書の綴りは
  guodiveaulx。これは「アンドゥイエット(のようなもの)」と一般に解釈されている。ラブレーはこれに先立つ1534年「ガルガンチュア」(=第一の書)において
  gaudebillaux
  という表現を用いている。これについては「ゴドビヨとは、たっぷり肥育した牛のトリップ(胃と腸)のこと」と本文で説明している。これらを敷衍すると、ゴディヴォはもともと牛などの胃や腸を刻んで詰めた腸詰すなわちアンドゥイエットのことだった、と考えたくなっても不思議はない。しかし、たとえ16世紀のラブレーにおけるゴディヴォが当時アンドゥイエットと呼ばれるものとほぼ同じだったとしても、アンドゥイエット
  andouilette がアンドゥイユ andouille
  に縮小辞を付したものであることから、中世のアンドゥイユを確認する必要が出てくる。14世紀末に書かれた『ル・メナジエ・ド・パリ』においてアンドゥイユは確かに「細かく刻んだ胃や腸を、腸詰にする」という説明がまず出てくるが、その他に、牛の第1胃だけを詰めるもの、豚のコトレットを切り出した端肉を材料にするもの、胸腺肉やレバーを掃除した残りの肉を材料にするもの、が挙げられている(t.2,p.127)。これに従うなら、中世におけるアンドゥイユとは素材の定義があまりはっきりしていなかったと考えられる。ところが17世紀、ピエール・ド・リュヌ『新料理の本』(1660年)に「スペイン風アンドゥイエット」というレシピがある。概要を記すと、仔牛肉を細かく刻む。豚背脂少々、香草、卵黄、塩、こしょう、ナツメグ、粉にしたシナモンを加える。豚背脂のシートで巻いてアンドゥイエットの形状にする。串を刺してローストする。ローストする際に滴り落ちてくる肉汁は受け皿で受ける。火が通ったらその肉汁をかける。茹で卵の黄身8〜10個分と細かくおろしたパン粉を順につけて、しっかりした衣を作る。提供時にレモン汁と羊のジュをかけ、揚げたパセリを添える、というものだ。1693年刊マシアロ『宮廷および大ブルジョワ料理の本』では豚のアンドゥイユ、仔牛のアンドゥイユとともに、仔牛のアンドゥイエットというレシピが掲載されている。最後のものには材料として「細かく刻んだ仔牛肉、豚背脂、香草、卵黄、塩、こしょう、ナツメグ、シナモンを加えて作る」とある(pp.108-109)。また、1750年に出版された『食品、ワイン、リキュール事典』でも、アンドゥイエットは「細かく刻んだ仔牛肉を楕円形に巻いたもの」と定義されている。実際、17、18世紀の料理書に出てくるアンドゥイエットは腸詰であるかどうかは別にしても、仔牛肉を主材料にしたものが多い。18世紀ヴァンサン・ラ・シャペル『近代料理』第1巻のアンドゥイエットも細かく刻んだ仔牛肉を豚の腸に詰めて作る。さて、ゴディヴォに戻ると、17世紀、1653年刊の『フランスのパティスリの本』(ラ・ヴァレーヌが著者だと言われている)にはFaire
  un pasté de gaudiueau
  「ゴディヴォのパテの作り方」という節があり、仔牛腿肉あるいは他の肉と脂身を細かく刻んだもの、をパテ(≒パイ包み焼き)に入れる。つまりここでも「仔牛腿肉」の使用が前提となっている。したがって、これら勘案すれば、ラブレーのゴディヴォもまた仔牛肉を材料にしていたものだった可能性は充分に考えられるだろう。
  もちろんゴドビヨという別の巻で出てくる名詞との関連性は無視出来ないものだが、中世〜ルネサンス期において、食にかかわる名詞、概念がしばしば曖昧だったことを考えると、多少のわかりにくさは許容せざるを得ない。したがって、本書において仔羊腿肉とケンネ脂を使うゴディヴォを「古典的」なファルスとして扱っているのはまことに正鵠を射ていると言えよう。}/仔牛肉とケンネ脂のファルス}{ゴディヴォ/仔牛肉とケンネ脂のファルス}}\label{godiveau}}

\frsecb{Farce de Veau à la Graisse de boeuf, ou Godiveau}

\index{farce!veau graisse de boeuf@--- de veau à la graisse de boeuf}
\index{farce!veau glodiveau@Godiveau}
\index{ふあるす@ファルス!こうしにくとけんねあふらのふあるす@牛仔牛肉とケンネ脂の---/ゴディヴォ}
\index{ふあるす@ファルス!こていうお@ゴディヴォ} \index{godiveau}
\index{こていうお@ゴディヴォ}

\end{Main}

\begin{recette}

\hypertarget{godiveau-mouille-a-la-glace}{%
\subsubsection[A. 氷を入れて作るゴディヴォ]{\texorpdfstring{A.
氷を入れて作るゴディヴォ\footnote{氷を入れて作る方法についてはカレームが1815年刊『パリ風パティスリの本』の「シブレット入りゴディヴォ」原注において詳しく論じている。「不思議なことだが、氷を入れることでゴディヴォが滑らかなテクスチュアになり、素晴しくふんわりとしてとてもいい柔らかさに仕上がる。ゴディヴォが変質してしまうと、部分的とはいえそのクオリティはまったく失なわれてしまう。これは夏によく起こる事で、あまりに暑いとその熱で牛脂が仔牛肉としっかりつながらなくなってしまうからだ。一方(仔牛肉)は水分を含んでいて、もう一方(牛脂)は脂質そのものだからだ。だから、夏の暑い時期には必ず氷を加えて作るべきであり、逆に冬場はそこまでする必要はない(p.142)」。ほぼ同時期のヴィアール『王国料理の本』
  1817年版においてゴディヴォのレシピの末尾に、「夏に、水の代わりに少量でも氷を使えるならそのほうがずっといい仕上がりになる(p.145)」と書かれている。これは、製氷機、冷凍庫が実用化されるのが19世紀中頃なので、それよりやや早い時代ということになり、カレームの主たる活躍の舞台であった食卓外交というものが、いかに贅沢だったかを示しているとも言えよう。言うまでもなく、17〜18世紀の料理書、パティスリの本においてゴディヴォのレシピは多く見られるが、氷の使用について言及したものはいまのところ見つかっていない。}}{A. 氷を入れて作るゴディヴォ}}\label{godiveau-mouille-a-la-glace}}

\frsub{Godiveau mouillé à la glace}

\index{farce@farce!godiveau a@Godiveau A. Godeiveau mouillé à la glace}
\index{ふあるす@ファルス!こていうお@ゴディヴォ!a@A. 氷を入れて作るゴディヴォ}
\index{godiveau@godiveau!a@A. --- mouillé à la glace}
\index{こていうお@ゴディヴォ!a@A. 氷を入れて作る---}

\begin{itemize}
\item
  材料\ldots{}\ldots{}筋をきれいに取り除いた仔牛腿肉1
  kg、\ul{水気を含んでいない}牛ケンネ脂\footnote{腎臓の周囲を厚く覆っている脂肪。融解温度が低く、精製して牛脂(ヘット)の原料となる。}1.5
  kg、全卵8個、塩25 g、白こしょう5 g、ナツメグ1 g、透明な氷7〜800
  gまたは氷水7〜8 dL。
\item
  作業手順\ldots{}\ldots{}はじめに、仔牛肉とケンネ脂を別々に、細かく刻む。仔牛肉はさいの目に切り、調味料と合わせておく。牛脂は細かくして、薄皮は筋はきれいに取り除いておく。
\end{itemize}

仔牛肉と牛脂を別々の鉢に入れて、それぞれすり潰す。次にこれらを合わせてから、完全に混ざり合って一体化するまでよくすり潰し、卵を一個ずつ、すり潰す作業を止めずに加えていく。

裏漉しして、平皿に\footnote{大きなバット。}広げ、氷の上に置いて翌日まで休ませる。

翌日になったら、再度ファルスをすり潰す。この時、小さく割った氷を少しずつ加えていき、よく混ぜ合わせる。

ゴディヴォに氷を加え終わったら、必ずテスト\footnote{少量を、沸騰しない程度の温度で火を通し(ポシェ)て様子を見ること。}を行ない、必要に応じて修正する。固すぎるようなら水を少々加え、柔らかすぎるようなら卵白を少し加えること。

\hypertarget{nota-godiveau-a}{%
\subparagraph{【原注】}\label{nota-godiveau-a}}

ゴディヴォで作ったクネルはもっぱら、\protect\hyperlink{vol-au-vent}{ヴォロヴァン}の詰め物\footnote{原文
  garniture ガルニチュールの意味が広いことに注意。}にしたり、牛、羊の塊肉の料理に添える\protect\hyperlink{garniture-a-la-financiere}{ガルニチュール・フィナンシエール}に用いられる。

他のクネルがどれもそうであるように、沸騰しない程度の温度で茹でて\footnote{pocher
  (ポシェ)。}
火を通せばいいが、一般的には手で整形して塩を加えた沸騰しない程度の温度の湯で茹でる。

だが、「ポシャジャセック\footnote{pochage à sec
  直訳すると「乾燥した状態でポシェすること」。つまり水(湯)を用いずに、pocher
  と同様に低めの温度で加熱することを指している。}」と呼ばれる技法、すなわち弱火のオーブンで焼くのがいちばんいい。

以下に示す方法はとても短時間で出来るので特にお勧めだ。

ゴディヴォは充分に氷を加えて水気を含んだ状態にしておく。オーブンの天板に敷いたバターを塗った紙の上に、丸口金を付けた絞り袋から絞り出す。オーブンの天板にもバターを塗っておくこと。絞り出したクネルは触れ合うようにしていい。

これを低温のオーブンに入れて加熱する。

7〜8分すると、クネルの表面に脂が水滴状に浸み出してくる。これが、ちょうどいい具合に火が通った合図だ。オーブンから出して、クネルを別の銀製の盆か大理石の板の上に裏返しに広げる。クネルが\ruby{微温}{ぬる}くなるまで冷めたら、敷いてあった紙を端のほうから引き剥して取り除く。

クネルは完全に冷めるまで放置し、その後に皿に移すか、可能なら柳編みのすのこに載せてやるのがいい。

\atoaki{}

\hypertarget{godiveau-a-la-creme}{%
\subsubsection{B. 生クリーム入りゴディヴォ}\label{godiveau-a-la-creme}}

\frsub{Godiveau à la crème}

\index{farce@farce!godiveau b@Godiveau B. Godeiveau  à la crème}
\index{ふあるす@ファルス!こていうお@ゴディヴォ!b@B. 生クリーム入りゴディヴォ}
\index{godiveau@godiveau!b@B. --- à la crème}
\index{こていうお@ゴディヴォ!b@B. 生クリーム入り---}

\begin{itemize}
\item
  材料\ldots{}\ldots{}筋をきれいに取り除いた極上の白さの仔牛腿肉1
  kg、水気を含んでいない牛ケンネ脂1 kg、全卵4個、卵黄3個、生クリーム7
  dL、塩25 g、白こしょう5 g、ナツメグ1 g。
\item
  作業手順\ldots{}\ldots{}仔牛肉とケンネ脂は別々に、細かく刻む。これらを鉢に入れて合わせ、調味料、全卵、卵黄をひとつずつ加えながら、力強く全体をすり潰し、完全に一体化させる。
\end{itemize}

裏漉しして、天板に広げる。氷の上にのせて翌日まで休ませる。

翌日になったら、あらかじめ中に氷を入れて冷やしておいた鉢で再度すり潰す。この際に生クリームを少量ずつ加えていく。

クネルを整形する前にテストをして、必要があれば固さなどを修正してやること。

\atoaki{}

\hypertarget{godiveau-lyonnais}{%
\subsubsection[C.
リヨン風ゴディヴォ/ケンネ脂入りブロシェのファルス]{\texorpdfstring{C.
リヨン風ゴディヴォ\footnote{このレシピは第二版以降。このファルスが仔牛肉が材料ではなくパナードも使うにもかかわらずゴディヴォの名称である根拠はおそらく、ケンネ脂を用いていることだろう。なお、これを用いたブロシェのクネルの起源については、リヨンのシャルキュトリ(豚肉加工業者)であるオ・プチ・ヴァテルの店主ルイ・レグロスが1907年に創案したものだという説がある。しかしこの説は、1907年の本書第二版にファルスとクネル両方のレシピが収録されているといることで否定されよう。また、19世紀前半にオーヴェルニュ・ローヌ・アルプ地方にある宿屋の主J.-F.
  モワーヌなる人物が、宿泊客を呼び込むための料理としてブロシェの身と卵、小麦粉で作ったクネルを創案し、これがリヨンに伝わったという説もある。ただしこれは信憑性がさほど高くないうえに、そもそもケンネ脂を使わないのであれば、その後のリヨン風ゴディヴォとは似て非なるものということになろう。魚のすり身をクネルにすることはローマ時代後期の『アピキウス』(この場合は人物ではなく料理書の意)以来、ヨーロッパにおいてごくあたりまえのように行なわれてきたことだ。いずれにしても、本書では\protect\hyperlink{quenelles-de-brochet-lyonnaise}{ブロシェのクネル リヨン風}のレシピでのみこのファルスが用いられることになる。その意味でも、「ブロシェのクネル リヨン風」という料理が20世紀初頭に大流行したものだったことは間違いなく、そのことが理由で第二版においてレシピが追加されたと考えられる。}/ケンネ脂入りブロシェのファルス}{C. リヨン風ゴディヴォ/ケンネ脂入りブロシェのファルス}}\label{godiveau-lyonnais}}

\frsub{Godiveau Lyonnais ou Farce de Brochet à la graisse}

\index{farce@farce!godiveau c@Godiveau C. Godeiveau Lyonnais ou Farce de Brochet à la graisse}
\index{ふあるす@ファルス!こていうお@ゴディヴォ!c@C. リヨン風ゴディヴォ/ケンネ脂入りブロシェのファルス}
\index{godiveau@godiveau!c@C. --- Lyonnais ou Farce de Brochet à la graisse}
\index{こていうお@ゴディヴォ!c@C. リヨン風---/ケンネ脂入りブロシェのファルス}

\begin{itemize}
\item
  材料\ldots{}\ldots{}皮とアラをきれいに取り除いたブロシェ\footnote{brochet
    ノーザンパイク、和名キタカワカマス。カワカマス属の淡水、汽水魚。}の身(正味重量)500
  g、筋を取り除き細かく刻んだ水気を含んでいない牛ケンネ脂500
  g(またはケンネ脂と白い牛骨髄半量ずつ)、\protect\hyperlink{panade-c}{パナード
  C}500 g、卵白4 個分、塩15 g、こしょう4 g、ナツメグ1 g。
\item
  作業手順\ldots{}\ldots{}まず鉢でブロシェの身をすり潰す。これを取り出して、次にケンネ脂にパナード(よく冷やしたもの)を加えてすり潰し、卵白を少しずつ加えていく。ブロシェの身と調味料を入れ戻す。すりこ木で力強く練り、裏漉しする。
\end{itemize}

陶製の器に移し、ヘラで滑らかになるまで練る。使うまで、氷の上に置いておく。

次のように作ってもいい。ブロシェの身を調味料とともにすり潰し、そこにパナードを加える。裏漉しして、鉢に戻す。すりこ木で力強く練ってまとまるようになったらケンネ脂を少しずつ加えるか、溶かしたケンネ脂と牛骨髄を加えて、よくまとめる。陶製の器に移し、氷の上に置いておく。

\atoaki{}

\hypertarget{farce-de-veau-pour-bordures}{%
\subsubsection{盛り付けの縁飾りおよび底に敷いたり、詰め物をしたクネルに用いる仔牛のファルス}\label{farce-de-veau-pour-bordures}}

\frsub{Farce de veau pour Bordures de dressage, fonds, quenelles fourrées etc.}

\index{farce!veau@--- de veau pour Bordures de dressage, fonds, quenelles fourrées, etc.}
\index{ふあるす@ファルス!こうし@盛り付けの縁飾りおよび底に敷いたり、詰め物をしたクネルに用いる仔牛の---}

\begin{itemize}
\item
  材料\ldots{}\ldots{}筋をきれいに取り除いた\ul{極上の白さの\\仔牛腿肉}1
  kg、\protect\hyperlink{panade-e}{パナード E} 500 g、バター300
  g、全卵5個、卵黄8個、濃い冷えた\protect\hyperlink{sauce-bechamel}{ベシャメルソース}大さじ2杯、塩20
  g、白こしょう 3 g、ナツメグ1 g。
\item
  作業手順\ldots{}\ldots{}鉢に仔牛肉と調味料を入れてて細かくすり潰す。これを鉢から取り出す。
\end{itemize}

まだ温い状態のじゃがいものパナードを入れ、すりこ木でペースト状になるまで練り、だいたい冷めた頃に、先にすり潰した仔牛肉を戻し入れる。全体によく混ぜながら、バター、全卵、卵黄をひとつずつ加えていき、最後に冷たいベシャメルソースを加える。

裏漉しして、陶製の器に入れ、充分に滑らかになるまでヘラで練る\footnote{装飾用、あるいは中に別の食材を射込んだ大きなクネルを作る目的なので、加熱後はしっかりとしたテクスチュアとなる。クネルにする場合も、アトレと呼ばれる飾り串を刺して料理を飾るのが主要な目的で、トリュフを射込むなど、料理としてきちんと成立していた。なおアトレはエスコフィエの時代にフランスではほぼ用いられなくなっていたが、アメリカ経由で
  19世紀中頃のフランス料理をベースに始まった日本の西洋料理では、20世紀になってからも使われ続けていたという。}。

\index{farce@farce!gratin@--- gratin}
\index{gratin@gratin!farce@farce ---}
\index{ふあるす@ファルス!くらたん@ファルス・グラタン}
\index{くらたん@グラタン!ふあるす@ファルス・---}

\atoaki{}

\hypertarget{farce-gratin-a}{%
\subsubsection[ファルス・グラタン
A]{\texorpdfstring{ファルス・グラタン\footnote{ここでゴディヴォのように小見出しがあって然るべきところだが、初版には小見出しの類が一切なかったので、第二版改訂の際に見落とされてそのままになったのだろう。本書におけるファルス・グラタンの定義が、決して「グラタン用」ファルスではないことに注意。語源的には
  gratin \textless{} gratter
  (グラテ)引っ掻く、であり、元来はbouillie(ブイイ)という粥のようなものの鍋底や隅に貼り付いた部分のことをグラタンと呼んだ。
  18世紀マラン『コモス神の贈り物』には「グラタン」という名称のファルスがある。これは、鶏胸肉、レバー、牛の骨髄、香草などと卵黄をすり潰して練ったもの(t.1,
  p.143)。また仕立てとしてのグラタンは深皿にこのファルスを敷き詰め、その上に別途調理した素材をのせてソースをかけ、フルノーの端でファルスが容器に貼り付く程度に加熱する(仔牛の耳のグラタン(id.,
  p.209)、エクルヴィスのグラタン(id.,
  pp.171-172)がある。その後、グラタンという名称のファルスは他の料理書に記されなかったが、
  1868年のデュボワとベルナールの『古典料理』においてfarce à gratin de
  gibier, farce à gratin de foie-grasの2つのレシピが掲載され
  (p.125)、その約半世紀後『料理の手引き』において完全に復活したが、その頃にはグラタンという仕立てがまったく別の、こんにち我々がよく知っているものへと変わってしまっていた。このため、本書におけるグラタンの説明(原書pp.405-407)においてもこれらのファルス・グラタンは用いられない。}
A}{ファルス・グラタン A}}\label{farce-gratin-a}}

\frsub{Frace Gratin A}

\index{farce@farce!gratin a@--- Gratin A}
\index{ふあるす@ファルス!くらたんa@---・グラタン A}

(標準的な温製パテ\footnote{pâté
  とは本来、生地で素材を包んで焼いたもの全般を指す。こんにちではその意味が失なわれつつあり「パイ包み」のような表現をとることも多い。決して英語のpatty(小型のミートパイ、ハンバーガーのパティなど)と混同しないこと。}、大皿料理\footnote{Entrée
  アントレ。\protect\hyperlink{panade-a}{パナードとバターを用いるファルス}訳注参照。}の縁飾りなど)

\begin{itemize}
\item
  ファルス1 kg分の材料\ldots{}\ldots{}豚背脂250
  g、筋をきれいに取り除いた極上の白さの仔牛腿肉1
  kg、出来るだけ白い仔牛のレバー250 g、バター150
  g、マッシュルームの切りくず40
  g、トリュフの切りくず(可能なら生のもの) 25 g、卵黄6個、ローリエの葉
  \(\frac{1}{2}\)枚、タイム1枝、エシャロット4個、塩20 g、こしょう4
  g、ミックススパイス\footnote{原文は初版から一貫して、2 grammes
    d'épices 直訳すると「香辛料2
    g」としか記されていないが、フランスでもっともポピュラーなミックススパイスであるquatre-épicesカトルエピスの場合は、こしょう、ナツメグ、クローブ、シナモンの粉末のミックス。また「オールスパイス」単独を意味することもある。なお、1907年の英語版には、ローリエ5オンス、タイム3オンス、コリアンダー3オンス、シナモン4オンス、ナツメグ6オンス、クローブ4オンス、ジンジャーパウダー3オンス、メース3オンス、黒こしょうと白こしょう同量ずつ計10オンス、カイエンヌ1オンス、を粉末にして保存すべし(p.75)、とあるが、フランス語原書にこのミックススパイスのレシピはいずれの版でも記されていない。}2
  g、マデイラ酒1 \(\frac{1}{2}\)
  dL、\protect\hyperlink{sauce-espagnole}{ソース・エスパニョル}1
  \(\frac{1}{2}\) dL(よく煮詰めてあって、冷やしてあること)。
\item
  作業手順\ldots{}\ldots{}豚背脂をさいの目に切る。ソテー鍋に50gのバターを熱し、強火で色よく焼く。
\end{itemize}

背脂が色付いたらすぐに取り出して余分な脂をきり、同じ鍋で、大きめのさいの目に切った仔牛肉を色よく焼く。同様してに余分な脂はきる。

同じく強火で、仔牛肉と同様に切ったレバーを色よく焼く。仔牛肉と背脂を鍋に戻し入れ、マッシュルームの切りくず、トリュフの切りくず、タイム、ローリエの葉、みじん切りにしたエシャロットと調味料を加える。2分程火にかけたままにし、バットにあける。ソテー鍋にマデイラ酒を注いでデグラセ\footnote{肉を焼く際に肉から浸み出た肉汁が濃縮して鍋底に貼り付いているのを、何らかの液体を注いで溶かし出すこと。意味としては「焦げ」を取ることではないので注意。}する。

鉢に背脂、仔牛肉、レバーなどを入れて細かくすり潰しながら、バターの残り(100
g)と卵黄をひとつずつ加えていく。さらに煮詰めたソース・エスパニョルとデグラセしたマデイラ酒を加える。裏漉しして、陶製の容器に入れ、ヘラで滑らかになるまで練る。

\hypertarget{nota-farce-gratin-a}{%
\subparagraph{【原注】}\label{nota-farce-gratin-a}}

このファルスのレシピでの仔牛のレバーは鶏や鴨、がちょう、七面鳥のレバーに代えてもいい。その場合は、胆汁および胆汁で汚れた部分を丁寧に取り除く必要がある。

\atoaki{}

\hypertarget{farce-gratin-b}{%
\subsubsection{ファルス・グラタン B}\label{farce-gratin-b}}

\frsub{Frace Gratin B}

(ジビエの温製パテ用)

\index{farce@farce!gratin b@--- Gratin B}
\index{ふあるす@ファルス!くらたんb@---・グラタン B}

\begin{itemize}
\item
  ファルス1 kg分の材料\ldots{}\ldots{}塩漬け豚バラ肉250
  g、穴うさぎの\footnote{lapin de garenne
    (ラパンドガレーヌ)、野生の穴うさぎ。いわゆる野うさぎlièvre(リエーヴル)とは肉質も違い、まったく別のものとして扱われる。この穴うさぎを家畜化したものが、いわゆるlapin(ラパン)。}肉(正味重量)250
  g、鶏とジビエのレバー250
  g、マッシュルーム、トリュフ、タイム、ローリエ、エシャロット、塩こしょうは\protect\hyperlink{farce-gratin-a}{ファルス・グラタン
  A}と同じ。バター50 g、生あるいは加熱済みのフォワグラ 100
  g、卵黄6個、マデイラ酒1 \(\frac{1}{2}\)
  dL、ジビエで作った\protect\hyperlink{sauce-espagnole}{ソース・エスパニョル}または\protect\hyperlink{sauce-salmis}{ソース・サルミ}をよく煮詰めて冷ましたもの1
  \(\frac{1}{2}\) dL。
\item
  作業手順\ldots{}\ldots{}前項で説明したように、バターで3種の素材、つまり豚バラ、うさぎ肉、レバーを別々に色よく焼く。これらをソテー鍋に調味料、香辛料とともに入れ、軽く炒めたらマデイラ酒を注ぎ蓋をして弱火で5分程蒸し煮\footnote{étuver
    (エチュヴェ)。}する。よく水気をきってから鉢に入れてすり潰す。充分に滑らかになったら、フォワグラと卵黄、冷めたソースとマデイラ酒を加える。裏漉しして、ヘラで滑らかになるまで混ぜる。
\end{itemize}

\atoaki{}

\hypertarget{farce-gratin-c}{%
\subsubsection{ファルス・グラタン C}\label{farce-gratin-c}}

\frsub{Frace Gratin C}

\index{farce@farce!gratin c@--- Gratin C}
\index{ふあるす@ファルス!くらたんc@---・グラタン C}

(詰め物をしたクルトン、カナペ、小型ジビエ、仔鴨用)

\begin{itemize}
\item
  ファルス1 kg分の材料\ldots{}\ldots{}生のフレッシュな豚背脂\footnote{塩漬けなどの加工をしていないということ。なお、lard
    (gras) (ラール グラ)は「豚背脂」を意味し、lard maigre
    (ラールメーグル)またはlard de
    poitrine(ラールドポワトリーヌ)は塩漬け豚ばら肉およびそれを冷燻したものを意味する。後者はしばしば日本語で「ベーコン」と誤訳されるが、日本語でいう「ベーコン」は温燻、熱燻されたものであり、風味などが大きく異なるので注意。近年は「生ベーコン」という商品名のものもあるらしく、紛らわしいので注意が必要だろう。いずれにしても、豚背脂は薄いシート状または長い棒状、拍子木状にして、素材の油脂分と風味を補う目的で使われることが多く、豚ばら肉の塩漬けおよびその冷燻品は拍子木状に切って(lardon
    ラルドン)各種料理に使われる。既に拍子木状にカットされたものがごく一般的に市販されており、それぞれ
    lardon(ラルドン)、lardon
    fumé(ラルドンフュメ)と呼ばれ非常にポピュラーな食材。}を器具を用いておろしたもの\footnote{râper
    (ラペ) \textless{} râpe (ラープ)という器具を用いておろすこと。
    Mandeline
    (マンドリーヌ)と呼ばれる野菜スライサーにこの機能が付属しているものは非常に多い。}300
  g、鶏レバー600 g、エシャロット4〜5個の薄切り \footnote{émincé
    \textless{} émincer (エマンセ)薄切りにする、スライスする。}、マッシュルームの切りくず\footnote{マッシュルームは通常、料理として提供する際にはtourner
    (トゥルネ)と呼ばれる、螺旋状の切れ込みを入れて装飾したものが使われる。この際に少なくない量の切りくずが発生する(具体的には軸込みで15〜20%の廃棄率だが、このファルス・グラタンにおいては、口あたりを損ねる可能性があるので軸、石突きは使わないと考えるべき)のでそれを利用する。なおtournerの原義は「回す」であり、包丁を持った側の手は動かさずに材料を回すようにして切れ目を入れたり皮を剥いたりすることを意味する料理用語。}25
  g、ローリエの葉 \(\frac{1}{2}\)枚、タイム1枝、塩18 g、こしょう3
  g、ミックススパイス3 g。
\item
  作業手順\ldots{}\ldots{}ソテー鍋に豚背脂を熱して溶かす。レバーと香辛料、調味料を加え、強火で\ul{色付かないように}炒める。
\end{itemize}

いま、\ul{色付かないように}\footnote{原文 raidir ou saisir
  (レディール ウ セジール)。前者は油脂を熱したフライパン等で、材料が色付かないように表面を焼き固めること。後者「セジール」は焼く、炒める、茹でるなど方法は問わないが、熱によって表面だけを固める(タンパク質の熱変性)ことを指す。}と書いたように、焼き色を付けないようにすることがポイント。レバーはレアな焼き加減で血が滴るくらいにすると、バラ色のきれいなファルスに仕上がる\footnote{現代の衛生学的知見からすると、充分に加熱調理していないレバーには食中毒あるいは肝炎などのリスクがあるので注意。}。

材料がだいたい冷めたら鉢に入れてすり潰す。裏漉しして、陶製の容器に移してヘラで練って滑らかにする。バターを塗った紙で蓋をして冷蔵する。

\end{recette}

\begin{Main}

\hypertarget{farce-pour-les-pieces-froides}{%
\subsection{冷製料理用のファルス}\label{farce-pour-les-pieces-froides}}

\vspace{-1.5\zw}
\begin{center}
\textbf{(ガランティーヌ、パテアンクルート、テリーヌ)}
\end{center}
\vspace{.5\zw}
\frsecb{Farces pour Pièces froides}
\begin{center}
\vspace{-1\zw}
\hspace{1\zw}\textbf{(Galantines --- Pâtés --- Terrines)}
\end{center}

\index{farce@farce!froides@--- pour les pièces froides}
\index{ふあるす@ファルス!れいせいりようりよう@冷製料理用の---}

\normalfont

\end{Main}

\begin{recette}

\hypertarget{assaisonnement-et-liaison}{%
\subsubsection{味付けと「つなぎ」}\label{assaisonnement-et-liaison}}

\frsub{Assaisonnement et Liaison}

ガランティーヌや、パテアンクルート、テリーヌに用いる標準的なファルスは、ファルス1
kgあたり25〜30 gのスパイスソルトで調味する。最後に、肉1
kgあたりコニャック1 \(\frac{1}{2}\) dLを振りかける。

\vspace*{.5\zw}

冷製料理用のファルスは以下のように3つに分類される。これらは前述の滑らかな口あたりのファルスやファルス・グラタンとはまったく違うものである。

「つなぎ」が必要な場合には、ファルス1 kgあたり全卵2個を加えて調整する。

\atoaki{}

\hypertarget{sel-epice}{%
\subsubsection{スパイスソルト}\label{sel-epice}}

\frsub{Sel épicé}

\index{sel epice@sel épicé} \index{すぱいすそると@スパイスソルト}
\index{こうしんりよういりしお@香辛料入りの塩 ⇒ スパイスソルト}

スパイスソルトはよく乾燥した細かい塩100 gと、こしょう20
g、ミックススパイス\footnote{\protect\hyperlink{farce-gratin-a}{ファルス・グラタン
  A}訳注参照。}20 gを混ぜて作る。

すぐに使わない場合は、密閉できる缶に入れて乾燥した場所で保存すること。

\atoaki{}

\hypertarget{farce-froide-a}{%
\subsubsection{ファルス A (豚肉)}\label{farce-froide-a}}

\frsub{Farce A (Porc)}

\index{farce@farce!froide a@--- pour les pièces froides A (Porc)}
\index{ふあるす@ファルス!れいせいa@冷製料理用--- A}

これは豚肉の脂身のない部分と、フレッシュな背脂を同量ずつ用いる。別々に細かく刻むこと。それを鉢に入れて合わせてすり潰し、調味と風味付けを上記の分量比率で行なう。

ごく標準的なパテアンクルートやテリーヌに用いらる。

これは「\protect\hypertarget{chair-a-saucisse}{ソーセージ用の挽肉}」\footnote{chair
  à saucisses
  (シェラソシス)。料理書によってはよく出てくる表現なので覚えておくといいだろう。}としても使われる。\label{chair-a-saucisse}

\atoaki{}

\hypertarget{farce-froide-b}{%
\subsubsection{ファルス B (仔牛肉と豚肉)}\label{farce-froide-b}}

\frsub{Farce A (Porc)}

\index{farce@farce!froide b@--- pour les pièces froides B (Veau et Porc)}
\index{ふあるす@ファルス!れいせいb@冷製料理用--- B}

\begin{itemize}
\item
  材料\ldots{}\ldots{}仔牛腿肉の輪切り250
  g、さいの目に切った豚肉の脂身を含まない部分250
  g、フレッシュな豚背脂500
  g、全卵2個、調味料とコニャックは上記のとおり。
\item
  作業手順\ldots{}\ldots{}仔牛肉、豚肉、背脂を別々に細かく刻む。調味料とともに鉢に入れてよくすり潰し、最後に、火を点けてアルコールをとばした\footnote{flamber
    (フロンベ)。フランベする。鍋に入れて火にかけるとコニャックのようにアルコール度の高い酒類はすぐにアルコール分が揮発して非常に燃えやすくなる。}
  コニャックを加える。裏漉しする。
\end{itemize}

このファルスは主としてガランティーヌに使うが、パテアンクルートやテリーヌに用いてもいい。

\atoaki{}

\hypertarget{farce-froide-c}{%
\subsubsection{ファルス C (鶏とジビエ)}\label{farce-froide-c}}

\frsub{Farce C (Volaille et Gibier)}

\index{farce@farce!froide c@--- pour les pièces froides C (Volaille et Gibier)}
\index{ふあるす@ファルス!れいせいc@冷製料理用--- C}

このファルスの素材はいろいろだから、分量比率は使用する鶏とジビエの肉の正味重量\footnote{poids
  net (ポワネット)。}から調節することになる。

例えば、中抜きしただけの丸鶏の重量\footnote{廃棄分なども含めた全重量は
  poids brut (ポワブリュット)。}が1.5
kgの場合、ガルニチュールに使うフィレの量は500〜600
gに減ってしまうことになる。そのため、ファルスの材料の分量比率は以下のようになる。

鶏肉550 g、きれいに筋取りした仔牛肉200 g、豚肉の脂身のないところ200
g、生の豚背脂900
g、全卵4個、\protect\hyperlink{sel-epice}{スパイスソルト}50〜60
g、コニャック3 dL。

作業手順\ldots{}\ldots{}肉と背脂は別々にして、それぞれ細かく刻む。これを鉢に入れて合わせ、調味料を加える。細かくすり潰しながら卵を一個ずつ加えていく。コニャックは最後に加えること。裏漉しする。

ジビエのファルスも同様の材料の比率で、同じように作る。

\atoaki{}

\hypertarget{observation-sur-les-farces}{%
\subsubsection{冷製料理用ファルスの補足}\label{observation-sur-les-farces}}

場合によっては、ファルスB(仔牛と豚)およびファルスC(鶏)に、ファルス 1
kgあたりフォワグラ125
gを加えることがある。その場合フォワグラは出来るだけ新鮮なものを用いて、裏漉しして加えること。あるいはトリュフのみじん切り50
gを加えることもある。

ジビエのファルスCを極上の滑らかな仕上りにするには、\(\frac{1}{4}\)量の\protect\hyperlink{farce-gratin-b}{ファルス・グラタンB}と、ファルスのベースにしたジビエのフュメをよく煮詰めて少量加えるといい。

\end{recette}

\begin{Main}

\hypertarget{farces-speciales-pour-garnir-les-poissons-braises}{%
\subsection[魚のブレゼのガルニチュール用ファルス]{\texorpdfstring{魚のブレゼ\footnote{本書において魚を「煮る」あるいは「茹でる」場合、通常はクールブイヨンか塩水で沸騰させない程度の温度で火入れをする(ポシェ)。料理の仕立てとしての「ブレゼ」は基本が牛、羊の赤身肉であり、仔羊、仔牛、家禽などはやや例外的な位置付けとして「ブレゼ」が存在する。同様に、サーモン、大型のトラウト、チュルボ、チュルボタンなどについても「ブレゼ」という仕立ての方法が\protect\hyperlink{cuisson-des-poissons-par-le-braisage}{「第6章魚料理」}において説明されているので併せて読んでおきたい。}のガルニチュール用ファルス}{魚のブレゼのガルニチュール用ファルス}}\label{farces-speciales-pour-garnir-les-poissons-braises}}

\frsecb{Farces spéciales pour garnir les Poissons Braisés}

\end{Main}

\begin{recette}

\hypertarget{farces-poissons-braises-a}{%
\subsubsection{ファルス A}\label{farces-poissons-braises-a}}

\frsub{Farce A}

\index{farce@farce!poissons braises a@---s spéciales pour les poissons braisés A}
\index{ふあるす@ファルス!さかなのふれせa@魚のブレゼのガルニチュール用--- A}

\begin{itemize}
\item
  材料\ldots{}\ldots{}細かく刻んだ生の白子\footnote{laitance
    (レトンス)。伝統的な高級料理では鯉の白子が一般的に使用された。他に鯖や鰊の白子も食用とするが、日本のようにスケトウダラの白子を食材とするケースはほとんどないと思われる。}250
  g、白いパンの身180 gを牛乳に浸して絞ったもの、塩5g、こしょう1
  g、ナツメグごく少量、シブレット 10gとパセリの葉5 g、セルフイユ20
  gをみじん切りにしたもの。バター50 g、全卵1個、卵黄3個。
\item
  作業手順\ldots{}\ldots{}陶製の鉢に材料をすべて入れ、木のヘラで全体をよく練り、完全にまとまるようにする。
\end{itemize}

\atoaki{}

\hypertarget{farces-poissons-braises-b}{%
\subsubsection{ファルス B}\label{farces-poissons-braises-b}}

\frsub{Farce B}

\index{farce@farce!poissons braises b@---s spéciales pour les poissons braisés B}
\index{ふあるす@ファルス!さかなのふれせb@魚のブレゼのガルニチュール用--- B}

\begin{itemize}
\item
  材料\ldots{}\ldots{}白いパンの身200
  gを牛乳に浸して絞ったもの。玉ねぎ50 gとエシャロット25
  gを細かいみじん切りにしてバターで炒めたもの。ごく新鮮なマッシュルームをみじん切りにし、圧して余分な水分を絞ったもの。パセリのみじん切り大さじ1杯、叩き潰したにんにく1片、全卵1個、卵黄3個、塩8
  g、こしょう2 g、ナツメグごく少量。
\item
  作業手順\ldots{}\ldots{}ファルスAと同じ。
\end{itemize}

\end{recette}

\begin{Main}

\hypertarget{quenelles-diverses}{%
\subsection[クネル]{\texorpdfstring{クネル\footnote{ローマ時代後期に成立した料理書アピキウスにも甲殻類やイカをはじめとした各種素材のすり身を丸めて作るクネルとも呼ぶべきレシピが多く見られるように、とても古くからある調理だが、フランス語のquenelleという語それ自体は意外と新しく、18世紀頃に定着したと思われる。語源はドイツ語の
  Knödel
  (クヌーデル)すなわちボール状にした食べものを意味する語からの移入と考えられている。荘厳で華麗な装飾を施した大掛かりな仕立てがとりわけ好まれた17、18世紀の宮廷料理においてその装飾の一部としてクネルの利用が広まり、発達したのだろう。また、\protect\hyperlink{godiveau}{ゴディヴォ}の訳注において触れたように、ピエール・ド・リュヌのアンドゥイエットなどは仔牛肉をすり潰したものを棒状にして豚背脂で包んで焼くという、まさしく本書におけるゴディヴォの調理法に近いものであり、これもまた一種のクネルと言えるだろう。}}{クネル}}\label{quenelles-diverses}}

\index{quenelle} \index{garniture!quenelle} \index{クネル}
\index{かるにちゆーる@ガルニチュール!くねる@クネル}

\frsecb{Quenelles diverses}

クネルは大きさや形状がさまざま。

\begin{enumerate}
\def\labelenumi{\arabic{enumi}.}
\item
  粉を打った台の上で転がして小さな円筒形にする
\item
  絞り袋に詰めてバターを塗った天板に絞り出す
\item
  スプーンを使って整形する
\item
  指で丸めて、雄鶏のロニョン\footnote{ロニョンrognonは通常は腎臓のことだが、rognon
    de coq は精巣のこと。高級食材として珍重された。}のような形状にする
\end{enumerate}

クネルの作り方のその他の詳細はよく知られていることだから、本書ではこれ以上は述べないことにする。加熱方法についても同様としたい。

ただ、以下の点には留意していただきたい。\protect\hyperlink{garniture-a-la-financiere}{フィナンシエール}や\protect\hyperlink{garniture-toulouse}{トゥールーズ}といった標準的なガルニチュールに加えるクネルはコーヒースプーンを使って整形するか、丸口あるいは刻み模様が入る口金を使って絞り出すこと。

こうやって作る場合のクネルは平均で、ひとつ12〜15 g程度となる。

\protect\hyperlink{garniture-godard}{ガルニチュール・ゴダール}や\protect\hyperlink{garniture-regence}{レジャンス}、\protect\hyperlink{garniture-chambord}{シャンボール}に使うような大きなクネルの場合は、必ずスプーンを用いて整形し、20〜22
gの大きさにすること。

上記のような大がかりなガルニチュールでよく用いられる、装飾を施したクネルの場合、大きさは40〜50
g、球形か卵形、あるいはやや長い卵形にすること。

装飾に用いる素材は、ほとんど常にトリュフ、\protect\hyperlink{saumure-liquide-pour-langues}{赤く漬けた舌肉}のどちらか、あるいは両方を用いて、生の卵白でクネルに貼り付けて固定する。

\protect\hyperlink{godiveau}{ゴディヴォ}のクネルは茹でずに低めの温度のオーブンで加熱していいが、それ以外は1
Lあたり10
gの塩を加えた湯で、沸騰しない程度の温度で茹でること。整形したクネルを並べたソテー鍋や天板に、沸騰した塩湯を注ぎ、沸騰寸前の温度を保つようにして火を通すこと。

\end{Main}
\newpage
\href{未、原文対照チェック}{} \href{未、日本語表現校正}{}
\href{未、その他修正}{} \href{未、原稿最終校正}{}
\begin{Main}
\hypertarget{serie-des-appareiles-et-preparations-diverses-pour-garnitures-chaudes}{%
\section[温製ガルニチュール用アパレイユなど]{\texorpdfstring{温製ガルニチュール用アパレイユ\footnote{料理用語としての
  appareil
  アパレイユとは、具体的な何かを指す言葉ではなく、\ul{ある料理を作る過程において用いられる、複数\\の材料を組み
  合わせたもの}、という一種の概念。現実には、キッシュのアパレイユ(生クリームと卵、塩漬け豚バラ肉など)、クレーム・ブリュレのアパレイユ(卵黄、砂糖、生クリーム、牛乳)というように用いられるが、概ね、
  \ul{加熱して凝固する液体\\または半液状のもの、およびそれらを「つなぎ」
  として固形物をあえたものを指す}、と考えていい。アパレイユの概念としては、まったくの固形物である、3〜4
  mmのさいの目に切った香味野菜(場合によってはハムも入る)である\protect\hyperlink{matignon}{マティニョン}も
  appareil à matignon
  と表現されることはフランスの料理書においては珍しくないし、本節の\protect\hyperlink{duxelles-seche}{デュクセル・セッシュ}もまたアパレイユの一種に含められる。実際のところ、アパレイユという語はそれぞれの調理現場および料理人によって使い方がさまざまであり、概念としての理解も必ずしも共通しているとは限らない。本書では基本的に、上述のように加熱凝固する液体の場合と、半固形状あるいはクリーム状のものを指す場合がほとんど。また、既出の\protect\hyperlink{sauce-villeroy}{ソース・ヴィルロワ}などもまた、ソースというよりはむしろアパレイユと呼んでおかしくないものだろう。}など}{温製ガルニチュール用アパレイユなど}}\label{serie-des-appareiles-et-preparations-diverses-pour-garnitures-chaudes}}

\frsec{Série des Appareils et Préparations diverses pour Garnitures chaudes}

\index{garniture@garniture!appareils chaudes@appareils et préparations diverses pour ---s chaudes}
\index{appareil@appareil!garnitures chaudes@--- et préparations diverses pour garnitures chaudes}
\index{かるにちゆーる@ガルニチュール!あはれいゆおんせい@温製---のためのアパレイユなど}
\index{あはれいゆ@アパレイユ!おんせいかるにちゆーる@温製ガルニチュールのための---など}
\end{Main}
\begin{recette}
\hypertarget{appareils-a-cromesquis-et-a-croquettes}{%
\subsubsection[クロメスキとクロケットのアパレイユ]{\texorpdfstring{クロメスキとクロケット\footnote{クロケットは日本のコロッケの原型となったもので、細かく切った素材をじゃがいものピュレや\protect\hyperlink{sauce-bechamel}{ベシャメルソース}であえて円盤または円筒形に整形してパン粉衣を付けて揚げたもの。クロメスキは正六面体(サイコロ形)にすることが多く、コロッケとアパレイユが共通のため、形状が違うだけでクロケットのバリエーションという見方もあるが、ポーランド語のkromesk(薄く切ったもの)が語源とされる。}のアパレイユ}{クロメスキとクロケットのアパレイユ}}\label{appareils-a-cromesquis-et-a-croquettes}}

\frsub{Appareils à Cromesquis et à Croquettes}

\index{appareil@appareil!cromesquis croquettes@---s à cromesquis et à croquettes}
\index{cromesqui@cromesqui!appareil@appareils à --- et à croquettes}
\index{croquette@croquette!appareil@appareils à cromesquis et à ---}
\index{あはれいゆ@アパレイユ!くろめすきとくろけつと@クロケットとクロメスキの---}
\index{くろめすき@クロメスキ!あはれいゆ@---とクロケットのアパレイユ}
\index{くろけつと@クロケット!あはれいゆ@クロメスキと---のアパレイユ}

⇒ \protect\hyperlink{hors-d-oeuvres-chauds}{温製オードブル}の章を参照。

\atoaki{}

\hypertarget{appareils-a-pomme-dauphine-duchesse-marquise}{%
\subsubsection[じゃがいものドフィーヌ、デュシェス、マルキーズのアパレイユ]{\texorpdfstring{じゃがいものドフィーヌ、デュシェス、マルキーズ\footnote{dauphin(王太子)、dauphine(王太子妃)、duc(公爵)、
  duchesse(公爵夫人)、mariquis(侯爵)、marquise(侯爵夫人)。いずれも王家、貴族の位階(爵位)を表わす語だが、特に理由もなく料理名に付けられることが非常に多い。}のアパレイユ}{じゃがいものドフィーヌ、デュシェス、マルキーズのアパレイユ}}\label{appareils-a-pomme-dauphine-duchesse-marquise}}

\frsub{Appareils à pomme Dauphine, Duchesse et Marquise}

\index{appareil@appareil!pomme dauphine@---s à pomme Dauphine, Duchesse et Marquise}
\index{dauphin@dauphin(e)!appareil@appareil à pomme ---e}
\index{duc@duc / duchesse!appareil@appareil à pomme duchesse}
\index{marquis@marquis(s)!appareil@appareil à pomme ---e}
\index{あはれいゆ@アパレイユ!しやかいものとふいーぬと@じゃがいものドフィーヌ、デュシェス、マルキーズの---}
\index{とふいーぬ@ドフィーヌ!あはれいゆ@アパレイユ!しやかいものとふいーぬ@じゃがいもの---、デュシェス、マルキーズのアパレイユ}
\index{てゆしえす@デュシェス!あはれいゆ@アパレイユ!しやかいものてゆしえす@じゃがいものドフィーヌ、---、マルキーズのアパレイユ}
\index{まるきーす@マルキーズ!あはれいゆ@アパレイユ!しやかいものまるきーす@じゃがいものドフィーヌ、デュシェス、---のアパレイユ}

⇒ \protect\hyperlink{legumes}{野菜料理}の章、\protect\hyperlink{pommes-de-terre}{じゃがいも}の項を参照。

\atoaki{}

\hypertarget{appareil-maintenon}{%
\subsubsection[アパレイユ・マントノン]{\texorpdfstring{アパレイユ・マントノン\footnote{マントノン夫人(出生名フランソワーズ・ドビニェ
  1635〜1719)。はじめはマントノン侯爵夫人としてルイ14世とモンテスパン夫人の間に生まれた子どもたちの非公式な教育係となり、モンテスパン夫人の死後、ルイ
  14世と結婚した。彼女の名を冠した料理はここで言及されている\protect\hyperlink{cotelettes-maintenon}{羊のコトレット マントノン}の他、卵料理、菓子などにある。「羊のコトレット マントノン」は彼女自身が考案したとも、ルイ14世付の料理人の考案ともいわれているが、いずれも憶測の域を出ない。なお、côtelette(コトレット)とは仔牛、羊の背肉を骨付きで肋骨1本ずつに切り分けたもの。日本語では、仔羊の場合ラムチョップと呼ばれることも多い。}}{アパレイユ・マントノン}}\label{appareil-maintenon}}

\frsub{Appareils Maintenon}

\index{appareil@appareil!maintenon@--- Maintenon}
\index{maintenon@Maintenon!appareil@appareil ---}
\index{あはれいゆ@アパレイユ!まんとのん@---・マントノン}
\index{まんとのん@マントノン!あはれいゆ@アパレイユ・---}

(\protect\hyperlink{cotelettes-maintenon}{羊のコトレット マントノン}用)

\protect\hyperlink{sauce-bechamel}{ベシャメルソース}4
dLと\protect\hyperlink{sauce-soubise}{スビーズ}1
dLを半量になるまで煮詰める。

卵黄3個を加えてとろみを付ける。あらかじめマッシュルーム100
gを薄切りにしてバターでごく弱火で鍋に蓋をして蒸し煮\footnote{étuver
  (エチュヴェ)。}したものを加える。

\atoaki{}

\hypertarget{appareil-montglas}{%
\subsubsection[アパレイユ・モングラ]{\texorpdfstring{アパレイユ・モングラ\footnote{Salpicon
  à la
  Monglas(サルピコンアラモングラ)とも呼ばれものとほぼ同じ。サルピコンはせいぜい5
  mm角くらいの小さなさいの目に切ったもののこと。他の用途としては、ブシェ(パイ生地で作ったケースに詰め物をしたもの。本書ではオードブルに分類されている)やタルトレット(小さなタルト)の\ul{アパレイユ}にする。カレーム『19世紀フランス料理』にはプロフィットロール(小さな丸いパンの中身を刳り貫いたもの)にフォワグラと赤く漬けた舌肉とマッシュルームのサルピコンを詰めた「プロフィットロールのポタージュ モングラ」が掲載されている(t.1,
  p.180)。また1806年刊のヴィアール『帝国料理の本』にはローストしたペルドローの胸肉とマッシュルーム、トリュフのサルピコンをソース・エスパニョルなどであえた「ペルドロー モングラ」が掲載されている
  (pp.265-266)。それ以前の主な料理書にこの料理名は見当たらないが、17
  世紀のモングラ侯爵 François Clermont Marquis de Montglas
  (生年不詳〜1675)の名を冠したものらしい。}}{アパレイユ・モングラ}}\label{appareil-montglas}}

\frsub{Appareils à la Montglas}

\index{appareil@appareil!montglas@--- à l Montglas}
\index{montglas@Montglas!appareil@appareil à la ---}
\index{あはれいゆ@アパレイユ!もんくら@---・モングラ}
\index{もんくら@モングラ!あはれいゆ@アパレイユ・---}

(\protect\hyperlink{cotelettes-monglas}{羊のコトレット モングラ}用など)

\protect\hyperlink{saumure-liquide-pour-langues}{赤く漬けた舌肉}150
g、フォワグラ150 g、茹でたマッシュルーム100 g、トリュフ100
gを通常より太めで短かい千切り\footnote{julienne (ジュリエーヌ)。}にする。

これらを、マデイラ酒風味の充分に煮詰めた\protect\hyperlink{sauce-demi-glace}{ソース・ドゥミグラス}2
\(\frac{1}{2}\)
dLであえ、バターを塗った平皿に広げて使うまでそのまま冷ます。

\atoaki{}

\hypertarget{appareil-provencal}{%
\subsubsection[プロヴァンス風アパレイユ]{\texorpdfstring{プロヴァンス風アパレイユ\footnote{\protect\hyperlink{appareil-maintenon}{アパレイユ・マントノン}からこれまでの3種のアパレイユはいずれも、羊のコトレット(ラムチョップ)の片面だけを焼いて、その表面をよく拭い、まだ焼いていない面を下にして、焼いた側の面にこれらのアパレイユを塗る、あるいは盛り上げてからオーブンに入れるという同工異曲とも言うべき仕立てに用いられる。ここで、アパレイン・マントノンとこのプロヴァンス風アパレイユの「用途」の部分の原文には動詞farcirあるいはその過去分詞farci(es)が用いられているのはとても興味深いと言えよう。farcirを日本語の「詰め物をする」と等価と考えてはうまく理解できないケースのひとつで、日本語としてはこの場合「盛る」のほうがむしろ適切だろう。}}{プロヴァンス風アパレイユ}}\label{appareil-provencal}}

\frsub{Appareils à la Provençale}

\index{appareil@appareil!provencale@--- à la Provençale}
\index{provençal@provençale(e)!appareil@appareil à la ---e}
\index{あはれいゆ@アパレイユ!ふろうあんすふう@プロヴァンス風---}
\index{ふろうあんすふう@プロヴァンス風!あはれいゆ@---アパレイユ}

(\protect\hyperlink{cotelettes-provencale}{羊のコトレット プロヴァンス風}用)

\protect\hyperlink{sauce-soubise}{ソース・スビーズ}5
dLを充分に固くなるまで煮詰める。潰したにんにく1片を加え、卵黄3個を加えてとろみを付ける。

\atoaki{}

\hypertarget{bordures-en-farce}{%
\subsubsection{ファルスで作る縁飾り}\label{bordures-en-farce}}

\frsub{Bordures en farce}

\index{bordure@bordure!farce@--- en farce}
\index{farce@farce!bordures@bordures en ---}
\index{ふあるす@ファルス!ふちかざり@---で作る縁飾り}
\index{ほるてゆーる@ボルデュール ⇒ 縁飾り!ふぁるす@ファルスで作る縁飾り}
\index{ふちかさり@縁飾り!ふあるす@ファルスで作る---}

この縁飾りは、飾り付ける料理の素材とおなじ材料を中心にしたファルス\footnote{本文に指定はないが、原則としては\protect\hyperlink{farce-a}{ファルス
  A}か、\protect\hyperlink{farce-de-veau-pour-bordures}{盛り付けの縁飾りおよび底に敷いたり、詰め物をしたクネルに用いる仔牛のファルス}を用いることになるだろう。}を使う。縁飾り用の型\footnote{moule
  à
  bordure(ムーラボルデュール)、ボルデュール型ともいう。大きなリング型で、表面に山形の刻み目(浮き彫り模様)の入ったタイプ(moule
  historié ムールイストリエ、またはmoule cannelé
  ムールカヌレ)と、特に模様の入っていないプレーンなもの(moule uni
  ムールユニ) の2種に大別される。}はプレーンなものでも浮き彫り模様のあるものでもいいが、たっぷりとバターを塗ってからファルスを詰めて低めの温度で火を通す\footnote{原文pocher(ポシェ)。ここまでにも何度も出てきた表現だが、茹でる場合は「沸騰しない程度の温度で加熱すること」であり、このように型に詰めた場合には湯をはった天板に型をのせてやや低温のオーブンに入れてゆっくり加熱することになる。}。

プレーンな型を使う場合、きれいに切ったトリュフのスライスやポシェした
\footnote{原文 oeuf poché
  をそのまま訳したが、表面に飾りとして用いるのは固茹で卵の白身をスライスして型抜きあるいはナイフできれいに切ったものを使うことが多い。}卵の白身、\protect\hyperlink{saumure-liquide-pour-langues}{赤く漬けた舌肉}、ピスタチオなどで表面を装飾するといい。

浮き彫り模様の型を使うなら、上記のような装飾は省いていい。

このようなファルスで作った縁飾りを使うのはとりわけ、鶏肉料理、魚料理、牛や羊肉のソテーなど。

\atoaki{}

\hypertarget{bordures-en-legumes}{%
\subsubsection{野菜で作る縁飾り}\label{bordures-en-legumes}}

\frsub{Bordures en Légumes}

\index{bordure@bordure!legumes@--- en légumes}
\index{legume@légume!bordures@bordures en ---s}
\index{やさい@野菜!ふちかざり@---で作る縁飾り}
\index{ほるてゆーる@ボルデュール ⇒ 縁飾り!やさい@野菜で作る縁飾り}
\index{ふちかさり@縁飾り!やさい@野菜で作る---}

プレーンなボルデュール型の内側にたっぷりとバターを塗り、下拵えしたさまざまな野菜を型の底面と側面にシャルトルーズ\footnote{chartreuse
  本文にあるように、野菜を装飾に用いた仕立てのひとつ。シャルトル会修道院で作られている同名のスピリッツがあるが、料理におけるシャルトルーズ仕立てもシャルトル会修道院に由来しているという。シャルトル会は大斉、小斉の決まりに厳格で、野菜を多く食べる修道生活を送っていたことで有名。そのことにちなんだ仕立ての名称と言われている。この仕立ての文献上の初出は1914年刊ボヴィリエ『調理技術』第2巻の「りんごのシャルトルーズ仕立て」と思われる。これは今でいうデザートに位置するもので、りんごをサフランやアンゼリカとともに煮て黄色や緑に染め、もとの白い果肉、皮の赤など、それら色合いを組み合わせて美しく型の底面を側面に貼り付け、内部をりんごのマーマレード(≒ジャム)で満たす、というもの(t.2,
  pp.149-150)。このボヴィリエのシャルトルーズは「原型」というよりはむしろ「バリエーション」的なものであることが、レシピ本文の文面から伺える。そのため、いつごろ成立した仕立てなのかは不明だが、いずれにしてもシャルトルーズはカレームが「アントレの女王」と呼んだ程に手の込んだ華やかな仕立てとして19世紀前半には定着していた。基本的には、円筒形の型に拍子木に切ってそれぞれ下茹でしたにんじん、さやいんげん、かぶ、などの野菜をびっしりと貼りつけて崩れないようにファルスで塗り固める。その内側に、「ペルドリのシャルトルーズ」の場合は、下茹でしたサヴォイキャベツとペルドリ(ペルドロー
  ≒山うずら、の成鳥)をブレゼしたものを詰め、型の上面(提供するときは底面になる)に蓋をするようにファルスを塗ってから、湯煎にかけてファルスに火を通して固める。裏返して型から外して供する、というもの。野菜の配置、配色が重要で技術のいる仕立て(ヘリンボーンのようなパターンが比較的多かったようだ)。なお、「ペルドローのシャルトルーズ」と「ペルドリとサヴォイキャベツのブレゼ」を混同しているケースが日本でよく見られるが、シャルトルーズとはあくまでも数種類の野菜とファルスで表面を装飾する仕立てを意味しているので注意。}状に貼り付けるように敷き詰める。型の中にやや固めに作った\protect\hyperlink{farce-de-veau-pour-bordures}{じゃがいもを「つなぎ」にした仔牛のファルス}をいっぱいに詰める(\protect\hyperlink{farce-de-veau-pour-bordures}{「縁飾り用仔牛のファルス」}参照)。低めのオーブンで湯煎焼きして火を通す。

この縁飾りはもっぱら、牛、羊肉の料理で野菜のガルニチュールをともなうものに使う。

\atoaki{}

\hypertarget{bordures-en-pate-blanche}{%
\subsubsection[白い生地で作る縁飾り]{\texorpdfstring{白い生地で作る縁飾り\footnote{おなじ縁飾り(ボルデュール)でも、美味しく出来得るものと、食べもので出来てはいるけれども実際には食べないことを前提とした装飾では、本書において明らかに扱いが異なる。この「白い生地で作る縁飾り」および次項「ヌイユ生地で作る縁飾り」は後者にあたるため、さして重きを置いた説明になっていない。}}{白い生地で作る縁飾り}}\label{bordures-en-pate-blanche}}

\frsub{Bordures en pâte blanche}

\index{bordure@bordure!pate blanche@--- en pâte blanche}
\index{pate blanche@pâte blanche!bordures@bordures en ---s}
\index{きし@生地!しろいふちかざり@白い生地で作る縁飾り}
\index{ほるてゆーる@ボルデュール ⇒ 縁飾り!しろいきし@白い生地で作る縁飾り}
\index{ふちかさり@縁飾り!しろいきし@白い生地で作る野菜---}
\index{はーと@パート ⇒ 生地!しろいふちかざり@白い生地で作る縁飾り}

片手鍋に水1 dLと塩5 g、ラード\footnote{saindoux
  (サンドゥー)精製した豚の脂}30
gを入れ、火にかけて沸騰させる。ふるった小麦粉100
gを加えて、余分な水分をとばし、大理石板の上に広げる。

捏ねながらでんぷん\footnote{原文では fécule
  (フェキュール)すなわち「でんぷん」としか指示がないが、fécule de maïs
  (フェキュールドマイス)コンスターチがいいだろう。}を練り込んでいく。10回生地を折ってから、生地を休ませる。

生地を厚さ7
mm程度にのす。これを専用の抜き型で抜いて飾りのパーツをつくる。エチューヴ\footnote{野菜などを乾燥させるためなどの目的で使用する低温で用いるオーブンの一種。}に入れて乾燥させる。これを卵白に小麦粉を加えた糊\footnote{repère
  (ルペール)ここでは小麦粉を卵白に加えて混ぜた糊のこと。通常は銀などの金属製の皿に装飾を貼り付ける際に用いる。この場合は事前に皿を熱しておき、手早く装飾のパーツを貼る。現代ではほとんど行なわれていない手法。小麦粉と水で作り鍋の蓋に目張りをするための生地も同じ用語だが、いずれのケースについても「ルペール」という用語は現代の日本の調理現場であまり多用されていない。}で皿の縁に貼り付ける。

\atoaki{}

\hypertarget{bordures-en-pate-a-nouille}{%
\subsubsection{ヌイユ生地で作る縁飾り}\label{bordures-en-pate-a-nouille}}

\frsub{Bordures en pâte à nouille}

\index{bordure@bordure!pate nouille@--- en pâte à nouille}
\index{nouille@nouille!bordures@bordures en pâte à ---}
\index{きし@生地!ぬいゆふちかざり@ヌイユ生地で作る縁飾り}
\index{ほるてゆーる@ボルデュール ⇒ 縁飾り!ぬいゆきし@ヌイユ生地で作る縁飾り}
\index{ふちかさり@縁飾り!ぬいゆきし@ヌイユ生地で作る野菜---}
\index{はーと@パート ⇒ 生地!ぬいゆふちかざり@ヌイユ生地で作る縁飾り}

ごく固めに捏ねた\protect\hyperlink{nouilles}{ヌイユ生地}を用いて作る縁飾り。上記のように抜き型で抜いてもいいし、あるいは厚さ6〜7
mmで高さ4〜5
cmの帯状に切ってもいい。後者は「エヴィドワール」と呼ばれる専用の小さな抜き型を用いて模様をつけた帯状の生地を皿の縁にしっかりと貼り付ける。

どちらの方法でも、ヌイユ生地を用いた縁飾りには溶いた卵黄を塗ってから、乾燥させる。

\atoaki{}

\hypertarget{croutons}{%
\subsubsection{クルトン}\label{croutons}}

\frsub{Croûtons}

\index{croutons@croûtons} \index{くるとん@クルトン}

クルトンはいわゆる食パン\footnote{フランス語でpain(パン)とだけ言う場合はバゲットに代表されるリーンなパンを指すのが普通で、イギリス式およびアメリカ式の「食パン」は
  pain de mie(パンドミー)と呼ばれて区別される。}で作る。形状や大きさは、どんな料理に合わせるかで決まってくる。これを澄ましバター\footnote{バターには少なからずカゼインなどの不純物が含まれており、それらが焦げや色むらの原因となるので、充分よく澄んだバターを使うこと。}で揚げるが、揚げるのは必ず提供直前にすること。

\atoaki{}

\hypertarget{duxelles-seche}{%
\subsubsection[デュクセル・セッシュ]{\texorpdfstring{デュクセル\footnote{俗説では17世紀にユクセル侯爵
  Marquis d'Uxelles(マルキ
  デュクセル)に料理長として仕えていたラ・ヴァレーヌが創案し、主人の名を付けたとされている。d'
  は de +
  母音の短縮形(フランス語文法ではエリジオンという)。貴族の場合は領地名の前に
  de (≒ of, from) を付ける慣習があり、爵位 de
  領地名、というのが正式な呼び名として用いられていた。Uxellesは母音で始まるからd'Uxellesとなり、それが料理用語としてひとつの単語となりduxellesとして定着したという。しかし、
  duxelles(デュクセル)あるいはそれに類似する名称が用いられるようになったのは19世紀以降であり、文献によって綴りも安定していない。19世紀末のファーヴル『料理および食品衛生事典』では
  duxel
  という綴りで項目が立てられている(なお、ファーヴルはデュクセルをアパレイユの一種と明確に定義している)。さらに時代を遡っていくと、オドの1858年版ではDurcelle(デュルセル)またはDuxelleという名称で呼ばれていると記述がある(p.167)。1856年刊グフェ『料理の本』ではd'Uxelles
  (p.72)。 1833年刊カレーム『19世紀フランス料理』第3巻には、sauce à la
  Duxelle「ソース・デュクセル」が掲載されている。これはあくまでも「ソース」ではあるが、ベースとしてマッシュルームのみじん切りを使っている点は他と同様。さらにヴィアール『王国料理の本』1820年版(p.74)
  および1814年刊ボヴィリエ『調理技術』(p.73)には、のちのデュクセル・セッシュとほぼ同様のものがDurcelleの綴りで掲載されている。カレームは\protect\hyperlink{mayonnaise}{マヨネーズ}の訳注でも見たとおり、料理名の綴りに独自のこだわりを持つ傾向が強かったので、あるいはカレームがdurcelleからduxelleへの転換点として存在している可能性はある。Durcelleの語としての成り立ちは不明だが、人名(名字)に時折見られる綴りのため、何かの由来があったことまでは推察される。以上を考慮すると、ユクセル侯爵の名を冠したという説がどんなに早くとも19世紀中葉以降のものだとわかる。フランス語の/R/と/k/の音がやや似て聞こえることがあるために、はじめdurcelleと呼ばれていたアパレイユがduxelleとなり、ひいては歴史上の人物Marquis
  d'Uxellesユクセル侯爵に結びつけられるようになった、と考えられよう。とはいえ、17世紀はいわゆるマッシュルームの人工栽培が実用化され、食材として流行した時代でもあっため、ラ・ヴァレーヌとユクセル侯爵をこのアパレイユに関連付けたもまったくの見当違いでとは言えまい。}・セッシュ\footnote{sec
  / sèche (セック/セッシュ)乾燥した、水気のない、の意。}}{デュクセル・セッシュ}}\label{duxelles-seche}}

\frsub{Duxelles sèche}

\index{champignon@champignon!duxelles seche@duxelles sèche}
\index{duxelles@duxelles!seche@--- sèche}
\index{てゆくせる@デュクセル!せつしゆ@---・セッシュ}
\index{まつしゆるーむ@マッシュルーム!てゆくせる@デュクセル!せつしゆ@デュクセル・セッシュ}

デュクセルはベースとして必ず、みじん切りにした茸を用いるが、食用のものならどんな茸でも構わない。

バター30 gと植物油30
gを鍋に熱し、玉ねぎのみじん切りとエシャロットのみじん切りを各大さじ1杯ずつ入れて、軽く炒める。マッシュルームの切りくずと軸を細かくみじん切りにしたもの250
gを加え、よく圧して水気を出させる。水分が完全に蒸発するまで強火で炒め続ける。塩こしょうで調味し、パセリのみじん切り1つまみを加えて仕上げる。陶製の器に移し入れ、バターを塗った紙で蓋をする。

デュクセル・セッシュは多くの料理で使われる。

\atoaki{}

\hypertarget{duxelles-pour-legumes-farcis}{%
\subsubsection[野菜のファルシ用デュクセル]{\texorpdfstring{野菜のファルシ\footnote{farci
  (ファルシ)詰め物をした、の意。}用デュクセル}{野菜のファルシ用デュクセル}}\label{duxelles-pour-legumes-farcis}}

\frsub{Duxelles pour légumes farcis}

\index{champignon@champignon!duxelles legumes farcis@duxelles pour légumes farcis}
\index{duxelles@duxelles!legumes farcis@--- pour légumes farcis}
\index{てゆくせる@デュクセル!やさいのふあるしよう@野菜のファルシ用---}
\index{まつしゆるーむ@マッシュルーム!てゆくせる@デュクセル!やさいのふあるしよう@野菜のファルシ用デュクセル}

(トマト、茸などの詰め物用)

\protect\hyperlink{duxelles-seche}{デュクセル・セッシュ}100
g、すなわち大さじ\footnote{本書における「大さじ1杯」の表現は非常にあいまいで、ざっくりとした分量表示であることに注意。}4杯を用意する。白ワイン
\(\frac{1}{2}\)
dLを加えてほぼ完全に煮詰める。次に、トマトを\protect\hyperlink{sauce-demi-glace}{ソース・ドゥミグラス}1
dLと小さめのにんにく1片をつぶしたもの、パンの身25 gを加える。

ごく弱火にかけて煮込み、詰め物をするのにちょうどいい固さになるまで煮詰める。

\atoaki{}

\hypertarget{duxelles-pour-farnitures-diverses}{%
\subsubsection{ガルニチュール用デュクセル}\label{duxelles-pour-farnitures-diverses}}

\frsub{Duxelles pour garnitures diverses}

\index{champignon@champignon!duxelles pour garnitures diverses@duxelles pour garnitures diverses}
\index{duxelles@duxelles!garnitures diverses@--- pour garnitures diverses}
\index{てゆくせる@デュクセル!かるにちゆーるよう@ガルニチュール用---}
\index{まつしゆるーむ@マッシュルーム!てゆくせる@デュクセル!かるにちゆーるよう@ガルニチュール用デュクセル}

(タルトレット、玉ねぎ\footnote{玉ねぎには、完熟、乾燥させた際に表皮が黄色いタイプと白いもの、赤紫色の3系統がある。黄色系統の玉ねぎはフォンなどに用いられることが多い(日本ではこのタイプがほとんど。また「泉州黄」という品種はフランスの野菜栽培の専門書でも言及がある程に栽培特性とクオリティが高く評価されて、フランスでも栽培されている)。白玉ねぎ(oignon
  blanche
  オニョンブロンシュ)は生食やその他の調理、とりわけ小さいものは下茹でしてからバターで色よく炒めて(グラセ)ガルニチュールに用いられる。火が通りやすく、甘いものが多い。赤紫のものは品種によって特性が違うが、加熱調理、生食いずれにも用いられる。}、きゅうり\footnote{20世紀末頃から日本の種苗メーカーが育種した品種も栽培されるようになってきているため、あえて「きゅうり」と訳したが、伝統的な
  concombre (コンコンブル)は太さ4〜5 cm、長さ30〜45
  cm程度まで大きくするのが一般的で、日本の現代品種と異なり表皮は固く、苦味やアクは少ない。種の部分をスプーンなどで取り除いて、そこに詰め物して加熱調理する。また、生のまま輪切りにして食べることも多い。}、などの詰め物用)

\protect\hyperlink{duxelles-seche}{デュクセル・セッシュ}100
gに、\protect\hyperlink{farce-c}{ファルス・ムスリーヌ}または\protect\hyperlink{farce-a}{パナードを用いたファルス}もしくは\protect\hyperlink{farce-gratin-a}{ファルス・グラタン}60
gのいずれかを料理に合わせて加える。

このデュクセルを野菜の詰め物として用いた場合は、表面を焦がさないように\footnote{表面に焦げ目を付けることを
  gratiner (グラティネ)という。}、低温のオーブンに入れて加熱すること\footnote{pocher
  (ポシェ)。本来は沸騰しない程度の温度で茹でることを指すが、この場合は比較的低温のオーブンで加熱調理するという意味。}。

\atoaki{}

\hypertarget{duxelles-bonne-femme}{%
\subsubsection[デュクセル・ボヌファム]{\texorpdfstring{デュクセル・ボヌファム\footnote{bonne
  femme(ボヌファム)は「おばさん」くらいの意。家庭風、田舎風の素朴さを感じさせる料理に付けられる名称。}}{デュクセル・ボヌファム}}\label{duxelles-bonne-femme}}

\frsub{Duxelles à la bonne femme}

\index{champignon@champignon!duxelles bonne femme@duxelles à la bonne femme}
\index{duxelles@duxelles!bonne femme@--- à la bonne femme}
\index{bonne femme@bonne femme!duxelles@duxelles à la ---}
\index{てゆくせる@デュクセル!ほぬふあむ@---・ボヌファム}
\index{ほぬふあむ@ボヌファム!てゆくせる@デュクセル・---}
\index{まつしゆるーむ@マッシュルーム!てゆくせる@デュクセル!ほぬふあむ@デュクセル・ボヌファム}

(家庭料理用)

生のデュクセルに、しっかり味付けをした\protect\hyperlink{chair-a-saucisse}{ソーセージ用挽肉}を同量加えるだけ。

\atoaki{}

\hypertarget{essence-de-tomate}{%
\subsubsection{トマトエッセンス}\label{essence-de-tomate}}

\frsub{Essence de tomate}

\index{tomate@tomate!essence@essence de ---}
\index{essence@essence!tomate@--- de tomate}
\index{とまと@トマト!えつせんす@---エッセンス}
\index{えつせんす@エッセンス!トマト---}

よく熟したトマトのジュースを漉し器で漉す。これを片手鍋に入れて、弱火にかけてゆっくりと、シロップ状になるまで煮詰める。

布で漉るが、圧したり絞らないこと。保存しておく。

\hypertarget{nota-essence-de-tomate}{%
\subparagraph{【原注】}\label{nota-essence-de-tomate}}

このトマトエッセンスはブラウン系の派生ソースの仕上げに色合いを調節するのにとても便利だ\footnote{ブラウン系の派生ソースの節で、明示的にこのトマトエッセンスの使用に言及しているレシピは2つのみだが、必ずしもそのことにこだわず、適宜、必要に応じて使うのがいい。}。

\atoaki{}

\hypertarget{fonds-de-plats}{%
\subsubsection{料理をのせる台、トンポン、クルスタード}\label{fonds-de-plats}}

\frsub{Fonds de plats, Tampons et Croustades}

\index{fonds@fonds!plats@--- de plats} \index{tampon@tampon}
\index{croustade@croustade}
\index{たい@台!さらにしいてりようりをのせる@皿に敷いて料理をのせる---}
\index{とんぽん@トンポン} \index{くるすたーと@クルスタード}

皿に敷いて料理をのせる台、トンポン、クルスタードの重要性は日々ますます失なわれつつある。新しいサーヴィスの方式ではこれらをほぼ完全に用いなてはいない。これらの装飾的な台はパンや、一番多いケースは米を材料に作られる\footnote{実際、本書においてこれらを用いる指示は非常に少ないが、まったくないわけでもない。ただ、エスコフィエが乗り越えたいと願ったデュボワとベルナールの『古典料理』がこれらの装飾的な台の作り方にかなりのページを割いていることと比較すると、驚くほどに素気なく短かい説明で終わっている。}。

パンを使った台は、固くなったパンの身を切って作る。これをバターで揚げ
{[}\^{}31{]}、小麦粉を卵白に加えて作った糊\footnote{説明的に訳したが、原文は
  repèreの1語。\protect\hyperlink{bordures-en-pate-blanche}{白い生地で作る縁飾り}および訳注参照。}で皿の底に貼り付ける。

\hypertarget{ux7c73ux3067ux4f5cux308bux30c8ux30f3ux30ddux30f3ux3068ux30afux30ebux30b9ux30bfux30fcux30c9}{%
\subparagraph{米で作るトンポンとクルスタード}\label{ux7c73ux3067ux4f5cux308bux30c8ux30f3ux30ddux30f3ux3068ux30afux30ebux30b9ux30bfux30fcux30c9}}

\ldots{}\ldots{}パトナ米2 kgを、水が完全に澄むまでよく洗う。

たっぷりの水に入れて火にかけ、5分間茹でる。鍋の湯を捨て、別の湯に漬けて米を洗う。再度湯をきる。大きな片手鍋に丈夫で清潔な布または豚背脂のシートを敷き、入れてみょうばん10
\nolinebreak[4]gを加え、布または豚背脂のシートを折り畳んで米を包む。鍋に蓋をして、弱火のオーブンかエチューヴ\footnote{étuve
  主として野菜の乾燥などを目的とした低温専用のオーブン。}に入れ、3時間加熱する。

その後、米を力をこめてすり潰す。ラードを塗った布のナフキンで包んで揉み、ラードを塗った器に手早く詰めて、冷ます。

充分に冷めたら、米の塊を彫って装飾する。みょうばんを加えた水に漬けて、こまめに水を替えてやれば長期保存も可能だ。

\atoaki{}

\hypertarget{portugaise}{%
\subsubsection[ポルチュゲーズ/トマトのフォンデュ]{\texorpdfstring{ポルチュゲーズ\footnote{portugais(e)
  (ポルチュゲ/ポルチュゲーズ)は形容詞の場合は「ポルトガルの」の意。名詞の場合はポルトガル人。ここでは大文字で書き出していることから名詞と考えられる(なお現代フランス語の正書法では文頭以外の語は固有名詞のみ大文字で始めることになっており、普通名詞を文中で大文字にすることはないが、料理名などの場合は比較的自由に大文字を使う傾向にある)。すなわち「ポルトガルの女」くらいの意味にとることが可能。ちなみに、このレシピとはまったく関係ないが、、\emph{Lettres
  Portugaises}
  (レットルポルチュゲーズ)『ぽるとがる\ruby{文}{ぶみ}』という題名の本が17世紀にフランスで出版され人々の感動を誘った。リルケや佐藤春夫が自国語に翻訳、翻案したものも有名。実在したポルトガルの修道女マリアナ・アルコフォラドがフランス軍人に宛てた恋文をまとめた、事実にもとづく書簡集と考えられていたが、20世紀になってから、ガブリエル・ド・ギユラーグという男性文筆家によるまったくの創作であることが証明された。いわゆる「書簡体小説」である。とはいえ作品の文学的価値はまったく減じることのない名作。書簡体小説という形式は18世紀に流行し、ゲーテ『若きウェルテルの悩み』やラクロ『危険な関係』、ルソー『新エロイーズ』などの名作がある。19世紀前半にはその流行も落ち着き、バルザック『二人の若妻の手記』などはこの小説形式の流行の最後を飾る名作のひとつとして名高い。なお、トマトは16世紀に既にフランスにもたらされており、16世紀末に出版されたオリヴィエ・ド・セール『農業経営論』では「美しいが食べても美味しくない」と記されている。食材として広く普及したのは19世紀以降であり、爆発的な流行現象とさえいえるほどだった。第二帝政期を代表する小説家のひとりフロベールの遺作『ブヴァールとペキュシェ』にも農業に挑戦した2人の主人公がトマトの芽掻きをする必要があることを知らなかったために失敗したエピソードが描かれている。「オマール・アメリケーヌ」や「舌びらめ デュグレレ」などトマトが重要な役割を果している料理が多く創案され、フランス料理の歴史において19世紀という時代を象徴する食材のひとつともいえる。}/トマトのフォンデュ}{ポルチュゲーズ/トマトのフォンデュ}}\label{portugaise}}

\frsub{Fondue de tomate ou Portugaise}

\index{tomate@tomate!fondue@fondue de --- ou Portugaise}
\index{portugais@portugais(e)!fondue de tomate@foudue de tomate ou Portugaise}
\index{とまと@トマト!ふおんてゆ@---のフォンデュ/ポルチュゲーズ}
\index{ふおんてゆ@フォンデュ!とまと@トマトの---/ポルチュゲーズ}
\index{ほるちゆけーす@ポルチュゲーズ/トマトのフォンデュ}
\index{ほるとかるふう@ポルトガル風!ほるちゆけーす@ポルチュゲーズ/トマトのフォンデュ}

玉ねぎ大1個をみじん切りにしてバターまたは植物油で炒める。トマト500
gは皮を剥いて潰し、粗みじん切りにして鍋に加える。潰したにんにく1片と塩、こしょうを加える。弱火にかけて水分がすっかりなくなるまで煮詰める\footnote{トマトは品種にもよるが、混ぜずに弱火で加熱すると固形物が沈殿し、水分が上澄みになる。ここでは濃縮トマトペーストになるほどは煮詰めず、その上澄みがなくなるまで、という解釈でいいだろう。}。

時季、つまりトマトの熟し具合に応じて必要なら粉砂糖をほんの1つまみ加えるといい。

\atoaki{}

\hypertarget{kache-de-sarrazin-pour-potage}{%
\subsubsection[ポタージュ用そば粉のカーシャ]{\texorpdfstring{ポタージュ用そば粉のカーシャ\footnote{フランス語では
  kache (カシュ)、Kacha
  (カシャ)とも。日本語ではカーシャと呼ばれるほうが多いようだ。もとはロシアなどスラブ諸国における粥の総称でロシア語では
  \ltjsetparameter{jacharrange={-2}}каша\ltjsetparameter{jacharrange={+2}}
  。フランス料理に取り入れられ、そば粉やセモリナ粉でつくったクレープのようなものを意味するようになった。このカーシャはポタージュのガルニチュール、つまり「浮き実」となる。}}{ポタージュ用そば粉のカーシャ}}\label{kache-de-sarrazin-pour-potage}}

\frsub{Kache de Sarrazin pour Potages}

\index{kache@kache!sarrazin@--- de Sarrazin pour Potages}
\index{sarrazin@sarrazin!kache@Kache de --- pour Potages}
\index{かーしや@カーシャ!そはこ@ポタージュ用そば粉の---}
\index{そはこ@そば粉!かーしや@ポタージュ用---のカーシャ}

(仕上がり約10人分\footnote{原文 pour un service (プランセルヴィス)
  フランス宮廷料理の時代から、ロシア式サービスの普及しはじめた頃まで、格式ある宴席での料理を作る際の単位としてserviceが用いられた。1
  service
  は概ね10人分。現実には8〜12人くらいの間で融通を効かせて運用されていたようだ。本書のレシピの分量は多くが1
  serviceすなわち約10人分で書かれている。})

粗挽きのそば粉1 kgに塩を加えたゆるま湯を7〜8
dL加えてデトランプ\footnote{ここでは動詞 détremper
  (デトロンペ)が使われているが、faire un détrempe
  (フェランデトロンプ)と同義で粉が吸水して捏ねる前の状態(塊)のこと。}を作ってまとめる。これを深手の片手鍋\footnote{casserole
  russe
  (カスロールリュス)直訳すると「ロシアの片手鍋」だが、通常は深い片手鍋をそう呼ぶ。}に入れて押し潰す。高温のオーブンに入れて約2時間加熱する。

オーブンから出したら、表面の固くなった皮の部分は取り除く。鍋の中のパン状になったものを、鍋の周囲にこびりついた焦げの部分に触れないようにして取り出す。

これにバター100 gを加えて捏ねる。厚さ1
cmになるように重しをして冷ます。直径26〜27 mm位の\footnote{原文 un
  emporte-pièce rond de la grandeur d'une pièce de 2 francs
  「2フラン硬貨の大きさの円形の抜き型」。フランはヨーロッパ通貨統合前のフランスの通貨単位。2フラン硬貨は概ね26〜27
  mm。}の丸い型抜きで抜く。これを澄ましバターで色よく焼く。オードブル皿か、ナフキンに盛り付けて供する。

\hypertarget{nota-kache-de-sarrazin-pour-potage}{%
\subparagraph{【原注】}\label{nota-kache-de-sarrazin-pour-potage}}

このカーシャをオーブンから出してそのままの状態で供してもいい。その場合は専用の容器に盛りつける。

\atoaki{}

\hypertarget{kache-de-semoule-pour-coulibiac}{%
\subsubsection[クリビヤック用セモリナ粉のカーシャ]{\texorpdfstring{クリビヤック\footnote{サーモンなどをブリオシュ生地で包んで焼いた料理。これを作る際に、厚さ1
  cmくらいに切ったサーモンの身とこのカーシャまたは米を互いに層になるようにして、ブリオシュ生地で包んで焼く。}用セモリナ粉のカーシャ}{クリビヤック用セモリナ粉のカーシャ}}\label{kache-de-semoule-pour-coulibiac}}

\frsub{Kache de Semoule pour le Coulibiac}

\index{kache@kache!semoule@--- de Semoule pour le Coulibiac}
\index{semoule@semoule!kache@Kache de --- pour le Coulibiac}
\index{かーしや@カーシャ!せもりなこ@クリビヤック用セモリナ粉の---}
\index{せもりなこ@セモリナ粉!かーしや@クリビヤック用セモリナ粉---のカーシャ}
\index{くりひやつく@クリビヤック!かーしや@---そば粉のカーシャ}

(仕上がり約10人分)

大粒のセモリナ粉200
gに溶き卵1個をよく混ぜる。天板の上に広げて弱火で乾燥させる。

これを目の粗い漉し器で裏漉しする。コンソメに入れて約20分間、沸騰しない程度の温度で加熱する\footnote{pocher
  (ポシェ)。}。気をつけて水気をきる。

\atoaki{}

\hypertarget{matignon}{%
\subsubsection[マティニョン]{\texorpdfstring{マティニョン\footnote{1935年以来首相官邸として使われているマティニョン館を18世紀に所有していたジャック・ド・マティニョンの料理人が創案したものといわれているが真偽は不明。料理用語としての初出はおそらく1856年刊デュボワ、ベルナール共著『古典料理』。ここでは「マティニョンのフォン」として、「器具でお削りおろした豚背脂、同量のバター、生ハムのスライス数枚、薄切りにしたにんじんと玉ねぎ、マッシュリュームの切りくずを弱火にかけて軽く色付くまで炒め、ローリエの葉、塩、こしょうを加えてからマデイラ酒かソテルヌのワインをひたひたに注ぎ、強火でグラス状になるまで煮詰める。これを串焼きあるいはオーブン焼きにする塊肉を覆うのに使う。魚料理の場合には豚背脂とハムをバターか植物油に代える(p.71)」となっている。これ以前の主要な料理書にmatignonの語はまったく見られないが、
  1858年版のオドにおいて「ミルポワとマティニョンは野菜と豚背脂、ハムを煮てグラス状に煮詰めたガルニチュール。鶏やジビエを串刺しでローストする際にこれで覆ってさらにバターを塗った紙で包む。高級料理でしかほとんど用いられない(p.167)」とされている。}}{マティニョン}}\label{matignon}}

\frsub{Matignon}

\index{matignon@matignon} \index{まていによん@マティニョン}

にんじん125 g、玉ねぎ125 g、セロリ50 g、生ハム100 gを1
cm弱のさいの目に刻む。ローリエの葉1枚とタイム1枝とともに鍋に入れて、バターで弱火にかけ蓋をして蒸し煮し、少量の白ワインでデグラセする。

\atoaki{}

\hypertarget{mirepoix}{%
\subsubsection[ミルポワ]{\texorpdfstring{ミルポワ\footnote{18世紀にガストン・ピエール・レヴィ・ミルポワ公爵(1699〜1757)の料理人が考案したといわれているが真偽は不明。料理書における初出はおそらく1814年刊ボヴィリエ『調理技術』(p.61)だが、非常に厄介な問題を含んでいる。というのも、まずpoêle(ポワル)という名称のソースがあり(これがのちのà
  la poêle \textgreater{} poêlé という調理の歴史につながる)、 ~
  それは、さいの目に切った仔牛腿肉2 kgとハム750
  g、器具を使っておろすかさいの目に刻んだ豚背脂750
  g、さいの目に切ったにんじん5〜6本、玉ねぎ8個は切らずにそのまま、ブーケガルニとしてパセリ、シブール(≒葱)、クローブ、ローリエの葉2枚、タイム、バジル少々と外皮を剥いて種を取り除いたレモンのスライスを、500
  gのバターで弱火で炒め、ブイヨンかコンソメを注ぎ、4〜5時間アクを引きながら煮込み、漉す、というもの。そして、ミルポワとはこのポワルにブイヨンの\(\frac{1}{4}\)
  量をシャンパーニュか上等の白ワインにして作ったもの、となっている。
  1833年のカレーム『19世紀フランス料理』第1巻におけるミルポワも同工異曲であり、さいの目に切った材料をブイヨンで煮込んで布で絞り漉したもの。さて、1856年のデュボワ、ベルナール共著『古典料理』においては「ミルポワのフォン」としてソースのベースとして掲載されている。概要は、器具を用いておろすか細かく刻んだ豚背脂300
  gを鍋に入れて溶かし、玉ねぎ1個とにんじん1本の薄切りを加え、弱火でゆっくり色付かないよう炒める。さらに大きめのさいの目に切ったハム250
  gとブーケガルニ、パセリ、マッシュルームの切りくず、にんにく、クローブを加えて2
  Lのブイヨンと \(\frac{1}{2}\)
  Lの白ワインを注ぐ。強火にかけ沸騰したら端に寄せて弱火にし、沸騰状態を保ったまま\(\frac{2}{3}\)量まで煮詰める。最後に漉し器で漉す、というもの(pp.70-71)。1867年のグフェ『料理の本』においても「ミルポワすなわち肉と野菜のエッセンス」となっている
  (p.406)。つまり、デュボワとベルナールあるいはグフェの頃、つまり19
  世紀後半まで、ミルポワとは「出汁」の一種あるいは液体調味料のようなものだったと考えていい。前項のマティニョンの訳注でも見たように、
  1858年版のオドでやや違った認識がされていることは注目に値しよう。19
  世紀末のファーヴルの『料理および食品衛生事典』ではアパレイユとして定義している。さいの目に刻んだハム、にんじん、玉ねぎを白こしょう、タイム、バジル、ローリエの葉、クローブとともに色付くまで炒め、ソースやブレゼの調理に用いる、とある。おそらくはファーヴルの示したミルポワがもっとも本書のものに近いが、ファーヴルはマティニョンに言及していないため、さいの目に刻む大きさによって呼び名を変えているのは本書が文献上最初のものと思われる。なお、ファーヴルはミルポワ公爵の料理人が創案したという説をとっている。いずれにしてもミルポワという言葉の指す内容、用途が19世紀後半の30年くらいの間に大きく変化したと考えていいだろう。なお、現代日本の調理現場ではミルポワとマティニョンを厳密に区別することなく、また、豚背脂やハムは用いず、にんじんや玉ねぎなどの香味野菜を細かいさいの目に刻んだものをミルポワの用語で統一しているケースも多いようだ。}}{ミルポワ}}\label{mirepoix}}

\frsub{Mirepoix}

\index{mirepoix@mirepoix} \index{みるほわ@ミルポワ}

材料は\protect\hyperlink{matignon}{マティニョン}とまったく同じだが、より小さなさいの目
\footnote{brunoise (ブリュノワーズ)厳密には1〜2
  mmのさいの目に刻んだものを指す。}に刻むことと、ハムではなく塩漬け豚バラ肉の脂身の少ないところをさいの目に切って下茹でしたものを使う場合もある。

バターで色よく炒める\footnote{原文 faire revenir
  (フェールルヴニール)熱した油脂で色付くまで焼く、炒める ≒ rissoler
  (リソレ)。}。

\atoaki{}

\hypertarget{mirepoix-fine}{%
\subsubsection{ボルドー風ミルポワ}\label{mirepoix-fine}}

\frsub{Mirepoix fine, dite à la Bordelaise}

\index{mirepoix@mirepoix!fine@ --- fine, dite à la Bordelaise}
\index{bordelais@bordelais(e)!mirepoix fine@mirepoix fine, dite à la ---e}
\index{みるほわ@ミルポワ!ほるとーふう@ボルドー風ミルポワ}
\index{ほるとーふう@ボルドー風!みるほわ@---ミルポワ}

標準的な大きさに刻んだミルポワを料理に加えると、普通は即座にその料理にふさわしい香り付けが出来るが、ボルドー風ミルポワはとりわけエクルヴィス
\footnote{ecrevisse ヨーロッパザリガニ。}やオマール\footnote{homard
  ロブスター。}の料理の風味付けにいい。これはあらかじめ用意しておくべきもので、次のように作業する。

にんじん125 gと玉ねぎ125 g、パセリ1枝を出来るだけ細かいさいの目に刻む
\footnote{原文 brunoise excessivement fine
  直訳すると「過度なまでに細かいブリュノワーズ(1〜2
  mm角のさいの目)」。}。これにタイム1つまみと粉末にしたローリエの葉1つまみを加える。

材料をバター50
gとともに片手鍋に入れ、完全に火が通るまで蓋をして弱火で蒸し煮する\footnote{étuver
  (エチュヴェ)。}。

小さな陶製の器に広げ、フォークの背を使って器に押し込む。バターを塗った白い円形の紙で蓋をして、使用するまで保存する。

\hypertarget{nota-mirepoix-fine}{%
\subparagraph{【原注】}\label{nota-mirepoix-fine}}

より細かいミルポワを作るには、材料をみじん切りにして、トーション\footnote{\protect\hyperlink{sauce-verte}{ソース・ヴェルト}訳注参照。}
の端で材料を強く圧して野菜の水気を出してしまうだけでいい。こうすると蒸し煮している間にその水分は蒸発しきれないで残る。ただし、こうしてミルポワに残った水分は、長い時間保存する場合にはカビや腐敗の原因になるので注意すること。

\atoaki{}

\hypertarget{orge-perle-pour-volailles-farcies}{%
\subsubsection[丸鶏の詰め物その他に用いる真珠麦]{\texorpdfstring{丸鶏の詰め物その他に用いる真珠麦\footnote{このレシピは第四版のみ。}}{丸鶏の詰め物その他に用いる真珠麦}}\label{orge-perle-pour-volailles-farcies}}

\frsub{Orge perlé pour volailles farcies et autres usages}

\index{orge perle@orge perlé}
\index{しんしゆむき@真珠麦!まるとりのつめもの@丸鶏の詰め物その他に用いる真珠麦}
\index{おおむぎ@大麦!せいはく@精白--- ⇒ 丸鶏の詰め物その他に用いる真珠麦}
\index{まるむぎ@丸麦 ⇒ 真珠麦!まるとりのつめもの@丸鶏の詰め物その他に用いる真珠麦}

玉ねぎのみじん切り75
gをバターでブロンド色になるまで炒める。皮を剥いて洗い、水気をきってさらに布で水気を取り除いた大麦250
gを加える。木のヘラで混ぜながら炒める。沸かした\protect\hyperlink{consomme-blanc}{白いブイヨン}\footnote{原文どおりに訳したが、第四版に
  bouillon blanc
  は掲載されていない。ここでは「白いコンソメ」すなわちコンソメ・サンプルと解釈するのがいいだろう。}\(\frac{3}{4}\)
Lを注ぐ。こしょう1つまみを加えたら蓋をしてごく弱火のオーブンで約2時間加熱する。焦がしバター50
gをかけて仕上げる。

\atoaki{}

\hypertarget{pate-a-chou-d-office}{%
\subsubsection{調理用シュー生地}\label{pate-a-chou-d-office}}

\frsub{Pâte à chou d'office}

\index{pate@pâte!chou@chou!office@pâte à chou d'office}
\index{choupate@chou (pâte)!office@pâte à --- d'office}
\index{しゆうきし@シュー生地!ちようりよう@調理用---}
\index{きし@生地!ちようりようしゆー@調理用シュー生地}

水1 Lとバター200 g、塩 10
gを片手鍋に入れて火にかけ、沸騰したら火から外す。ふるった小麦粉625
gを加える。強火にかけて混ぜながら余計な水分をとばす。次に、卵の大きさによって12〜14個の全卵を生地に加える。

このシュー生地は\protect\hyperlink{pomme-dauphine}{じゃがいものドフィーヌ}やニョッキなどのアパレイユとして使用されるのがほとんどなので、通常のシュー生地よりも固く作らなくてはいけない。

\atoaki{}

\hypertarget{pate-a-frire-pour-beignet-de-cervelles}{%
\subsubsection[脳、白子のベニェやフリトー用の揚げ衣]{\texorpdfstring{脳、白子のベニェやフリトー\footnote{かえるの腿、牡蠣、ムール貝、サーモン、鶏のレバーなどをマリネして揚げ衣を付けて油で揚げた料理。friteau
  (フリトー)とも綴る。とくにfrite(s)(フリット)と混同しないように注意したい。名詞としての
  fritesはフライドポテトのこと。過去分詞(形容詞)としての
  frit(e)(フリ/フリット)は「油で揚げた」の意。例えばcourgette
  frite(クルジェットフリット)は油で揚げたズッキーニのこと(friteは形容詞)だが、steak
  frites(ステックフリット)フライドポテト添えのステーキを意味する(この場合のfritesは名詞)。}用の揚げ衣}{脳、白子のベニェやフリトー用の揚げ衣}}\label{pate-a-frire-pour-beignet-de-cervelles}}

\frsub{Pâte à frire pour Beignets de cervelles et de laitances, fritots, etc.}

\index{pate@pâte!frire@--- à frire!beignet cervelles@pâte à frire pour Beignets de cervelles et de laitances, fritots, etc.}
\index{frire@frire!pate@pâte à ---!beignet cervelles@pâte à frire pour Beignet de cervelles et de laitances, fritots; etc.}
\index{fritot@fritot!pate frire@pâte à frire pour Beignet de cervelles et de laitances, fritots; etc.}
\index{cervelle@cervelle!pate frire@pâte à frire pour Beignet de cervelles et de laitances, fritots; etc.}
\index{laitance@laitance!pate frire@pâte à frire pour Beignet de cervelles et de laitances, fritots; etc.}
\index{あけころも@揚げ衣!のうしらこのへにえやふりとー@脳、白子のベニェやフリトー用の揚げ衣}
\index{ふりとー@フリトー!のうしらこのへにえやふりとようあけころも@脳、白子のベニェやフリトー用の揚げ衣}
\index{のう@脳!のうしらこのへにえやふりとーようあけころも@脳、白子のベニェやフリトー用の揚げ衣}
\index{しらこ@白子!のうしらこのへにえやふりとーようあけころも@脳、白子のベニェやフリトー用の揚げ衣}

陶製の器に、ふるった小麦粉125 g、塩1つまみ、植物油か溶かしバター大さじ
2杯、微温湯2
dLを入れる。木のヘラで生地を持ち上げながら混ぜる。すぐに使う場合は決して生地を捏ねまわさないこと。弾力が出てしまい、揚げる具材を漬けたときに生地が上手く付かなくなってしまうからだ。事前に用意しておく場合には、捏ねまわしても大丈夫。生地を休ませている間に弾力性は失なわれる\footnote{いったん形成されたグルテンはそうそう崩れないので、内容としてはやや疑問に思う部分だが、日本の「てんぷら」の常識をここで適用すべきではない。実際のところ、フリトーの衣はグルテンが形成されていてもまったく問題ないだろう。}。

この生地は、使う直前に、ふんわりと泡立てた卵白2個分を加える。

\atoaki{}

\hypertarget{pate-a-frire-pour-legumes}{%
\subsubsection{野菜用の揚げ衣}\label{pate-a-frire-pour-legumes}}

\frsub{Pâte à frire pour Légumes}

\index{pate@pâte!frire@--- à frire!legumes@pâte à frire pour Légumes}
\index{frire@frire!pate legumes@pâte à frire pour Légumes}
\index{legumes@legumes!pate frire@pâte à frire pour Légumes}
\index{salsifis@salsifis!pate frire@pâte à frire pour Légumes}
\index{celeris@céleris!pate frire@pâte à frire pour Légumes}
\index{crosnes@crosnes!pate frire@pâte à frire pour Légumes}
\index{あけころも@揚げ衣!やさいよう@野菜用の揚げ衣}
\index{さすしふい@サルシフィ!やさいようのあけころも@野菜用の揚げ衣}
\index{せろり@セロリ!やさいようのあけころも@野菜用の揚げ衣}
\index{ちよろき@チョロギ!やさいようのあけころも@野菜用の揚げ衣}

(サルシフィ\footnote{salsifis
  キク科の根菜、見た目は牛蒡に似ているが風味や調理特性はまったく異なる。}、セロリ、クローヌ\footnote{crosne
  ちょろぎ。シソ科の根菜(正確には塊茎が食用となる)。中国原産で日本には江戸時代に伝わった。同様に中国からヨーロッパにも伝わり、フランスでは最初に栽培された地名からcrosne、あるいは日本由来のものとしてcrosne
  du
  Japon(クローヌデュジャポン)と呼ばれる。絵画などの分野でジャポニスムが流行したこともあって、日本の食材として注目を浴びたためか、(à
  la) japonaise (アラ
  ジャポネーズ)「日本風」を冠したものには、このちょろぎを用いた料理が多い。}など)

陶製の器に小麦粉125
gと塩1つまみ、溶かしバター大さじ2杯、全卵1個、水適量を混ぜて薄めの衣をつくる。

出来るだけ、1時間前に用意しておくこと。

\atoaki{}

\hypertarget{riz-pour-farcir-les-volailles-servies-en-releve-ou-en-entree}{%
\subsubsection{大皿仕立ての丸鶏に詰める米}\label{riz-pour-farcir-les-volailles-servies-en-releve-ou-en-entree}}

\frsub{Riz pour farcir les volailles servies en Relevé ou en Entrée}

\index{riz@riz!farcir@--- pour farcir les volailles servies en Relevé ou en Entrée}
\index{farce@farce!riz@riz pour farcir les volailles servies en Relevé ou en Entrée}
\index{ふあるす@ファルス!おおさらしたてのまるとりにつめるこめ@大皿仕立ての丸鶏に詰める米}
\index{こめ@米!つめもの@大皿仕立ての丸鶏に詰める米}

玉ねぎ \(\frac{1}{2}\)個のみじん切りをバター50
gでさっと炒める。カロライナ米またはパトナ米250
gを加え、米が白くなるまで混ぜながら炒める。

\protect\hyperlink{consomme-blanc}{白いコンソメ} \(\frac{1}{2}\)
Lを注ぎ、蓋をして15分間煮る。生クリーム1 \(\frac{1}{2}\)
dLとフォワグラの脂\footnote{フォワグラのテリーヌなどを作る際に余分な脂が出るのでそれを利用するといい。}またはバター125
g、\protect\hyperlink{sauce-supreme}{ソース・シュプレーム}大さじ数杯と、この米を詰める鶏料理に添えることになっているガルニチュールの一部を加える。

\hypertarget{nota-riz-pour-farcir-les-volailles-servies-en-releve-ou-en-entree}{%
\subparagraph{【原注】}\label{nota-riz-pour-farcir-les-volailles-servies-en-releve-ou-en-entree}}

米は鶏を焼いている間に完全に火が通るよう、詰め物をする段階では\(\frac{3}{4}\)程度に火が通っているようにする。鶏に詰めた米は膨らむので、きっちりとは鶏に詰め込まないこと。

\atoaki{}

\hypertarget{salpicons-divers}{%
\subsubsection[サルピコン]{\texorpdfstring{サルピコン\footnote{この項は第二版で全面的に書き換えられ、分量も大幅に増えた。初版の記述は以下のとおり。「この用語は一般的に火を通した肉、フアルス、マッシュルーム、トリュフなどをさいの目切りにしたもののこと。大きさは合わせる料理に応じて加減する。たった1種類の肉、あるいは野菜のさいの目切りにしたものでもサルピコンと呼ぶ。(例)フォワグラのサルピコン、トリュフのサルピコン、など(p.188)」。}}{サルピコン}}\label{salpicons-divers}}

\frsub{Salpicons divers}

\index{salpicon@salpicon} \index{さるひこん@サルピコン}

サルピコンという用語は普通、ある調理の種類を指すものと理解されよう。

サルピコンにはサンプルとコンポゼ\footnote{simple
  (サンプル)単一の、シンプルな。composé(e) (コンポゼ)組み合わせた。}がある。

素材が1種類だけの場合はサンプルと呼ぶ。例えば鶏やジビエの肉、羊や牛の肉、仔牛胸腺肉\footnote{ris
  de veau (リドヴォ))}、あるいはフォワグラ、魚、甲殻類、ハム、舌肉など。

素材が複数からなる場合はコンポゼと呼ぶ。本書に掲載されている組み合わせのほか、相性のよさそうなものの組み合わせ、マッシュルームやトリュフで嵩を増したもの、などがそうだ。

サルピコンの作り方は、各種の素材を、小さな規則正しいサイズ、すなわち一辺が0.5
cm程度のさいの目に刻む。

各種サルピコンのレシピ集を作るとしたら\footnote{原文は直説法現在という時制で書かれており「事実を述べる」ニュアンスだが、本書にサルピコンのレシピをまとめた章も節もないため、やや仮定法的に訳した。なお『ラルース・ガストロノミック』初版には
  salpiconの項に代表的なレシピがまとめられている。}、上記のような素材の組み合わせから始まり、それによって使い途も名称も決まることになる。例えば\footnote{ここに挙げられている例が第二版での加筆者の「思い付き」かそれとも本書の全体の構想にかかわるものだったかは不明だが、結果として本書第四版にはおろか肝心の第二版にさえ具体的な素材が記されていない例が含まれている。「ロワイヤル」と「シャスール」がそれにあたる。以下、ひとつずつ見ていくと、(1)ロワイヤルroyale
  王宮風、王家風、の意で、これほど料理そのものと関連なく料理名に濫用されている語も珍しいとさえ言えるが、サルピコン・ロワイヤルsalpicon
  à la
  royaleの場合は『ラルース・ガストロノミック』初版によると「トリュフとマッシュルームを鶏のピュレであえたもの」を指す。(2)フィナンシーエル
  salpicon à la
  financière\ldots{}\ldots{}クネル、雄鶏のとさかとロニョン(精巣)、マッシュルーム、トリュフ、すなわち\protect\hyperlink{garniture-a-la-financiere}{ガルニチュール・フィナンシエール}の構成素材をさいの目に切ってを煮詰めたソース・フィナンシエールであえたもの。(3)パリ風
  salpicon à la
  parisienneは\protect\hyperlink{garniture-parisienne}{パリ風ガルニチュール}参照。(4)
  モングラsalpicon à la
  Monglasは本節冒頭の\protect\hyperlink{appareil-montglas}{アパレイユ・モングラ}そのもの。(5)シャスールsalpicon
  chasseur\ldots{}\ldots{}さいの目に切ってバターで炒めた鶏のレバーとマッシュルームを、煮詰めた\protect\hyperlink{sauce-chasseur}{ソース・シャスール}であえたもの。}\ul{ロワイヤル}、\ul{フィ
ナンシエール}、\ul{パリ風}、\ul{モングラ}、\ul{シャスール}など。

\atoaki{}

\hypertarget{ux30d4ux30edux30b7ux30adux7528ux30c8ux30f4ux30a1ux30edux30fcux30b072}{%
\subsubsection[ピロシキ用トヴァローグ]{\texorpdfstring{ピロシキ用トヴァローグ\footnote{このレシピは第三版から。なお第四版=現行版の綴りはtawrogueになっているが、明らかに第三版にあるtwarogueの誤植。ロシア語の綴りは
  \ltjsetparameter{jacharrange={-2}}творог\ltjsetparameter{jacharrange={+2}}。}}{ピロシキ用トヴァローグ}}\label{ux30d4ux30edux30b7ux30adux7528ux30c8ux30f4ux30a1ux30edux30fcux30b072}}

\frsub{Twatogue pour Piroguis}

\index{piroguis@piroguie!twarogue@twarogue pour ---}
\index{twarogue@twaroguel!piroguis@--- pour piroguis}
\index{cuisine russe@cuisine russe!twarogue@twarogue pour piroguis}
\index{とうあろーく@トヴァローグ}
\index{ろしあふう@ロシア風!ひろしきようとうあろーく@ピロシキ用トヴァローグ}
\index{ひろしき@ピロシキ!とうあろーく@---用トゥヴァローグ}

よく水気をきったフロマージュ・ブラン\footnote{ヨーグルトに見た目のよく似た半固形チーズ。デザートなどとして砂糖をかけて食べるなどが一般的。}250
gをナフキンでしっかり絞る。これを陶製の器に入れ、ヘラで滑らかになるまで練る。あらかじめ捏ねてポマード状に柔らかくしておいたバター250
gと全卵1個を加える。

塩、こしょうで調味する。
\end{recette}\newpage
\begin{Main}

\hypertarget{serie-des-appareiles-et-preparations-diverses-pour-garnitures-froides}{%
\section[冷製ガルニチュール用アパレイユなど]{\texorpdfstring{冷製ガルニチュール用アパレイユなど\footnote{この節は、初版で「冷製料理」の章の冒頭に概説としてまとめられていたものを、第二版の改訂時に、ほぼそのままの内容で現在の位置に移動させられている。もちろん順序および内容の加筆も行なわれており、異同は少なくない。}}{冷製ガルニチュール用アパレイユなど}}\label{serie-des-appareiles-et-preparations-diverses-pour-garnitures-froides}}

\frsec{Série des Appareils et Préparations diverses pour Garnitures froides}

\index{garniture@garniture!appareils garnitures froides@appareils et préparations diverses pour ---s froides}
\index{appareil@appareil!garnitures froides@--- et préparations diverses pour garnitures froides}
\index{かるにちゆーる@ガルニチュール!あはれいゆれいせい@冷製---のためのアパレイユなど}
\index{あはれいゆ@アパレイユ!れいせいかるにちゆーる@冷製ガルニチュールのための---など}

\hypertarget{mousses-mousselines-et-souffles-froids}{%
\subsection{冷製のムース、ムスリーヌ、スフレ}\label{mousses-mousselines-et-souffles-froids}}

\frsecb{Mousse, Moussseline, et Soufflé froids}

\index{mousse@mousse!froide@--- froide}
\index{mousseline@mousseline!froide@--- froide}
\index{souffle@soufflé!froid@--- froid}
\index{むーす@ムース!れいせい@冷製の---}
\index{むすりーぬ@ムスリーヌ!れいせい@冷製の---}
\index{すふれ@スフレ!れいせい@冷製の---}

温製の場合でも冷製の場合でも、\ul{ムースとムスリーヌはどちらも同じ材料から作られる}。

ムースとムスリーヌの違いは、温製でも冷製でも、通常は10人分が入る大きな型に詰めて作るのが\ul{ムース}と呼ばれ、いっぽう、\ul{ムスリーヌ}はスプーンで整形したり絞り袋を使ったり、あるいは大きなクネルの形をした専用の型に入れたりして作るが、基本的に\ul{1つ}で1人分と決まっている。スフレは小さなスフレ型に詰める。

\end{Main}

\begin{recette}

\hypertarget{composition-de-l-appareil-pour-mousses-et-mousseline-froides}{%
\subsubsection{冷製のムースとムスリーヌのアパレイユ}\label{composition-de-l-appareil-pour-mousses-et-mousseline-froides}}

\frsub{Composition de l'Appareil pour Mousses et Mousseline froides}

\index{appareil@appareil!mousses mousselines froides@composition de l'--- pour mousses et mousselines froides}
\index{mousse@mousse!composition appareil froides@Composition de l'appareil pour --- et mousseline froides}
\index{mousseline@mousseline!composition appareil froides @Composition de l'appareil pour mousses et --- froides}
\index{あはれいゆ@アパレイユ!れいせいのむーすとむすりーぬ@冷製のムースとムスリーヌの---}
\index{むーす@ムース!れいせいむーすのあぱれいゆ@冷製の---とムスリーヌのアパレイユ}
\index{むすりーぬ@ムスリーヌ!れいせいむすりーぬのあぱれいゆ@冷製のムースと---のアパレイユ}

\begin{itemize}
\tightlist
\item
  材料\ldots{}\ldots{}主素材のピュレ\footnote{本書では加熱した肉や魚、甲殻類のピュレを作る方法への言及はないが、\textbf{本章冒頭にある\protect\hyperlink{farce-mousseline}{ファルス・ムスリーヌ}をそのまま使おうなどと考えてはいけない。ここで説明されている冷製のムース、ムスリーヌ、スフレの作り方に加熱の工程がまったく含まれていないのは、主素材のピュレが既に加熱済みであることを当然の前提としている}からだ。つまりここで材料として示されているピュレは\textbf{すべて加熱済みのものをピュレにしたものだ}と考えなければならない。『料理の手引き』の当時はローストするか茹でるなどの加熱後に、鉢に入れてすり潰し、裏漉ししてから何らかのソース(ここではヴルテ)を加えて漉さ(固さ)を調節するなどしていた。現代ではフードプロセッサーや冷凍粉砕調理機などを利用すればより容易に滑らかなピュレを作ることが可能だろう。また、第3章ポタージュに\protect\hyperlink{les-purees}{ポタージュ・ピュレ}についての概説があるが、そこではポタージュにすることを前提として「つなぎ」の使用が作業のプロセスに組込まれて説明されているために、あくまで参考程度に読むのがいいだろう。}1
  Lすなわち鶏のピュレ、ジビエ、フォワグラや魚、甲殻類のピュレ。溶かした\protect\hyperlink{gelees-ordinaires}{ジュレ}2
  \(\frac{1}{2}\) dL、\protect\hyperlink{veloute}{ヴルテ}4
  dL、生クリーム4 dLはちょうどいい固さに立てて6 dL相当にしておく。
\end{itemize}

素材の特性によって、これらの分量比率は多少変更してもいい。同様に、ある種のムースを作る際にはジュレまたはヴルテのどちらかしか用いなくてもいい。

\begin{itemize}
\tightlist
\item
  作業手順\ldots{}\ldots{}まずベースとなるピュレを入れたボウルを氷の上に置いて、軽く混ぜながら、ジュレとヴルテを加える(どちらかしか使わない場合は使うもののみ)。次に泡立てた生クリームを加える。
\end{itemize}

味付けを確認する。これは冷製料理ではとても重要なことだ。いつも気をつけて確認し、修正を加えるようにすること。

\hypertarget{nota-composition-de-l-appareil-pour-mousses-et-mousseline-froides}{%
\subparagraph{【原注】}\label{nota-composition-de-l-appareil-pour-mousses-et-mousseline-froides}}

生クリームは五分立てすること。完全に立ててしまうと、ムースに滑らかさが失なわれてパサついた仕上りになってしまう。

\atoaki{}

\hypertarget{moulage-des-mousses-froides}{%
\subsubsection{冷製ムースの型詰め}\label{moulage-des-mousses-froides}}

\frsub{Moulage des Mousses froides}

\index{moulage@moulage!mousses foirdes@--- des mousses froides}
\index{mousse@mousse!moulage froides@moulage des ---s froides}
\index{かたつめ@型詰め!れいせいむーす@冷製ムースの---}
\index{むーす@ムース!れいせいかたつめ@冷製---の型詰め}

いまもそうしている料理人は少なくないようだが、かつては、プレーンな型あるいは浮き彫り模様の付いた型の中に透明なジュレを流して層をつくってやり\footnote{chemiser
  (シュミゼ)ジュレなどを型の内側に流して薄い層を作ること。}、ムースの主素材と関連あるものを装飾要素として貼り付けていた。

こんにちでは次の方法がむしろ好ましい。銀製のタンバル型\footnote{timbale
  (タンバル)円筒形の比較的浅い型。野菜料理用の深皿もこの語で呼ぶので注意。}の底面だけに透明なジュレの薄い層をつくる。型の側面の外側に紙の帯を冷たいバターで貼り付ける。型の\ruby{縁}{ふち}から2〜3
cmくらい高くなるようにすること。そうするとスフレのような見た目のムースになる。紙の帯は型の内側に貼り付けてもいい。この紙の帯は提供直前に、ぬるま湯で濡らしてナイフの刃を使ってムースからそっと引き剥してやる。

タンバル型の用意が整ったら、ムースを詰めて冷やす。アイスクリーム用の冷凍庫に入れるほうがいいだろう。この方法は、小さな銀製のスフレ型に詰めてやってもいいが、それは冷製のスフレにとっておいたほうがいいだろう。アパレイユの構成が同じであるにもかかわらず、冷製ムースと冷製スフレの違いをはっきりさせることが出来るからだ。

とりわけジビエのムースやフォワグラのムースについては、近代的な料理の提供方法に合わせて作られた銀製かガラス製の容器を用いてもいい。その場合は、型の底面だけジュレの層をつくってやり、アパレイユをそのまま流し込めばいい。表面はパレットナイフなどで丁寧に滑らかにならしてやってから、ムースを冷やす。その後、ムースに直接装飾を施し、ジュレをかけて艶を出させる。

ジビエのムースの場合には、そのジビエの胸肉を冷やして、ムースの周囲に飾るようにする。

\atoaki{}

\hypertarget{moulage-des-mousselines-froides}{%
\subsubsection{冷製ムスリーヌの型詰め}\label{moulage-des-mousselines-froides}}

\frsub{Moulage des Mousselines froides}

\index{moulage@moulage!mousselines froides@--- des mousselines froides}
\index{mousseline@mousseline!moulage froides@moulage des --- froides}

\index{かたつめ@型詰め!れいせいむすりーぬ@冷製ムスリーヌの---}
\index{むすりーぬ@ムスリーヌ!れいせいかたつめ@冷製---の型詰め}
\srcChaudFroidBlancheOrdinaire{moulage mousselines froides}{Moulage des Mousseline froides}{れいせいむすりーむのかたつめ}{冷製ムスリーヌの型詰め}

冷製ムスリーヌの型詰めには2つの方法がある。たんに、型にジュレの層を作ってやるか、ソース・ショフロワの層を作ってやるかの違いでしかない。どちらの場合でも、卵形の型に詰めるか、大きなクネルの形状のものにするか、ということになる。

\hypertarget{procede-un-moulage-des-mousselines-froides}{%
\subparagraph{方法1\ldots{}\ldots{}}\label{procede-un-moulage-des-mousselines-froides}}

型の内側に透明なジュレを流して薄い層を作ってやる\footnote{chemiser
  (シュミゼ)。}。その上にアパレイユを張るように塗り、アパレイユのベースとなっている素材とおなじもの---
鶏、ジビエ、甲殻類の身など、とトリュフ---で構成された\protect\hyperlink{salpicons-divers}{サルピコン}を盛り込む。その上からアパレイユを塗って覆い、パレットナイフなどを使ってドーム形に滑らかにならす。冷蔵庫に入れて冷し固める。

\hypertarget{procede-deux-moulage-des-mousselines-froides}{%
\subparagraph{方法2\ldots{}\ldots{}}\label{procede-deux-moulage-des-mousselines-froides}}

型の内側にアパレイユを詰め、さらにサルピコンをその内側に射込む。アパレイユで覆って、冷し固める。

型から外す。ムスリーヌのアパレイユの素材と関連性のある\protect\hyperlink{sauce-chaud-froid-ordinaire}{ソース・ショフロワ}を表面を覆うように塗る\footnote{napper
  (ナペ)。覆いかける(ように塗る)こと。}。トリュフおよびその他の素材(これもムスリーヌと関連性があること)を装飾用に細工したものを飾り付ける。装飾が剥れないように、上からジュレを塗って艶を出させる。

銀製またはガラス製の深皿の底に透明なジュレの層を作り、その上にムスリーヌを並べる。再度ジュレを上からかけてやり、冷蔵庫に入れて提供するまで保管しておく。

\atoaki{}

\hypertarget{souffles-froids}{%
\subsubsection{冷製スフレ}\label{souffles-froids}}

\frsub{Soufflés froids}

\index{souffle@soufflé!froid@---s froids}
\index{すふれ@スフレ!れいせい@冷製---}

冷製スフレはムースそのものに他ならない。だから構成はまったく同じだ。ただ、先に見たようにスフレが10人分\footnote{1
  service
  (アンセルヴィス)、格式のある宴席料理などを作る際の単位。基本は10人分。}を確保できるだけの大きな型に詰めるのに対して、スフレはそもそも、小さなスフレ型に入れてひとり1つ宛で作るものだ。

アパレイユを型に詰める方法はムースの場合と同様、つまり、スフレ型の底にジュレの層を敷いてその上にアパレイユを盛り、型の縁より高くなるように周囲に巻いた紙の帯を利用して縁より高くアパレイユを盛る。そうすると、冷やし固めた後で紙の帯を取り除けば、まるで温製のスフレのように見えることになる。

\hypertarget{nota-souffles-froids}{%
\subparagraph{【原注】}\label{nota-souffles-froids}}

ここまで述べた3種の作り方の基礎はおなじだから、ポイントは次のようにまとめられる。

\begin{enumerate}
\def\labelenumi{\arabic{enumi}.}
\item
  ムースは「スフレ」の名称で供してもいいものだが、混同されるのを避けるために「ムース」の名称で約10人分をひとつの型に入れて作る。
\item
  ムスリーヌはサルピコンを射込んだものであってもそうでなくても、大きなクネルであって、ひとりあたり1つにする。
\item
  スフレは小さなムースであって、スフレ型あるいは似たような型に詰めて、これもひとりあたり1つとする。
\end{enumerate}

\end{recette}

\begin{Main}

\hypertarget{aspics}{%
\subsection[アスピック]{\texorpdfstring{アスピック\footnote{いわゆる「ゼリー寄せ」のやや大掛かりな仕立てだが、aspicという語は本来、フランスやイタリアに生息する蛇の名称。本文にあるように高さがあり中央に穴のあいたリング形にジュレとともに具材を詰め、装飾をおこなうが、その完成した姿が、アスピックという蛇がとぐろを巻いた姿を思わせるというところから付けられた名称といわれている。ジュレが柔らかいものであればそれだけ、大きな型に入れる場合、中空になっているリング型を用いないと自重で崩壊することになる。逆にいえば、リング型を使うのは自重で崩壊するのを防ぐための経験的な知恵なのだろう。}}{アスピック}}\label{aspics}}

\frsecb{Aspics}

\index{aspics@aspics!generalite@généralité}
\index{あすぴつく@アスピック!かいせつ@概説}

アスピックを作る際に、肝に銘じておくべき第一のポイントは、どんなアスピックでも、ジュレがジューシー\footnote{原文succulent(スュキュロン)はsuc(スュック=肉汁)から派生した形容詞で、もともとは「汁気の多い」の意味だったが、そこから転じて「美味な、滋味に富んだ」の意味で一般的に用いられている。ここでは、両方のニュアンスで表現されていると解釈できる。}で美味しく、完全に透き通ったもので、ちょうどいい加減に固まっていなければならないことである。

アスピックを作る際には、昔もそうだったが現代でも、中央に穴の空いたアスピック型\footnote{moule
  à douille (ムーラドゥイユ)サヴァラン型のような中央に穴が空いた型。
  現代では「アスピック型」というと楕円形で中央に穴のないものを指すことが多いが、それとは異なる。あるいはクグロフ型のようなものをイメージするとわかりやすいだろう。19世紀、アスピックには高さのある型が多く用いられたようだ。なお、現代では一般にサヴァラン型というと、型の高さや穴の大きさ等さまざまなタイプのものをまとめて指すことになるので注意。高さのない(低い)、中央の穴が大きな型について、エスコフィエはボルデュール型
  moule à bordure (ムーラボルデュール)と呼んで区別している。}でプレーンなもの、波模様等の装飾のあるものが用いられている。

ボルデュール型\footnote{moule à bordure
  (ムーラボルデュール)料理の縁り飾りを作るための、やや丈が低く中央の穴が大きいリング型。}も使われることがあるが、一般的に、アスピックの中心にガルニチュールを盛り込む場合のみである。

アスピックを型に入れる時には、まず、型の底と周囲に装飾をする。

そのために、型は砕いた氷の中に入れてよく冷やしておく。やや固まりかけたジュレ少量を流し入れ、型を氷の上で転がしながらジュレを周囲に貼り付かせる\footnote{chemiser
  (シュミゼ)。}。次に、装飾するパーツを、固まらない程度に冷たいジュレに浸してからすぐに貼り付ける。装飾については料理人のセンスとアイデア次第なので、ここで明確に述べておくべきことはほとんどない。ひとつだけ言えるのは、常に正確な作業をし、型からアスピックを出したときに装飾がはっきりと見えるようにすべき、ということのみ。

装飾に用いる素材はアスピックの主素材と関連性のあるものでなくてはならない。一般的には、トリュフ、ポシェした卵白、コルニション\footnote{cornichon
  主としてピクルスにする小型のきゅうり、およびそのピクルスのこと。日本では、ハンバーガーによく用いられているドイツ系のピクルス用品種であるガーキンス(英
  gherkins 独
  Einlegegurken)と混同されることがあるが、コルニションはより小さなサイズで収穫し、フレッシュな状態では「いぼ」が尖っているのが特徴。}、ケイパー、いろいろな香草の葉先、ラディッシュの薄い輪切り、オマールのコライユ\footnote{胴の背側にあるオレンジ色がかった「内子」。}、\protect\hyperlink{saumure-liquide-pour-langues}{赤く漬けた舌肉}、等。

アスピックの具材が種々のエスカロップ\footnote{escalope
  (エスカロップ)筋線維とは垂直方向に、厚さ1〜2
  cmに薄切りにした仔牛などの肉や魚の薄い切り身。}や長方形に切ったフォワグラ等で、型の大きさから何度も並べなければならない場合、ジュレの層と交互に重ねて型に入れていく。新しい層を並べる際には先に入れたジュレがある程度固まってからにする。

アスピックの型入れでは常に、最後のジュレの層を充分な厚みにする。できるだけ、型を氷に埋めるようにしながらジュレを流し込んでいくが、早く冷やすために氷に塩を加えてはいけない。塩を使うとジュレの透明さが損なわれるからである。

\noindent\textbf{型から外す方法}\ldots{}\ldots{}型を湯につけてただちに水気を拭い、折ったナフキンや彫刻した氷のブロック等に、アスピックを裏返して型から出す。

菱形や正方形に切ったジュレのクルトン\footnote{パンで作るクルトンとは別に、菱形やさいの目に切った冷製料理装飾用のジュレもクルトンと呼ぶ。}、またはアシェしたジュレで周囲を飾る。

\hypertarget{nota-aspics}{%
\subparagraph{【原注】}\label{nota-aspics}}

アスピックを型に入れて作るには、必然的に、ジュレが相当に固いものでなければならないが、これはまことによろしくない。そもそも固いジュレは口あたりがよくないのだ。だから現代の調理現場では、以下のような方法を採っている。タンバル型か、氷に嵌め込むようにした銀やガラスあるいは陶製の深皿の底に予めジュレの層を作って固めておき、その上にアスピックの素材を並べる。次に、固まりかけのジュレをたっぷり覆いかける。この方法では、装飾をする必要がある場合は、アスピックの調理をおこなう前に、主素材にじかに装飾することになる。

\hypertarget{chauds-froids}{%
\subsection[ショフロワ]{\texorpdfstring{ショフロワ\footnote{ショフロワという仕立てについては\protect\hyperlink{sauce-chaud-froid-brune}{茶色いソース・ショフロワ})訳注参照。なおこのchaud-froidという語の複数形は、それぞれにsを付けchauds-froidsとなる。合成語の複数形はいろいろなパターンがあるので、必要が出たらその都度覚えるようにしたほうがいい。}}{ショフロワ}}\label{chauds-froids}}

\frsecb{Chauds-froids}

\index{chaud-froid@chaud-froid!generalite@généralité}
\index{しよふろわ@ショフロワ!かいせつ@概説}
\srcChaudFroidBlancheOrdinaire{chaud-froid generalite}{Chauds-froids (Généralité)}{しよふろわかいせつ}{ショフロワ(概説)}

\protect\hyperlink{sauce-chaud-froid-ordinaire}{ソース・ショフロワ}には大抵の場合、切り分けた素材を浸す。が、時として大きな塊肉全体をソース・ショフロワで覆わなくてはならない場合もある。ただ、そういう仕立てにする場合には、別の料理名となっている。

ショフロワが複数のばらばらのパーツからなる場合には、それらをソース・ショフロワに漬けたら網の上に並べておく。ソースが冷えたら、それぞれのパーツに装飾をし、ジュレを覆いかけて艶を出してやる。さらに盛り付けの際にはみ出す余分なソースについてはきれいに取り除いておくこと。

大きな塊肉の場合は、よく冷えてはいるけれどまだ流動性のある状態のソース・ショフロワを一気に塗りつけて、その後に装飾をし、ジュレを塗って艶出しすること。

切り分けた素材からなるショフロワの盛り付けは、\protect\hyperlink{fonds-de-plats}{皿の上の台}の上に盛り付けてもいいし、縁飾りの内側に、パンまたは米、セモリナ粉で作った台を置いてその上に盛り付けてもいい。あるいは、銀製か陶製、ガラス製の深皿に盛り付けてもいい。

大きな塊肉のショフロワの場合、皿の上の台にのせてもいいし、あるいは、氷のブロックに料理が嵌まるようにブロックを削ってからそこに盛り付けるのもいい。

ショフロワ仕立ての鶏やジビエについては、正確に切り分けて\footnote{基本的に鶏および鳥類のジビエの可食部は胸肉のみとされていたことに留意。}皮は剥いでおくこと。手羽や下腿肉は使わないので、別の用途に取り置いておくといい。

細かく切った素材のショフロワ仕立ての場合、添えてやるマッシュルームや雄鶏のとさかとロニョン\footnote{rognon
  (ロニョン)牛、羊などの場合は腎臓だが、雄鶏の場合は精巣のこと。高級食材として珍重された。}にもソース・ショフロワを塗ってやること。トリュフはただジュレをかけて艶を出すだけでいい。

\hypertarget{pains-froids}{%
\subsection[パンフロワ]{\texorpdfstring{パンフロワ\footnote{pain froid
  直訳すると「冷たいパン」だが、いわゆるパンとはまったく違う。語の概念としては「パンに似た塊」のこと。}}{パンフロワ}}\label{pains-froids}}

\frsecb{Pains froids}

\index{pains froids@pains froids!generalite@généralité}
\index{はんふろわ@パンフロワ!かいせつ@概説}

古典料理におけるパンフロワとは、ファルスで出来たアパレイユを型に詰めて比較的低温で加熱調理し、冷ましてから型から出して装飾を施し、ジュレをかけて艶を出させたものでしかない。

近代の料理においてこの方法は用いられなくなっており、一般的にいって、パンフロワの代わりとしてムースが作られるようになったわけだ\footnote{この段落は第四版でかなり分量が減らされ、内容も書き換えられている。結果として大きく削られた後半部分の初版の文章は以下のとおり。「(近代の)パンフロワはいずれも、その中心となるアパレイユが次の構成になる。(1)そのパンフロワの主素材からひいた香りゆたかなフュメをほとんどグラス状に煮詰めたものと、卵黄とバターを\protect\hyperlink{sauce-hollandaise}{オランデーズソース}のように立てたもの。(2)このアパレイユが温いかどうかくらいまで冷めたら、溶かしたゼラチンを布で漉しながら流し入れ、さらに主素材のピュレと、それと同量の泡立てた生クリームを加える。
  (3)最後にこのアパレイユに、主素材から切り出した薄切り肉(エスカロップ)にトリュフのスライスを重ねていく。あるいは単純に、肉とトリュフをさいの目に切ったものでもいい。このようにして作ったアパレイユを、あらかじめジュレを内側に流して層を作っておいた型に流し入れ、冷やす、もしくは氷室に入れる。提供直前に、ぬるま湯にさっと型を浸していから米かセモリナ粉で作った台の上に裏返してのせてやる。あるいは皿の底にジュレを敷いただけでもいい。このパンフロワの周囲に、きちっと正確な形状に切ったジュレのクルトンを飾る。【原注】ジュレによるクルトンについては、冷製料理全般にあてはまる」(p.582)。}。

\hypertarget{garnitures-de-mets-froids}{%
\subsection{冷製料理のガルニチュール}\label{garnitures-de-mets-froids}}

\frsecb{Garnitures de Mets froids}

\index{garnitures mets froids@garnitures de mets froids (généralité)}
\index{れいせいりようりのかるにちゆーる@冷製料理のガルニチュール(概説)}

料理に合わせて、ガルニチュールは以下のようなもので構成すること。

\begin{itemize}
\tightlist
\item
  固茹で卵を半割りまたは四つ割りにして詰め物をし、装飾を施してジュレをかけて艶を出したもの
\item
  小さなトマトファルシが、いろいろな食材を添えたもの、または大きなトマトに何らかの詰め物をして正確に櫛切りにしたもの
\item
  小さな野菜皿または舟形の皿に盛った野菜サラダ
\item
  トマトピュレにジュレを混ぜて塗った小さなパンまたはタルトレット
\item
  真っ白なレチュ\footnote{いわゆる「サラダ菜」に属する系統の結球レタスのこと。}の中心部分
\item
  アンチョビのフィレ、オリーブなど\ldots{}\ldots{}
\end{itemize}

\newpage

\hypertarget{serie-des-garnitures}{%
\section{ガルニチュール}\label{serie-des-garnitures}}

\index{garniture@Garniture!serie@\textbf{Série des ---s}|(}

\frsec{Série des Garnitures}

\hypertarget{consideration-sur-la-modification-de-forme-que-peuvent-subir-les-garniture}{%
\subsection{ガルニチュールの見た目を変えることについて}\label{consideration-sur-la-modification-de-forme-que-peuvent-subir-les-garniture}}

\vspace{-1\zw}
\begin{center}
\textit{Considérations sur le modifications de forme que peuvent subir les Garnitures.}
\end{center}
\vspace{1\zw}

\index{garniture@garniture!consideration modification forme@Considération sur les modificqtions de forme que peuvent subir les ---s}
\index{かるにちゆーる@ガルニチュール!みためをかえることについて@---の見た目を変えることについて}

他のどんなレシピでもそうだが、それぞれのガルニチュールの構成上の約束事を勝手に変えてはいけない。もし、どうしても何らかの変更が必要なら、料理本体に合わせて、配置を変えるとか、見た目の形状を変えるだけにすること。ガルニチュールを構成している素材を変えてはいけない。

そうすれば、「牛フィレ肉」のような大きな塊で供する料理か、「トゥルヌド
\footnote{牛フィレ肉を厚さ約2
  cmに切ったもの。周囲に豚背脂のシートを巻いて調理することが多いが、アメリカもしくはイギリス経由で周囲にベーコンを巻く調理法が日本に伝わったために、混同されやすいので注意。\ul{フランス料理としては、豚背脂\\のシートを巻く}。}」のような調理かにかかわらず、同じガルニチュールを合わせることが出来るが、その場合は必然的に、ガルニチュールの形状や盛り付けにおける配置などは変更せざるを得ないわけだ。そうしないと、主素材とガルニチュールの関係性が保てなくなる。

これは、薄切りにしたフィレ肉とシャトーブリヤン\footnote{牛フィレ肉の太い部分、およびそれを約3
  cmの厚さに切ったもの。}の場合も同様だ。理屈からいって当然だろう。

だから、この節において示しているガルニチュールの分量は10人分を基本としているが、大きな塊肉の料理に添えるか、1人分ずつに切って調理して供するかで、量を増やしたり減らしたりすることになる。

これはとても重要なことだ。というのも、本書はフランス料理の伝統的な作り方を集めた本なのだから、多種多様なガルニチュールを収録せざるを得なかったが、その中には近代的な料理にはもはやふさわしくないものだって含まれている。近代的な料理は何よりもまず複雑さを\ruby{厭}{いと}い、ガルニチュールをシンプルなものにする傾向にある。そうすれば皿出しが早くなるし、結果は完璧だ。料理というのは熱々の状態で供されてこそ、完璧な状態で味わっていただけるものだ。ガルニチュールがごくシンプルなものなら、素早い盛り付けにも対応出来る。

同様に、もし可能なら、ガルニチュールを料理の周囲に配置するよりは、別添で供したほうがいいだろう\footnote{大皿に約10人分をまとめて盛り付けるケースを想定して言っていることに留意。}。そうすればどんな料理であっても、本体は事前に切り分けて、ソースにまみれていない状態で盛り付けられた姿を、お客様方にご覧いただくことが可能だ。それからすぐにガルニチュールとソースを回していけばいい。この方式以外に、盛り付けを素早くおこない、清潔で熱々の状態で料理をご提供する手段はなかろう。

これはとりわけ、ルルヴェ\footnote{\protect\hyperlink{releve}{第二版序文訳注}および本章「\protect\hyperlink{farces}{ファルス}」訳注参照。}と呼ばれる大掛かりな仕立ての料理の場合にあてはまることだ。ノワゼット\footnote{約80
  gの牛フィレ肉の筒切り、および、円筒形に切った羊、仔羊の背肉の中心部分。}やトゥルヌドのようなさして大規模ではない仕立てのアントレ\footnote{Entrée
  現代フランス語では「前菜」のことを指すが、かつては約10人分を大きな皿にまとめて盛り付け、給仕の際に取り分ける肉料理(さらに古くは魚料理も)を意味していた。}と呼ばれる料理については、給仕の際に切り分けてガルニチュールを盛り付けてからお客様にお出しするよりは、おひとり様分ずつ盛り付けて供することにすれば、「アントレ」の存在理由はますます低いものとなる。

それでも、アントレについてはそうしたほうがいい。この問題に関しては、料理本体の盛り付けとガルニチュールを切り離したほうが、毎回確実により早く料理をご提供できるのだから、どんな盛り付けの料理だろうと、ぜひためらうことなくこの方式を採用していただきたい。

\hypertarget{remarque-importante-sur-les-sauces-applicables-aux-entrees-de-boucherie-garnies-de-legumes}{%
\subsection{牛、羊肉料理に野菜を添える場合にふさわしいソースについて}\label{remarque-importante-sur-les-sauces-applicables-aux-entrees-de-boucherie-garnies-de-legumes}}

\vspace{-1\zw}
\begin{center}
\textit{Remarque importante sur les sauces applicables aux Entrées de Boucherie garnies de Légumes.}
\end{center}

\index{garniture@garniture!remarque sauce entrees boucherie legumes@Remarque importante sur les sauces applicables aux Entrées de Boucherie garnies de Légumes}
\index{かるにちゆーる@ガルニチュール!うしひつしにくりようりにやさいをそえるはあいにふさわしいそーす@牛、羊肉料理に野菜を添える場合にふさわしいソースについて}
\srcJusDeVeauLie{remarque importante sur les sauces applicables aux entrees de boucherie garnies de legumes}{Remarque importante sur les sauces applicables aus Entées de Boucherie garnies de Légumes}{うしひつしにくりようりにやさいをそえるはあいにふさわしいそーすについて}{牛、羊肉料理に野菜を添える場合にふさわしいソースについて}
\srcEspagnole{remarque importante sur les sauces applicables aux entrees de boucherie garnies de legumes}{Remarque importante sur les sauces applicables aus Entées de Boucherie garnies de Légumes}{うしひつしにくりようりにやさいをそえるはあいにふさわしいそーすについて}{牛、羊肉料理に野菜を添える場合にふさわしいソースについて}
\srcDemiGlace{remarque importante sur les sauces applicables aux entrees de boucherie garnies de legumes}{Remarque importante sur les sauces applicables aus Entées de Boucherie garnies de Légumes}{うしひつしにくりようりにやさいをそえるはあいにふさわしいそーすについて}{牛、羊肉料理に野菜を添える場合にふさわしいソースについて}
\srcChateaubriand{remarque importante sur les sauces applicables aux entrees de boucherie garnies de legumes}{Remarque importante sur les sauces applicables aus Entées de Boucherie garnies de Légumes}{うしひつしにくりようりにやさいをそえるはあいにふさわしいそーすについて}{牛、羊肉料理に野菜を添える場合にふさわしいソースについて}

\vspace{1\zw}

\protect\hyperlink{sauce-espagnole}{エスパニョル}系の派生ソースは野菜を添えた牛、羊肉料理にはふさわしくない。\protect\hyperlink{jus-de-veau-lie}{とろみを付けたジュ}のほうが圧倒的にいい。

だが、いちばんいいのは、軽く仕上げた\protect\hyperlink{glace-de-viande}{グラスドヴィアンド}1
dLに125 gのバターを加えて\footnote{ソースを仕上げる際にバターを加えてより滑らかで艶やかな仕上がりにする。monter
  au beurre (モンテオブール)日本ではブールモンテとも呼ばれる。}、レモン果汁ほんの数滴で仕上げたものだ。とはいえ、このバターを加えたグラスドヴィアンドは野菜を包み込んでしまわない程度に充分に軽い仕上がりにすること\footnote{グラスドヴィアンドとはフォンを煮詰めたものに他ならないから、ここで述べられているのは、\protect\hyperlink{jus-de-veau-lie}{とろみを付けたジュ}から「とろみ」の要素であるでんぷんを省き、やや多めのバターを加えて滑かに仕上げたもの、と考えていい。}。

アスパラガスの穂先とかプチポワ\footnote{petits pois
  いわゆるグリンピースのことだが、フランスではより若どりの、直径7〜8
  mm程度のものが好まれる。}、アリコヴェール\footnote{haricots verts
  いわゆる、さやいんげん。これもごく細い若どりのもの(太さ8〜9
  mm)程度のものが好まれる。}、マセドワーヌ \footnote{macédoine
  多くの場合、小さめのさいの目に切った蕪(navet
  ナヴェ)やアリコヴェール、プチポワ、にんじんなどを混ぜ合わせたもの。日本のマセドアンサラダの原型となった。ただし言葉の意味としては「各種の野菜を混ぜあわせたもの」であり、料理用語として切り方が決まっているわけではない。}などの野菜は、ソースをある意味、分解してしまう。それは野菜そのものが持つ水分によってだったり、野菜をあえているアパレイユのせいだったりする。

その結果、大皿から取り皿に分けてお客様のところに運ばれた時には、ほとんど食欲を失なわせるような見た目になってしまう。こういう事態は\protect\hyperlink{sauce-chateaubriand}{ソース・シャトーブリヤン}か、バターを加えたグラスドヴィアンドを料理に合わせれば解決する。これらのソースは分解しないどころか、野菜のガルニチュールととてもよく合う。同時に、野菜のガルニチュールにもこれらのソースはとても素晴らしいふんわりとした食感を与えてくれるからだ。

そんなわけで、以下の点にぜひとも留意していただきたい。出来るだけ、エスパニョル系の派生ソースやトマトソースは、\protect\hyperlink{garniture-financiere}{ガルニチュール・フィナンシエール}や\protect\hyperlink{garniture-godard}{ゴダール}のような、トリュフ、雄鶏のとさかとロニョン、クネル、マッシュルームなどを添える料理にとっておくべきだ。野菜のガルニチュールには、とろみを付けたジュ、もしくはバターを加えたグラスドヴィアンドのほうがずっと好ましい。

\end{Main}
%\newpage
\href{✓原稿下準備なし}{} \href{訳と注釈\%2020180420進行中}{}
\href{未、原文対照チェック}{} \href{未、日本語表現校正}{}
\href{未、注釈チェク}{} \href{未、原稿最終校正}{}

\hypertarget{garnitures-recettes}{%
\subsection{ガルニチュールのレシピ}\label{garnitures-recettes}}

\frsecb{Garnitures}

\begin{center}
\medlarge(ここで示す分量はすべて仕上がり10人分)
\end{center}
\normalsize
\begin{recette}
\hypertarget{garniture-algerienne}{%
\subsubsection{ガルニチュール・アルジェリア風}\label{garniture-algerienne}}

\frsub{Garniture à l'Algérienne}

\index{garniture@garniture!algerienne@--- à l'Algérienne}
\index{algerien@algérien(nne)!garuniture à l'---ne}
\index{かるにちゆーる@ガルニチュール!あるしえりあふう@---・アルジェリア風}
\index{あるしえりあふう@アルジェリア風!かるにちゆーる@ガルニチュール・---}

(牛、羊の塊肉\footnote{原文 Pour les pièce de boucherie
  より正確に訳すなら、「\ul{肉屋
  (boucherie)が伝統的に扱かってきた、白身肉を除く畜産精肉}、具体的には牛、羊(馬も含まれる)の塊肉」であり、牛の場合は基本的にランプ、イチボに相当する部位、羊の場合は鞍下肉から腿上部にかけての部位を塊のまま調理したものを意味することがほとんど。}の料理に添える)

\begin{itemize}
\item
  ワインの栓の形にしたさつまいもの\protect\hyperlink{croquettes}{クロケット}10個
\item
  小さなトマト10個は中をくり抜いて味付けをし、植物油少々で弱火で蒸し煮する
\item
  ソース\ldots{}\ldots{}薄く仕上げた\protect\hyperlink{sauce-tomate}{トマトソース}に、グリルして皮を剥き、細かい千切りにしたポワヴロン\footnote{いわゆる青果としてのパプリカ。}を加える
\end{itemize}

\hypertarget{garniture-alsacienne}{%
\subsubsection{ガルニチュール・アルザス風}\label{garniture-alsacienne}}

\frsub{Garniture à l'Alsacienne}

\index{garniture@garniture!alsacienne@--- à l'Alsacienne}
\index{alsacien@alsacien(ne)!garuniture à l'---ne}
\index{かるにちゆーる@ガルニチュール!あるさすふう@---・アルザス風}
\index{あるさすふう@アルザス風!かるにちゆーる@ガルニチュール・---}

(牛、羊の塊肉、牛フィレ、トゥルヌドに添える)

\begin{itemize}
\item
  ブレゼ\footnote{\protect\hyperlink{chou-braise}{キャベツのブレゼ}を参考にすること。}したシュークルート\footnote{生食出来ないくらい固くて大きな専用品種であるキャベツを千切りにして香辛料などとともに塩蔵、醗酵さたもの。ドイツのザワークラウトが原型だが、歴史的にフランスとドイツで領土の取り合いとなったアルザス地方で独自に発展した。温めたシュークルートにソーセージなどの豚肉加工品を添えたchoucoûte
    garnie(シュークルートガルニ)はアルザスの名物料理のひとつ。}を詰めてハムの脂身のないところを円く切ってのせたタルトレット10個
\item
  ソース\ldots{}\ldots{}\protect\hyperlink{jus-de-veau-lie}{とろみを付けた仔牛のジュ}
\end{itemize}

\hypertarget{garniture-americaine}{%
\subsubsection[ガルニチュール・アメリケーヌ]{\texorpdfstring{ガルニチュール・アメリケーヌ\footnote{\protect\hyperlink{sauce-americaine}{ソース・アメリケーヌ}も参照されたい。}}{ガルニチュール・アメリケーヌ}}\label{garniture-americaine}}

\frsub{Garniture à l'Américaine}

\index{garniture@garniture!americaine@--- à l'Américaine}
\index{americain@américain(e)!garuniture à l'---e}
\index{かるにちゆーる@ガルニチュール!あめりけーぬ@---・アメリケーヌ}
\index{あめりかん@アメリカン/アメリケーヌ!かるにちゆーる@ガルニチュール・アメリケーヌ}

(魚料理に添える)

\begin{itemize}
\item
  このガルニチュールは必ず、\protect\hyperlink{homard-americaine}{オマール・アメリケーヌ}の方法で調理した尾の身をやや斜めに1
  cm程度の薄切り\footnote{escalope
    (エスカロップ)肉などを筋線維と直角に、丸くスライスしたもの。}にして供する
\item
  ソース\ldots{}\ldots{}オマール・アメリケーヌのソース
\end{itemize}

\hypertarget{garniture-andalouse}{%
\subsubsection[ガルニチュール・アンダルシア風]{\texorpdfstring{ガルニチュール・アンダルシア風\footnote{アンダルシア風、つまりスペイン風といいながら、ギリシャ風ライスを使うという点からも、料理名に付けられた地名がしばしば不確かで大雑把な理由さえないことが多いことが理解されよう。}}{ガルニチュール・アンダルシア風}}\label{garniture-andalouse}}

\frsub{Garniture à l'Andalouse}

\index{garniture@garniture!andalouse@--- à l'Andalouse}
\index{andalou@andalou(se)!garuniture à l'---se}
\index{かるにちゆーる@ガルニチュール!あんたるしあふう@---・アンダルシア風}
\index{あんたるしあふう@アンダルシア風!かるにちゆーる@ガルニチュール・---}

(牛、羊の塊肉料理や鶏料理に添える)

\begin{itemize}
\item
  中位の大きさのポワヴロン10個をグリル焼きして中をくり抜き、\protect\hyperlink{riz-grecque}{ギリシャ風ライス}を詰める
\item
  なす\footnote{フランスで伝統的なタイプのなすはヘタが緑色で、風味や調理特性はいわゆる米なすに近いが、形状は比較的細長い。直径4〜6
    cm、長さ25 cmくらいのものが多い。}を4
  cmの厚さの輪切りにして面取りをし、中に窪みをつくって油で揚げ、提供直前に油で炒めたトマトをのせる
\item
  ソース\ldots{}\ldots{}\protect\hyperlink{jus-de-veau-lie}{とろみを付けたジュ}
\end{itemize}

\hypertarget{garniture-arlesienne}{%
\subsubsection[ガルニチュール・アルル風]{\texorpdfstring{ガルニチュール・アルル風\footnote{南フランスの都市
  Arles
  (アルル)の形容詞および名詞形。名詞の場合は「アルルの人」の意味になる。アルルはオランダ出身の印象派〜ポスト印象派の画家フィンセント・ファン・ゴッホ
  Vincent van Gogh
  (フランス語では昔からヴァンソンヴァンゴーグと呼ぶ習慣が付いてしまっており、現代フランス語の原語発音尊重の風潮にもかかわらず、そのように発音されることは多いようだ)が1888年から1889年までアトリエを構え、「ひまわり」など多くの傑作を描いた。有名な、自分の耳を切り落すという「事件」を起こしたのもアルルでのことだ。この時期の作品のひとつに、「アルルの女(ジヌー夫人)」と呼ばれる一連のものがある。モデルはアルルのカフェの経営者だといわれている。もっとも、フランスにおいて画家としてのゴッホおよび彼の作品は生前はほとんど評価されることがなく、生前に売れた絵は1枚だけだったとさえいわれている。このレシピは初版つまり1903年から収められているため、ゴッホの絵との関連はほぼないと考えていいだろう。むしろ、小説化アルフォンス・ドーデ原作を戯曲化してジョルジュ・ビゼーが劇音楽を付けた『アルルの女』(1872年初演、
  1878年再演)との関連があると見るのがいいだろう。この作品は初演時点ではあまり好評ではなかったが、再演で大ヒットとなった。\protect\hyperlink{sauce-bohemienne}{ソース・ボヘミアの娘}のように、人気のある劇やオペラのタイトルを料理名につけて、その人気にあやかろうという風潮が19世紀後半には比較的多かった。そのため、トマトとなすという南フランスを思わせる食材を使ってはいてもアルルという土地に何の関係もないと思われる、内容的にも凡庸なこのガルニチュールに、当時の人気作品の名をつけて、いかにも流行のものであるかのように供したのが定着した、と考えることも可能だろう。その場合は「\ul{ガルニチュール・アルルの女}」と訳すべきかも知れない。なお、ビゼーが最初に作曲したのは27曲からなる舞台音楽であって、独立した音楽作品でもなければ、オペラでもなかったが、そのなかから数曲を選んで編曲し(あるいは作曲しなおし)、『アルルの女 組曲』としてこんにち広く知られている。第1組曲と第2組曲があり、前者はビゼー自身によるオーケストラ用編曲。後者はビゼーの死後1879年に友人エルネスト・ギローが完成させた。第1組曲の「メヌエット」や第2
  組曲の「ファランドール」など、曲名は知らずとも、メロディーを聴いたことのある読者も少なくないとと思われる。}}{ガルニチュール・アルル風}}\label{garniture-arlesienne}}

\frsub{Garniture à l'Arlésienne}

\index{garniture@garniture!arlesienne@--- à l'Arlésienne}
\index{arlesien@arlésien(ne)!garuniture à l'---ne}
\index{かるにちゆーる@ガルニチュール!あるるふう@---・アルル風}
\index{あるるふう@アルル風!かるにちゆーる@ガルニチュール・---}

(トゥルヌドやノワゼットの料理に添える)

\begin{itemize}
\item
  なす\footnote{なす、トマト、玉ねぎの分量は記されていないので適宜判断すること。}は1
  cm程の厚さにスライスして塩こしょうをし、小麦粉をまぶして油で揚げる
\item
  トマト皮を剥いてスライスし、バターでソテーする
\item
  玉ねぎは輪切りにして指輪のようにばらばらにし、小麦粉をまぶして油で揚げ、花束のように盛る
\item
  ソース\ldots{}\ldots{}トマト風味の\protect\hyperlink{sauce-demi-glace}{ソース・ドゥミグラス}
\end{itemize}

\hypertarget{garniture-banquiere}{%
\subsubsection[ガルニチュール・銀行家夫人風]{\texorpdfstring{ガルニチュール・銀行家夫人風\footnote{原文の
  à la Banquière をここでは文字通り訳した。料理名において {[}à la +
  形容詞の女性形{]}は通常、à la manière/façon
  〜のmanièreもしくはfaçonが省力されたものと考えられている。これらmanière,
  façon
  いずれも女性名詞であるために、この後に付ける形容詞も女性形となる。ところが「〜風」」「〜を記念して/〜を称揚して」の意味で{[}à
  la + (固有)名詞{]}という用法もある。これは à la manière de + 名詞、の
  manière
  deが省略されたものと考える。Banquier(ボンキエ)は「銀行家」を意味する名詞であり、女性の場合はbanquièreとなり、女性銀行家あるいは銀行家夫人ということになる。そのため、従来は「銀行家風」と訳されていたが、あえて文法の原則に忠実に「銀行家夫人風」を訳した。さて、この料理名だが、日仏料理協会編『フランス 食の事典』(白水社、2000
  年)には「産業革命に伴う産業の隆盛を支えた銀行は、現代にいたるまで資本主義社会の根幹をなすもので、その経営者は19世紀において金持ちの代名詞ともなった。当時、「銀行家風」は王風、王妃風にかわる新しい表現だった(pp.162-163」と説明されている。ところが、料理書においてこのà
  la
  Banquièreという表現は1856年のデュボワ、ベルナール共著『古典料理』以前には見つからない。しかも、「冷製料理用ガルニチュール・銀行家夫人風」Garniture
  à la banquière, pour froid (t.1,
  p.259)および「若鶏のガランティーヌ・銀行家夫人風」Galantine de poulet
  à la banquière (t.2,
  p.40)の2つでのみ料理名に使われているのみ。ガルニチュールの概要は、オマール2尾の身をやや斜めの円形(エスカロップ)にスライスする。これをひとつずつ別々の陶製の器に入れ、小さなアーティチョークの基底部を茹でたもの、大きな黒または白トリュフのスライス、マッシュルームのスライス、コルニションのスライスを盛り込み、塩、こしょう、植物油、パセリとエストラゴンのみじん切りで味付けし、銘々に供する、というもの。本書のガルニチュールと温製、冷製の違いはあっても、同じ名称とは思い難いくらい異なった内容。その前後および以前については、毎年のように版を重ねながら増補されたために料理の流行、変遷を見るのに非常に便利なヴィアールにもオドにも収録されておらず、グフェ『料理の本』(1867年)にも見あたらない。本書よりやや時代が下って、
  1838年の『ラルース・ガストロノミック』初版の「ガルニチュール・銀行家夫人風」は「鶏、仔牛胸腺肉(リドヴォー)の料理、ヴォロヴァン用。クネル、マッシュルーム、トリュフのスライス、ソース・バンキエール
  (p.136)」と定義されている。ソース・バンキエールsauce
  banquièreについては「卵料理、鶏料理、牛や羊の副生物(リドヴォーなど)、ヴォロヴァン用。ソース・シュプレーム2
  dLにマデイラ酒 \(\frac{1}{2}\)
  dLを加え、布で漉す。トリュフのみじん切り大さじ2杯を加えて仕上げる(p.959)」とある。
  2007年版の『ラルース・ガストロノミック』でもほぼ同様の内容だが、ソース・バンキエールのレシピはこの版では欠落している。また、20世紀についても、1950年に刊行されたレシピ集『フランス料理技法』(Flammarion)
  にソース・バンキエールのレシピは見られるが(p.147)、これはモンタニェの『料理大全』(1929年)からの引用であり、ガルニチュール・バンキエールについては何も出ていない。1952年のペラプラ『近代料理技術』にも、
  1953年のキュルノンスキー編『フランスの料理とワイン』にもこれらへの言及なない。ところが2018年現在、インターネットで検索するとpoularde
  à la banquière
  「肥鶏 女銀行家風」のような、ここで見てきたものとはかなり内容の違うレシピが見つかる。「銀行家風」にしろ「女銀行家風」「銀行家夫人風」にしろ、銀行家という語には肯定的な「富の象徴」というイメージがあると同時に、「\ruby{吝嗇} {りんしょく}家」あるいは「カネ貸し」場合によっては「官僚主義的」のようなマイナスイメージが伴なわれ得ることもまた事実だろうし、銀行家が出席している宴席で「銀行家風」の料理を出す場合にはいろいろな誤解やトラブルの原因となる可能性さえあるかも知れない。このことから、『ラルース・ガストロノミック』が初版から2007年版までほぼ記述を変えなかった、つまり誰もこの名称のガルニチュールに手を加えなかった、ということの証左ともなろう。}}{ガルニチュール・銀行家夫人風}}\label{garniture-banquiere}}

\frsub{Garniture à la Banquière}

\index{garniture@garniture!banauiere@--- à la Banquière}
\index{banquier@banquier(ère)!garuniture à la Banquière}
\index{かるにちゆーる@ガルニチュール!きんこうかふしんふう@---・銀行家夫人風}
\index{きんこうかふしんふう@銀行家夫人風!かるにちゆーる@ガルニチュール・---}

(肥鶏の料理に添える)

\begin{itemize}
\item
  ひばり\footnote{mauviette
    (モヴィエット)、ひばりの食材としての名称。生物としてはalouette(アルエット)と呼ぶ。なお、オルレアネ地方の郷土料理に、
    pithiviers de mauviettes
    という、脳と鶏のファルスを詰めたひばりを折込みパイ生地で包んで焼いた料理があるが、pithiviers(ピティヴィエ)とだけ言う場合は、バターと砂糖、アーモンドパウダーなどを折込みパイ生地で包んで上部を渦巻模様に装飾したオルレアネ地方発祥の菓子を指すので注意。}10羽を背側から開いて骨をすべて取り除き\footnote{désosser
    (デゾセ)。日本の調理現場でも比較的よく使われる用語。この語に含まれるosは「骨」のこと、déは「反対、除去」などを意味する接頭辞、erは動詞であることを示す語尾。したがって、文字どおり「骨を取り除く」の意になる。}、\protect\hyperlink{farce-gratin-c}{ファルス・グラタン}を詰めて、表面を色よく焼き、カスロールで火を通す\footnote{en
    casserole
    (オンカスロール)カスロール仕立てと解釈も可能。\protect\hyperlink{sauce-smitane}{ソース・スミターヌ}訳注参照。}
\item
  \protect\hyperlink{farce-b}{鶏のファルス}で小さなクネル10個
\item
  トリュフのスライス10枚
\item
  ソース\ldots{}\ldots{}トリュフエッセンスを加えた\protect\hyperlink{sauce-demi-glace}{ソース・ドゥミグラス}
\end{itemize}

\hypertarget{garniture-berrichonne}{%
\subsubsection[ガルニチュール・ベリー風]{\texorpdfstring{ガルニチュール・ベリー風\footnote{berrichon(ne)(ベリション/ベリショーヌ)
  はフランス中央部にある地方名 Berry の形容詞。ここでは女性形
  berrichonneとなる。山羊乳のチーズで有名。なおフランス史関連の書物ににおいてよく見かける、ベリー公
  duc de Berry
  (デュックドベリー)という公爵位はフランスの王族(つまりその時の王の近縁者)に与えられた爵位で、その後フランス王となった者も多い。このため、いわゆる「世襲」はされてこなかった。また、中世フランスでもっとも豪華で美しい写本とされる数部の『\href{http://gallica.bnf.fr/ark:/12148/btv1b520004510}{ベリー公のいとも豪華なる時祷書}』(14世紀)は当時のベリー公ジャン1世が作成させたもの。}}{ガルニチュール・ベリー風}}\label{garniture-berrichonne}}

\frsub{Garniture à la Berrreichonne}

\index{garniture@garniture!berrichonne@--- à la Berrichonne}
\index{berrichon@berrichon(ne)!garuniture à la Berrichonne}
\index{かるにちゆーる@ガルニチュール!へりーふう@---・ベリー風}
\index{へりーふう@ベリー風!かるにちゆーる@ガルニチュール・---}

(牛、羊肉の大がかりな料理\footnote{ルルヴェ relevé
  のこと。\protect\hyperlink{releve}{第二版序文訳注}参照。}に添える)

\begin{itemize}
\item
  卵の大きさにした\protect\hyperlink{chou-braise}{サヴォイキャベツのブレゼ}20個
\item
  キャベツとともに火を通した塩漬け豚バラ肉の小さなスライス10枚
\item
  小玉ねぎ20個と大粒のマロン20個はこのガルニチュールを添える肉の煮汁で火を通す
\item
  ソース\ldots{}\ldots{}アロールート\footnote{Allow-root
    南米産クズウコンを原料とした良質のでんぷん。現代の日本ではコーンスターチで代用することがほとんど。}でとろみを付けた、ブレゼの煮汁
\end{itemize}

\hypertarget{garniture-berny}{%
\subsubsection[ガルニチュール・ベルニ]{\texorpdfstring{ガルニチュール・ベルニ\footnote{ピエール・ド・ベルニPierre
  de Bernis
  (1715〜1794)のこと。なぜか料理名としてはBernyの綴りが一般的だが、個人名なのでもちろん誤り。
  29才でアカデミーフランセーズに入った俊才。ポンパドゥール夫人の庇護のもとルイ15世からも重用された。駐ヴェネツィア大使として食卓外交を展開したが、フランス革命後、ローマで客死した。}}{ガルニチュール・ベルニ}}\label{garniture-berny}}

\frsub{Garniture à la Berny}

\index{garniture@garniture!berny@--- à la Berny}
\index{Berny@Berny (Bernis)!garuniture à la Berny}
\index{かるにちゆーる@ガルニチュール!へるに@---・ベルニ}
\index{へるに@ベルニ!かるにちゆーる@ガルニチュール・---}

(ジビエおよびマリネした牛、羊肉料理\footnote{シュヴルイユ仕立てのこと。\protect\hyperlink{sauce-porvrade}{ソース・ポワヴラード}および\protect\hyperlink{marinade-crue-pour-viandes-de-boucherie-ou-venaison}{マリナード}参照。}用)

\begin{itemize}
\item
  小さな俵形にしたじゃがいものクロケット・ベルニ\footnote{本書の温製オードブルの節に「クロケット・ベルニ」は掲載されていない。野菜料理の章にある「\protect\hyperlink{pommes-de-terre-berny}{じゃがいも・ベルニ}」をアパレイユとしてクロケットを作ることになる。}10個
\item
  空焼きしたタルトレット10個にバターを加えたマロンのピュレをドーム状に詰め、バターで軽くソテーして艶を出させたトリュフのスライスをタルトレットに1枚ずつのせる
\item
  軽く仕上げた\protect\hyperlink{sauce-poivrade}{ソース・ポワヴラード}。
\end{itemize}

\hypertarget{garniture-bezontinne}{%
\subsubsection[ガルニチュール・ブザンソン風]{\texorpdfstring{ガルニチュール・ブザンソン風\footnote{Besonçon
  (ブゾンソン)フランス東部、ブルゴーニュ=フランシュ=コンテ圏の都市。形容詞は通常bisontin(e)(ビゾンタン/ビゾンティーヌ)だが、本書のようにbizontin(e)と綴ることもある。なお、1980年代に画期的といわれたフランス語教材\emph{C'est
  le
  printemps}の第1課においてはじめて出てくる地名がブザンソンだった。この教材は会話例のリアリティや題材としてdocuments
  authentiques(ドキュモンオトンティック=現実にあるドキュメントすなわち言語を用いたさまざまな書類、看板、広告など)を積極的に採用したこととともに、アプレ68(フランスの学生運動および現代思想における転換期のひとつとなった1968年の「五月革命」以後に多方面において展開された時代特有の雰囲気)が強く表われているのが特徴だった。同時期のフランス語教材の傑作とされる(やや保守的な傾向の)通称「カペル」\emph{Le
  français en direct}
  と並び、フランス語教育・教授法において現在のEUおよびフランスで定められ運用されている「外国語としての言語コミュニケーション能力」の概念形成の先駆けとなった。アプレ68的なものは食文化、料理の世界においても、ゴ\&ミヨの批評と店の格付けにおける、既存のミシュランのガイドブックのオルタナティヴとしての方向性、ヌーヴェルキュイジーヌ宣言などによく表われている。}}{ガルニチュール・ブザンソン風}}\label{garniture-bezontinne}}

\frsub{Garniture à la Bizontine}

\index{garniture@garniture!bizontinne@--- à la Bizontine}
\index{bizontin@bizontin(e) ⇒ bisontin(e)!garuniture à la ---e}
\index{かるにちゆーる@ガルニチュール!ふさんそんふう@---・ブザンソン風}
\index{ふさんそん@ブザンソン!かるにちゆーる@ガルニチュール・---風}

(牛、羊の塊肉料理およびトゥルヌドに添える)

\begin{itemize}
\item
  \protect\hyperlink{croustade-en-pomme-duchesse}{クルスタード・ポム・デュシェス}\footnote{\protect\hyperlink{pomme-de-terre-duchesse}{ポム・デュシェス}をバターを塗ったダリオル型(小さな円筒形の型)に詰めて成形してからイギリス式パン粉衣を付けて油で揚げ、中をくり抜いてケースにする。詳細は温製オードブルの節参照。}10個は提供直前にドリュール\footnote{色艶よく焼き上げるために卵黄を溶いたもの、あるいは卵黄に水を加えて溶いたものをdorure(ドリュール)と呼び、それを塗ることをdorer
    (ドレ)という動詞で表現する。}を塗り、オーブンに入れて色よく焼く。生クリームを加えたカリフラワーのピュレを詰めてクルスタードの中に絞り袋を使って詰める
\item
  半割りにした\protect\hyperlink{laitues-farcies-pour-garniture}{ガルニチュール用レチュのファルシ}10個
\item
  ソース\ldots{}\ldots{}バターを加えて仕上げた\protect\hyperlink{jus-de-veau-lie}{とろみを付けたジュ}
\end{itemize}

\hypertarget{garniture-boulangere}{%
\subsubsection[ガルニチュール・ブランジェール]{\texorpdfstring{ガルニチュール・ブランジェール\footnote{boulanger/boulangère
  は「パン屋、パン職人」の意。}}{ガルニチュール・ブランジェール}}\label{garniture-boulangere}}

\frsub{Garniture à la Boulangère}

\index{garniture@garniture!boulangere@--- à la Boulangère}
\index{boulanger@boulanger/boulangère!garuniture à la ---ère}
\index{かるにちゆーる@ガルニチュール!ふらんしえーる@---・ブランジェール}
\index{ふらんしえ@ブランジェ/ブランジェール ⇒ パン屋!かるにちゆーる@ガルニチュール・ブランジェール}
\index{はんや@パン屋 ⇒ ブランジェ/ブランジェール!かるにちゆーる@ガルニチュール・ブランジェール}

(羊、乳呑み仔羊、鶏料理に添える)

\begin{enumerate}
\def\labelenumi{\arabic{enumi}.}
\item
  玉ねぎ250 gは薄切りにし\footnote{émincer (エマンセ)。}て、バターで色よく炒める
\item
  じゃがいも750 gは櫛切りか薄切りにする
\item
  塩15 gとこしょう5 g
\end{enumerate}

\begin{itemize}
\item
  1〜3を混ぜ合わせて、このガルニチュールを添える肉を油を熱したフライパンで表面を焼き固め\footnote{rissoler
    (リソレ)。}とともにオーヴンに入れて、一緒に火を通す
\item
  鶏の場合は、じゃがいもはオリーブ形に成形し\footnote{tourner
    (トゥルネ)}、小玉ねぎをあらかじめバターでこんがり焼き色を付けておく。
\item
  ソース\ldots{}\ldots{}美味しい肉汁(ジュ)少々
\end{itemize}

\hypertarget{garniture-bouquetiere}{%
\subsubsection[ガルニチュール・ブクティエール]{\texorpdfstring{ガルニチュール・ブクティエール\footnote{花売り娘、の意。}}{ガルニチュール・ブクティエール}}\label{garniture-bouquetiere}}

\frsub{Garniture à la Bouquetière}

\index{garniture@garniture!bouquetiere@--- à la Bouquetière}
\index{bouquetiere@bouquetière!garuniture à la ---}
\index{かるにちゆーる@ガルニチュール!ふくていえーる@---・ブクティエール}
\index{ふくていえーる@ブクティエール ⇒ 花売り娘!かるにちゆーる@ガルニチュール・ブクティエール}
\index{はなうりむすめ@花売り娘 ⇒ ブクティエール!かるにちゆーる@ガルニチュール・ブクティエール}

(牛、羊の大掛かりな仕立ての料理\footnote{ルルヴェ relevé
  のこと。\protect\hyperlink{releve}{第二版序文訳注}参照。}に添える)

\begin{itemize}
\item
  にんじん250 gと蕪250
  gはスプーンで中をくり抜いて下茹でし、バターで色艶よく炒める\footnote{glacer
    (グラセ)。}
\item
  小さなじゃがいも250 gはシャトー\footnote{長さ6
    cm程度の細長い樽の形状にすること。両端は切り落すので、ラグビーボール形ではない。}に成形する\footnote{いずれも適切に加熱調理するが、この節では細かく説明されていないので、対応する野菜のページを参照すること。}
\item
  プチポワ\footnote{petits pois
    (プティポワ)いわゆるグリンピースのことだが、日本でよく知られているものよりも若どりで小さく、風味も軽やかで甘みがある。}250
  gと、さいの目に切ったアリコヴェール\footnote{haricots verts
    さやいんげんのことだが、これも日本のものより若どりに適した品種が好まれる。}250
  g
\item
  カリフラワー250 gは花束の形状にバラしておく
\end{itemize}

以上の材料をそれぞれ加熱調理した後に、塊肉の周囲に、ブーケ状に、それぞれを離してニュアンスが明確になるように盛り付ける。カリフラワーのブーケには\protect\hyperlink{sauce-hollandaise}{オランデーズソース}を薄く塗ること。

\begin{itemize}
\tightlist
\item
  ソース\ldots{}\ldots{}塊肉を調理した際の肉汁の浮き脂を取り除き\footnote{dégraisser
    (デグレセ)。}、澄ませたもの
\end{itemize}

\hypertarget{garniture-bourgeoise}{%
\subsubsection[ガルニチュール・ブルジョワーズ]{\texorpdfstring{ガルニチュール・ブルジョワーズ\footnote{bourgeois(e)
  (ブルジョワ/ブルジョワーズ)。ブルジョワ風の意。中世においては都市に住む貴族ではないある種の特権階級を意味したが、
  19世紀以降は、肉体労働をせずに快適できわめて豊かな生活をおくれる社会階層、の意に変化した。社会が物質的に、経済的に豊かになるにともない
  petit bourgeois
  (プティブルジョワ)なる階層も出現したが、ブルジョワの本義はあくまでも「大金持ち」であり、現代日本語でいうところの「セレブ」に相当すると思っていい。}}{ガルニチュール・ブルジョワーズ}}\label{garniture-bourgeoise}}

\frsub{Garniture à la Bourgeoise}

\index{garniture@garniture!bourgeoise@--- à la Bourgeoise}
\index{bourgeois@bourgeois(e)!garuniture à la ---}
\index{かるにちゆーる@ガルニチュール!ふるしよわーす@---・ブルジョワーズ}
\index{ふるしよわーす@ブルジョワーズ!かるにちゆーる@ガルニチュール・ブルジョワーズ}
\index{ふるしよわふう@フルジョワ風 ⇒ ブルジョワーズ!かるにちゆーる@ガルニチュール・ブルジョワーズ}

(牛、羊の塊肉料理に添える)

\begin{itemize}
\item
  にんじん500 gは、にんにくのような形に成形して\footnote{tourner
    (トゥルネ)。}下茹でし、バターで色艶よく炒める\footnote{glacer
    (グラセ)。もともとは「鏡のようにする」ところから「艶を出す」の意となり、野菜の場合はもっぱら下茹でした後にバターで軽く炒めて艶を出すことをいうが、場合によっては茹でる段階で砂糖を煮含めたりもする。}
\item
  小玉ねぎ\footnote{日本のいわゆる「ペコロス」は黄色系品種が多いが、フランスの小さな玉ねぎはもっぱら白系品種であり、甘さや風味がまったく異なるので注意。}500
  gは下茹でした後にバターで色艶よく炒める
\item
  塩漬け豚バラ肉\footnote{原文 lard de poitrine
    (ラールドポワトリーヌ)は豚バラ肉のことだが、通常は塩蔵、熟成させたもの、およびそれを冷燻にかけたものを指す。しばしば「ベーコン」と誤訳されているが、日本語のいわゆるベーコンとは違うので注意。}125
  gはさいの目に切ってバターでこんがり炒める
\item
  このガルニチュールは、塊肉にほぼ火が通った段階で、鍋の中の肉の周囲に入れてやり、ブレゼの煮汁で火入れを完全にすること
\end{itemize}

\hypertarget{garniture-brabanconne}{%
\subsubsection[ガルニチュール・ブラバント風]{\texorpdfstring{ガルニチュール・ブラバント風\footnote{現在はベルギー中部の州ブラバントBrabantの、の意。なお、この名称のガルニチュールは『ラルース・ガストロノミック』初版にも掲載されているが、内容がまったく異なる。アンディーヴとじゃがいものピュレ、ホップの若芽を茹でてバターか生クリームであえたもので構成するという
  (p.239)。なおブラバントは中世においてブラバント公国として独立した国家であった。ベルギー王国成立後は、儀礼称号としてベルギー王家の法定推定相続人にブラバント公の称号が授けられるようになった。なお、エスコフィエによる\protect\hyperlink{peches-melba}{ピーチメルバ}創案のきっかけとなったといわれるワーグナーの楽劇『ローエングリン』においてネリー・メルバNellie
  Melba(1861〜1931)が演じていたエルザ・フォン・ブラバントはブラバント公国の公女という設定。}}{ガルニチュール・ブラバント風}}\label{garniture-brabanconne}}

\frsub{Garniture à la Brabançonne}

\index{garniture@garniture!brabanconne@--- à la Brabançonne}
\index{brabanconne@brabançon(ne)!garuniture à la ---ne}
\index{かるにちゆーる@ガルニチュール!ふらはんとふう@---・ブラバント風}
\index{ふらはんとふう@ブラバント風!かるにちゆーる@ガルニチュール・---}

(牛、羊の塊肉の料理に添える)

\begin{itemize}
\item
  空焼きしたタルトレット10個に、下茹でしてバターで蒸し煮した\footnote{étuver
    (エチュヴェ)。}芽キャベツ\footnote{芽キャベツはchoux de Bruxelles
    (シュドブリュクセル、ブリュッセルのキャベツの意)と呼ぶ。}をピュレにして詰め、\protect\hyperlink{sauce-mornay}{ソース・モルネー}を塗る
\item
  \protect\hyperlink{pommes-de-terre-duchesse}{ポムデュシェス}で作った小さな円盤形のクロケット10個
\item
  ソース\ldots{}\ldots{}\protect\hyperlink{jus-de-veau-lie}{とろみを付けたジュ}
\end{itemize}

\hypertarget{garniture-brehan}{%
\subsubsection[ガルニチュール・ブレオン]{\texorpdfstring{ガルニチュール・ブレオン\footnote{このガルニチュールについては、初版から掲載されているにもかかわらず、Bréhanがブルターニュ地方の町の名であることしかわかっていない。ファーヴルにもデュボワ、ベルナール『古典料理』にも言及は見られない。いささか疑問なのは、Bréhanの住人はbréhannaisという語で表わすことから、形容詞も同様であり、garniture
  à la bréhannaise
  (ガルニチュールアラブレアネーズ)の名称でもおかしくないのだが、第二版および第三版ではGarniture
  à la
  Bréhanとなっており、まるで人名のように扱われていることだろう。なお、ブルターニュ地方はアーティチョークの生産で有名だが旬は晩春から初夏にかけてであり、このガルニチュールの構成要素に初版はトリュフのスライスをそら豆のピュレを詰めたアーティチョークの上にのせる指示がある。カリフラワーも基本的には冬の野菜である。それに対してそら豆は乾物であれば1年中、フレッシュのものはやはり晩春から初夏が旬である。レシピには乾物を使うかフレッシュを使うかの指示がないが、「季節感」を演出するためには、フレッシュのそら豆を用いたいところだろう。}}{ガルニチュール・ブレオン}}\label{garniture-brehan}}

\frsub{Garniture Bréhan}

\index{garniture@garniture!brehan@--- Bréhan}
\index{brehan@Bréhan!garuniture ---}
\index{かるにちゆーる@ガルニチュール!ふれおん@---・ブレオン}
\index{ふれおん@ブレオン!かるにちゆーる@ガルニチュール・---}

(牛、仔牛の塊肉の料理に添える)

\begin{itemize}
\item
  小さなアーティチョークの基底部に、そら豆のピュレをドーム状に詰める。
\item
  カリフラワーの小房10個は\protect\hyperlink{sauce-hollandaise}{ソース・オランデーズ}を軽く塗っておく\footnote{茹でてよく水気をきっておくこと}
\item
  小さなじゃがいも10個はバターで火を通し、パセリのみじん切りを振る
\item
  ソース\ldots{}\ldots{}塊肉をブレゼした際の煮汁をソースに仕上げる
\end{itemize}

\hypertarget{garniture-bretonne}{%
\subsubsection{ガルニチュール・ブルターニュ風}\label{garniture-bretonne}}

\frsub{Garniture à la Bretonne}

\index{garniture@garniture!bretonne@--- à la Bretonne}
\index{breton@breton(ne)!garuniture à la ---ne}
\index{かるにちゆーる@ガルニチュール!ふるたーにゆふう@---・ブルターニュ風}
\index{ふるたーにゆふう@ブルターニュ風!かるにちゆーる@ガルニチュール・---}

(羊料理に添える)

\begin{itemize}
\item
  茹でた白いんげん豆またはフラジョレ\footnote{flageolet
    白いんげん豆の一種で、通常のものより小粒。}1
  Lを\protect\hyperlink{sauce-bretonne}{ブルターニュ風ソース}(ブラウン系の派生ソース参照)であえる、パセリのみじん切りを振りかける。
\item
  ソース\ldots{}\ldots{}塊肉の肉汁(ジュ)
\end{itemize}

\hypertarget{garniture-brillat-savarin}{%
\subsubsection[ガルニチュール・ブリヤサヴァラン]{\texorpdfstring{ガルニチュール・ブリヤサヴァラン\footnote{ジャン・アンテルム・ブリア=サヴァラン(Jean
  Anthelme
  Brillat-Savarin)(1755〜1826)。法律家であり、弁護士、一時はアメリカに亡命し、のちに裁判官として活躍したが、とりわけ、はじめ匿名で出版した『美味礼讃』\emph{Physiologie
  du
  Goût}(1825年、タイトルを直訳すれば「味覚の生理学」)で知られる。この著作は食をめぐる考察からなる随筆集だが、必ずしも生真面目な哲学的記述ばかりではない。むしろ「食をめぐる知的な面白読み物」ともいうべき内容であり、のちに「生理学もの」というジャンルが流行する嚆矢となった。これにインスパイアされたバルザックが『結婚の生理学』(1829年)を出版し文筆家バルザックとして最初のヒット作となった。その後に続けとばかりに「○○の生理学」と題した書物が19世紀中頃まで数多く出版された。その多くはほとんど文学的にも省みられることのないもので、「丸わかり○○」あるいは「○○
  のすべて」的なものばかりだった。このため、「生理学もの」のうちで文学史において一般的に価値を認められている作品は『美味礼讃』および『結婚の生理学』くらいしかない。}}{ガルニチュール・ブリヤサヴァラン}}\label{garniture-brillat-savarin}}

\frsub{Garniture Bréhan}

\index{garniture@garniture!brillat-savarin@--- Brillat-Savarin}
\index{brillat-savarin@Brillat-Savarin!garuniture ---}
\index{かるにちゆーる@ガルニチュール!ふりやさうあらん@---・ブリヤサヴァラン}
\index{ふりやさうあらん@ブリヤサヴァラン!かるにちゆーる@ガルニチュール・---}

(鳥類のジビエ料理に添える)

\begin{itemize}
\item
  空焼きしたごく小さなタルトレットに、トリュフを加えた\protect\hyperlink{souffle-de-becasse}{ベカスのスフレ}\footnote{現行版の原書でベカスのスフレの項を見ると、\protect\hyperlink{becasse-favart}{ベカス・ファヴァール}と同じ、とある。なお、ファヴァールFavartというのは劇場の名称で、オペラコミック座が19世紀以来本拠地にしていたが、
    2度の火災に遭い、その度に再建された。19世紀にはイタリアオペラを主な演目とする「イタリア座」(テアトル・イタリアン)が間借りのようなかたちでファヴァール劇場を本拠にしていた時期もある。現在のファヴァール劇場は1898年に再建され、2005年以降国立となったオペラコミック座の本拠地となっている。}のアパレイユをピラミッド形に盛り、提供直前にやや低温のオーブンで焦がさないように火を通す。
\item
  大きなトリュフのスライス。
\item
  ソース\ldots{}\ldots{}このガルニチュールを添える\protect\hyperlink{fonds-de-gibier}{ジビエのフュメ}で作った上等な\protect\hyperlink{sauce-demi-glace}{ソース・ドゥミグラス}
\end{itemize}

\hypertarget{garniture-bristol}{%
\subsubsection[ガルニチュール・ブリストル]{\texorpdfstring{ガルニチュール・ブリストル\footnote{Bristol
  はイギリス西部の港湾都市。このガルニチュールの名称となった由来などは不明。}}{ガルニチュール・ブリストル}}\label{garniture-bristol}}

\frsub{Garniture Bristol}

\index{garniture@garniture!bristol@--- Bristol}
\index{bristol@Bristol!garniture@garuniture ---}
\index{かるにちゆーる@ガルニチュール!ふりすとる@---・ブリストル}
\index{ふりすとる@ブリストル!かるにちゆーる@ガルニチュール・---}

(牛、羊の塊肉料理に添える)

\begin{itemize}
\item
  アプリコットの形状、大きさの\protect\hyperlink{croquettes-de-riz}{米のクロケット}\footnote{本書の「米のクロケット」はアントルメすなわちデザートとして砂糖を加えて甘くつくるレシピであり、そのとおりにすべきかどうかは一考の余地がある。}10個。
\item
  茹でたフラジョレ\footnote{\protect\hyperlink{garniture-bretonne}{ガルニチュール・ブルターニュ風}訳注参照。}
  \(\frac{1}{2}\) Lを\protect\hyperlink{veloute}{ヴルテ}であえる。
\item
  くるみ大の丸い小さなじゃがいも20個はバターで火を通し、溶かした\protect\hyperlink{glace-de-viande}{グラスドヴィアンド}を塗る。
\item
  塊肉をブレゼした煮汁をソースとして仕上げる。
\end{itemize}

\hypertarget{garniture-bluxelloise}{%
\subsubsection[ガルニチュール・ブリュッセル風]{\texorpdfstring{ガルニチュール・ブリュッセル風\footnote{芽キャベツchoux
  de Bruxelles
  とアンディーヴendiveはいずれもベルギーで品種改良、開発された野菜であり、これらを組み合わせてブリュッセル風とするのはいささか安易なようにも思われる。}}{ガルニチュール・ブリュッセル風}}\label{garniture-bluxelloise}}

\frsub{Garniture à la Bruxelloise}

\index{garniture@garniture!bruxelloise@--- à la Bruxelloise}
\index{bruxellois@bruxellois(e)!garniture@garuniture à la ---e}
\index{かるにちゆーる@ガルニチュール!ふりゆつせるふう@---・フリュッセル風}
\index{ふりゆつせるふう@ブリュッセル風!かるにちゆーる@ガルニチュール・---}

(牛、羊の塊肉料理に添える)

\begin{itemize}
\item
  アンディーヴ10個は白さを保つようにしてブレゼする。
\item
  シャトー\footnote{\protect\hyperlink{garniture-bouquetiere}{ガルニチュール・ブクティエール}訳注参照。}に成形したじゃがいも10個。
\item
  芽キャベツ500gは下茹でした後バターで蒸し煮する\footnote{étuver
    (エチュヴェ)。下茹での段階で \(\frac{2}{3}\)〜
    \(\frac{3}{4}\)くらいまで火を通しておくこと。サヴォイキャベツもそうだが、下茹でにはアクを除去する意味もあり、エチュヴェの段階で変色してしまうことがあるため、アクを充分に取り除いてから比較的短時間でエチュヴェするのが望ましい。}。
\item
  ソース\ldots{}\ldots{}やや薄めのマデイラ酒風味の\protect\hyperlink{sauce-demi-glace}{ソース・ドゥミグラス}。
\end{itemize}

\hypertarget{garniture-cancalaise}{%
\subsubsection[ガルニチュール・カンカル風]{\texorpdfstring{ガルニチュール・カンカル風\footnote{ブルターニュ地方の地名Cancale(カンカール)の形容詞
  cancalais(e) (カンカレ/カンカレーズ)。牡蠣の産地として知られ、
  cancaleという牡蠣の品種もある。17世紀、ルイ14世は、ヴェルサイユ宮殿へカンカル産カキを取り寄せていたといわれている。なお、ブルターニュ地方とはいえノルマンディ地方に非常に近い位置にあるため、牡蠣を中心にしたこのガルニチュールにブルターニュの地名を冠し、ノルマンディ風ソースを合わせるのは、一種の洒落とも考えられなくもないが、ブルターニュが言語文化的にフランスにおいてやや異質な歴史を持っていることを考慮すると、無神経な命名ともとられかねない。}}{ガルニチュール・カンカル風}}\label{garniture-cancalaise}}

\frsub{Garniture à la Cancalaise}

\index{garniture@garniture!cancalaise@--- à la Cancalaise}
\index{cancalais@cancalais(e)!garniture@garuniture à la ---e}
\index{かるにちゆーる@ガルニチュール!かんかるふう@---・カンカル風}
\index{かんかるふう@カンカル風!かるにちゆーる@ガルニチュール・---}

(魚料理に添える)

\begin{itemize}
\item
  牡蠣20個の剥き身は、沸騰しない程度の温度の湯で火を通し、周囲をきれいに掃除する。殻を剥いたクルヴェットの尾125g
\item
  ノルマンディ風ソース
\end{itemize}

\hypertarget{garniture-cardinal}{%
\subsubsection[ガルニチュール・カルディナル]{\texorpdfstring{ガルニチュール・カルディナル\footnote{カトリック教会における枢機卿のこと。枢機卿の衣が真紅であることからオマールを用いた料理に付けられた名称とも、オマールが「海の枢機卿」と呼ばれるから、ともいわれている。なお、\ul{à la + 男性名詞}
  の形態は、固有名詞の場合および、対応する女性名詞がない場合にも成立する。これは
  \ul{à la manière de + 名詞} のmanière de
  が省略されたものと解釈される。さらに、料理名において à la
  も省略される傾向にあるため、garuniture Cardinal あるいは garniture
  cardinal という表現も\ul{料理名においては}正しいとされている。}}{ガルニチュール・カルディナル}}\label{garniture-cardinal}}

\frsub{Garniture à la Cardinal}

\index{garniture@garniture!cardinal@--- à la Cardinal}
\index{cardinal@cardinal!garniture@garuniture à la ---}
\index{かるにちゆーる@ガルニチュール!かるていなる@---・カルディナル}
\index{かるていなる@カルディナル!かるにちゆーる@ガルニチュール・---}
\index{すうききよう@枢機卿 ⇒ カルディナル!かるにちゆーる@ガルニチュール・カルディナル}

(魚料理に添える)

\begin{itemize}
\item
  立派なオマールの尾の身をやや斜めに厚さ1cm程度にスライスしたもの10枚。
\item
  真黒なトリュフのスライス10枚。
\item
  さいの目に切ったオマールの身60 gとトリュフ50 g。
\item
  \protect\hyperlink{sauce-cardinal}{ソース・カルディナル}
\end{itemize}

\hypertarget{garniture-castillane}{%
\subsubsection[ガルニチュール・カスティリア風]{\texorpdfstring{ガルニチュール・カスティリア風\footnote{Castilla
  (カスティーリャ、カスティージャ)はスペイン中部の地域で、中世はカスティリア王国だった。「カステラ」の語源ともいわれる。}}{ガルニチュール・カスティリア風}}\label{garniture-castillane}}

\frsub{Garniture à la Castillane}

\index{garniture@garniture!castillane@--- à la Castillane}
\index{castillan@castillan(e)!garniture@garuniture à la ---e}
\index{かるにちゆーる@ガルニチュール!かすていりあふう@---・カスティリア風}
\index{かすていりあふう@カスティリア風!かるにちゆーる@ガルニチュール・---}

(トゥルヌド、ノワゼットに添える)

\begin{itemize}
\item
  \protect\hyperlink{pommes-de-terre-duchesse}{ポム・デュシェス}で作ったた小さなケースにドリュールを塗ってオーブンで焼き色を付ける。そこに、軽くにんにく風味を効かせた\protect\hyperlink{portugaise}{トマトのフォンデュ}を詰める。
\item
  皿の周囲に、輪切りにして塩こしょうし、小麦粉をまぶして油で揚げた玉ねぎを飾る。
\item
  トマト風味を加えたデグラセした肉汁(ジュ)\footnote{トゥルヌド、ノワゼットをフライパンでソテーし、デグラセしてトマトピュレまたは本文にあるトマトのフォンデュを加えてソースにするということ。}
\end{itemize}

\hypertarget{garniture-chambord}{%
\subsubsection[ガルニチュール・シャンボール]{\texorpdfstring{ガルニチュール・シャンボール\footnote{シャンボールとは16世紀、ロワール河の近くに建てられた瀟洒な城の名。このガルニチュールを添えた場合、料理名にシャンボールが冠される。鯉、サーモンが代表的だが、とりわけ19世紀は鯉が好まれ、カレーム『19
  世紀フランス料理』第2巻では鯉のシャンボールだけで近代風、ロヤイヤル、レジャンスの3種の仕立てについて詳述されている(pp.181-189)。なお、このガルニチュールの構成も時代や料理人によって多少の変化があり、『ラルース・ガストロノミック』初版では、魚でつくった大小のクネル、マッシュルーム、舌びらめのフィレ、バターでソテーした白子、オリーヴ形に成形したトリュフ、クールブイヨンで火を通したエクルヴィス、揚げたクルトン、となっている(p.516)。}}{ガルニチュール・シャンボール}}\label{garniture-chambord}}

\frsub{Garniture Chambord}

\index{garniture@garniture!chambord@--- Chambord}
\index{chambord@Chambord!garniture@garuniture ---}
\index{かるにちゆーる@ガルニチュール!しやんほーる@---・シャンボール}
\index{しやんほーる@シャンボール!かるにちゆーる@ガルニチュール・---}

(魚のブレゼの大掛かりな仕立てに添える\footnote{ルルヴェのこと。\protect\hyperlink{releve}{第二版序文訳注}参照。19世紀前半くらいまではカトリックの習慣としての「小斉」が比較的厳格に守られており、料理人たちは四旬節やその他の小斉の日の献立としていかに豪華で美味な魚料理を提供するかに腐心していたのが、17〜18世紀の料理書を読むとよくわかる。カレームの著書にも魚の大掛かりな仕立てのレシピが数多く収められている。})

\begin{itemize}
\item
  トリュフを加えてスプーンで成形した魚のファルスで作ったクネル10個。
\item
  長卵形の大きな、表面に装飾を施したクネル4個。
\item
  渦巻模様を付けた\footnote{原文 canneler (カヌレ)。この場合はtourner
    (トゥルネ)とほぼ同義だが、凹凸の刻み模様を付けた、の意。}小さなマッシュルーム200
  g。
\item
  鯉の白子を1
  cm程度の厚さにスライスして塩こしょうし、小麦粉をまぶしてソテーしたもの10枚。
\item
  オリーブ形に成形した\footnote{tourner
    (トゥルネ)。原義は「回す」。野菜などを包丁ではなく材料を回すようにして皮を剥いたり成形するところから。}トリュフ200
  g。
\item
  エクルヴィス\footnote{ecrevisse ヨーロッパザリガニ。}6尾は\protect\hyperlink{courtbouillon-a}{クールブイヨン}\footnote{court-bouillon
    (クールブイヨン)。court
    は少量の意。つまり、原則としてはできるだけ少量の液体を煮汁として魚介類その他を加熱調理するのに用いる。また、とりわけ魚介類の場合は沸騰しない程度の温度で火を通す(pocher
    ポシェする)のが原則。たんなる水、塩水だけでなく、ワインや香味野菜、香辛料などを加えて風味付け(および場合によっては臭みのマスキング効果)も兼ねて事前に用意しておくこともある。ただしこれらはあくまでも原則論にすぎない。詳細は\protect\hyperlink{poissons}{魚料理}の\protect\hyperlink{serie-de-courts-bouillons-de-poisson}{クールブイヨン}および\protect\hyperlink{ecrevisse-a-la-nage}{エクルヴィス・ナージュ}参照のこと。なお、エクルヴィスの場合は上記の「少量」にあまりこだわらず、後ではさみを背に回しやすくなるように鍋に入れて加熱すればいいだろう。エクルヴィスはジストマ(寄生虫)のリスクがあるためしっかり加熱すること。またエクルヴィスは腕が取れやすいが、その場合でも可食部である尾の身には問題がないので装飾以外の利用はもちろん可能であり、装飾用としてはロス分を見込んで用意しておくのがいいだろう。}で火を通し、はさみを背に回すように成形する\footnote{trousser
    (トゥルセ)。}(しなくてもよい)。
\item
  食パンを鶏のとさかの形に切りバターで揚げたクルトン6枚。
\item
  魚をブレゼした際の煮汁をベースにしたソース。
\end{itemize}

\hypertarget{garniture-chatelaine}{%
\subsubsection[ガルニチュール・シャトレーヌ]{\texorpdfstring{ガルニチュール・シャトレーヌ\footnote{châtelain(e)
  (シャトラン/シャトレーヌ)。城館の主の意。城館に住む者を思わせる豪華な、の意で料理名として使われるようになったようだ。}}{ガルニチュール・シャトレーヌ}}\label{garniture-chatelaine}}

\frsub{Garniture Châtelaine}

\index{garniture@garniture!chatelaine@--- Châtelaine}
\index{chatelaine@Châtelaine!garniture@garuniture ---}
\index{かるにちゆーる@ガルニチュール!しやとれーぬ@---・シャトレーヌ}
\index{しやとれーぬ@シャトレーヌ!かるにちゆーる@ガルニチュール・---}

(牛、羊の塊肉や鶏料理に添える)

\begin{itemize}
\item
  アーティチョークの基底部10個に、固く作った\protect\hyperlink{sauce-soubise}{スビーズ}を詰める。
\item
  殻を剥いて塊肉をブレゼした煮汁で蒸し煮したマロン30個。
\item
  \protect\hyperlink{pommes-de-terre-noisette}{じゃがいものノワゼット}300
  g。
\item
  ブレゼした煮汁を加えた\protect\hyperlink{sauce-madere}{ソース・マデール}
\end{itemize}

\hypertarget{garniture-chipolata}{%
\subsubsection[ガルニチュール・シポラタ]{\texorpdfstring{ガルニチュール・シポラタ\footnote{もとはイタリアで玉ねぎとソーセージを煮込んだ料理(cipollata
  チポッラータ \textless{} cipolla
  チポッラ=玉ねぎ)を意味していたが、フランスに伝わった際に、語本来の意味に含まれていた玉ねぎが脱落して、羊腸に豚挽肉を詰めた小さなソーセージをこう呼ぶようになったといわれている。}}{ガルニチュール・シポラタ}}\label{garniture-chipolata}}

\frsub{Garniture à la Chipolata}

\index{garniture@garniture!chipolata@--- à la Chipolata}
\index{chipolata@chipolata!garniture@garuniture à la ---}
\index{かるにちゆーる@ガルニチュール!しほらた@---・シポラタ}
\index{しほらた@シポラタ!かるにちゆーる@ガルニチュール・---}

(牛、羊の塊肉および鶏料理に添える)

\begin{itemize}
\item
  小玉ねぎ20個は下茹でしてバターで色艶よく炒める\footnote{glacer
    (グラセ)。本文下のにんじんも同様の指示。}。
\item
  シポラタソーセージ10本。
\end{itemize}

コンソメで煮たマロン10個。

塩漬け豚バラ肉125 gはさいの目に切って、強火でこんがり炒める。

\begin{itemize}
\item
  オリーブ形に成形して下茹でし、バターで色艶よく炒めたにんじん20個(なくてもよい)。
\item
  ソース\ldots{}\ldots{}このガルニチュールを添える料理の煮汁を加えた\protect\hyperlink{sauce-demi-glace}{ソース・ドゥミグラス}
\end{itemize}

\hypertarget{garniture-choisy}{%
\subsubsection[ガルニチュール・ショワジー]{\texorpdfstring{ガルニチュール・ショワジー\footnote{パリのセーヌ川上流(=東側)約12
  kmのところにある Choisy-le-Roi
  の地名に由来。17世紀にショワジー城が建てられ、18世紀にこれを相続したルイ15世が狩りの際に使う邸宅として利用し、現在の名称ショワジールロワになった。その後、ポンパドゥール夫人がここに移り住み、豪華な夕食会がしばしば開かれたという。ショワジーの名称はレチュを用いた料理に付けられることが多い。}}{ガルニチュール・ショワジー}}\label{garniture-choisy}}

\frsub{Garniture Choisy}

\index{garniture@garniture!choisy@--- Choisy}
\index{choisy@Choisy!garniture@garuniture ---}
\index{かるにちゆーる@ガルニチュール!しよわしー@---・ショワジー}
\index{しよわしー@ショワジー!かるにちゆーる@ガルニチュール・---}

(トゥルヌドおよびノワゼットに添える)

\begin{itemize}
\item
  半割りにした\protect\hyperlink{laitue-braise}{レチュのブレゼ}10個。
\item
  シャトーに成形した小さなじゃがいも20個。
\item
  ソース\ldots{}\ldots{}バターを加えた\protect\hyperlink{glace-de-viande}{グラスドビアンド}
\end{itemize}

\hypertarget{garniture-choron}{%
\subsubsection[ガルニチュール・ショロン]{\texorpdfstring{ガルニチュール・ショロン\footnote{19世紀にあったパリの有名レストラン、ヴォワザンの料理長の名。\protect\hyperlink{sauce-bearnaise-tomatee}{ソース・ショロン}も参照。}}{ガルニチュール・ショロン}}\label{garniture-choron}}

\frsub{Garniture Choron}

\index{garniture@garniture!choron@--- Choron}
\index{choron@Choron!garniture@garuniture ---}
\index{かるにちゆーる@ガルニチュール!しよろん@---・ショロン}
\index{しよろん@ショロン!かるにちゆーる@ガルニチュール・---}

(トゥルヌドおよびノワゼットに添える)

\begin{itemize}
\item
  中位か小さいアーティチョークの基底部をにバターであえたアスパラガスの穂先を詰める。アスパラガスがなければ、バターであえた小粒のプチポワでもいい。
\item
  \protect\hyperlink{pommes-de-terre-noisette}{じゃがいものノワゼット}30個。
\item
  \protect\hyperlink{sauce-bearnaise-tomatee}{トマト入りソース・ベアルネーズ}。
\end{itemize}

\hypertarget{garniture-clamart}{%
\subsubsection[ガルニチュール・クラマール]{\texorpdfstring{ガルニチュール・クラマール\footnote{パリ郊外の町の名。プチポワを使った料理にこの名が冠されるものがいくつかある。}}{ガルニチュール・クラマール}}\label{garniture-clamart}}

\frsub{Garniture à la Clamart}

\index{garniture@garniture!clamart@--- à la Clamart}
\index{clamart@Clamart!garniture@garuniture à la ---}
\index{かるにちゆーる@ガルニチュール!くらまーる@---・クラマール}
\index{くらまーる@クラマール!かるにちゆーる@ガルニチュール・---}

(牛、羊の塊肉の料理に添える)

\begin{itemize}
\item
  \protect\hyperlink{petits-pois-francaise}{プチポワ・アラフランセーズ}に細かく刻んだレチュの葉を加えてバターであえ、空焼きしたタルトレット10個に詰める。
\item
  \protect\hyperlink{pommes-de-terre-macaire}{じゃがいものマケール}で作った円形の小さな台の上に、タルトレットをひとつずつのせる。
\item
  ソース\ldots{}\ldots{}\protect\hyperlink{jus-de-veau-lie}{とろみを付けたジュ}\footnote{このガルニチュールを添える料理がポワレ(\protect\hyperlink{sauce-bigarade}{ソース・ビガラード}訳注および\protect\hyperlink{releves-et-entrees}{肉料理}参照)の場合には、鍋に残った香味野菜(マティニョン)にフォン少量を注いで風味を引き出し、それにコーンスターチでとろみを付けることになるだろう。}
\end{itemize}

\hypertarget{garniture-de-compote}{%
\subsubsection[ガルニチュール・コンポート]{\texorpdfstring{ガルニチュール・コンポート\footnote{compote
  (コンポート)。果物のシロップ煮のイメージが強いが、肉や野菜をばらばらになるまで煮込んだ料理のことも指す。}}{ガルニチュール・コンポート}}\label{garniture-de-compote}}

\frsub{Garniture de Compote}

\index{garniture@garniture!compote@--- de Compote}
\index{compote@compote!garniture@garuniture de ---}
\index{かるにちゆーる@ガルニチュール!こんほーと@---・コンポート}
\index{こんほーと@コンポート!かるにちゆーる@ガルニチュール・---}

(鳩およびプレ・ド・グラン\footnote{poulet de grain
  鶏の大きさや飼育方法による区別については\protect\hyperlink{sauce-chaud-froid-vert-pre}{ソース・ショフロワ・ヴェールプレ}参照。}に添える)

\begin{itemize}
\item
  塩漬け豚バラ肉\footnote{lard de poitrine
    (ラールドポワトリーヌ)、lard maigre
    (ラールメーグル)あるいは原文のように合わせて lard de poitrine
    maigre
    (ラールドポワトリーヌメーグル)とも呼ぶが、塩漬けにして熟成させた豚バラ肉のこと。通常、lard
    だけの場合は lard gras
    (ラールグラ)すなわち豚背脂のことを意味するので注意。}250は拍子木\footnote{lardon
    (ラルドン)。たんに lardon
    というだけで、この豚バラ肉の塩漬けを拍子木状に切ったものを意味することはごく一般的で、塩漬け後に冷燻にかけた
    lard de poitrine fumé を拍子木に切ったものは lardon fumé
    (ラルドンフュメ)と呼ばれる。}に切り、下茹でしてからバターでこんがり炒める\footnote{rissoler
    (リソレ)油脂を熱して、素材の表面をこんがり焼くこと。語源は中世からある
    rissole
    (リソール)という円形または半円形、塩味または甘い焼き菓子(揚げ菓子)---
    つまり時代や地域とともに非常にバリエーションに富むものだが、こんがりとした色合いに仕上げるのは共通している。}。
\item
  小玉ねぎ300 gは下茹でしてバターで色艶よく炒める\footnote{glacer
    (グラセ)。}。
\item
  小さなマッシュルーム300 gは生のまま2つに切り、バターで炒める。
\item
  これらは鳩とともに火入れを仕上げ、供する際には鳩を覆うようにガルニチュールを盛る。
\end{itemize}

\hypertarget{garniture-conti}{%
\subsubsection[ガルニチュール・コンティ]{\texorpdfstring{ガルニチュール・コンティ\footnote{ブルボン王家のひとつCondé(コンデ)家の傍流(いわゆる分家筋)で、代々のコンティ大公
  le prince de Conti
  (ルプランスドコンティ)がいる。もとはピカルディ地方アミアンの近くにあるContyというところを領地にしていたのが家名の起源。コンティの名は本書でも、レンズ豆のポタージュ、\protect\hyperlink{puree-conti}{ピュレ・コンティ}が収められている。このガルニチュールは18世紀のコンティ大公ルイ・フランソワ・ド・ブルボン(1727〜
  1776)の料理人が考案したと伝えられているが、たとえ事実であったとしても、あまりにシンプルなものなので、ガルニチュールとして供することを考えた、というのがせいぜいのところか。}}{ガルニチュール・コンティ}}\label{garniture-conti}}

\frsub{Garniture Conti}

\index{garniture@garniture!conti@--- Conti}
\index{conti@Conti!garniture@garuniture de ---}
\index{かるにちゆーる@ガルニチュール!こんてい@---・コンティ}
\index{こんてい@コンティ!かるにちゆーる@ガルニチュール・---}

(牛、羊の塊肉のブレゼに添える)

\begin{itemize}
\item
  レンズ豆\footnote{lentille
    (ロンティーユ)。西アジア原産。レンズ豆はおそらく農耕がはじまったごく初期からの作物で、エジプトや地中海沿岸で多く栽培されていた。温暖な気候に向いた作物であり、その意味ではフランス北部に縁があるコンティ大公の名はふさわしくないかも知れない。旧約聖書の「創世記」にも出てくる。アブラハムの息子イサクの双子のうちのひとりエサウはすぐれた狩人に、もうひとりのヤコブは「穏かな人で天幕の周りで働くのを常として」いた。(中略)ある日のこと、ヤコブが煮物をしていると、エサウが疲れきって野原から帰ってきた。エサウはヤコブに言った。『お願いだ、その赤いもの(アドム)、そこの赤いものを食べさせてほしい。わたしは疲れきっているんだ。』(中略)エサウは誓い、長子の権利をヤコブに譲ってしまった。ヤコブはエサウにパンとレンズ豆の煮物を与えた。(中略)こうしてエサウは長子の権利を軽んじた。(「創世記」
    25-27〜34、新共同訳『聖書』)」。これを踏まえると、Condé
    すなわちコンデ大公の名を冠した料理、とりわけポタージュ\protect\hyperlink{puree-conde}{ピュレ・コンデ}が赤いんげん豆のポタージュであることと、レンズ豆を主素材とした「ガルニチュール・コンティ」およびポタージュ「ピュレ・コンティ」の関係には考えさせられるところがある。}のピュレ750
  g。
\item
  脂身のほとんどない豚バラ肉の塩漬け250
  gは長方形に切って、レンズ豆を煮る際に一緒に煮ておく。
\item
  ブレゼの煮汁をソースとして仕上げて添える。
\end{itemize}

\hypertarget{garniture-a-la-commodore}{%
\subsubsection[ガルニチュール・コモドール]{\texorpdfstring{ガルニチュール・コモドール\footnote{もとは英語
  commodore であり、イギリスでは艦隊司令官、アメリカでは准将の意。}}{ガルニチュール・コモドール}}\label{garniture-a-la-commodore}}

\frsub{Garniture à la Commodore}

\index{garniture@garniture!commodore@--- à la Commodore}
\index{commodore@commodore!garniture@garuniture de ---}
\index{かるにちゆーる@ガルニチュール!こもとーる@---・コモドール}
\index{こもとーる@コモドール!かるにちゆーる@ガルニチュール・---}

(魚の大掛かりな仕立てに添える)

\begin{itemize}
\item
  エクルヴィスの尾の身を入れた小さなグラタン皿10個。
\item
  メルラン\footnote{merlan (メルロン)タラ科の海水魚。}の\protect\hyperlink{farce-a}{ファルス}に\protect\hyperlink{beurre-d-ecrevissse}{エクルヴィスバター}を加え、スプーンで成形したクネル10個。
\item
  大きな\protect\hyperlink{moules-a-la-villeroy}{ムール貝のヴィルロワ}10個。
\item
  仕上げにエクルヴィスバターを加えた\protect\hyperlink{sauce-normande}{ノルマンディ風ソース}。
\end{itemize}

\hypertarget{garniture-cussy}{%
\subsubsection[ガルニチュール・キュシー]{\texorpdfstring{ガルニチュール・キュシー\footnote{キュシー侯爵(1767〜1841)。\protect\hyperlink{osbservation-sur-la-sauce}{基本ソース 概説}訳注参照。}}{ガルニチュール・キュシー}}\label{garniture-cussy}}

\frsub{Garniture Cussy}

\index{garniture@garniture!cussy@--- Cussuy}
\index{cussy@Cussy (marquis de)!garniture@garuniture ---}
\index{かるにちゆーる@ガルニチュール!きゆしー@---・キュシー}
\index{きゆしー@キュシー!かるにちゆーる@ガルニチュール・---}

(トゥルヌド、ノワゼット、鶏料理に添える)

\begin{itemize}
\item
  マロンのピュレを詰めてグリル焼きした大きなマッシュルーム10個。
\item
  完全に球形に成形し、マデイラ酒風味で火を通した小さなトリュフ10個。
\item
  大きな雄鶏のロニョン\footnote{rognon
    仔牛などでは腎臓のこと。鶏の場合はrognon
    blanc(ロニョンブロン)とも呼び、精巣のこと。この場合は後者。もちろんきちんと加熱調理したものをガルニチュールの構成要素とする。}20個。
\item
  \protect\hyperlink{sauce-madere}{ソース・マデール}
\end{itemize}

\hypertarget{garniture-Daumont}{%
\subsubsection[ガルニチュール・ドモン]{\texorpdfstring{ガルニチュール・ドモン\footnote{ドモン公爵家duché
  d'Aumon(デュシェドモン)にちなんだ名称をいわれている。}}{ガルニチュール・ドモン}}\label{garniture-Daumont}}

\frsub{Garniture Daumont}

\index{garniture@garniture!daumont@--- Daumont}
\index{daumont@Daumont!garniture@garuniture ---}
\index{かるにちゆーる@ガルニチュール!ともん@---・ドモン}
\index{ともん@ドモン!かるにちゆーる@ガルニチュール・---}

(魚料理に添える)

\begin{itemize}
\item
  バターで鍋に蓋をして弱火で火を通した\footnote{étuver au beurre
    (エチュヴェオブール)}マッシュルーム10個に、それぞれエクルヴィスの尾の身を半分に切ったもの6枚ずつ添える。
\item
  \protect\hyperlink{farce-c}{生クリーム入り魚のファルス}を小さな球形にし、トリュフで装飾を施したクネル10個。
\item
  厚さ1cm程にスライスした\footnote{escalope (エスカロップ)肉や魚を1〜2
    cmの厚さで、筋腺維と直角にスライスした円形または楕円形にスライスしたもの。}白子10枚はイギリス式パン粉衣\footnote{paner
    à l'anglaise
    (パネアロングレーズ)。素材に小麦粉をまぶしてから、卵液に浸し、細かいパン粉で衣を付けること。日本でフライなどをつくる際に一般的な方法とよく似ているが、日本では粗いパン粉が好まれるのに対して、フランスやイギリスでは細かいパン粉を使うのが一般的。}を付けて油で揚げる。
\item
  \protect\hyperlink{sauce-nantua}{ソース・ナンチュア}
\end{itemize}

\hypertarget{garniture-a-la-dauphine}{%
\subsubsection[ガルニチュール・ドフィーヌ]{\texorpdfstring{ガルニチュール・ドフィーヌ\footnote{à
  la Dauphine
  (アラドフィーヌ)王太子妃風、の意。この料理名には由来や理由がないことがほとんど。あえていえば「豪華」であるという程度だが、存外、簡素な仕立ての料理にも付けられることがある。}}{ガルニチュール・ドフィーヌ}}\label{garniture-a-la-dauphine}}

\frsub{Garniture à la Dauphine}

\index{garniture@garniture!dauphine@--- à la Dauphine}
\index{dauphin@dauphin(e)!garniture@garuniture ---e}
\index{かるにちゆーる@ガルニチュール!とふいぬ@---・ドフィーヌ}
\index{とふあん@ドファン/ドフィーヌ!かるにちゆーる@ガルニチュール・ドフィーヌ}

(牛、羊の塊肉の料理に添える)

\begin{itemize}
\item
  \protect\hyperlink{pomme-de-terres-dauphine}{じゃがいものドフィーヌ}をアパレイユにした\protect\hyperlink{croquettes}{クロケット}20個。大きな塊肉料理に添える場合はコルクの栓の形状に、トゥルヌドやノワゼットに添えるときは平たい円盤の形にする。
\item
  マデイラ酒風味の\protect\hyperlink{sauce-demi-glace}{ソース・ドゥミグラス}。
\end{itemize}

\hypertarget{garniture-a-la-dieppoise}{%
\subsubsection[ガルニチュール・ディエープ風]{\texorpdfstring{ガルニチュール・ディエープ風\footnote{dieppois(e)
  (ディエポワ/ディエポワーズ) \textless{} Dieppe
  (ディエープ)ノルマンディ地方の港町の名。}}{ガルニチュール・ディエープ風}}\label{garniture-a-la-dieppoise}}

\frsub{Garniture à la Dieppoise}

\index{garniture@garniture!dieppoise@--- à la Dieppoise}
\index{dieppois@dieppois(e)!garniture@garuniture ---e}
\index{かるにちゆーる@ガルニチュール!ていえーふふう@---・ディエープ風}
\index{ていえーふふう@ディエープ風!かるにちゆーる@ガルニチュール・---}

(魚料理に添える)

\begin{itemize}
\item
  殻を剥いたクルヴェット\footnote{小海老。小さめのcrevette
    grise(クルヴェットグリーズ)とやや大きめのcrevette
    rose(クルヴェットローズ)が代表的。}の尾の身100g。
\item
  \(\frac{3}{4}\)
  L(約30個)のムール貝は白ワインを加えた湯で沸騰させない程度の温度で火を通し\footnote{pocher
    (ポシェ)}、周囲をきれいに掃除する\footnote{ébarber
    (エバルベ)貝類の身の周囲をきれいにすること。帆立貝の場合は「ひもを取る」ともいう。}。
\item
  このガルニチュールを添える魚の煮汁を煮詰めて加えた\protect\hyperlink{sauce-vin-blanc}{白ワインソース}。
\end{itemize}

\hypertarget{garniture-doria}{%
\subsubsection[ガルニチュール・ドリア]{\texorpdfstring{ガルニチュール・ドリア\footnote{原書現行版ではDorlaとなっているが初版〜第三版はDoria。19世紀パリのカフェ・アングレの顧客として知られていた名家ドリアの名を冠したといわれている。このドリア家は12世紀ジェノヴァの
  de Auria (ラテン語の filiis
  Auriaeすなわちアウリアの子孫の意)から発する由緒ある家系として有名。なお、日本の洋食のドリアは1930年頃横浜ホテルニューグランド総料理長サリー・ワイルが発案したものといわれており、上記のドリア家とはまったく関係がない。また古代ギリシア時代の民族ドーリア人とも関係がない。ちなみに、バルザックの小説『幻滅』におなじ発音の名の
  Dauriatという登場人物がいる。}}{ガルニチュール・ドリア}}\label{garniture-doria}}

\frsub{Garniture Doria}

\index{garniture@garniture!doria@--- Doria}
\index{doria@Doria!garniture@garuniture ---e}
\index{かるにちゆーる@ガルニチュール!とりあ@---・ドリア}
\index{とりあ@ドリア!かるにちゆーる@ガルニチュール・---}

(魚料理に添える)

\begin{itemize}
\item
  オリーブ形に剥いたきゅうり\footnote{concombre
    (コンコンブル)日本で一般的なきゅうりと品種系統も異なるものが多く、サイズも太さ4〜5
    cm、長さ30〜45
    cmで収穫する(品種によって異なる)。青臭さがなく、加熱調理することが多い。}30個をバターで蒸し煮する\footnote{étuver
    (エチュヴェ)。}。
\item
  表皮を剥いて種を取り除いたレモンのスライスを魚の上に並べる。魚は\protect\hyperlink{meuniere}{ムニエル}にしたもの。
\end{itemize}

\hypertarget{garniture-dubarry}{%
\subsubsection[ガルニチュール・デュバリー]{\texorpdfstring{ガルニチュール・デュバリー\footnote{Madame
  du Barry (マダムデュバリー)デュバリー夫人
  (1743〜1793)のこと。ルイ15世の公妾であり、フランス革命により断頭台に送られ命を落したことで知られる。もとはシャンパーニュ地方の貧しい家庭の生まれ。パリに出てのち「お針子」などの仕事や娼婦をしていたが、デュ・バリー子爵に囲われ、いわゆるdemi-mondaine(ドゥミモンデーヌ)、
  courtisane
  (クルティザーヌ)すなわち高等娼婦として知られるようになる。その後、ポンパドゥール夫人を亡くしたルイ15世が彼女を妾にすることにし、形式上、デュ・バリー子爵の弟と結婚したことにして、正式な社交界デビューを果たした。フランス史において「女性」であることを最大限利用して社会的にのしあがった典型例のひとつ。フランス革命のさなか、捕えられて断頭台へ連れていかれる際に、ほとんどの貴族の女性が取り乱さず凛として死に臨んだのに対し、デュバリー夫人ただひとりだけが狂乱し泣き叫んで命乞いした、という逸話が残っている。ただし、それはロベスピエールによる恐怖政治への警鐘になり得たという見解も少なくない。}}{ガルニチュール・デュバリー}}\label{garniture-dubarry}}

\frsub{Garniture Dubarry}

\index{garniture@garniture!dubarry@--- Dubarry}
\index{dubarry@Dubarry!garniture@garuniture ---}
\index{かるにちゆーる@ガルニチュール!とゆはりー@---・デュバリー}
\index{とゆはりー@デュバリー!かるにちゆーる@ガルニチュール・---}

(牛、羊の塊肉やノワゼット、トゥルヌドに添える)

\begin{itemize}
\item
  小さく分けたカリフラワーの花房を小さなボウルに詰め半球形にまとめて裏返し\protect\hyperlink{sauce-mornay}{ソース・モルネー}で覆ったもの10個。おろした
  \footnote{râper (ラペ)}チーズを振りかけて高温のオーブンでこんがり焼く\footnote{gratiner
    (グラティネ)。また、原文moulés en
    boulesを文字通りに読むと「完全な球形」にするようにも解釈出来なくはないが、そのためには強力な「つなぎ」が必要になる。ソース・モルネー以外に「つなぎ」の役割を果たすものの指定がないため、これでは球形を維持する「つなぎ」として熱に弱過ぎるだろう。ここは\protect\hyperlink{chou-fleur-au-gratin}{カリフラワーのグラタン}にあるように
    moulé dans un bol
    「ボウルに詰める」と同様と解釈していいと思われる。。}。
\item
  \protect\hyperlink{pommes-de-terre-fondantes}{じゃがいものフォンダント}10個。
\item
  塊肉をブレゼあるいはポワレした際のフォン、もしくはノワゼットやトゥルヌドをソテした後にデグラセしてソースに仕上げる。
\end{itemize}

\hypertarget{garniture-a-la-duchesse}{%
\subsubsection[ガルニチュール・デュシェス]{\texorpdfstring{ガルニチュール・デュシェス\footnote{duc
  (デュック=公爵)、duchesse(デュシェス=公爵夫人)。ここではたんに、\protect\hyperlink{pommes-de-terre-duchesse}{じゃがいものデュシェス}を用いるからこの名称になっているが、デュシェスそれ自体にも料理名としての由来や根拠はほとんどない。}}{ガルニチュール・デュシェス}}\label{garniture-a-la-duchesse}}

\frsub{Garniture à la Duchesse}

\index{garniture@garniture!duchesse@--- à la Duchesse}
\index{duc@duc / duchesse!garniture@garuniture à la Duchesse}
\index{かるにちゆーる@ガルニチュール!てゆせす@---・デュシェス}
\index{てゆしえす@デュシェス!かるにちゆーる@ガルニチュール・---}

(牛、羊の塊肉の料理やノワゼット、トゥルヌドに添える)

\begin{itemize}
\item
  じゃがいものデュシェスを舟形または円盤状かブリオシュ型に詰めて成形し、溶き卵\footnote{dorer
    (ドレ)\textless{} dorure
    (ドリュール)焼いた際に艶を出すために塗る溶き卵。水や牛乳などを混ぜることもある。}を塗って、提供直前にオーブンでこんがり焼いたもの20個。
\item
  \protect\hyperlink{sauce-madere}{ソース・マデール}
\end{itemize}

\hypertarget{garniture-a-la-favorite}{%
\subsubsection[ガルニチュール・ラファヴォリータ]{\texorpdfstring{ガルニチュール・ラファヴォリータ\footnote{『愛の妙薬』や『ランメルモールのルチア』で知られる作曲家ガエタノ・ドニゼッティ(1797〜1848)のグランドオペラ\emph{La
  Favorite}(1840
  年初演)にあやかって付けられた名称。グランドオペラ(grand opéra
  グロントペラ、複数形grands
  opérasグロンゾペラ)とは19世紀前半から中葉にかけて、パリのオペラ座において、豪華な舞台装置と派手な演出、大編成のオーケストラ、歴史的題材などをテーマとしたわかりやすい悲劇的筋書きなどを特徴としたオペラ作品の様式のこと。ジャコモ・マイアベーア『悪魔ロベール』(1831年)や『ユグノー教徒』(1836年)などが代表的。なお、ロッシーニはこの様式が流行る前のオペラ作曲家と位置付けられていることが多いが、『ウィリアム・テル』(1829年。フランス語原題
  \emph{Guillaume
  Tell}ギヨーム・テル)あるいはそれに先立つ1827年の『モーセとファラオン』をこのジャンルの嚆矢と見なす場合もある。その他の代表的なグランドオペラの作曲家にダニエル=フランソワ・オーベール(1782〜
  1871)やジャック=フロマンタル・アレヴィ(1799〜1862)がいる。ドニゼッティのこの作品もロッシーニやマイアベーアの諸作品同様、フランス語の台本、歌詞であり、原題もフランス語で
  \emph{La
  Favorite}(ラファヴォリット)だが、どういうわけか、こんにちの日本ではイタリア語式に直した『ラファヴォリータ』と呼ばれることが多いためにここではそれに合わせた。なおこのオペラのリブレット(台本、歌詞)はアルフォンス・ロワイエとギュスターヴ・ヴァエズによるものだが、18世紀バキュラール・ダルノー(1718〜1805)の戯曲『不幸な恋人たち』を原作としている。さらにいえば、バキュラール・ダルノーの戯曲もまた、クロディーヌ・ゲラン・ド・トンサン(1682〜1749)の小説『コマンジュ伯爵の手記』を翻案したもの。『ロメオとジュリエット』の物語のバリエーションのひとつともいえるこの小説は18世紀に大きな反響を呼び、多くの小説、戯曲に影響を与えた。ダルノーの戯曲はその代表例。}}{ガルニチュール・ラファヴォリータ}}\label{garniture-a-la-favorite}}

\frsub{Garniture à la Favorite}

\index{garniture@garniture!favorite@--- à la Favorite}
\index{favorite@Favorige (La)!garniture@garuniture à la Favorite}
\index{かるにちゆーる@ガルニチュール!らふあうおりーた@---・ラファヴォリータ}
\index{らふあうおりーた@ラファヴォリータ!かるにちゆーる@ガルニチュール・---}

(ノワゼット、トゥルヌドに添える)

\begin{itemize}
\item
  小さめのフォワグラを厚さ1 cm程にスライス\footnote{éscalope
    (エスカロップ)。}し、塩こしょうしてから小麦粉をまぶしてバターでソテーしたもの10枚。
\item
  大きなトリュフのスライスをソテーしたフォラグラに1枚ずつのせる。
\item
  アスパラガスの穂先を束にしたもの。
\item
  ソース\ldots{}\ldots{}\protect\hyperlink{jus-de-veau-lie}{とろみを付けたジュ}。
\end{itemize}

\hypertarget{garniture-a-la-fermiere}{%
\subsubsection[ガルニチュール・フェルミエール]{\texorpdfstring{ガルニチュール・フェルミエール\footnote{日本語にすれば「農場主風」。野菜を厚さ1
  mmくらい、長さ1 cm程度の四角形に切ることを détailler en paysanne
  (デタイエオンペイザーヌ)というが、そのpaysanneとはpaysan(ペイゾン=農民)の女性形であり、このガルニチュールでは野菜をすべてそのように切るところにかけての名称。}}{ガルニチュール・フェルミエール}}\label{garniture-a-la-fermiere}}

\frsub{Garniture à la Fernière}

\index{garniture@garniture!fermiere@--- à la Fermière}
\index{fermier@fermier/fermière!garniture@garuniture à la Fermière}
\index{かるにちゆーる@ガルニチュール!ふえるみえーる@---・フェルミエール}
\index{ふえるみえ@フェルミエ/フェルミエール!かるにちゆーる@ガルニチュール・フェルミエール}

(鶏料理に添える)

\begin{itemize}
\item
  にんじん150 gと蕪150 gは厚さ1 mm程度、長さ1
  cm程度の四角形に切る\footnote{原文émincer en paysanne
    (エマンセオンペイザーヌ)。ペイザーヌに切る場合、動詞にはémincer
    薄くスライスする、も使われる。}。
\item
  玉ねぎ50 gとセロリ50 gも同様に切る。
\item
  これらを鍋に入れてバターと、塩3g、粉砂糖5
  gを加えて蓋をして弱火で軽く蒸し煮する\footnote{étuver (エチュヴェ)。}。
\item
  野菜を鶏の周囲に盛り、野菜の火入れを仕上げる\footnote{このガルニチュールは\protect\hyperlink{poulet-saute-a-la-fermiere}{鶏のソテー・フェルミエール}に添えるという前提がある。表面に焼き色を付けた鶏を、あらかじめ軽く蒸し煮しておいたこのガルニチュール・フェルミエールとともに陶製の鍋に入れて、さいの目に切ったハムを加え、蓋をしてオーブンに入れて鶏と野菜の火入れを仕上げることになる。}。
\end{itemize}

\hypertarget{garniture-a-la-financiere}{%
\subsubsection[ガルニチュール・フィナンシエール]{\texorpdfstring{ガルニチュール・フィナンシエール\footnote{徴税官風の意。名称について詳しくは\protect\hyperlink{sauce-financiere}{ソース・フィナンシエール}訳注参照のこと。なお、カレームは『19世紀フランス料理』で、多少の違いはあるが、これらの具材とソースを合わせることで「ラグー・フィナンシエール」と呼んでいる(t.3,
  pp.146-148)。これはつまり、ソース・フィナンシエールの訳注でも述べたように、もともとはガルニチュールとソースが別々のものではなく、一体化したものとして調理されていたことを示唆している。実際、このフィナンシエールという料理名の初出と思われる1755年ムノン『宮廷の晩餐』第2巻「肥鶏・フィナンシエール」および第3巻「鯉・フィナンシエール」は19世紀のものと内容、素材は違えどラグーとして扱われている。前者は肥鶏を掃除して中抜きした後、背から開いて骨を取り除き、大きなトリュフ4個とフォワグラとマッシュルーム、おろした豚背脂、卵黄、粒こしょう、バジルの粉末を混ぜて詰める。これを豚背脂のシートで包んで鍋に入れ、液体は注がずに熱い灰の上に鍋を置く。加熱していると肉汁などが出てくる。そこにビガラード(南フランスのビターオレンジ)の搾り汁と塩、こしょうで調味する(p.280)。後者は大きな鯉を掃除し、舌を残すようにしてエラは取り除く。片側の包丁を入れていない面の皮を剥がし、細かく刻んだ豚背脂を表面に刺す。鯉の中に詰めるラグーを作る。仔牛胸腺肉、トリュフ、フォワグラ、マッシュルームを鍋にバター1片とともに入れ、パセリ、シブール、にんにく、クローブ、バジルのブーケガルニを加える。鍋を火にかけて、小麦粉を振りかけ、シャンパーニュをグラス2杯注ぐ。塩こしょうで調味し、具材に火を通す。浮き脂を取り除き、冷めたら鯉の腹に詰め、ラグーが出てこないようしっかり鯉の腹を縫う。鯉の大きさにぴったりの魚用鍋に、ハムのスライスとたっぷりの仔牛腿肉のスライスを敷き、その上に鯉をのせる。豚背脂のシートで多い、玉ねぎのスライス、根菜の皮、パセリ、シブール、にんにく、クローブ、タイム、ローリエの葉、バルジのブーケガルニを入れる。中火にかけて汗をかかせるイメージで少し火を通し、それから上等のブイヨンとシャンパーニュを同量ずつ、鯉が液体に浸るまで注ぐ。塩こしょう。弱火にかける。火が通ったら鍋から鯉を取り出して水気をきる。縫った糸を取り除き、仔牛のグラスを塗って艶を出す。周囲にはお好みでエクルヴィスやまるごとのトリュフ、大きな鶏のとさか、鶏の胸肉、ペルドロー、こんがり焼いた鳩などをセンスよく配する。ソース・エスパニョルを添えて供する(pp.43-44)。その後、19世紀になるとヴィアール『帝国料理の本』1806年において「鳩・フィナンシエール」のレシピが掲載される。概要は、鳩6羽をバター、塩こしょう、レモン果汁を入れた鍋でさっと表面を色付けないように焼き固めたら、豚背脂で包んで鍋に入れ、ポワル(ここではソースの一種と考えていい)を注ぎ、柔らかく火を通す。提供直前に水気をきり、皿の周囲に鳩を配置する。その中央に雄鶏のとさかとロニョン、フォワグラ、トリュフのラグーを流し入れる
  (p.332)。残念ながらヴィアールの1806、1807年版は不完全なもののため、このラグーのレシピそのものは1820の第10版でようやく掲載に至る。概要は、マッシュルーム大24個、ボール形にしたトリュフ24個は辛口のマデイラ酒\(\frac{1}{2}\)瓶とともに鍋に入れ、唐辛子2本、トマト少々、仔牛のグラス1オンスを加えて火にかけてほとんどシロップ状になるまで煮詰める。それからソース・エスパニョルをレードル4杯、仔牛のブロンド(ソース)スプーン2杯を注いでよく混ぜる。沸騰させたら火の弱い場所に移して浮き脂を取り除き、煮詰めていく。このソースを布で漉し、きれいな鍋にマッシュルームとトリュフを移し入れて漉したソースを注ぐ。ここに雄鶏のとさかとロニョン24個ずつ、スプーンで成形したクネル24個、仔羊または仔牛の胸腺肉のスライス24枚を入れる(p.67)。この時点つまり1820年頃には、カレームが記した「ラグー・フィナンシエール」とほぼ同じ内容になっていることが注目されよう。カレームの「ラグー・フィナンシエール」は、トリュフ500
  gを円く成形し、マデイラ酒を加えて10分間弱火で蒸し煮する。ここにソース・フィナンシエールを注ぐ。ひと煮立ちさせたら、マデイラ酒から出たアクを取り除き、マッシュルーム12個、鶏のとさか12個(マデイラ酒少々を煮立たせて火を通しておく)、雄鶏のロニョン
  12個を加える。ひと煮立ちさせたらバター少々、鶏のクネル、フォラグラの1〜2
  cmのスライス、仔羊胸腺肉を加える。このラグーの半量はこれを添える料理上に盛り、周囲に白い立派な鶏のとさかとロニョンを配する。ソースが皿の上の料理をのせる台からはみ出ないようにしないと優美さが失なわれ、皿の縁飾りが乱れてしまうことに注意。ラグーの残りはソース入れで別添で供する(pp.146-147)。カレームはもうひとつ、「フォワグラのラグー・フィナンシエール」というレシピも残している(pp.147-148)。デュボワ、ベルナール共著『古典料理』(1868年)では「ガルニチュール・フィナンシエール」の項目は見られず、「サルピコン・フィナンシエ」
  (p.65)\ldots{}\ldots{}
  これは黒トリュフ、鶏胸肉、赤く漬けた舌肉、マッシュルームに火を通して小さなさいの目に切り、茶色いソース・フィナンシエールであえたもの、となっている。この他、「仔牛の耳・フィナンシエール
  (p.167)」「ほろほろ鳥のフィレ・フィナンシエール(p.179)」「ベカシーヌのグラタン・フィナンシエール(p.187)」「雉のクネル・フィナンシエール(p.190)」「温製パイ包み焼き・フィナンシエール(p.210)」「うずら・フィナンシエール(p.228)」「鳩・フィナンシエール(p.223)」といったレシピが収録されている。これらのレシピを見ると「ガルニチュール・フィナンシエール」を用いる指示になっているものがほとんどのため「ガルニチュール・フィナンシエール」の項が抜けているのは執筆あるいは何らかのミスによるものに過ぎないだろうと思われる。}}{ガルニチュール・フィナンシエール}}\label{garniture-a-la-financiere}}

\frsub{Garniture à la Financière}

\index{garniture@garniture!financiere@--- à la Financière}
\index{financier@financier/financère!garniture@garuniture à la Financière}
\index{かるにちゆーる@ガルニチュール!ふいなんしえーる@---・フィナンシエール}
\index{ふいなんしえ@フィナンシエ/フィナンシエール!かるにちゆーる@ガルニチュール・フィナンシエール}

(牛、羊の塊肉あるいは鶏料理に添える)

\begin{itemize}
\item
  仔牛か鶏のファルスでつくった標準的なクネル20個。ファルスに仔牛を使うか鶏を使うかは、このガルニチュールを添える料理に合わせて決めること。
\item
  渦巻状の刻み模様を入れた小さなマッシュルーム150 g。
\item
  雄鶏のとさかとロニョン\footnote{rognonは通常「腎臓」を指すが、雄鶏の場合はrognon
    blanc(ロニョンブロン)=
    testicule(テスティキュル)すなわち精巣のこと。}100 g。
\item
  トリュフのスライス50 g。
\item
  皮を剥いて下茹でしたオリーブ12個。
\item
  \protect\hyperlink{sauce-financiere}{ソース・フィナンシエール}
\end{itemize}

\hypertarget{garniture-a-la-flamande}{%
\subsubsection[ガルニチュール・フランドル風]{\texorpdfstring{ガルニチュール・フランドル\footnote{flamand(e)
  (フラモン/フラモンド) \textless{} Flandre
  (フロンドル)フランドル地方 =
  現在のベルギー西部からフランス北部にかけての北海に面する地域。フランダース。ただし『フランダースの犬』はイギリスの児童文学なので、フランスおよびベルギーではあまり知られていない。}風}{ガルニチュール・フランドル風}}\label{garniture-a-la-flamande}}

\frsub{Garniture à la Flamande}

\index{garniture@garniture!flamande@--- à la Flamande}
\index{flamand@flamand(e)!garniture@garuniture à la ---e}
\index{かるにちゆーる@ガルニチュール!ふらんとるふう@---・フランドル風}
\index{ふらんとるふう@フランドル風!かるにちゆーる@ガルニチュール・---}

\begin{itemize}
\item
  球形に成形した小さな\protect\hyperlink{chou-braise}{サヴォイキャベツのブレゼ}10個。
\item
  オリーブ形に成形し、コンソメで煮たにんじんと蕪、各10個ずつ。
\item
  \protect\hyperlink{pommes-de-terres-a-l-anglaise}{じゃがいものアラングレーズ}\footnote{à
    l'anglaise
    イギリス風、の意だが、必ずしもイギリス料理に由来するとは限らない。野菜の場合、アラングレーズとはすなわち「塩を加えた湯で茹でる」ことを意味するが、本書の該当個所にも、イギリスでは塩を加えない、とある。}小10個。
\item
  塩漬け豚バラ肉250
  gは10枚の長方形の板状に切り、キャベツとともにブレゼする。
\item
  輪切りにしたソシソン\footnote{熟成、乾燥させてつくる太いソーセージ。多くの場合、調理せず薄切りにして食べる。}10枚(150
  g)。
\item
  塊肉をブレゼした煮汁\footnote{このガルニチュールは\protect\hyperlink{piece-de-boeuf-a-la-flammande}{牛塊肉・フランドル風}に添えるのを前提に書かれているため、ブレゼと特定出来るが、本書における\protect\hyperlink{les-poeles}{ポワレ}の手法でももちろん可能だろう。}をソースに仕上げる。
\end{itemize}

\hypertarget{garniture-a-la-florentine}{%
\subsubsection[ガルニチュール・フィレンツェ風]{\texorpdfstring{ガルニチュール・フィレンツェ風\footnote{florentin(e)
  (フロロンタン/フロロンティーヌ)\textless{} Florence
  (フロロンス)フィレンツェのこと。}}{ガルニチュール・フィレンツェ風}}\label{garniture-a-la-florentine}}

\frsub{Garniture à la Florentine}

\index{garniture@garniture!florentine@--- à la Florentine}
\index{florentin@florentin(e)!garniture@garuniture à la ---e}
\index{かるにちゆーる@ガルニチュール!ふいれんつえふう@---・フィレンツェ風}
\index{ふいれんつえふう@フィレンツェ風!かるにちゆーる@ガルニチュール・---}

(魚料理に添える場合)

\begin{itemize}
\item
  ほうれんそうの葉250 gは下茹でしてから、バターで蒸し煮する\footnote{étuver
    au beurre (エチュヴェオブール)}。
\item
  このほうれんそうを皿の底に敷き、その上に煮立たせないように茹で\footnote{pocher
    (ポシェ)。}
  て火を通した魚をのせ、\protect\hyperlink{sauce-mornay}{ソース・モルネー}を覆いかける。高温のオーブンに入れて焼き色を付ける。
\end{itemize}

(牛、羊の塊肉の料理に添える場合)

\begin{itemize}
\item
  \protect\hyperlink{sucric-d-epinards}{ほうれんそうのシュブリック}10個。
\item
  セモリナ粉を獣脂で加熱し、卵とおろしたチーズを混ぜ込んだアパレイユで円盤状につくった小さな\protect\hyperlink{croquettes}{クロケット}10個。
\item
  トマト風味を効かせ、ただしよく澄んだ状態の\protect\hyperlink{sauce-demi-glace}{ソース・ドゥミグラス}。
\end{itemize}

\hypertarget{garniture-Florian}{%
\subsubsection[ガルニチュール・フローリアン]{\texorpdfstring{ガルニチュール・フローリアン\footnote{ヴェネツィア、サンマルコ広場にある18世紀からあるカフェ。}}{ガルニチュール・フローリアン}}\label{garniture-Florian}}

\frsub{Garniture Florian}

\index{garniture@garniture!florian@--- Florian}
\index{florian@Florian!garniture@garuniture ---}
\index{かるにちゆーる@ガルニチュール!ふろーりあん@---・フローリアン}
\index{ふろーりあん@フローリアン!かるにちゆーる@ガルニチュール・---}

(乳呑仔羊\footnote{本書で
  agneauという場合には、いわゆるプレサレ(agneau de pré-salé
  アニョドプレサレ)はmouton(ムトン=羊の成獣)に準ずる扱いであり、それ以外は基本的にagneau
  de lait (アニョドレ)乳呑仔羊を指すことに留意。}の料理に添える\footnote{\protect\hyperlink{epaule-d-agneau-florian}{乳呑仔羊肩肉・フローリアン}に添えるのを前提としたガルニチュールであることに留意。。})

\begin{itemize}
\item
  大きめのレチュ3個は四つ割りにして外葉を取り除き、ブレゼする\footnote{\protect\hyperlink{laitue-braisee}{レチュのブレゼ}参照。}。
\item
  オリーブの大きさと形にしたにんじん20個は下茹でしてバターで色艶よく火を通す\footnote{glacer
    (グラセ)。これらの野菜の場合は下茹でして半ば火を通しておくことと、必要に応じて砂糖を加える場合があることに留意。}。
\item
  小玉ねぎ20個は下茹でしてバターで色艶よく炒める。
\item
  小さな\protect\hyperlink{pommes-de-terre-fondantes}{じゃがいものフォンダント}10個。
\item
  ソース\ldots{}\ldots{}仔羊の肉汁\footnote{原文ではfondsとなっているが、前提となっている仕立て「乳呑仔羊肩肉・フローリアン」の場合はバターをかけながらローストするので、いわゆる「ジュ」と考えていい。}。
\end{itemize}

\hypertarget{garniture-a-la-Forestiere}{%
\subsubsection[ガルニチュール・フォレスティエール]{\texorpdfstring{ガルニチュール・フォレスティエール\footnote{forestier
  (フォレスティエ)形容詞は森林の、の意。名詞の場合は森林管理人。一般には「森番風」などと訳されることが多いようだ。}}{ガルニチュール・フォレスティエール}}\label{garniture-a-la-Forestiere}}

\frsub{Garniture à la Forestière}

\index{garniture@garniture!forestiere@--- à la Forestière}
\index{forestier@forestier/forestière!garniture@garuniture à la Forestière}
\index{かるにちゆーる@ガルニチュール!ふおれすていえーる@---・フォレスティエール}
\index{ふおれすていえーる@フォレスティエール!かるにちゆーる@ガルニチュール・---}

(牛、羊の塊肉や鶏の料理に添える)

\begin{itemize}
\item
  モリーユ\footnote{茸の一種。和名アミガサタケ。生食出来ないので注意。}300
  gはバターと植物油同量ずつでソテーする。
\item
  脂身の少ない豚バラ肉の塩漬け125
  gは拍子木に切って下茹でし、バターでこんがりと焼く\footnote{rissoler
    (リソレ)。油脂を鍋に熱し、高温で素材の表面に焼き色を付けること。}。
\item
  じゃがいも300 gは大きめのさいの目に切ってバターでソテーする。
\item
  ブレゼの煮汁あるいはデグラセした液体を加えた\protect\hyperlink{sauce-duxelles}{ソース・デュクセル}
\end{itemize}

\hypertarget{garniture-frascati}{%
\subsubsection[ガルニチュール・フラスカーティ]{\texorpdfstring{ガルニチュール・フラスカーティ\footnote{フラスカーティは古代ローマの避暑地として有名だったところ。18世紀末にナポリ出身のアイスクリーム職人ガルキがパリのブルヴァール・モンマルトルにカフェ・フラスカティという店名のカジノ兼レストラン、パティスリを開き盛況だったという。とりわけアイスクリームが評判を呼んだらしい。その後経営者が何度か代わり、1857年に建物は取り壊された。このガルニチュールおよび「牛フィレ肉・フラスカーティ」がどちらに由来しているかは不明。フランス語風に発音するなら「フラスカティ」となる。}}{ガルニチュール・フラスカーティ}}\label{garniture-frascati}}

\frsub{Garniture Frascati}

\index{garniture@garniture!forestiere@--- Frascati}
\index{frascati@Frascati!garniture@garuniture ---}
\index{かるにちゆーる@ガルニチュール!ふらすかーてい@---・フラスカーティ}
\index{ふらすかーてい@フラスカーティ!かるにちゆーる@ガルニチュール・---}

(牛、羊の塊肉の豪華な仕立てに添える\footnote{直訳すると「牛、羊の塊肉のルルヴェ用」(ルルヴェについては「\protect\hyperlink{releve}{第二版序文訳注}」参照)だが、本書では「\protect\hyperlink{filet-de-boeur-frascati}{牛フィレ肉・フラスカーティ}」くらいしか目ぼしいレシピがない。})

\begin{itemize}
\item
  厚さ1〜2 cm程度スライスした\footnote{escalope (エスカロップ)。}フォワグラ(出来るだけ生のものがいい)10枚に小麦粉をまぶし、バターでソテーする。
\item
  アスパラガスの穂先300 gは茹でてからバターであえる。
\item
  小さめの真っ白なマッシュルーム10個は軸を落とし、渦巻状に飾り模様を入れる。
\item
  大きめのオリーブくらいのサイズに成形したトリュフ10個はバターでかるく炒めて艶を出す。
\item
  トリュフ風味にした\protect\hyperlink{pommes-de-terre-duchesse}{じゃがいものデュシェス}をアパレイユにして細長く作ったクロワッサン10個は提供直前に溶き卵を塗り、オーブンで焼いて艶を出す。このクロワッサンを並べてでガルニチュールの外枠にする。
\item
  軽くとろみを付けた肉汁(ジュ)\footnote{「牛フィレ肉・フラスカーティ」の場合は\protect\hyperlink{les-poeles}{ポワレ}するので、適量のフォンを肉を加熱する際鍋の底に敷いたマティニョンに注ぎ、肉汁の風味をひき出してから布で漉し、でんぷんでとろみを付けることになる。}。
\end{itemize}

\hypertarget{garniture-a-la-gastronome}{%
\subsubsection[ガルニチュール・ガストロノーム]{\texorpdfstring{ガルニチュール・ガストロノーム\footnote{美食家、食通、の意。}}{ガルニチュール・ガストロノーム}}\label{garniture-a-la-gastronome}}

\frsub{Garniture à la Gastronome}

\index{garniture@garniture!gastronome@--- à la Gastronome}
\index{gastronome@gastronome!garniture@garuniture à la ---}
\index{かるにちゆーる@ガルニチュール!かすとろのーむ@---・ガストロノーム}
\index{かすとろのーむ@ガストロノーム!かるにちゆーる@ガルニチュール・---}
\index{ひしよくかふう@美食家風 ⇒ ガストロノーム!かるにちゆーる@ガルニチュール・---}

(牛、羊の塊肉および鶏の料理に添える)

\begin{itemize}
\item
  大きめのマロン20個は、皮を剥いてコンソメで煮、小玉ねぎのようにバターで色艶よく炒める\footnote{glacer
    (グラセ)。}。
\item
  中位のサイズのトリュフ10個はシャンパーニュ風味に茹でる。
\item
  立派な雄鶏のロニョン\footnote{rognons de coq
    (ロニョンドコック)ここでは鶏の睾丸のこと。}20個はブロンド色の\protect\hyperlink{glace-de-viande}{グラスドヴィアンド}でコーティングする。
\item
  大きなモリーユ\footnote{morille
    茸の一種。和名アミガサタケ。生食不可なのでよく加熱する必要がある。}は縦二つ割りにし、バターでソテーする。
\item
  トリュフエッセンス入り\protect\hyperlink{sauce-demi-glace}{ソース・ドゥミグラス}。
\end{itemize}

\hypertarget{garniture-godard}{%
\subsubsection[ガルニチュール・ゴダール]{\texorpdfstring{ガルニチュール・ゴダール\footnote{18世紀の徴税官(つまりフィナンシエ)であり作家としても活動したクロード・ゴダール・ドクール
  Claude Godard d'Aucour (1716〜
  1795)の名を冠したものと考えられる。\protect\hyperlink{sauce-godard}{ソース・ゴダール}も参照のこと。本書のレシピだけを見ていると\protect\hyperlink{garniture-a-la-financiere}{ガルニチュール・フィナンシエール}と非常によく似ているけれどもソースの違うパターン、くらいにしか見えないかも知れぬが、このガルニチュールのほうが圧倒的に大掛かりで豪華な仕立てにすることを前提としており、ガルニチュール・ゴダールあるいはゴダールという名称の仕立てはフィナンシエールの完成形というべきか、究極の到達点のひとつだったのではないか?
  19世紀前半をとおして版を重ね、そのたびに増補されたヴィアールの本を版ごとに見ていくと、初版から
  1817年の第9版まではフィナンシエールのみ。1820年の第10版以降から「牛アロワイヨ・ゴダール」というレシピが登場する。長いレシピなので要点だけ見ると、約7〜8
  kgの牛アロワイヨ(日本語では「腰肉」すなわちフィレを含むサーロインからランプ、イチボにかけての部分)を四角形に切り整えて骨は取り除き、紐で縛ってからマデイラ酒を加えて6時間ブレゼする。肉を取り出したら煮汁を漉して、卵白でクラリフィエし、さら布で漉して煮詰める。その煮汁の半分にコンソメを足して、肉を鍋に戻し入れてさらに2時間弱火で煮込む。肉を皿に盛り付け、周囲に若鳩4羽、拍子木に切った豚背脂やトリュフ、赤く漬けた舌肉を表面に刺して装飾した仔羊胸腺肉4枚、スプーンで成形した大きなクネル8個、大きなエクルヴィス8尾、鶏またはその他の揚げものを刺した飾り串8本をアロワイヨの上から刺す。ドゥミグラス半量と合わせたラグー・フィナンシエールをかける。強火のオーブンで照りを付け、熱々を供する(p.101)。これを見るかぎり、本質的にはフィナンシエールの変形もしくは豪華版と考えていいだろう。カレーム『19世紀フランス料理』では「牛アロワイヨのブレゼ・ゴダール」に2種のレシピが記述されており、ひとつは上記と似たアロワイヨ全体をブレゼしたもの。もうひとつはアロワイヨの一部をブレゼし、残りはローストにした仕立てになっている。いずれにしても、非常に豪華な仕立てであり、ものすごいコストがかかるため、きわめて格式の高い荘厳な宴席でしか出来ないだろうが「食卓外交に携わる料理人はこうした料理の知識を大切にして、これらの料理を供すべきだ」と述べている(t.3,
  p.327)。デュボワ、ベルナール共著『古典料理』に至るとむしろゴダールという仕立ては簡略される方向に向かい、本書と同様に「ルルヴェ用ガルニチュール・ゴダール」として記述される。「このガルニチュールはトリュフで装飾を施した大きなクネル、表面に装飾をしてソースをかけてオーブンで焼き色を付けた仔牛胸腺肉、トリュフ、マッシュルームで構成される。これらを各まとまりごとに料理の周囲に添える。クネルとマッシュルームには軽くソースをかけてやり、トリュフと仔牛胸腺肉には艶を出させてやる(グラセ)こと」(p.94)とある。こうしたことから、フィナンシエールおよびその発展形としての仕立てであるゴダールが19世紀後半にむかってだんだんと広まっていき、盛んに作られるようになったが、ゴダールについてはそのコストゆえに簡素化していく傾向にあった。いずれにしても両者ともにきわめて19世紀的なソースとガルニチュールの組み合わせ、あるいは料理の仕立てといえよう。}}{ガルニチュール・ゴダール}}\label{garniture-godard}}

\frsub{Garniture Godard}

\index{garniture@garniture!godard@--- Godard}
\index{godard@Godard!garniture@garuniture ---}
\index{かるにちゆーる@ガルニチュール!こたーる@---・ゴダール}
\index{こたーる@ゴダール!かるにちゆーる@ガルニチュール・---}

(牛、羊、鶏の大掛かりで豪華な仕立てに添える)

\begin{itemize}
\item
  マッシュルームとトリュフのみじん切りを加えた、\protect\hyperlink{farce-a}{バター入りのファルス}をスプーンで成形したクネル10個。
\item
  トリュフと赤く漬けた舌肉で装飾を施した大きな楕円形のクネル4個。
\item
  小さめのマッシュルーム10個は軸を除き、螺旋状に切れ込み模様を付ける。
\item
  雄鶏のとさかとロニョン125 g。
\item
  上等の仔羊胸腺肉200
  gは高温のオーブンで焼き色を付ける。または仔牛胸腺肉の喉側を高温のオーブンで焼き色を付け、スライスする
\item
  オリーブ形に成形したトリュフ10個。
\item
  \protect\hyperlink{sauce-godard}{ソース・ゴダール}
\end{itemize}

\hypertarget{garniture-grand-duc}{%
\subsubsection[ガルニチュール・グランデュック]{\texorpdfstring{ガルニチュール・グランデュック\footnote{grand-duc
  (グロンデュック)大公およびロシアの皇太子の意。
  Prince(プランス)大公とほぼ同義だが使われる場面などで違いがある。料理においてはアスパラガスの穂先とトリュフを用いた料理に付されることが多い。}}{ガルニチュール・グランデュック}}\label{garniture-grand-duc}}

\frsub{Garniture Grand-Duc}

\index{garniture@garniture!grand-duc@--- Grand-Duc}
\index{grand-duc@grand-duc!garniture@garuniture ---}
\index{かるにちゆーる@ガルニチュール!くらんていゆつく@---・グランデュック}
\index{くらんていゆすく@グランデュック!かるにちゆーる@ガルニチュール・---}
\index{たいこう@大公(風)⇒グランデュック!かるにちゆーる@ガルニチュール・---}

(魚料理に添える)

\begin{itemize}
\item
  アスパラガスの穂先200gは下茹でしてバターであえる。
\item
  殻をむいたエクルヴィスの尾の身10。
\item
  大きなトリュフのスライス10枚。
\end{itemize}

\hypertarget{garniture-a-la-grecque}{%
\subsubsection[ガルニチュール・ギリシア風]{\texorpdfstring{ガルニチュール・ギリシア風\footnote{grec
  / grecque
  は「ギリシアの」の意。ここではあえて「ギリシア風」訳したが、ギリシアに起源あるいは縁のないと思われる調理が少なくないので注意。}}{ガルニチュール・ギリシア風}}\label{garniture-a-la-grecque}}

\frsub{Garniture à la Grecque}

\index{garniture@garniture!grecque@--- à la  Grecque}
\index{grec@grec/grecque!garniture@garuniture à la grecque}
\index{かるにちゆーる@ガルニチュール!くれつく@---・グレック}
\index{くれいつく@グレック!かるにちゆーる@ガルニチュール・---}
\index{きりしあふう@ギリシャ風⇒グレック!かるにちゆーる@ガルニチュール・---}

(乳呑仔羊および鶏料理に添える)

\begin{itemize}
\item
  \protect\hyperlink{riz-a-la-grecque}{ギリシア風ライス}\footnote{ピラフの一種だが、実際のところまったくギリシア風ではないことに注意。}250
  g(\protect\hyperlink{riz}{野菜料理「米」の項}参照)。
\item
  \protect\hyperlink{sauce-tomate}{トマトソース}
\end{itemize}

\hypertarget{garniture-henri-iv}{%
\subsubsection{ガルニチュール・アンリ4世亭風}\label{garniture-henri-iv}}

\frsub{Garniture Henri IV }

\index{garniture@garniture!henri iV@--- Henri IV}
\index{henri iv@Henri IV!garniture@garuniture ---}
\index{かるにちゆーる@ガルニチュール!あんりよんせいていふう@---・アンリ4世亭風}
\index{あんりよんせいていふう@アンリ4世亭風!かるにちゆーる@ガルニチュール・---}

(ノワゼットやトゥルヌドに添える)

\begin{itemize}
\tightlist
\item
  肉に合わせて中くらいから小さめのアーティチョークの芯\footnote{比較的小ぶりであっても完熟のアーティチョーク(開花がやや近い状態のもの)は下茹で後に花萼をすべて取り除く。この状態を
    fond d'artichaut (フォンダルティショー)またはcul d'artichaut
    (キュダルティショー)と呼ぶ。とりわけ大きなアーティチョークは花萼部が完全に固いことが多いために、上半分よりやや下で切り離して、繊毛を取り除いてから下茹でする。花萼を全て剥いて皿のような形状の基底部のみを取り出す。丸い皿のような底面になるので、そこに詰め物をすることが多い。小さく比較的若どりのアーティチョークは花萼を全部は取り除かず、周囲の固いところを2周くらい剥いて使う。これを
    coeur d'artichaut
    (クールダルティショー)と呼ぶ。サイズによっては縦半分または四つ割りにして繊毛を取り除いてから下茹でする。四つ割りの場合はquartiers
    d'artichaut
    (カルティエダルティショー)と呼ぶ。若どりのアーティチョークの内側の花萼は柔らかく火が通り、とても美味。また、生食できるくらい若どりのアーティチョークはpoivrade(ポワヴラード)とも呼ばれる。ただし若どりであればそれだけ、アーティチョーク特有の風味は弱い。}に、溶かした\protect\hyperlink{glace-de-viande}{グラスドヴィアンド}の中に入れて転がしてグラスをコーティングさせた小さな\protect\hyperlink{pommes-de-terre-noisette}{じゃがいものノワゼット}を詰める。
\end{itemize}

\hypertarget{garniture-a-la-hongroise}{%
\subsubsection[ガルニチュール・ハンガリー風]{\texorpdfstring{ガルニチュール・ハンガリー風\footnote{この名称の根拠となっているのはパプリカを使用していることのみ。\protect\hyperlink{sauce-hongroise}{ハンガリー風ソース}訳注も参照。}}{ガルニチュール・ハンガリー風}}\label{garniture-a-la-hongroise}}

\frsub{Garniture à la Hongroise}

\index{garniture@garniture!hongroise@--- à la Hongroise}
\index{hongrois@hongrois(e)!garniture@garuniture à la ---e}
\index{かるにちゆーる@ガルニチュール!はんかりーふう@---・ハンガリー風}
\index{はんかりーふう@ハンガリー風!かるにちゆーる@ガルニチュール・---}

(いろいろな料理に添えられる)

\begin{itemize}
\item
  小房に分けたカリフラワーをクリームであえて、いくつかの小さな容器に詰め、バターを塗ったグラタン皿に裏返して並べ、上からおろしたチーズを振りかけ、みじん切りにしたハムを加えたパプリカ風味の\protect\hyperlink{sauce-mornay}{ソース・モルネー}で覆い、高温のオーブンに入れてこんがりと焼く。
\item
  パプリカで風味付けした軽いソースを添える。
\end{itemize}

\hypertarget{garniture-a-l-italienne}{%
\subsubsection{ガルニチュール・イタリア風}\label{garniture-a-l-italienne}}

\frsub{Garniture à l'Italienne}

\index{garniture@garniture!italienne@--- à l'Italienne}
\index{italien@italien(ne)!garniture@garuniture à l' ---ne}
\index{かるにちゆーる@ガルニチュール!いたりあふう@---・イタリア風}
\index{いたりあふう@イタリア風!かるにちゆーる@ガルニチュール・---}

(牛、羊の塊肉および鶏料理に添える)

\begin{itemize}
\item
  小さなアーティチョークを縦4つに切って、\protect\hyperlink{quartiers-d-artichauts-a-l-italienne}{イタリア風}に調理する(野菜料理「\protect\hyperlink{artichauts}{アーティチョーク}」参照)20個。
\item
  茹でたマカロニにたっぷりチーズをあえて円盤型にしたクロケット10個。
\item
  \protect\hyperlink{sauce-italienne}{イタリア風ソース}。
\end{itemize}

\hypertarget{garniture-a-l-indienne}{%
\subsubsection{ガルニチュール・インド風}\label{garniture-a-l-indienne}}

\frsub{Garniture à l'Indienne}

\index{garniture@garniture!indienne@--- à l'Indienne}
\index{indien@indien(ne)!garniture@garuniture à l' ---ne}
\index{かるにちゆーる@ガルニチュール!いんとふう@---・インド風}
\index{いんとふう@インド風!かるにちゆーる@ガルニチュール・---}

(魚、牛、羊の塊肉や鶏料理に添える)

\begin{itemize}
\item
  \protect\hyperlink{riz-a-l-indienne}{インド風に調理}したパトナ米\footnote{パトナはコメの品種名。いわゆる「長粒種」だがバスマティのような香り米ではない。}125
  g(野菜料理「\protect\hyperlink{riz}{米}」参照)。
\item
  \protect\hyperlink{sauce-a-l-indienne}{インド風ソース}。
\end{itemize}

\hypertarget{garniture-a-la-japonaise}{%
\subsubsection[ガルニチュール・日本風]{\texorpdfstring{ガルニチュール・日本風\footnote{このガルニチュールが「日本風」であるのは、ちょろぎを用いているから。中国原産のシソ科の根菜で、現代日本では慶事などの際に用いられる程度だが、どういうわけか日本原産と誤解されたまま19世紀にフランスで栽培されるようになり、以来、日本風の名を付けた料理にはほとんど必ずといっていい程、ちょろぎが用いられる。}}{ガルニチュール・日本風}}\label{garniture-a-la-japonaise}}

\frsub{Garniture à la Japonaise}

\index{garniture@garniture!japonaise@--- à la Japonaise}
\index{japonais@japonais(e)!garniture@garuniture à la ---e}
\index{かるにちゆーる@ガルニチュール!にほんふう@---・日本風}
\index{にほんふう@日本風!かるにちゆーる@ガルニチュール・---}

(牛、羊の塊肉の料理に添える)

\begin{itemize}
\item
  ちょろぎ625
  gは\protect\hyperlink{veloute}{ヴルテ}であえ、ブリオシュ型でつくりオーブンでこんがり焼いた\protect\hyperlink{croustades}{クルスタード}に詰める。
\item
  米の\protect\hyperlink{croquettes}{クロケット}10個\footnote{初版〜第三版は「\protect\hyperlink{croquettes-de-pommes-de-terre}{じゃがいものクロケット}10個」となっている。「じゃがいものクロケット」は初版からレシピが掲載されているが、初版〜第四版すなわち現行版において「\href{Ecroquettes-de-riz}{米のクロケット}」のレシピはアントルメすなわちデザートとして砂糖を加えて甘くつくるレシピしか掲載されておらず、そのとおりにすべきかは一考の余地がある。また、「\protect\hyperlink{croquettes-a-l-indienne}{インド風クロケット}」も米を使用している。}。
\end{itemize}

\hypertarget{garniture-a-la-jardiniere}{%
\subsubsection[ガルニチュール・ジャルディニエール]{\texorpdfstring{ガルニチュール・ジャルディニエール\footnote{jardinier/jardinière
  (ジャルディニエ、ジャルディニエール)には名詞で「庭師」の意味もあるが、ここでは
  jardin potager (ジャルダンポタジェ)すなわち野菜畑、菜園、の意。}}{ガルニチュール・ジャルディニエール}}\label{garniture-a-la-jardiniere}}

\frsub{Garniture à la Jardinière}

\index{garniture@garniture!jardiniere@--- à la Jardinière}
\index{jardinier@jardinier/jardinière!garniture@garuniture à la Jardinière}
\index{かるにちゆーる@ガルニチュール!しやるていにえーる@---・ジャルディニエール}
\index{しやるていにえーる@ジャルディニエール!かるにちゆーる@ガルニチュール・---}

(牛、羊の塊肉の料理に添える)

\begin{itemize}
\item
  にんじん125 gと蕪125
  gは、プレーンな、あるいは刻み模様の入ったスプーンでくり抜く。あるいは円柱形にしてもいい。これをコンソメで煮て、最後にバターで色艶よく炒める。
\item
  プチポワ125 g。小さなフラジョレ125 g。アリコヴェール125
  gは小さな菱形に切る。これらを別々にバターであえる\footnote{しっかり加熱調理してからバターであえること。}。
\item
  茹であげたばかりのカリフラワーの小房10個。
\item
  以上の構成要素を肉の周囲に、別々にはっきりとニュアンスが代わるように配置する。カリフラワーの小房は\protect\hyperlink{sauce-hollandaise}{オランデーズソース}小さじ1杯程度をそれぞれに塗ってやる。
\item
  ソース\ldots{}\ldots{}澄んだジュ(肉汁)。
\end{itemize}

\hypertarget{garniture-joinville}{%
\subsubsection[ガルニチュール・ジョワンヴィル]{\texorpdfstring{ガルニチュール・ジョワンヴィル\footnote{フランソワ・ドルレアン・ジョワンヴィル海軍大将(1818〜
  1900)のこと。\protect\hyperlink{sauce-joinville}{ソース・ジョワンヴィル}も参照。}}{ガルニチュール・ジョワンヴィル}}\label{garniture-joinville}}

\frsub{Garniture Joinville}

\index{garniture@garniture!joinville@--- Joinville}
\index{joinville@Joinille!garniture@garuniture ---}
\index{かるにちゆーる@ガルニチュール!しよわんういる@---・ジョワンヴィル}
\index{しよわんういる@ジョワンヴィル!かるにちゆーる@ガルニチュール・---}

(魚料理に添える)

\begin{itemize}
\item
  以下のものを5 mm角くらいの小さな角切り\footnote{salpicon
    (サルピコン)。}か短かい拍子木状\footnote{julienne courte
    (ジュリエーヌクルト)。}に刻む\ldots{}\ldots{}茹でたマッシュルーム125
  g、トリュフ50 g。これにクルヴェットの尾の身125
  gを加え、スプーン数杯の\protect\hyperlink{sauce-joinville}{ソース・ジョワンヴィル}であえる。
\item
  追加として\ldots{}\ldots{}トリュフのスライス10枚。白くて大きなマッシュルームに殻をむいたクルヴェット8尾を刺す。
\item
  ソース・ジョワンヴィル。
\end{itemize}

\hypertarget{garniture-judic}{%
\subsubsection[ガルニチュール・ジュディック]{\texorpdfstring{ガルニチュール・ジュディック\footnote{女優アンナ・ジュディック(1849〜1911)の名を冠したもの。}}{ガルニチュール・ジュディック}}\label{garniture-judic}}

\frsub{Garniture Judic}

\index{garniture@garniture!judic@--- Judic}
\index{judic@Judic!garniture@garuniture ---}
\index{かるにちゆーる@ガルニチュール!しゆていつく@---・ジュディック}
\index{しゆていつく@ジュディック!かるにちゆーる@ガルニチュール・---}

(ノワゼットやトゥルヌド、鶏料理に添える)

\begin{itemize}
\item
  小さめのレチュを縦半割りにしてきれいに掃除し、ブレゼしたもの10個。
\item
  大きな雄鶏のロニョン10個。
\item
  トリュスのスライス10枚。
\item
  上等な仕上りの\protect\hyperlink{sauce-demi-glace}{ソース・ドゥミグラス}。
\end{itemize}

\hypertarget{garniture-languedocienne}{%
\subsubsection[ガルニチュール・ラングドック風]{\texorpdfstring{ガルニチュール・ラングドック風\footnote{languedocien(ne)
  (ラングドスィヤン、ラングドスィエーヌ)\textless{} Languedoc
  ラングドック地方。フランス南西部の地方名。もとは「オック語」langue
  d'oc から。中世プロヴァンス語と考えていい。オック oc
  とは古語で、現代フランス語の oui
  に相当する肯定の語。ラテン語の格変化の消失が比較的遅かった。バスク地方を除く(バスク語は別言語として扱われる)ロワール河以南で話された諸語の総称。これに対し、オイル語
  langue d'oil (ラングドイル)があり、oil
  が肯定の語であるというところが代表的な違い。主としてロワール河以北で話された諸語の総称。現代フランス語は後者の系統にあたるが、語彙の面などではラングドックの影響を大きく受けている。}}{ガルニチュール・ラングドック風}}\label{garniture-languedocienne}}

\frsub{Garniture Languedocienne}

\index{garniture@garniture!languedocienne@--- Languedocienne}
\index{languedocien@languedocien(ne)!garniture@garuniture ---}
\index{かるにちゆーる@ガルニチュール!らんくとつくふう@---・ラングドック風}
\index{らんくとつくふう@ラングドック風!かるにちゆーる@ガルニチュール・---}

(牛、羊の塊肉、鶏料理に添える)

\begin{itemize}
\item
  なすは1 cm厚の輪切りを10枚用意し、小麦粉をまぶして油で揚げる。
\item
  セープ\footnote{cèpe
    (セープ)、茸の一種、和名ヤマドリタケ。イタリアのポルチーニと同種だが、フランス産、イタリア産で風味や調理特性が異なる。また日本に多いのはヤマドリタケモドキという種で、食用できるが風味などはまったく及ばないという。類似のものにウツロイグチ、ドクヤマドリという毒茸があるので注意が必要。}400
  gはスライスして植物油でソテーする。
\item
  トマト400
  gは河を剥いて圧しつぶし、粗く刻んで、にんにく1片を加えて油でソテーする。
\item
  パセリのみじん切り。
\item
  ソース\ldots{}\ldots{}\protect\hyperlink{jus-de-veau-lie}{とろみを付けたジュ}。
\end{itemize}

\hypertarget{garniture-lorette}{%
\subsubsection[ガルニチュール・ロレット]{\texorpdfstring{ガルニチュール・ロレット\footnote{19世紀の7月王政期に娼婦たちの一部がロレットと呼ばれていた。ノートルダム・ド・ロレット(現在のパリ9区にある19世紀に建てられた教会)に因んでいるという。つまりこの「ロレット」とは特定の女性の固有名ではなく「ある職業および階層の女性たち」を意味する集合名詞。かつて「洗濯娘」を意味するgrisettes(グリゼット)と呼ばれた社会階層のやや下に位置する者が多かったため、いわゆる高等娼婦を意味する
  courisane(クルティザーヌ)と区別されることもある。グリゼットおよびロレットの例としては、バルザックの小説『幻滅』や『高等娼婦の栄華と悲惨(浮かれ女盛衰記)』にこの種の娼婦が主要人物として登場する。また、ロレットと呼ばれる娼婦たちは、第二帝政期にはcocotte(ココット)と呼ばれる高等娼婦にとって代わられた(たとえばゾラの『ナナ』やデュマ・フィス原作ヴェルディ作曲のオペラ『椿姫』の主人公\ldots{}\ldots{}原作ではマルグリット、オペラではヴィオレッタ、などがこれに相当する)。また、高等娼婦の中には、18世紀末のデュバリー夫人(\protect\hyperlink{garuniture-dubarry}{ガルニチュール・デュバリー}
  訳注参照)や、19世紀末から20世紀初頭にかけての女優サラ・ベルナールのように社会的に成功した例も少なくはない。貴族やブルジョワがこうした高等娼婦との「社交の場」として
  19世紀にはレストランや高級カフェを利用することが多かった。これを
  demi-monde(ドゥミモンド、半社交界)と呼び、そこでの華やかな主役たる高等娼婦たちはdemi-mondaine(ドゥミモンデーヌ)とも呼ばれた。このため、19世紀〜20世紀初頭にかけて創案された料理のなかには高等娼婦の名を冠したものも存外少なくない。}}{ガルニチュール・ロレット}}\label{garniture-lorette}}

\frsub{Garniture Lorette}

\index{garniture@garniture!lorette@--- Lorette}
\index{lorette@lorette!garniture@garuniture ---}
\index{かるにちゆーる@ガルニチュール!ろれつと@---・ロレット}
\index{ろれつと@ロレット!かるにちゆーる@ガルニチュール・---}

(ノワゼット、トゥルヌドに添える)

\begin{itemize}
\item
  小さくつくった\protect\hyperlink{croquettes-de-volaille}{鶏のクロケット}10個。
\item
  アスパラガスの穂先またはプチポワをバターであえる。
\item
  トリュフのスライス。
\item
  \protect\hyperlink{jus-de-veau-lie}{とろみを付けたジュ}
\end{itemize}

\hypertarget{garniture-louisiane}{%
\subsubsection[ガルニチュール・ルイジアナ風]{\texorpdfstring{ガルニチュール・ルイジアナ風\footnote{アメリカ合衆国のルイジアナ州のこと。とうもろこしは伝統的なフランス料理ではあまり好まれる食材ではないが、それを用いているところからの命名だろう。}}{ガルニチュール・ルイジアナ風}}\label{garniture-louisiane}}

\frsub{Garniture Louisiane}

\index{garniture@garniture!louisiane@--- Louisiane}
\index{louisiane@Louisiane!garniture@garuniture ---}
\index{かるにちゆーる@ガルニチュール!るいしあなふう@---・ルイジアナ風}
\index{るいしあなふう@ルイジアナ風!かるにちゆーる@ガルニチュール・---}

(家禽\footnote{原文は volaille
  (ヴォライユ)。本来は家禽全般つまり鶏、鴨(あひる)、七面鳥、鳩なども含まれるが、本書ではほとんどの場合、鶏(大きさや肥育方法により名称が多数ある)を意味するため、そのように訳しているが、ここでは七面鳥もしくはがちょうを前提としていると考えるのが妥当であり、調理方法も、ソースの指示からブレゼであると解釈される。なお、七面鳥は16世紀にアメリカ大陸からフランスにもたらされ、17世紀には流行の食材となった。当初は
  poulet d'Inde
  (プレダンド、インドの鶏の意)などと呼ばれていたが、やがてdinde
  (ダンド、七面鳥のメス)、dindon(ダンドン、七面鳥のオス)、dindonneau
  (ダンドノー、七面鳥の雛、若いオスの七面鳥)のように用語が定着していった。}の料理に添える)

\begin{itemize}
\item
  とうもろこし500 gはクリームであえる。
\item
  \protect\hyperlink{riz-au-gras}{リオグラ}\footnote{riz au gras
    直訳すると脂気の多い米、だが、実際には下茹でした米を脂気がやや残ったままのブイヨンで煮込んだもの。}をダリオル型\footnote{小さな円筒形の型。}に詰めて形づくった小さなタンバル\footnote{timbale
    直径に対してやや背の低い円筒形にする仕立て。大きなものはタンバル型
    moule à timbale
    (ムーラタンバール)に料理を詰めるが、ここでは小さなものを10個つくるので、ダリオル型を用いている。なお、
    timbale
    は元来「小太鼓、スネアドラム」の意。ただし、料理においては上述のような仕立てとともに、野菜料理用のやや深い皿のこともタンバルと呼ぶ。}10個。
\item
  バナナの輪切り20個を油で揚げる。
\item
  家禽を調理した際のフォンを煮詰めて仕上げたソース。
\end{itemize}

\hypertarget{garniture-lucullus}{%
\subsubsection[ガルニチュール・ルクッルス]{\texorpdfstring{ガルニチュール・ルクッルス\footnote{ルキウス・ルキニウス・ルクッルス(前118〜前56)。共和政ローマの軍人、政治家で美食家として知られた。}}{ガルニチュール・ルクッルス}}\label{garniture-lucullus}}

\frsub{Garniture Lucullus}

\index{garniture@garniture!lucullus@--- Lucullus}
\index{lucculus@Lucullus!garniture@garuniture ---}
\index{かるにちゆーる@ガルニチュール!るくつるす@---・ルクッルス}
\index{るくつるす@ルクッルス!かるにちゆーる@ガルニチュール・---}

(牛、羊の塊肉や鶏料理に添える)

\begin{enumerate}
\def\labelenumi{\arabic{enumi}.}
\item
  平均60
  gのトリュフ10個は\protect\hyperlink{mirepoix}{ミルポワ}にマデイラ酒風味で火を通す。\ul{中はくり抜き、蓋になる\\部分を取り置く}。雄鶏のロニョンをトリュフ1つにつき2つ、バターを加えた\protect\hyperlink{glace-de-viande}{グラスドヴィアンド}をまぶしてコーティングして詰める。取り置いておいた蓋をして、鶏のファルスでつくった小さなリボンで蓋とトリュフをつなぐ。低温のオーブンでファルスに火を通す。
\item
  トリュフをくり抜いた中身を混ぜ込んだ\protect\hyperlink{farce-c}{鶏の滑らかなファルス}をスプーンで成形して10個のクネルを用意する。混ぜ込むトリュフはあらかじめすり潰して裏漉しすること。
\end{enumerate}

\begin{itemize}
\item
  カールさせた大きな鶏のとさか10個。
\item
  トリュフエッセンス入り\protect\hyperlink{sauce-demi-glace}{ソース・ドゥミグラス}。
\end{itemize}

\hypertarget{garniture-macedoine}{%
\subsubsection[ガルニチュール・マセドワーヌ]{\texorpdfstring{ガルニチュール・マセドワーヌ\footnote{この語の初出は匿名で出版された\href{http://gallica.bnf.fr/ark:/12148/bpt6k1511832p}{『ガスコーニュ料理の本』}(1740年)であり、Macedoine
  à la Paysanne
  というレシピが掲載されている。ただしこの本はいわゆる「偽書」あるいは「奇書」に類するもので、ガスコーニュ地方の料理などひとつも掲載されていない。パリで印刷、出版されたにもかかわらず、アムステルダム(18世紀にアムステルダム版といえば「海賊版」の代名詞だった)出版として匿名で上梓された。匿名なので著者名はないのだが、献辞に「ドンブ大公閣下へ」とあり、実際の著者はまさにそのドンブ大公であるルイ・オーギュスト・ド・ブルボンそのひとであったと考えられている(17世紀にルイ14世の子で同名の者がいるが混同しないよう注意)。狩猟と料理が趣味であったという。一般的な言い回しとして「料理上手」の意味でcuisinier
  gascon(キュイジニエガスコン)「ガスコーニュの料理人」ということがあり、しかも内容は料理書として見た場合、どこまで真面目でどの程度冗談めいたものなのか判断に苦しむところがある。要は「殿様」の道楽本ともいえる。一例として「徴税官風鶏の袋詰め」\emph{Poulette
  en musette à la Financière}
  というほとんど冗談としか思えぬ、けれども歴史的に非常に興味深いレシピがある。茹でたプレ・レーヌ(若鶏と肥鶏の中間くらいの大きさ)を羊の膀胱にサルピコンとともに詰めて、息を吹き込んで膀胱を膨らませて口を縛る。1皿に3袋のせるべし。というもの(pp.128-129)。20世紀にフェルナン・ポワンが「鶏の膀胱包み」という歴史に残る名作料理をスペシャリテのひとつとしていたが、おそらくは膀胱に鶏を入れて膨らませるというプレゼンテーションについてはこれが最初の例であろう(この例では調理において膀胱を用いる必然性はまったくなく、たんに見せ方だけの問題だが)。また、「徴税官風」すなわちフィナンシエールという語が、後代のラグー・フィナンシエールあるいはフィナンシエール仕立てとはまったく関係ない文脈で使われている点も興味深い。要は税として徴収した財貨を詰め込んだ袋のイメージを演出するための小道具に過ぎないということ。マセドワーヌについては、えんどう豆とそら豆とアリコヴェール(これらはえんどう豆とおなじ大きさに刻む)、細切りにしたにんじんをバターを入れた鍋で弱火にかけ汗をかかせるようなイメージで蒸し煮し、時々混ぜながら、火が通ったら味付けしてソースを少量のソースとともに供するというもの(pp.139-140)。マセドワーヌの語は料理とは関係なく、同じ18世紀のラクロの小説『危険な関係』において「カードを混ぜる行為」という意味で用いられており、よく混ぜる、という意味において間違いはない。なお、一般にマセドワーヌというと\ul{小さなさいの目に切った}蕪やにんじん、アリコヴェール、プチポワなどを混ぜたものであり、日本のマセドアンサラダの原型にもなったが、料理用語の原義としては、必ずしも「さいの目」に刻む必要はない。このガルニチュール・マセドワーヌも『料理の手引き』における「ガルニチュール・ジャルディニエール」の指示どおりに野菜を下ごしらえして混ぜても成立するだろうし、切り方を揃えるという方法もあるだろう。}}{ガルニチュール・マセドワーヌ}}\label{garniture-macedoine}}

\frsub{Garniture Macédoine}

\index{garniture@garniture!macedoine@--- Macédoine}
\index{macedoine@macédoine!garniture@garuniture ---}
\index{かるにちゆーる@ガルニチュール!ませとわーぬ@---・マセドワーヌ}
\index{ませとわーぬ@マセドワーヌ!かるにちゆーる@ガルニチュール・---}

(牛、羊の塊肉の料理に添える)

\begin{itemize}
\tightlist
\item
  このガルニチュールは\protect\hyperlink{garniture-jardiniere}{ジャルディニエール}とまったくおなじ構成要素だが、すべての材料を混ぜてあえてしまう点が異なる。これを野菜料理用の深皿に別途盛り付けるか、アーティチョークの基底部に詰めるか、もしくは皿の中心にドーム状に盛り、その周囲に肉料理を並べるようにして盛り付ける。
\end{itemize}

\hypertarget{garniture-madeleine}{%
\subsubsection[ガルニチュール・マドレーヌ]{\texorpdfstring{ガルニチュール・マドレーヌ\footnote{マドレーヌといえば誰もが焼き菓子を想い浮かべるだろうが、このガルニチュールにはそれと類似する、あるいは想起させる要素がまったくない。マドレーヌは聖マドレーヌに由来し、教会の名称として珍しくないばかりか、女性の名前としてもごく一般的なものだ。他の料理書に同名のガルニチュールが見あたらないこと、本書初版から掲載されているものであることを考えると、あえていうなら、オクターヴ・ミルボー作の戯曲『\href{http://gallica.bnf.fr/ark:/12148/bpt6k203007v}{酷い羊飼いども}』初演(1897年)の際にサラ・ベルナールが演じて話題となった主人公の名がマドレーヌであることくらいか。資本家に虐げられた労働者階級が反乱を起こして失敗するという悲劇で、テーマとしてはゾラの『ジェルミナル』に近い。もしこのガルニチュールがミルボーの戯曲の登場人物を示唆しているなら、その料理を食べる側すなわち富裕層、資本家と、その料理を作る側の労働者との対立の図式が透けて見える、いわば強烈な風刺とも考えられるだろう。もっとも、エスコフィエは、その真意まではわからぬが常に資本家、富裕層の側に寄り添った料理人であったのもまた事実だ。オクターヴ・ミルボーについては、自然主義文学の作家としてスタートしたとはいえ、1964年にルイス・ブニュエルがジャンヌ・モロー主演で映画化した小説『小間使いの日記』で知られるように、いわゆる自然主義文学の枠にとどまることなく、独自の文学活動を展開した。画家ファン・ゴッホをモデルとした小説『天空にて』(新聞連載1892〜1893年)や、発表後にバルザックの遺族の抗議により作品の撤回を余儀なくされた『バルザックの死』などにより、フランス文学史においては世紀末文学の作家として位置付けられている。}}{ガルニチュール・マドレーヌ}}\label{garniture-madeleine}}

\frsub{Garniture Madelaine}

\index{garniture@garniture!madeleine@--- Madeleine}
\index{madeleine@Madeleine!garniture@garuniture ---}
\index{かるにちゆーる@ガルニチュール!まとれーぬ@---・マドレーヌ}
\index{まとれーぬ@マドレーヌ!かるにちゆーる@ガルニチュール・---}

(牛、羊の塊肉、鶏料理に添える)

\begin{itemize}
\item
  小さめのアーティチョークの基底部10個に固めにつくった\protect\hyperlink{sauce-soubise}{スビーズ}を詰める。
\item
  白いんげん豆のピュレ1
  Lあたり卵黄6個と全卵1個を加えてとろみを付け、仕上げにバター150
  gを加えたものを、ダリオル型に詰めて湯をはった天板にのせて低めの温度のオーブンで火を通したタンバル10個。
\item
  \protect\hyperlink{sauce-demi-glace}{ソース・ドゥミグラス}。
\end{itemize}

\hypertarget{garniture-a-la-maillot}{%
\subsubsection[ガルニチュール・マイヨ]{\texorpdfstring{ガルニチュール・マイヨ\footnote{maillot
  (マイヨ、男性名詞)には、産着、肌着、maillot de danseuse
  (マイヨドドンスーズ、踊り子のタイツ)、maillot
  jaune(マイヨジョーヌ、トゥールドフランスでトップの走者が着る黄色いウェア)などいろいろな意味があるが、ここではハムの料理に合わせること(ハムは本来、豚腿肉を加工したもの)からcancan(コンコン、いわゆるフレンチカンカン\ldots{}\ldots{}例えばロートレックの版画、絵画に描かれたような)で踊り子がタイツを履いた脚を高く上げて踊る姿を示唆していると解釈されよう。}}{ガルニチュール・マイヨ}}\label{garniture-a-la-maillot}}

\frsub{Garniture Madelaine}

\index{garniture@garniture!madeleine@--- Madeleine}
\index{madeleine@Madeleine!garniture@garuniture ---}
\index{かるにちゆーる@ガルニチュール!まとれーぬ@---・マドレーヌ}
\index{まとれーぬ@マドレーヌ!かるにちゆーる@ガルニチュール・---}

(牛、羊の塊肉、とりわけハムの料理に添える)

\begin{itemize}
\item
  大きなオリーブ形に成形した\footnote{tourner (トゥルネ)。}にんじん10個と蕪10個は、コンソメで煮る。
\item
  小玉ねぎ20個は下茹でしてからバターで色艶よく炒める\footnote{glacer
    (グラセ)。}。
\item
  縦半割りにした\protect\hyperlink{laitues-braisees-au-jus}{レチュのブレゼ}10個。
\item
  プチポワ100 gとアリコヴェール100 gはバターであえる。
\item
  \protect\hyperlink{jus-de-veau-lie}{とろみを付けたジュ}。
\end{itemize}

\hypertarget{garniture-a-la-maraichere}{%
\subsubsection[ガルニチュール・マレシェール]{\texorpdfstring{ガルニチュール・マレシェール\footnote{maraîcher/maraîchère
  比較的小規模な野菜生産者のこと。そのため、既出のガルニチュール・ジャルディニエールと非常に似た意味、すなわちあえて日本語にするならどちらも「菜園風」くらいの訳になろう。}}{ガルニチュール・マレシェール}}\label{garniture-a-la-maraichere}}

\frsub{Garniture à la Maraîchère}

\index{garniture@garniture!maraichere@--- à la Maraîchère}
\index{maraicher@maraîcher/maraîchère!garniture@garuniture à la Maraîchère}
\index{かるにちゆーる@ガルニチュール!まれしえーる@---・マレシェール}
\index{まれしえーる@マレシェール!かるにちゆーる@ガルニチュール・---}

(牛、羊の塊肉の料理に添える)

\begin{itemize}
\item
  サルシフィ\footnote{Salsifis
    こんにちフランス語で一般的にサルシフィと呼ばれているのは和名キバナバラモンジン。キク科の根菜。表皮が黒く、ごぼうに似ているが風味は異なる。また、じっくり時間をかけて加熱すれば筋っぽさがなくなり、とても柔らかくなる。別名scorzonère(スコルゾネール)。本来のサルシフィは表皮がやや白く、風味もやや異なるが、生産量はスコルゾネールと逆転するかたちで減少しつつある。}は長さ4
  cmの筒切りにして柔らかく茹で、固めにつくった\protect\hyperlink{veloute}{ヴルテ}であえる。
\item
  大きめのじゃがいも10個は\protect\hyperlink{pommes-de-terre-chateau}{シャトー}\footnote{大きめのオリーブのような形に剥き、塩こしょうして澄ましバターでゆっくり柔らかく火を通す。仕上げにパセリのみじん切りを散らす。}にする。
\item
  芽キャベツ(小)300
  gは下茹でしてからバターを入れた鍋で弱火で蒸し煮する\footnote{étuver
    (エテュヴェ)。}。
\item
  肉をブレゼまたはポワレした際のフォンをソースに仕上げて添える\protect\hyperlink{garniture-a-la-bouquetiere}{ガルニチュール・ブクティエール}訳注参照。
\end{itemize}

\hypertarget{garniture-marechal}{%
\subsubsection[ガルニチュール・マレシャル]{\texorpdfstring{ガルニチュール・マレシャル\footnote{元帥の意。}}{ガルニチュール・マレシャル}}\label{garniture-marechal}}

\frsub{Garniture Maréchal}

\index{garniture@garniture!marrechal@--- Maréchal}
\index{marechal@maréchal!garniture@garuniture ---}
\index{かるにちゆーる@ガルニチュール!まれしやる@---・マレシャル}
\index{まれしやる@マレシャル!かるにちゆーる@ガルニチュール・---}

\hypertarget{a.}{%
\subparagraph{A.}\label{a.}}

仔牛胸腺肉、牛、羊の塊肉の料理に添える場合\ldots{}\ldots{}

\begin{itemize}
\item
  トリュフ入りの鶏のファルスをスプーンで成形したクネル10個。
\item
  50〜60gのトリュフをスライスし、\protect\hyperlink{sauce-italienne}{イタリア風ソース}であえる\footnote{原文
    lié à l'italienne
    訳文は英訳第5版の「\protect\hyperlink{sauce-italienne}{イタリア風ソース}であえる」としているのに倣ったが、この表現自体は細かいさいの目に刻んだマッシュルーム(茸)であえる、の意。そのため、可能性としては\protect\hyperlink{duxelles-seche}{デュクセル・セッシュ}であえるということもあり得るが、実際には「つなぎ」に相当するものが必要になるだろう。}。
\item
  マデイラ酒風味の\protect\hyperlink{sauce-demi-glace}{ソース・ドゥミグラス}
\end{itemize}

\hypertarget{b.}{%
\subparagraph{B.}\label{b.}}

鶏胸肉のフィレ、仔牛胸腺肉の薄切り、ノワゼット、乳呑仔羊の骨付き背肉に添える場合\ldots{}\ldots{}

\begin{itemize}
\item
  ガルニチュールはバターで色艶よく炒めたトリュフの大きなスライスのみをメインの素材の上にのせる。バターであえたアスパラガスの穂先、季節でない場合にはごく小さなプチポワを添える。

  このガルニチュールを合わせる料理は必ず、細かい生パン粉に
  \(\frac{1}{3}\)量のトリュフのみじん切りを混ぜたイギリス式パン粉衣\footnote{素材に小麦粉をまぶして卵液にくぐらせ、パン粉衣を付けて油で揚げる方法。ただし日本と異なりパン粉は粒子の細かいものが一般的。}を付けて揚げ焼きしたもの\footnote{初版「これらの素材はパン粉衣を付けるか、トリュフのみじん切りをまぶし付けるか、パン粉の\(\frac{1}{3}\)量のトリュフを混ぜた衣を付けて調理する」。第二版〜第三版「これらの素材は必ず、トリュフのみじん切りをまぶし付けるか、細かい生パン粉に\(\frac{1}{3}\)量のトリュフのみじん切りを混ぜた衣を付けて調理する」。なお、このように直接素材にパン粉衣がうまく付くとはかぎらないので、通常は溶かしバターを素材に塗ってからパン粉の衣を付ける。これをフランス式パン粉衣
    pané à la française (パネアラフロンセーズ)という。}。
\end{itemize}

\hypertarget{garniture-marie-louise}{%
\subsubsection[ガルニチュール・マリ=ルイーズ]{\texorpdfstring{ガルニチュール・マリ=ルイーズ\footnote{マリア・ルイーザ(1791〜1847)。神聖ローマ皇帝フランツ2世の娘で、フランス皇帝ナポレオン1世の皇后。ナポレオンを憎み恐れて育ったにもかかわらず、ナポレオンがジョゼフィーヌとの離婚後に名家との婚姻を望んだため、オーストリアの外務大臣メテルニヒの計略により婚姻させられる。ナポレオン失脚後はパルマ公国の女公となる(在位1814〜1847)。ドイツ語ではMaria
  Ludovica von Österreich (マリア ルドウィカ フォン
  エスターライヒ)、フランス語ではMarie-Louise d'Autriche (マリルイーズ
  ドートリッシュ)と呼ばれる。}}{ガルニチュール・マリ=ルイーズ}}\label{garniture-marie-louise}}

\frsub{Garniture Marie-Louise}

\index{garniture@garniture!marie-louise@--- Marie-Louise}
\index{marie-louise@Marie-Louise (d'Autriche)!garniture@garuniture ---}
\index{かるにちゆーる@ガルニチュール!まりるいーす@---・マリ=ルイーズ}
\index{まりるいーす@マリ=ルイーズ!かるにちゆーる@ガルニチュール・---}

(ノワゼット、トゥルヌド、鶏料理に添える)

\begin{itemize}
\item
  アーティチョークの基底部\footnote{fond d'artichaut
    (フォンダルティショー)。}は添える肉に応じてサイズを選び、バターで蒸し煮して、\protect\hyperlink{sauce-soubise}{スビーズ}を\(\frac{1}{4}\)量加えたマッシュルームの固めのピュレを絞り袋でドーム状に詰める。
\item
  ソース\ldots{}\ldots{}\protect\hyperlink{jus-de-veau-lie}{とろみを付けたジュ}。
\end{itemize}

\hypertarget{garniture-marniere}{%
\subsubsection[ガルニチュール・マリニエール]{\texorpdfstring{ガルニチュール・マリニエール\footnote{marinier/marinière
  \textless{} mare
  ラテン語「海」から派生した語。\protect\hyperlink{sauce-mariniere}{ソース・マリニエール}も参照。}}{ガルニチュール・マリニエール}}\label{garniture-marniere}}

\frsub{Garniture à la Marinière}

\index{garniture@garniture!mariniere@--- à la Marinière}
\index{mariniere@marinière!garniture@garuniture à la ---}
\index{かるにちゆーる@ガルニチュール!まりにえーる@---・マリニエール}
\index{まりにえーる@マリニエール!かるにちゆーる@ガルニチュール・---}

(魚料理に添える)

\begin{itemize}
\item
  小さなムール貝\(\frac{3}{4}\)
  L(35個)は白ワインで蒸し煮して、身の周囲をきれいに掃除する\footnote{ébarber
    (エバルベ)。}。
\item
  殻をむいたクルヴェット\footnote{crevette
    小海老。小さめで生のときは灰色がかった crevette grise
    (クルヴェットグリーズ)とやや大きめで美味なcrevette
    rose(クルヴェットローズ)の2種が代表的。}の尾の身100 g。
\end{itemize}

\hypertarget{garniture-marquise}{%
\subsubsection[ガルニチュール・マルキーズ]{\texorpdfstring{ガルニチュール・マルキーズ\footnote{marquis
  / marquise 侯爵、侯爵夫人の意。}}{ガルニチュール・マルキーズ}}\label{garniture-marquise}}

\frsub{Garniture Marquise}

\index{garniture@garniture!marquise@--- Marquise}
\index{marquise@marquise!garniture@garuniture ---}
\index{かるにちゆーる@ガルニチュール!まるきーす@---・マルキーズ}
\index{まるきーす@マルキーズ!かるにちゆーる@ガルニチュール・---}

(ノワゼット、トゥルヌド、鶏料理に添える)

\begin{enumerate}
\def\labelenumi{\arabic{enumi}.}
\item
  縁に波形の飾り模様の付いた小さなタルトレット10個を空焼きする。これに以下を詰める。アムレット\footnote{牛、仔牛、仔羊の脊髄(=moelle
    モワル)。}250
  gをやや低温で火を通して小さな筒切りにする。アスパラガスの穂先125
  g、太さ1〜2 mmの千切り\footnote{julienne (ジュリエーヌ)。}にしたトリュフ
  50
  g、これらを、\protect\hyperlink{beurre-d-ecrevisse}{エクルヴィスバター}を仕上げに加えた\protect\hyperlink{sauce-allemande}{ソース・アルマンド}1
  \(\frac{1}{2}\) dLであえる。
\item
  濃いトマトピュレを混ぜ込んだ\protect\hyperlink{pommes-de-terre-duchesse}{じゃがいものデュシェス}を絞り袋で天板に小さな卵形に絞り出し、縦中央にナイフなどで切れ込みを入れ、オーブンに入れて提供数分前にこんがりと焼きあげたもの\footnote{ここで説明されているのは、\protect\hyperlink{pommes-de-terre-marquise}{じゃがいものマルキーズ}のレシピそのものといっていい内容で、かなりの部分が野菜料理の節にあるレシピと重複している。じゃがいものマルキーズにおいて形状は2パターン提示されており、これはそのうちのひとつであり、こんにちではあまりつくられなくなったパターンの方。大雑把なイメージとしては大きなコーヒー豆のあるいはキドニービーンズのような形を想像すればいいだろう。ここではかなり意訳したが、直訳すると「じゃがいものデュセスでつくったPain
    de la Mecque
    (パンドラメック=メッカのパンの意)」とある。本来これはシュー生地を卵形に天板に絞り出して溶き卵を塗り、グラニュー糖を振ってから縦中央にナイフで切れ込みを入れて焼くという、19世紀には比較的ポピュラーだった焼き菓子。グフェ『パティスリの本』(1873年)にもレシピが2種掲載されている
    (pp.287〜288)。シュー生地とほぼ同じものを使うため、内部に空洞が出来るが、そこにクレーム・シャンティイなどを絞り袋を使って詰めるバリエーションもあった。この焼き菓子になぜ「メッカ」の地名が付けられているのか、また、トマトピュレを加えたじゃがいものデュシェスをその形状に似せて焼いたものをなぜ、じゃがいものマルキーズ(=侯爵夫人)と呼ぶのかといった理由、由来は不明。ちなみに貴族の格としては公爵夫人(デュシェス)のほうが侯爵夫人よりも一般的に上位とされた。}20個。
\end{enumerate}

\hypertarget{garniture-a-la-marseillaise}{%
\subsubsection[ガルニチュール・マルセイエーズ]{\texorpdfstring{ガルニチュール・マルセイエーズ\footnote{marseillais(e)
  (マルセイエ / マルセイエーズ)マルセイユ Marseille
  の、の意。現在のフランス国歌 \emph{La Marseillaise}
  はフランス革命期の1792年に作曲され、同年8月10日のチュイルリー宮襲撃事件の際にマルセイユの義勇兵たちが口ずさんでいたことをきっかけにパリ市民の間で流行した。ただしマルセイユの義勇兵たちが作曲、作詞したわけでもなければ、その内容がマルセイユと関連があるわけでもない。いずれにしても第一帝政から王政復古期にかけては歌詞の「暴君を倒せ」という部分に問題があるいう理由から禁止され、1830年の七月革命以降解禁、第三共和政(1870〜1940)において正式に国歌として制定された。日本ではあまり知られていないが、決して平和的な内容の歌詞ではなく、むしろ激しい戦意を鼓舞する内容。とりわけルフラン(繰り返し部分)の「武器を取れ 市民らよ 隊列を組め 進もう 進もう 穢れた血が 我らの畝を満たさんことを」にその激烈さがよく表われている。1979年にセルジュ・ゲンズブールがこの曲をレゲエ風に編曲して発表したところ、「愛国者」からの脅迫が相次いだというエピソードは有名。}}{ガルニチュール・マルセイエーズ}}\label{garniture-a-la-marseillaise}}

\frsub{Garniture à la Marseillaise}

\index{garniture@garniture!marseillaise@--- à la Marseillaise}
\index{marseillais@marseillais(e)!garniture@garuniture ---e}
\index{かるにちゆーる@ガルニチュール!まるせいえーす@---・マルセイエーズ}
\index{まるせいえーす@マルセイエーズ!かるにちゆーる@ガルニチュール・---}

(牛、羊の塊肉の料理に添える)

\begin{itemize}
\item
  小さめのトマト5個を半割りにして中をくり抜き、にんにく1片と油をひとたらししてオーブンで焼く。立派なアンチョビのフィレを円環状になるようトマトに盛り込み、さらに成形した大きなオリーブを詰める。
\item
  それぞれのトマトの間に、細かく切って揚げたフライドポテトを配する。
\item
  \protect\hyperlink{sauce-provencale}{プロヴァンス風ソース}
\end{itemize}

\hypertarget{garniture-mascotte}{%
\subsubsection[ガルニチュール・マスコット]{\texorpdfstring{ガルニチュール・マスコット\footnote{エドモン・オドロン(1842〜1901)作曲のオペラコミック『ラ・マスコット』(1880年初演)にちなんだ名称。}}{ガルニチュール・マスコット}}\label{garniture-mascotte}}

\frsub{Garniture Mascotte}

\index{garniture@garniture!mascotte@--- Mascotte}
\index{mascotte@Mascotte!garniture@garuniture ---}
\index{かるにちゆーる@ガルニチュール!ますこつと@---・マスコット}
\index{ますこつと@マスコット!かるにちゆーる@ガルニチュール・---}

(ノワゼット、トゥルヌド、鶏料理に添える)

\begin{itemize}
\item
  アーティチョークの芯\footnote{原文は fonds d'artichaut
    なので字義通りに解すれば「アーティチョークの基底部」だが、下茹でしないということは比較的若どりのアーティチョークを用いる必要がある。その場合は花萼部の上半分程を切り捨て、茎の皮を剥いて四つ割りにするのが一般的。}10個は生のまま四つ割りに切り、バターでソテーする。
\item
  小さなじゃがいも20個はオリーブ形に剥き、バターで火を通す。
\item
  小さな玉にくり抜いたトリュフ10個。
\item
  肉を焼いた鍋に白ワインを注いでデグラセ\footnote{déglasser
    ソテーしている際に肉から流れ出た肉汁が煮詰まって鍋底にシロップ状に貼り付いたのを、何らかの液体を注いで溶かし出すこと。決して「焦げ」を取ることではないので注意。}し、\protect\hyperlink{fonds-de-veau-brun}{仔牛のフォン}を加えてソースに仕上げる。
\end{itemize}

\hypertarget{nota-garniture-mascotte}{%
\subparagraph{【原注】}\label{nota-garniture-mascotte}}

このガルニチュール・マスコットは、必ずココット仕立て\footnote{こんにちのココット仕立てを同じだが、本書においては「ポワレ」のバリエーションのひとつとして位置付けられている。\protect\hyperlink{poeles-speciaux-dits-en-casserole-ou-en-cocotte}{特殊なポワレ、カスロール仕立て、ココット仕立て}参照。}の肉の周囲を飾るようにして供する。

\hypertarget{garniture-massena}{%
\subsubsection[ガルニチュール・マセナ]{\texorpdfstring{ガルニチュール・マセナ\footnote{ナポレオン軍の元帥を務めたアンドレ・マセナ(1758〜1817)のこと。スイス戦役や半島戦争で著しい功績をあげた。ナポレオン失脚後の王政復古の際にもマルセイユの司令官を務め続けた。いわゆる百日天下の際に、軍に加わることはなかったがナポレオンを支持したために、全ての軍務を解任された。また、1898年就役、1915年退役となったフランス海軍の戦艦マセナは彼の名にちなんで命名された。}}{ガルニチュール・マセナ}}\label{garniture-massena}}

\frsub{Garniture Masséna}

\index{garniture@garniture!massena@--- Masséna}
\index{massena@Masséna!garniture@garuniture ---}
\index{かるにちゆーる@ガルニチュール!ませな@---・マセナ}
\index{ませな@マセナ!かるにちゆーる@ガルニチュール・---}

(ノワゼット、トゥルヌドに添える)

\begin{itemize}
\item
  中位か小さめのアーティチョークの基底部\footnote{あらかじめ適切に下処理をし、火を通しておくこと。}に固く仕上げた\protect\hyperlink{sauce-bearnaise}{ソース・ベアルネーズ}を詰める。
\item
  新鮮で大きな牛骨髄を輪切りにしてコンソメで沸騰しないよう火を通した\footnote{pocher
    (ポシェ)。}もの10枚。
\item
  \protect\hyperlink{sauce-tomate}{トマトソース}。
\end{itemize}

\hypertarget{garniture-matelote}{%
\subsubsection[ガルニチュール・マトロット]{\texorpdfstring{ガルニチュール・マトロット\footnote{水夫風の意。\protect\hyperlink{sauce-matelote}{ソース・マトロット}および魚料理「\protect\hyperlink{matelotes-types}{マトロット}」も参照。}}{ガルニチュール・マトロット}}\label{garniture-matelote}}

\frsub{Garniture Matelote}

\index{garniture@garniture!matelote@--- Matelote}
\index{matelote@matelote!garniture@garuniture ---}
\index{かるにちゆーる@ガルニチュール!まとろつと@---・マトロット}
\index{まとろつと@マトロット!かるにちゆーる@ガルニチュール・---}

(魚料理その他に添える)

\begin{itemize}
\item
  小玉ねぎ300 gは下茹でしてバターで色艶よく炒める\footnote{glacer
    (グラセ)。}。
\item
  マッシュルーム(小)200 gは茹でる。
\item
  食パンをハート形に切ってバターで揚げたクルトン10枚。場合によっては\protect\hyperlink{court-bouillon-c}{クールブイヨン}で火を通したエクルヴィスも添える。
\end{itemize}

\hypertarget{garniture-medicis}{%
\subsubsection[ガルニチュール・メディシス]{\texorpdfstring{ガルニチュール・メディシス\footnote{ルネサンス期のイタリア、フィレンツェにおいて金融業などにより実質的な支配者として君臨した名家。フランス史においてもっとも有名なカトリーヌ・ド・メディシス(1519〜1589)は後のフランス国王アンリ2世のもとに嫁ぎ、アンリ2世の死後15才で即位した長男フランソワ2世の摂政として権力を掌握、フランソワ2世が病死してシャルル9世が即位した後も実権を握り続けた。カトリックとプロテスタントが対立したユグノー戦争の時代であり、サン・バルテルミの虐殺(1572年)に関与したさえ当時はまことしやかに語られたという。また、カトリーヌがフランスの宮廷にフィレンツェの、とりわけ食文化を紹介、導入したという逸話は、アーティチョークからフォークにいたるまで非常に多いが、逸話の域を出ないものがほとんどで、史料として残っているものは非常に少ない。逆に言えば、ルネサンス期に文化的先進国だったイタリアから王妃を娶るということそれ自体が象徴するように、イタリア文化がさまざまなルート、形態でフランス文化に吸収された時代と見るのが妥当だろう。また、メディチ家はカトリーヌの後もフランス王アンリ4世(1553〜1610)にマリー・ド・メディシス(1575〜1642)を嫁がせている。なお、イタリア語の家名Medici(メディチ)はフランス語で伝統的にMédicis(メディシス)と呼ばれているため、ここではその慣習に倣って表記した。}}{ガルニチュール・メディシス}}\label{garniture-medicis}}

\frsub{Garniture Médicis}

\index{garniture@garniture!medicis@--- Médicis}
\index{medicis@médicis!garniture@garuniture ---}
\index{かるにちゆーる@ガルニチュール!めていしす@---・メディシス}
\index{めていしす@メディシス!かるにちゆーる@ガルニチュール・---}
\index{めていちけ@メディチ家 ⇒ メディシス!かるにちゆーる@ガルニチュール・---}

(牛、羊の塊肉、ノワゼット、トゥルヌドに添える)

\begin{itemize}
\item
  空焼きしたタルトレット10個に、マカロニとさいの目に切ったトリュフをフォワグラのピュレであえて詰める。
\item
  バターであえたプチポワ。
\item
  ソース\ldots{}\ldots{}\protect\hyperlink{jus-de-veau-lie}{とろみを付けたジュ}。
\end{itemize}

\hypertarget{garniture-a-la-mexicaine}{%
\subsubsection{ガルニチュール・メキシコ風}\label{garniture-a-la-mexicaine}}

\frsub{Garniture à la Mexicaine}

\index{garniture@garniture!mexicaine@--- à la Mexicaine}
\index{mexicain@mexicain(e)!garniture@garuniture à la ---}
\index{かるにちゆーる@ガルニチュール!めきしこふう@---・メキシコ風}
\index{めきしこふう@メキシコ風!かるにちゆーる@ガルニチュール・---}

(牛、羊の塊肉、鶏料理に添える)

\begin{itemize}
\item
  マッシュルーム10個はグリル焼きし、濃く煮詰めた\protect\hyperlink{portugaise}{トマトのフォンデュ}を添える。
\item
  ポワヴロン10個はグリル焼きする。
\item
  ソース\ldots{}\ldots{}カイエンヌを強く効かせた\protect\hyperlink{jus-lie-tomate}{トマト風味のジュ}。
\end{itemize}

\hypertarget{garniture-mignon}{%
\subsubsection{ガルニチュール・ミニョン}\label{garniture-mignon}}

\frsub{Garniture Mignon}

\index{garniture@garniture!mignon@--- Mignon}
\index{mignon@mignon!garniture@garuniture ---}
\index{かるにちゆーる@ガルニチュール!みによん@---・ミニョン}
\index{みによん@ミニョン!かるにちゆーる@ガルニチュール・---}

(ノワゼット、トゥルヌドに添える)

\begin{itemize}
\item
  小さなアーティチョークの基底部10個はバターで蒸し煮\footnote{étuver
    (エチュヴェ)。}し、バターであえたプチポワを詰める。
\item
  \protect\hyperlink{farce-b}{鶏の滑らかなファルス}で丸く作った小さなクネル10個に、それぞれトリュフのスライスをのせる。
\item
  デグラセ\footnote{déglacer
    (デグラセ)。肉を焼いた際に流れ出た肉汁が煮詰まって鍋底に濃いペースト状に貼り付いているのを酒類やフォンを加えて溶かし出すこと。焦げを取ることではないので注意。}したフォンにバターを加えて滑らかなソースに仕上げる。
\end{itemize}

\hypertarget{garniture-a-la-milanaise}{%
\subsubsection{ガルニチュール・ミラノ風}\label{garniture-a-la-milanaise}}

\frsub{Garniture à la Milanaise}

\index{garniture@garniture!milanaise@--- à la Milanaise}
\index{milanais@milanais(e)!garniture@garuniture ---e}
\index{かるにちゆーる@ガルニチュール!みらのふう@---・ミラノ風}
\index{みらのふう@ミラノ風!かるにちゆーる@ガルニチュール・---}

(牛、羊の塊肉料理に添える)

\begin{itemize}
\item
  マカロニ400 gは茹でて4 cmの長さに切る。
\item
  \protect\hyperlink{saumure-liquide-pour-langues}{赤く漬けた舌肉}50 g。
\item
  ハムとマッシュルーム各50 g。
\item
  トリュフ40 g\ldots{}\ldots{}これらは千切り\footnote{julienne
    (ジュリエーヌ)。}にする。
\item
  おろしたグリュイエールチーズ 50 gとパルメザンチーズ50 g。
\item
  トマトピュレ1 dL。
\item
  バター100 g。
\item
  澄んだ\protect\hyperlink{sauce-tomate}{トマトソース}を添える。
\end{itemize}

\hypertarget{garniture-mirabeau}{%
\subsubsection[ガルニチュール・ミラボー]{\texorpdfstring{ガルニチュール・ミラボー\footnote{フランス革命期、立憲君主主義を主張したオノレ・ガブリエル・ミラボー伯爵(1748〜1891)の名を冠した料理名だが、多くの日本人にとっては「
  ミラボー橋の下を セーヌ川が流れ われらの恋が流れる」という堀口大學の美しい訳で知られたアポリネールの詩のほうがなじみがあるだろうか。実際にパリにかかる橋のなかでは下流寄りに位置し、幅の広い橋ではあるが、それ自体や周囲の風景にとくに風情があるとは言い難い。実際に文学散歩としてミラボー橋を訪ずれた日本人観光客や留学生のほとんどが、がっかりしたという感想を持つというが、これは堀口訳が原詩以上といっていいくらい美しく、日本人の琴線に触れたがゆえに過度の情緒的な期待をさせてしまうことによるのだろう。}}{ガルニチュール・ミラボー}}\label{garniture-mirabeau}}

\frsub{Garniture Mirabeau}

\index{garniture@garniture!mirabeau@--- Mirabeau}
\index{mirabeau@Mirabeau!garniture@garuniture ---}
\index{かるにちゆーる@ガルニチュール!みらほー@---・ミラボー}
\index{みらほー@ミラボー!かるにちゆーる@ガルニチュール・---}

(牛、羊の赤身肉のグリルに添える)

\begin{itemize}
\item
  アンチョビのフィレ20枚を網目状に肉の上に配置する。
\item
  大きなオリーブ10個は種を抜いておく。
\item
  茹がいたエストラゴンの葉で皿の周囲を飾る。
\item
  \protect\hyperlink{beurre-d-anchois}{アンチョビバター}125 g。
\end{itemize}

\hypertarget{garniture-mirette}{%
\subsubsection[ガルニチュール・ミレット]{\texorpdfstring{ガルニチュール・ミレット\footnote{目、瞳、瞼、の意。}}{ガルニチュール・ミレット}}\label{garniture-mirette}}

\frsub{Garniture Mirette}

\index{garniture@garniture!mirette@--- Mirette}
\index{mirette@mirette!garniture@garuniture ---}
\index{かるにちゆーる@ガルニチュール!みれつと@---・ミレット}
\index{みれつと@ミレット!かるにちゆーる@ガルニチュール・---}

\begin{itemize}
\item
  \protect\hyperlink{pommes-de-terre-mirette}{じゃがいものミレット}で作った小さなタンバル\footnote{円筒形にする仕立て。大きなものはmoule
    à
    timbale(ムーラタンバール)という専用の型を用いる。原義は「小太鼓」であり、直径に比べてやや高さのないものが本来の姿だが、19世紀にはかなり高さのある円筒形の仕立てにもこの名称が用いられた。}仕立て10個(\protect\hyperlink{legumes}{野菜料理}、\protect\hyperlink{pommes-de-terre-mirette}{じゃがいものミレット}参照)。
\item
  \protect\hyperlink{sauce-chateaubriand}{ソース・シャトーブリヤン}。
\end{itemize}

\hypertarget{garniture-a-la-moderne}{%
\subsubsection[ガルニチュール・アラモデルヌ]{\texorpdfstring{ガルニチュール・アラモデルヌ\footnote{近代風、の意。}}{ガルニチュール・アラモデルヌ}}\label{garniture-a-la-moderne}}

\frsub{Garniture à la Moderne}

\index{garniture@garniture!moderne@--- à la Moderne}
\index{moderne@moderne!garniture@garuniture à la ---}
\index{かるにちゆーる@ガルニチュール!あらもてるぬ@---・アラモデルヌ}
\index{もてるぬ@モデルヌ(アラ)!かるにちゆーる@ガルニチュール・アラモデルヌ}
\index{きんたいふう@近代風 ⇒ アラモデルヌ!かるにちゆーる@ガルニチュール・アラモデルヌ}

(牛、羊の塊肉の料理に添える)

\begin{itemize}
\item
  六角形の小さな型10個の底にトリュフのスライスを敷き、シャルトルーズ仕立て\footnote{\protect\hyperlink{bordures-en-legumes}{野菜で作る縁飾り}および訳注参照。この本文は非常に簡潔に記述されているが、実際の工程としては、\protect\hyperlink{chou-braise}{サヴォイキャベツを下茹でした後にブレゼ}し、バターを塗った型の底にトリュフのスライスを敷いた後、それぞれに下茹でして小さな拍子木に切ったにんじん、蕪、アリコヴェールなどを配色よく周囲に貼り付け、さらに\protect\hyperlink{farce-a}{ファルス}を塗ってからブレゼしたサヴォイキャベツを詰め、湯煎にかけて加熱して固める、という手の込んだもの。}に周囲を装飾して、ブレゼしたサヴォイキャベツを詰める。加熱後に型を裏返して型から外す。
\item
  半割りにして\protect\hyperlink{laitues-farcies-pour-garniture}{詰め物をしたレチュのブレゼ}10個。
\item
  \protect\hyperlink{farce-b}{バター入り仔牛のファルス}をスプーンで楕円型に成形し、\protect\hyperlink{saumure-liquide-pour-langues}{赤く漬けた舌肉}で装飾を施した小さなクネル10個。
\item
  ソース\ldots{}\ldots{}\protect\hyperlink{jus-de-veau-lie}{とろみを付けたジュ}。
\end{itemize}

\hypertarget{garniture-montbazon}{%
\subsubsection[ガルニチュール・モンバゾン]{\texorpdfstring{ガルニチュール・モンバゾン\footnote{トゥール近郊の町の名だが、このガルニチュールの由来は不明。}}{ガルニチュール・モンバゾン}}\label{garniture-montbazon}}

\frsub{Garniture Montbazon}

\index{garniture@garniture!montbazon@--- Montbazon}
\index{montbazon@montbazon!garniture@garuniture ---}
\index{かるにちゆーる@ガルニチュール!もんはそん@---・モンバゾン}
\index{もんはそん@モンバゾン!かるにちゆーる@ガルニチュール・---}

(鶏料理に添える)

\begin{itemize}
\item
  大きさを揃えた子羊胸腺肉10個に拍子木あるいは楔形に切ったトリュフを刺し、\protect\hyperlink{les-poeles}{ポワレ}する。
\item
  \protect\hyperlink{farce-b}{バター入りの鶏のファルス}で楕円形に作りトリュフで装飾を施したクネル10個。
\item
  表面に渦巻状の刻み模様をつけた白いマッシュルーム10個。
\item
  トリュフのスライス10枚。
\item
  \protect\hyperlink{sauce-supreme}{ソース・シュプレーム}
\end{itemize}

\hypertarget{garniture-a-la-montmorency}{%
\subsubsection[ガルニチュール・モンモランシー]{\texorpdfstring{ガルニチュール・モンモランシー\footnote{10世紀頃から続く貴族の名家。歴史上重要な活躍をした人物を多く輩出した。イルドフランス県にある街の名称でもある。}}{ガルニチュール・モンモランシー}}\label{garniture-a-la-montmorency}}

\frsub{Garniture à la Montmorency}

\index{garniture@garniture!montmorency@--- à la Montmorency}
\index{montbazon@montbazon!garniture@garuniture à la ---}
\index{かるにちゆーる@ガルニチュール!もんもらんしー@---・モンモランシー}
\index{もんもらんしー@モンモランシー!かるにちゆーる@ガルニチュール・---}

(牛、羊の塊肉料理、鶏料理に添える)

\begin{itemize}
\item
  アーティチョークの基底部10個はバターで蒸し煮\footnote{étuver
    (エチュヴェ)}して、オランデーズソースでであえた\protect\hyperlink{garniture-macedoine}{マセドワーヌ}を詰める。
\item
  アスパラガスの小さな穂先10束。
\item
  肉の煮汁を加えて仕上げた{[}ソース・マデール{]}(\#sauce-madere。
\end{itemize}

\hypertarget{garniture-a-la-moissonneuse}{%
\subsubsection[ガルニチュール・モワソヌーズ]{\texorpdfstring{ガルニチュール・モワソヌーズ\footnote{moissonneur/moissoneuse
  (モワソヌール/モワソヌーズ)麦などの刈り入れをする人の意。基本的にレシピはアルファベット順の配列だが、ここはその原則が崩れている。}}{ガルニチュール・モワソヌーズ}}\label{garniture-a-la-moissonneuse}}

\frsub{Garniture à la moissonneuse}

\index{garniture@garniture!moissoneuse@--- à la Moissonneuse}
\index{moissoneur@moissoneur/moissoneuse!garniture@garuniture à la moissoneuse}
\index{かるにちゆーる@ガルニチュール!もわそぬーす@---・モワソヌーズ}
\index{もわそぬーる@モワソヌール/モワソヌーズ!かるにちゆーる@ガルニチュール・モワソヌーズ}

(牛、羊の塊肉料理に添える)

\begin{itemize}
\item
  千切りにしたレチュを加えた\protect\hyperlink{petits-pois-a-la-francaise}{プチポワ・アラフランセーズ}1
  L。
\item
  じゃがいも2個はスライスする。
\item
  脂身の少ない豚ばら肉の塩漬125 gはさいの目に切って下茹でする。
\item
  上記全部をまとめて火入れを仕上げる。
\item
  \protect\hyperlink{beurre-manie}{ブールマニエ}を加えてごく軽くとろみ付けする。
\end{itemize}

\hypertarget{garniture-montreuil}{%
\subsubsection[ガルニチュール・モントルイユ]{\texorpdfstring{ガルニチュール・モントルイユ\footnote{フランス北端の海に近い町、Montreuil-sur-Mer(モントルイユシュルメール)のこと。内陸部のセーヌサンドゥニ県には
  Montreuil-sous-Bois(モントルイユスーボワ)という非常によく似た名の町があり、その他にもモントルイユの地名はいくつかあるので注意。}}{ガルニチュール・モントルイユ}}\label{garniture-montreuil}}

\frsub{Garniture Montreuil}

\index{garniture@garniture!montreuil@--- Montreuil}
\index{montreuil@Montreuil!garniture@garuniture ---} \index{かるにちゆー
る@ガルニチュール!もんとるいゆ@---・モントルイユ} \index{もんとるいゆ@
モントルイユ!かるにちゆーる@ガルニチュール・---}

(魚料理に添える)

\begin{itemize}
\item
  じゃがいも20個を剥いて成形し、\protect\hyperlink{pommes-de-terres-a-l-anglaise}{アラングレーズ}\footnote{à
    l'anglaise
    イギリス風の意だが、フランス料理の用語としては、単に塩を加えた湯で火を通すことを指す。野菜の調理法のうちで基本のひとつ。なお、実際にイギリスで塩茹でがポピュラーな調理法かというと必ずしもそうではないらしく、1907年の英語版にはただし書きが付けられている。}にする。これを魚の周囲に縁飾りのように配する。
\item
  魚には\protect\hyperlink{sauce-vin-blanc}{白ワインソース}を、じゃがいもには\protect\hyperlink{sauce-aux-crevettes}{ソース・クルヴェット}を塗るように覆いかける。
\end{itemize}

\hypertarget{garniture-montpensier}{%
\subsubsection[ガルニチュール・モンパンシエ]{\texorpdfstring{ガルニチュール・モンパンシエ\footnote{15世紀まで遡ることの出来る名家で、とりわけ後の7月王政期で国王となったルイ・フィリップの弟で、フランス革命期には革命派であったアントワーヌ・フィリップ・ドルレアン・モンパンシエ公爵(1775〜1807)の名を冠した料理名。なおフランス語の発音を無理矢理カナで書くと
  Montpensierはモンポンシエのほうが近い音に聞こえるケースも多い。}}{ガルニチュール・モンパンシエ}}\label{garniture-montpensier}}

\frsub{Garniture Montpensier}

\index{garniture@garniture!montpensier@--- Montpensier}
\index{montpensier@Montpensier!garniture@garuniture ---}
\index{かるにちゆーる@ガルニチュール!もんはんしえ@---・モンパンシエ}
\index{もんはんしえ@モンパンシエ!かるにちゆーる@ガルニチュール・---}

(ノワゼット、トゥルヌド、鶏料理に添える)

\begin{itemize}
\item
  バターであえたアスパラガスの穂先の束。
\item
  ノワゼットまたはトゥルヌドにトリュフのスライスをのせる。
\item
  ソテーした鍋をデグラセし、バターを加えて滑らかな口あたりのソースに仕上げる。
\end{itemize}

\hypertarget{garniture-nantua}{%
\subsubsection[ガルニチュール・ナンチュア]{\texorpdfstring{ガルニチュール・ナンチュア\footnote{エクルヴィスが獲れることで知られるローヌ・アルプ地方のナンチュア湖に由来した名称。\protect\hyperlink{sauce-nantua}{ソース・ナンチュア}も参照}}{ガルニチュール・ナンチュア}}\label{garniture-nantua}}

\frsub{Garniture Nantua}

\index{garniture@garniture!nantua@--- Nantua}
\index{nantua@Nantua!garniture@garuniture ---}
\index{かるにちゆーる@ガルニチュール!なんちゆあ@---・ナンチュア}
\index{なんちゆあ@ナンチュア!かるにちゆーる@ガルニチュール・---}

(魚料理に添える)

\begin{itemize}
\item
  エクルヴィスの尾の身30は\protect\hyperlink{sauce-nantua}{ソース・ナンチュア}であえる。
\item
  トリュフのスライス20枚。
\item
  ソース・ナンチュア。
\end{itemize}

\hypertarget{garniture-a-la-napolitaine}{%
\subsubsection{ガルニチュール・ナポリ風}\label{garniture-a-la-napolitaine}}

\frsub{Garniture à la Napolitaine}

\index{garniture@garniture!napolitaine@--- à la Napolitaine}
\index{napolitain@napolitain(e)!garniture@garuniture à la  ---e}
\index{かるにちゆーる@ガルニチュール!なほりふう@---・ナポリ風}
\index{なほりふう@ナポリ風!かるにちゆーる@ガルニチュール・---}

(牛、羊の塊肉や鶏料理に添える)

\begin{itemize}
\item
  スパゲッティ500 gを茹でて\footnote{原文は Spaghetti pochés
    つまり「沸騰しない程度の温度で茹でる」とあり、塩を加えることなどは一切記されていない。もっとも、20世紀初頭までスパゲッティをナポリの庶民の多くは「手づかみ」で食べていたのであり、高級料理の食材としてはあまりなじみのないものだった。なお、日本の「ナポリタン」の発祥には諸説あるが、少なくともこのガルニチュールとは関係がないだろう。もっとも有力なのは太平洋戦争後に日本を一時占領した駐日米軍のイタリア系アメリカ人兵士から広まった説だろうか。}、おろしたグリュイエールチーズ50
  gとパルメザンチーズ50 g、トマトピュレ1 dLであえる。バター100
  gを加えて仕上げる。
\item
  肉をブレゼあるいはポワレ、ポシェしたフォンをソースに仕上げる。
\end{itemize}

\hypertarget{garniture-aux-navets}{%
\subsubsection[蕪のガルニチュール]{\texorpdfstring{蕪\footnote{navet
  (ナヴェ)「蕪」と訳してはいるが、日本で主流の「金町小かぶ」系とはまったく系統が違い、いくつかの系統が存在する。いずれの系統も、調理特性、味わいが日本のものと異なる点に注意。ちなみに俗語で
  C'est un navet.
  「味気ない、つまらない」という言い廻わしがあるが、野菜としてはむしろ、よく火を入れることで味わいを引き出すものと考えるべきであり、その点では日本料理における大根とやや近いところもあるだろう。}のガルニチュール}{蕪のガルニチュール}}\label{garniture-aux-navets}}

\frsub{Garniture aux Navets}

\index{garniture@garniture!navets@--- aux Navets}
\index{navet@navet!garniture@garuniture aux ---s}
\index{かるにちゆーる@ガルニチュール!かふ@蕪の---}
\index{かふ@蕪!かるにちゆーる@---のガルニチュール}

(羊や仔鴨の料理に添える)

\begin{itemize}
\item
  細長い大きめのオリーブのように成形した蕪30個は、バターと粉糖1つまみを加えてフライパンで色よく炒める。
\item
  バターで色よく炒めた小玉ねぎ20個
\item
  これらの野菜は、添える料理そのものに加えて火入れを仕上げるようにする。
\end{itemize}

\hypertarget{garniture-a-la-nicoise}{%
\subsubsection{ガルニチュール・ニース風}\label{garniture-a-la-nicoise}}

\frsub{Garniture à la Niçoise}

\index{garniture@garniture!nicoise@--- à la Niçoise}
\index{nicois@niçois(e)!garniture@garuniture à la  ---e}
\index{かるにちゆーる@ガルニチュール!にーすふう@---・ニース風}
\index{にーすふう@ニース風!かるにちゆーる@ガルニチュール・---}

(魚料理に添える場合)

\begin{itemize}
\item
  トマト250
  gは皮を剥き、潰して、叩いたにんにく1編とともにバター\footnote{オリーブオイルではなくバターを使っている点に注目すべきだろう。}でソテーする。最後にエストラゴンのみじん切りを1つまみ加える。
\item
  アンチョビのフィレ10枚。
\item
  黒オリーブ10個。
\item
  ケイパー大さじ1杯。
\item
  \protect\hyperlink{beurre-d-anchois}{アンチョビバター}30 g。
\item
  粗く皮を剥いて、種子を取り除いたレモンのスライス。
\end{itemize}

(牛、羊の塊肉や鶏料理に添える場合)

\begin{itemize}
\item
  トマト250 gは上述のとおりにする。
\item
  アリコ・ヴェール 300 gはバターであえる\footnote{豆類は未熟なものであっても消化阻害酵素を含むため、必ず下茹でして不活性化させる必要がある。アリコ・ヴェールすなわち「さやいんげん」やpois
    mange-tout
    (ポワモンジュトゥ)さやえんどう、も同様であり、その点は日本の「枝豆」と何ら変わるところがない(枝豆は大豆の未成熟のもの)。}。
\item
  \protect\hyperlink{pommes-de-terre-chateau}{じゃがいものシャトー}400
  g。
\end{itemize}

盛り付け\ldots{}\ldots{}トマトは肉の上にのせ、アリコ・ヴェールとじゃがいもは適量ずつ交互にまとめて配すること。

\begin{itemize}
\tightlist
\item
  \protect\hyperlink{jus-de-veau-lie}{とろみを付けたジュ}
\end{itemize}

\hypertarget{garniture-a-la-nivernaise}{%
\subsubsection[ガルニチュール・ニヴェルネ風]{\texorpdfstring{ガルニチュール・ニヴェルネ\footnote{ニヴェルネとはフランス革命以前の地方名であり、北がオルレアネ、南はブルボネ、東はブルゴーニュ、西がベリー領に椄していた。現在はニエーヴル県
  le département de la Nièvre (ルデパルトモン ドラ
  ニエーヴル)の一部。食文化的にはブルゴーニュ地方に近いとされている。}風}{ガルニチュール・ニヴェルネ風}}\label{garniture-a-la-nivernaise}}

\frsub{Garniture à la Nivernaise}

\index{garniture@garniture!nivernaise@--- à la Nivernaise}
\index{nivernais@nivernais(e)!garniture@garuniture à la  ---e}
\index{かるにちゆーる@ガルニチュール!にうえるねふう@---・ニヴェルネ風}
\index{にうえるねふう@ニヴェルネ風!かるにちゆーる@ガルニチュール・---}

(牛、羊の塊肉料理に添える)

\begin{itemize}
\item
  オリーヴ形に成形したにんじん500
  gは\protect\hyperlink{consomme-blanc-simple}{コンソメ}で煮てから、バターで色艶よく炒める。
\item
  小玉ねぎ300 gもバターで色艶よく炒める\footnote{必要に応じて下茹でしてから炒めるといい。とりわけ日本の「ペコロス」を用いる場合はそのほとんどが黄色系の品種で辛味が強いため、下茹ですることが望ましいだろう。逆に、オニオンブランとも呼ばれる白系品種で比較的若どりの場合は下茹でしない方がいいだろう。}。
\item
  塊肉をブレゼした煮汁をソースとして仕上げる。
\end{itemize}

\hypertarget{garniture-a-la-normande}{%
\subsubsection{ガルニチュール・ノルマンディ風}\label{garniture-a-la-normande}}

\frsub{Garniture à la Normande}

\index{garniture@garniture!normande@--- à la Normande}
\index{normand@normand(e)!garniture@garuniture à la  ---e}
\index{かるにちゆーる@ガルニチュール!のるまんていふう@---・ノルマンディ風}
\index{のるまんていふう@ノルマンディ風!かるにちゆーる@ガルニチュール・---}

(魚料理に添える)

\begin{itemize}
\item
  牡蠣10個とムール貝10個は沸騰しない温度で加熱して周囲を掃除する。
\item
  マッシュルーム(小)10個。
\item
  殻を剥いたエクルヴィスの尾の身100 g。
\item
  トリュフのスライス10枚。
\item
  クールブイヨンで火を通し、殻を剥かずに、はさみを後ろに回した\footnote{trousser
    (トゥルセ)。}中位のサイズのエクルヴィス10尾。
\item
  グジョンまたは小さなエペルランを尾の部分を残してパン粉衣を付けて揚げたもの10尾。
\item
  食パンを小さな菱形に切って提供直前に揚げたクルトン、または折り込みパイ生地を抜き型で三日月形や葉の形にして素焼きしたもの\footnote{fleurons
    de feuilletage cuit à blanc
    (フルロンドフイユタージュキュイタブロン)フルロンは折り込みパイ生地を三日月形や葉の形に抜いて、卵黄を塗って焼いたものだが、ここでは
    à blanc 「白く」とあるので説明的に訳した。}10個。
\item
  \protect\hyperlink{sauce-normande}{ノルマンディ風ソース}
\end{itemize}

\hypertarget{ux539fux6ce8}{%
\subparagraph{【原注】}\label{ux539fux6ce8}}

トリュフは省いてもいい。

\hypertarget{garniture-de-nouilles}{%
\subsubsection[ヌイユのガルニチュール]{\texorpdfstring{ヌイユ\footnote{卵入りの平麺。イタリアのフェットッチーネに近い。}のガルニチュール}{ヌイユのガルニチュール}}\label{garniture-de-nouilles}}

\frsub{Garniture de Nouilles}

\index{garniture@garniture!nouilles@--- de Nouilles}
\index{nouilles@nouilles!garniture@garuniture de ---}
\index{かるにちゆーる@ガルニチュール!ぬいゆの@ヌイユの---}
\index{ぬいゆ@ヌイユ!かるにちゆーる@---のガルニチュール}

(牛、羊の塊肉、鶏料理に添える)

\begin{itemize}
\item
  生の\protect\hyperlink{nouilles}{ヌイユ}をやや固めに茹で、おろしたグリュイエールチーズとパルメザンチーズ各50
  gであえ、仕上げにバター50 gを混ぜ込む。
\item
  肉をブレゼした際の煮汁をソースとして仕上げる。
\end{itemize}

\hypertarget{garniture-opera}{%
\subsubsection[ガルニチュール・オペラ]{\texorpdfstring{ガルニチュール・オペラ\footnote{現在は「オペラ・ガルニエ」と呼ばれているシャルル・ガルニエ設計、
  1862年着工、1874年竣工の劇場のことを指す。第二帝政期に大々的に行なわれたパリ都市改造いわゆるオスマン計画においてエトワール広場の凱旋門とともに象徴的な建物。豪華な内装と、シャガールによる天井画で有名。現在も国立オペラ座として、1989年に完成したバスティーユのオペラ座とともに使用されている。なお、opéra
  (オペラ)というフランス語はもとはラテン語の opera
  (仕事、作品の意)が、ルネサンス以降、とりわけ古典期にギリシア演劇の当時の解釈、ある意味では誤解から成立した音楽劇で、イタリア語で「正式なオペラ」opera
  seria
  と呼ばれる4〜5幕の悲劇と、その幕間に上演された軽い内容の1幕または2幕の喜劇、「幕間劇」(イタリア語
  entratto フランス語entracteオントラクト)から発展したオペラ・ブッファ
  opéra buffa
  に大別される。初期のオペラセリアにおいては、基本的に女性歌手ではなく、カストラートと呼ばれる幼少期に去勢された歌手がソプラノ、コントラルトなどの役を歌うことが多く、カストラートたちはオペラの「花形」として女性観客にとってアイドル的存在だったと言われている。18世紀末には人道的理由などからカストラートの養成もほとんど行なわれなくなった。オペラセリアとオペラ・ブッファの区別は18世紀前半のジョヴァンニ・バッティスタ・ペルゴレージ(とりわけオペラ・ブッファ『奥様女中』\emph{La
  Serva padrona}
  1733年)以降、あまり明確な意味を持たなくなり、喜劇的内容の作品が「幕合劇」の範疇に留まらず、独立した作品として上演されるようになった。モーツァルトの比較的後期のイタリアスタイルのオペラ『フィガロの結婚』『ドン・ジョヴァンニ』『コジ・ファン・トゥッテ』や、19世紀ロッシーニの『ラ・チェネレントラ』(シンデレラ)1817年、ドニゼッティ『愛の妙薬』1832年などがそうである。また、『ドン・ジョヴァンニ』などは喜劇的枠組みを利用しつつも、悲劇的側面をも備えた傑作として名高い。モーツァルトの『魔笛』はドイツ伝統のジングシュピールという音楽劇の形式によるものであり、上記のオペラの歴史の流れにおいてはやや特殊な位置にある。また、フランスのオペラコミックopéra-comique
  と呼ばれる形式は台詞部分にメロディをつけたもの(レチタティーヴォ)ではなく、普通の台詞として演じられる形式であり、ビゼーの『カルメン』やオッフェンバック『天国と地獄』などが代表例として挙げられよう。また、台本が喜劇的内容であれ悲劇的なものであれ、それまではひとつひとつが独立した曲としての体裁を保っていた歌手の独唱や重唱と、台詞に相当するレチタティーヴォの区別が19世紀後半、とりわけワーグナー以降あまり意味を持たなくなり、ヴェルディやプッチーニ以降のオペラではもはや判然としなくなるのが19世紀後半から20世紀初頭にかけてのオペラ作品の特徴のひとつと言えよう。ちなみにチョコレートケーキの一種であるオペラはオーストリア=ハンガリー帝国の首都ウィーンのオペラ座の外観に似ているという理由でつけられた名称であり、厳密にはフランス菓子ではない。}}{ガルニチュール・オペラ}}\label{garniture-opera}}

\frsub{Garniture Opéra}

\index{garniture@garniture!opera@--- Opéra}
\index{opera@Opéra!garniture@garuniture ---}
\index{かるにちゆーる@ガルニチュール!おへら@---・オペラ}
\index{おへら@オペラ!かるにちゆーる@ガルニチュール・---}

(ノワゼット、トゥルヌドに添える)

\begin{itemize}
\item
  添える肉の大きさに合わせたサイズの空焼きしたタルトレットに、鶏レバーをソテーしてマデイラ風味に仕上げたものを詰める。
\item
  \protect\hyperlink{pommes-de-teres-duchesse}{じゃがいものデュシェス}をアパレイユとして成形し、パン粉衣を付けて揚げ、中をくり抜いてクルスタードを作る。そこに、バターであえたアスパラガスの穂先を詰める。
\item
  肉を焼いた後にデグラセした肉汁を、バターを加えて滑かな口あたりに仕上げてソースとする。
\end{itemize}

\hypertarget{garniture-a-l-orientale}{%
\subsubsection{ガルニチュール・オリエント風}\label{garniture-a-l-orientale}}

\frsub{Garniture à l'Orientale}

\index{garniture@garniture!oriental@--- à l'Orientale}
\index{oriental@oriental(e)!garniture@garuniture à l'---e}
\index{かるにちゆーる@ガルニチュール!おりえんとふう@---・オリエント風}
\index{おりえんとふう@オリエント風!かるにちゆーる@ガルニチュール・---}

(鶏料理に添える)

\begin{itemize}
\item
  \protect\hyperlink{riz-a-la-grecque}{ギリシア風ライス}を小さな円筒形の型に詰めて10個のタンバルに仕立て、それぞれを半割りにして味付けし、ソテーしたトマトの上に盛る。
\item
  コルクの栓の形状にした、\protect\hyperlink{patates-douces}{さつまいも}のクロケット10
  個。
\item
  \protect\hyperlink{sauce-tomate}{トマトソース}。
\end{itemize}
\end{recette}\newpage



%%% Chapitre III. Potages
%% エスコフィエ『料理の手引き』全注解
% 五島 学
%% III. potages
\href{原稿下準備20180414五島、連載からコピー}{} \href{訳と注釈}{}
\href{未、原文対照チェック}{} \href{未、日本語表現校正}{}
\href{未、その他修正}{} \href{未、原稿最終校正}{}

\hypertarget{potages}{%
\chapter{III. ポタージュ Potages}\label{potages}}

\hypertarget{considerations-generales-potages}{%
\section{概説}\label{considerations-generales-potages}}

\frsec{Considérations Générales}

こんにちポタージュと呼ばれている料理は、少なくともいま目にする形態のも
のとしては、比較的新しい料理であり、せいぜい19世紀初頭までしか起源を遡
ることができない。

古典料理におけるポタージュは一皿に何もかもが入った料理だった\footnote{potage
  の語源は pot「壺、鍋」。古くは、鍋に肉や野菜を入れて煮込
  んだ料理(シヴェ、ラグー、ブルゥエなど)の総称であった。こんにちのよ
  うにポタージュが専ら液体料理の意味で用いられるようになったのは、エ
  スコフィエがここで述べているように、19世紀以降である。なお、日本語
  では一般的に「ポタージュ」という語はとろみがあるスープを意味し、
  「コンソメ」が澄んだスープを指すが、これは英語由来。}。こん
にちポタージュという名称は液体の料理だけを意味するが、古くは、その液体
を作るのに用いた牛、羊肉、鶏、、ジビエ、魚そのものと、こんにちでは浮き
実として使われている野菜もポタージュの一部として必ず含まれていた。

いくつか例を挙げるなら、フランドルの「オシュポ\footnote{hochepotフランドルの地方料理としては、牛の尾を主素材としたポトフ
  の一種を指す。他に、豚の耳と尾、煮込み用牛肉と野菜の煮込みを意味す
  る場合もある。料理名自体は非常に古く、14世紀のタイユヴァン『ル・ヴィ
  アンディエ』には「鶏のオシュポ」hochepot de poullaille がある。}」、スペインの「オジャ
\footnote{olla (olla podrida)
  カタカナではオジャ、オヤとも表記される。豚肉
  と各種内臓肉、豆、野菜の煮込み料理。南西フランスのウイヤ ouillat
  (オイユ oilleとも)の原型になったと言われている。また、日本語の「お
  じや」の語源になったという説もある。}」、そして我が国の「プチットマルミット\footnote{petite
  marmite小ぶりの陶製の鍋に具材と汁を入れてオーヴンで熱して
  供するポトフに似た料理。牛肉、牛の尾、骨髄、鶏などをブイヨンで煮込
  む。19世紀パリのレストランで大流行した。澄んだポタージュ(コンソメ)
  に分類されるが、クラリフィエ(澄ませる作業)は行なわない。また、必ず
  鶏を用いるのが特徴。}」もまた、古くから残され
てきたポタージュの代表例と言える。もっとも、こんにちではこれらの料理を
作る際に、多かれ少なかれ単純な構成にしているから、こう書いただけでは明
確なイメージが得られないかも知れない。

これらの料理にはごった煮のようなイメージがあるだろうが、大昔の献立という
のはつまるところそういうものだったのだ。食べ進むにつれ食欲がだんだん満
たされていくのに合わせて、順序よく料理を進行させるようなことなどしなかっ
た\footnote{現代のように、料理を食べる順に提供する「\protect\hyperlink{service-russe}{ロシア式サーヴィ
  ス}」が行なわれるようになったのは19世紀後半のこと。
  それ以前は大きな皿に盛られた複数の料理をまとめて食卓に供する「フラ
  ンス式サーヴィス」であった。}。昔の献立に非常に多くの種類のポタージュが見られるのは、適切に料
理を配した結果というよりは、まさにそのこと自体が献立の特徴なのだと言え
る\footnote{典拠不明。確かに、17世紀以前の料理書ではポタージュに多くのページ
  を割いているものも少なくないが、17世紀にL.S.R.という筆名で出版され
  た『饗応術』や、同じく17世紀マシアロの著作にある献立例では、必ずし
  もポタージュの数が突出して多いとは言えない。大規模な宴席を除けば、
  通常の献立におけるポタージュは2〜3種であり、同時に供されるアントレ、
  オル・ドゥーヴルの方が種類が多い。また、『ル・ヴィアンディエ』巻末
  の献立例では料理が4回に分けて供されるが、1回目はポタージュ、2回目
  はロ(ロティ)があてられており、料理の種類はほぼ同数ずつになっている。}。

他の多くの調理技術についても同じだが、ポタージュにおいてカレームの功績
は大きい。文字通りの意味でカレームが近代のポタージュを発案したわけでは
ないにしても、少なくとも、こんにちのポタージュのつくり方に移行する導き
手のひとりとして新たな料理理論の普及に大いに貢献したのだ。

けれど、カレーム以後の料理人たちはポタージュをこんにちの姿に完成させる
まで、1世紀近くかかってしまった。

彼らは風味豊かで軽やかな、理想的なまでに繊細で美味な料理を創案したのだ
が、それらの新しいポタージュに正しい料理名をつけることにはあまり頓着し
なかったのだろう。とりわけ、とろみをつけたポタージュに関しては、しばし
ば同じルセットについて\ul{ビスク}や\ul{ピュレ}、\ul{クリ}、\ul{ヴルテ}、
\ul{クレーム}の名称を頓着なく与えていた。理屈から言って、これらの名称
はそれぞれまったく異なった料理を意味しなくてはいけないにもかかわらず、
だ。結果として、残念なことに用語の混乱が起きてしまったわけだ。本書では、
それぞれのポタージュの特徴をはっきりさせて、さまざまなレシピを合理的に
分類することによってこの問題を正してある。

基本的な考え方は次のとおり。\ul{ヴルテ}と\ul{クレーム}の語がポタージュ
について用いられるようになったのは比較的近年に過ぎない。その理由として
は、\ul{ビスク}や\ul{クーリ}の語感が古めかしい印象を与えるうえに、
\ul{ピュレ}はあまりに俗っぽくて品位に欠ける印象だから、ビスク、クーリ、
ピュレの代りに\ul{ヴルテ}と\ul{クレーム}の語が用いられるようになったの
だろう。

だから、ポタージュのそれぞれの種類をはっきりと定義し、調理技術体系の欠
落を埋める必要がある。

それぞれの種類のポタージュの特徴について以下にまとめておいた。これをお
読みいただければ、本書で企図したこの改革の意義がお解りいただけることだ
ろう。

\hypertarget{classification-des-potages}{%
\section{ポタージュの種類}\label{classification-des-potages}}

\frsec{Classification des Potages}

料理を提供するという観点からは、ポタージュは大きく2つに分類される。

\textbf{澄んだポタージュ}と\textbf{とろみのあるポタージュ}である。

格式ある正餐の献立では常に、それぞれの分類から1つ以上のポタージュ
が供される。通常の献立では、ポタージュを1つだけにする場合、献立全体の
流れに応じて、上記2つの分類のうちどちらか一方とする。

\hypertarget{classification-potages-claires}{%
\subsection{澄んだポタージュ}\label{classification-potages-claires}}

\frsecb{Les Potages claires}

澄んだポタージュは、主素材として畜肉、家禽、ジビエ、魚、甲殻類、海亀な
どのどれを用いたものでも、分類としてはただひとつとなる。つまり、澄んだ
コンソメ\footnote{consommé 語源は動詞
  consommer「完遂する、完成させる」。つまり
  「完全に仕上げたもの」の意。もともとは必ずしも液体料理を指す語では
  なかった。例えば18世紀マラン『食の贈り物』には以下のような「コンソ
  メ」が出ている。まず「最初のブイヨン」をとり、それを用いて「ブイヨ
  ン・ミトナージュ」をとり、ブイヨン・ミトナージュを用いて「ブイヨン・
  コルディアル」をとり、さらにブイヨン・コルディアルを用いて「コンソ
  メ」を作る。最終的にはとろみがつく程度に煮詰める。このコンソメは単
  体で料理として供するものではなく、調味料的にポタージュにこくを与え
  たり、ソースを作るのに用いると記されている。つまりは本書における
  \protect\hyperlink{glace-de-viande}{グラスドヴィアンド}に近いものと考えられる。}である。タピオカでんぷんでごく軽くとろみをつける場合もある
が、いずれにせよ、それぞれのポタージュの性格に合わせて浮き実は少量とす
る。

\hypertarget{classification-potages-lies}{%
\subsection{とろみを付けたポタージュ}\label{classification-potages-lies}}

\frsecb{Les Potages Liés}

とろみのあるポタージュは5つに分類される。すなわち

\begin{enumerate}
\def\labelenumi{\arabic{enumi}.}
\item
  ピュレ、クーリ、ビスク
\item
  クレーム
\item
  ヴルテ
\item
  とろみをつけたコンソメ
\item
  特殊なポタージュ。上記のうち複数のポタージュの性格を持ち、バリエー
  ションを展開出来ないもの。
\end{enumerate}

より単純な分類にするため、本書では、\protect\hyperlink{potage-germiny}{ジェルミニ}のよ
うなとろみを付けたコンソメは特殊なポタージュに含めている。その理由は
「\protect\hyperlink{series-de-potages-lie-speciaux}{とろみを付けた特殊なポタージュ}」
の節の冒頭で述べることにする。

上記のポタージュのうち最初の3種はベースとして何らかのピュレを用いるが、
主に何によってとろみをつけるかで違いが出てくる。

「ピュレ」「クーリ」「ビスク」でとろみをつけるのは、主素材に応じて、米、
油で揚げたパン、じゃがいも、いんげん豆、レンズ豆などのでんぷん質の野菜。
これらのつなぎと主素材との分量比率はきちんと守らなければならないので、
「\protect\hyperlink{potages-lies}{とろみをつけたポタージュ}」の節の冒頭を参照されたい
\footnote{原書 p.135}。

「クレーム」と「ヴルテ」のとろみは白いルーがベースになるが、実際に用い
るつなぎは違う\footnote{ポタージュ・ヴルテは基本ソースのヴルテをつなぎとして用いるのに
  対し、ポタージュ・クレームはベシャメル(原書pp.136-138)。}。また、仕上げ方も異なる。

ヴルテの仕上げには必ず卵黄とバターを加えてとろみ付けをする。クレームは
とろみ付けの要素は足さず、バターではなく良質の生クリーム適量を加えて仕
上げる。

つまり、最終的にとろみ付けの要素が違うわけだから、クレームとヴルテはまっ
たく違うものとして区別すべきなのだ。

ピュレ、クーリ、ビスクは、いずれも作り方がほぼ同じわけだが、にもかかわ
らず、これらの語は同義ではない。むしろ明らかに違う意味を持っていること
に注意。

慣習として\ul{ピュレ}の名称は野菜をベースにしたものに用いるが、この名称
には俗っぽい印象があるため、避けられる傾向にある。

\ul{クリ}の名称は鶏、ジビエ、魚および甲殻類のピュレについて用いる。

甲殻類のピュレについては、\u{ビスク}の名称を用いる方が多い。むしろ、
ビスクと言えば甲殻類のピュレを指すことになっている。ただし、ビスクはそ
の発祥から18世紀末まで、鶏やと鳩を主素材にしたポタージュを意味する語だっ
た。

とろみを付けたポタージュの多くは、主素材はそのままで単に調理法を変えれ
ば、ピュレ、クレームおよびヴルテとして展開出来るのだが、これについては
後で記すことにする\footnote{原書 pp.134-138}。

本書の初版ではポタージュはこれらのとろみを付けたポタージュを3つに分類
していた。すなわち

\begin{enumerate}
\def\labelenumi{\arabic{enumi}.}
\item
  ピュレとしてでもヴルテあるいはクレームとしてでも提供可能なもの。
\item
  ピュレもしくはクレームとしてのみ提供可能なもの。
\item
  ヴルテまたはクレームとして調理可能なもの。
\end{enumerate}

このように分類することはしたが、読者がレシピを探すのを容易にするために
も、もっとシンプルな構成にする必要があることがわかった。そのため、上記
3つの分類にしていたとろみを付けたポタージュは1つにまとめて、ポタージュ
の名称のアルファベット順に掲載し、それぞれのレシピの最後にバリエーショ
ンが可能な場合は付記することにした。

結果として、ポタージュのレシピのパートは以下のような構成となった。

\begin{enumerate}
\def\labelenumi{\arabic{enumi}.}
\item
  ガルニチュール(浮き実)を添えた澄んだコンソメ
\item
  ピュレ、ヴルテ、クレームのかたちで調理可能なポタージュ、およびピュレかクレームとして調理可能なもの、ヴルテかクレームとしてしか調理できないもの
\item
  特殊なポタージュ、つまり作り方のバリエーションがないもの。および、とろみを付けたコンソメ
\item
  ブルジョワ家庭料理やいろいろな地方料理からそのまま採り上げたスープとポタージュ
\item
  外国のポタージュ
\end{enumerate}

\hypertarget{consideration-sur-le-service-des-potages}{%
\subsection{ポタージュを供するにあたっての注意}\label{consideration-sur-le-service-des-potages}}

\vspace{-1\zw} \begin{center}
\hspace{1\zw}\large\textit{Considérations sur le Service des Potages}
\normalsize \end{center}
\newpage
\href{原稿下準備20180414五島、連載からコピー}{} \href{訳と注釈}{}
\href{未、原文対照チェック}{} \href{未、日本語表現校正}{}
\href{未、その他修正}{} \href{未、原稿最終校正}{}

\hypertarget{ux30ddux30bfux30fcux30b8ux30e5ux306eux57faux790e}{%
\section{ポタージュの基礎}\label{ux30ddux30bfux30fcux30b8ux30e5ux306eux57faux790e}}

\vspace{-.5\zw}
\begin{center}
\setstretch{0.5}
\headfont\medlarge プチットマルミット、グランドブイヨン、\\コンソメのクラリフィエ \normalfont\normalsize
\end{center}
\setstretch{1.0}
\vspace{2ex}
\frsec{Précis des éléments nurtirifs, aromatiques}
\vspace{-1ex}
\begin{center}
\setstretch{0.5}
\headfont et de l'assaisonnement pour la Petite Marmite,\\ les Grands Bouillons,
et la clarification des Consommé divers

\end{center}
\normalfont
\setstretch{1.0}
\begin{recette}
\hypertarget{ux767dux3044ux30b3ux30f3ux30bdux30e1ux30b5ux30f3ux30d7ux30eb1}{%
\subsubsection[白いコンソメ・サンプル]{\texorpdfstring{白いコンソメ・サンプル\footnote{consommé
  simple「単純な(簡素な)コンソメ」の意。肉や魚、野菜を煮
  て漉しただけのもの。ここでは具体的な作業手順は記されていないが、モ
  ンタニェの『ラルース・ガストロノミーク』初版(1937年)の記述は概ね以
  下のとおり(材料はエスコフィエとほぼ同じ)。(a) 牛肉を紐で縛り、大鍋
  (陶製が良い)に入れて水7 Lを注ぐ。火にかけて沸騰したら、表面にアルブ
  ミンの軽く固まった膜が張るので、丁寧にこの膜を取り除く。鍋に野菜を
  加える。かすかに沸騰する火加減で5時間煮る。浮き脂を丁寧に取り除き、
  布または目の細かい漉し器で漉す。5時間以上煮込んではいけない。だが、
  5時間では骨に含まれているおいしさを全て抽出出来ないので、砕いた骨
  を長時間煮て第1のブイヨンをとり、これで肉と野菜を煮るようにすると
  良い。(b) 鍋に砕いた骨を入れ、水をかぶる程度注ぐ。沸騰させ、あくを
  引き、塩を加える。弱火で2時間半煮る。この「沸騰したブイヨン」に、
  骨を外して紐で縛った肉を入れる。再び沸騰させ、あくを引いて味を調え
  る。野菜を加え、弱火で約4時間煮る。塩は最初に全量を入れないこと。
  必要なら作業の最終段階でも塩を加える。}}{白いコンソメ・サンプル}}\label{ux767dux3044ux30b3ux30f3ux30bdux30e1ux30b5ux30f3ux30d7ux30eb1}}

(仕上がり 10 L分)

\begin{itemize}
\item
  主素材\ldots{}\ldots{}牛赤身肉4 kgと牛骨付きすね肉3 kg。
\item
  香味素材\ldots{}\ldots{}にんじん1.1 kg (5〜6本)、かぶ900
  g(5〜6ヶ)、ポワロー200 g、パース ニップ\footnote{panaisパネ。和名アメリカボウフウ。セリ科の根菜で、香りが良い。
    白く、にんじんに似た円錐形のため、俗に「白にんじん」と呼ばれること
    もあるが、にんじんとは別種。でんぷん質が豊富で、ピュレ等の調理にも
    適している。}200 g、玉ねぎ(中)2ヶ(200
  g)、クローブ3本、にんにく3片(20 g)、 セロリ120 g。
\item
  加える液体\ldots{}\ldots{}水14 L。
\item
  調味料\ldots{}\ldots{}粗塩70 g。
\item
  加熱時間\ldots{}\ldots{}5時間。
\end{itemize}

\hypertarget{ux4f5cux308aux65b9ux306bux95a2ux3059ux308bux88dcux8db33}{%
\paragraph[作り方に関する補足]{\texorpdfstring{作り方に関する補足\footnote{この部分は第二版で加筆された。}}{作り方に関する補足}}\label{ux4f5cux308aux65b9ux306bux95a2ux3059ux308bux88dcux8db33}}

\ldots{}\ldots{}コンソメ・サンプルを作る際、一般的には5時間かけて煮ることになっている。
肉汁を抽出するには充分な時間である。

しかし、骨の組織を壊して可溶性物質を確実に抽出するには5時間では絶対に
足りない。骨から可溶性物質を抽出することはとても重要だが、そのためには
弱火で12〜15時間煮る必要がある。

だから、グランド・キュイジーヌでは、粗く砕いた骨を12時間以上煮て第1の
コンソメをとるようになってきている。

この第1のコンソメを第2のコンソメをとる鍋に注ぐ。この鍋で肉を約4時間、
すなわち肉を煮るのに最低限必要な時間、火にかける。

肉を野菜を塊のままではなく細かく刻めば、2つめの作業時間をさらに短かく
することも可能だ。その場合は、通常のクラリフィエと同様の作業となる。
(「クラリフィエ」の項参照)。
\end{recette}
\hypertarget{ux30afux30e9ux30eaux30d5ux30a3ux30a84}{%
\subsection[クラリフィエ]{\texorpdfstring{クラリフィエ\footnote{原文
  clarifications クラリフィカシオン (動詞 clarifier「澄ませ
  る」の名詞形)。字義通りには「澄ませる作業」だが、実際にはコンソメ・
  ドゥーブルconsommé double (コンソメ・リッシュ consommé richeコンソ
  メ・クラリフィエ consommé clarifiéとも呼ばれる)を作ることを意味す
  る。本来はその工程のひとつであった「澄ませる作業」が作業全体を指す
  語として定着したのだろう。}}{クラリフィエ}}\label{ux30afux30e9ux30eaux30d5ux30a3ux30a84}}

\frsecb{Clarifications}

\hypertarget{ux901aux5e38ux306eux30b3ux30f3ux30bdux30e1}{%
\subsubsection{通常のコンソメ}\label{ux901aux5e38ux306eux30b3ux30f3ux30bdux30e1}}

(仕上がり4 L分)

\begin{itemize}
\item
  白いコンソメ・サンプル\ldots{}\ldots{}5 L。
\item
  主素材\ldots{}\ldots{}牛赤身肉1.5
  kg。丁寧に筋を除き、挽いておく\footnote{原文 hacher
    アシェ(細かく刻む)。語源は hacheアーシュ(斧)。日本
    語の「刻む」は包丁を用い、「挽く」はミートチョッパーのような器具を
    用いる場合を指すが、フランス語では区別せずどちらもhacher と表現す
    る。}。
\item
  香味素材\ldots{}\ldots{}にんじん100 g、ポワロー200
  g。小さなさいの目\footnote{brunoise ブリュノワーズ}に刻んでおく。
\item
  澄ませるための素材\ldots{}\ldots{}卵白2ヶ分。
\item
  所要時間\ldots{}\ldots{}1時間半。
\item
  作業\ldots{}\ldots{}片手鍋\footnote{casserole カスロール}または小ぶりの寸胴鍋\footnote{marmiteマルミート。一般的には、大型で深さが直径以上ある両手鍋を
    指す。}に牛挽肉、小さなさいの目に刻
  んだ野菜、卵白を入れ、全体をよく混ぜる。白いコンソメ・サンプルを注ぎ入
  れ、時々混ぜながら\footnote{ここは原文に忠実に訳したが、実際には常に混ぜ続けないと卵白が鍋
    底にくっついて無駄になってしまう。『ラルース・ガストロノミック』初
    版では「絶えず混ぜる」ように指示されている。}沸騰させる。軽く沸騰させながら1時間半煮る。
\end{itemize}

布で漉して仕上げる。

\hypertarget{ux9d8fux306eux30b3ux30f3ux30bdux30e1}{%
\subsubsection{鶏のコンソメ}\label{ux9d8fux306eux30b3ux30f3ux30bdux30e1}}

(仕上がり4 L分)**

*白いコンソメ・サンプル\ldots{}\ldots{}同上。

\begin{itemize}
\item
  主素材と香味素材\ldots{}\ldots{}同上に、以下を加える。オーヴンで軽く色づけた鶏1羽。
  鶏の首づる、手羽先、足など\footnote{原文
    abatisアバティ。鶏肉として食べられる以外の部位の総称。
    鶏の「内臓」と訳されることが多いが、とさか、頭、首づる、手羽先、足
    なども含まれる。}を刻んだもの6羽分。ロティールした鶏
  のがら\footnote{鶏のロティ(ローストチキン)を提供した際に出る「がら」。}2羽分。
\item
  澄ませるための素材、方法、時間は通常のコンソメと同様にする。
\end{itemize}
\newpage
\href{原稿下準備20180414五島、連載からコピー}{} \href{訳と注釈}{}
\href{未、原文対照チェック}{} \href{未、日本語表現校正}{}
\href{未、その他修正}{} \href{未、原稿最終校正}{}

\hypertarget{ux30ddux30bfux30fcux30b8ux30e5ux306eux4e3bux306aux6d6eux304dux5b9fux30acux30ebux30cbux30c1ux30e5ux30fcux30eb}{%
\section{ポタージュの主な浮き実(ガルニチュール)}\label{ux30ddux30bfux30fcux30b8ux30e5ux306eux4e3bux306aux6d6eux304dux5b9fux30acux30ebux30cbux30c1ux30e5ux30fcux30eb}}

\frsec{Elémtent divers de Garnitures pour Potages}
\newpage
\href{スミ、原稿下準備\%20amanojack0615-20180510}{}
\href{未、原文対照チェック}{} \href{未、日本語表現校正}{}
\href{未、その他修正}{} \href{未、原稿最終校正}{}

\hypertarget{potages-claires-et-consommes-garnis}{%
\section{澄んだポタージュ、ガルニチュール入りコンソメ}\label{potages-claires-et-consommes-garnis}}

\frsec{Série des Potages claires et Consommés garnis}

\index{potages@potages!claire@---s claires}
\index{consomme@consommé!garnis@---s garnis}
\index{ほたーしゆ@ポタージュ!すんた@澄んだ---}
\index{こんそめ@コンソメ!かるにちゆーるいり@ガルニチュール入り---}

\hypertarget{nota-potages-claires-et-consommes-garnis}{%
\paragraph{【原注】}\label{nota-potages-claires-et-consommes-garnis}}

\href{コメント\ldots{}\ldots{}この部分は無視してください。この下からスタートしてください}{}
\begin{recette}
\hypertarget{consomme-aux-ailerons}{%
\subsubsection{コンソメ・鶏手羽入り}\label{consomme-aux-ailerons}}

\frsub{Consommé aux ailerons}

\index{consomme@consommé!ailerons@--- ailerons}
\index{aileron@aileron!consomme@consommé ---}
\index{こんそめ@コンソメ!とりてはいり@---・鶏手羽入り}
\index{とりては@鶏手羽!こんそめ@コンソメ・---入り}

\hypertarget{consomme-alexandra}{%
\subsubsection{コンソメ・アレクサンドラ}\label{consomme-alexandra}}

\frsub{Consommé Alexandra}

\index{consomme@consommé!alexandra@--- Alexandra}
\index{alexandra@Alexandra!consomme@consommé ---}
\index{こんそめ@コンソメ!あれくさんとら@---・アレクサンドラ}
\index{あれくさんとら@アレクサンドラ!こんそめ@コンソメ・---}

\hypertarget{consomme-a-l-ancienne}{%
\subsubsection{コンソメ・クラシック}\label{consomme-a-l-ancienne}}

\frsub{Consommé à l'Ancienne}

\index{consomme@consommé!ancienne@--- Ancienne}
\index{ancienne@Ancienne!consomme@consommé ---}
\index{こんそめ@コンソメ!くらしつく@---・クラシック}
\index{くらしつく@クラシック!こんそめ@コンソメ・---}

\hypertarget{consomme-d-arenberg}{%
\subsubsection{コンソメ・アーレンベルク}\label{consomme-d-arenberg}}

\frsub{Consommé d'Arenberg}

\index{consomme@consommé!arenberg@--- Arenberg}
\index{arenberg@Arenberg!consomme@consommé ---}
\index{こんそめ@コンソメ!あーれんへるく@---・アーレンベルク}
\index{あーれんへるく@アーレンベルク!こんそめ@コンソメ・---}

\hypertarget{consomme-a-l-aurore}{%
\subsubsection{コンソメ・オーロール}\label{consomme-a-l-aurore}}

\frsub{Consommé à l'Aurore}

\index{consomme@consommé!aurore@--- Aurore}
\index{aurore@Aurore!consomme@consommé ---}
\index{こんそめ@コンソメ!おーろーる@---・オーロール}
\index{おーろーる@オーロール!こんそめ@コンソメ・---}

\hypertarget{consomme-belle-fermiere}{%
\subsubsection{コンソメ・ベル・フェルミエール}\label{consomme-belle-fermiere}}

\frsub{Consommé Belle Fermière}

\index{consomme@consommé!bellefermiere@--- Belle Fermière}
\index{bellefermiere@Bell Fermière!consomme@consommé ---}
\index{こんそめ@コンソメ!へるふえるみえーる@---・ベル・フェルミエール}
\index{へるふえるみえーる@ベル・フェルミエール!こんそめ@コンソメ・---}

\hypertarget{consomme-bellini}{%
\subsubsection{コンソメ・ベッリーニ}\label{consomme-bellini}}

\frsub{Consommé Bellini}

\index{consomme@consommé!bellini@--- Bellini}
\index{bellini@Bellini!consomme@consommé ---}
\index{こんそめ@コンソメ!へつりーに@---・ベッリーニ}
\index{へつりーに@ベッリーニ!こんそめ@コンソメ・---}

\hypertarget{consomme-a-la-bouchere}{%
\subsubsection{コンソメ・ブーシェール}\label{consomme-a-la-bouchere}}

\frsub{Consommé à la Bouchère}

\index{consomme@consommé!bouchere@--- Bouchère}
\index{bouchere@Bouchère!consomme@consommé ---}
\index{こんそめ@コンソメ!ふーしえーる@---・ブーシェール}
\index{ふーしえーる@ブーシェール!こんそめ@コンソメ・---}

\hypertarget{consomme-a-la-bouquetiere}{%
\subsubsection{コンソメ・ブクティエール}\label{consomme-a-la-bouquetiere}}

\frsub{Consommé à la Bouquetière}

\index{consomme@consommé!bouquetiere@--- Bouquetière}
\index{bouquetiere@Bouquetière!consomme@consommé ---}
\index{こんそめ@コンソメ!ふくていえーる@---・ブクティエール}
\index{ふくていえーる@ブクティエール!こんそめ@コンソメ・---}

\href{ガルニチュールで「ブクティエール」にしているので合わさせてください}{}

\hypertarget{consomme-a-la-brunoise}{%
\subsubsection{コンソメ・ブリュノワーズ}\label{consomme-a-la-brunoise}}

\frsub{Consommé à la Brunoise}

\index{consomme@consommé!brunoise@--- Brunoise}
\index{brunoise@Brunoise!consomme@consommé ---}
\index{こんそめ@コンソメ!ふりゆのわーす@---・ブリュノワーズ}
\index{ふりゆのわーす@ブリュノワーズ!こんそめ@コンソメ・---}

\hypertarget{consomme-carmen}{%
\subsubsection{コンソメ・カルメン}\label{consomme-carmen}}

\frsub{Consommé Carmen}

\index{consomme@consommé!carmen@--- Carmen}
\index{carmen@Carmen!consomme@consommé ---}
\index{こんそめ@コンソメ!かるめん@---・カルメン}
\index{かるめん@カルメン!こんそめ@コンソメ・---}

\href{ココからセレスティーヌまで項目のダブリがあったので削除しました。20180407五島}{}

\hypertarget{consomme-celestine}{%
\subsubsection{コンソメ・ケレスティヌス}\label{consomme-celestine}}

\frsub{Consommé Célestine}

\index{consomme@consommé!celestine@--- Célestine}
\index{celestine@Célestine!consomme@consommé ---}
\index{こんそめ@コンソメ!けれすていぬす@---・ケレスティヌス}
\index{けれすていぬす@ケレスティヌス!こんそめ@コンソメ・---}

\hypertarget{consomme-cendrillon}{%
\subsubsection{コンソメ・シンデレラ}\label{consomme-cendrillon}}

\frsub{Consommé Cendrillon}

\index{consomme@consommé!cendrillon@--- Cendrillon}
\index{cendrillon@Cendrillon!consomme@consommé ---}
\index{こんそめ@コンソメ!しんてれら@---・シンデレラ}
\index{しんてれら@シンデレラ!こんそめ@コンソメ・---}

\hypertarget{consomme-chanceliere}{%
\subsubsection{コンソメ・ションスリエール}\label{consomme-chanceliere}}

\frsub{Consommé Chancelière}

\index{consomme@consommé!chanceliere@--- Chancelière}
\index{chanceliere@Chancelière!consomme@consommé ---}
\index{こんそめ@コンソメ!しよんすりえーる@---・ションスリエール}
\index{しよんすりえーる@ションスリエール!こんそめ@コンソメ・---}

\hypertarget{consomme-au-chasseur}{%
\subsubsection{コンソメ・シャッスール}\label{consomme-au-chasseur}}

\frsub{Consommé au Chasseur}

\index{consomme@consommé!chasseur@--- Chasseur}
\index{chasseur@Chasseur!consomme@consommé ---}
\index{こんそめ@コンソメ!しやつすーる@---・シャッスール}
\index{しやつすーる@シャッスール!こんそめ@コンソメ・---}

\hypertarget{consomme-chatelaine}{%
\subsubsection{コンソメ・シャトレーヌ}\label{consomme-chatelaine}}

\frsub{Consommé Châtelaine }

\index{consomme@consommé!chatelaine@---Châtelaine}
\index{chatelaine@Châtelaine!consomme@consommé ---}
\index{こんそめ@コンソメ!しやとれーぬ@---・シャトレーヌ}
\index{しやとれーぬ@シャトレーヌ!こんそめ@コンソメ・---}

\href{以下項目のダブリがあったので削除しました}{}

\hypertarget{consomme-aux-cheveux-d-ange}{%
\subsubsection{コンソメ・ヴァミセリ入り}\label{consomme-aux-cheveux-d-ange}}

\frsub{Consommé aux Cheveux d'ange}

\index{consomme@consommé!cheveud'ange@--- Cheveux d'ange}
\index{cheveud'ange@Cheveux d'ange!consomme@consommé ---}
\index{こんそめ@コンソメ!うあみせり@---・ヴァミセリ}
\index{うあみせり@ヴァミセリ!こんそめ@コンソメ・---}

\hypertarget{consomme-colbert}{%
\subsubsection{コンソメ・コルべール}\label{consomme-colbert}}

\frsub{Consommé Colbert}

\index{consomme@consommé!colbert@--- Colbert}
\index{colbert@Colbert!consomme@consommé ---}
\index{こんそめ@コンソメ!こるへーる@---・コルベール}
\index{こるへーる@コルベール!こんそめ@コンソメ・---}

\hypertarget{consomme-colombine}{%
\subsubsection{コンソメ・コロンバン}\label{consomme-colombine}}

\frsub{Consommé Colombine}

\index{consomme@consommé!colombine@--- Colombine}
\index{colombine@Colombine!consomme@consommé ---}
\index{こんそめ@コンソメ!ころんはん@---・コロンバン}
\index{ころんはん@コロンバン!こんそめ@コンソメ・---}

\hypertarget{consomme-croute-au-pot}{%
\subsubsection{コンソメ・クルトン入り}\label{consomme-croute-au-pot}}

\frsub{Consommé Croûte au pot}

\index{consomme@consommé!crouteaupot@--- Croûte au pot}
\index{croûteaupot@Croûte au pot!consomme@consommé ---}
\index{こんそめ@コンソメ!くるとん@---・クルトン入り}
\index{くるとん@クルトン入り!こんそめ@コンソメ・---}

\hypertarget{consomme-cyrano}{%
\subsubsection{コンソメ・シラノ}\label{consomme-cyrano}}

\frsub{Consommé Cyrano}

\index{consomme@consommé!cyrano@--- Cyrano}
\index{cyrano@Cyrano!consomme@consommé ---}
\index{こんそめ@コンソメ!しらの@---・シラノ}
\index{しらの@シラノ!こんそめ@コンソメ・---}

\hypertarget{consomme-dame-blanche}{%
\subsubsection{コンソメ・ダム・ブランシュ}\label{consomme-dame-blanche}}

\frsub{Consommé Dame Blanche}

\index{consomme@consommé!dameblanche@--- Dame Blanche}
\index{dameblanche@Dame Blanche!consomme@consommé ---}
\index{こんそめ@コンソメ!たむふらんしゆ@---・ダム・ブランシュ}
\index{たむふらんしゆ@ダム・ブランシュ!こんそめ@コンソメ・---}

\hypertarget{consomme-demidoff}{%
\subsubsection{コンソメ・ドゥミドフ}\label{consomme-demidoff}}

\frsub{Consommé Demidoff}

\index{consomme@consommé!demidoff@--- Domidoff}
\index{demidoff@Demidoff!consomme@consommé ---}
\index{こんそめ@コンソメ!とうみとふ@---・ドゥミドフ}
\index{とうみとふ@ドゥミドフ!こんそめ@コンソメ・---}

\hypertarget{consomme-deslignac}{%
\subsubsection{コンソメ・デリニャック}\label{consomme-deslignac}}

\frsub{Consommé Deslignac}

\index{consomme@consommé!deslignac@--- Deslignac}
\index{deslignac@Deslignac!consomme@consommé ---}
\index{こんそめ@コンソメ!てりにやつく@---・デリニャック}
\index{てりにやつく@デリニャック!こんそめ@コンソメ・---}

\hypertarget{consomme-aux-diablotins}{%
\subsubsection{コンソメ・小悪魔風}\label{consomme-aux-diablotins}}

\frsub{Consommé aux Diablotins}

\index{consomme@consommé!diablotin@--- Diablotins}
\index{diablotin@Diablotins!consomme@consommé ---}
\index{こんそめ@コンソメ!こあくまふう@---・小悪魔風}
\index{こあくまふう@小悪魔風!こんそめ@コンソメ・---}

\hypertarget{consomme-diane}{%
\subsubsection{コンソメ・ディアーヌ}\label{consomme-diane}}

\frsub{Consommé Diane}

\index{consomme@consommé!diane@--- Diane}
\index{diane@Diane!consomme@consommé ---}
\index{こんそめ@コンソメ!ていあーぬ@---・ディアーヌ}
\index{ていあーぬ@ディアーヌ!こんそめ@コンソメ・---}

\href{ソース・ディアーヌとしているのに合わさせてもらいました20180407五島}{}

\hypertarget{consomme-diplomate}{%
\subsubsection{コンソメ・ディプロマ}\label{consomme-diplomate}}

\frsub{Consommé Diplomate}

\index{consomme@consommé!diplomate@--- Diplomate}
\index{diplomate@Diplomate!consomme@consommé ---}
\index{こんそめ@コンソメ!ていふろま@---・ディプロマ}
\index{ていふろま@ディプロマ!こんそめ@コンソメ・---}

\hypertarget{consomme-divette}{%
\subsubsection{コンソメ・ディヴェット}\label{consomme-divette}}

\frsub{Consommé Divette}

\index{consomme@consommé!divette@--- Divette}
\index{divette@Divette!consomme@consommé ---}
\index{こんそめ@コンソメ!ていうえつと@---・ディヴェット}
\index{ていうえつと@ディヴェット!こんそめ@コンソメ・---}

\hypertarget{consomme-dominicaine}{%
\subsubsection{コンソメ・ドミニコ修道会風}\label{consomme-dominicaine}}

\frsub{Consommé Dominicaine}

\index{consomme@consommé!dominicaine@--- Dominicaine}
\index{dominicaine@Dominicaine!consomme@consommé ---}
\index{こんそめ@コンソメ!とみにこしゆうとうかいふう@---・ドミニコ修道会風}
\index{とみにこしゆうとうかいふう@ドミニコ修道会風!こんそめ@コンソメ・---}

\hypertarget{consomme-doria}{%
\subsubsection{コンソメ・ドリア}\label{consomme-doria}}

\frsub{Consommé Doria}

\index{consomme@consommé!doria@--- Doria}
\index{doria@Doria!consomme@consommé ---}
\index{こんそめ@コンソメ!とりあ@---・ドリア}
\index{とりあ@ドリア!こんそめ@コンソメ・---}

\hypertarget{consomme-douglas}{%
\subsubsection{コンソメ・ドゥーグラ}\label{consomme-douglas}}

\frsub{Consommé Douglas}

\index{consomme@consommé!douglas@--- Douglas}
\index{douglas@Douglas!consomme@consommé ---}
\index{こんそめ@コンソメ!とうーくら@---・ドゥーグラ}
\index{とうーくら@ドゥーグラ!こんそめ@コンソメ・---}

\hypertarget{consomme-dubarry}{%
\subsubsection{コンソメ・デュバリー}\label{consomme-dubarry}}

\frsub{Consommé Dubarry}

\index{consomme@consommé!dubarry@--- Dubarry}
\index{dubarry@Dubarry!consomme@consommé ---}
\index{こんそめ@コンソメ!てゆはりー@---・デュバリー}
\index{てゆはりー@デュバリー!こんそめ@コンソメ・---}

\hypertarget{consomme-a-l-ecossaise}{%
\subsubsection{コンソメ・スコットランド風}\label{consomme-a-l-ecossaise}}

\frsub{Consommé à l'Ecossaise}

\index{consomme@consommé!ecossaise@--- Ecossaise}
\index{ecossaise@Ecossaise!consomme@consommé ---}
\index{こんそめ@コンソメ!すこつとらんとふう@---・スコットランド風}
\index{すこつとらんとふう@スコットランド風!こんそめ@コンソメ・---}

\hypertarget{consomme-edouard-vii}{%
\subsubsection{コンソメ・エドワード7世}\label{consomme-edouard-vii}}

\frsub{Consommé Edouard VII}

\index{consomme@consommé!edouard vii@--- Edouard VII}
\index{edouard vii@Edouard VII!consomme@consommé ---}
\index{こんそめ@コンソメ!えとわーとななせい@---・エドワード7世}
\index{えとわーとななせい@エドワード7世!こんそめ@コンソメ・---}

\hypertarget{consomme-flavigny}{%
\subsubsection{コンソメ・フラヴィニー}\label{consomme-flavigny}}

\frsub{Consommé Flavigny}

\index{consomme@consommé!flavigny@--- Flavigny}
\index{flavigny@Flavigny!consomme@consommé ---}
\index{こんそめ@コンソメ!ふらういにー@---・フラヴィニー}
\index{ふらういにー@フラヴィニー!こんそめ@コンソメ・---}

\hypertarget{consomme-floreal}{%
\subsubsection{コンソメ・フロレアル}\label{consomme-floreal}}

\frsub{Consommé Floréal}

\index{consomme@consommé!floreal@--- Floréal}
\index{floréal@Floréal!consomme@consommé ---}
\index{こんそめ@コンソメ!ふろれある@---・フロレアル}
\index{ふろれある@フロレアル!こんそめ@コンソメ・---}

\href{パッチのコメントに書いたとおりの理由からフロレアルとさせていただきました。20180407五島}{}

\hypertarget{consomme-florentine}{%
\subsubsection{コンソメ・フロランタン}\label{consomme-florentine}}

\frsub{Consommé Florentine}

\index{consomme@consommé!frolentine@--- Florentine}
\index{florentine@Florentine!consomme@consommé ---}
\index{こんそめ@コンソメ!ふろらんたん@---・フロランタン}
\index{ふろらんたん@フロランタン!こんそめ@コンソメ・---}

\hypertarget{consomme-florian}{%
\subsubsection{コンソメ・フロリアン}\label{consomme-florian}}

\frsub{Consommé Florian}

\index{consomme@consommé!florian@--- Florian}
\index{florian@Florian!consomme@consommé ---}
\index{こんそめ@コンソメ!ふろりあん@---・フロリアン}
\index{ふろりあん@フロリアン!こんそめ@コンソメ・---}

\hypertarget{consomme-a-la-gauloise}{%
\subsubsection{コンソメ・ガリア風}\label{consomme-a-la-gauloise}}

\frsub{Consommé à la Gauloise}

\index{consomme@consommé!gauloise@--- Gauloise}
\index{gauloise@Gauloise!consomme@consommé ---}
\index{こんそめ@コンソメ!かりあふう@---・ガリア風}
\index{かりあふう@ガリア風!こんそめ@コンソメ・---}

\hypertarget{consomme-georges-V}{%
\subsubsection{コンソメ・ジョルジュⅤ世風}\label{consomme-georges-V}}

\frsub{Consommé Georges V}

\index{consomme@consommé!georgesv@--- Georges V}
\index{georgesv@Georges V!consomme@consommé ---}
\index{こんそめ@コンソメ!しよるしゆこせいふう@---・ジョルジュⅤ世風}
\index{しよるしゆこせいふう@ジョルジュⅤ世風!こんそめ@コンソメ・---}

\hypertarget{consomme-germinal}{%
\subsubsection{コンソメ・ジェルミナル}\label{consomme-germinal}}

\frsub{Consommé Germinal}

\index{consomme@consommé!germinal@--- Germinal}
\index{germinal@Germinal!consomme@consommé ---}
\index{こんそめ@コンソメ!しえるみなる@---・ジェルミナル}
\index{しえるみなる@ジェルミナル!こんそめ@コンソメ・---}

\hypertarget{consomme-gladiateur}{%
\subsubsection{コンソメ・剣闘士風}\label{consomme-gladiateur}}

\frsub{Consommé Gladiateur}

\index{consomme@consommé!gladiateur@--- Gladiateur}
\index{gladiateur@Gladiateur!consomme@consommé ---}
\index{こんそめ@コンソメ!けんとうしふう@---・剣闘士風}
\index{けんとうしふう@剣闘士風!こんそめ@コンソメ・---}

\hypertarget{consomme-grimaldi}{%
\subsubsection{コンソメ・グリマルディ}\label{consomme-grimaldi}}

\frsub{Consommé Grimaldi}

\index{consomme@consommé!grimaldi@--- Grimaldi}
\index{grimaldi@Grimaldi!consomme@consommé ---}
\index{こんそめ@コンソメ!くりまるてい@---・くりまるてい}
\index{グリマルディ@くりまるてい!こんそめ@コンソメ・---}

\hypertarget{consomme-helene}{%
\subsubsection{コンソメ・エレーヌ}\label{consomme-helene}}

\frsub{Consommé Helène}

\index{consomme@consommé!helène@--- Helène}
\index{helène@Helène!consomme@consommé ---}
\index{こんそめ@コンソメ!えれーぬ@---・エレーヌ}
\index{えれーぬ@エレーヌ!こんそめ@コンソメ・---}

\hypertarget{consomme-henriette}{%
\subsubsection{コンソメ・アンリエッテ}\label{consomme-henriette}}

\frsub{Consommé Henriette}

\index{consomme@consommé!henriette@--- Henriette}
\index{henriette@Henriette!consomme@consommé ---}
\index{こんそめ@コンソメ!あんりえつて@---・アンリエッテ}
\index{アンリエッテ@あんりえつて!こんそめ@コンソメ・---}

\hypertarget{consomme-a-indienne}{%
\subsubsection{コンソメ・インド風}\label{consomme-a-indienne}}

\frsub{Consommé à l'Indienne}

\index{consomme@consommé!indien(ne)@--- à l'Indienne}
\index{indien(ne)@Indienne!consomme@consommé à l' ---}
\index{こんそめ@コンソメ!いんとふう@---・インド風}
\index{いんとふう@インド風!こんそめ@コンソメ・---}

\hypertarget{consomme-a-l-infante}{%
\subsubsection{コンソメ・親王風}\label{consomme-a-l-infante}}

\frsub{Consommé à l'Infante}

\index{consomme@consommé!infante@--- à l'Infante}
\index{infante@Infante!consomme@consommé à l' ---}
\index{こんそめ@コンソメ!しんのうふう@---・親王風}
\index{しんのうふう@親王風!こんそめ@コンソメ・---}

\hypertarget{consomme-isabelle-de-france}{%
\subsubsection{コンソメ・イザベル・ド・フランス風}\label{consomme-isabelle-de-france}}

\frsub{Consommé Isabelle de France}

\index{consomme@consommé!isabelledefrance@--- Isabelle de France}
\index{isabelledefrance@Isabelle de France!consomme@consommé ---}
\index{こんそめ@コンソメ!いさへるとふらんすふう@---・イザベル・ド・フランス風}
\index{いさへるとふらんすふう@イザベル・ド・フランス風!こんそめ@コンソメ・---}

\hypertarget{consomme-ivan}{%
\subsubsection{コンソメ・イヴァン}\label{consomme-ivan}}

\frsub{Consommé Ivan}

\index{consomme@consommé!ivan@--- Ivan}
\index{ivan@Ivan!consomme@consommé ---}
\index{こんそめ@コンソメ!いうあん@---・イヴァン}
\index{いうあん@イヴァン!こんそめ@コンソメ・---}

\hypertarget{consomme-jeanne-garnier}{%
\subsubsection{コンソメ・ジャン・ガルニエ}\label{consomme-jeanne-garnier}}

\frsub{Consommé Jeanne Garnier}

\index{consomme@consommé!jeanne garnier@--- Jeanne Garnier}
\index{jeanne garnier@Jeanne Garnier!consomme@consommé ---}
\index{こんそめ@コンソメ!しやんかるにえ@---・ジャン・ガルニエ}
\index{しやんかるにえ@ジャン・ガルニエ!こんそめ@コンソメ・---}

\hypertarget{consomme-judic}{%
\subsubsection{コンソメ・ジュディク}\label{consomme-judic}}

\frsub{Consommé Judic}

\index{consomme@consommé!judic@--- Judic}
\index{judic@Judic!consomme@consommé ---}
\index{こんそめ@コンソメ!しゆていく@---・ジュディク}
\index{しゆていく@ジュディク!こんそめ@コンソメ・---}

\hypertarget{potage-julienne}{%
\subsubsection{ポタージュ・ジュリエンヌ}\label{potage-julienne}}

\frsub{Potage Julienne}

\index{potage@potage!julienne@--- Julienne}
\index{julienne@Julienne!potage@potage ---}
\index{ほたーしゆ@ポタージュ!しゆりえんぬ@---・ジュリエンヌ}
\index{しゆりえんぬ@ジュリエンヌ!ほたーしゆ@ポタージュ・---}

\hypertarget{consomme-juliette}{%
\subsubsection{コンソメ・ジュリエット}\label{consomme-juliette}}

\frsub{Consommé Juliette}

\index{consomme@consommé!juliette@--- Juliette}
\index{juliette@Juliette!consomme@consommé ---}
\index{こんそめ@コンソメ!しゆりえつと@---・ジュリエット}
\index{しゆりえつと@ジュリエット!こんそめ@コンソメ・---}

\hypertarget{consomme-kleber}{%
\subsubsection{コンソメ・クレーベル}\label{consomme-kleber}}

\frsub{Consommé Kléber}

\index{consomme@consommé!kleber@--- Kléber}
\index{kleber@Kléber!consomme@consommé ---}
\index{こんそめ@コンソメ!くれーへる@---・クレーベル}
\index{くれーへる@クレーベル!こんそめ@コンソメ・---}

\hypertarget{consomme-la-perouse}{%
\subsubsection{コンソメ・ラ・ペルーズ風}\label{consomme-la-perouse}}

\frsub{Consommé La Pérouse}

\index{consomme@consommé!laperouse@--- La Pérouse}
\index{laperouse@La Pérouse!consomme@consommé ---}
\index{こんそめ@コンソメ!らへるーすふう@---・ラ・ペルーズ風}
\index{らへるーすふう@ラ・ペルーズ風!こんそめ@コンソメ・---}

\hypertarget{consomme-lorette}{%
\subsubsection{コンソメ・ロレット}\label{consomme-lorette}}

\frsub{Consommé Lorette}

\index{consomme@consommé!lorette@--- Lorette}
\index{lorette@Lorette!consomme@consommé ---}
\index{こんそめ@コンソメ!ろれつと@---・ロレット}
\index{ろれつと@ロレット!こんそめ@コンソメ・---}

\hypertarget{consomme-lucette}{%
\subsubsection{コンソメ・ルセット}\label{consomme-lucette}}

\frsub{Consommé Lucette}

\index{consomme@consommé!lucette@--- Lucette}
\index{lucette@Lucette!consomme@consommé ---}
\index{こんそめ@コンソメ!るせつと@---・ルセット}
\index{るせつと@ルセット!こんそめ@コンソメ・---}

\hypertarget{consomme-lucullus}{%
\subsubsection{コンソメ・ルクッルス}\label{consomme-lucullus}}

\frsub{Consommé Lucullus}

\index{consomme@consommé!lucullus@--- Lucullus}
\index{lucullus@Lucullus!consomme@consommé ---}
\index{こんそめ@コンソメ!るくつるす@---・ルクッルス}
\index{るくつるす@ルクッルス!こんそめ@コンソメ・---}

\hypertarget{consomme-a-la-madrilene}{%
\subsubsection{コンソメ・マドリッド風}\label{consomme-a-la-madrilene}}

\frsub{Consommé à la Madrilène}

\index{consomme@consommé!madrilene@--- Madrilène}
\index{madrilene@Madrilène!consomme@consommé à la ---}
\index{こんそめ@コンソメ!まとりつとふう@---・マドリッド風}
\index{まとりつとふう@マドリッド風!こんそめ@コンソメ・---}

\hypertarget{consomme-maintenon}{%
\subsubsection{コンソメ・マントノン}\label{consomme-maintenon}}

\frsub{Consommé Maintenon}

\index{consomme@consommé!maintenon@--- Maintenon}
\index{maintenon@Maintenon!consomme@consommé ---}
\index{こんそめ@コンソメ!まんとのん@---・マントノン}
\index{まんとのん@マントノン!こんそめ@コンソメ・---}

\hypertarget{consomme-messaline}{%
\subsubsection{コンソメ・メッザリーヌ}\label{consomme-messaline}}

\frsub{Consommé Messaline}

\index{consomme@consommé!messaline@--- Messaline}
\index{messaline@Messaline!consomme@consommé ---}
\index{こんそめ@コンソメ!めつさりーぬ@---・メッザリーヌ}
\index{めつさりーぬ@メッザリーヌ!こんそめ@コンソメ・---}

\hypertarget{consomme-midinette}{%
\subsubsection{コンソメ・ミディネット}\label{consomme-midinette}}

\frsub{Consommé Midinette}

\index{consomme@consommé!midinette@--- Midinette}
\index{midinette@Midinette!consomme@consommé ---}
\index{こんそめ@コンソメ!みていねつと@---・ミディネット}
\index{みていねつと@ミディネット!こんそめ@コンソメ・---}

\hypertarget{consomme-mikado}{%
\subsubsection{コンソメ・ミカド}\label{consomme-mikado}}

\frsub{Consommé Mikado}

\index{consomme@consommé!mikado@--- Mikado}
\index{mikado@Mikado!consomme@consommé ---}
\index{こんそめ@コンソメ!みかと@---・ミカド}
\index{みかと@ミカド!こんそめ@コンソメ・---}

\hypertarget{consomme-mireille}{%
\subsubsection{コンソメ・ミレイユ}\label{consomme-mireille}}

\frsub{Consommé Mireille}

\index{consomme@consommé!mireille@--- Mireille}
\index{mireille@Mireille!consomme@consommé ---}
\index{こんそめ@コンソメ!みれいゆ@---・ミレイユ}
\index{みれいゆ@ミレイユ!こんそめ@コンソメ・---}

\hypertarget{consomme-mirette}{%
\subsubsection{コンソメ・ミレット}\label{consomme-mirette}}

\frsub{Consommé Mirette}

\index{consomme@consommé!mirette@--- Mirette}
\index{mirette@Mirette!consomme@consommé ---}
\index{こんそめ@コンソメ!みれつと@---・ミレット}
\index{みれつと@ミレット!こんそめ@コンソメ・---}

\hypertarget{consomme-mistral}{%
\subsubsection{コンソメ・ミストラル}\label{consomme-mistral}}

\frsub{Consommé Mistral}

\index{consomme@consommé!mistral@--- Mistral}
\index{mistral@Mistral!consomme@consommé ---}
\index{こんそめ@コンソメ!みすとらる@---・ミストラル}
\index{みすとらる@ミストラル!こんそめ@コンソメ・---}

\hypertarget{consomme-monsigny}{%
\subsubsection{コンソメ・モンシニー}\label{consomme-monsigny}}

\frsub{Consommé Monsigny}

\index{consomme@consommé!monsigny@--- Monsigny}
\index{monsigny@Monsigny!consomme@consommé ---}
\index{こんそめ@コンソメ!もんしにー@---・モンシニー}
\index{もんしにー@モンシニー!こんそめ@コンソメ・---}

\hypertarget{consomme-montespan}{%
\subsubsection{コンソメ・モンテスパン}\label{consomme-montespan}}

\frsub{Consommé Montespan}

\index{consomme@consommé!montespan@--- Montespan}
\index{montespan@Montespan!consomme@consommé ---}
\index{こんそめ@コンソメ!もんてすはん@---・モンテスパン}
\index{もんてすはん@モンテスパン!こんそめ@コンソメ・---}

\hypertarget{consomme-montmorency}{%
\subsubsection{コンソメ・モンモランシー}\label{consomme-montmorency}}

\frsub{Consommé Montmorency}

\index{consomme@consommé!montmorency@--- Montmorency}
\index{montmorency@Montmorency!consomme@consommé ---}
\index{こんそめ@コンソメ!もんもらんしー@---・モンモランシー}
\index{もんもらんしー@モンモランシー!こんそめ@コンソメ・---}

\hypertarget{consomme-murat}{%
\subsubsection{コンソメ・ミュラ}\label{consomme-murat}}

\frsub{Consommé Murat}

\index{consomme@consommé!murat@--- Murat}
\index{murat@Murat!consomme@consommé ---}
\index{こんそめ@コンソメ!みゆら@---・ミュラ}
\index{みゆら@ミュラ!こんそめ@コンソメ・---}

\hypertarget{consomme-murillo}{%
\subsubsection{コンソメ・ムリーリョ}\label{consomme-murillo}}

\frsub{Consommé Murillo}

\index{consomme@consommé!murillo@--- Murillo}
\index{murillo@Murillo!consomme@consommé ---}
\index{こんそめ@コンソメ!むりーりよ@---・ムリーリョ}
\index{むりーりよ@ムリーリョ!こんそめ@コンソメ・---}

\hypertarget{consomme-nana}{%
\subsubsection{コンソメ・ナナ}\label{consomme-nana}}

\frsub{Consommé Nana}

\index{consomme@consommé!nana@--- Nana}
\index{nana@Nana!consomme@consommé ---}
\index{こんそめ@コンソメ!なな@---・ナナ}
\index{なな@ナナ!こんそめ@コンソメ・---}

\hypertarget{consomme-nantua}{%
\subsubsection{コンソメ・ナンチュア}\label{consomme-nantua}}

\frsub{Consommé Nantua}

\index{consomme@consommé!nantua@--- Nantua}
\index{nantua@Nantua!consomme@consommé ---}
\index{こんそめ@コンソメ!なんちゆあ@---・ナンチュア}
\index{なんちゆあ@ナンチュア!こんそめ@コンソメ・---}

\hypertarget{consomme-a-la-neige-de-florence}{%
\subsubsection{コンソメ・ネージュ・ド・フローレンス}\label{consomme-a-la-neige-de-florence}}

\frsub{Consommé à la Neige de Florence}

\index{consomme@consommé!neigedeflorence@--- à la Neige de Florence}
\index{neigedeflorence@Neige de Florence!consomme@consommé à la ---}
\index{こんそめ@コンソメ!ねーしゆとふろーれんす@---・ネージュ・ド・フローレンス}
\index{ねーしゆとふろーれんす@ネージュ・ド・フローレンス!こんそめ@コンソメ・---}

\hypertarget{consomme-nelson}{%
\subsubsection{コンソメ・ネルソン}\label{consomme-nelson}}

\frsub{Consommé Nelson}

\index{consomme@consommé!nelson@--- Nelson}
\index{nelson@Nelson!consomme@consommé ---}
\index{こんそめ@コンソメ!ねるそん@---・ネルソン}
\index{ねるそん@ネルソン!こんそめ@コンソメ・---}

\hypertarget{consomme-nesselrode}{%
\subsubsection{コンソメ・ネッセルローデ}\label{consomme-nesselrode}}

\frsub{Consommé Nesselrode}

\index{consomme@consommé!nesselrode@--- Nesselrode}
\index{nesselrode@Nesselrode!consomme@consommé ---}
\index{こんそめ@コンソメ!ねつせるろーて@---・ネッセルローデ}
\index{ねつせるろーて@ネッセルローデ!こんそめ@コンソメ・---}

\hypertarget{consomme-aux-nids-d-hirondelles}{%
\subsubsection{コンソメ・ツバメの巣入り}\label{consomme-aux-nids-d-hirondelles}}

\frsub{Consommé aux Nids d'hirondelles}

\index{consomme@consommé!niddhirondelle@--- aux-nids-d-hirondelles}
\index{niddhirondelle@Nids d'hirondelles!consomme@consommé aux ---}
\index{こんそめ@コンソメ!つはめのす@---・ツバメの巣入り}
\index{つはめのす@ツバメの巣入り!こんそめ@コンソメ・---}

\hypertarget{consomme-ninon}{%
\subsubsection{コンソメ・ニノン}\label{consomme-ninon}}

\frsub{Consommé Ninon}

\index{consomme@consommé!ninon@--- Ninon}
\index{ninon@Ninon!consomme@consommé ---}
\index{こんそめ@コンソメ!にのん@---・ニノン}
\index{にのん@ニノン!こんそめ@コンソメ・---}

\hypertarget{consomme-a-l-orge-perle}{%
\subsubsection{コンソメ・大麦入り}\label{consomme-a-l-orge-perle}}

\frsub{Consommé à l'Orge perlé}

\index{consomme@consommé!orgeperle@--- à l'Orge perlé}
\index{orgeperle@Orge perlé!consomme@consommé à l'---}
\index{こんそめ@コンソメ!おおむき@---・大麦入り}
\index{おおむき@大麦入り!こんそめ@コンソメ・---}

\hypertarget{consomme-a-l-oriental}{%
\subsubsection{コンソメ・オリエンタル}\label{consomme-a-l-oriental}}

\frsub{Consommé à l'Orientale}

\index{consomme@consommé!oriental@--- à l'Orientale}
\index{oriental@Orientale!consomme@consommé à l'---}
\index{こんそめ@コンソメ!おりえんたる@---・オリエンタル}
\index{おりえんたる@オリエンタル!こんそめ@コンソメ・---}

\hypertarget{consomme-olga}{%
\subsubsection{コンソメ・オルガ}\label{consomme-olga}}

\frsub{Consommé Olga}

\index{consomme@consommé!olga@--- Olga}
\index{olga@Olga!consomme@consommé ---}
\index{こんそめ@コンソメ!おるか@---・オルガ}
\index{おるか@オルガ!こんそめ@コンソメ・---}

\hypertarget{consomme-a-la-d-orleans}{%
\subsubsection{コンソメ・オルレアン風}\label{consomme-a-la-d-orleans}}

\frsub{Consommé à la d'Orléans}

\index{consomme@consommé!orlean@--- à la d'Orléans}
\index{orlean@Orléans!consomme@consommé à la d'---}
\index{こんそめ@コンソメ!おるれあんふう@---・オルレアン風}
\index{おるれあんふう@オルレアン風!こんそめ@コンソメ・---}

\hypertarget{consomme-orloff}{%
\subsubsection{コンソメ・オルロフ}\label{consomme-orloff}}

\frsub{Consommé Orloff}

\index{consomme@consommé!orloff@--- Orloff}
\index{rloff@Orloff!consomme@consommé ---}
\index{こんそめ@コンソメ!おるろふ@---・オルロフ}
\index{おるろふ@オルロフ!こんそめ@コンソメ・---}

\hypertarget{consomme-d-orsay}{%
\subsubsection{コンソメ・オルセイ}\label{consomme-d-orsay}}

\frsub{Consommé d'Orsay}

\index{consomme@consommé!orsay@--- d'Orsay}
\index{orsay@Orsay!consomme@consommé d'---}
\index{こんそめ@コンソメ!おるせい@---・オルセイ}
\index{おるせい@オルセイ!こんそめ@コンソメ・---}

\hypertarget{consomme-otello}{%
\subsubsection{コンソメ・オテッロ}\label{consomme-otello}}

\frsub{Consommé Otello}

\index{consomme@consommé!otello@--- Otello}
\index{otello@Otello!consomme@consommé ---}
\index{こんそめ@コンソメ!おてつろ@---・オテッロ}
\index{おてつろ@オテッロ!こんそめ@コンソメ・---}

\hypertarget{consomme-otero}{%
\subsubsection{コンソメ・オテロ}\label{consomme-otero}}

\frsub{Consommé Otero}

\index{consomme@consommé!otero@--- Otero}
\index{otero@Otero!consomme@consommé ---}
\index{こんそめ@コンソメ!おてろ@---・オテロ}
\index{おてろ@オテロ!こんそめ@コンソメ・---}

\hypertarget{consomme-palestro}{%
\subsubsection{コンソメ・パレストロ}\label{consomme-palestro}}

\frsub{Consommé Palestro}

\index{consomme@consommé!palestro@--- Palestro}
\index{palestro@Palestro!consomme@consommé ---}
\index{こんそめ@コンソメ!はれすとろ@---・パレストロ}
\index{はれすとろ@パレストロ!こんそめ@コンソメ・---}

\hypertarget{consomme-aux-pates-diverses}{%
\subsubsection{コンソメ・パスタ入り}\label{consomme-aux-pates-diverses}}

\frsub{Consommé aux Pâtes diverses}

\index{consomme@consommé!patediverse@--- aux Pâtes diverses}
\index{patediverse@Pâtes diverses!consomme@consommé aux ---}
\index{こんそめ@コンソメ!はすたいり@---・パスタ入り}
\index{はすたいり@パスタ入り!こんそめ@コンソメ・---}

\hypertarget{consomme-petite-mariee}{%
\subsubsection{コンソメ・新婦風}\label{consomme-petite-mariee}}

\frsub{Consommé Petite Mariée}

\index{consomme@consommé!petitemariee@--- Petite Mariée}
\index{petitemariee@Petite Mariée!consomme@consommé ---}
\index{こんそめ@コンソメ!しんふふう@---・新婦風}
\index{しんふふう@新婦風!こんそめ@コンソメ・---}

\hypertarget{petite-marmite}{%
\subsubsection{プティット・マルミット}\label{petite-marmite}}

\frsub{Petite Marmite}

\index{petite marmite@Petite Marmite}
\index{こんそめ@コンソメ!ふていつとまるみつと@プティットマルミット}
\index{ふていつとまるみつと@プティットマルミット}

\hypertarget{consomme-polaire}{%
\subsubsection{コンソメ・ポレール}\label{consomme-polaire}}

\frsub{Consommé Polaire}

\index{consomme@consommé!polaire@--- Polaire}
\index{polaire@Polaire!consomme@consommé ---}
\index{こんそめ@コンソメ!ほれーる@---・ポレール}
\index{ほれーる@ポレール!こんそめ@コンソメ・---}

\hypertarget{consomme-pompadour}{%
\subsubsection{コンソメ・ポンパドール夫人風}\label{consomme-pompadour}}

\frsub{Consommé Pompadour}

\index{consomme@consommé!pompadour@--- Pompadour}
\index{pompadour@Pompadour!consomme@consommé ---}
\index{こんそめ@コンソメ!ほんはとーるふしんふう@---・ポンパドール夫人風}
\index{ほんはとーるふしんふう@ポンパドール夫人風!こんそめ@コンソメ・---}

\hypertarget{consomme-portalis}{%
\subsubsection{コンソメ・ポルタリ風}\label{consomme-portalis}}

\frsub{Consommé Portalis}

\index{consomme@consommé!portalis@--- Portalis}
\index{portalis@Portalis!consomme@consommé ---}
\index{こんそめ@コンソメ!ほるたりふう@---・ポルタリ風}
\index{ほるたりふう@ポルタリ風!こんそめ@コンソメ・---}

\hypertarget{consomme-printanier}{%
\subsubsection{コンソメ・プランタニエ}\label{consomme-printanier}}

\frsub{Consommé Printanier}

\index{consomme@consommé!printanier@--- Printanier}
\index{printanier@Printanier!consomme@consommé ---}
\index{こんそめ@コンソメ!ふらんたにえ@---・プランタニエ}
\index{ふらんたにえ@プランタニエ!こんそめ@コンソメ・---}

\hypertarget{consomme-aux-quenelles-a-la-moelle}{%
\subsubsection{コンソメ・モワルのクネル入り}\label{consomme-aux-quenelles-a-la-moelle}}

\frsub{Consommé aux Quenelles à la moelle}

\index{consomme@consommé!aux quenelles a la moelle@--- aux Quenelles à la moelle}
\index{quenelles a la moelle@Quenelles à la moelle!consomme@consommé aux ---}
\index{こんそめ@コンソメ!もわるのくねるいり@---・モワルのクネル入り}
\index{もわるのくねるいり@モワルのクネル入り!こんそめ@コンソメ・---}

\hypertarget{potage-queue-de-boeuf-a-la-francaise}{%
\subsubsection{牛テールのコンソメ・アラフランセズ}\label{potage-queue-de-boeuf-a-la-francaise}}

\frsub{Potage Queue de Boeuf à la française}

\index{potage queue de boeuf a la francaise@Potage Queue de Boeuf à la française}
\index{こんそめ@コンソメ!きゆうてーるのこんそめあらふらんせす@---・牛テールのコンソメアラフランセズ}
\index{きゆうてーるのこんそめあらふらんせす@牛テールのコンソメアラフランセズ}

\hypertarget{consomme-rabelais}{%
\subsubsection{コンソメ・ラブレ}\label{consomme-rabelais}}

\frsub{Consommé Rabelais}

\index{consomme@consommé!rabelais@--- Rabelais}
\index{rabelais@Rabelais!consomme@consommé ---}
\index{こんそめ@コンソメ!らふれ@---・ラブレ}
\index{らふれ@ラブレ!こんそめ@コンソメ・---}

\hypertarget{consomme-rachel}{%
\subsubsection{コンソメ・ラケル}\label{consomme-rachel}}

\frsub{Consommé Rachel}

\index{consomme@consommé!rachel@--- Rachel}
\index{rachel@Rachel!consomme@consommé ---}
\index{こんそめ@コンソメ!らける@---・ラケル}
\index{らける@ラケル!こんそめ@コンソメ・---}

\hypertarget{consomme-aux-raviolis}{%
\subsubsection{コンソメ・ラヴィオリ入り}\label{consomme-aux-raviolis}}

\frsub{Consommé aux Raviolis}

\index{consomme@consommé!au ravioli@--- au(x) ravioli(s)}
\index{ravioli@!consomme@consommé aux ---(s)}
\index{こんそめ@コンソメ!らういおりいり@---・ラヴィオリ入り}
\index{らういおりいり@ラヴィオリ入り!こんそめ@コンソメ・---}

\hypertarget{consomme-recamier}{%
\subsubsection{コンソメ・レカミエ}\label{consomme-recamier}}

\frsub{Consommé Récamier}

\index{consomme@consommé!recamier@--- Récamier}
\index{recamier@Récamier!consomme@consommé ---}
\index{こんそめ@コンソメ!れかみえ@---・レカミエ}
\index{れかみえ@レカミエ!こんそめ@コンソメ・---}

\hypertarget{consomme-a-la-reine}{%
\subsubsection{コンソメ・女王風}\label{consomme-a-la-reine}}

\frsub{Consommé à la Reine}

\index{consomme@consommé!a la reine@--- à la Reine}
\index{reine@Reine!consomme@consommé à la ---}
\index{こんそめ@コンソメ!しよおうふう@---・女王風}
\index{しよおうふう@女王風!こんそめ@コンソメ・---}

\hypertarget{consomme-renaissance}{%
\subsubsection{コンソメ・ルネッサンス}\label{consomme-renaissance}}

\frsub{Consommé Renaissance}

\index{consomme@consommé!renaissance@--- Renaissance}
\index{renaissance@Renaissance!consomme@consommé ---}
\index{こんそめ@コンソメ!るねつさんす@---・ルネッサンス}
\index{るねつさんす@ルネッサンス!こんそめ@コンソメ・---}

\hypertarget{consomme-rossini}{%
\subsubsection{コンソメ・ロッシーニ}\label{consomme-rossini}}

\frsub{Consommé Rossini}

\index{consomme@consommé!rossini@--- Rossini}
\index{rossini@Rossini!consomme@consommé ---}
\index{こんそめ@コンソメ!ろつしーに@---・ロッシーニ}
\index{ろつしーに@ロッシーニ!こんそめ@コンソメ・---}

\hypertarget{consomme-a-la-royale}{%
\subsubsection{コンソメ・ロワイヤル}\label{consomme-a-la-royale}}

\frsub{Consommé à la Royale}

\index{consomme@consommé!royale@--- à la Royale}
\index{royale@Royale!consomme@consommé à la ---}
\index{こんそめ@コンソメ!ろわいやる@---・ロワイヤル}
\index{ろわいやる@ロワイヤル!こんそめ@コンソメ・---}

\hypertarget{consomme-au-sagou}{%
\subsubsection{コンソメ・サゴ澱粉入り}\label{consomme-au-sagou}}

\frsub{Consommé au Sagou}

\index{consomme@consommé!sagou@--- au Sagou}
\index{sagou@Sagou!consomme@consommé au ---}
\index{こんそめ@コンソメ!さこてんふんいり@---・サゴ澱粉入り}
\index{さこてんふんいり@サゴ澱粉入り!こんそめ@コンソメ・---}

\hypertarget{consomme-saint-hubert}{%
\subsubsection{コンソメ・サンテュベール}\label{consomme-saint-hubert}}

\frsub{Consommé Saint Hubert}

\index{consomme@consommé!saint hubert@--- Saint Hubert}
\index{saint hubert@Saint Hubert!consomme@consommé ---}
\index{こんそめ@コンソメ!さんてゆべーる@---・サンテュベール}
\index{さんてゆべーる@サンテュベール!こんそめ@コンソメ・---}

\hypertarget{consomme-au-salep}{%
\subsubsection{コンソメ・サレプ粉入り}\label{consomme-au-salep}}

\frsub{Consommé au Salep}

\index{consomme@consommé!salep@--- au Salep}
\index{salep@Salep!consomme@consommé au ---}
\index{こんそめ@コンソメ!されふこいり@---・サレプ粉入り}
\index{されふこいり@サレプ粉入り!こんそめ@コンソメ・---}

\hypertarget{consomme-sapho}{%
\subsubsection{コンソメ・サフォ}\label{consomme-sapho}}

\frsub{Consommé Sapho}

\index{consomme@consommé!sapho@--- Sapho}
\index{sapho@Sapho!consomme@consommé ---}
\index{こんそめ@コンソメ!さふお@---・サフォ}
\index{さふお@サフォ!こんそめ@コンソメ・---}

\hypertarget{potage-sarah-bernhardt}{%
\subsubsection{コンソメ・サラ・ベルンハルト風}\label{potage-sarah-bernhardt}}

\frsub{Potage Sarah Bernhardt}

\index{potage sarah bernhardt@Potage Sarah Bernhardt}
\index{こんそめ@コンソメ!さらへるんはるとふう@---・サラ・ベルンハルト風}
\index{さらへるんはるとふう@サラ・ベルンハルト風!こんそめ@コンソメ・---}

\hypertarget{consomme-severine}{%
\subsubsection{コンソメ・セヴェリン}\label{consomme-severine}}

\frsub{Consommé Séverine}

\index{consomme@consommé!severine@--- Séverine}
\index{severine@Séverine!consomme@consommé ---}
\index{こんそめ@コンソメ!せうえりん@---・セヴェリン}
\index{せうえりん@セヴェリン!こんそめ@コンソメ・---}

\hypertarget{consomme-sevigne}{%
\subsubsection{コンソメ・セヴィニェ}\label{consomme-sevigne}}

\frsub{Consommé Sévigné}

\index{consomme@consommé!sevigne@--- Sévigné}
\index{sevigne@Sévigné!consomme@consommé ---}
\index{こんそめ@コンソメ!せういにえ@---・セヴィニェ}
\index{せういにえ@セヴィニェ!こんそめ@コンソメ・---}

\hypertarget{consomme-solange}{%
\subsubsection{コンソメ・ソランジュ}\label{consomme-solange}}

\frsub{Consommé Solange}

\index{consomme@consommé!solange@--- Solange}
\index{solange@Solange!consomme@consommé ---}
\index{こんそめ@コンソメ!そらんしゆ@---・ソランジュ}
\index{そらんしゆ@ソランジュ!こんそめ@コンソメ・---}

\hypertarget{consomme-stael}{%
\subsubsection{コンソメ・スタール}\label{consomme-stael}}

\frsub{Consommé Staël}

\index{consomme@consommé!stael@--- Staël}
\index{stael@Staël!consomme@consommé ---}
\index{こんそめ@コンソメ!すたーる@---・スタール}
\index{すたーる@スタール!こんそめ@コンソメ・---}

\hypertarget{consomme-stanley}{%
\subsubsection{コンソメ・スタンレイ}\label{consomme-stanley}}

\frsub{Consommé Stanley}

\index{consomme@consommé!stanley@--- Stanley}
\index{stanley@Stanley!consomme@consommé ---}
\index{こんそめ@コンソメ!すたんれい@---・スタンレイ}
\index{すたんれい@スタンレイ!こんそめ@コンソメ・---}

\hypertarget{consomme-suzette}{%
\subsubsection{コンソメ・シュゼット}\label{consomme-suzette}}

\frsub{Consommé Suzette}

\index{consomme@consommé!suzette@--- Suzette}
\index{suzette@Suzette!consomme@consommé ---}
\index{こんそめ@コンソメ!しゆせつと@---・シュゼット}
\index{しゆせつと@シュゼット!こんそめ@コンソメ・---}

\hypertarget{consomme-talleyrand}{%
\subsubsection{コンソメ・タレーラン}\label{consomme-talleyrand}}

\frsub{Consommé Talleyrand}

\index{consomme@consommé!talleyrand@--- Talleyrand}
\index{talleyrand@Talleyrand!consomme@consommé ---}
\index{こんそめ@コンソメ!たれーらん@---・タレーラン}
\index{たれーらん@タレーラン!こんそめ@コンソメ・---}

\hypertarget{consomme-au-tapioca}{%
\subsubsection{コンソメ・タピオカ入り}\label{consomme-au-tapioca}}

\frsub{Consommé au Tapioca}

\index{consomme@consommé!au tapioca@--- au Tapioca}
\index{tapioca@Tapioca!consomme@consommé au ---}
\index{こんそめ@コンソメ!たひおかいり@---・タピオカ入り}
\index{たひおかいり@タピオカ入り!こんそめ@コンソメ・---}

\hypertarget{consomme-theodora}{%
\subsubsection{コンソメ・テオドラ}\label{consomme-theodora}}

\frsub{Consommé Théodora}

\index{consomme@consommé!theodora@--- Théodora}
\index{theodora@Théodora!consomme@consommé ---}
\index{こんそめ@コンソメ!ておとら@---・テオドラ}
\index{ておとら@テオドラ!こんそめ@コンソメ・---}

\hypertarget{consomme-toreador}{%
\subsubsection{コンソメ・闘牛士風}\label{consomme-toreador}}

\frsub{Consommé Toréador}

\index{consomme@consommé!toreador@--- Toréador}
\index{toreador@Toréador!consomme@consommé ---}
\index{こんそめ@コンソメ!とうきゆうしふう@---・闘牛士風}
\index{とうきゆうしふう@闘牛士風!こんそめ@コンソメ・---}

\hypertarget{consomme-tosca}{%
\subsubsection{コンソメ・トスカ}\label{consomme-tosca}}

\frsub{Consommé Tosca}

\index{consomme@consommé!tosca@--- Tosca}
\index{tosca@Tosca!consomme@consommé ---}
\index{こんそめ@コンソメ!とすか@---・トスカ}
\index{とすか@トスカ!こんそめ@コンソメ・---}

\hypertarget{consomme-toulousaine}{%
\subsubsection{コンソメ・トゥールーズ風}\label{consomme-toulousaine}}

\frsub{Consommé Toulousaine}

\index{consomme@consommé!toulousaine@--- Toulousaine}
\index{toulousaine@Toulousaine!consomme@consommé ---}
\index{こんそめ@コンソメ!とうーるーすふう@---・トゥールーズ風}
\index{とうーるーすふう@トゥールーズ風!こんそめ@コンソメ・---}

\hypertarget{consomme-a-la-trevise}{%
\subsubsection{コンソメ・トレヴィーゾ風}\label{consomme-a-la-trevise}}

\frsub{Consommé à la Trévise}

\index{consomme@consommé!a la trevise@--- à la Trévise}
\index{trevise@Trévise!consomme@consommé à la ---}
\index{こんそめ@コンソメ!とれういーそふう@---・トレヴィーゾ風}
\index{とれうーそふう@トレヴィーゾ風!こんそめ@コンソメ・---}

\hypertarget{consomme-tyrolienne}{%
\subsubsection{コンソメ・チロリエンヌ}\label{consomme-tyrolienne}}

\frsub{Consommé Tyrolienne}

\index{consomme@consommé!tyrolienne@--- Tyrolienne}
\index{tyrolienne@Tyrolienne!consomme@consommé ---}
\index{こんそめ@コンソメ!ちろりえんぬ@---・チロリエンヌ}
\index{ちろりえんぬ@チロリエンヌ!こんそめ@コンソメ・---}

\hypertarget{consomme-d-uzes}{%
\subsubsection{コンソメ・ユゼス}\label{consomme-d-uzes}}

\frsub{Consommé d'Uzès}

\index{consomme@consommé!d'uzes@--- d'Uzès}
\index{uzes@Uzès!consomme@consommé d'---}
\index{こんそめ@コンソメ!ゆせす@---・ユゼス}
\index{ゆせす@ユゼス!こんそめ@コンソメ・---}

\hypertarget{consomme-valromey}{%
\subsubsection{コンソメ・ヴァルロメ}\label{consomme-valromey}}

\frsub{Consommé Valromey}

\index{consomme@consommé!valromey@--- Valromey}
\index{valromey@Valromey!consomme@consommé ---}
\index{こんそめ@コンソメ!うあるろめ@---・ヴァルロメ}
\index{うあるろめ@ヴァルロメ!こんそめ@コンソメ・---}

\hypertarget{consomme-vendome}{%
\subsubsection{コンソメ・ヴァンドーム}\label{consomme-vendome}}

\frsub{Consommé Vendôme}

\index{consomme@consommé!vendome@--- Vendôme}
\index{vendome@Vendôme!consomme@consommé ---}
\index{こんそめ@コンソメ!うあんとーむ@---・ヴァンドーム}
\index{うあんとーむ@ヴァンドーム!こんそめ@コンソメ・---}

\hypertarget{consomme-verdi}{%
\subsubsection{コンソメ・ヴェルディ}\label{consomme-verdi}}

\frsub{Consommé Verdi}

\index{consomme@consommé!verdi@--- Verdi}
\index{verdi@Verdi!consomme@consommé ---}
\index{こんそめ@コンソメ!うえるてい@---・ヴェルディ}
\index{うえるてい@ヴェルディ!こんそめ@コンソメ・---}

\hypertarget{consomme-vermandoise}{%
\subsubsection{コンソメ・ヴェルマンドワズ}\label{consomme-vermandoise}}

\frsub{Consommé Vermandoise}

\index{consomme@consommé!vermandoise@--- Vermandoise}
\index{vermandoise@Vermandoise!consomme@consommé ---}
\index{こんそめ@コンソメ!うえるまんとわす@---・ヴェルマンドワズ}
\index{うえるまんとわす@ヴェルマンドワズ!こんそめ@コンソメ・---}

\hypertarget{consomme-au-vermicelle}{%
\subsubsection{コンソメ・ヴァミセリ入り}\label{consomme-au-vermicelle}}

\frsub{Consommé au Vermicelle}

\index{consomme@consommé!au vermicelle@--- au Vermicelle}
\index{vermicelle@Vermicelle!consomme@consommé au ---}
\index{こんそめ@コンソメ!うあみせりいり@---・ヴァミセリ入り}
\index{うあみせりいり@ヴァミセリ入り!こんそめ@コンソメ・---}

\hypertarget{consomme-des-viveurs}{%
\subsubsection{コンソメ・ヴィブール}\label{consomme-des-viveurs}}

\frsub{Consommé des Viveurs}

\index{consomme@consommé!des viveurs@--- des Viveurs}
\index{viveurs@Viveurs!consomme@consommé des---}
\index{こんそめ@コンソメ!ういふーる@---・ヴィブール}
\index{ういふーる@ヴィブール!こんそめ@コンソメ・---}

\hypertarget{consomme-warwick}{%
\subsubsection{コンソメ・ヴァルヴィック}\label{consomme-warwick}}

\frsub{Consommé Warwick}

\index{consomme@consommé!warwick@--- Warwick}
\index{warwick@Warwick!consomme@consommé ---}
\index{こんそめ@コンソメ!うあるういつく@---・ヴァルヴィック}
\index{うあるういつく@ヴァルヴィック!こんそめ@コンソメ・---}

\hypertarget{consomme-washington}{%
\subsubsection{コンソメ・ワシントン}\label{consomme-washington}}

\frsub{Consommé Washington}

\index{consomme@consommé!washington@--- Washington}
\index{washington@Washington!consomme@consommé ---}
\index{こんそめ@コンソメ!わしんとん@---・ワシントン}
\index{わしんとん@ワシントン!こんそめ@コンソメ・---}

\hypertarget{consomme-wladimir}{%
\subsubsection{コンソメ・ワディミール}\label{consomme-wladimir}}

\frsub{Consommé Wladimir}

\index{consomme@consommé!wladimir@--- Wladimir}
\index{wladimir@Wladimir!consomme@consommé ---}
\index{こんそめ@コンソメ!わていみーる@---・ワディミール}
\index{わていみーる@ワディミール!こんそめ@コンソメ・---}

\hypertarget{consomme-yvetot}{%
\subsubsection{コンソメ・イヴト}\label{consomme-yvetot}}

\frsub{Consommé Yvetot}

\index{consomme@consommé!yvetot@--- Yvetot}
\index{yvetot@vetot!consomme@consommé ---}
\index{こんそめ@コンソメ!いうと@---・イヴト}
\index{いうと@イヴト!こんそめ@コンソメ・---}

\hypertarget{consomme-zola}{%
\subsubsection{コンソメ・ゾラ}\label{consomme-zola}}

\frsub{Consommé Zola}

\index{consomme@consommé!zola@--- Zola}
\index{zola@Zola!consomme@consommé ---}
\index{こんそめ@コンソメ!そら@---・ゾラ}
\index{そら@ゾラ!こんそめ@コンソメ・---}

\hypertarget{consomme-zorilla}{%
\subsubsection{コンソメ・ゾリヤ}\label{consomme-zorilla}}

\frsub{Consommé Zorilla}

\index{consomme@consommé!zorilla@--- Zorilla}
\index{zorilla@Zorilla!consomme@consommé ---}
\index{こんそめ@コンソメ!そりや@---・ゾリヤ}
\index{そりや@ゾリヤ!こんそめ@コンソメ・---}
\end{recette}%newpage
\hypertarget{consomme-divers-speciaux-pour-soupers}{%
\section{夜食用のコンソメなど}\label{consomme-divers-speciaux-pour-soupers}}

\frsec{Consommés divers, spéciaux pour Soupers}
\begin{recette}
\hypertarget{consome-a-l-essence-de-caille}{%
\subsubsection{コンソメ・鶉のエッセンス入り}\label{consome-a-l-essence-de-caille}}

\frsub{Consommé à l'essence de Caille}

\index{consomme@consommé!caille@--- à l'essence de Caille}
\index{caille@caille!consomme@consommé à l'essence de ---}
\index{こんそめ@コンソメ!うすらのえつせんすいり@鶉のエッセンス入り}
\index{うすら@鶉!こんそめえつせんすいり@コンソメ・---のエッセンス入り}

コンソメ1 Lあたり鶉4羽を用いる。鶉をローストし、胸肉は別に取り置いて別
の用途に使うこと。残りの部位を使ってコンソメをクラリフィエする。

\hypertarget{consome-a-l-essence-de-celeri}{%
\subsubsection{コンソメ・セロリのエッセンス入り}\label{consome-a-l-essence-de-celeri}}

\frsub{Consommé à l'essence de Céleri}

\index{consomme@consommé!celeri@--- à l'essence de Céleri}
\index{celeri@céleri!consomme@consommé à l'essence de ---}
\index{こんそめ@コンソメ!せろりのえつせんす@---・セロリのエッセンス入り}
\index{せろり@せろり!こんそめえつせんすいり@コンソメ・---のエッセンス入り}

コンソメに風味を付けるために必要なセロリの量は、概ね 1 Lあたり 100
gとして、クラリフィエの際に加える。

\hypertarget{consome-a-l-essence-d-estragon}{%
\subsubsection{コンソメ・エストラゴンのエッセンス入り}\label{consome-a-l-essence-d-estragon}}

\frsub{Consommé à l'essence d'Estragon}

\index{consomme@consommé!estragon@--- à l'essence d'Estragon}
\index{estragon@Estragon!consomme@consommé à l'essence d' ---}
\index{こんそめ@コンソメ!えすとらこんのえつせんす@エストラゴンのエッセンス入り}
\index{えすとらこん@エストラゴン!こんそめえつせんすいり@コンソメ・---のエッセンス入り}

この香草は使い過ぎないようにするのががいい。風味付けには1
Lあたり数枚の葉で充分。

\hypertarget{consome-ivan-pour-soupers}{%
\subsubsection{コンソメ・イヴァン}\label{consome-ivan-pour-soupers}}

\frsub{Consommé Ivan}

\index{consomme@consommé!ivan@--- Ivan (pour soupers)}
\index{ivan@Ivan!consomme@consommé --- (pour soupers)}
\index{こんそめ@コンソメ!いうあんやしよくよう@イヴァン(夜食用)}
\index{いうあん@イヴァン!こんそめやしよくよう@コンソメ・---(夜食用)}

\protect\hyperlink{consomme-ivan}{温かいコンソメの節のアルファベット順に出ているレシピ}参照。

\hypertarget{consome-a-l-essence-de-morille}{%
\subsubsection{コンソメ・モリーユのエッセンス入り}\label{consome-a-l-essence-de-morille}}

\frsub{Consommé à l'essence de Morille}

\index{consomme@consommé!morille@--- à l'essence de Morille}
\index{morille@morille!consomme@consommé à l'essence de ---}
\index{こんそめ@コンソメ!もりーゆのえつせんすいり@---・モリーユのエッセンス入り}
\index{もりーゆ@モリーユ!こんそめえつせんすいり@コンソメ・---のエッセンス入り}

コンソメ1 Lあたり、生なら150 g、乾燥なら90 gの網笠茸を用意する。ごく薄
くスライスし、さらに細かくすり潰して加える。
\end{recette}%newpage
%\href{原稿下準備20180414五島、連載からコピー}{} \href{訳と注釈}{}
\href{未、原文対照チェック}{} \href{未、日本語表現校正}{}
\href{未、その他修正}{} \href{未、原稿最終校正}{}

\hypertarget{ux3068ux308dux307fux3092ux4ed8ux3051ux305fux30ddux30bfux30fcux30b8ux30e5}{%
\section{とろみを付けたポタージュ}\label{ux3068ux308dux307fux3092ux4ed8ux3051ux305fux30ddux30bfux30fcux30b8ux30e5}}

\frsec{Potages Liés}

\hypertarget{ux30ddux30bfux30fcux30b8ux30e5ux30d4ux30e5ux30ec}{%
\subsection{ポタージュ・ピュレ}\label{ux30ddux30bfux30fcux30b8ux30e5ux30d4ux30e5ux30ec}}

\frsecb{les Purées}

主素材とつなぎ:ポタージュ・ピュレの主素材として用いるのは次のとおり。
1種類または数種を組み合わせた野菜、鶏、ジビエ、甲殻類。

ほぼ全てのポタージュ・ピュレにはつなぎを加える。すなわち、

米\ldots{}\ldots{}鶏、甲殻類のポタージュ・ピュレおよび野菜のポタージュ・ピュレ
のいくつか。

じゃがいも\ldots{}\ldots{}香草や、かぼちゃのように水分の多い野菜のポタージュ・
ピュレ。

レンズ豆\ldots{}\ldots{}ジビエのポタージュ・ピュレ。

バターで揚げたクルトン\ldots{}\ldots{}クラシックなポタージュ・ピュレ。

昔の料理では、他にもつなぎに用いるものはあったが、とりわけクーリとビス
ク\footnote{「昔の料理」におけるビスクは甲殻類のポタージュ・ピュレのことで
  はなく、鳩などの煮込み料理のこと(本連載「ポタージュ(1)」2012年6月
  号p.115 参照)。}には、クルトンが主に用いられていた\footnote{中世〜18世紀には、とろみをつけるために、硬くなったパンを加えて
  弱火で煮込む(mitonnerミトネ)ことが一般的だった。}。とてもまろやかな仕上り
になるので、現代でもこの手法を用いる価値はある。

いんげん豆やレンズ豆、じゃがいものようなでんぷん質の素材のポタージュ・
ピュレにはつなぎを加える必要はない。主素材である野菜それ自体がつなぎと
なるからだ。

加える液体とつなぎの分量:ポタージュ・ピュレに加える液体は、主素材の種
類に応じて、白いコンソメ、ジビエのコンソメ、魚のコンソメを用いる。野菜
のポタージュ・ピュレでは牛乳を用いる場合もある。

加える液体\ldots{}\ldots{}ベースとなるピュレ1 Lに対して2 L。

つなぎ\ldots{}\ldots{}

\begin{enumerate}
\def\labelenumi{\arabic{enumi}.}
\item
  米\ldots{}\ldots{}野菜500 gあたり85〜120 g。鶏、ジビエ、甲殻類の身500
  gあ たり75〜100 g。
\item
  レンズ豆\ldots{}\ldots{}ジビエの肉500 gあたり190 g。
\item
  じゃがいも\ldots{}\ldots{}香草と野菜500 gあたり250 g。
\item
  バターで揚げたクルトン\ldots{}\ldots{}野菜または甲殻類の身500
  gあたり270 g。
\end{enumerate}

作業と仕上げ:野菜は次のいずれかの方法で処理する。(a)薄切りにした野菜
600〜700 gあたり80〜100 gのバターでエテュヴェする。(b)薄切りにした野菜を
湯通し\footnote{原文 blanchir
  (ブランシール)。下茹でする、湯がくこと。野菜類の
  ブランシールは塩を加えた湯で行なうが、素材の性質により2種に分けら
  れる。ひとつは大量の湯で素材に完全に火が通るまで茹でること。もうひ
  とつは刳味(アク)を除くための下茹で、湯通し(原書p.726)。ここでは後 者。}してからバターでエテュヴェする。どちらの方法を用いるかは、
本書では個々のルセットに記してある。

ジビエはサルミを調理する際と同様にロティール\footnote{鶏、猟鳥の胸肉の部分を豚背脂のシートで包んでセニャンにロティー
  ルする。なお、「サルミ」は古くは「猟鳥肉の煮込み」の意であった。
  『ル・ギード・キュリネール』では、ロティールした猟鳥の肉を切り分け
  て保温し、摺り潰したガラと端肉を煮込んで作ったソースと合わせる(本
  連載「雉のサルミ」2011年11月号pp.128-129 参照)。}してから、レンズ豆と
ともに煮る。火が通ったら骨を外す。肉とレンズ豆を摺り潰し、布漉しした後、
濃さを調節する。

鶏は白いコンソメでポシェする。つなぎに用いる米も一緒に煮る。火が通った
ら骨を外し、その後はジビエのピュレと同様にする。鶏およびジビエにちょう
ど火が通ったところで、浮き実にする分の胸肉は別にとっておくこと。

野菜のポタージュ・ピュレは、濃さを調節したらデプイエ、つまり微沸騰の状
態で25〜30分間かけて不純物を取り除く。

このデプイエの作業の際、時折、冷たいコンソメを若干量加えるとよい。ピュ
レの中に紛れている不純物が表面に浮かび上がって、取り除きやすくなる。

鶏、ジビエ、甲殻類のピュレは沸騰したら湯煎にかける。デプイエする必要は
ない。

どのポタージュ・ピュレも、仕上げにバターを加える直前に、目の細かいシノ
ワで漉すこと。

仕上げは提供直前に行なう。火から外し、ポタージュ1ℓあたり80〜100
gのバター を加える。

つなぎに白いんげん豆、じゃがいも、米などのような白いでんぷん質やクルト
ンを用いるポタージュは、さらにつなぎとして卵黄を補ってもよい。

バターを加えたら、再沸騰させてはいけないと肝に銘じること。沸騰するとバ
ターの風味が失なわれてしまう。ポタージュにおいて、バターの風味は明瞭で
フレッシュでなくてはいけない。

(略)

ピュレの展開\footnote{本連載「ポタージュ(1)」2012年6月号 p.115 参照。}:以下に記す方法で、ピュレの多くはポタージュ・ヴルテ、
ポタージュ・クレームにすることが出来る。ポタージュ・ピュレに用いるつな
ぎの代わりに、鶏または魚のヴルテや薄いソース・ベシャメルを主素材に加え
るのだ。

ただし、素材によっては、ポタージュ・ピュレ以外の仕立てに出来ないものも
ある。

\hypertarget{ux30ddux30bfux30fcux30b8ux30e5ux30f4ux30ebux30c6}{%
\subsection{ポタージュ・ヴルテ}\label{ux30ddux30bfux30fcux30b8ux30e5ux30f4ux30ebux30c6}}

\frsecb{les Veloutés}

ベースとなるヴルテ\footnote{基本ソースとしてのヴルテ(原書
  p.15)がベースとなる。}:

\begin{enumerate}
\def\labelenumi{\arabic{enumi}.}
\item
  野菜のポタージュ・ヴルテの場合は、やや薄い通常のヴルテ。
\item
  鶏や魚のポタージュ・ヴルテの場合は、それぞれ対応するヴルテ。
\end{enumerate}

ポタージュのベースにするヴルテは、主素材となる野菜、鶏、ジビエおよび魚
に応じて、通常の白いコンソメ、鶏のコンソメ、ジビエのコンソメ、魚のコン
ソメ1ℓあたり白いルゥ100 gを加えて作る。

材料比率:この方法で作るポタージュは全て、次の分量配分となる。

・ベースとなるヴルテはポタージュ全体の半量。

・ポタージュの性格を決めるピュレは全体の\textbf{1/4}。

・濃さを整えるのに加えるコンソメも\textbf{1/4}。ただし、つなぎとして加える
生クリームの分量もこれに含める。

例えば、仕上がり2ℓの「ポタージュ・ヴルテ王妃風\footnote{à la reine
  (ア・ラ・レーヌ)優美で繊細な料理に用いる表現。この名
  称の料理には鶏を素材としたものが多い。「ポタージュ・ピュレ王妃風」
  のルセットは原書p.146。}」の場合には分量は 次のようになる。

\begin{quote}
鶏のヴルテ\textbf{1}ℓ。鶏のピュレ\textbf{5dl}。仕上げに加える白いコンソメ
\textbf{3dl}。つなぎ\textbf{(}生クリーム\textbf{)2dl}。計\textbf{2}ℓ。
\end{quote}

作業:

\begin{enumerate}
\def\labelenumi{(\arabic{enumi})}
\item
  主素材が鶏や魚の場合は、予め骨を外してからベースとなるヴルテで素材
  を煮る。次に、肉を取り出して摺り潰し、肉を煮たヴルテでのばしてから布漉
  しする。このピュレにコンソメを加えて濃さを整える。
\item
  野菜の場合は、素材の性質に応じて、湯通ししたものをバターでエテュヴェ
  するか、生の野菜をバターでエテュヴェしてから、ベースとなるヴルテに加え
  る。野菜に火が通った後は上記と同様にする。
\item
  甲殻類の場合は、通常どおりミルポワを用いて火を通し、細かく摺り潰し
  てからベースとなるヴルテに加えて煮、布漉しする。
\end{enumerate}

つなぎと仕上げ:ポタージュ・ヴルテのつなぎには、仕上り1ℓあたり卵黄3ヶ
と生クリーム1dlを加える。

提供直前に、鍋を火から外して、1ℓあたりバター80〜100 gを加えて仕上げる。
(略)

\hypertarget{ux30ddux30bfux30fcux30b8ux30e5ux30afux30ecux30fcux30e0}{%
\subsection{ポタージュ・クレーム}\label{ux30ddux30bfux30fcux30b8ux30e5ux30afux30ecux30fcux30e0}}

\frsecb{les Crèmes}

ポタージュ・クレームの作り方はポタージュ・ヴルテと同じだが、以下の点が
違う。

\begin{enumerate}
\def\labelenumi{(\arabic{enumi})}
\item
  ヴルテではなく薄いソース・ベシャメルをベースとして用いる。牛乳1ℓあ
  たり白いルゥ100 gで作る。
\item
  多くの場合、仕上げに濃さを調節する際、コンソメではなく牛乳を加える。
\end{enumerate}

材料比率:ポタージュ・ヴルテと同様。つまり、ベシャメルはポタージュ全体
の半量、ポタージュの性格を決めるピュレが1/4、濃さを整えるための白いコ
ンソメまたは牛乳が1/4(仕上げに加える生クリームもこれに含める)。

作業:主素材が鶏、ジビエ、野菜、甲殻類いずれの場合も、作業はポタージュ・
ヴルテの項で示したのと同じ。(略)

仕上げ:提供直前に、ポタージュ1ℓあたり2dlの生クリームを加える。

原注:ポタージュ・ヴルテもポタージュ・クレームもデプイエは行なわない。
ポタージュの濃さを整えたら、沸騰寸前まで温め、湯煎にかけて保温しておく。
表面が乾かないようバターのかけら数片を載せる。ポタージュ・ヴルテは、供
する前に卵黄、生クリーム、バターを加えて仕上げる。ポタージュ・クレーム
の仕上げは供する前に生クリームだけを加える。
%newpage
%\input{03-potages/03-07-pp139-146}
%\input{03-potages/03-08-pp146-150}
%\input{03-potages/03-09-pp151-156}
%\input{03-potages/03-10-pp157-162}
%\input{03-potages/03-11-pp163-169}
%\input{03-potages/03-12-pp170-171}
%\input{03-potages/03-13-pp172-185}



%%% Chapitre IV. Hors-d'oeuvres
%%%% エスコフィエ『料理の手引き』全注解
% 五島 学

\href{✓原稿下準備なし}{} \href{訳と注釈\%2020180420進行中}{}
\href{未、原文対照チェック}{} \href{未、日本語表現校正}{}
\href{未、注釈チェク}{} \href{未、原稿最終校正}{}

\hypertarget{poissons}{%
\chapter{VI 魚料理 Poissons}\label{poissons}}

\hypertarget{serie-de-courts-bouillons-de-poisson}{%
\section{クールブイヨン}\label{serie-de-courts-bouillons-de-poisson}}

\begin{recette}
\hypertarget{ux30afux30fcux30ebux30d6ux30a4ux30e8ux30f3-a}{%
\subsubsection{クールブイヨン
A}\label{ux30afux30fcux30ebux30d6ux30a4ux30e8ux30f3-a}}

水5 L に対し、ヴィネガー2.5 dL、粗塩60 g、薄切りにしたにんじん600
gと玉ねぎ500 g、タイム1枝、ローリエの小さな葉2枚、パセリの茎100
g、粒こしょう20
g(こしょうを加えるのはクールブイヨンを漉す10分前)。材料を全て鍋に入れ、火にかけて1時間弱火で煮、漉す(原書
p.277)。

\hypertarget{ux30afux30fcux30ebux30d6ux30a4ux30e8ux30f3-b-4-ux9c52ux3046ux306aux304eux30d6ux30edux30b7ux30a7ux7b49-ux539fux66f8p.277}{%
\subsubsection[クールブイヨン B (鱒、うなぎ、ブロシェ等)
原書p.277]{\texorpdfstring{クールブイヨン B \footnote{クールブイヨンは用途に応じ、AからEまでの5種が挙げられている(原書pp.277-278)。}
(鱒、うなぎ、ブロシェ等)
原書p.277}{クールブイヨン B  (鱒、うなぎ、ブロシェ等) 原書p.277}}\label{ux30afux30fcux30ebux30d6ux30a4ux30e8ux30f3-b-4-ux9c52ux3046ux306aux304eux30d6ux30edux30b7ux30a7ux7b49-ux539fux66f8p.277}}

5 L 分の材料\ldots{}\ldots{}白ワイン2.5 L 。水2.5 L
。薄切りにした玉ねぎ600 g。パセリの茎80g
。タイムの小枝1本。ローリエの葉(小) \(\frac{1}{2}\) 枚。粗塩
60g。大粒のこしょう15 g(クールブイヨンを漉す10分前に加える)。

作業手順\ldots{}\ldots{}作業:液体、香味素材、調味料を鍋に入れ、沸かす。弱火で30分程煮て、漉す。

\href{欠落アリ}{}

原注:クールブイヨンBとC\footnote{クールブイヨンBの白ワインを赤ワインに代え、香味素材としてにんじん400gを加える。鱒、鯉、マトロート用(原書pp.277-278)。}で調理した魚はクールブイヨン添えとして供する。つまり、少量の煮汁とクールブイヨンに用いた野菜を添える。野菜はよく火が通っていること。煮汁はしっかり煮詰め、提供直前に新鮮なバター少量を加えて仕上げる。
\end{recette}
\hypertarget{ux30afux30fcux30ebux30d6ux30a4ux30e8ux30f3ux306eux4f7fux3044ux65b9-ux539fux66f8-p.278}{%
\subsection{クールブイヨンの使い方 原書
p.278}\label{ux30afux30fcux30ebux30d6ux30a4ux30e8ux30f3ux306eux4f7fux3044ux65b9-ux539fux66f8-p.278}}

\begin{enumerate}
\def\labelenumi{\arabic{enumi}.}
\item
  加熱時間が30分以内の場合は、クールブイヨンは必ず事前に用意しておくこと。
\item
  加熱時間が30分を越える場合は、クールブイヨンの材料は冷たい状態のままで合わせせておく。香味素材はポワソニエールの網の下に入れる。
\item
  ごく少量のクールブイヨンでポシェ\footnote{原文 pochage à court
    mouillement『ル・ギード・キュリネール』では、この表現はテュルボタン(小型のテュルボ)、バルビュ、舌びらめ等の平たい魚をポシェする際に用いられる。本連載「舌びらめのボヌ・ファム」
    2011年3月号pp.110-111 参照。}する場合、材料は(白または赤ワインを含む場合も)魚を火にかける際に合わせる。クールブイヨンの量は魚の
  \(\frac{1}{3}\)
  の高さとし、加熱中ひんぱんに煮汁を魚にかけてやること\footnote{arroser
    アロゼ。}。この調理法の場合は通常、クールブイヨンは上で記したように
  \footnote{「クールブイヨンB」原注。}、提供直前に軽くバターを加えて仕上げ、魚に添える。
\item
  冷製にする場合は、必ずクールブイヨンに魚が浸った状態で冷ますこと。当然ながら、火にかけている時間は短かくなる\footnote{余熱で火が通るため。}。
\end{enumerate}

\hypertarget{ux539fux6ce8}{%
\subparagraph{【原注】}\label{ux539fux6ce8}}

いくつかの魚種の加熱時間は該当する項で示してある。

\hypertarget{ux9b5aux306eux8abfux7406ux6cd5}{%
\section{魚の調理法}\label{ux9b5aux306eux8abfux7406ux6cd5}}

魚料理は全て、下記のいずれかの調理法による。

\begin{enumerate}
\def\labelenumi{\arabic{enumi}.}
\item
  塩水(湯)またはクールブイヨン\footnote{court-bouillon直訳は「量の少ない煮汁」。魚の他、甲殻類、鶏などの白身肉、野菜などをポシェするのに用いる。とりわけ魚や鶏を丸ごとポシェする場合には、その名称のとおり、できるだけ少量でポシェする必要がある。また、ポシェに用いたクールブイヨンをベースにソースを作る場合が多い。}Bを用いたポシェ\ldots{}\ldots{}大きな魚丸ごと、および切り身。
\item
  ごく少量のクールブイヨンを用いたポシェ\ldots{}\ldots{}魚のフィレ、またはやや小さい魚。
\item
  ブレゼ\ldots{}\ldots{}もっぱら大きな魚。
\item
  オ・ブルー\footnote{比較的小さめの淡水魚に主として用いられる調理法。生きたままの魚の表面のぬめりをとらないように洗い、内臓を取り除いたらすぐに、塩とヴィネガーを加えたクールブイヨンで茹でる。冷製、温製どちらでも供する。原書p.281参照。}\ldots{}\ldots{}とりわけ\ruby{鱒}{ます}、鯉、ブロシェ\footnote{川かますの一種。本連載「ブロシェのクネル」2011年10月号
    pp.124-125 参照。}に合う。
\item
  揚げもの\ldots{}\ldots{}もっぱら小さい魚、切り身。
\item
  ムニエール\ldots{}\ldots{}揚げものにするのと同じ小さい魚、切り身。
\item
  グリエ\ldots{}\ldots{}小さい魚、および切り身。
\item
  グラタン\ldots{}\ldots{}小さい魚、切り身。
\end{enumerate}

\hypertarget{ux5869ux6c34ux6e6fux304aux3088ux3073ux30afux30fcux30ebux30d6ux30a4ux30e8ux30f3bux3092ux7528ux3044ux305fux52a0ux71b1ux8abfux7406}{%
\subsection{塩水(湯)およびクールブイヨンBを用いた加熱調理}\label{ux5869ux6c34ux6e6fux304aux3088ux3073ux30afux30fcux30ebux30d6ux30a4ux30e8ux30f3bux3092ux7528ux3044ux305fux52a0ux71b1ux8abfux7406}}

魚を丸ごと調理する場合は、魚に合ったポワソニエール\footnote{大きな魚を丸ごと煮るための細長い鍋。魚の形を崩さずに取り出せるよう、中に専用の網を敷いて使う。似たものに、舌びらめ等の平たい魚にぴったり合う菱形をしたテュルボティエールがある。いずれも、できるだけ少量の煮汁で魚を加熱できるように工夫されたもの。(図参照)}を用いる。魚を掃除し(テュルボは水にさらして血抜きをし)、ひれ等を切り落して形を整え、ポワソニエールの網に乗せる。魚種に応じて塩水または冷たいクールブイヨンをかぶるまで注ぐ。強火にかけて沸騰したらすぐにレンジの火の弱いところに鍋を移動させ、ポシェする。

切り身(薄すぎは絶対にいけない)の場合、沸騰した液体(塩湯またはクールブイヨン)に投入したらすぐにレンジの火の弱いところに鍋を移動させ、沸騰しない程度の温度でゆっくりと火を通す。

こうするのは、魚の身のエキスを閉じこめるためである。冷水から火にかけた場合にはエキスの大部分が流れ出してしまう。大きな魚丸ごとの場合にはこのやり方はしない。沸騰した液体に魚を投入すると身が収縮するので、大きな魚の場合は身が割れたり形が崩れたりするからだ。

塩湯あるいはクールブイヨンでポシェした魚は、ナフキンまたは専用の網に盛る。周囲をパセリで飾り、塩茹でしたじゃがいもと1種類または数種のソースを添えて供する。ガルニテュールがパセリのみの場合、魚の周囲にパセリを飾るのは客に料理を見せる\footnote{当時の宴席で主流だったロシア式サーヴィスでは、大きな銀盆に盛った料理をまず食客に見せてから、とり分けて給仕する。}直前にすること。どんな場合でも、ガルニテュールを添えたらクロッシュ\footnote{銀または陶製の保温用皿カバー。}は被せないこと。

\hypertarget{ux3054ux304fux5c11ux91cfux306eux30afux30fcux30ebux30d6ux30a4ux30e8ux30f3ux3092ux7528ux3044ux305fux30ddux30b7ux30a7-ux539fux66f8-pp.279-280}{%
\subsection{ごく少量のクールブイヨンを用いたポシェ 原書
pp.279-280}\label{ux3054ux304fux5c11ux91cfux306eux30afux30fcux30ebux30d6ux30a4ux30e8ux30f3ux3092ux7528ux3044ux305fux30ddux30b7ux30a7-ux539fux66f8-pp.279-280}}

この火入れの方法は主としてテュルボタン、バルビュ、舌びらめ、丸ごとおよびそれぞれの魚のフィレで用いる。バターを塗った天板あるいはソテ鍋に魚丸ごとあるいはそのフィレを置き、軽く塩をして、所要量の魚のフュメかマッシュルームの煮汁を注ぐ。フュメとマッシュルームの煮汁を合わせたものを用いる場合もある。蓋をして、中温のオーヴンに入れる。魚丸ごとの場合は時折煮汁をかけてやる。

魚(丸ごとでもフィレでも)に火が通ったら、注意して汁気をきり、皿に盛る。ガルニテュールを含む料理の場合、ガルニテュールを魚の周囲に盛り、ソースをかける\footnote{原文通りの順で訳したが、実際にはソースをかけてからガルニテュールを盛ったほうが良い場合もあるだろう。}。多くの場合、魚の煮汁を煮詰めてソースに加える。

\href{欠落アリ}{}

\hypertarget{ux9b5aux306eux30d6ux30ecux30bc17-ux539fux66f8-p.280}{%
\subsection[魚のブレゼ 原書 p.280]{\texorpdfstring{魚のブレゼ\footnote{ここではブレゼの語が限定的な意味で用いられていることに注意。牛のアロワイヨのような大きな塊肉のブレゼと同様の調理法、ということである。それは、香味素材を色づくまで炒めてから用いることや、主素材に豚背脂やトリュフをピケ針で差したり、豚背脂のシートで覆って加熱するという点によく表れている。ただし、これらは必須というわけではないため、事実上は「ごく少量のクールブイヨンを用いたポシェ」と区別がつきにくい。実際、モンタニェ『ラルース・ガストロノミーク』初版では、魚のブレゼについて「本来的な意味でのブレゼというよりは、ごく少量のクールブイヨンを用いたポシェである」と述べられている。逆に言えば、こんにち魚の加熱方法についてしばしば「ブレゼ」と呼ばれているものが、エスコフィエやモンタニェにおいては「少量のクールブイヨンを用いたポシェ」と表現されていたということである。}
原書
p.280}{魚のブレゼ 原書 p.280}}\label{ux9b5aux306eux30d6ux30ecux30bc17-ux539fux66f8-p.280}}

この調理法を用いるのは通常、丸ごとまたは筒切りにした鮭、大ぶりの鱒、テュルボ、テュルボタンのうち大きなもの、である。

場合によっては、魚の片面に、小さく切った豚背脂、トリュフ、コルニション、にんじん等をピケ針で差し込む。

香味素材等\footnote{原文 fonds de braisage フォン・ド・ブレザージュ
  (fonds de
  braiseフォン・ド・ブレーズ、とも)。通常は、厚い輪切りにしたにんじんと玉ねぎをバターか獣脂で色づくまで炒め、ブーケガルニ、下茹でした豚皮を合わせる(原書pp.394-395)。また、これを用いた煮汁のことも指す。}は肉料理のブレゼの場合と同じように用意するが、豚皮は用いない。提供方法に応じて、白または赤ワインと軽い魚のフュメ同量ずつを、魚の厚みの
\(\frac{3}{4}\) またはひたひたの高さまで注ぐ。厳密に肉断ち \footnote{カトリックの生活習慣として、四旬節(復活祭までの46日間)および週1
  回程度、肉類を食べないということが行なわれた(本連載2012年5月号「ソース・エスパニョル(4)」p.110、訳注1参照)。}のための仕立てにする場合を除いて、薄くスライスした豚背脂のシートを魚にかぶせる。加熱中\footnote{鍋を火にかけ、沸騰したら蓋をして中火のオーヴンに入れ、加熱する。}こまめに煮汁を魚にかけてやる\footnote{arroser
  アロゼ。}。また、完全には蓋をせず、加熱中に煮汁が煮詰まるようにしてやる。

ほぼ火が通ったら、鍋の蓋をとり、魚にかけた煮汁の水分をオーヴンの熱で蒸発させて表面につやを出す\footnote{glacer
  グラセ。}。魚を鍋から出して汁気をきり、皿に盛り保温しておく。

煮汁\footnote{原文 fonds de braisage (訳注2参照)。}を漉し、しばらく休ませたら浮き脂を取り除き、必要なら煮詰める。これを加えてソースを仕上げる。

魚のブレゼには通常、各ルセットに示してあるガルニテュールを添える。

\hypertarget{ux30aaux30d6ux30ebux30fc36-ux539fux66f8-pp.280-281}{%
\subsection[オ・ブルー 原書
pp.280-281]{\texorpdfstring{オ・ブルー\footnote{au bleu
  ヴィネガーを加えることで魚の表面のぬめりが青みがかることから。} 原書
pp.280-281}{オ・ブルー 原書 pp.280-281}}\label{ux30aaux30d6ux30ebux30fc36-ux539fux66f8-pp.280-281}}

オ・ブルーは鱒、鯉、ブロシェ\footnote{川かますの一種。}のみに用いられる特殊な調理法で、基本的なポイントは以下のとおり。

\begin{enumerate}
\def\labelenumi{\arabic{enumi}.}
\item
  必ず、生きた魚を使う。
\item
  魚の表面のぬめりをとらないように、なるべく手で触れずに、わたを抜く。鱗も引かない。
\item
  魚が大きい場合は、専用の網を敷いたポワソニーエルに入れ、「沸騰したヴィネガーをかける」。ヴィネガーは、通常のクールブイヨンに加える分量\footnote{以下の「クールブイヨンA」の分量比率を参照。}。次に、ヴィネガーを入れずに用意した温かい\footnote{原文
    tiède ぬるい、温かい。}クールブイヨンを注ぎ入れる。これは、なるべく身が割れないようにするためである。その後は通常どおり加熱する\footnote{レンジで沸騰させたらオーヴンに入れる。}。
\item
  小さい鱒の場合は、生きたままのものを手早く中抜きし、塩、ヴィネガーを加えただけの沸騰したクールブイヨンで煮る。
\item
  オ・ブルーは冷製、温製どちらの仕立てにしてもいい。実際の作り方の項で示してあるソースを添えて供する。
\end{enumerate}

\href{欠落アリ}{}

\hypertarget{ux30e0ux30cbux30a8ux30fcux30eb43-ux539fux66f8-p.282}{%
\subsection[ムニエール 原書 p.282]{\texorpdfstring{ムニエール\footnote{à
  la meunière 「粉挽き職人風」の意。} 原書
p.282}{ムニエール 原書 p.282}}\label{ux30e0ux30cbux30a8ux30fcux30eb43-ux539fux66f8-p.282}}

ムニエールは素晴らしい調理法だが、小型の魚と、大きな魚の場合は切り身にしか用いない。とはいえ、丁寧にやれば1.5
kg以下のテュルボタンはムニエールで調理できる。

魚丸ごと、あるいは切り身、フィレに味つけをして小麦粉をまぶし、バターを熱したフライパンで焼く。

魚が小さい場合は普通のバターでいいが、大きい場合は澄ましバターを使った方がいい。

魚の両面を焼き、程良く火が通ったら、予め熱しておいた皿に盛る。

飾り切りにした半割りのレモンを添えて、そのまま供することも可能である。ただし、このような提供方法の場合は本来の「ムニエール」と区別するために「黄金色に焼いた\footnote{doré
  (ドレ)
  一般的な色の表現として「黄金色」の意だが、ムニエールの場合、通常は大きい魚についてのみこの表現を用いる。}」と表現する。

「ムニエール」の場合には、焼き上がった魚に少量のレモン汁をふり、塩、こしょう少々で味を整える。粗みじん切りにして湯通ししたパセリを魚の表面に散らし、焦がしバターをかけてすぐに供する。湯通ししたパセリの水分に熱いバターが触れて泡がたつので、それが消えないうちに客に料理を見せるようにする。

%\input{06-poissons/06-02-p-284}
%\input{06-poissons/06-03-pp285-286}
%\input{06-poissons/06-04-p286}
%\input{06-poissons/06-05-pp286-290}

%\href{✓原稿下準備なし}{} \href{訳と注釈\%2020180420進行中}{}
\href{未、原文対照チェック}{} \href{未、日本語表現校正}{}
\href{未、注釈チェク}{} \href{未、原稿最終校正}{}

\hypertarget{poissons}{%
\chapter{VI 魚料理 Poissons}\label{poissons}}

\hypertarget{serie-de-courts-bouillons-de-poisson}{%
\section{クールブイヨン}\label{serie-de-courts-bouillons-de-poisson}}

\begin{recette}
\hypertarget{ux30afux30fcux30ebux30d6ux30a4ux30e8ux30f3-a}{%
\subsubsection{クールブイヨン
A}\label{ux30afux30fcux30ebux30d6ux30a4ux30e8ux30f3-a}}

水5 L に対し、ヴィネガー2.5 dL、粗塩60 g、薄切りにしたにんじん600
gと玉ねぎ500 g、タイム1枝、ローリエの小さな葉2枚、パセリの茎100
g、粒こしょう20
g(こしょうを加えるのはクールブイヨンを漉す10分前)。材料を全て鍋に入れ、火にかけて1時間弱火で煮、漉す(原書
p.277)。

\hypertarget{ux30afux30fcux30ebux30d6ux30a4ux30e8ux30f3-b-4-ux9c52ux3046ux306aux304eux30d6ux30edux30b7ux30a7ux7b49-ux539fux66f8p.277}{%
\subsubsection[クールブイヨン B (鱒、うなぎ、ブロシェ等)
原書p.277]{\texorpdfstring{クールブイヨン B \footnote{クールブイヨンは用途に応じ、AからEまでの5種が挙げられている(原書pp.277-278)。}
(鱒、うなぎ、ブロシェ等)
原書p.277}{クールブイヨン B  (鱒、うなぎ、ブロシェ等) 原書p.277}}\label{ux30afux30fcux30ebux30d6ux30a4ux30e8ux30f3-b-4-ux9c52ux3046ux306aux304eux30d6ux30edux30b7ux30a7ux7b49-ux539fux66f8p.277}}

5 L 分の材料\ldots{}\ldots{}白ワイン2.5 L 。水2.5 L
。薄切りにした玉ねぎ600 g。パセリの茎80g
。タイムの小枝1本。ローリエの葉(小) \(\frac{1}{2}\) 枚。粗塩
60g。大粒のこしょう15 g(クールブイヨンを漉す10分前に加える)。

作業手順\ldots{}\ldots{}作業:液体、香味素材、調味料を鍋に入れ、沸かす。弱火で30分程煮て、漉す。

\href{欠落アリ}{}

原注:クールブイヨンBとC\footnote{クールブイヨンBの白ワインを赤ワインに代え、香味素材としてにんじん400gを加える。鱒、鯉、マトロート用(原書pp.277-278)。}で調理した魚はクールブイヨン添えとして供する。つまり、少量の煮汁とクールブイヨンに用いた野菜を添える。野菜はよく火が通っていること。煮汁はしっかり煮詰め、提供直前に新鮮なバター少量を加えて仕上げる。
\end{recette}
\hypertarget{ux30afux30fcux30ebux30d6ux30a4ux30e8ux30f3ux306eux4f7fux3044ux65b9-ux539fux66f8-p.278}{%
\subsection{クールブイヨンの使い方 原書
p.278}\label{ux30afux30fcux30ebux30d6ux30a4ux30e8ux30f3ux306eux4f7fux3044ux65b9-ux539fux66f8-p.278}}

\begin{enumerate}
\def\labelenumi{\arabic{enumi}.}
\item
  加熱時間が30分以内の場合は、クールブイヨンは必ず事前に用意しておくこと。
\item
  加熱時間が30分を越える場合は、クールブイヨンの材料は冷たい状態のままで合わせせておく。香味素材はポワソニエールの網の下に入れる。
\item
  ごく少量のクールブイヨンでポシェ\footnote{原文 pochage à court
    mouillement『ル・ギード・キュリネール』では、この表現はテュルボタン(小型のテュルボ)、バルビュ、舌びらめ等の平たい魚をポシェする際に用いられる。本連載「舌びらめのボヌ・ファム」
    2011年3月号pp.110-111 参照。}する場合、材料は(白または赤ワインを含む場合も)魚を火にかける際に合わせる。クールブイヨンの量は魚の
  \(\frac{1}{3}\)
  の高さとし、加熱中ひんぱんに煮汁を魚にかけてやること\footnote{arroser
    アロゼ。}。この調理法の場合は通常、クールブイヨンは上で記したように
  \footnote{「クールブイヨンB」原注。}、提供直前に軽くバターを加えて仕上げ、魚に添える。
\item
  冷製にする場合は、必ずクールブイヨンに魚が浸った状態で冷ますこと。当然ながら、火にかけている時間は短かくなる\footnote{余熱で火が通るため。}。
\end{enumerate}

\hypertarget{ux539fux6ce8}{%
\subparagraph{【原注】}\label{ux539fux6ce8}}

いくつかの魚種の加熱時間は該当する項で示してある。

\hypertarget{ux9b5aux306eux8abfux7406ux6cd5}{%
\section{魚の調理法}\label{ux9b5aux306eux8abfux7406ux6cd5}}

魚料理は全て、下記のいずれかの調理法による。

\begin{enumerate}
\def\labelenumi{\arabic{enumi}.}
\item
  塩水(湯)またはクールブイヨン\footnote{court-bouillon直訳は「量の少ない煮汁」。魚の他、甲殻類、鶏などの白身肉、野菜などをポシェするのに用いる。とりわけ魚や鶏を丸ごとポシェする場合には、その名称のとおり、できるだけ少量でポシェする必要がある。また、ポシェに用いたクールブイヨンをベースにソースを作る場合が多い。}Bを用いたポシェ\ldots{}\ldots{}大きな魚丸ごと、および切り身。
\item
  ごく少量のクールブイヨンを用いたポシェ\ldots{}\ldots{}魚のフィレ、またはやや小さい魚。
\item
  ブレゼ\ldots{}\ldots{}もっぱら大きな魚。
\item
  オ・ブルー\footnote{比較的小さめの淡水魚に主として用いられる調理法。生きたままの魚の表面のぬめりをとらないように洗い、内臓を取り除いたらすぐに、塩とヴィネガーを加えたクールブイヨンで茹でる。冷製、温製どちらでも供する。原書p.281参照。}\ldots{}\ldots{}とりわけ\ruby{鱒}{ます}、鯉、ブロシェ\footnote{川かますの一種。本連載「ブロシェのクネル」2011年10月号
    pp.124-125 参照。}に合う。
\item
  揚げもの\ldots{}\ldots{}もっぱら小さい魚、切り身。
\item
  ムニエール\ldots{}\ldots{}揚げものにするのと同じ小さい魚、切り身。
\item
  グリエ\ldots{}\ldots{}小さい魚、および切り身。
\item
  グラタン\ldots{}\ldots{}小さい魚、切り身。
\end{enumerate}

\hypertarget{ux5869ux6c34ux6e6fux304aux3088ux3073ux30afux30fcux30ebux30d6ux30a4ux30e8ux30f3bux3092ux7528ux3044ux305fux52a0ux71b1ux8abfux7406}{%
\subsection{塩水(湯)およびクールブイヨンBを用いた加熱調理}\label{ux5869ux6c34ux6e6fux304aux3088ux3073ux30afux30fcux30ebux30d6ux30a4ux30e8ux30f3bux3092ux7528ux3044ux305fux52a0ux71b1ux8abfux7406}}

魚を丸ごと調理する場合は、魚に合ったポワソニエール\footnote{大きな魚を丸ごと煮るための細長い鍋。魚の形を崩さずに取り出せるよう、中に専用の網を敷いて使う。似たものに、舌びらめ等の平たい魚にぴったり合う菱形をしたテュルボティエールがある。いずれも、できるだけ少量の煮汁で魚を加熱できるように工夫されたもの。(図参照)}を用いる。魚を掃除し(テュルボは水にさらして血抜きをし)、ひれ等を切り落して形を整え、ポワソニエールの網に乗せる。魚種に応じて塩水または冷たいクールブイヨンをかぶるまで注ぐ。強火にかけて沸騰したらすぐにレンジの火の弱いところに鍋を移動させ、ポシェする。

切り身(薄すぎは絶対にいけない)の場合、沸騰した液体(塩湯またはクールブイヨン)に投入したらすぐにレンジの火の弱いところに鍋を移動させ、沸騰しない程度の温度でゆっくりと火を通す。

こうするのは、魚の身のエキスを閉じこめるためである。冷水から火にかけた場合にはエキスの大部分が流れ出してしまう。大きな魚丸ごとの場合にはこのやり方はしない。沸騰した液体に魚を投入すると身が収縮するので、大きな魚の場合は身が割れたり形が崩れたりするからだ。

塩湯あるいはクールブイヨンでポシェした魚は、ナフキンまたは専用の網に盛る。周囲をパセリで飾り、塩茹でしたじゃがいもと1種類または数種のソースを添えて供する。ガルニテュールがパセリのみの場合、魚の周囲にパセリを飾るのは客に料理を見せる\footnote{当時の宴席で主流だったロシア式サーヴィスでは、大きな銀盆に盛った料理をまず食客に見せてから、とり分けて給仕する。}直前にすること。どんな場合でも、ガルニテュールを添えたらクロッシュ\footnote{銀または陶製の保温用皿カバー。}は被せないこと。

\hypertarget{ux3054ux304fux5c11ux91cfux306eux30afux30fcux30ebux30d6ux30a4ux30e8ux30f3ux3092ux7528ux3044ux305fux30ddux30b7ux30a7-ux539fux66f8-pp.279-280}{%
\subsection{ごく少量のクールブイヨンを用いたポシェ 原書
pp.279-280}\label{ux3054ux304fux5c11ux91cfux306eux30afux30fcux30ebux30d6ux30a4ux30e8ux30f3ux3092ux7528ux3044ux305fux30ddux30b7ux30a7-ux539fux66f8-pp.279-280}}

この火入れの方法は主としてテュルボタン、バルビュ、舌びらめ、丸ごとおよびそれぞれの魚のフィレで用いる。バターを塗った天板あるいはソテ鍋に魚丸ごとあるいはそのフィレを置き、軽く塩をして、所要量の魚のフュメかマッシュルームの煮汁を注ぐ。フュメとマッシュルームの煮汁を合わせたものを用いる場合もある。蓋をして、中温のオーヴンに入れる。魚丸ごとの場合は時折煮汁をかけてやる。

魚(丸ごとでもフィレでも)に火が通ったら、注意して汁気をきり、皿に盛る。ガルニテュールを含む料理の場合、ガルニテュールを魚の周囲に盛り、ソースをかける\footnote{原文通りの順で訳したが、実際にはソースをかけてからガルニテュールを盛ったほうが良い場合もあるだろう。}。多くの場合、魚の煮汁を煮詰めてソースに加える。

\href{欠落アリ}{}

\hypertarget{ux9b5aux306eux30d6ux30ecux30bc17-ux539fux66f8-p.280}{%
\subsection[魚のブレゼ 原書 p.280]{\texorpdfstring{魚のブレゼ\footnote{ここではブレゼの語が限定的な意味で用いられていることに注意。牛のアロワイヨのような大きな塊肉のブレゼと同様の調理法、ということである。それは、香味素材を色づくまで炒めてから用いることや、主素材に豚背脂やトリュフをピケ針で差したり、豚背脂のシートで覆って加熱するという点によく表れている。ただし、これらは必須というわけではないため、事実上は「ごく少量のクールブイヨンを用いたポシェ」と区別がつきにくい。実際、モンタニェ『ラルース・ガストロノミーク』初版では、魚のブレゼについて「本来的な意味でのブレゼというよりは、ごく少量のクールブイヨンを用いたポシェである」と述べられている。逆に言えば、こんにち魚の加熱方法についてしばしば「ブレゼ」と呼ばれているものが、エスコフィエやモンタニェにおいては「少量のクールブイヨンを用いたポシェ」と表現されていたということである。}
原書
p.280}{魚のブレゼ 原書 p.280}}\label{ux9b5aux306eux30d6ux30ecux30bc17-ux539fux66f8-p.280}}

この調理法を用いるのは通常、丸ごとまたは筒切りにした鮭、大ぶりの鱒、テュルボ、テュルボタンのうち大きなもの、である。

場合によっては、魚の片面に、小さく切った豚背脂、トリュフ、コルニション、にんじん等をピケ針で差し込む。

香味素材等\footnote{原文 fonds de braisage フォン・ド・ブレザージュ
  (fonds de
  braiseフォン・ド・ブレーズ、とも)。通常は、厚い輪切りにしたにんじんと玉ねぎをバターか獣脂で色づくまで炒め、ブーケガルニ、下茹でした豚皮を合わせる(原書pp.394-395)。また、これを用いた煮汁のことも指す。}は肉料理のブレゼの場合と同じように用意するが、豚皮は用いない。提供方法に応じて、白または赤ワインと軽い魚のフュメ同量ずつを、魚の厚みの
\(\frac{3}{4}\) またはひたひたの高さまで注ぐ。厳密に肉断ち \footnote{カトリックの生活習慣として、四旬節(復活祭までの46日間)および週1
  回程度、肉類を食べないということが行なわれた(本連載2012年5月号「ソース・エスパニョル(4)」p.110、訳注1参照)。}のための仕立てにする場合を除いて、薄くスライスした豚背脂のシートを魚にかぶせる。加熱中\footnote{鍋を火にかけ、沸騰したら蓋をして中火のオーヴンに入れ、加熱する。}こまめに煮汁を魚にかけてやる\footnote{arroser
  アロゼ。}。また、完全には蓋をせず、加熱中に煮汁が煮詰まるようにしてやる。

ほぼ火が通ったら、鍋の蓋をとり、魚にかけた煮汁の水分をオーヴンの熱で蒸発させて表面につやを出す\footnote{glacer
  グラセ。}。魚を鍋から出して汁気をきり、皿に盛り保温しておく。

煮汁\footnote{原文 fonds de braisage (訳注2参照)。}を漉し、しばらく休ませたら浮き脂を取り除き、必要なら煮詰める。これを加えてソースを仕上げる。

魚のブレゼには通常、各ルセットに示してあるガルニテュールを添える。

\hypertarget{ux30aaux30d6ux30ebux30fc36-ux539fux66f8-pp.280-281}{%
\subsection[オ・ブルー 原書
pp.280-281]{\texorpdfstring{オ・ブルー\footnote{au bleu
  ヴィネガーを加えることで魚の表面のぬめりが青みがかることから。} 原書
pp.280-281}{オ・ブルー 原書 pp.280-281}}\label{ux30aaux30d6ux30ebux30fc36-ux539fux66f8-pp.280-281}}

オ・ブルーは鱒、鯉、ブロシェ\footnote{川かますの一種。}のみに用いられる特殊な調理法で、基本的なポイントは以下のとおり。

\begin{enumerate}
\def\labelenumi{\arabic{enumi}.}
\item
  必ず、生きた魚を使う。
\item
  魚の表面のぬめりをとらないように、なるべく手で触れずに、わたを抜く。鱗も引かない。
\item
  魚が大きい場合は、専用の網を敷いたポワソニーエルに入れ、「沸騰したヴィネガーをかける」。ヴィネガーは、通常のクールブイヨンに加える分量\footnote{以下の「クールブイヨンA」の分量比率を参照。}。次に、ヴィネガーを入れずに用意した温かい\footnote{原文
    tiède ぬるい、温かい。}クールブイヨンを注ぎ入れる。これは、なるべく身が割れないようにするためである。その後は通常どおり加熱する\footnote{レンジで沸騰させたらオーヴンに入れる。}。
\item
  小さい鱒の場合は、生きたままのものを手早く中抜きし、塩、ヴィネガーを加えただけの沸騰したクールブイヨンで煮る。
\item
  オ・ブルーは冷製、温製どちらの仕立てにしてもいい。実際の作り方の項で示してあるソースを添えて供する。
\end{enumerate}

\href{欠落アリ}{}

\hypertarget{ux30e0ux30cbux30a8ux30fcux30eb43-ux539fux66f8-p.282}{%
\subsection[ムニエール 原書 p.282]{\texorpdfstring{ムニエール\footnote{à
  la meunière 「粉挽き職人風」の意。} 原書
p.282}{ムニエール 原書 p.282}}\label{ux30e0ux30cbux30a8ux30fcux30eb43-ux539fux66f8-p.282}}

ムニエールは素晴らしい調理法だが、小型の魚と、大きな魚の場合は切り身にしか用いない。とはいえ、丁寧にやれば1.5
kg以下のテュルボタンはムニエールで調理できる。

魚丸ごと、あるいは切り身、フィレに味つけをして小麦粉をまぶし、バターを熱したフライパンで焼く。

魚が小さい場合は普通のバターでいいが、大きい場合は澄ましバターを使った方がいい。

魚の両面を焼き、程良く火が通ったら、予め熱しておいた皿に盛る。

飾り切りにした半割りのレモンを添えて、そのまま供することも可能である。ただし、このような提供方法の場合は本来の「ムニエール」と区別するために「黄金色に焼いた\footnote{doré
  (ドレ)
  一般的な色の表現として「黄金色」の意だが、ムニエールの場合、通常は大きい魚についてのみこの表現を用いる。}」と表現する。

「ムニエール」の場合には、焼き上がった魚に少量のレモン汁をふり、塩、こしょう少々で味を整える。粗みじん切りにして湯通ししたパセリを魚の表面に散らし、焦がしバターをかけてすぐに供する。湯通ししたパセリの水分に熱いバターが触れて泡がたつので、それが消えないうちに客に料理を見せるようにする。

%\href{✓原稿下準備なし}{} \href{訳と注釈\%2020180420進行中}{}
\href{未、原文対照チェック}{} \href{未、日本語表現校正}{}
\href{未、注釈チェク}{} \href{未、原稿最終校正}{}

\hypertarget{poissons}{%
\chapter{VI 魚料理 Poissons}\label{poissons}}

\hypertarget{serie-de-courts-bouillons-de-poisson}{%
\section{クールブイヨン}\label{serie-de-courts-bouillons-de-poisson}}

\begin{recette}
\hypertarget{ux30afux30fcux30ebux30d6ux30a4ux30e8ux30f3-a}{%
\subsubsection{クールブイヨン
A}\label{ux30afux30fcux30ebux30d6ux30a4ux30e8ux30f3-a}}

水5 L に対し、ヴィネガー2.5 dL、粗塩60 g、薄切りにしたにんじん600
gと玉ねぎ500 g、タイム1枝、ローリエの小さな葉2枚、パセリの茎100
g、粒こしょう20
g(こしょうを加えるのはクールブイヨンを漉す10分前)。材料を全て鍋に入れ、火にかけて1時間弱火で煮、漉す(原書
p.277)。

\hypertarget{ux30afux30fcux30ebux30d6ux30a4ux30e8ux30f3-b-4-ux9c52ux3046ux306aux304eux30d6ux30edux30b7ux30a7ux7b49-ux539fux66f8p.277}{%
\subsubsection[クールブイヨン B (鱒、うなぎ、ブロシェ等)
原書p.277]{\texorpdfstring{クールブイヨン B \footnote{クールブイヨンは用途に応じ、AからEまでの5種が挙げられている(原書pp.277-278)。}
(鱒、うなぎ、ブロシェ等)
原書p.277}{クールブイヨン B  (鱒、うなぎ、ブロシェ等) 原書p.277}}\label{ux30afux30fcux30ebux30d6ux30a4ux30e8ux30f3-b-4-ux9c52ux3046ux306aux304eux30d6ux30edux30b7ux30a7ux7b49-ux539fux66f8p.277}}

5 L 分の材料\ldots{}\ldots{}白ワイン2.5 L 。水2.5 L
。薄切りにした玉ねぎ600 g。パセリの茎80g
。タイムの小枝1本。ローリエの葉(小) \(\frac{1}{2}\) 枚。粗塩
60g。大粒のこしょう15 g(クールブイヨンを漉す10分前に加える)。

作業手順\ldots{}\ldots{}作業:液体、香味素材、調味料を鍋に入れ、沸かす。弱火で30分程煮て、漉す。

\href{欠落アリ}{}

原注:クールブイヨンBとC\footnote{クールブイヨンBの白ワインを赤ワインに代え、香味素材としてにんじん400gを加える。鱒、鯉、マトロート用(原書pp.277-278)。}で調理した魚はクールブイヨン添えとして供する。つまり、少量の煮汁とクールブイヨンに用いた野菜を添える。野菜はよく火が通っていること。煮汁はしっかり煮詰め、提供直前に新鮮なバター少量を加えて仕上げる。
\end{recette}
\hypertarget{ux30afux30fcux30ebux30d6ux30a4ux30e8ux30f3ux306eux4f7fux3044ux65b9-ux539fux66f8-p.278}{%
\subsection{クールブイヨンの使い方 原書
p.278}\label{ux30afux30fcux30ebux30d6ux30a4ux30e8ux30f3ux306eux4f7fux3044ux65b9-ux539fux66f8-p.278}}

\begin{enumerate}
\def\labelenumi{\arabic{enumi}.}
\item
  加熱時間が30分以内の場合は、クールブイヨンは必ず事前に用意しておくこと。
\item
  加熱時間が30分を越える場合は、クールブイヨンの材料は冷たい状態のままで合わせせておく。香味素材はポワソニエールの網の下に入れる。
\item
  ごく少量のクールブイヨンでポシェ\footnote{原文 pochage à court
    mouillement『ル・ギード・キュリネール』では、この表現はテュルボタン(小型のテュルボ)、バルビュ、舌びらめ等の平たい魚をポシェする際に用いられる。本連載「舌びらめのボヌ・ファム」
    2011年3月号pp.110-111 参照。}する場合、材料は(白または赤ワインを含む場合も)魚を火にかける際に合わせる。クールブイヨンの量は魚の
  \(\frac{1}{3}\)
  の高さとし、加熱中ひんぱんに煮汁を魚にかけてやること\footnote{arroser
    アロゼ。}。この調理法の場合は通常、クールブイヨンは上で記したように
  \footnote{「クールブイヨンB」原注。}、提供直前に軽くバターを加えて仕上げ、魚に添える。
\item
  冷製にする場合は、必ずクールブイヨンに魚が浸った状態で冷ますこと。当然ながら、火にかけている時間は短かくなる\footnote{余熱で火が通るため。}。
\end{enumerate}

\hypertarget{ux539fux6ce8}{%
\subparagraph{【原注】}\label{ux539fux6ce8}}

いくつかの魚種の加熱時間は該当する項で示してある。

\hypertarget{ux9b5aux306eux8abfux7406ux6cd5}{%
\section{魚の調理法}\label{ux9b5aux306eux8abfux7406ux6cd5}}

魚料理は全て、下記のいずれかの調理法による。

\begin{enumerate}
\def\labelenumi{\arabic{enumi}.}
\item
  塩水(湯)またはクールブイヨン\footnote{court-bouillon直訳は「量の少ない煮汁」。魚の他、甲殻類、鶏などの白身肉、野菜などをポシェするのに用いる。とりわけ魚や鶏を丸ごとポシェする場合には、その名称のとおり、できるだけ少量でポシェする必要がある。また、ポシェに用いたクールブイヨンをベースにソースを作る場合が多い。}Bを用いたポシェ\ldots{}\ldots{}大きな魚丸ごと、および切り身。
\item
  ごく少量のクールブイヨンを用いたポシェ\ldots{}\ldots{}魚のフィレ、またはやや小さい魚。
\item
  ブレゼ\ldots{}\ldots{}もっぱら大きな魚。
\item
  オ・ブルー\footnote{比較的小さめの淡水魚に主として用いられる調理法。生きたままの魚の表面のぬめりをとらないように洗い、内臓を取り除いたらすぐに、塩とヴィネガーを加えたクールブイヨンで茹でる。冷製、温製どちらでも供する。原書p.281参照。}\ldots{}\ldots{}とりわけ\ruby{鱒}{ます}、鯉、ブロシェ\footnote{川かますの一種。本連載「ブロシェのクネル」2011年10月号
    pp.124-125 参照。}に合う。
\item
  揚げもの\ldots{}\ldots{}もっぱら小さい魚、切り身。
\item
  ムニエール\ldots{}\ldots{}揚げものにするのと同じ小さい魚、切り身。
\item
  グリエ\ldots{}\ldots{}小さい魚、および切り身。
\item
  グラタン\ldots{}\ldots{}小さい魚、切り身。
\end{enumerate}

\hypertarget{ux5869ux6c34ux6e6fux304aux3088ux3073ux30afux30fcux30ebux30d6ux30a4ux30e8ux30f3bux3092ux7528ux3044ux305fux52a0ux71b1ux8abfux7406}{%
\subsection{塩水(湯)およびクールブイヨンBを用いた加熱調理}\label{ux5869ux6c34ux6e6fux304aux3088ux3073ux30afux30fcux30ebux30d6ux30a4ux30e8ux30f3bux3092ux7528ux3044ux305fux52a0ux71b1ux8abfux7406}}

魚を丸ごと調理する場合は、魚に合ったポワソニエール\footnote{大きな魚を丸ごと煮るための細長い鍋。魚の形を崩さずに取り出せるよう、中に専用の網を敷いて使う。似たものに、舌びらめ等の平たい魚にぴったり合う菱形をしたテュルボティエールがある。いずれも、できるだけ少量の煮汁で魚を加熱できるように工夫されたもの。(図参照)}を用いる。魚を掃除し(テュルボは水にさらして血抜きをし)、ひれ等を切り落して形を整え、ポワソニエールの網に乗せる。魚種に応じて塩水または冷たいクールブイヨンをかぶるまで注ぐ。強火にかけて沸騰したらすぐにレンジの火の弱いところに鍋を移動させ、ポシェする。

切り身(薄すぎは絶対にいけない)の場合、沸騰した液体(塩湯またはクールブイヨン)に投入したらすぐにレンジの火の弱いところに鍋を移動させ、沸騰しない程度の温度でゆっくりと火を通す。

こうするのは、魚の身のエキスを閉じこめるためである。冷水から火にかけた場合にはエキスの大部分が流れ出してしまう。大きな魚丸ごとの場合にはこのやり方はしない。沸騰した液体に魚を投入すると身が収縮するので、大きな魚の場合は身が割れたり形が崩れたりするからだ。

塩湯あるいはクールブイヨンでポシェした魚は、ナフキンまたは専用の網に盛る。周囲をパセリで飾り、塩茹でしたじゃがいもと1種類または数種のソースを添えて供する。ガルニテュールがパセリのみの場合、魚の周囲にパセリを飾るのは客に料理を見せる\footnote{当時の宴席で主流だったロシア式サーヴィスでは、大きな銀盆に盛った料理をまず食客に見せてから、とり分けて給仕する。}直前にすること。どんな場合でも、ガルニテュールを添えたらクロッシュ\footnote{銀または陶製の保温用皿カバー。}は被せないこと。

\hypertarget{ux3054ux304fux5c11ux91cfux306eux30afux30fcux30ebux30d6ux30a4ux30e8ux30f3ux3092ux7528ux3044ux305fux30ddux30b7ux30a7-ux539fux66f8-pp.279-280}{%
\subsection{ごく少量のクールブイヨンを用いたポシェ 原書
pp.279-280}\label{ux3054ux304fux5c11ux91cfux306eux30afux30fcux30ebux30d6ux30a4ux30e8ux30f3ux3092ux7528ux3044ux305fux30ddux30b7ux30a7-ux539fux66f8-pp.279-280}}

この火入れの方法は主としてテュルボタン、バルビュ、舌びらめ、丸ごとおよびそれぞれの魚のフィレで用いる。バターを塗った天板あるいはソテ鍋に魚丸ごとあるいはそのフィレを置き、軽く塩をして、所要量の魚のフュメかマッシュルームの煮汁を注ぐ。フュメとマッシュルームの煮汁を合わせたものを用いる場合もある。蓋をして、中温のオーヴンに入れる。魚丸ごとの場合は時折煮汁をかけてやる。

魚(丸ごとでもフィレでも)に火が通ったら、注意して汁気をきり、皿に盛る。ガルニテュールを含む料理の場合、ガルニテュールを魚の周囲に盛り、ソースをかける\footnote{原文通りの順で訳したが、実際にはソースをかけてからガルニテュールを盛ったほうが良い場合もあるだろう。}。多くの場合、魚の煮汁を煮詰めてソースに加える。

\href{欠落アリ}{}

\hypertarget{ux9b5aux306eux30d6ux30ecux30bc17-ux539fux66f8-p.280}{%
\subsection[魚のブレゼ 原書 p.280]{\texorpdfstring{魚のブレゼ\footnote{ここではブレゼの語が限定的な意味で用いられていることに注意。牛のアロワイヨのような大きな塊肉のブレゼと同様の調理法、ということである。それは、香味素材を色づくまで炒めてから用いることや、主素材に豚背脂やトリュフをピケ針で差したり、豚背脂のシートで覆って加熱するという点によく表れている。ただし、これらは必須というわけではないため、事実上は「ごく少量のクールブイヨンを用いたポシェ」と区別がつきにくい。実際、モンタニェ『ラルース・ガストロノミーク』初版では、魚のブレゼについて「本来的な意味でのブレゼというよりは、ごく少量のクールブイヨンを用いたポシェである」と述べられている。逆に言えば、こんにち魚の加熱方法についてしばしば「ブレゼ」と呼ばれているものが、エスコフィエやモンタニェにおいては「少量のクールブイヨンを用いたポシェ」と表現されていたということである。}
原書
p.280}{魚のブレゼ 原書 p.280}}\label{ux9b5aux306eux30d6ux30ecux30bc17-ux539fux66f8-p.280}}

この調理法を用いるのは通常、丸ごとまたは筒切りにした鮭、大ぶりの鱒、テュルボ、テュルボタンのうち大きなもの、である。

場合によっては、魚の片面に、小さく切った豚背脂、トリュフ、コルニション、にんじん等をピケ針で差し込む。

香味素材等\footnote{原文 fonds de braisage フォン・ド・ブレザージュ
  (fonds de
  braiseフォン・ド・ブレーズ、とも)。通常は、厚い輪切りにしたにんじんと玉ねぎをバターか獣脂で色づくまで炒め、ブーケガルニ、下茹でした豚皮を合わせる(原書pp.394-395)。また、これを用いた煮汁のことも指す。}は肉料理のブレゼの場合と同じように用意するが、豚皮は用いない。提供方法に応じて、白または赤ワインと軽い魚のフュメ同量ずつを、魚の厚みの
\(\frac{3}{4}\) またはひたひたの高さまで注ぐ。厳密に肉断ち \footnote{カトリックの生活習慣として、四旬節(復活祭までの46日間)および週1
  回程度、肉類を食べないということが行なわれた(本連載2012年5月号「ソース・エスパニョル(4)」p.110、訳注1参照)。}のための仕立てにする場合を除いて、薄くスライスした豚背脂のシートを魚にかぶせる。加熱中\footnote{鍋を火にかけ、沸騰したら蓋をして中火のオーヴンに入れ、加熱する。}こまめに煮汁を魚にかけてやる\footnote{arroser
  アロゼ。}。また、完全には蓋をせず、加熱中に煮汁が煮詰まるようにしてやる。

ほぼ火が通ったら、鍋の蓋をとり、魚にかけた煮汁の水分をオーヴンの熱で蒸発させて表面につやを出す\footnote{glacer
  グラセ。}。魚を鍋から出して汁気をきり、皿に盛り保温しておく。

煮汁\footnote{原文 fonds de braisage (訳注2参照)。}を漉し、しばらく休ませたら浮き脂を取り除き、必要なら煮詰める。これを加えてソースを仕上げる。

魚のブレゼには通常、各ルセットに示してあるガルニテュールを添える。

\hypertarget{ux30aaux30d6ux30ebux30fc36-ux539fux66f8-pp.280-281}{%
\subsection[オ・ブルー 原書
pp.280-281]{\texorpdfstring{オ・ブルー\footnote{au bleu
  ヴィネガーを加えることで魚の表面のぬめりが青みがかることから。} 原書
pp.280-281}{オ・ブルー 原書 pp.280-281}}\label{ux30aaux30d6ux30ebux30fc36-ux539fux66f8-pp.280-281}}

オ・ブルーは鱒、鯉、ブロシェ\footnote{川かますの一種。}のみに用いられる特殊な調理法で、基本的なポイントは以下のとおり。

\begin{enumerate}
\def\labelenumi{\arabic{enumi}.}
\item
  必ず、生きた魚を使う。
\item
  魚の表面のぬめりをとらないように、なるべく手で触れずに、わたを抜く。鱗も引かない。
\item
  魚が大きい場合は、専用の網を敷いたポワソニーエルに入れ、「沸騰したヴィネガーをかける」。ヴィネガーは、通常のクールブイヨンに加える分量\footnote{以下の「クールブイヨンA」の分量比率を参照。}。次に、ヴィネガーを入れずに用意した温かい\footnote{原文
    tiède ぬるい、温かい。}クールブイヨンを注ぎ入れる。これは、なるべく身が割れないようにするためである。その後は通常どおり加熱する\footnote{レンジで沸騰させたらオーヴンに入れる。}。
\item
  小さい鱒の場合は、生きたままのものを手早く中抜きし、塩、ヴィネガーを加えただけの沸騰したクールブイヨンで煮る。
\item
  オ・ブルーは冷製、温製どちらの仕立てにしてもいい。実際の作り方の項で示してあるソースを添えて供する。
\end{enumerate}

\href{欠落アリ}{}

\hypertarget{ux30e0ux30cbux30a8ux30fcux30eb43-ux539fux66f8-p.282}{%
\subsection[ムニエール 原書 p.282]{\texorpdfstring{ムニエール\footnote{à
  la meunière 「粉挽き職人風」の意。} 原書
p.282}{ムニエール 原書 p.282}}\label{ux30e0ux30cbux30a8ux30fcux30eb43-ux539fux66f8-p.282}}

ムニエールは素晴らしい調理法だが、小型の魚と、大きな魚の場合は切り身にしか用いない。とはいえ、丁寧にやれば1.5
kg以下のテュルボタンはムニエールで調理できる。

魚丸ごと、あるいは切り身、フィレに味つけをして小麦粉をまぶし、バターを熱したフライパンで焼く。

魚が小さい場合は普通のバターでいいが、大きい場合は澄ましバターを使った方がいい。

魚の両面を焼き、程良く火が通ったら、予め熱しておいた皿に盛る。

飾り切りにした半割りのレモンを添えて、そのまま供することも可能である。ただし、このような提供方法の場合は本来の「ムニエール」と区別するために「黄金色に焼いた\footnote{doré
  (ドレ)
  一般的な色の表現として「黄金色」の意だが、ムニエールの場合、通常は大きい魚についてのみこの表現を用いる。}」と表現する。

「ムニエール」の場合には、焼き上がった魚に少量のレモン汁をふり、塩、こしょう少々で味を整える。粗みじん切りにして湯通ししたパセリを魚の表面に散らし、焦がしバターをかけてすぐに供する。湯通ししたパセリの水分に熱いバターが触れて泡がたつので、それが消えないうちに客に料理を見せるようにする。

%\href{✓原稿下準備なし}{} \href{訳と注釈\%2020180420進行中}{}
\href{未、原文対照チェック}{} \href{未、日本語表現校正}{}
\href{未、注釈チェク}{} \href{未、原稿最終校正}{}

\hypertarget{poissons}{%
\chapter{VI 魚料理 Poissons}\label{poissons}}

\hypertarget{serie-de-courts-bouillons-de-poisson}{%
\section{クールブイヨン}\label{serie-de-courts-bouillons-de-poisson}}

\begin{recette}
\hypertarget{ux30afux30fcux30ebux30d6ux30a4ux30e8ux30f3-a}{%
\subsubsection{クールブイヨン
A}\label{ux30afux30fcux30ebux30d6ux30a4ux30e8ux30f3-a}}

水5 L に対し、ヴィネガー2.5 dL、粗塩60 g、薄切りにしたにんじん600
gと玉ねぎ500 g、タイム1枝、ローリエの小さな葉2枚、パセリの茎100
g、粒こしょう20
g(こしょうを加えるのはクールブイヨンを漉す10分前)。材料を全て鍋に入れ、火にかけて1時間弱火で煮、漉す(原書
p.277)。

\hypertarget{ux30afux30fcux30ebux30d6ux30a4ux30e8ux30f3-b-4-ux9c52ux3046ux306aux304eux30d6ux30edux30b7ux30a7ux7b49-ux539fux66f8p.277}{%
\subsubsection[クールブイヨン B (鱒、うなぎ、ブロシェ等)
原書p.277]{\texorpdfstring{クールブイヨン B \footnote{クールブイヨンは用途に応じ、AからEまでの5種が挙げられている(原書pp.277-278)。}
(鱒、うなぎ、ブロシェ等)
原書p.277}{クールブイヨン B  (鱒、うなぎ、ブロシェ等) 原書p.277}}\label{ux30afux30fcux30ebux30d6ux30a4ux30e8ux30f3-b-4-ux9c52ux3046ux306aux304eux30d6ux30edux30b7ux30a7ux7b49-ux539fux66f8p.277}}

5 L 分の材料\ldots{}\ldots{}白ワイン2.5 L 。水2.5 L
。薄切りにした玉ねぎ600 g。パセリの茎80g
。タイムの小枝1本。ローリエの葉(小) \(\frac{1}{2}\) 枚。粗塩
60g。大粒のこしょう15 g(クールブイヨンを漉す10分前に加える)。

作業手順\ldots{}\ldots{}作業:液体、香味素材、調味料を鍋に入れ、沸かす。弱火で30分程煮て、漉す。

\href{欠落アリ}{}

原注:クールブイヨンBとC\footnote{クールブイヨンBの白ワインを赤ワインに代え、香味素材としてにんじん400gを加える。鱒、鯉、マトロート用(原書pp.277-278)。}で調理した魚はクールブイヨン添えとして供する。つまり、少量の煮汁とクールブイヨンに用いた野菜を添える。野菜はよく火が通っていること。煮汁はしっかり煮詰め、提供直前に新鮮なバター少量を加えて仕上げる。
\end{recette}
\hypertarget{ux30afux30fcux30ebux30d6ux30a4ux30e8ux30f3ux306eux4f7fux3044ux65b9-ux539fux66f8-p.278}{%
\subsection{クールブイヨンの使い方 原書
p.278}\label{ux30afux30fcux30ebux30d6ux30a4ux30e8ux30f3ux306eux4f7fux3044ux65b9-ux539fux66f8-p.278}}

\begin{enumerate}
\def\labelenumi{\arabic{enumi}.}
\item
  加熱時間が30分以内の場合は、クールブイヨンは必ず事前に用意しておくこと。
\item
  加熱時間が30分を越える場合は、クールブイヨンの材料は冷たい状態のままで合わせせておく。香味素材はポワソニエールの網の下に入れる。
\item
  ごく少量のクールブイヨンでポシェ\footnote{原文 pochage à court
    mouillement『ル・ギード・キュリネール』では、この表現はテュルボタン(小型のテュルボ)、バルビュ、舌びらめ等の平たい魚をポシェする際に用いられる。本連載「舌びらめのボヌ・ファム」
    2011年3月号pp.110-111 参照。}する場合、材料は(白または赤ワインを含む場合も)魚を火にかける際に合わせる。クールブイヨンの量は魚の
  \(\frac{1}{3}\)
  の高さとし、加熱中ひんぱんに煮汁を魚にかけてやること\footnote{arroser
    アロゼ。}。この調理法の場合は通常、クールブイヨンは上で記したように
  \footnote{「クールブイヨンB」原注。}、提供直前に軽くバターを加えて仕上げ、魚に添える。
\item
  冷製にする場合は、必ずクールブイヨンに魚が浸った状態で冷ますこと。当然ながら、火にかけている時間は短かくなる\footnote{余熱で火が通るため。}。
\end{enumerate}

\hypertarget{ux539fux6ce8}{%
\subparagraph{【原注】}\label{ux539fux6ce8}}

いくつかの魚種の加熱時間は該当する項で示してある。

\hypertarget{ux9b5aux306eux8abfux7406ux6cd5}{%
\section{魚の調理法}\label{ux9b5aux306eux8abfux7406ux6cd5}}

魚料理は全て、下記のいずれかの調理法による。

\begin{enumerate}
\def\labelenumi{\arabic{enumi}.}
\item
  塩水(湯)またはクールブイヨン\footnote{court-bouillon直訳は「量の少ない煮汁」。魚の他、甲殻類、鶏などの白身肉、野菜などをポシェするのに用いる。とりわけ魚や鶏を丸ごとポシェする場合には、その名称のとおり、できるだけ少量でポシェする必要がある。また、ポシェに用いたクールブイヨンをベースにソースを作る場合が多い。}Bを用いたポシェ\ldots{}\ldots{}大きな魚丸ごと、および切り身。
\item
  ごく少量のクールブイヨンを用いたポシェ\ldots{}\ldots{}魚のフィレ、またはやや小さい魚。
\item
  ブレゼ\ldots{}\ldots{}もっぱら大きな魚。
\item
  オ・ブルー\footnote{比較的小さめの淡水魚に主として用いられる調理法。生きたままの魚の表面のぬめりをとらないように洗い、内臓を取り除いたらすぐに、塩とヴィネガーを加えたクールブイヨンで茹でる。冷製、温製どちらでも供する。原書p.281参照。}\ldots{}\ldots{}とりわけ\ruby{鱒}{ます}、鯉、ブロシェ\footnote{川かますの一種。本連載「ブロシェのクネル」2011年10月号
    pp.124-125 参照。}に合う。
\item
  揚げもの\ldots{}\ldots{}もっぱら小さい魚、切り身。
\item
  ムニエール\ldots{}\ldots{}揚げものにするのと同じ小さい魚、切り身。
\item
  グリエ\ldots{}\ldots{}小さい魚、および切り身。
\item
  グラタン\ldots{}\ldots{}小さい魚、切り身。
\end{enumerate}

\hypertarget{ux5869ux6c34ux6e6fux304aux3088ux3073ux30afux30fcux30ebux30d6ux30a4ux30e8ux30f3bux3092ux7528ux3044ux305fux52a0ux71b1ux8abfux7406}{%
\subsection{塩水(湯)およびクールブイヨンBを用いた加熱調理}\label{ux5869ux6c34ux6e6fux304aux3088ux3073ux30afux30fcux30ebux30d6ux30a4ux30e8ux30f3bux3092ux7528ux3044ux305fux52a0ux71b1ux8abfux7406}}

魚を丸ごと調理する場合は、魚に合ったポワソニエール\footnote{大きな魚を丸ごと煮るための細長い鍋。魚の形を崩さずに取り出せるよう、中に専用の網を敷いて使う。似たものに、舌びらめ等の平たい魚にぴったり合う菱形をしたテュルボティエールがある。いずれも、できるだけ少量の煮汁で魚を加熱できるように工夫されたもの。(図参照)}を用いる。魚を掃除し(テュルボは水にさらして血抜きをし)、ひれ等を切り落して形を整え、ポワソニエールの網に乗せる。魚種に応じて塩水または冷たいクールブイヨンをかぶるまで注ぐ。強火にかけて沸騰したらすぐにレンジの火の弱いところに鍋を移動させ、ポシェする。

切り身(薄すぎは絶対にいけない)の場合、沸騰した液体(塩湯またはクールブイヨン)に投入したらすぐにレンジの火の弱いところに鍋を移動させ、沸騰しない程度の温度でゆっくりと火を通す。

こうするのは、魚の身のエキスを閉じこめるためである。冷水から火にかけた場合にはエキスの大部分が流れ出してしまう。大きな魚丸ごとの場合にはこのやり方はしない。沸騰した液体に魚を投入すると身が収縮するので、大きな魚の場合は身が割れたり形が崩れたりするからだ。

塩湯あるいはクールブイヨンでポシェした魚は、ナフキンまたは専用の網に盛る。周囲をパセリで飾り、塩茹でしたじゃがいもと1種類または数種のソースを添えて供する。ガルニテュールがパセリのみの場合、魚の周囲にパセリを飾るのは客に料理を見せる\footnote{当時の宴席で主流だったロシア式サーヴィスでは、大きな銀盆に盛った料理をまず食客に見せてから、とり分けて給仕する。}直前にすること。どんな場合でも、ガルニテュールを添えたらクロッシュ\footnote{銀または陶製の保温用皿カバー。}は被せないこと。

\hypertarget{ux3054ux304fux5c11ux91cfux306eux30afux30fcux30ebux30d6ux30a4ux30e8ux30f3ux3092ux7528ux3044ux305fux30ddux30b7ux30a7-ux539fux66f8-pp.279-280}{%
\subsection{ごく少量のクールブイヨンを用いたポシェ 原書
pp.279-280}\label{ux3054ux304fux5c11ux91cfux306eux30afux30fcux30ebux30d6ux30a4ux30e8ux30f3ux3092ux7528ux3044ux305fux30ddux30b7ux30a7-ux539fux66f8-pp.279-280}}

この火入れの方法は主としてテュルボタン、バルビュ、舌びらめ、丸ごとおよびそれぞれの魚のフィレで用いる。バターを塗った天板あるいはソテ鍋に魚丸ごとあるいはそのフィレを置き、軽く塩をして、所要量の魚のフュメかマッシュルームの煮汁を注ぐ。フュメとマッシュルームの煮汁を合わせたものを用いる場合もある。蓋をして、中温のオーヴンに入れる。魚丸ごとの場合は時折煮汁をかけてやる。

魚(丸ごとでもフィレでも)に火が通ったら、注意して汁気をきり、皿に盛る。ガルニテュールを含む料理の場合、ガルニテュールを魚の周囲に盛り、ソースをかける\footnote{原文通りの順で訳したが、実際にはソースをかけてからガルニテュールを盛ったほうが良い場合もあるだろう。}。多くの場合、魚の煮汁を煮詰めてソースに加える。

\href{欠落アリ}{}

\hypertarget{ux9b5aux306eux30d6ux30ecux30bc17-ux539fux66f8-p.280}{%
\subsection[魚のブレゼ 原書 p.280]{\texorpdfstring{魚のブレゼ\footnote{ここではブレゼの語が限定的な意味で用いられていることに注意。牛のアロワイヨのような大きな塊肉のブレゼと同様の調理法、ということである。それは、香味素材を色づくまで炒めてから用いることや、主素材に豚背脂やトリュフをピケ針で差したり、豚背脂のシートで覆って加熱するという点によく表れている。ただし、これらは必須というわけではないため、事実上は「ごく少量のクールブイヨンを用いたポシェ」と区別がつきにくい。実際、モンタニェ『ラルース・ガストロノミーク』初版では、魚のブレゼについて「本来的な意味でのブレゼというよりは、ごく少量のクールブイヨンを用いたポシェである」と述べられている。逆に言えば、こんにち魚の加熱方法についてしばしば「ブレゼ」と呼ばれているものが、エスコフィエやモンタニェにおいては「少量のクールブイヨンを用いたポシェ」と表現されていたということである。}
原書
p.280}{魚のブレゼ 原書 p.280}}\label{ux9b5aux306eux30d6ux30ecux30bc17-ux539fux66f8-p.280}}

この調理法を用いるのは通常、丸ごとまたは筒切りにした鮭、大ぶりの鱒、テュルボ、テュルボタンのうち大きなもの、である。

場合によっては、魚の片面に、小さく切った豚背脂、トリュフ、コルニション、にんじん等をピケ針で差し込む。

香味素材等\footnote{原文 fonds de braisage フォン・ド・ブレザージュ
  (fonds de
  braiseフォン・ド・ブレーズ、とも)。通常は、厚い輪切りにしたにんじんと玉ねぎをバターか獣脂で色づくまで炒め、ブーケガルニ、下茹でした豚皮を合わせる(原書pp.394-395)。また、これを用いた煮汁のことも指す。}は肉料理のブレゼの場合と同じように用意するが、豚皮は用いない。提供方法に応じて、白または赤ワインと軽い魚のフュメ同量ずつを、魚の厚みの
\(\frac{3}{4}\) またはひたひたの高さまで注ぐ。厳密に肉断ち \footnote{カトリックの生活習慣として、四旬節(復活祭までの46日間)および週1
  回程度、肉類を食べないということが行なわれた(本連載2012年5月号「ソース・エスパニョル(4)」p.110、訳注1参照)。}のための仕立てにする場合を除いて、薄くスライスした豚背脂のシートを魚にかぶせる。加熱中\footnote{鍋を火にかけ、沸騰したら蓋をして中火のオーヴンに入れ、加熱する。}こまめに煮汁を魚にかけてやる\footnote{arroser
  アロゼ。}。また、完全には蓋をせず、加熱中に煮汁が煮詰まるようにしてやる。

ほぼ火が通ったら、鍋の蓋をとり、魚にかけた煮汁の水分をオーヴンの熱で蒸発させて表面につやを出す\footnote{glacer
  グラセ。}。魚を鍋から出して汁気をきり、皿に盛り保温しておく。

煮汁\footnote{原文 fonds de braisage (訳注2参照)。}を漉し、しばらく休ませたら浮き脂を取り除き、必要なら煮詰める。これを加えてソースを仕上げる。

魚のブレゼには通常、各ルセットに示してあるガルニテュールを添える。

\hypertarget{ux30aaux30d6ux30ebux30fc36-ux539fux66f8-pp.280-281}{%
\subsection[オ・ブルー 原書
pp.280-281]{\texorpdfstring{オ・ブルー\footnote{au bleu
  ヴィネガーを加えることで魚の表面のぬめりが青みがかることから。} 原書
pp.280-281}{オ・ブルー 原書 pp.280-281}}\label{ux30aaux30d6ux30ebux30fc36-ux539fux66f8-pp.280-281}}

オ・ブルーは鱒、鯉、ブロシェ\footnote{川かますの一種。}のみに用いられる特殊な調理法で、基本的なポイントは以下のとおり。

\begin{enumerate}
\def\labelenumi{\arabic{enumi}.}
\item
  必ず、生きた魚を使う。
\item
  魚の表面のぬめりをとらないように、なるべく手で触れずに、わたを抜く。鱗も引かない。
\item
  魚が大きい場合は、専用の網を敷いたポワソニーエルに入れ、「沸騰したヴィネガーをかける」。ヴィネガーは、通常のクールブイヨンに加える分量\footnote{以下の「クールブイヨンA」の分量比率を参照。}。次に、ヴィネガーを入れずに用意した温かい\footnote{原文
    tiède ぬるい、温かい。}クールブイヨンを注ぎ入れる。これは、なるべく身が割れないようにするためである。その後は通常どおり加熱する\footnote{レンジで沸騰させたらオーヴンに入れる。}。
\item
  小さい鱒の場合は、生きたままのものを手早く中抜きし、塩、ヴィネガーを加えただけの沸騰したクールブイヨンで煮る。
\item
  オ・ブルーは冷製、温製どちらの仕立てにしてもいい。実際の作り方の項で示してあるソースを添えて供する。
\end{enumerate}

\href{欠落アリ}{}

\hypertarget{ux30e0ux30cbux30a8ux30fcux30eb43-ux539fux66f8-p.282}{%
\subsection[ムニエール 原書 p.282]{\texorpdfstring{ムニエール\footnote{à
  la meunière 「粉挽き職人風」の意。} 原書
p.282}{ムニエール 原書 p.282}}\label{ux30e0ux30cbux30a8ux30fcux30eb43-ux539fux66f8-p.282}}

ムニエールは素晴らしい調理法だが、小型の魚と、大きな魚の場合は切り身にしか用いない。とはいえ、丁寧にやれば1.5
kg以下のテュルボタンはムニエールで調理できる。

魚丸ごと、あるいは切り身、フィレに味つけをして小麦粉をまぶし、バターを熱したフライパンで焼く。

魚が小さい場合は普通のバターでいいが、大きい場合は澄ましバターを使った方がいい。

魚の両面を焼き、程良く火が通ったら、予め熱しておいた皿に盛る。

飾り切りにした半割りのレモンを添えて、そのまま供することも可能である。ただし、このような提供方法の場合は本来の「ムニエール」と区別するために「黄金色に焼いた\footnote{doré
  (ドレ)
  一般的な色の表現として「黄金色」の意だが、ムニエールの場合、通常は大きい魚についてのみこの表現を用いる。}」と表現する。

「ムニエール」の場合には、焼き上がった魚に少量のレモン汁をふり、塩、こしょう少々で味を整える。粗みじん切りにして湯通ししたパセリを魚の表面に散らし、焦がしバターをかけてすぐに供する。湯通ししたパセリの水分に熱いバターが触れて泡がたつので、それが消えないうちに客に料理を見せるようにする。



%%% Chapitre V. OEufs
%%%% エスコフィエ『料理の手引き』全注解
% 五島 学

\href{✓原稿下準備なし}{} \href{訳と注釈\%2020180420進行中}{}
\href{未、原文対照チェック}{} \href{未、日本語表現校正}{}
\href{未、注釈チェク}{} \href{未、原稿最終校正}{}

\hypertarget{poissons}{%
\chapter{VI 魚料理 Poissons}\label{poissons}}

\hypertarget{serie-de-courts-bouillons-de-poisson}{%
\section{クールブイヨン}\label{serie-de-courts-bouillons-de-poisson}}

\begin{recette}
\hypertarget{ux30afux30fcux30ebux30d6ux30a4ux30e8ux30f3-a}{%
\subsubsection{クールブイヨン
A}\label{ux30afux30fcux30ebux30d6ux30a4ux30e8ux30f3-a}}

水5 L に対し、ヴィネガー2.5 dL、粗塩60 g、薄切りにしたにんじん600
gと玉ねぎ500 g、タイム1枝、ローリエの小さな葉2枚、パセリの茎100
g、粒こしょう20
g(こしょうを加えるのはクールブイヨンを漉す10分前)。材料を全て鍋に入れ、火にかけて1時間弱火で煮、漉す(原書
p.277)。

\hypertarget{ux30afux30fcux30ebux30d6ux30a4ux30e8ux30f3-b-4-ux9c52ux3046ux306aux304eux30d6ux30edux30b7ux30a7ux7b49-ux539fux66f8p.277}{%
\subsubsection[クールブイヨン B (鱒、うなぎ、ブロシェ等)
原書p.277]{\texorpdfstring{クールブイヨン B \footnote{クールブイヨンは用途に応じ、AからEまでの5種が挙げられている(原書pp.277-278)。}
(鱒、うなぎ、ブロシェ等)
原書p.277}{クールブイヨン B  (鱒、うなぎ、ブロシェ等) 原書p.277}}\label{ux30afux30fcux30ebux30d6ux30a4ux30e8ux30f3-b-4-ux9c52ux3046ux306aux304eux30d6ux30edux30b7ux30a7ux7b49-ux539fux66f8p.277}}

5 L 分の材料\ldots{}\ldots{}白ワイン2.5 L 。水2.5 L
。薄切りにした玉ねぎ600 g。パセリの茎80g
。タイムの小枝1本。ローリエの葉(小) \(\frac{1}{2}\) 枚。粗塩
60g。大粒のこしょう15 g(クールブイヨンを漉す10分前に加える)。

作業手順\ldots{}\ldots{}作業:液体、香味素材、調味料を鍋に入れ、沸かす。弱火で30分程煮て、漉す。

\href{欠落アリ}{}

原注:クールブイヨンBとC\footnote{クールブイヨンBの白ワインを赤ワインに代え、香味素材としてにんじん400gを加える。鱒、鯉、マトロート用(原書pp.277-278)。}で調理した魚はクールブイヨン添えとして供する。つまり、少量の煮汁とクールブイヨンに用いた野菜を添える。野菜はよく火が通っていること。煮汁はしっかり煮詰め、提供直前に新鮮なバター少量を加えて仕上げる。
\end{recette}
\hypertarget{ux30afux30fcux30ebux30d6ux30a4ux30e8ux30f3ux306eux4f7fux3044ux65b9-ux539fux66f8-p.278}{%
\subsection{クールブイヨンの使い方 原書
p.278}\label{ux30afux30fcux30ebux30d6ux30a4ux30e8ux30f3ux306eux4f7fux3044ux65b9-ux539fux66f8-p.278}}

\begin{enumerate}
\def\labelenumi{\arabic{enumi}.}
\item
  加熱時間が30分以内の場合は、クールブイヨンは必ず事前に用意しておくこと。
\item
  加熱時間が30分を越える場合は、クールブイヨンの材料は冷たい状態のままで合わせせておく。香味素材はポワソニエールの網の下に入れる。
\item
  ごく少量のクールブイヨンでポシェ\footnote{原文 pochage à court
    mouillement『ル・ギード・キュリネール』では、この表現はテュルボタン(小型のテュルボ)、バルビュ、舌びらめ等の平たい魚をポシェする際に用いられる。本連載「舌びらめのボヌ・ファム」
    2011年3月号pp.110-111 参照。}する場合、材料は(白または赤ワインを含む場合も)魚を火にかける際に合わせる。クールブイヨンの量は魚の
  \(\frac{1}{3}\)
  の高さとし、加熱中ひんぱんに煮汁を魚にかけてやること\footnote{arroser
    アロゼ。}。この調理法の場合は通常、クールブイヨンは上で記したように
  \footnote{「クールブイヨンB」原注。}、提供直前に軽くバターを加えて仕上げ、魚に添える。
\item
  冷製にする場合は、必ずクールブイヨンに魚が浸った状態で冷ますこと。当然ながら、火にかけている時間は短かくなる\footnote{余熱で火が通るため。}。
\end{enumerate}

\hypertarget{ux539fux6ce8}{%
\subparagraph{【原注】}\label{ux539fux6ce8}}

いくつかの魚種の加熱時間は該当する項で示してある。

\hypertarget{ux9b5aux306eux8abfux7406ux6cd5}{%
\section{魚の調理法}\label{ux9b5aux306eux8abfux7406ux6cd5}}

魚料理は全て、下記のいずれかの調理法による。

\begin{enumerate}
\def\labelenumi{\arabic{enumi}.}
\item
  塩水(湯)またはクールブイヨン\footnote{court-bouillon直訳は「量の少ない煮汁」。魚の他、甲殻類、鶏などの白身肉、野菜などをポシェするのに用いる。とりわけ魚や鶏を丸ごとポシェする場合には、その名称のとおり、できるだけ少量でポシェする必要がある。また、ポシェに用いたクールブイヨンをベースにソースを作る場合が多い。}Bを用いたポシェ\ldots{}\ldots{}大きな魚丸ごと、および切り身。
\item
  ごく少量のクールブイヨンを用いたポシェ\ldots{}\ldots{}魚のフィレ、またはやや小さい魚。
\item
  ブレゼ\ldots{}\ldots{}もっぱら大きな魚。
\item
  オ・ブルー\footnote{比較的小さめの淡水魚に主として用いられる調理法。生きたままの魚の表面のぬめりをとらないように洗い、内臓を取り除いたらすぐに、塩とヴィネガーを加えたクールブイヨンで茹でる。冷製、温製どちらでも供する。原書p.281参照。}\ldots{}\ldots{}とりわけ\ruby{鱒}{ます}、鯉、ブロシェ\footnote{川かますの一種。本連載「ブロシェのクネル」2011年10月号
    pp.124-125 参照。}に合う。
\item
  揚げもの\ldots{}\ldots{}もっぱら小さい魚、切り身。
\item
  ムニエール\ldots{}\ldots{}揚げものにするのと同じ小さい魚、切り身。
\item
  グリエ\ldots{}\ldots{}小さい魚、および切り身。
\item
  グラタン\ldots{}\ldots{}小さい魚、切り身。
\end{enumerate}

\hypertarget{ux5869ux6c34ux6e6fux304aux3088ux3073ux30afux30fcux30ebux30d6ux30a4ux30e8ux30f3bux3092ux7528ux3044ux305fux52a0ux71b1ux8abfux7406}{%
\subsection{塩水(湯)およびクールブイヨンBを用いた加熱調理}\label{ux5869ux6c34ux6e6fux304aux3088ux3073ux30afux30fcux30ebux30d6ux30a4ux30e8ux30f3bux3092ux7528ux3044ux305fux52a0ux71b1ux8abfux7406}}

魚を丸ごと調理する場合は、魚に合ったポワソニエール\footnote{大きな魚を丸ごと煮るための細長い鍋。魚の形を崩さずに取り出せるよう、中に専用の網を敷いて使う。似たものに、舌びらめ等の平たい魚にぴったり合う菱形をしたテュルボティエールがある。いずれも、できるだけ少量の煮汁で魚を加熱できるように工夫されたもの。(図参照)}を用いる。魚を掃除し(テュルボは水にさらして血抜きをし)、ひれ等を切り落して形を整え、ポワソニエールの網に乗せる。魚種に応じて塩水または冷たいクールブイヨンをかぶるまで注ぐ。強火にかけて沸騰したらすぐにレンジの火の弱いところに鍋を移動させ、ポシェする。

切り身(薄すぎは絶対にいけない)の場合、沸騰した液体(塩湯またはクールブイヨン)に投入したらすぐにレンジの火の弱いところに鍋を移動させ、沸騰しない程度の温度でゆっくりと火を通す。

こうするのは、魚の身のエキスを閉じこめるためである。冷水から火にかけた場合にはエキスの大部分が流れ出してしまう。大きな魚丸ごとの場合にはこのやり方はしない。沸騰した液体に魚を投入すると身が収縮するので、大きな魚の場合は身が割れたり形が崩れたりするからだ。

塩湯あるいはクールブイヨンでポシェした魚は、ナフキンまたは専用の網に盛る。周囲をパセリで飾り、塩茹でしたじゃがいもと1種類または数種のソースを添えて供する。ガルニテュールがパセリのみの場合、魚の周囲にパセリを飾るのは客に料理を見せる\footnote{当時の宴席で主流だったロシア式サーヴィスでは、大きな銀盆に盛った料理をまず食客に見せてから、とり分けて給仕する。}直前にすること。どんな場合でも、ガルニテュールを添えたらクロッシュ\footnote{銀または陶製の保温用皿カバー。}は被せないこと。

\hypertarget{ux3054ux304fux5c11ux91cfux306eux30afux30fcux30ebux30d6ux30a4ux30e8ux30f3ux3092ux7528ux3044ux305fux30ddux30b7ux30a7-ux539fux66f8-pp.279-280}{%
\subsection{ごく少量のクールブイヨンを用いたポシェ 原書
pp.279-280}\label{ux3054ux304fux5c11ux91cfux306eux30afux30fcux30ebux30d6ux30a4ux30e8ux30f3ux3092ux7528ux3044ux305fux30ddux30b7ux30a7-ux539fux66f8-pp.279-280}}

この火入れの方法は主としてテュルボタン、バルビュ、舌びらめ、丸ごとおよびそれぞれの魚のフィレで用いる。バターを塗った天板あるいはソテ鍋に魚丸ごとあるいはそのフィレを置き、軽く塩をして、所要量の魚のフュメかマッシュルームの煮汁を注ぐ。フュメとマッシュルームの煮汁を合わせたものを用いる場合もある。蓋をして、中温のオーヴンに入れる。魚丸ごとの場合は時折煮汁をかけてやる。

魚(丸ごとでもフィレでも)に火が通ったら、注意して汁気をきり、皿に盛る。ガルニテュールを含む料理の場合、ガルニテュールを魚の周囲に盛り、ソースをかける\footnote{原文通りの順で訳したが、実際にはソースをかけてからガルニテュールを盛ったほうが良い場合もあるだろう。}。多くの場合、魚の煮汁を煮詰めてソースに加える。

\href{欠落アリ}{}

\hypertarget{ux9b5aux306eux30d6ux30ecux30bc17-ux539fux66f8-p.280}{%
\subsection[魚のブレゼ 原書 p.280]{\texorpdfstring{魚のブレゼ\footnote{ここではブレゼの語が限定的な意味で用いられていることに注意。牛のアロワイヨのような大きな塊肉のブレゼと同様の調理法、ということである。それは、香味素材を色づくまで炒めてから用いることや、主素材に豚背脂やトリュフをピケ針で差したり、豚背脂のシートで覆って加熱するという点によく表れている。ただし、これらは必須というわけではないため、事実上は「ごく少量のクールブイヨンを用いたポシェ」と区別がつきにくい。実際、モンタニェ『ラルース・ガストロノミーク』初版では、魚のブレゼについて「本来的な意味でのブレゼというよりは、ごく少量のクールブイヨンを用いたポシェである」と述べられている。逆に言えば、こんにち魚の加熱方法についてしばしば「ブレゼ」と呼ばれているものが、エスコフィエやモンタニェにおいては「少量のクールブイヨンを用いたポシェ」と表現されていたということである。}
原書
p.280}{魚のブレゼ 原書 p.280}}\label{ux9b5aux306eux30d6ux30ecux30bc17-ux539fux66f8-p.280}}

この調理法を用いるのは通常、丸ごとまたは筒切りにした鮭、大ぶりの鱒、テュルボ、テュルボタンのうち大きなもの、である。

場合によっては、魚の片面に、小さく切った豚背脂、トリュフ、コルニション、にんじん等をピケ針で差し込む。

香味素材等\footnote{原文 fonds de braisage フォン・ド・ブレザージュ
  (fonds de
  braiseフォン・ド・ブレーズ、とも)。通常は、厚い輪切りにしたにんじんと玉ねぎをバターか獣脂で色づくまで炒め、ブーケガルニ、下茹でした豚皮を合わせる(原書pp.394-395)。また、これを用いた煮汁のことも指す。}は肉料理のブレゼの場合と同じように用意するが、豚皮は用いない。提供方法に応じて、白または赤ワインと軽い魚のフュメ同量ずつを、魚の厚みの
\(\frac{3}{4}\) またはひたひたの高さまで注ぐ。厳密に肉断ち \footnote{カトリックの生活習慣として、四旬節(復活祭までの46日間)および週1
  回程度、肉類を食べないということが行なわれた(本連載2012年5月号「ソース・エスパニョル(4)」p.110、訳注1参照)。}のための仕立てにする場合を除いて、薄くスライスした豚背脂のシートを魚にかぶせる。加熱中\footnote{鍋を火にかけ、沸騰したら蓋をして中火のオーヴンに入れ、加熱する。}こまめに煮汁を魚にかけてやる\footnote{arroser
  アロゼ。}。また、完全には蓋をせず、加熱中に煮汁が煮詰まるようにしてやる。

ほぼ火が通ったら、鍋の蓋をとり、魚にかけた煮汁の水分をオーヴンの熱で蒸発させて表面につやを出す\footnote{glacer
  グラセ。}。魚を鍋から出して汁気をきり、皿に盛り保温しておく。

煮汁\footnote{原文 fonds de braisage (訳注2参照)。}を漉し、しばらく休ませたら浮き脂を取り除き、必要なら煮詰める。これを加えてソースを仕上げる。

魚のブレゼには通常、各ルセットに示してあるガルニテュールを添える。

\hypertarget{ux30aaux30d6ux30ebux30fc36-ux539fux66f8-pp.280-281}{%
\subsection[オ・ブルー 原書
pp.280-281]{\texorpdfstring{オ・ブルー\footnote{au bleu
  ヴィネガーを加えることで魚の表面のぬめりが青みがかることから。} 原書
pp.280-281}{オ・ブルー 原書 pp.280-281}}\label{ux30aaux30d6ux30ebux30fc36-ux539fux66f8-pp.280-281}}

オ・ブルーは鱒、鯉、ブロシェ\footnote{川かますの一種。}のみに用いられる特殊な調理法で、基本的なポイントは以下のとおり。

\begin{enumerate}
\def\labelenumi{\arabic{enumi}.}
\item
  必ず、生きた魚を使う。
\item
  魚の表面のぬめりをとらないように、なるべく手で触れずに、わたを抜く。鱗も引かない。
\item
  魚が大きい場合は、専用の網を敷いたポワソニーエルに入れ、「沸騰したヴィネガーをかける」。ヴィネガーは、通常のクールブイヨンに加える分量\footnote{以下の「クールブイヨンA」の分量比率を参照。}。次に、ヴィネガーを入れずに用意した温かい\footnote{原文
    tiède ぬるい、温かい。}クールブイヨンを注ぎ入れる。これは、なるべく身が割れないようにするためである。その後は通常どおり加熱する\footnote{レンジで沸騰させたらオーヴンに入れる。}。
\item
  小さい鱒の場合は、生きたままのものを手早く中抜きし、塩、ヴィネガーを加えただけの沸騰したクールブイヨンで煮る。
\item
  オ・ブルーは冷製、温製どちらの仕立てにしてもいい。実際の作り方の項で示してあるソースを添えて供する。
\end{enumerate}

\href{欠落アリ}{}

\hypertarget{ux30e0ux30cbux30a8ux30fcux30eb43-ux539fux66f8-p.282}{%
\subsection[ムニエール 原書 p.282]{\texorpdfstring{ムニエール\footnote{à
  la meunière 「粉挽き職人風」の意。} 原書
p.282}{ムニエール 原書 p.282}}\label{ux30e0ux30cbux30a8ux30fcux30eb43-ux539fux66f8-p.282}}

ムニエールは素晴らしい調理法だが、小型の魚と、大きな魚の場合は切り身にしか用いない。とはいえ、丁寧にやれば1.5
kg以下のテュルボタンはムニエールで調理できる。

魚丸ごと、あるいは切り身、フィレに味つけをして小麦粉をまぶし、バターを熱したフライパンで焼く。

魚が小さい場合は普通のバターでいいが、大きい場合は澄ましバターを使った方がいい。

魚の両面を焼き、程良く火が通ったら、予め熱しておいた皿に盛る。

飾り切りにした半割りのレモンを添えて、そのまま供することも可能である。ただし、このような提供方法の場合は本来の「ムニエール」と区別するために「黄金色に焼いた\footnote{doré
  (ドレ)
  一般的な色の表現として「黄金色」の意だが、ムニエールの場合、通常は大きい魚についてのみこの表現を用いる。}」と表現する。

「ムニエール」の場合には、焼き上がった魚に少量のレモン汁をふり、塩、こしょう少々で味を整える。粗みじん切りにして湯通ししたパセリを魚の表面に散らし、焦がしバターをかけてすぐに供する。湯通ししたパセリの水分に熱いバターが触れて泡がたつので、それが消えないうちに客に料理を見せるようにする。

%\input{06-poissons/06-02-p-284}
%\input{06-poissons/06-03-pp285-286}
%\input{06-poissons/06-04-p286}
%\input{06-poissons/06-05-pp286-290}

%\href{✓原稿下準備なし}{} \href{訳と注釈\%2020180420進行中}{}
\href{未、原文対照チェック}{} \href{未、日本語表現校正}{}
\href{未、注釈チェク}{} \href{未、原稿最終校正}{}

\hypertarget{poissons}{%
\chapter{VI 魚料理 Poissons}\label{poissons}}

\hypertarget{serie-de-courts-bouillons-de-poisson}{%
\section{クールブイヨン}\label{serie-de-courts-bouillons-de-poisson}}

\begin{recette}
\hypertarget{ux30afux30fcux30ebux30d6ux30a4ux30e8ux30f3-a}{%
\subsubsection{クールブイヨン
A}\label{ux30afux30fcux30ebux30d6ux30a4ux30e8ux30f3-a}}

水5 L に対し、ヴィネガー2.5 dL、粗塩60 g、薄切りにしたにんじん600
gと玉ねぎ500 g、タイム1枝、ローリエの小さな葉2枚、パセリの茎100
g、粒こしょう20
g(こしょうを加えるのはクールブイヨンを漉す10分前)。材料を全て鍋に入れ、火にかけて1時間弱火で煮、漉す(原書
p.277)。

\hypertarget{ux30afux30fcux30ebux30d6ux30a4ux30e8ux30f3-b-4-ux9c52ux3046ux306aux304eux30d6ux30edux30b7ux30a7ux7b49-ux539fux66f8p.277}{%
\subsubsection[クールブイヨン B (鱒、うなぎ、ブロシェ等)
原書p.277]{\texorpdfstring{クールブイヨン B \footnote{クールブイヨンは用途に応じ、AからEまでの5種が挙げられている(原書pp.277-278)。}
(鱒、うなぎ、ブロシェ等)
原書p.277}{クールブイヨン B  (鱒、うなぎ、ブロシェ等) 原書p.277}}\label{ux30afux30fcux30ebux30d6ux30a4ux30e8ux30f3-b-4-ux9c52ux3046ux306aux304eux30d6ux30edux30b7ux30a7ux7b49-ux539fux66f8p.277}}

5 L 分の材料\ldots{}\ldots{}白ワイン2.5 L 。水2.5 L
。薄切りにした玉ねぎ600 g。パセリの茎80g
。タイムの小枝1本。ローリエの葉(小) \(\frac{1}{2}\) 枚。粗塩
60g。大粒のこしょう15 g(クールブイヨンを漉す10分前に加える)。

作業手順\ldots{}\ldots{}作業:液体、香味素材、調味料を鍋に入れ、沸かす。弱火で30分程煮て、漉す。

\href{欠落アリ}{}

原注:クールブイヨンBとC\footnote{クールブイヨンBの白ワインを赤ワインに代え、香味素材としてにんじん400gを加える。鱒、鯉、マトロート用(原書pp.277-278)。}で調理した魚はクールブイヨン添えとして供する。つまり、少量の煮汁とクールブイヨンに用いた野菜を添える。野菜はよく火が通っていること。煮汁はしっかり煮詰め、提供直前に新鮮なバター少量を加えて仕上げる。
\end{recette}
\hypertarget{ux30afux30fcux30ebux30d6ux30a4ux30e8ux30f3ux306eux4f7fux3044ux65b9-ux539fux66f8-p.278}{%
\subsection{クールブイヨンの使い方 原書
p.278}\label{ux30afux30fcux30ebux30d6ux30a4ux30e8ux30f3ux306eux4f7fux3044ux65b9-ux539fux66f8-p.278}}

\begin{enumerate}
\def\labelenumi{\arabic{enumi}.}
\item
  加熱時間が30分以内の場合は、クールブイヨンは必ず事前に用意しておくこと。
\item
  加熱時間が30分を越える場合は、クールブイヨンの材料は冷たい状態のままで合わせせておく。香味素材はポワソニエールの網の下に入れる。
\item
  ごく少量のクールブイヨンでポシェ\footnote{原文 pochage à court
    mouillement『ル・ギード・キュリネール』では、この表現はテュルボタン(小型のテュルボ)、バルビュ、舌びらめ等の平たい魚をポシェする際に用いられる。本連載「舌びらめのボヌ・ファム」
    2011年3月号pp.110-111 参照。}する場合、材料は(白または赤ワインを含む場合も)魚を火にかける際に合わせる。クールブイヨンの量は魚の
  \(\frac{1}{3}\)
  の高さとし、加熱中ひんぱんに煮汁を魚にかけてやること\footnote{arroser
    アロゼ。}。この調理法の場合は通常、クールブイヨンは上で記したように
  \footnote{「クールブイヨンB」原注。}、提供直前に軽くバターを加えて仕上げ、魚に添える。
\item
  冷製にする場合は、必ずクールブイヨンに魚が浸った状態で冷ますこと。当然ながら、火にかけている時間は短かくなる\footnote{余熱で火が通るため。}。
\end{enumerate}

\hypertarget{ux539fux6ce8}{%
\subparagraph{【原注】}\label{ux539fux6ce8}}

いくつかの魚種の加熱時間は該当する項で示してある。

\hypertarget{ux9b5aux306eux8abfux7406ux6cd5}{%
\section{魚の調理法}\label{ux9b5aux306eux8abfux7406ux6cd5}}

魚料理は全て、下記のいずれかの調理法による。

\begin{enumerate}
\def\labelenumi{\arabic{enumi}.}
\item
  塩水(湯)またはクールブイヨン\footnote{court-bouillon直訳は「量の少ない煮汁」。魚の他、甲殻類、鶏などの白身肉、野菜などをポシェするのに用いる。とりわけ魚や鶏を丸ごとポシェする場合には、その名称のとおり、できるだけ少量でポシェする必要がある。また、ポシェに用いたクールブイヨンをベースにソースを作る場合が多い。}Bを用いたポシェ\ldots{}\ldots{}大きな魚丸ごと、および切り身。
\item
  ごく少量のクールブイヨンを用いたポシェ\ldots{}\ldots{}魚のフィレ、またはやや小さい魚。
\item
  ブレゼ\ldots{}\ldots{}もっぱら大きな魚。
\item
  オ・ブルー\footnote{比較的小さめの淡水魚に主として用いられる調理法。生きたままの魚の表面のぬめりをとらないように洗い、内臓を取り除いたらすぐに、塩とヴィネガーを加えたクールブイヨンで茹でる。冷製、温製どちらでも供する。原書p.281参照。}\ldots{}\ldots{}とりわけ\ruby{鱒}{ます}、鯉、ブロシェ\footnote{川かますの一種。本連載「ブロシェのクネル」2011年10月号
    pp.124-125 参照。}に合う。
\item
  揚げもの\ldots{}\ldots{}もっぱら小さい魚、切り身。
\item
  ムニエール\ldots{}\ldots{}揚げものにするのと同じ小さい魚、切り身。
\item
  グリエ\ldots{}\ldots{}小さい魚、および切り身。
\item
  グラタン\ldots{}\ldots{}小さい魚、切り身。
\end{enumerate}

\hypertarget{ux5869ux6c34ux6e6fux304aux3088ux3073ux30afux30fcux30ebux30d6ux30a4ux30e8ux30f3bux3092ux7528ux3044ux305fux52a0ux71b1ux8abfux7406}{%
\subsection{塩水(湯)およびクールブイヨンBを用いた加熱調理}\label{ux5869ux6c34ux6e6fux304aux3088ux3073ux30afux30fcux30ebux30d6ux30a4ux30e8ux30f3bux3092ux7528ux3044ux305fux52a0ux71b1ux8abfux7406}}

魚を丸ごと調理する場合は、魚に合ったポワソニエール\footnote{大きな魚を丸ごと煮るための細長い鍋。魚の形を崩さずに取り出せるよう、中に専用の網を敷いて使う。似たものに、舌びらめ等の平たい魚にぴったり合う菱形をしたテュルボティエールがある。いずれも、できるだけ少量の煮汁で魚を加熱できるように工夫されたもの。(図参照)}を用いる。魚を掃除し(テュルボは水にさらして血抜きをし)、ひれ等を切り落して形を整え、ポワソニエールの網に乗せる。魚種に応じて塩水または冷たいクールブイヨンをかぶるまで注ぐ。強火にかけて沸騰したらすぐにレンジの火の弱いところに鍋を移動させ、ポシェする。

切り身(薄すぎは絶対にいけない)の場合、沸騰した液体(塩湯またはクールブイヨン)に投入したらすぐにレンジの火の弱いところに鍋を移動させ、沸騰しない程度の温度でゆっくりと火を通す。

こうするのは、魚の身のエキスを閉じこめるためである。冷水から火にかけた場合にはエキスの大部分が流れ出してしまう。大きな魚丸ごとの場合にはこのやり方はしない。沸騰した液体に魚を投入すると身が収縮するので、大きな魚の場合は身が割れたり形が崩れたりするからだ。

塩湯あるいはクールブイヨンでポシェした魚は、ナフキンまたは専用の網に盛る。周囲をパセリで飾り、塩茹でしたじゃがいもと1種類または数種のソースを添えて供する。ガルニテュールがパセリのみの場合、魚の周囲にパセリを飾るのは客に料理を見せる\footnote{当時の宴席で主流だったロシア式サーヴィスでは、大きな銀盆に盛った料理をまず食客に見せてから、とり分けて給仕する。}直前にすること。どんな場合でも、ガルニテュールを添えたらクロッシュ\footnote{銀または陶製の保温用皿カバー。}は被せないこと。

\hypertarget{ux3054ux304fux5c11ux91cfux306eux30afux30fcux30ebux30d6ux30a4ux30e8ux30f3ux3092ux7528ux3044ux305fux30ddux30b7ux30a7-ux539fux66f8-pp.279-280}{%
\subsection{ごく少量のクールブイヨンを用いたポシェ 原書
pp.279-280}\label{ux3054ux304fux5c11ux91cfux306eux30afux30fcux30ebux30d6ux30a4ux30e8ux30f3ux3092ux7528ux3044ux305fux30ddux30b7ux30a7-ux539fux66f8-pp.279-280}}

この火入れの方法は主としてテュルボタン、バルビュ、舌びらめ、丸ごとおよびそれぞれの魚のフィレで用いる。バターを塗った天板あるいはソテ鍋に魚丸ごとあるいはそのフィレを置き、軽く塩をして、所要量の魚のフュメかマッシュルームの煮汁を注ぐ。フュメとマッシュルームの煮汁を合わせたものを用いる場合もある。蓋をして、中温のオーヴンに入れる。魚丸ごとの場合は時折煮汁をかけてやる。

魚(丸ごとでもフィレでも)に火が通ったら、注意して汁気をきり、皿に盛る。ガルニテュールを含む料理の場合、ガルニテュールを魚の周囲に盛り、ソースをかける\footnote{原文通りの順で訳したが、実際にはソースをかけてからガルニテュールを盛ったほうが良い場合もあるだろう。}。多くの場合、魚の煮汁を煮詰めてソースに加える。

\href{欠落アリ}{}

\hypertarget{ux9b5aux306eux30d6ux30ecux30bc17-ux539fux66f8-p.280}{%
\subsection[魚のブレゼ 原書 p.280]{\texorpdfstring{魚のブレゼ\footnote{ここではブレゼの語が限定的な意味で用いられていることに注意。牛のアロワイヨのような大きな塊肉のブレゼと同様の調理法、ということである。それは、香味素材を色づくまで炒めてから用いることや、主素材に豚背脂やトリュフをピケ針で差したり、豚背脂のシートで覆って加熱するという点によく表れている。ただし、これらは必須というわけではないため、事実上は「ごく少量のクールブイヨンを用いたポシェ」と区別がつきにくい。実際、モンタニェ『ラルース・ガストロノミーク』初版では、魚のブレゼについて「本来的な意味でのブレゼというよりは、ごく少量のクールブイヨンを用いたポシェである」と述べられている。逆に言えば、こんにち魚の加熱方法についてしばしば「ブレゼ」と呼ばれているものが、エスコフィエやモンタニェにおいては「少量のクールブイヨンを用いたポシェ」と表現されていたということである。}
原書
p.280}{魚のブレゼ 原書 p.280}}\label{ux9b5aux306eux30d6ux30ecux30bc17-ux539fux66f8-p.280}}

この調理法を用いるのは通常、丸ごとまたは筒切りにした鮭、大ぶりの鱒、テュルボ、テュルボタンのうち大きなもの、である。

場合によっては、魚の片面に、小さく切った豚背脂、トリュフ、コルニション、にんじん等をピケ針で差し込む。

香味素材等\footnote{原文 fonds de braisage フォン・ド・ブレザージュ
  (fonds de
  braiseフォン・ド・ブレーズ、とも)。通常は、厚い輪切りにしたにんじんと玉ねぎをバターか獣脂で色づくまで炒め、ブーケガルニ、下茹でした豚皮を合わせる(原書pp.394-395)。また、これを用いた煮汁のことも指す。}は肉料理のブレゼの場合と同じように用意するが、豚皮は用いない。提供方法に応じて、白または赤ワインと軽い魚のフュメ同量ずつを、魚の厚みの
\(\frac{3}{4}\) またはひたひたの高さまで注ぐ。厳密に肉断ち \footnote{カトリックの生活習慣として、四旬節(復活祭までの46日間)および週1
  回程度、肉類を食べないということが行なわれた(本連載2012年5月号「ソース・エスパニョル(4)」p.110、訳注1参照)。}のための仕立てにする場合を除いて、薄くスライスした豚背脂のシートを魚にかぶせる。加熱中\footnote{鍋を火にかけ、沸騰したら蓋をして中火のオーヴンに入れ、加熱する。}こまめに煮汁を魚にかけてやる\footnote{arroser
  アロゼ。}。また、完全には蓋をせず、加熱中に煮汁が煮詰まるようにしてやる。

ほぼ火が通ったら、鍋の蓋をとり、魚にかけた煮汁の水分をオーヴンの熱で蒸発させて表面につやを出す\footnote{glacer
  グラセ。}。魚を鍋から出して汁気をきり、皿に盛り保温しておく。

煮汁\footnote{原文 fonds de braisage (訳注2参照)。}を漉し、しばらく休ませたら浮き脂を取り除き、必要なら煮詰める。これを加えてソースを仕上げる。

魚のブレゼには通常、各ルセットに示してあるガルニテュールを添える。

\hypertarget{ux30aaux30d6ux30ebux30fc36-ux539fux66f8-pp.280-281}{%
\subsection[オ・ブルー 原書
pp.280-281]{\texorpdfstring{オ・ブルー\footnote{au bleu
  ヴィネガーを加えることで魚の表面のぬめりが青みがかることから。} 原書
pp.280-281}{オ・ブルー 原書 pp.280-281}}\label{ux30aaux30d6ux30ebux30fc36-ux539fux66f8-pp.280-281}}

オ・ブルーは鱒、鯉、ブロシェ\footnote{川かますの一種。}のみに用いられる特殊な調理法で、基本的なポイントは以下のとおり。

\begin{enumerate}
\def\labelenumi{\arabic{enumi}.}
\item
  必ず、生きた魚を使う。
\item
  魚の表面のぬめりをとらないように、なるべく手で触れずに、わたを抜く。鱗も引かない。
\item
  魚が大きい場合は、専用の網を敷いたポワソニーエルに入れ、「沸騰したヴィネガーをかける」。ヴィネガーは、通常のクールブイヨンに加える分量\footnote{以下の「クールブイヨンA」の分量比率を参照。}。次に、ヴィネガーを入れずに用意した温かい\footnote{原文
    tiède ぬるい、温かい。}クールブイヨンを注ぎ入れる。これは、なるべく身が割れないようにするためである。その後は通常どおり加熱する\footnote{レンジで沸騰させたらオーヴンに入れる。}。
\item
  小さい鱒の場合は、生きたままのものを手早く中抜きし、塩、ヴィネガーを加えただけの沸騰したクールブイヨンで煮る。
\item
  オ・ブルーは冷製、温製どちらの仕立てにしてもいい。実際の作り方の項で示してあるソースを添えて供する。
\end{enumerate}

\href{欠落アリ}{}

\hypertarget{ux30e0ux30cbux30a8ux30fcux30eb43-ux539fux66f8-p.282}{%
\subsection[ムニエール 原書 p.282]{\texorpdfstring{ムニエール\footnote{à
  la meunière 「粉挽き職人風」の意。} 原書
p.282}{ムニエール 原書 p.282}}\label{ux30e0ux30cbux30a8ux30fcux30eb43-ux539fux66f8-p.282}}

ムニエールは素晴らしい調理法だが、小型の魚と、大きな魚の場合は切り身にしか用いない。とはいえ、丁寧にやれば1.5
kg以下のテュルボタンはムニエールで調理できる。

魚丸ごと、あるいは切り身、フィレに味つけをして小麦粉をまぶし、バターを熱したフライパンで焼く。

魚が小さい場合は普通のバターでいいが、大きい場合は澄ましバターを使った方がいい。

魚の両面を焼き、程良く火が通ったら、予め熱しておいた皿に盛る。

飾り切りにした半割りのレモンを添えて、そのまま供することも可能である。ただし、このような提供方法の場合は本来の「ムニエール」と区別するために「黄金色に焼いた\footnote{doré
  (ドレ)
  一般的な色の表現として「黄金色」の意だが、ムニエールの場合、通常は大きい魚についてのみこの表現を用いる。}」と表現する。

「ムニエール」の場合には、焼き上がった魚に少量のレモン汁をふり、塩、こしょう少々で味を整える。粗みじん切りにして湯通ししたパセリを魚の表面に散らし、焦がしバターをかけてすぐに供する。湯通ししたパセリの水分に熱いバターが触れて泡がたつので、それが消えないうちに客に料理を見せるようにする。

%\href{✓原稿下準備なし}{} \href{訳と注釈\%2020180420進行中}{}
\href{未、原文対照チェック}{} \href{未、日本語表現校正}{}
\href{未、注釈チェク}{} \href{未、原稿最終校正}{}

\hypertarget{poissons}{%
\chapter{VI 魚料理 Poissons}\label{poissons}}

\hypertarget{serie-de-courts-bouillons-de-poisson}{%
\section{クールブイヨン}\label{serie-de-courts-bouillons-de-poisson}}

\begin{recette}
\hypertarget{ux30afux30fcux30ebux30d6ux30a4ux30e8ux30f3-a}{%
\subsubsection{クールブイヨン
A}\label{ux30afux30fcux30ebux30d6ux30a4ux30e8ux30f3-a}}

水5 L に対し、ヴィネガー2.5 dL、粗塩60 g、薄切りにしたにんじん600
gと玉ねぎ500 g、タイム1枝、ローリエの小さな葉2枚、パセリの茎100
g、粒こしょう20
g(こしょうを加えるのはクールブイヨンを漉す10分前)。材料を全て鍋に入れ、火にかけて1時間弱火で煮、漉す(原書
p.277)。

\hypertarget{ux30afux30fcux30ebux30d6ux30a4ux30e8ux30f3-b-4-ux9c52ux3046ux306aux304eux30d6ux30edux30b7ux30a7ux7b49-ux539fux66f8p.277}{%
\subsubsection[クールブイヨン B (鱒、うなぎ、ブロシェ等)
原書p.277]{\texorpdfstring{クールブイヨン B \footnote{クールブイヨンは用途に応じ、AからEまでの5種が挙げられている(原書pp.277-278)。}
(鱒、うなぎ、ブロシェ等)
原書p.277}{クールブイヨン B  (鱒、うなぎ、ブロシェ等) 原書p.277}}\label{ux30afux30fcux30ebux30d6ux30a4ux30e8ux30f3-b-4-ux9c52ux3046ux306aux304eux30d6ux30edux30b7ux30a7ux7b49-ux539fux66f8p.277}}

5 L 分の材料\ldots{}\ldots{}白ワイン2.5 L 。水2.5 L
。薄切りにした玉ねぎ600 g。パセリの茎80g
。タイムの小枝1本。ローリエの葉(小) \(\frac{1}{2}\) 枚。粗塩
60g。大粒のこしょう15 g(クールブイヨンを漉す10分前に加える)。

作業手順\ldots{}\ldots{}作業:液体、香味素材、調味料を鍋に入れ、沸かす。弱火で30分程煮て、漉す。

\href{欠落アリ}{}

原注:クールブイヨンBとC\footnote{クールブイヨンBの白ワインを赤ワインに代え、香味素材としてにんじん400gを加える。鱒、鯉、マトロート用(原書pp.277-278)。}で調理した魚はクールブイヨン添えとして供する。つまり、少量の煮汁とクールブイヨンに用いた野菜を添える。野菜はよく火が通っていること。煮汁はしっかり煮詰め、提供直前に新鮮なバター少量を加えて仕上げる。
\end{recette}
\hypertarget{ux30afux30fcux30ebux30d6ux30a4ux30e8ux30f3ux306eux4f7fux3044ux65b9-ux539fux66f8-p.278}{%
\subsection{クールブイヨンの使い方 原書
p.278}\label{ux30afux30fcux30ebux30d6ux30a4ux30e8ux30f3ux306eux4f7fux3044ux65b9-ux539fux66f8-p.278}}

\begin{enumerate}
\def\labelenumi{\arabic{enumi}.}
\item
  加熱時間が30分以内の場合は、クールブイヨンは必ず事前に用意しておくこと。
\item
  加熱時間が30分を越える場合は、クールブイヨンの材料は冷たい状態のままで合わせせておく。香味素材はポワソニエールの網の下に入れる。
\item
  ごく少量のクールブイヨンでポシェ\footnote{原文 pochage à court
    mouillement『ル・ギード・キュリネール』では、この表現はテュルボタン(小型のテュルボ)、バルビュ、舌びらめ等の平たい魚をポシェする際に用いられる。本連載「舌びらめのボヌ・ファム」
    2011年3月号pp.110-111 参照。}する場合、材料は(白または赤ワインを含む場合も)魚を火にかける際に合わせる。クールブイヨンの量は魚の
  \(\frac{1}{3}\)
  の高さとし、加熱中ひんぱんに煮汁を魚にかけてやること\footnote{arroser
    アロゼ。}。この調理法の場合は通常、クールブイヨンは上で記したように
  \footnote{「クールブイヨンB」原注。}、提供直前に軽くバターを加えて仕上げ、魚に添える。
\item
  冷製にする場合は、必ずクールブイヨンに魚が浸った状態で冷ますこと。当然ながら、火にかけている時間は短かくなる\footnote{余熱で火が通るため。}。
\end{enumerate}

\hypertarget{ux539fux6ce8}{%
\subparagraph{【原注】}\label{ux539fux6ce8}}

いくつかの魚種の加熱時間は該当する項で示してある。

\hypertarget{ux9b5aux306eux8abfux7406ux6cd5}{%
\section{魚の調理法}\label{ux9b5aux306eux8abfux7406ux6cd5}}

魚料理は全て、下記のいずれかの調理法による。

\begin{enumerate}
\def\labelenumi{\arabic{enumi}.}
\item
  塩水(湯)またはクールブイヨン\footnote{court-bouillon直訳は「量の少ない煮汁」。魚の他、甲殻類、鶏などの白身肉、野菜などをポシェするのに用いる。とりわけ魚や鶏を丸ごとポシェする場合には、その名称のとおり、できるだけ少量でポシェする必要がある。また、ポシェに用いたクールブイヨンをベースにソースを作る場合が多い。}Bを用いたポシェ\ldots{}\ldots{}大きな魚丸ごと、および切り身。
\item
  ごく少量のクールブイヨンを用いたポシェ\ldots{}\ldots{}魚のフィレ、またはやや小さい魚。
\item
  ブレゼ\ldots{}\ldots{}もっぱら大きな魚。
\item
  オ・ブルー\footnote{比較的小さめの淡水魚に主として用いられる調理法。生きたままの魚の表面のぬめりをとらないように洗い、内臓を取り除いたらすぐに、塩とヴィネガーを加えたクールブイヨンで茹でる。冷製、温製どちらでも供する。原書p.281参照。}\ldots{}\ldots{}とりわけ\ruby{鱒}{ます}、鯉、ブロシェ\footnote{川かますの一種。本連載「ブロシェのクネル」2011年10月号
    pp.124-125 参照。}に合う。
\item
  揚げもの\ldots{}\ldots{}もっぱら小さい魚、切り身。
\item
  ムニエール\ldots{}\ldots{}揚げものにするのと同じ小さい魚、切り身。
\item
  グリエ\ldots{}\ldots{}小さい魚、および切り身。
\item
  グラタン\ldots{}\ldots{}小さい魚、切り身。
\end{enumerate}

\hypertarget{ux5869ux6c34ux6e6fux304aux3088ux3073ux30afux30fcux30ebux30d6ux30a4ux30e8ux30f3bux3092ux7528ux3044ux305fux52a0ux71b1ux8abfux7406}{%
\subsection{塩水(湯)およびクールブイヨンBを用いた加熱調理}\label{ux5869ux6c34ux6e6fux304aux3088ux3073ux30afux30fcux30ebux30d6ux30a4ux30e8ux30f3bux3092ux7528ux3044ux305fux52a0ux71b1ux8abfux7406}}

魚を丸ごと調理する場合は、魚に合ったポワソニエール\footnote{大きな魚を丸ごと煮るための細長い鍋。魚の形を崩さずに取り出せるよう、中に専用の網を敷いて使う。似たものに、舌びらめ等の平たい魚にぴったり合う菱形をしたテュルボティエールがある。いずれも、できるだけ少量の煮汁で魚を加熱できるように工夫されたもの。(図参照)}を用いる。魚を掃除し(テュルボは水にさらして血抜きをし)、ひれ等を切り落して形を整え、ポワソニエールの網に乗せる。魚種に応じて塩水または冷たいクールブイヨンをかぶるまで注ぐ。強火にかけて沸騰したらすぐにレンジの火の弱いところに鍋を移動させ、ポシェする。

切り身(薄すぎは絶対にいけない)の場合、沸騰した液体(塩湯またはクールブイヨン)に投入したらすぐにレンジの火の弱いところに鍋を移動させ、沸騰しない程度の温度でゆっくりと火を通す。

こうするのは、魚の身のエキスを閉じこめるためである。冷水から火にかけた場合にはエキスの大部分が流れ出してしまう。大きな魚丸ごとの場合にはこのやり方はしない。沸騰した液体に魚を投入すると身が収縮するので、大きな魚の場合は身が割れたり形が崩れたりするからだ。

塩湯あるいはクールブイヨンでポシェした魚は、ナフキンまたは専用の網に盛る。周囲をパセリで飾り、塩茹でしたじゃがいもと1種類または数種のソースを添えて供する。ガルニテュールがパセリのみの場合、魚の周囲にパセリを飾るのは客に料理を見せる\footnote{当時の宴席で主流だったロシア式サーヴィスでは、大きな銀盆に盛った料理をまず食客に見せてから、とり分けて給仕する。}直前にすること。どんな場合でも、ガルニテュールを添えたらクロッシュ\footnote{銀または陶製の保温用皿カバー。}は被せないこと。

\hypertarget{ux3054ux304fux5c11ux91cfux306eux30afux30fcux30ebux30d6ux30a4ux30e8ux30f3ux3092ux7528ux3044ux305fux30ddux30b7ux30a7-ux539fux66f8-pp.279-280}{%
\subsection{ごく少量のクールブイヨンを用いたポシェ 原書
pp.279-280}\label{ux3054ux304fux5c11ux91cfux306eux30afux30fcux30ebux30d6ux30a4ux30e8ux30f3ux3092ux7528ux3044ux305fux30ddux30b7ux30a7-ux539fux66f8-pp.279-280}}

この火入れの方法は主としてテュルボタン、バルビュ、舌びらめ、丸ごとおよびそれぞれの魚のフィレで用いる。バターを塗った天板あるいはソテ鍋に魚丸ごとあるいはそのフィレを置き、軽く塩をして、所要量の魚のフュメかマッシュルームの煮汁を注ぐ。フュメとマッシュルームの煮汁を合わせたものを用いる場合もある。蓋をして、中温のオーヴンに入れる。魚丸ごとの場合は時折煮汁をかけてやる。

魚(丸ごとでもフィレでも)に火が通ったら、注意して汁気をきり、皿に盛る。ガルニテュールを含む料理の場合、ガルニテュールを魚の周囲に盛り、ソースをかける\footnote{原文通りの順で訳したが、実際にはソースをかけてからガルニテュールを盛ったほうが良い場合もあるだろう。}。多くの場合、魚の煮汁を煮詰めてソースに加える。

\href{欠落アリ}{}

\hypertarget{ux9b5aux306eux30d6ux30ecux30bc17-ux539fux66f8-p.280}{%
\subsection[魚のブレゼ 原書 p.280]{\texorpdfstring{魚のブレゼ\footnote{ここではブレゼの語が限定的な意味で用いられていることに注意。牛のアロワイヨのような大きな塊肉のブレゼと同様の調理法、ということである。それは、香味素材を色づくまで炒めてから用いることや、主素材に豚背脂やトリュフをピケ針で差したり、豚背脂のシートで覆って加熱するという点によく表れている。ただし、これらは必須というわけではないため、事実上は「ごく少量のクールブイヨンを用いたポシェ」と区別がつきにくい。実際、モンタニェ『ラルース・ガストロノミーク』初版では、魚のブレゼについて「本来的な意味でのブレゼというよりは、ごく少量のクールブイヨンを用いたポシェである」と述べられている。逆に言えば、こんにち魚の加熱方法についてしばしば「ブレゼ」と呼ばれているものが、エスコフィエやモンタニェにおいては「少量のクールブイヨンを用いたポシェ」と表現されていたということである。}
原書
p.280}{魚のブレゼ 原書 p.280}}\label{ux9b5aux306eux30d6ux30ecux30bc17-ux539fux66f8-p.280}}

この調理法を用いるのは通常、丸ごとまたは筒切りにした鮭、大ぶりの鱒、テュルボ、テュルボタンのうち大きなもの、である。

場合によっては、魚の片面に、小さく切った豚背脂、トリュフ、コルニション、にんじん等をピケ針で差し込む。

香味素材等\footnote{原文 fonds de braisage フォン・ド・ブレザージュ
  (fonds de
  braiseフォン・ド・ブレーズ、とも)。通常は、厚い輪切りにしたにんじんと玉ねぎをバターか獣脂で色づくまで炒め、ブーケガルニ、下茹でした豚皮を合わせる(原書pp.394-395)。また、これを用いた煮汁のことも指す。}は肉料理のブレゼの場合と同じように用意するが、豚皮は用いない。提供方法に応じて、白または赤ワインと軽い魚のフュメ同量ずつを、魚の厚みの
\(\frac{3}{4}\) またはひたひたの高さまで注ぐ。厳密に肉断ち \footnote{カトリックの生活習慣として、四旬節(復活祭までの46日間)および週1
  回程度、肉類を食べないということが行なわれた(本連載2012年5月号「ソース・エスパニョル(4)」p.110、訳注1参照)。}のための仕立てにする場合を除いて、薄くスライスした豚背脂のシートを魚にかぶせる。加熱中\footnote{鍋を火にかけ、沸騰したら蓋をして中火のオーヴンに入れ、加熱する。}こまめに煮汁を魚にかけてやる\footnote{arroser
  アロゼ。}。また、完全には蓋をせず、加熱中に煮汁が煮詰まるようにしてやる。

ほぼ火が通ったら、鍋の蓋をとり、魚にかけた煮汁の水分をオーヴンの熱で蒸発させて表面につやを出す\footnote{glacer
  グラセ。}。魚を鍋から出して汁気をきり、皿に盛り保温しておく。

煮汁\footnote{原文 fonds de braisage (訳注2参照)。}を漉し、しばらく休ませたら浮き脂を取り除き、必要なら煮詰める。これを加えてソースを仕上げる。

魚のブレゼには通常、各ルセットに示してあるガルニテュールを添える。

\hypertarget{ux30aaux30d6ux30ebux30fc36-ux539fux66f8-pp.280-281}{%
\subsection[オ・ブルー 原書
pp.280-281]{\texorpdfstring{オ・ブルー\footnote{au bleu
  ヴィネガーを加えることで魚の表面のぬめりが青みがかることから。} 原書
pp.280-281}{オ・ブルー 原書 pp.280-281}}\label{ux30aaux30d6ux30ebux30fc36-ux539fux66f8-pp.280-281}}

オ・ブルーは鱒、鯉、ブロシェ\footnote{川かますの一種。}のみに用いられる特殊な調理法で、基本的なポイントは以下のとおり。

\begin{enumerate}
\def\labelenumi{\arabic{enumi}.}
\item
  必ず、生きた魚を使う。
\item
  魚の表面のぬめりをとらないように、なるべく手で触れずに、わたを抜く。鱗も引かない。
\item
  魚が大きい場合は、専用の網を敷いたポワソニーエルに入れ、「沸騰したヴィネガーをかける」。ヴィネガーは、通常のクールブイヨンに加える分量\footnote{以下の「クールブイヨンA」の分量比率を参照。}。次に、ヴィネガーを入れずに用意した温かい\footnote{原文
    tiède ぬるい、温かい。}クールブイヨンを注ぎ入れる。これは、なるべく身が割れないようにするためである。その後は通常どおり加熱する\footnote{レンジで沸騰させたらオーヴンに入れる。}。
\item
  小さい鱒の場合は、生きたままのものを手早く中抜きし、塩、ヴィネガーを加えただけの沸騰したクールブイヨンで煮る。
\item
  オ・ブルーは冷製、温製どちらの仕立てにしてもいい。実際の作り方の項で示してあるソースを添えて供する。
\end{enumerate}

\href{欠落アリ}{}

\hypertarget{ux30e0ux30cbux30a8ux30fcux30eb43-ux539fux66f8-p.282}{%
\subsection[ムニエール 原書 p.282]{\texorpdfstring{ムニエール\footnote{à
  la meunière 「粉挽き職人風」の意。} 原書
p.282}{ムニエール 原書 p.282}}\label{ux30e0ux30cbux30a8ux30fcux30eb43-ux539fux66f8-p.282}}

ムニエールは素晴らしい調理法だが、小型の魚と、大きな魚の場合は切り身にしか用いない。とはいえ、丁寧にやれば1.5
kg以下のテュルボタンはムニエールで調理できる。

魚丸ごと、あるいは切り身、フィレに味つけをして小麦粉をまぶし、バターを熱したフライパンで焼く。

魚が小さい場合は普通のバターでいいが、大きい場合は澄ましバターを使った方がいい。

魚の両面を焼き、程良く火が通ったら、予め熱しておいた皿に盛る。

飾り切りにした半割りのレモンを添えて、そのまま供することも可能である。ただし、このような提供方法の場合は本来の「ムニエール」と区別するために「黄金色に焼いた\footnote{doré
  (ドレ)
  一般的な色の表現として「黄金色」の意だが、ムニエールの場合、通常は大きい魚についてのみこの表現を用いる。}」と表現する。

「ムニエール」の場合には、焼き上がった魚に少量のレモン汁をふり、塩、こしょう少々で味を整える。粗みじん切りにして湯通ししたパセリを魚の表面に散らし、焦がしバターをかけてすぐに供する。湯通ししたパセリの水分に熱いバターが触れて泡がたつので、それが消えないうちに客に料理を見せるようにする。

%\href{✓原稿下準備なし}{} \href{訳と注釈\%2020180420進行中}{}
\href{未、原文対照チェック}{} \href{未、日本語表現校正}{}
\href{未、注釈チェク}{} \href{未、原稿最終校正}{}

\hypertarget{poissons}{%
\chapter{VI 魚料理 Poissons}\label{poissons}}

\hypertarget{serie-de-courts-bouillons-de-poisson}{%
\section{クールブイヨン}\label{serie-de-courts-bouillons-de-poisson}}

\begin{recette}
\hypertarget{ux30afux30fcux30ebux30d6ux30a4ux30e8ux30f3-a}{%
\subsubsection{クールブイヨン
A}\label{ux30afux30fcux30ebux30d6ux30a4ux30e8ux30f3-a}}

水5 L に対し、ヴィネガー2.5 dL、粗塩60 g、薄切りにしたにんじん600
gと玉ねぎ500 g、タイム1枝、ローリエの小さな葉2枚、パセリの茎100
g、粒こしょう20
g(こしょうを加えるのはクールブイヨンを漉す10分前)。材料を全て鍋に入れ、火にかけて1時間弱火で煮、漉す(原書
p.277)。

\hypertarget{ux30afux30fcux30ebux30d6ux30a4ux30e8ux30f3-b-4-ux9c52ux3046ux306aux304eux30d6ux30edux30b7ux30a7ux7b49-ux539fux66f8p.277}{%
\subsubsection[クールブイヨン B (鱒、うなぎ、ブロシェ等)
原書p.277]{\texorpdfstring{クールブイヨン B \footnote{クールブイヨンは用途に応じ、AからEまでの5種が挙げられている(原書pp.277-278)。}
(鱒、うなぎ、ブロシェ等)
原書p.277}{クールブイヨン B  (鱒、うなぎ、ブロシェ等) 原書p.277}}\label{ux30afux30fcux30ebux30d6ux30a4ux30e8ux30f3-b-4-ux9c52ux3046ux306aux304eux30d6ux30edux30b7ux30a7ux7b49-ux539fux66f8p.277}}

5 L 分の材料\ldots{}\ldots{}白ワイン2.5 L 。水2.5 L
。薄切りにした玉ねぎ600 g。パセリの茎80g
。タイムの小枝1本。ローリエの葉(小) \(\frac{1}{2}\) 枚。粗塩
60g。大粒のこしょう15 g(クールブイヨンを漉す10分前に加える)。

作業手順\ldots{}\ldots{}作業:液体、香味素材、調味料を鍋に入れ、沸かす。弱火で30分程煮て、漉す。

\href{欠落アリ}{}

原注:クールブイヨンBとC\footnote{クールブイヨンBの白ワインを赤ワインに代え、香味素材としてにんじん400gを加える。鱒、鯉、マトロート用(原書pp.277-278)。}で調理した魚はクールブイヨン添えとして供する。つまり、少量の煮汁とクールブイヨンに用いた野菜を添える。野菜はよく火が通っていること。煮汁はしっかり煮詰め、提供直前に新鮮なバター少量を加えて仕上げる。
\end{recette}
\hypertarget{ux30afux30fcux30ebux30d6ux30a4ux30e8ux30f3ux306eux4f7fux3044ux65b9-ux539fux66f8-p.278}{%
\subsection{クールブイヨンの使い方 原書
p.278}\label{ux30afux30fcux30ebux30d6ux30a4ux30e8ux30f3ux306eux4f7fux3044ux65b9-ux539fux66f8-p.278}}

\begin{enumerate}
\def\labelenumi{\arabic{enumi}.}
\item
  加熱時間が30分以内の場合は、クールブイヨンは必ず事前に用意しておくこと。
\item
  加熱時間が30分を越える場合は、クールブイヨンの材料は冷たい状態のままで合わせせておく。香味素材はポワソニエールの網の下に入れる。
\item
  ごく少量のクールブイヨンでポシェ\footnote{原文 pochage à court
    mouillement『ル・ギード・キュリネール』では、この表現はテュルボタン(小型のテュルボ)、バルビュ、舌びらめ等の平たい魚をポシェする際に用いられる。本連載「舌びらめのボヌ・ファム」
    2011年3月号pp.110-111 参照。}する場合、材料は(白または赤ワインを含む場合も)魚を火にかける際に合わせる。クールブイヨンの量は魚の
  \(\frac{1}{3}\)
  の高さとし、加熱中ひんぱんに煮汁を魚にかけてやること\footnote{arroser
    アロゼ。}。この調理法の場合は通常、クールブイヨンは上で記したように
  \footnote{「クールブイヨンB」原注。}、提供直前に軽くバターを加えて仕上げ、魚に添える。
\item
  冷製にする場合は、必ずクールブイヨンに魚が浸った状態で冷ますこと。当然ながら、火にかけている時間は短かくなる\footnote{余熱で火が通るため。}。
\end{enumerate}

\hypertarget{ux539fux6ce8}{%
\subparagraph{【原注】}\label{ux539fux6ce8}}

いくつかの魚種の加熱時間は該当する項で示してある。

\hypertarget{ux9b5aux306eux8abfux7406ux6cd5}{%
\section{魚の調理法}\label{ux9b5aux306eux8abfux7406ux6cd5}}

魚料理は全て、下記のいずれかの調理法による。

\begin{enumerate}
\def\labelenumi{\arabic{enumi}.}
\item
  塩水(湯)またはクールブイヨン\footnote{court-bouillon直訳は「量の少ない煮汁」。魚の他、甲殻類、鶏などの白身肉、野菜などをポシェするのに用いる。とりわけ魚や鶏を丸ごとポシェする場合には、その名称のとおり、できるだけ少量でポシェする必要がある。また、ポシェに用いたクールブイヨンをベースにソースを作る場合が多い。}Bを用いたポシェ\ldots{}\ldots{}大きな魚丸ごと、および切り身。
\item
  ごく少量のクールブイヨンを用いたポシェ\ldots{}\ldots{}魚のフィレ、またはやや小さい魚。
\item
  ブレゼ\ldots{}\ldots{}もっぱら大きな魚。
\item
  オ・ブルー\footnote{比較的小さめの淡水魚に主として用いられる調理法。生きたままの魚の表面のぬめりをとらないように洗い、内臓を取り除いたらすぐに、塩とヴィネガーを加えたクールブイヨンで茹でる。冷製、温製どちらでも供する。原書p.281参照。}\ldots{}\ldots{}とりわけ\ruby{鱒}{ます}、鯉、ブロシェ\footnote{川かますの一種。本連載「ブロシェのクネル」2011年10月号
    pp.124-125 参照。}に合う。
\item
  揚げもの\ldots{}\ldots{}もっぱら小さい魚、切り身。
\item
  ムニエール\ldots{}\ldots{}揚げものにするのと同じ小さい魚、切り身。
\item
  グリエ\ldots{}\ldots{}小さい魚、および切り身。
\item
  グラタン\ldots{}\ldots{}小さい魚、切り身。
\end{enumerate}

\hypertarget{ux5869ux6c34ux6e6fux304aux3088ux3073ux30afux30fcux30ebux30d6ux30a4ux30e8ux30f3bux3092ux7528ux3044ux305fux52a0ux71b1ux8abfux7406}{%
\subsection{塩水(湯)およびクールブイヨンBを用いた加熱調理}\label{ux5869ux6c34ux6e6fux304aux3088ux3073ux30afux30fcux30ebux30d6ux30a4ux30e8ux30f3bux3092ux7528ux3044ux305fux52a0ux71b1ux8abfux7406}}

魚を丸ごと調理する場合は、魚に合ったポワソニエール\footnote{大きな魚を丸ごと煮るための細長い鍋。魚の形を崩さずに取り出せるよう、中に専用の網を敷いて使う。似たものに、舌びらめ等の平たい魚にぴったり合う菱形をしたテュルボティエールがある。いずれも、できるだけ少量の煮汁で魚を加熱できるように工夫されたもの。(図参照)}を用いる。魚を掃除し(テュルボは水にさらして血抜きをし)、ひれ等を切り落して形を整え、ポワソニエールの網に乗せる。魚種に応じて塩水または冷たいクールブイヨンをかぶるまで注ぐ。強火にかけて沸騰したらすぐにレンジの火の弱いところに鍋を移動させ、ポシェする。

切り身(薄すぎは絶対にいけない)の場合、沸騰した液体(塩湯またはクールブイヨン)に投入したらすぐにレンジの火の弱いところに鍋を移動させ、沸騰しない程度の温度でゆっくりと火を通す。

こうするのは、魚の身のエキスを閉じこめるためである。冷水から火にかけた場合にはエキスの大部分が流れ出してしまう。大きな魚丸ごとの場合にはこのやり方はしない。沸騰した液体に魚を投入すると身が収縮するので、大きな魚の場合は身が割れたり形が崩れたりするからだ。

塩湯あるいはクールブイヨンでポシェした魚は、ナフキンまたは専用の網に盛る。周囲をパセリで飾り、塩茹でしたじゃがいもと1種類または数種のソースを添えて供する。ガルニテュールがパセリのみの場合、魚の周囲にパセリを飾るのは客に料理を見せる\footnote{当時の宴席で主流だったロシア式サーヴィスでは、大きな銀盆に盛った料理をまず食客に見せてから、とり分けて給仕する。}直前にすること。どんな場合でも、ガルニテュールを添えたらクロッシュ\footnote{銀または陶製の保温用皿カバー。}は被せないこと。

\hypertarget{ux3054ux304fux5c11ux91cfux306eux30afux30fcux30ebux30d6ux30a4ux30e8ux30f3ux3092ux7528ux3044ux305fux30ddux30b7ux30a7-ux539fux66f8-pp.279-280}{%
\subsection{ごく少量のクールブイヨンを用いたポシェ 原書
pp.279-280}\label{ux3054ux304fux5c11ux91cfux306eux30afux30fcux30ebux30d6ux30a4ux30e8ux30f3ux3092ux7528ux3044ux305fux30ddux30b7ux30a7-ux539fux66f8-pp.279-280}}

この火入れの方法は主としてテュルボタン、バルビュ、舌びらめ、丸ごとおよびそれぞれの魚のフィレで用いる。バターを塗った天板あるいはソテ鍋に魚丸ごとあるいはそのフィレを置き、軽く塩をして、所要量の魚のフュメかマッシュルームの煮汁を注ぐ。フュメとマッシュルームの煮汁を合わせたものを用いる場合もある。蓋をして、中温のオーヴンに入れる。魚丸ごとの場合は時折煮汁をかけてやる。

魚(丸ごとでもフィレでも)に火が通ったら、注意して汁気をきり、皿に盛る。ガルニテュールを含む料理の場合、ガルニテュールを魚の周囲に盛り、ソースをかける\footnote{原文通りの順で訳したが、実際にはソースをかけてからガルニテュールを盛ったほうが良い場合もあるだろう。}。多くの場合、魚の煮汁を煮詰めてソースに加える。

\href{欠落アリ}{}

\hypertarget{ux9b5aux306eux30d6ux30ecux30bc17-ux539fux66f8-p.280}{%
\subsection[魚のブレゼ 原書 p.280]{\texorpdfstring{魚のブレゼ\footnote{ここではブレゼの語が限定的な意味で用いられていることに注意。牛のアロワイヨのような大きな塊肉のブレゼと同様の調理法、ということである。それは、香味素材を色づくまで炒めてから用いることや、主素材に豚背脂やトリュフをピケ針で差したり、豚背脂のシートで覆って加熱するという点によく表れている。ただし、これらは必須というわけではないため、事実上は「ごく少量のクールブイヨンを用いたポシェ」と区別がつきにくい。実際、モンタニェ『ラルース・ガストロノミーク』初版では、魚のブレゼについて「本来的な意味でのブレゼというよりは、ごく少量のクールブイヨンを用いたポシェである」と述べられている。逆に言えば、こんにち魚の加熱方法についてしばしば「ブレゼ」と呼ばれているものが、エスコフィエやモンタニェにおいては「少量のクールブイヨンを用いたポシェ」と表現されていたということである。}
原書
p.280}{魚のブレゼ 原書 p.280}}\label{ux9b5aux306eux30d6ux30ecux30bc17-ux539fux66f8-p.280}}

この調理法を用いるのは通常、丸ごとまたは筒切りにした鮭、大ぶりの鱒、テュルボ、テュルボタンのうち大きなもの、である。

場合によっては、魚の片面に、小さく切った豚背脂、トリュフ、コルニション、にんじん等をピケ針で差し込む。

香味素材等\footnote{原文 fonds de braisage フォン・ド・ブレザージュ
  (fonds de
  braiseフォン・ド・ブレーズ、とも)。通常は、厚い輪切りにしたにんじんと玉ねぎをバターか獣脂で色づくまで炒め、ブーケガルニ、下茹でした豚皮を合わせる(原書pp.394-395)。また、これを用いた煮汁のことも指す。}は肉料理のブレゼの場合と同じように用意するが、豚皮は用いない。提供方法に応じて、白または赤ワインと軽い魚のフュメ同量ずつを、魚の厚みの
\(\frac{3}{4}\) またはひたひたの高さまで注ぐ。厳密に肉断ち \footnote{カトリックの生活習慣として、四旬節(復活祭までの46日間)および週1
  回程度、肉類を食べないということが行なわれた(本連載2012年5月号「ソース・エスパニョル(4)」p.110、訳注1参照)。}のための仕立てにする場合を除いて、薄くスライスした豚背脂のシートを魚にかぶせる。加熱中\footnote{鍋を火にかけ、沸騰したら蓋をして中火のオーヴンに入れ、加熱する。}こまめに煮汁を魚にかけてやる\footnote{arroser
  アロゼ。}。また、完全には蓋をせず、加熱中に煮汁が煮詰まるようにしてやる。

ほぼ火が通ったら、鍋の蓋をとり、魚にかけた煮汁の水分をオーヴンの熱で蒸発させて表面につやを出す\footnote{glacer
  グラセ。}。魚を鍋から出して汁気をきり、皿に盛り保温しておく。

煮汁\footnote{原文 fonds de braisage (訳注2参照)。}を漉し、しばらく休ませたら浮き脂を取り除き、必要なら煮詰める。これを加えてソースを仕上げる。

魚のブレゼには通常、各ルセットに示してあるガルニテュールを添える。

\hypertarget{ux30aaux30d6ux30ebux30fc36-ux539fux66f8-pp.280-281}{%
\subsection[オ・ブルー 原書
pp.280-281]{\texorpdfstring{オ・ブルー\footnote{au bleu
  ヴィネガーを加えることで魚の表面のぬめりが青みがかることから。} 原書
pp.280-281}{オ・ブルー 原書 pp.280-281}}\label{ux30aaux30d6ux30ebux30fc36-ux539fux66f8-pp.280-281}}

オ・ブルーは鱒、鯉、ブロシェ\footnote{川かますの一種。}のみに用いられる特殊な調理法で、基本的なポイントは以下のとおり。

\begin{enumerate}
\def\labelenumi{\arabic{enumi}.}
\item
  必ず、生きた魚を使う。
\item
  魚の表面のぬめりをとらないように、なるべく手で触れずに、わたを抜く。鱗も引かない。
\item
  魚が大きい場合は、専用の網を敷いたポワソニーエルに入れ、「沸騰したヴィネガーをかける」。ヴィネガーは、通常のクールブイヨンに加える分量\footnote{以下の「クールブイヨンA」の分量比率を参照。}。次に、ヴィネガーを入れずに用意した温かい\footnote{原文
    tiède ぬるい、温かい。}クールブイヨンを注ぎ入れる。これは、なるべく身が割れないようにするためである。その後は通常どおり加熱する\footnote{レンジで沸騰させたらオーヴンに入れる。}。
\item
  小さい鱒の場合は、生きたままのものを手早く中抜きし、塩、ヴィネガーを加えただけの沸騰したクールブイヨンで煮る。
\item
  オ・ブルーは冷製、温製どちらの仕立てにしてもいい。実際の作り方の項で示してあるソースを添えて供する。
\end{enumerate}

\href{欠落アリ}{}

\hypertarget{ux30e0ux30cbux30a8ux30fcux30eb43-ux539fux66f8-p.282}{%
\subsection[ムニエール 原書 p.282]{\texorpdfstring{ムニエール\footnote{à
  la meunière 「粉挽き職人風」の意。} 原書
p.282}{ムニエール 原書 p.282}}\label{ux30e0ux30cbux30a8ux30fcux30eb43-ux539fux66f8-p.282}}

ムニエールは素晴らしい調理法だが、小型の魚と、大きな魚の場合は切り身にしか用いない。とはいえ、丁寧にやれば1.5
kg以下のテュルボタンはムニエールで調理できる。

魚丸ごと、あるいは切り身、フィレに味つけをして小麦粉をまぶし、バターを熱したフライパンで焼く。

魚が小さい場合は普通のバターでいいが、大きい場合は澄ましバターを使った方がいい。

魚の両面を焼き、程良く火が通ったら、予め熱しておいた皿に盛る。

飾り切りにした半割りのレモンを添えて、そのまま供することも可能である。ただし、このような提供方法の場合は本来の「ムニエール」と区別するために「黄金色に焼いた\footnote{doré
  (ドレ)
  一般的な色の表現として「黄金色」の意だが、ムニエールの場合、通常は大きい魚についてのみこの表現を用いる。}」と表現する。

「ムニエール」の場合には、焼き上がった魚に少量のレモン汁をふり、塩、こしょう少々で味を整える。粗みじん切りにして湯通ししたパセリを魚の表面に散らし、焦がしバターをかけてすぐに供する。湯通ししたパセリの水分に熱いバターが触れて泡がたつので、それが消えないうちに客に料理を見せるようにする。




%%% Chapitre VI. Poissons
%% エスコフィエ『料理の手引き』全注解
% 五島 学

\href{✓原稿下準備なし}{} \href{訳と注釈\%2020180420進行中}{}
\href{未、原文対照チェック}{} \href{未、日本語表現校正}{}
\href{未、注釈チェク}{} \href{未、原稿最終校正}{}

\hypertarget{poissons}{%
\chapter{VI 魚料理 Poissons}\label{poissons}}

\hypertarget{serie-de-courts-bouillons-de-poisson}{%
\section{クールブイヨン}\label{serie-de-courts-bouillons-de-poisson}}

\begin{recette}
\hypertarget{ux30afux30fcux30ebux30d6ux30a4ux30e8ux30f3-a}{%
\subsubsection{クールブイヨン
A}\label{ux30afux30fcux30ebux30d6ux30a4ux30e8ux30f3-a}}

水5 L に対し、ヴィネガー2.5 dL、粗塩60 g、薄切りにしたにんじん600
gと玉ねぎ500 g、タイム1枝、ローリエの小さな葉2枚、パセリの茎100
g、粒こしょう20
g(こしょうを加えるのはクールブイヨンを漉す10分前)。材料を全て鍋に入れ、火にかけて1時間弱火で煮、漉す(原書
p.277)。

\hypertarget{ux30afux30fcux30ebux30d6ux30a4ux30e8ux30f3-b-4-ux9c52ux3046ux306aux304eux30d6ux30edux30b7ux30a7ux7b49-ux539fux66f8p.277}{%
\subsubsection[クールブイヨン B (鱒、うなぎ、ブロシェ等)
原書p.277]{\texorpdfstring{クールブイヨン B \footnote{クールブイヨンは用途に応じ、AからEまでの5種が挙げられている(原書pp.277-278)。}
(鱒、うなぎ、ブロシェ等)
原書p.277}{クールブイヨン B  (鱒、うなぎ、ブロシェ等) 原書p.277}}\label{ux30afux30fcux30ebux30d6ux30a4ux30e8ux30f3-b-4-ux9c52ux3046ux306aux304eux30d6ux30edux30b7ux30a7ux7b49-ux539fux66f8p.277}}

5 L 分の材料\ldots{}\ldots{}白ワイン2.5 L 。水2.5 L
。薄切りにした玉ねぎ600 g。パセリの茎80g
。タイムの小枝1本。ローリエの葉(小) \(\frac{1}{2}\) 枚。粗塩
60g。大粒のこしょう15 g(クールブイヨンを漉す10分前に加える)。

作業手順\ldots{}\ldots{}作業:液体、香味素材、調味料を鍋に入れ、沸かす。弱火で30分程煮て、漉す。

\href{欠落アリ}{}

原注:クールブイヨンBとC\footnote{クールブイヨンBの白ワインを赤ワインに代え、香味素材としてにんじん400gを加える。鱒、鯉、マトロート用(原書pp.277-278)。}で調理した魚はクールブイヨン添えとして供する。つまり、少量の煮汁とクールブイヨンに用いた野菜を添える。野菜はよく火が通っていること。煮汁はしっかり煮詰め、提供直前に新鮮なバター少量を加えて仕上げる。
\end{recette}
\hypertarget{ux30afux30fcux30ebux30d6ux30a4ux30e8ux30f3ux306eux4f7fux3044ux65b9-ux539fux66f8-p.278}{%
\subsection{クールブイヨンの使い方 原書
p.278}\label{ux30afux30fcux30ebux30d6ux30a4ux30e8ux30f3ux306eux4f7fux3044ux65b9-ux539fux66f8-p.278}}

\begin{enumerate}
\def\labelenumi{\arabic{enumi}.}
\item
  加熱時間が30分以内の場合は、クールブイヨンは必ず事前に用意しておくこと。
\item
  加熱時間が30分を越える場合は、クールブイヨンの材料は冷たい状態のままで合わせせておく。香味素材はポワソニエールの網の下に入れる。
\item
  ごく少量のクールブイヨンでポシェ\footnote{原文 pochage à court
    mouillement『ル・ギード・キュリネール』では、この表現はテュルボタン(小型のテュルボ)、バルビュ、舌びらめ等の平たい魚をポシェする際に用いられる。本連載「舌びらめのボヌ・ファム」
    2011年3月号pp.110-111 参照。}する場合、材料は(白または赤ワインを含む場合も)魚を火にかける際に合わせる。クールブイヨンの量は魚の
  \(\frac{1}{3}\)
  の高さとし、加熱中ひんぱんに煮汁を魚にかけてやること\footnote{arroser
    アロゼ。}。この調理法の場合は通常、クールブイヨンは上で記したように
  \footnote{「クールブイヨンB」原注。}、提供直前に軽くバターを加えて仕上げ、魚に添える。
\item
  冷製にする場合は、必ずクールブイヨンに魚が浸った状態で冷ますこと。当然ながら、火にかけている時間は短かくなる\footnote{余熱で火が通るため。}。
\end{enumerate}

\hypertarget{ux539fux6ce8}{%
\subparagraph{【原注】}\label{ux539fux6ce8}}

いくつかの魚種の加熱時間は該当する項で示してある。

\hypertarget{ux9b5aux306eux8abfux7406ux6cd5}{%
\section{魚の調理法}\label{ux9b5aux306eux8abfux7406ux6cd5}}

魚料理は全て、下記のいずれかの調理法による。

\begin{enumerate}
\def\labelenumi{\arabic{enumi}.}
\item
  塩水(湯)またはクールブイヨン\footnote{court-bouillon直訳は「量の少ない煮汁」。魚の他、甲殻類、鶏などの白身肉、野菜などをポシェするのに用いる。とりわけ魚や鶏を丸ごとポシェする場合には、その名称のとおり、できるだけ少量でポシェする必要がある。また、ポシェに用いたクールブイヨンをベースにソースを作る場合が多い。}Bを用いたポシェ\ldots{}\ldots{}大きな魚丸ごと、および切り身。
\item
  ごく少量のクールブイヨンを用いたポシェ\ldots{}\ldots{}魚のフィレ、またはやや小さい魚。
\item
  ブレゼ\ldots{}\ldots{}もっぱら大きな魚。
\item
  オ・ブルー\footnote{比較的小さめの淡水魚に主として用いられる調理法。生きたままの魚の表面のぬめりをとらないように洗い、内臓を取り除いたらすぐに、塩とヴィネガーを加えたクールブイヨンで茹でる。冷製、温製どちらでも供する。原書p.281参照。}\ldots{}\ldots{}とりわけ\ruby{鱒}{ます}、鯉、ブロシェ\footnote{川かますの一種。本連載「ブロシェのクネル」2011年10月号
    pp.124-125 参照。}に合う。
\item
  揚げもの\ldots{}\ldots{}もっぱら小さい魚、切り身。
\item
  ムニエール\ldots{}\ldots{}揚げものにするのと同じ小さい魚、切り身。
\item
  グリエ\ldots{}\ldots{}小さい魚、および切り身。
\item
  グラタン\ldots{}\ldots{}小さい魚、切り身。
\end{enumerate}

\hypertarget{ux5869ux6c34ux6e6fux304aux3088ux3073ux30afux30fcux30ebux30d6ux30a4ux30e8ux30f3bux3092ux7528ux3044ux305fux52a0ux71b1ux8abfux7406}{%
\subsection{塩水(湯)およびクールブイヨンBを用いた加熱調理}\label{ux5869ux6c34ux6e6fux304aux3088ux3073ux30afux30fcux30ebux30d6ux30a4ux30e8ux30f3bux3092ux7528ux3044ux305fux52a0ux71b1ux8abfux7406}}

魚を丸ごと調理する場合は、魚に合ったポワソニエール\footnote{大きな魚を丸ごと煮るための細長い鍋。魚の形を崩さずに取り出せるよう、中に専用の網を敷いて使う。似たものに、舌びらめ等の平たい魚にぴったり合う菱形をしたテュルボティエールがある。いずれも、できるだけ少量の煮汁で魚を加熱できるように工夫されたもの。(図参照)}を用いる。魚を掃除し(テュルボは水にさらして血抜きをし)、ひれ等を切り落して形を整え、ポワソニエールの網に乗せる。魚種に応じて塩水または冷たいクールブイヨンをかぶるまで注ぐ。強火にかけて沸騰したらすぐにレンジの火の弱いところに鍋を移動させ、ポシェする。

切り身(薄すぎは絶対にいけない)の場合、沸騰した液体(塩湯またはクールブイヨン)に投入したらすぐにレンジの火の弱いところに鍋を移動させ、沸騰しない程度の温度でゆっくりと火を通す。

こうするのは、魚の身のエキスを閉じこめるためである。冷水から火にかけた場合にはエキスの大部分が流れ出してしまう。大きな魚丸ごとの場合にはこのやり方はしない。沸騰した液体に魚を投入すると身が収縮するので、大きな魚の場合は身が割れたり形が崩れたりするからだ。

塩湯あるいはクールブイヨンでポシェした魚は、ナフキンまたは専用の網に盛る。周囲をパセリで飾り、塩茹でしたじゃがいもと1種類または数種のソースを添えて供する。ガルニテュールがパセリのみの場合、魚の周囲にパセリを飾るのは客に料理を見せる\footnote{当時の宴席で主流だったロシア式サーヴィスでは、大きな銀盆に盛った料理をまず食客に見せてから、とり分けて給仕する。}直前にすること。どんな場合でも、ガルニテュールを添えたらクロッシュ\footnote{銀または陶製の保温用皿カバー。}は被せないこと。

\hypertarget{ux3054ux304fux5c11ux91cfux306eux30afux30fcux30ebux30d6ux30a4ux30e8ux30f3ux3092ux7528ux3044ux305fux30ddux30b7ux30a7-ux539fux66f8-pp.279-280}{%
\subsection{ごく少量のクールブイヨンを用いたポシェ 原書
pp.279-280}\label{ux3054ux304fux5c11ux91cfux306eux30afux30fcux30ebux30d6ux30a4ux30e8ux30f3ux3092ux7528ux3044ux305fux30ddux30b7ux30a7-ux539fux66f8-pp.279-280}}

この火入れの方法は主としてテュルボタン、バルビュ、舌びらめ、丸ごとおよびそれぞれの魚のフィレで用いる。バターを塗った天板あるいはソテ鍋に魚丸ごとあるいはそのフィレを置き、軽く塩をして、所要量の魚のフュメかマッシュルームの煮汁を注ぐ。フュメとマッシュルームの煮汁を合わせたものを用いる場合もある。蓋をして、中温のオーヴンに入れる。魚丸ごとの場合は時折煮汁をかけてやる。

魚(丸ごとでもフィレでも)に火が通ったら、注意して汁気をきり、皿に盛る。ガルニテュールを含む料理の場合、ガルニテュールを魚の周囲に盛り、ソースをかける\footnote{原文通りの順で訳したが、実際にはソースをかけてからガルニテュールを盛ったほうが良い場合もあるだろう。}。多くの場合、魚の煮汁を煮詰めてソースに加える。

\href{欠落アリ}{}

\hypertarget{ux9b5aux306eux30d6ux30ecux30bc17-ux539fux66f8-p.280}{%
\subsection[魚のブレゼ 原書 p.280]{\texorpdfstring{魚のブレゼ\footnote{ここではブレゼの語が限定的な意味で用いられていることに注意。牛のアロワイヨのような大きな塊肉のブレゼと同様の調理法、ということである。それは、香味素材を色づくまで炒めてから用いることや、主素材に豚背脂やトリュフをピケ針で差したり、豚背脂のシートで覆って加熱するという点によく表れている。ただし、これらは必須というわけではないため、事実上は「ごく少量のクールブイヨンを用いたポシェ」と区別がつきにくい。実際、モンタニェ『ラルース・ガストロノミーク』初版では、魚のブレゼについて「本来的な意味でのブレゼというよりは、ごく少量のクールブイヨンを用いたポシェである」と述べられている。逆に言えば、こんにち魚の加熱方法についてしばしば「ブレゼ」と呼ばれているものが、エスコフィエやモンタニェにおいては「少量のクールブイヨンを用いたポシェ」と表現されていたということである。}
原書
p.280}{魚のブレゼ 原書 p.280}}\label{ux9b5aux306eux30d6ux30ecux30bc17-ux539fux66f8-p.280}}

この調理法を用いるのは通常、丸ごとまたは筒切りにした鮭、大ぶりの鱒、テュルボ、テュルボタンのうち大きなもの、である。

場合によっては、魚の片面に、小さく切った豚背脂、トリュフ、コルニション、にんじん等をピケ針で差し込む。

香味素材等\footnote{原文 fonds de braisage フォン・ド・ブレザージュ
  (fonds de
  braiseフォン・ド・ブレーズ、とも)。通常は、厚い輪切りにしたにんじんと玉ねぎをバターか獣脂で色づくまで炒め、ブーケガルニ、下茹でした豚皮を合わせる(原書pp.394-395)。また、これを用いた煮汁のことも指す。}は肉料理のブレゼの場合と同じように用意するが、豚皮は用いない。提供方法に応じて、白または赤ワインと軽い魚のフュメ同量ずつを、魚の厚みの
\(\frac{3}{4}\) またはひたひたの高さまで注ぐ。厳密に肉断ち \footnote{カトリックの生活習慣として、四旬節(復活祭までの46日間)および週1
  回程度、肉類を食べないということが行なわれた(本連載2012年5月号「ソース・エスパニョル(4)」p.110、訳注1参照)。}のための仕立てにする場合を除いて、薄くスライスした豚背脂のシートを魚にかぶせる。加熱中\footnote{鍋を火にかけ、沸騰したら蓋をして中火のオーヴンに入れ、加熱する。}こまめに煮汁を魚にかけてやる\footnote{arroser
  アロゼ。}。また、完全には蓋をせず、加熱中に煮汁が煮詰まるようにしてやる。

ほぼ火が通ったら、鍋の蓋をとり、魚にかけた煮汁の水分をオーヴンの熱で蒸発させて表面につやを出す\footnote{glacer
  グラセ。}。魚を鍋から出して汁気をきり、皿に盛り保温しておく。

煮汁\footnote{原文 fonds de braisage (訳注2参照)。}を漉し、しばらく休ませたら浮き脂を取り除き、必要なら煮詰める。これを加えてソースを仕上げる。

魚のブレゼには通常、各ルセットに示してあるガルニテュールを添える。

\hypertarget{ux30aaux30d6ux30ebux30fc36-ux539fux66f8-pp.280-281}{%
\subsection[オ・ブルー 原書
pp.280-281]{\texorpdfstring{オ・ブルー\footnote{au bleu
  ヴィネガーを加えることで魚の表面のぬめりが青みがかることから。} 原書
pp.280-281}{オ・ブルー 原書 pp.280-281}}\label{ux30aaux30d6ux30ebux30fc36-ux539fux66f8-pp.280-281}}

オ・ブルーは鱒、鯉、ブロシェ\footnote{川かますの一種。}のみに用いられる特殊な調理法で、基本的なポイントは以下のとおり。

\begin{enumerate}
\def\labelenumi{\arabic{enumi}.}
\item
  必ず、生きた魚を使う。
\item
  魚の表面のぬめりをとらないように、なるべく手で触れずに、わたを抜く。鱗も引かない。
\item
  魚が大きい場合は、専用の網を敷いたポワソニーエルに入れ、「沸騰したヴィネガーをかける」。ヴィネガーは、通常のクールブイヨンに加える分量\footnote{以下の「クールブイヨンA」の分量比率を参照。}。次に、ヴィネガーを入れずに用意した温かい\footnote{原文
    tiède ぬるい、温かい。}クールブイヨンを注ぎ入れる。これは、なるべく身が割れないようにするためである。その後は通常どおり加熱する\footnote{レンジで沸騰させたらオーヴンに入れる。}。
\item
  小さい鱒の場合は、生きたままのものを手早く中抜きし、塩、ヴィネガーを加えただけの沸騰したクールブイヨンで煮る。
\item
  オ・ブルーは冷製、温製どちらの仕立てにしてもいい。実際の作り方の項で示してあるソースを添えて供する。
\end{enumerate}

\href{欠落アリ}{}

\hypertarget{ux30e0ux30cbux30a8ux30fcux30eb43-ux539fux66f8-p.282}{%
\subsection[ムニエール 原書 p.282]{\texorpdfstring{ムニエール\footnote{à
  la meunière 「粉挽き職人風」の意。} 原書
p.282}{ムニエール 原書 p.282}}\label{ux30e0ux30cbux30a8ux30fcux30eb43-ux539fux66f8-p.282}}

ムニエールは素晴らしい調理法だが、小型の魚と、大きな魚の場合は切り身にしか用いない。とはいえ、丁寧にやれば1.5
kg以下のテュルボタンはムニエールで調理できる。

魚丸ごと、あるいは切り身、フィレに味つけをして小麦粉をまぶし、バターを熱したフライパンで焼く。

魚が小さい場合は普通のバターでいいが、大きい場合は澄ましバターを使った方がいい。

魚の両面を焼き、程良く火が通ったら、予め熱しておいた皿に盛る。

飾り切りにした半割りのレモンを添えて、そのまま供することも可能である。ただし、このような提供方法の場合は本来の「ムニエール」と区別するために「黄金色に焼いた\footnote{doré
  (ドレ)
  一般的な色の表現として「黄金色」の意だが、ムニエールの場合、通常は大きい魚についてのみこの表現を用いる。}」と表現する。

「ムニエール」の場合には、焼き上がった魚に少量のレモン汁をふり、塩、こしょう少々で味を整える。粗みじん切りにして湯通ししたパセリを魚の表面に散らし、焦がしバターをかけてすぐに供する。湯通ししたパセリの水分に熱いバターが触れて泡がたつので、それが消えないうちに客に料理を見せるようにする。

%\input{06-poissons/06-02-p-284}
%\input{06-poissons/06-03-pp285-286}
%\input{06-poissons/06-04-p286}
%\input{06-poissons/06-05-pp286-290}

%\href{✓原稿下準備なし}{} \href{訳と注釈\%2020180420進行中}{}
\href{未、原文対照チェック}{} \href{未、日本語表現校正}{}
\href{未、注釈チェク}{} \href{未、原稿最終校正}{}

\hypertarget{poissons}{%
\chapter{VI 魚料理 Poissons}\label{poissons}}

\hypertarget{serie-de-courts-bouillons-de-poisson}{%
\section{クールブイヨン}\label{serie-de-courts-bouillons-de-poisson}}

\begin{recette}
\hypertarget{ux30afux30fcux30ebux30d6ux30a4ux30e8ux30f3-a}{%
\subsubsection{クールブイヨン
A}\label{ux30afux30fcux30ebux30d6ux30a4ux30e8ux30f3-a}}

水5 L に対し、ヴィネガー2.5 dL、粗塩60 g、薄切りにしたにんじん600
gと玉ねぎ500 g、タイム1枝、ローリエの小さな葉2枚、パセリの茎100
g、粒こしょう20
g(こしょうを加えるのはクールブイヨンを漉す10分前)。材料を全て鍋に入れ、火にかけて1時間弱火で煮、漉す(原書
p.277)。

\hypertarget{ux30afux30fcux30ebux30d6ux30a4ux30e8ux30f3-b-4-ux9c52ux3046ux306aux304eux30d6ux30edux30b7ux30a7ux7b49-ux539fux66f8p.277}{%
\subsubsection[クールブイヨン B (鱒、うなぎ、ブロシェ等)
原書p.277]{\texorpdfstring{クールブイヨン B \footnote{クールブイヨンは用途に応じ、AからEまでの5種が挙げられている(原書pp.277-278)。}
(鱒、うなぎ、ブロシェ等)
原書p.277}{クールブイヨン B  (鱒、うなぎ、ブロシェ等) 原書p.277}}\label{ux30afux30fcux30ebux30d6ux30a4ux30e8ux30f3-b-4-ux9c52ux3046ux306aux304eux30d6ux30edux30b7ux30a7ux7b49-ux539fux66f8p.277}}

5 L 分の材料\ldots{}\ldots{}白ワイン2.5 L 。水2.5 L
。薄切りにした玉ねぎ600 g。パセリの茎80g
。タイムの小枝1本。ローリエの葉(小) \(\frac{1}{2}\) 枚。粗塩
60g。大粒のこしょう15 g(クールブイヨンを漉す10分前に加える)。

作業手順\ldots{}\ldots{}作業:液体、香味素材、調味料を鍋に入れ、沸かす。弱火で30分程煮て、漉す。

\href{欠落アリ}{}

原注:クールブイヨンBとC\footnote{クールブイヨンBの白ワインを赤ワインに代え、香味素材としてにんじん400gを加える。鱒、鯉、マトロート用(原書pp.277-278)。}で調理した魚はクールブイヨン添えとして供する。つまり、少量の煮汁とクールブイヨンに用いた野菜を添える。野菜はよく火が通っていること。煮汁はしっかり煮詰め、提供直前に新鮮なバター少量を加えて仕上げる。
\end{recette}
\hypertarget{ux30afux30fcux30ebux30d6ux30a4ux30e8ux30f3ux306eux4f7fux3044ux65b9-ux539fux66f8-p.278}{%
\subsection{クールブイヨンの使い方 原書
p.278}\label{ux30afux30fcux30ebux30d6ux30a4ux30e8ux30f3ux306eux4f7fux3044ux65b9-ux539fux66f8-p.278}}

\begin{enumerate}
\def\labelenumi{\arabic{enumi}.}
\item
  加熱時間が30分以内の場合は、クールブイヨンは必ず事前に用意しておくこと。
\item
  加熱時間が30分を越える場合は、クールブイヨンの材料は冷たい状態のままで合わせせておく。香味素材はポワソニエールの網の下に入れる。
\item
  ごく少量のクールブイヨンでポシェ\footnote{原文 pochage à court
    mouillement『ル・ギード・キュリネール』では、この表現はテュルボタン(小型のテュルボ)、バルビュ、舌びらめ等の平たい魚をポシェする際に用いられる。本連載「舌びらめのボヌ・ファム」
    2011年3月号pp.110-111 参照。}する場合、材料は(白または赤ワインを含む場合も)魚を火にかける際に合わせる。クールブイヨンの量は魚の
  \(\frac{1}{3}\)
  の高さとし、加熱中ひんぱんに煮汁を魚にかけてやること\footnote{arroser
    アロゼ。}。この調理法の場合は通常、クールブイヨンは上で記したように
  \footnote{「クールブイヨンB」原注。}、提供直前に軽くバターを加えて仕上げ、魚に添える。
\item
  冷製にする場合は、必ずクールブイヨンに魚が浸った状態で冷ますこと。当然ながら、火にかけている時間は短かくなる\footnote{余熱で火が通るため。}。
\end{enumerate}

\hypertarget{ux539fux6ce8}{%
\subparagraph{【原注】}\label{ux539fux6ce8}}

いくつかの魚種の加熱時間は該当する項で示してある。

\hypertarget{ux9b5aux306eux8abfux7406ux6cd5}{%
\section{魚の調理法}\label{ux9b5aux306eux8abfux7406ux6cd5}}

魚料理は全て、下記のいずれかの調理法による。

\begin{enumerate}
\def\labelenumi{\arabic{enumi}.}
\item
  塩水(湯)またはクールブイヨン\footnote{court-bouillon直訳は「量の少ない煮汁」。魚の他、甲殻類、鶏などの白身肉、野菜などをポシェするのに用いる。とりわけ魚や鶏を丸ごとポシェする場合には、その名称のとおり、できるだけ少量でポシェする必要がある。また、ポシェに用いたクールブイヨンをベースにソースを作る場合が多い。}Bを用いたポシェ\ldots{}\ldots{}大きな魚丸ごと、および切り身。
\item
  ごく少量のクールブイヨンを用いたポシェ\ldots{}\ldots{}魚のフィレ、またはやや小さい魚。
\item
  ブレゼ\ldots{}\ldots{}もっぱら大きな魚。
\item
  オ・ブルー\footnote{比較的小さめの淡水魚に主として用いられる調理法。生きたままの魚の表面のぬめりをとらないように洗い、内臓を取り除いたらすぐに、塩とヴィネガーを加えたクールブイヨンで茹でる。冷製、温製どちらでも供する。原書p.281参照。}\ldots{}\ldots{}とりわけ\ruby{鱒}{ます}、鯉、ブロシェ\footnote{川かますの一種。本連載「ブロシェのクネル」2011年10月号
    pp.124-125 参照。}に合う。
\item
  揚げもの\ldots{}\ldots{}もっぱら小さい魚、切り身。
\item
  ムニエール\ldots{}\ldots{}揚げものにするのと同じ小さい魚、切り身。
\item
  グリエ\ldots{}\ldots{}小さい魚、および切り身。
\item
  グラタン\ldots{}\ldots{}小さい魚、切り身。
\end{enumerate}

\hypertarget{ux5869ux6c34ux6e6fux304aux3088ux3073ux30afux30fcux30ebux30d6ux30a4ux30e8ux30f3bux3092ux7528ux3044ux305fux52a0ux71b1ux8abfux7406}{%
\subsection{塩水(湯)およびクールブイヨンBを用いた加熱調理}\label{ux5869ux6c34ux6e6fux304aux3088ux3073ux30afux30fcux30ebux30d6ux30a4ux30e8ux30f3bux3092ux7528ux3044ux305fux52a0ux71b1ux8abfux7406}}

魚を丸ごと調理する場合は、魚に合ったポワソニエール\footnote{大きな魚を丸ごと煮るための細長い鍋。魚の形を崩さずに取り出せるよう、中に専用の網を敷いて使う。似たものに、舌びらめ等の平たい魚にぴったり合う菱形をしたテュルボティエールがある。いずれも、できるだけ少量の煮汁で魚を加熱できるように工夫されたもの。(図参照)}を用いる。魚を掃除し(テュルボは水にさらして血抜きをし)、ひれ等を切り落して形を整え、ポワソニエールの網に乗せる。魚種に応じて塩水または冷たいクールブイヨンをかぶるまで注ぐ。強火にかけて沸騰したらすぐにレンジの火の弱いところに鍋を移動させ、ポシェする。

切り身(薄すぎは絶対にいけない)の場合、沸騰した液体(塩湯またはクールブイヨン)に投入したらすぐにレンジの火の弱いところに鍋を移動させ、沸騰しない程度の温度でゆっくりと火を通す。

こうするのは、魚の身のエキスを閉じこめるためである。冷水から火にかけた場合にはエキスの大部分が流れ出してしまう。大きな魚丸ごとの場合にはこのやり方はしない。沸騰した液体に魚を投入すると身が収縮するので、大きな魚の場合は身が割れたり形が崩れたりするからだ。

塩湯あるいはクールブイヨンでポシェした魚は、ナフキンまたは専用の網に盛る。周囲をパセリで飾り、塩茹でしたじゃがいもと1種類または数種のソースを添えて供する。ガルニテュールがパセリのみの場合、魚の周囲にパセリを飾るのは客に料理を見せる\footnote{当時の宴席で主流だったロシア式サーヴィスでは、大きな銀盆に盛った料理をまず食客に見せてから、とり分けて給仕する。}直前にすること。どんな場合でも、ガルニテュールを添えたらクロッシュ\footnote{銀または陶製の保温用皿カバー。}は被せないこと。

\hypertarget{ux3054ux304fux5c11ux91cfux306eux30afux30fcux30ebux30d6ux30a4ux30e8ux30f3ux3092ux7528ux3044ux305fux30ddux30b7ux30a7-ux539fux66f8-pp.279-280}{%
\subsection{ごく少量のクールブイヨンを用いたポシェ 原書
pp.279-280}\label{ux3054ux304fux5c11ux91cfux306eux30afux30fcux30ebux30d6ux30a4ux30e8ux30f3ux3092ux7528ux3044ux305fux30ddux30b7ux30a7-ux539fux66f8-pp.279-280}}

この火入れの方法は主としてテュルボタン、バルビュ、舌びらめ、丸ごとおよびそれぞれの魚のフィレで用いる。バターを塗った天板あるいはソテ鍋に魚丸ごとあるいはそのフィレを置き、軽く塩をして、所要量の魚のフュメかマッシュルームの煮汁を注ぐ。フュメとマッシュルームの煮汁を合わせたものを用いる場合もある。蓋をして、中温のオーヴンに入れる。魚丸ごとの場合は時折煮汁をかけてやる。

魚(丸ごとでもフィレでも)に火が通ったら、注意して汁気をきり、皿に盛る。ガルニテュールを含む料理の場合、ガルニテュールを魚の周囲に盛り、ソースをかける\footnote{原文通りの順で訳したが、実際にはソースをかけてからガルニテュールを盛ったほうが良い場合もあるだろう。}。多くの場合、魚の煮汁を煮詰めてソースに加える。

\href{欠落アリ}{}

\hypertarget{ux9b5aux306eux30d6ux30ecux30bc17-ux539fux66f8-p.280}{%
\subsection[魚のブレゼ 原書 p.280]{\texorpdfstring{魚のブレゼ\footnote{ここではブレゼの語が限定的な意味で用いられていることに注意。牛のアロワイヨのような大きな塊肉のブレゼと同様の調理法、ということである。それは、香味素材を色づくまで炒めてから用いることや、主素材に豚背脂やトリュフをピケ針で差したり、豚背脂のシートで覆って加熱するという点によく表れている。ただし、これらは必須というわけではないため、事実上は「ごく少量のクールブイヨンを用いたポシェ」と区別がつきにくい。実際、モンタニェ『ラルース・ガストロノミーク』初版では、魚のブレゼについて「本来的な意味でのブレゼというよりは、ごく少量のクールブイヨンを用いたポシェである」と述べられている。逆に言えば、こんにち魚の加熱方法についてしばしば「ブレゼ」と呼ばれているものが、エスコフィエやモンタニェにおいては「少量のクールブイヨンを用いたポシェ」と表現されていたということである。}
原書
p.280}{魚のブレゼ 原書 p.280}}\label{ux9b5aux306eux30d6ux30ecux30bc17-ux539fux66f8-p.280}}

この調理法を用いるのは通常、丸ごとまたは筒切りにした鮭、大ぶりの鱒、テュルボ、テュルボタンのうち大きなもの、である。

場合によっては、魚の片面に、小さく切った豚背脂、トリュフ、コルニション、にんじん等をピケ針で差し込む。

香味素材等\footnote{原文 fonds de braisage フォン・ド・ブレザージュ
  (fonds de
  braiseフォン・ド・ブレーズ、とも)。通常は、厚い輪切りにしたにんじんと玉ねぎをバターか獣脂で色づくまで炒め、ブーケガルニ、下茹でした豚皮を合わせる(原書pp.394-395)。また、これを用いた煮汁のことも指す。}は肉料理のブレゼの場合と同じように用意するが、豚皮は用いない。提供方法に応じて、白または赤ワインと軽い魚のフュメ同量ずつを、魚の厚みの
\(\frac{3}{4}\) またはひたひたの高さまで注ぐ。厳密に肉断ち \footnote{カトリックの生活習慣として、四旬節(復活祭までの46日間)および週1
  回程度、肉類を食べないということが行なわれた(本連載2012年5月号「ソース・エスパニョル(4)」p.110、訳注1参照)。}のための仕立てにする場合を除いて、薄くスライスした豚背脂のシートを魚にかぶせる。加熱中\footnote{鍋を火にかけ、沸騰したら蓋をして中火のオーヴンに入れ、加熱する。}こまめに煮汁を魚にかけてやる\footnote{arroser
  アロゼ。}。また、完全には蓋をせず、加熱中に煮汁が煮詰まるようにしてやる。

ほぼ火が通ったら、鍋の蓋をとり、魚にかけた煮汁の水分をオーヴンの熱で蒸発させて表面につやを出す\footnote{glacer
  グラセ。}。魚を鍋から出して汁気をきり、皿に盛り保温しておく。

煮汁\footnote{原文 fonds de braisage (訳注2参照)。}を漉し、しばらく休ませたら浮き脂を取り除き、必要なら煮詰める。これを加えてソースを仕上げる。

魚のブレゼには通常、各ルセットに示してあるガルニテュールを添える。

\hypertarget{ux30aaux30d6ux30ebux30fc36-ux539fux66f8-pp.280-281}{%
\subsection[オ・ブルー 原書
pp.280-281]{\texorpdfstring{オ・ブルー\footnote{au bleu
  ヴィネガーを加えることで魚の表面のぬめりが青みがかることから。} 原書
pp.280-281}{オ・ブルー 原書 pp.280-281}}\label{ux30aaux30d6ux30ebux30fc36-ux539fux66f8-pp.280-281}}

オ・ブルーは鱒、鯉、ブロシェ\footnote{川かますの一種。}のみに用いられる特殊な調理法で、基本的なポイントは以下のとおり。

\begin{enumerate}
\def\labelenumi{\arabic{enumi}.}
\item
  必ず、生きた魚を使う。
\item
  魚の表面のぬめりをとらないように、なるべく手で触れずに、わたを抜く。鱗も引かない。
\item
  魚が大きい場合は、専用の網を敷いたポワソニーエルに入れ、「沸騰したヴィネガーをかける」。ヴィネガーは、通常のクールブイヨンに加える分量\footnote{以下の「クールブイヨンA」の分量比率を参照。}。次に、ヴィネガーを入れずに用意した温かい\footnote{原文
    tiède ぬるい、温かい。}クールブイヨンを注ぎ入れる。これは、なるべく身が割れないようにするためである。その後は通常どおり加熱する\footnote{レンジで沸騰させたらオーヴンに入れる。}。
\item
  小さい鱒の場合は、生きたままのものを手早く中抜きし、塩、ヴィネガーを加えただけの沸騰したクールブイヨンで煮る。
\item
  オ・ブルーは冷製、温製どちらの仕立てにしてもいい。実際の作り方の項で示してあるソースを添えて供する。
\end{enumerate}

\href{欠落アリ}{}

\hypertarget{ux30e0ux30cbux30a8ux30fcux30eb43-ux539fux66f8-p.282}{%
\subsection[ムニエール 原書 p.282]{\texorpdfstring{ムニエール\footnote{à
  la meunière 「粉挽き職人風」の意。} 原書
p.282}{ムニエール 原書 p.282}}\label{ux30e0ux30cbux30a8ux30fcux30eb43-ux539fux66f8-p.282}}

ムニエールは素晴らしい調理法だが、小型の魚と、大きな魚の場合は切り身にしか用いない。とはいえ、丁寧にやれば1.5
kg以下のテュルボタンはムニエールで調理できる。

魚丸ごと、あるいは切り身、フィレに味つけをして小麦粉をまぶし、バターを熱したフライパンで焼く。

魚が小さい場合は普通のバターでいいが、大きい場合は澄ましバターを使った方がいい。

魚の両面を焼き、程良く火が通ったら、予め熱しておいた皿に盛る。

飾り切りにした半割りのレモンを添えて、そのまま供することも可能である。ただし、このような提供方法の場合は本来の「ムニエール」と区別するために「黄金色に焼いた\footnote{doré
  (ドレ)
  一般的な色の表現として「黄金色」の意だが、ムニエールの場合、通常は大きい魚についてのみこの表現を用いる。}」と表現する。

「ムニエール」の場合には、焼き上がった魚に少量のレモン汁をふり、塩、こしょう少々で味を整える。粗みじん切りにして湯通ししたパセリを魚の表面に散らし、焦がしバターをかけてすぐに供する。湯通ししたパセリの水分に熱いバターが触れて泡がたつので、それが消えないうちに客に料理を見せるようにする。

%\href{✓原稿下準備なし}{} \href{訳と注釈\%2020180420進行中}{}
\href{未、原文対照チェック}{} \href{未、日本語表現校正}{}
\href{未、注釈チェク}{} \href{未、原稿最終校正}{}

\hypertarget{poissons}{%
\chapter{VI 魚料理 Poissons}\label{poissons}}

\hypertarget{serie-de-courts-bouillons-de-poisson}{%
\section{クールブイヨン}\label{serie-de-courts-bouillons-de-poisson}}

\begin{recette}
\hypertarget{ux30afux30fcux30ebux30d6ux30a4ux30e8ux30f3-a}{%
\subsubsection{クールブイヨン
A}\label{ux30afux30fcux30ebux30d6ux30a4ux30e8ux30f3-a}}

水5 L に対し、ヴィネガー2.5 dL、粗塩60 g、薄切りにしたにんじん600
gと玉ねぎ500 g、タイム1枝、ローリエの小さな葉2枚、パセリの茎100
g、粒こしょう20
g(こしょうを加えるのはクールブイヨンを漉す10分前)。材料を全て鍋に入れ、火にかけて1時間弱火で煮、漉す(原書
p.277)。

\hypertarget{ux30afux30fcux30ebux30d6ux30a4ux30e8ux30f3-b-4-ux9c52ux3046ux306aux304eux30d6ux30edux30b7ux30a7ux7b49-ux539fux66f8p.277}{%
\subsubsection[クールブイヨン B (鱒、うなぎ、ブロシェ等)
原書p.277]{\texorpdfstring{クールブイヨン B \footnote{クールブイヨンは用途に応じ、AからEまでの5種が挙げられている(原書pp.277-278)。}
(鱒、うなぎ、ブロシェ等)
原書p.277}{クールブイヨン B  (鱒、うなぎ、ブロシェ等) 原書p.277}}\label{ux30afux30fcux30ebux30d6ux30a4ux30e8ux30f3-b-4-ux9c52ux3046ux306aux304eux30d6ux30edux30b7ux30a7ux7b49-ux539fux66f8p.277}}

5 L 分の材料\ldots{}\ldots{}白ワイン2.5 L 。水2.5 L
。薄切りにした玉ねぎ600 g。パセリの茎80g
。タイムの小枝1本。ローリエの葉(小) \(\frac{1}{2}\) 枚。粗塩
60g。大粒のこしょう15 g(クールブイヨンを漉す10分前に加える)。

作業手順\ldots{}\ldots{}作業:液体、香味素材、調味料を鍋に入れ、沸かす。弱火で30分程煮て、漉す。

\href{欠落アリ}{}

原注:クールブイヨンBとC\footnote{クールブイヨンBの白ワインを赤ワインに代え、香味素材としてにんじん400gを加える。鱒、鯉、マトロート用(原書pp.277-278)。}で調理した魚はクールブイヨン添えとして供する。つまり、少量の煮汁とクールブイヨンに用いた野菜を添える。野菜はよく火が通っていること。煮汁はしっかり煮詰め、提供直前に新鮮なバター少量を加えて仕上げる。
\end{recette}
\hypertarget{ux30afux30fcux30ebux30d6ux30a4ux30e8ux30f3ux306eux4f7fux3044ux65b9-ux539fux66f8-p.278}{%
\subsection{クールブイヨンの使い方 原書
p.278}\label{ux30afux30fcux30ebux30d6ux30a4ux30e8ux30f3ux306eux4f7fux3044ux65b9-ux539fux66f8-p.278}}

\begin{enumerate}
\def\labelenumi{\arabic{enumi}.}
\item
  加熱時間が30分以内の場合は、クールブイヨンは必ず事前に用意しておくこと。
\item
  加熱時間が30分を越える場合は、クールブイヨンの材料は冷たい状態のままで合わせせておく。香味素材はポワソニエールの網の下に入れる。
\item
  ごく少量のクールブイヨンでポシェ\footnote{原文 pochage à court
    mouillement『ル・ギード・キュリネール』では、この表現はテュルボタン(小型のテュルボ)、バルビュ、舌びらめ等の平たい魚をポシェする際に用いられる。本連載「舌びらめのボヌ・ファム」
    2011年3月号pp.110-111 参照。}する場合、材料は(白または赤ワインを含む場合も)魚を火にかける際に合わせる。クールブイヨンの量は魚の
  \(\frac{1}{3}\)
  の高さとし、加熱中ひんぱんに煮汁を魚にかけてやること\footnote{arroser
    アロゼ。}。この調理法の場合は通常、クールブイヨンは上で記したように
  \footnote{「クールブイヨンB」原注。}、提供直前に軽くバターを加えて仕上げ、魚に添える。
\item
  冷製にする場合は、必ずクールブイヨンに魚が浸った状態で冷ますこと。当然ながら、火にかけている時間は短かくなる\footnote{余熱で火が通るため。}。
\end{enumerate}

\hypertarget{ux539fux6ce8}{%
\subparagraph{【原注】}\label{ux539fux6ce8}}

いくつかの魚種の加熱時間は該当する項で示してある。

\hypertarget{ux9b5aux306eux8abfux7406ux6cd5}{%
\section{魚の調理法}\label{ux9b5aux306eux8abfux7406ux6cd5}}

魚料理は全て、下記のいずれかの調理法による。

\begin{enumerate}
\def\labelenumi{\arabic{enumi}.}
\item
  塩水(湯)またはクールブイヨン\footnote{court-bouillon直訳は「量の少ない煮汁」。魚の他、甲殻類、鶏などの白身肉、野菜などをポシェするのに用いる。とりわけ魚や鶏を丸ごとポシェする場合には、その名称のとおり、できるだけ少量でポシェする必要がある。また、ポシェに用いたクールブイヨンをベースにソースを作る場合が多い。}Bを用いたポシェ\ldots{}\ldots{}大きな魚丸ごと、および切り身。
\item
  ごく少量のクールブイヨンを用いたポシェ\ldots{}\ldots{}魚のフィレ、またはやや小さい魚。
\item
  ブレゼ\ldots{}\ldots{}もっぱら大きな魚。
\item
  オ・ブルー\footnote{比較的小さめの淡水魚に主として用いられる調理法。生きたままの魚の表面のぬめりをとらないように洗い、内臓を取り除いたらすぐに、塩とヴィネガーを加えたクールブイヨンで茹でる。冷製、温製どちらでも供する。原書p.281参照。}\ldots{}\ldots{}とりわけ\ruby{鱒}{ます}、鯉、ブロシェ\footnote{川かますの一種。本連載「ブロシェのクネル」2011年10月号
    pp.124-125 参照。}に合う。
\item
  揚げもの\ldots{}\ldots{}もっぱら小さい魚、切り身。
\item
  ムニエール\ldots{}\ldots{}揚げものにするのと同じ小さい魚、切り身。
\item
  グリエ\ldots{}\ldots{}小さい魚、および切り身。
\item
  グラタン\ldots{}\ldots{}小さい魚、切り身。
\end{enumerate}

\hypertarget{ux5869ux6c34ux6e6fux304aux3088ux3073ux30afux30fcux30ebux30d6ux30a4ux30e8ux30f3bux3092ux7528ux3044ux305fux52a0ux71b1ux8abfux7406}{%
\subsection{塩水(湯)およびクールブイヨンBを用いた加熱調理}\label{ux5869ux6c34ux6e6fux304aux3088ux3073ux30afux30fcux30ebux30d6ux30a4ux30e8ux30f3bux3092ux7528ux3044ux305fux52a0ux71b1ux8abfux7406}}

魚を丸ごと調理する場合は、魚に合ったポワソニエール\footnote{大きな魚を丸ごと煮るための細長い鍋。魚の形を崩さずに取り出せるよう、中に専用の網を敷いて使う。似たものに、舌びらめ等の平たい魚にぴったり合う菱形をしたテュルボティエールがある。いずれも、できるだけ少量の煮汁で魚を加熱できるように工夫されたもの。(図参照)}を用いる。魚を掃除し(テュルボは水にさらして血抜きをし)、ひれ等を切り落して形を整え、ポワソニエールの網に乗せる。魚種に応じて塩水または冷たいクールブイヨンをかぶるまで注ぐ。強火にかけて沸騰したらすぐにレンジの火の弱いところに鍋を移動させ、ポシェする。

切り身(薄すぎは絶対にいけない)の場合、沸騰した液体(塩湯またはクールブイヨン)に投入したらすぐにレンジの火の弱いところに鍋を移動させ、沸騰しない程度の温度でゆっくりと火を通す。

こうするのは、魚の身のエキスを閉じこめるためである。冷水から火にかけた場合にはエキスの大部分が流れ出してしまう。大きな魚丸ごとの場合にはこのやり方はしない。沸騰した液体に魚を投入すると身が収縮するので、大きな魚の場合は身が割れたり形が崩れたりするからだ。

塩湯あるいはクールブイヨンでポシェした魚は、ナフキンまたは専用の網に盛る。周囲をパセリで飾り、塩茹でしたじゃがいもと1種類または数種のソースを添えて供する。ガルニテュールがパセリのみの場合、魚の周囲にパセリを飾るのは客に料理を見せる\footnote{当時の宴席で主流だったロシア式サーヴィスでは、大きな銀盆に盛った料理をまず食客に見せてから、とり分けて給仕する。}直前にすること。どんな場合でも、ガルニテュールを添えたらクロッシュ\footnote{銀または陶製の保温用皿カバー。}は被せないこと。

\hypertarget{ux3054ux304fux5c11ux91cfux306eux30afux30fcux30ebux30d6ux30a4ux30e8ux30f3ux3092ux7528ux3044ux305fux30ddux30b7ux30a7-ux539fux66f8-pp.279-280}{%
\subsection{ごく少量のクールブイヨンを用いたポシェ 原書
pp.279-280}\label{ux3054ux304fux5c11ux91cfux306eux30afux30fcux30ebux30d6ux30a4ux30e8ux30f3ux3092ux7528ux3044ux305fux30ddux30b7ux30a7-ux539fux66f8-pp.279-280}}

この火入れの方法は主としてテュルボタン、バルビュ、舌びらめ、丸ごとおよびそれぞれの魚のフィレで用いる。バターを塗った天板あるいはソテ鍋に魚丸ごとあるいはそのフィレを置き、軽く塩をして、所要量の魚のフュメかマッシュルームの煮汁を注ぐ。フュメとマッシュルームの煮汁を合わせたものを用いる場合もある。蓋をして、中温のオーヴンに入れる。魚丸ごとの場合は時折煮汁をかけてやる。

魚(丸ごとでもフィレでも)に火が通ったら、注意して汁気をきり、皿に盛る。ガルニテュールを含む料理の場合、ガルニテュールを魚の周囲に盛り、ソースをかける\footnote{原文通りの順で訳したが、実際にはソースをかけてからガルニテュールを盛ったほうが良い場合もあるだろう。}。多くの場合、魚の煮汁を煮詰めてソースに加える。

\href{欠落アリ}{}

\hypertarget{ux9b5aux306eux30d6ux30ecux30bc17-ux539fux66f8-p.280}{%
\subsection[魚のブレゼ 原書 p.280]{\texorpdfstring{魚のブレゼ\footnote{ここではブレゼの語が限定的な意味で用いられていることに注意。牛のアロワイヨのような大きな塊肉のブレゼと同様の調理法、ということである。それは、香味素材を色づくまで炒めてから用いることや、主素材に豚背脂やトリュフをピケ針で差したり、豚背脂のシートで覆って加熱するという点によく表れている。ただし、これらは必須というわけではないため、事実上は「ごく少量のクールブイヨンを用いたポシェ」と区別がつきにくい。実際、モンタニェ『ラルース・ガストロノミーク』初版では、魚のブレゼについて「本来的な意味でのブレゼというよりは、ごく少量のクールブイヨンを用いたポシェである」と述べられている。逆に言えば、こんにち魚の加熱方法についてしばしば「ブレゼ」と呼ばれているものが、エスコフィエやモンタニェにおいては「少量のクールブイヨンを用いたポシェ」と表現されていたということである。}
原書
p.280}{魚のブレゼ 原書 p.280}}\label{ux9b5aux306eux30d6ux30ecux30bc17-ux539fux66f8-p.280}}

この調理法を用いるのは通常、丸ごとまたは筒切りにした鮭、大ぶりの鱒、テュルボ、テュルボタンのうち大きなもの、である。

場合によっては、魚の片面に、小さく切った豚背脂、トリュフ、コルニション、にんじん等をピケ針で差し込む。

香味素材等\footnote{原文 fonds de braisage フォン・ド・ブレザージュ
  (fonds de
  braiseフォン・ド・ブレーズ、とも)。通常は、厚い輪切りにしたにんじんと玉ねぎをバターか獣脂で色づくまで炒め、ブーケガルニ、下茹でした豚皮を合わせる(原書pp.394-395)。また、これを用いた煮汁のことも指す。}は肉料理のブレゼの場合と同じように用意するが、豚皮は用いない。提供方法に応じて、白または赤ワインと軽い魚のフュメ同量ずつを、魚の厚みの
\(\frac{3}{4}\) またはひたひたの高さまで注ぐ。厳密に肉断ち \footnote{カトリックの生活習慣として、四旬節(復活祭までの46日間)および週1
  回程度、肉類を食べないということが行なわれた(本連載2012年5月号「ソース・エスパニョル(4)」p.110、訳注1参照)。}のための仕立てにする場合を除いて、薄くスライスした豚背脂のシートを魚にかぶせる。加熱中\footnote{鍋を火にかけ、沸騰したら蓋をして中火のオーヴンに入れ、加熱する。}こまめに煮汁を魚にかけてやる\footnote{arroser
  アロゼ。}。また、完全には蓋をせず、加熱中に煮汁が煮詰まるようにしてやる。

ほぼ火が通ったら、鍋の蓋をとり、魚にかけた煮汁の水分をオーヴンの熱で蒸発させて表面につやを出す\footnote{glacer
  グラセ。}。魚を鍋から出して汁気をきり、皿に盛り保温しておく。

煮汁\footnote{原文 fonds de braisage (訳注2参照)。}を漉し、しばらく休ませたら浮き脂を取り除き、必要なら煮詰める。これを加えてソースを仕上げる。

魚のブレゼには通常、各ルセットに示してあるガルニテュールを添える。

\hypertarget{ux30aaux30d6ux30ebux30fc36-ux539fux66f8-pp.280-281}{%
\subsection[オ・ブルー 原書
pp.280-281]{\texorpdfstring{オ・ブルー\footnote{au bleu
  ヴィネガーを加えることで魚の表面のぬめりが青みがかることから。} 原書
pp.280-281}{オ・ブルー 原書 pp.280-281}}\label{ux30aaux30d6ux30ebux30fc36-ux539fux66f8-pp.280-281}}

オ・ブルーは鱒、鯉、ブロシェ\footnote{川かますの一種。}のみに用いられる特殊な調理法で、基本的なポイントは以下のとおり。

\begin{enumerate}
\def\labelenumi{\arabic{enumi}.}
\item
  必ず、生きた魚を使う。
\item
  魚の表面のぬめりをとらないように、なるべく手で触れずに、わたを抜く。鱗も引かない。
\item
  魚が大きい場合は、専用の網を敷いたポワソニーエルに入れ、「沸騰したヴィネガーをかける」。ヴィネガーは、通常のクールブイヨンに加える分量\footnote{以下の「クールブイヨンA」の分量比率を参照。}。次に、ヴィネガーを入れずに用意した温かい\footnote{原文
    tiède ぬるい、温かい。}クールブイヨンを注ぎ入れる。これは、なるべく身が割れないようにするためである。その後は通常どおり加熱する\footnote{レンジで沸騰させたらオーヴンに入れる。}。
\item
  小さい鱒の場合は、生きたままのものを手早く中抜きし、塩、ヴィネガーを加えただけの沸騰したクールブイヨンで煮る。
\item
  オ・ブルーは冷製、温製どちらの仕立てにしてもいい。実際の作り方の項で示してあるソースを添えて供する。
\end{enumerate}

\href{欠落アリ}{}

\hypertarget{ux30e0ux30cbux30a8ux30fcux30eb43-ux539fux66f8-p.282}{%
\subsection[ムニエール 原書 p.282]{\texorpdfstring{ムニエール\footnote{à
  la meunière 「粉挽き職人風」の意。} 原書
p.282}{ムニエール 原書 p.282}}\label{ux30e0ux30cbux30a8ux30fcux30eb43-ux539fux66f8-p.282}}

ムニエールは素晴らしい調理法だが、小型の魚と、大きな魚の場合は切り身にしか用いない。とはいえ、丁寧にやれば1.5
kg以下のテュルボタンはムニエールで調理できる。

魚丸ごと、あるいは切り身、フィレに味つけをして小麦粉をまぶし、バターを熱したフライパンで焼く。

魚が小さい場合は普通のバターでいいが、大きい場合は澄ましバターを使った方がいい。

魚の両面を焼き、程良く火が通ったら、予め熱しておいた皿に盛る。

飾り切りにした半割りのレモンを添えて、そのまま供することも可能である。ただし、このような提供方法の場合は本来の「ムニエール」と区別するために「黄金色に焼いた\footnote{doré
  (ドレ)
  一般的な色の表現として「黄金色」の意だが、ムニエールの場合、通常は大きい魚についてのみこの表現を用いる。}」と表現する。

「ムニエール」の場合には、焼き上がった魚に少量のレモン汁をふり、塩、こしょう少々で味を整える。粗みじん切りにして湯通ししたパセリを魚の表面に散らし、焦がしバターをかけてすぐに供する。湯通ししたパセリの水分に熱いバターが触れて泡がたつので、それが消えないうちに客に料理を見せるようにする。

%\href{✓原稿下準備なし}{} \href{訳と注釈\%2020180420進行中}{}
\href{未、原文対照チェック}{} \href{未、日本語表現校正}{}
\href{未、注釈チェク}{} \href{未、原稿最終校正}{}

\hypertarget{poissons}{%
\chapter{VI 魚料理 Poissons}\label{poissons}}

\hypertarget{serie-de-courts-bouillons-de-poisson}{%
\section{クールブイヨン}\label{serie-de-courts-bouillons-de-poisson}}

\begin{recette}
\hypertarget{ux30afux30fcux30ebux30d6ux30a4ux30e8ux30f3-a}{%
\subsubsection{クールブイヨン
A}\label{ux30afux30fcux30ebux30d6ux30a4ux30e8ux30f3-a}}

水5 L に対し、ヴィネガー2.5 dL、粗塩60 g、薄切りにしたにんじん600
gと玉ねぎ500 g、タイム1枝、ローリエの小さな葉2枚、パセリの茎100
g、粒こしょう20
g(こしょうを加えるのはクールブイヨンを漉す10分前)。材料を全て鍋に入れ、火にかけて1時間弱火で煮、漉す(原書
p.277)。

\hypertarget{ux30afux30fcux30ebux30d6ux30a4ux30e8ux30f3-b-4-ux9c52ux3046ux306aux304eux30d6ux30edux30b7ux30a7ux7b49-ux539fux66f8p.277}{%
\subsubsection[クールブイヨン B (鱒、うなぎ、ブロシェ等)
原書p.277]{\texorpdfstring{クールブイヨン B \footnote{クールブイヨンは用途に応じ、AからEまでの5種が挙げられている(原書pp.277-278)。}
(鱒、うなぎ、ブロシェ等)
原書p.277}{クールブイヨン B  (鱒、うなぎ、ブロシェ等) 原書p.277}}\label{ux30afux30fcux30ebux30d6ux30a4ux30e8ux30f3-b-4-ux9c52ux3046ux306aux304eux30d6ux30edux30b7ux30a7ux7b49-ux539fux66f8p.277}}

5 L 分の材料\ldots{}\ldots{}白ワイン2.5 L 。水2.5 L
。薄切りにした玉ねぎ600 g。パセリの茎80g
。タイムの小枝1本。ローリエの葉(小) \(\frac{1}{2}\) 枚。粗塩
60g。大粒のこしょう15 g(クールブイヨンを漉す10分前に加える)。

作業手順\ldots{}\ldots{}作業:液体、香味素材、調味料を鍋に入れ、沸かす。弱火で30分程煮て、漉す。

\href{欠落アリ}{}

原注:クールブイヨンBとC\footnote{クールブイヨンBの白ワインを赤ワインに代え、香味素材としてにんじん400gを加える。鱒、鯉、マトロート用(原書pp.277-278)。}で調理した魚はクールブイヨン添えとして供する。つまり、少量の煮汁とクールブイヨンに用いた野菜を添える。野菜はよく火が通っていること。煮汁はしっかり煮詰め、提供直前に新鮮なバター少量を加えて仕上げる。
\end{recette}
\hypertarget{ux30afux30fcux30ebux30d6ux30a4ux30e8ux30f3ux306eux4f7fux3044ux65b9-ux539fux66f8-p.278}{%
\subsection{クールブイヨンの使い方 原書
p.278}\label{ux30afux30fcux30ebux30d6ux30a4ux30e8ux30f3ux306eux4f7fux3044ux65b9-ux539fux66f8-p.278}}

\begin{enumerate}
\def\labelenumi{\arabic{enumi}.}
\item
  加熱時間が30分以内の場合は、クールブイヨンは必ず事前に用意しておくこと。
\item
  加熱時間が30分を越える場合は、クールブイヨンの材料は冷たい状態のままで合わせせておく。香味素材はポワソニエールの網の下に入れる。
\item
  ごく少量のクールブイヨンでポシェ\footnote{原文 pochage à court
    mouillement『ル・ギード・キュリネール』では、この表現はテュルボタン(小型のテュルボ)、バルビュ、舌びらめ等の平たい魚をポシェする際に用いられる。本連載「舌びらめのボヌ・ファム」
    2011年3月号pp.110-111 参照。}する場合、材料は(白または赤ワインを含む場合も)魚を火にかける際に合わせる。クールブイヨンの量は魚の
  \(\frac{1}{3}\)
  の高さとし、加熱中ひんぱんに煮汁を魚にかけてやること\footnote{arroser
    アロゼ。}。この調理法の場合は通常、クールブイヨンは上で記したように
  \footnote{「クールブイヨンB」原注。}、提供直前に軽くバターを加えて仕上げ、魚に添える。
\item
  冷製にする場合は、必ずクールブイヨンに魚が浸った状態で冷ますこと。当然ながら、火にかけている時間は短かくなる\footnote{余熱で火が通るため。}。
\end{enumerate}

\hypertarget{ux539fux6ce8}{%
\subparagraph{【原注】}\label{ux539fux6ce8}}

いくつかの魚種の加熱時間は該当する項で示してある。

\hypertarget{ux9b5aux306eux8abfux7406ux6cd5}{%
\section{魚の調理法}\label{ux9b5aux306eux8abfux7406ux6cd5}}

魚料理は全て、下記のいずれかの調理法による。

\begin{enumerate}
\def\labelenumi{\arabic{enumi}.}
\item
  塩水(湯)またはクールブイヨン\footnote{court-bouillon直訳は「量の少ない煮汁」。魚の他、甲殻類、鶏などの白身肉、野菜などをポシェするのに用いる。とりわけ魚や鶏を丸ごとポシェする場合には、その名称のとおり、できるだけ少量でポシェする必要がある。また、ポシェに用いたクールブイヨンをベースにソースを作る場合が多い。}Bを用いたポシェ\ldots{}\ldots{}大きな魚丸ごと、および切り身。
\item
  ごく少量のクールブイヨンを用いたポシェ\ldots{}\ldots{}魚のフィレ、またはやや小さい魚。
\item
  ブレゼ\ldots{}\ldots{}もっぱら大きな魚。
\item
  オ・ブルー\footnote{比較的小さめの淡水魚に主として用いられる調理法。生きたままの魚の表面のぬめりをとらないように洗い、内臓を取り除いたらすぐに、塩とヴィネガーを加えたクールブイヨンで茹でる。冷製、温製どちらでも供する。原書p.281参照。}\ldots{}\ldots{}とりわけ\ruby{鱒}{ます}、鯉、ブロシェ\footnote{川かますの一種。本連載「ブロシェのクネル」2011年10月号
    pp.124-125 参照。}に合う。
\item
  揚げもの\ldots{}\ldots{}もっぱら小さい魚、切り身。
\item
  ムニエール\ldots{}\ldots{}揚げものにするのと同じ小さい魚、切り身。
\item
  グリエ\ldots{}\ldots{}小さい魚、および切り身。
\item
  グラタン\ldots{}\ldots{}小さい魚、切り身。
\end{enumerate}

\hypertarget{ux5869ux6c34ux6e6fux304aux3088ux3073ux30afux30fcux30ebux30d6ux30a4ux30e8ux30f3bux3092ux7528ux3044ux305fux52a0ux71b1ux8abfux7406}{%
\subsection{塩水(湯)およびクールブイヨンBを用いた加熱調理}\label{ux5869ux6c34ux6e6fux304aux3088ux3073ux30afux30fcux30ebux30d6ux30a4ux30e8ux30f3bux3092ux7528ux3044ux305fux52a0ux71b1ux8abfux7406}}

魚を丸ごと調理する場合は、魚に合ったポワソニエール\footnote{大きな魚を丸ごと煮るための細長い鍋。魚の形を崩さずに取り出せるよう、中に専用の網を敷いて使う。似たものに、舌びらめ等の平たい魚にぴったり合う菱形をしたテュルボティエールがある。いずれも、できるだけ少量の煮汁で魚を加熱できるように工夫されたもの。(図参照)}を用いる。魚を掃除し(テュルボは水にさらして血抜きをし)、ひれ等を切り落して形を整え、ポワソニエールの網に乗せる。魚種に応じて塩水または冷たいクールブイヨンをかぶるまで注ぐ。強火にかけて沸騰したらすぐにレンジの火の弱いところに鍋を移動させ、ポシェする。

切り身(薄すぎは絶対にいけない)の場合、沸騰した液体(塩湯またはクールブイヨン)に投入したらすぐにレンジの火の弱いところに鍋を移動させ、沸騰しない程度の温度でゆっくりと火を通す。

こうするのは、魚の身のエキスを閉じこめるためである。冷水から火にかけた場合にはエキスの大部分が流れ出してしまう。大きな魚丸ごとの場合にはこのやり方はしない。沸騰した液体に魚を投入すると身が収縮するので、大きな魚の場合は身が割れたり形が崩れたりするからだ。

塩湯あるいはクールブイヨンでポシェした魚は、ナフキンまたは専用の網に盛る。周囲をパセリで飾り、塩茹でしたじゃがいもと1種類または数種のソースを添えて供する。ガルニテュールがパセリのみの場合、魚の周囲にパセリを飾るのは客に料理を見せる\footnote{当時の宴席で主流だったロシア式サーヴィスでは、大きな銀盆に盛った料理をまず食客に見せてから、とり分けて給仕する。}直前にすること。どんな場合でも、ガルニテュールを添えたらクロッシュ\footnote{銀または陶製の保温用皿カバー。}は被せないこと。

\hypertarget{ux3054ux304fux5c11ux91cfux306eux30afux30fcux30ebux30d6ux30a4ux30e8ux30f3ux3092ux7528ux3044ux305fux30ddux30b7ux30a7-ux539fux66f8-pp.279-280}{%
\subsection{ごく少量のクールブイヨンを用いたポシェ 原書
pp.279-280}\label{ux3054ux304fux5c11ux91cfux306eux30afux30fcux30ebux30d6ux30a4ux30e8ux30f3ux3092ux7528ux3044ux305fux30ddux30b7ux30a7-ux539fux66f8-pp.279-280}}

この火入れの方法は主としてテュルボタン、バルビュ、舌びらめ、丸ごとおよびそれぞれの魚のフィレで用いる。バターを塗った天板あるいはソテ鍋に魚丸ごとあるいはそのフィレを置き、軽く塩をして、所要量の魚のフュメかマッシュルームの煮汁を注ぐ。フュメとマッシュルームの煮汁を合わせたものを用いる場合もある。蓋をして、中温のオーヴンに入れる。魚丸ごとの場合は時折煮汁をかけてやる。

魚(丸ごとでもフィレでも)に火が通ったら、注意して汁気をきり、皿に盛る。ガルニテュールを含む料理の場合、ガルニテュールを魚の周囲に盛り、ソースをかける\footnote{原文通りの順で訳したが、実際にはソースをかけてからガルニテュールを盛ったほうが良い場合もあるだろう。}。多くの場合、魚の煮汁を煮詰めてソースに加える。

\href{欠落アリ}{}

\hypertarget{ux9b5aux306eux30d6ux30ecux30bc17-ux539fux66f8-p.280}{%
\subsection[魚のブレゼ 原書 p.280]{\texorpdfstring{魚のブレゼ\footnote{ここではブレゼの語が限定的な意味で用いられていることに注意。牛のアロワイヨのような大きな塊肉のブレゼと同様の調理法、ということである。それは、香味素材を色づくまで炒めてから用いることや、主素材に豚背脂やトリュフをピケ針で差したり、豚背脂のシートで覆って加熱するという点によく表れている。ただし、これらは必須というわけではないため、事実上は「ごく少量のクールブイヨンを用いたポシェ」と区別がつきにくい。実際、モンタニェ『ラルース・ガストロノミーク』初版では、魚のブレゼについて「本来的な意味でのブレゼというよりは、ごく少量のクールブイヨンを用いたポシェである」と述べられている。逆に言えば、こんにち魚の加熱方法についてしばしば「ブレゼ」と呼ばれているものが、エスコフィエやモンタニェにおいては「少量のクールブイヨンを用いたポシェ」と表現されていたということである。}
原書
p.280}{魚のブレゼ 原書 p.280}}\label{ux9b5aux306eux30d6ux30ecux30bc17-ux539fux66f8-p.280}}

この調理法を用いるのは通常、丸ごとまたは筒切りにした鮭、大ぶりの鱒、テュルボ、テュルボタンのうち大きなもの、である。

場合によっては、魚の片面に、小さく切った豚背脂、トリュフ、コルニション、にんじん等をピケ針で差し込む。

香味素材等\footnote{原文 fonds de braisage フォン・ド・ブレザージュ
  (fonds de
  braiseフォン・ド・ブレーズ、とも)。通常は、厚い輪切りにしたにんじんと玉ねぎをバターか獣脂で色づくまで炒め、ブーケガルニ、下茹でした豚皮を合わせる(原書pp.394-395)。また、これを用いた煮汁のことも指す。}は肉料理のブレゼの場合と同じように用意するが、豚皮は用いない。提供方法に応じて、白または赤ワインと軽い魚のフュメ同量ずつを、魚の厚みの
\(\frac{3}{4}\) またはひたひたの高さまで注ぐ。厳密に肉断ち \footnote{カトリックの生活習慣として、四旬節(復活祭までの46日間)および週1
  回程度、肉類を食べないということが行なわれた(本連載2012年5月号「ソース・エスパニョル(4)」p.110、訳注1参照)。}のための仕立てにする場合を除いて、薄くスライスした豚背脂のシートを魚にかぶせる。加熱中\footnote{鍋を火にかけ、沸騰したら蓋をして中火のオーヴンに入れ、加熱する。}こまめに煮汁を魚にかけてやる\footnote{arroser
  アロゼ。}。また、完全には蓋をせず、加熱中に煮汁が煮詰まるようにしてやる。

ほぼ火が通ったら、鍋の蓋をとり、魚にかけた煮汁の水分をオーヴンの熱で蒸発させて表面につやを出す\footnote{glacer
  グラセ。}。魚を鍋から出して汁気をきり、皿に盛り保温しておく。

煮汁\footnote{原文 fonds de braisage (訳注2参照)。}を漉し、しばらく休ませたら浮き脂を取り除き、必要なら煮詰める。これを加えてソースを仕上げる。

魚のブレゼには通常、各ルセットに示してあるガルニテュールを添える。

\hypertarget{ux30aaux30d6ux30ebux30fc36-ux539fux66f8-pp.280-281}{%
\subsection[オ・ブルー 原書
pp.280-281]{\texorpdfstring{オ・ブルー\footnote{au bleu
  ヴィネガーを加えることで魚の表面のぬめりが青みがかることから。} 原書
pp.280-281}{オ・ブルー 原書 pp.280-281}}\label{ux30aaux30d6ux30ebux30fc36-ux539fux66f8-pp.280-281}}

オ・ブルーは鱒、鯉、ブロシェ\footnote{川かますの一種。}のみに用いられる特殊な調理法で、基本的なポイントは以下のとおり。

\begin{enumerate}
\def\labelenumi{\arabic{enumi}.}
\item
  必ず、生きた魚を使う。
\item
  魚の表面のぬめりをとらないように、なるべく手で触れずに、わたを抜く。鱗も引かない。
\item
  魚が大きい場合は、専用の網を敷いたポワソニーエルに入れ、「沸騰したヴィネガーをかける」。ヴィネガーは、通常のクールブイヨンに加える分量\footnote{以下の「クールブイヨンA」の分量比率を参照。}。次に、ヴィネガーを入れずに用意した温かい\footnote{原文
    tiède ぬるい、温かい。}クールブイヨンを注ぎ入れる。これは、なるべく身が割れないようにするためである。その後は通常どおり加熱する\footnote{レンジで沸騰させたらオーヴンに入れる。}。
\item
  小さい鱒の場合は、生きたままのものを手早く中抜きし、塩、ヴィネガーを加えただけの沸騰したクールブイヨンで煮る。
\item
  オ・ブルーは冷製、温製どちらの仕立てにしてもいい。実際の作り方の項で示してあるソースを添えて供する。
\end{enumerate}

\href{欠落アリ}{}

\hypertarget{ux30e0ux30cbux30a8ux30fcux30eb43-ux539fux66f8-p.282}{%
\subsection[ムニエール 原書 p.282]{\texorpdfstring{ムニエール\footnote{à
  la meunière 「粉挽き職人風」の意。} 原書
p.282}{ムニエール 原書 p.282}}\label{ux30e0ux30cbux30a8ux30fcux30eb43-ux539fux66f8-p.282}}

ムニエールは素晴らしい調理法だが、小型の魚と、大きな魚の場合は切り身にしか用いない。とはいえ、丁寧にやれば1.5
kg以下のテュルボタンはムニエールで調理できる。

魚丸ごと、あるいは切り身、フィレに味つけをして小麦粉をまぶし、バターを熱したフライパンで焼く。

魚が小さい場合は普通のバターでいいが、大きい場合は澄ましバターを使った方がいい。

魚の両面を焼き、程良く火が通ったら、予め熱しておいた皿に盛る。

飾り切りにした半割りのレモンを添えて、そのまま供することも可能である。ただし、このような提供方法の場合は本来の「ムニエール」と区別するために「黄金色に焼いた\footnote{doré
  (ドレ)
  一般的な色の表現として「黄金色」の意だが、ムニエールの場合、通常は大きい魚についてのみこの表現を用いる。}」と表現する。

「ムニエール」の場合には、焼き上がった魚に少量のレモン汁をふり、塩、こしょう少々で味を整える。粗みじん切りにして湯通ししたパセリを魚の表面に散らし、焦がしバターをかけてすぐに供する。湯通ししたパセリの水分に熱いバターが触れて泡がたつので、それが消えないうちに客に料理を見せるようにする。



%%% Chapitre VII. Relevés et Entrées
%% エスコフィエ『料理の手引き』全注解
% 五島 学


\href{✓原稿下準備なし}{} \href{訳と注釈\%2020180424古い原稿のコピペ}{}
\href{未、原文対照チェック}{} \href{未、日本語表現校正}{}
\href{未、注釈チェク}{} \href{未、原稿最終校正}{}

\hypertarget{vii.ux8089ux6599ux7406-relevuxe9s-et-entruxe9es}{%
\chapter{VII. 肉料理 Relevés et
Entrées}\label{vii.ux8089ux6599ux7406-relevuxe9s-et-entruxe9es}}

\begin{center}
\frsec{ブレゼ、ポワレ、ソテ、ポシェの基礎、下茹で\\グラタン、グリル、揚げものの調理理論}
\end{center}
\normalsize
\vspace{1\zw}
\frsec{Princie généraux de la Conduite de Braisée --- des Poêlés --- des Sautés --- et des Pochés --- Blanchissages\\\vspace{.5ex}Théorie des Gratins, des Grillades et des Fritures}
\normalsize

\hypertarget{les-braises}{%
\section{ブレゼ}\label{les-braises}}

いろいろな調理法の中でも、ブレゼはとりわけコストがかかり、しかもきわめて高度な技術が要求される。この調理法を習得するには時間をかけて、注意深く実践を重ねていくしかない。細心の注意を払って調理しなければならないのは勿論のこと、他の調理法もそうだが、主素材となる肉の品質は非常に重要だ。美味しいブレゼを作るにはさらに、加熱の際に上質のフォンを用い、香味素材
(フォン・ド・ブレーズ)もきちんと仕込みをしておく必要がある。

\hypertarget{ux30d6ux30ecux30bcux306bux7528ux3044ux308bux8089}{%
\subsection{ブレゼに用いる肉}\label{ux30d6ux30ecux30bcux306bux7528ux3044ux308bux8089}}

ブレゼの通常の調理法としてここで述べるのは、牛肉、羊肉を用いる。仔牛、乳飲み仔羊、家禽のブレゼについては後述。ブレゼの素材は、ロティの場合と違って、若い畜肉でなくていい。牛の場合は3〜6才、羊は1〜2才のものが最良だ。上記の年齢を過ぎたものはいい肉質を期待できない。そのような肉を使う場合は加熱時間をかなり長くとる必要が出てくるし、それでも大抵は、筋っぽくてぱさついた仕上りになってしまう。実際問題として、料理に老齢の畜肉を用いるのは、コンソメと各種フォンをとる場合のみと考えること。

\hypertarget{ux30e9ux30ebux30c7}{%
\subsection{ラルデ}\label{ux30e9ux30ebux30c7}}

アロワイヨや背肉の塊は、ペルシエつまり脂肪の筋が入ったものであれば良質で柔らかい。牛や羊のもも肉の場合は事情が異なる。牛、羊のもも肉はそれ自体が脂をあまり含んでいないため、長時間加熱するとぱさついてしまう。脂を補うために、約1cm角の棒状に切った豚背脂を、繊維方向に沿って肉の内部に刺し込む(ラルデ)。背脂は予{[}あらかじ{]}め、こしょう、ナツメグ、その他の香辛料で味つけし、パセリのみじん切りを振ってから、コニャック適量で2時間マリネしておくこと。

\hypertarget{ux30deux30eaux30cd}{%
\subsection{マリネ}\label{ux30deux30eaux30cd}}

背脂を刺す場合も、そうでない場合も、ワインと香味素材で素材を数時間マリネする。ワインは煮汁として、香味素材はフォン・ド・ブレーズとして用いることになる。マリネする前に、素材を塩、こしょう、その他の香辛料で味つけする。肉を転がし、調味料がよく浸み込むようにする。素材が丁度入る大きさの容器の底に香味素材を敷いて、その上に肉を置き、さらに香味素材で覆う。煮汁に用いるワインをひたひたの高さまで注ぐ。一般的には、白でも赤でも普通のワインを用いる。分量は肉1kgあたり3㎗。5〜6時間マリネする。途中、何度か肉を裏返す。

\hypertarget{ux9999ux5473ux7d20ux6750ux30d5ux30a9ux30f3ux30c9ux30d6ux30ecux30fcux30ba}{%
\subsection{香味素材(フォン・ド・ブレーズ)}\label{ux9999ux5473ux7d20ux6750ux30d5ux30a9ux30f3ux30c9ux30d6ux30ecux30fcux30ba}}

にんじんと玉ねぎを厚めの輪切りにし、バターまたはグレス・ド・マルミートを用いて強火でこんがり炒める(野菜の量は肉1kgあたり各60g)。ブーケガルニ
(にんにくを入れること)。肉1kgあたり50gの生豚皮(下茹でしておく)。

\hypertarget{ux30eaux30bdux30ecux30d6ux30ecux30bc}{%
\subsection{リソレ、ブレゼ}\label{ux30eaux30bdux30ecux30d6ux30ecux30bc}}

いい具合にマリネした状態になったら、肉を取り出して網にあげ、30分間水気をきる。さらに布などで水気を拭き取る。丁度いい大きさの、厚手の片手鍋またはブレゼ鍋に、不純物を取り除いたグレス・ド・マルミートを熱する。素材を鍋に入れ、強火で表面をまんべんなくこんがりと焼く(リソレ)。こうすることで、肉に一種の鎧をまとわせ、肉汁があまり早く外に流れ出ないようにする。そうでないと、ブレゼではなくブイイになってしまう。素材が大きければ大きい程、表面の焼き固めた層がそれだけ丈夫でなくてはいけないから、素材の表面を焼く時間は長くなることになる。

表面を焼き固めたら素材を鍋から取り出す。脂の少ない肉の場合は、豚背脂のシートでくるみ、紐で縛る。牛のアロワイヨや背肉の場合、この作業は必要ない。そのままの状態で、脂の層に覆われているからだ。

素材の大きさにぴったり合うサイズの鍋に、香味野菜(マリネに用いたもの)、豚皮、ブーケガルニを入れ、その上に素材を置く。マリネに用いたワインを注ぎ、強火にかけてシロップ状になるまで煮詰める。上質の茶色いフォンを素材がかぶるまで注ぎ、沸騰させる。鍋に蓋をして中火のオーヴンに入れ、安定した微沸騰の状態を保つようにする。

細いブリデ針を深く刺してみて、穴から血が上ってこなくなるまで火を通す。この時点で、ブレゼの第1段階は終了。以下が第2段階だが、その調理メカニズムについては後述する。

煮汁については、目的に応じて以下のどちらかの方法をとる。

煮汁を澄んだままにしておきたい場合は、肉を充分な大きさのきれいな別鍋に移し、布漉しした煮汁を注ぐ。この鍋をオーヴンに入れ、こまめに煮汁をかけてやりながら加熱を仕上げる。

通常の「とろみを付けたジュ」と同様に、コーンスターチでとろみを付ける。ブレゼのソースとして仕上げる場合は、煮汁を半量まで煮詰める。その⅔量のソース・エスパニョルと⅓量のトマトピュレまたは同等量の生のトマトを加え、もとの煮汁と同じ量にする。これを、上で述べたように別鍋に移しておいた肉にかけてオーヴンに入れ、こまめにソースをかけてやりながら加熱を仕上げる。

肉にナイフを刺してみて、抵抗なく入っていくようであれば、ほどよく火が通っているので、肉をソースから取り出す。ソースは布で漉し、脂がすっかり表面に浮いてくるまで10分程休ませる。浮いてきた脂は徹底的に取り除く。最後に、ソースが濃いようなら上質のフォン少量を加え、薄過ぎるようなら煮詰めて、ソースを仕上げる。

\hypertarget{ux30d6ux30ecux30bcux7b2cux6bb5ux968e-ux306eux8abfux7406ux30e1ux30abux30cbux30baux30e0-ux539fux66f8-p.396}{%
\subsection{ブレゼ第2段階 の調理メカニズム 原書
p.396}\label{ux30d6ux30ecux30bcux7b2cux6bb5ux968e-ux306eux8abfux7406ux30e1ux30abux30cbux30baux30e0-ux539fux66f8-p.396}}

作業の第1段階のところで述べたように、肉の塊が大きければ大きい程、表面をしっかり焼き固める必要がある。表面を焼き固めるのは、外に逃げ出そうとする肉汁を内側に押し返すことと、表面に一種の鎧を作ることが目的だ。この鎧は、加熱が進むにつれ周囲から中心に向かって厚いものになっていく。

鍋の液体が熱せられると、肉の筋繊維が締まり肉汁は中心に向かうが、やがて熱が中心まで届くと、そこに圧縮された肉汁は分解して余計な水分が分離される。水分は蒸気となり、筋繊維を膨らませてほぐすのだ。

つまり、第1段階では明らかに、肉の塊の中心に向かって肉汁が濃縮されていく。

第2段階ではこれとは逆向きの現象が起こる。

肉の塊の中心部に集まった肉汁の水分が蒸発するだけの温度に達すると、筋繊維がほぐれ始める。肉汁から分離した水分が、逃げ場がないために蒸気圧をかけるのだ。筋繊維はかなりの圧力の影響を受けるわけだが、この圧力は第1段階とは逆に、肉の中心から周囲に向かうことになる。

だから、加熱が進み、肉の内部の圧力が高まるとその結果、筋繊維はゆるんでいく。肉汁の水分は、外側の焼き固めた層に少しずつ達してその筋繊維をゆるめ、内部の肉汁が流れ出す通り道が出来る。肉汁はソースと混ざり、同時に、毛管現象によってソースも肉の内部に浸み込む。ここが、ブレゼの調理でもっとも注意を払わなければならないところだ。作業は最終段階で、煮汁はかなり煮詰まってきて、肉を覆ってはいない。
何も覆うものがないから肉はとてもすぐに乾燥してしまうから、こまめに煮汁をかけてやり、肉を裏返す。肉の組織が常にソースを吸い込んだ状態にしてやるのだ。こうすることで、他の調理法とは一線を画すブレゼの特徴、柔らかくてとろけるような仕上りとなる。

\hypertarget{ux7167ux308aux3092ux3064ux3051ux308bux30b0ux30e9ux30bb}{%
\subsection{照りをつける(グラセ)}\label{ux7167ux308aux3092ux3064ux3051ux308bux30b0ux30e9ux30bb}}

ブレゼを塊のまま客にプレゼンテーションする場合には絶対に照りをつける必要があるが、切り分けてから供する場合には必須ではないし、不要とも言える。

照りをつける場合、ほど良く火が通ったらすぐに鍋から肉を取り出し、平鍋に移してオーヴンの入口近くに入れる。煮汁かジュ、ソースを軽くかけると、オーヴンの熱で煮詰まって、表面に薄い膜が出来る。この作業を、肉が艶のある層ですっかり覆われるまで繰り返す。オーヴンから出したら皿に盛り、供するまでクロッシュをかぶせておく。

\hypertarget{ux6ce8ux610fux4e8bux9805}{%
\subsection{注意事項}\label{ux6ce8ux610fux4e8bux9805}}

例えばブフ・アラモードのように野菜を添える場合には、バターで色良く炒めてから、手順の第2段階で肉と一緒に煮るか、あるいは、煮汁の一部を取り分けて肉とは別に煮る。

一番いいのは最初のやり方だが、緻密な盛り付けをするには向かない。だから、臨機応変にどちらがいいか判断する必要がある。ブレゼの作り方について、一般的に行なわれてはいるけれども絶対に間違っていることが2つある。ひとつ目は、香味野菜(フォン・ド・ブレーズ)をパンセすることだ。

香味野菜をパンセするというのは、予{[}あらかじ{]}め色良く炒めておりた野菜の上に、焼き色を付けた肉を置くのではなく、ブレゼ鍋の底に生のままの野菜を並べ、その上に肉を置くが、多くの場合は肉にも焼き色を付けない。溶かしたグレス・ド・マルミートを少量かけてやり、野菜が鍋底に軽く焦げ付くまで加熱する。----
厳密に言えば、上手にやるならこの方法も許容されるだろう。けれど、片面しか焼き色を付けていない野菜では両面に焼き色を付けた場合ほどは風味は出ないし、その上、加熱時間が長過ぎるとほとんど黒焦げになりかねず、苦味が出てソースの風味を損ねてしまう。

実際のところ、このパンセという作業は、煮汁に用いるフォンを事前に仕込んでおくこともせず、そのフォンの材料をブレゼそれ自体と一緒に煮ていた大昔の料理のうわべだけ真似たものに過ぎない。

昔のブレゼの作り方は素晴しいけれど、とてもコストのかかるものだった。というのも、厚くスライスした生ハムと仔牛のもも肉にのせて主素材を同時にブレゼしていたのだ。コストの問題でこのやり方でブレゼを作らなくなって久しいのだが、本質的なところは無視して、形式的な手順だけが慣習として残ったのだ。しかも、昔はフォンの材料に肉を用いていたのを各種の獣骨で代用するようになってしまったのだから、誤りとしてますますひどいものになってしまった。

そこで、第2の誤りである。

よく知られているように、ブレゼに最もよく使われるのは仔牛の骨だが、それでさえ完全に煮出すのには10〜12時間かかる。まず5〜6時間煮てフォンをとった骨に、さらに液体を注いで6時間煮た方が、5〜6時間煮ただけのフォンよりも多くのグラス・ド・ヴィアンドが得られるのがその証拠だ。2番のフォンで作ったグラス・ド・ヴィアンドは風味は劣る一方で、ゼラチン質が多いのは事実だ。ブレゼに用いるフォンとしては、このゼラチン質は風味の要素に負けず劣らず意味がある。ゼラチン質によってブレゼはなめらかでとろけるような口当たりになる。これは他のものでは代用できないし、ソースの出来を決めるものなのだ。

ブレゼに生の骨を加えたとしても、ブレゼの加熱時間は最大でも4〜5時間以上には出来ないのだから、肉に火が通った時点で骨は表面しか煮えていないわけだ。骨に含まれているもののほとんどは煮出されないままになってしまう。----
つまり、ブレゼに骨を加えても全く意味がないのだ。

この間違った方法には、また別の欠点もある。素材を煮るのに大量の液体を用いなければならないということだ。ブレゼというのはソースがしっかり煮詰められてこくがないと完全ではない、というのは誰もが認めるところだろう。煮汁が多ければ多い程、ソースは薄味になるし、結果的に煮汁で肉を洗うようなことになってしまう。

だからこそ、既に述べたように、ブレゼでは素材の大きさにぴったり合った容量の鍋を用いるべきなのだ。肉は始めすっかり煮汁に浸っていなければならないわけだから、鍋の大きさが丁度良ければそれだけ煮汁の絶対量は少なくなり、素材から溶け出すものも加わってフォンに一層こくが出るのだ。

\hypertarget{ux767dux8eabux8089ux306eux30d6ux30ecux30bc}{%
\section{白身肉のブレゼ}\label{ux767dux8eabux8089ux306eux30d6ux30ecux30bc}}

当今の白身肉のブレゼは、本来的な意味ではブレゼとは呼べない。赤身肉のブレゼの作り方は2段階で作業を行なうのが特徴だが、その1段階で火入れを止めてしまうからである。

昔の料理ではブレゼの2段階の作業が行なわれていなかったのは事実だ。が、仔牛等の大きな塊肉はスプーンで切れるくらいまでよく火を通すことが多かったのだ。---
現代ではこうした調理は行なわれなくなってしまい、名称だけが残ったわけだ。

白身肉のブレゼは次のとおり。仔牛の背肉、鞍下肉、腰肉、もも肉。フリカンド、リ・ド・ヴォ、若い七面鳥、肥鶏。頻度は少ないが、仔羊のドゥーブルやバロン、鞍下肉でもよく作られる。

上記の肉すべて同じ作り方にするが、加熱時間は肉の大きさによって変わるので注意。

ブレゼに用いる香味素材は赤身肉のブレゼと同じだが、野菜は色づかないようバターで軽く炒めるだけにすること。煮汁には必ず白いフォンを用いる。

\hypertarget{ux767dux8eabux8089ux306eux30d6ux30ecux30bcux306eux4f5cux696dux624bux9806}{%
\subsection{白身肉のブレゼの作業手順}\label{ux767dux8eabux8089ux306eux30d6ux30ecux30bcux306eux4f5cux696dux624bux9806}}

リ・ド・ヴォは調理前に必ず下茹でするので別だが、ブレゼする白身肉や家禽は表面を全て、軽く色づく程度まで焼き固めてもいい。その方がパサつきにくくなる。とはいえ、この作業は省いても良い。

次に、底に香味素材を敷いた鍋に肉を入れる。鍋の大きさは肉がちょうど入るくらいで、蓋をした際に肉が蓋に当たらない程度の深さのものを用いる。

この鍋に仔牛のフォン少量を注ぎ、蓋をして弱火で煮詰める。再び同量のフォンを注いで同じように煮詰める。それから肉の半分の高さまで煮汁を注ぐ。沸騰したら弱火のオーヴンに入れる。弱火といっても、煮汁が微沸騰の状態を保つ程度の温度であること。

加熱中は肉の表面が乾かないように、こまめに煮汁を肉にかけてやること。フォンにはゼラチン質が多く含まれているので、表面に塗膜のようなものが出来て、熱による肉汁の蒸散を防いでくれる。肉の表面をごく軽く焼き固めただけでは保護層が充分に出来ていないからだ。

そのために、最終的に煮汁を注ぐ前に少量のフォンを煮詰めておいたわけだ。肉を鍋に入れてそのまま煮汁を注いだとして、上で述べたような膜が出来るほどフォンが濃くならないだろうから、肉はひどくパサついてしまうだろう。

白身肉のブレゼで火の通り具合をみるには、ブリデ針を深く刺す。穴から透明の肉汁が上がってくるようになったら程良く火が通っている。透明の肉汁が上がってくるというのは、肉の中心まで火が通って血が分解された証拠なのだ。

この火の通し加減という点で、赤身肉のブレゼと白身肉のブレゼは大きく異なるわけだ。実際のところ、白身肉のブレゼの火入れ加減はほとんどロティに近い。だから家禽とごく若い畜肉の脂がのっていて柔らかいものしか使わないのだ。というのも、この料理では、ロティと同じくらいの程良い火入れ加減を少しでも越えたらたちまちおいしくなくなってしまうからだ。

白身肉のブレゼは通常、照りをつけてやる。とりわけ、細く切った豚背脂をピケ針で刺している場合には照りをつけてやったほうがいい。豚背脂を刺すのは昔と比べて減ったが、まだこの方法を採る者も多い。

\hypertarget{les-poches}{%
\section{ポシェ}\label{les-poches}}

こんな表現が成り立つならの話だが、ポシェとは「沸騰させないで作るブイイ」とするのが最も正しい定義だろう。

「ポシェ」という用語は広い意味では、何らかの液体を用いて弱火でゆっくり火を通すことを指す。液体の量が多いか少ないかは問題とならない。だから、大きなテュルボや鮭を丸ごとクールブイヨンで煮るのはもとより、舌びらめの切り身を少量の魚のフュメで煮る場合や、温製のムースやムスリーヌ、クネル、クレーム、ロワイヤル等々の加熱についても、「ポシェ」と呼ばれる。

このようにポシェする対象は多岐にわたるので、それぞれの加熱時間は大きく異なる。けれども、全てに共通して絶対守るべき原則がある。ポシェする液体は決して沸騰させない、沸騰寸前の温度にするということだ。

もうひとつ大事なことだが、魚や鶏を丸ごとポシェする際は液体が冷たい状態で火にかけ、手早く所定の温度まで上げるようにする。ごく少量の液体で魚や鶏の切り身をポシェする場合も同様にしていい。これに対し、他のポシェの場合は事前に所定の温度にしておいた液体に投入する手順となる。

\hypertarget{preparation-des-volailles-a-pocher}{%
\subsection{鶏のポシェの下ごしらえ}\label{preparation-des-volailles-a-pocher}}

\frsecb{Préparation des Volailles à pocher}

鶏は下処理の後、指示があれば詰め物をし、ブリデ針を用い糸で手羽と脚を畳み込むように縛る。

細かく切ったトリュフ、ハム、ラング・エカルラートを鶏の胸や脚に刺す場合には、半割りにしたレモンで擦ってから、沸騰した白いフォンに数分間浸してやる。

こうすることで皮が締まり、トリュフ等を刺す作業がやり易くなる。

\hypertarget{pochage-de-la-volaille}{%
\subsection{鶏のポシェ}\label{pochage-de-la-volaille}}

\frsecb{Pochage de la Volaille}

詰め物をしたり、トリュフ等を刺すのは必要な時だけだが、どんな場合でも豚背脂のシートで包んでやる。丁度いい大きさの鍋に素材を入れ、事前に仕込んでおいた白いフォンをかぶるまで注ぐ。

火にかけて沸騰したらアクを引き、蓋をして所定の温度、つまり目で見てほとんどわからない程度の微沸騰の状態を保つようにする。熱がだんだん伝わって鶏に火を通すにはこれで充分な温度だ。

明らかな沸騰状態にしてしまうといろいろ不都合が起きる。とりわけ (1)
水分の蒸発が激しすぎて煮汁が煮詰まり、澄んだ状態を保てなくなる。(2)
詰め物をしている場合は特に、皮が弾けやすくなる。

鶏のポシェで火の通り加減を見るには、ドラムスティックに近い腿の裏側を刺してみる。完全に透明な汁が上がってくれば程良く火が通っている。

\hypertarget{nota-sur-les-pochages-de-volaille}{%
\subsection{注意事項}\label{nota-sur-les-pochages-de-volaille}}

\begin{itemize}
\item
  鶏をポシェするのに丁度いい大きさの鍋を使うべき理由は\ldots{}\ldots{}

  \begin{enumerate}
  \def\labelenumi{\arabic{enumi}.}
  \item
    加熱中、素材が常にフォンに浸っていなければならない。
  \item
    煮汁そのものをソースに用いるので、煮汁の全体量が少なければ少ない程、鶏から流れる肉汁が薄まりにくくなる。結果として、ソースの風味が良くなる。
  \end{enumerate}
\item
  {[}その他{]}\footnote{原書ではここに項目名はないが、箇条書きを見やすくするために訳者が補った。}  

  \begin{enumerate}
  \def\labelenumi{\arabic{enumi}.}
  \item
    ポシェに使う白いフォンは必ず事前に仕込んでおく。充分に澄んだフォンを用いること。
  \item
    もしもフォンをとる材料と鶏を一緒に火にかけたら、フォンの材料がどんなにたくさんでも、いい結果は得られないだろう。理由は、鶏の加熱時間は最大でも1時間〜1時間半であるのに対して、フォンの材料から香りと栄養素を充分に引き出すには最低6時間はかかる。その結果、単なるお湯に近い液体で鶏のポシェが完了し、その煮汁から作ったソースも味気ないものになってしまうだろう。
  \end{enumerate}
\end{itemize}

\hypertarget{les-poeles}{%
\section{ポワレ}\label{les-poeles}}

\frsec{Les Poêlés}

ポワレは事実上ロティの一種と言える。ロティもポワレも目指す火入れは同じだ。

ここで記すポワレは次のような古い調理法を単純化したものだ。古い調理法では、あらかじめ素材の表面に焼き色をつけ、たっぷりのマティニョンで覆ってから豚背脂のシートやバターを塗った紙で包み、オーヴンまたは串を刺して直火で、溶かしバターをかけながら焼いていた。

火が通ったらすぐに包みを外して脂をきる。マティニョンをブレゼ鍋または片手鍋に移し、マデイラと煮詰めたフォンを加える。

マティニョンの香味がフォンに移ったら、フォンを漉し、提供直前に浮き脂を取り除いて仕上げていた。

家禽を丸ごと調理する仕立てのいくつかについては、今なおこの古い方法で作る価値がある。

\hypertarget{ux30ddux30efux30ecux306eux4f5cux696dux624bux9806}{%
\subsection{ポワレの作業手順}\label{ux30ddux30efux30ecux306eux4f5cux696dux624bux9806}}

素材に対して余裕のある大きさの厚手の深鍋の底にマティニョンを敷きつめる
(マティニョンについては「ガルニテュールの仕込み」参照)。

畜肉あるいは家禽にしっかり味付けをし、野菜の上に置く。溶かしバターをたっぷりかけてやる。鍋に蓋をして、やや高温のオーヴンに入れる。

そうして、小まめにバターをかけながら、蓋をした状態でじっくり火を入れる。

火が通ったら鍋の蓋を取り、オーヴンの熱で素材に焼き色をつける。皿に移し、クロッシュをかぶせて保温しておく。

野菜(焦げていないこと)に充分な量の煮詰めた澄んだフォンを注ぐ。弱火で10
分間煮てから布で漉し、丁寧に浮き脂を取り除く。これをソース容器に入れ、主素材の周囲にガルニテュールを盛って供する。

\hypertarget{ux30ddux30efux30ecux306bux3064ux3044ux3066ux306eux6ce8ux610f}{%
\subsection{ポワレについての注意}\label{ux30ddux30efux30ecux306bux3064ux3044ux3066ux306eux6ce8ux610f}}

\begin{enumerate}
\def\labelenumi{\arabic{enumi}.}
\item
  ポワレは火入れに液体を用いないのが重要ポイント。液体を用いたら白身肉のブレゼと同じ風味になってしまう。ポワレは火入れにバターしか用いない。ただし、雉、ペルドロ、うずら等の猟鳥のポワレでは概ね火が通った時点で少量のコニャックを注いでフランベする。
\item
  フォンを注ぐ前に野菜から脂を取り除かないことも大事なポイントだ。
\end{enumerate}

実際、ポワレに用いたバターには主素材と野菜の風味が溶け込んでいるわけだ。この風味を取り出すためには、フォンを注いで最低10分以上バターと接しているようにする必要がある。その後であれば、バターを取り除いてもフォンの香味を損なうことはない。

\hypertarget{ux7279ux6b8aux306aux30ddux30efux30ec-ux30abux30b9ux30edux30fcux30ebux30b3ux30b3ux30c3ux30c8}{%
\subsection{特殊なポワレ ----
カスロール、ココット}\label{ux7279ux6b8aux306aux30ddux30efux30ec-ux30abux30b9ux30edux30fcux30ebux30b3ux30b3ux30c3ux30c8}}

カスロール、ココットという調理は畜肉、家禽、ジビエを専用の陶製の鍋で火入れし、鍋ごと供するが、これはまさにポワレそのものと言える。

一般的に、「カスロール」は野菜を加えずバターだけを用いて素材に火を通す。素材に程良く火が通ったら主素材を取り出し、鍋に仔牛のフォン少量を注ぐ。数分間沸かしてから、浮いている余分なバターを取り除く。主素材を鍋に戻し、保温しておくが、沸騰させないこと。

「ココット」も同様に調理するが、マッシュルーム、アーティチョークの萼の基底部、小玉ねぎ、にんじん、かぶ等の野菜を加えて調理する。野菜はそれぞれの性質に応じて形を整え、バターで炒めて半ば火を通しておくこと。

出来るだけ新野菜を使うようにすること。野菜は主素材の周りに入れるが、主素材と同時に火入れが終了するタイミングで加えること。「ココット」に用いる陶製の鍋はある程度使い込んだものの方がいい。重曹や洗剤は用いずに水できれいに洗って手入れすること。

新品の鍋を用いなければならない場合、軽く沸かした湯をいっぱいに注ぎ、12
時間以上は微沸騰の状態を保つようにしてやる。その後、水気を拭き取る。さらに、冷水をいっぱいに注ぎしばらく放置してから使用すること。

\hypertarget{ux30bdux30c6-les-sautes}{%
\section{ソテ \{les-sautes\}}\label{ux30bdux30c6-les-sautes}}

ソテと呼ばれる調理法は水を使わないで火入れをするのが特徴だ。バター、植物油や精製した獣脂だけを用いる。

ソテに用いるのは、捌いた家禽やジビエ、あるいはソテに適するようにカットした畜肉である。

ソテに用いる素材は全て、強く熱した油脂で表面を焼き固める。表面に層を作り、肉汁を外に流れ出させずに内部に留めておくのが目的だ。この作業はとりわけ牛や羊のような赤身肉の場合に行なう。

家禽とジビエのソテは、素材に焼き色をつけたらソテ鍋に蓋をしてレンジで、あるいは蓋をせずオーヴンに入れてロティールと同様に焼き脂をかけながら火入れを仕上げる。

次に、素材を鍋から取り出してデグラセする。主素材を鍋に戻してソースあるいは付合せとからめる場合はごく短時間、つまりソースの風味がなじむ程度にとどめる。

トゥルヌド、ノワゼート、コトレート、フィレ、アントルコートのような赤身肉のソテでは、少量の澄ましバターで表面を焼き固め、そのままレンジで加熱を行なう。

表面を焼き固める際には、素材が小さく、薄く切ったものであればそれだけ強火にする。

焼いていない生のままの面に血が滲み出てきたら裏返す。先に焼いた面にピンク色の肉汁が出てきたら程良く火が通っている。

ソテ鍋から肉を取り出し、脂を捨ててから、ソースの一部となるワイン等の液体を鍋に注いで沸騰させ、鍋底についた肉汁を溶かし出す。こうしてデグラセした液体にソースを加える
----
場合によってはその逆で、別途用意しておいたソースやガルニテュールに加えることもある。仕立てとしての「ソテ」ではデグラセを必ず行なうこと。

ソテ鍋は素材の大きさにぴったり合ったものを使用する。大き過ぎると、肉が接していない部分が高温になり、デグラセが上手く出来なくなってしまう。デグラセは、肉を焼いた際に流れ出て固形化した肉汁を液体で溶かし出し、それによってソースがおいしくなるわけだから、デグラセが上手く出来ないとソースがおいしくなくなってしまうのだ。

仔牛、仔羊のような白身肉のソテは、まず表面を焼き固めてから弱火で火を通す。

白身肉の他の調理法と同様、しっかりと火を通すこと。

「ソテ」という名称は、ソテとブレゼ両方の特徴を兼ね備えた料理にも使われる。これは実際のところ「ラグー」と呼ぶのがふさわしいものだ。

こうした料理には牛、仔牛、仔羊、ジビエ等が用いられる。本書ではエストゥファード、グーラーシュ、仔牛のソテ、仔羊のソテ、カルボナード、ナヴァラン、シヴェ等の名称でまとめてある。

調理の第1段階では通常のソテと同様に小さめに切った肉に焼き色を付ける。第2段階はソースやガルニテュールと合わせて時間をかけて火を通すという点でブレゼと良く似ている。

\hypertarget{ux4e0bux8339ux3067}{%
\section{下茹で}\label{ux4e0bux8339ux3067}}

\hypertarget{ux30b0ux30e9ux30bfux30f3}{%
\section{グラタン}\label{ux30b0ux30e9ux30bfux30f3}}

\hypertarget{ux30b0ux30eaux30eb}{%
\section{グリル}\label{ux30b0ux30eaux30eb}}

グリルにおける調理上の働きとして重要なのは「凝縮」である。

グリルで重要なのは肉汁であり、殆どの場合、肉汁を内部に凝縮させることをまず目指すべきだ。

グリルとは要するに直火で焼くロティと同じであり、人類が料理ということを始めた遥か遠い起源に遡れるだろう。原始人の堅い頭に最初に生じたこのグリエというアイデアは、「もっと美味しいものを食べたい」という本能的な欲求から生まれた進化であり、食べ物を加熱調理するのに用いられた初めての方法なのだ。

やがて、その当然の帰結としてグリルから串焼きのロティが発生した。その頃には既に人類は本能ではなく知性で考えるようになっており、理屈によって結果を推論し、実地から結論を導き出すようになっていた。こうして、料理は進歩への道を歩みはじめたのだ。

\hypertarget{ux30b0ux30eaux30ebux306eux71b1ux6e90}{%
\subsection{グリルの熱源}\label{ux30b0ux30eaux30ebux306eux71b1ux6e90}}

もっとも一般的に用いられており、明らかに最良のものと言える燃料は熾火、または小さめの木炭である。どんな種類の燃料でも、重要なのは煙を出さないということだ。火力を強めるために風を送る場合は、その風で煙が外に流れ去るようにしてやる。

ましてや、あまりないことだが、自然と火がくすぶってしまい人工的に風を送ってやらねばならない時でも、煙が出てはいけない。熱源以外の物が燃えたり、炭に脂が落ちて煙が出てしまったら、人工的に風を送ろうと、強い風が吹こうと、きちんと煙を追い出せない限りは、どうしようもなく不味いグリルになってしまうからだ。

とはいえ、他の種類の熱源をグリルに用いても構わない。本書はこの点で絶対を主張はしない。反対に、適切に使うならばどんな熱源でも良いと言える。

\hypertarget{ux706bux5e8a}{%
\subsection{火床}\label{ux706bux5e8a}}

火床あるいはグリル台の造りも重要だ。グリルする素材の性質や大きさはもとより、状況によって火力を強めたり弱めたり自在に出来なければならない。

だから火床は炉の中央に平らに配する。ただし火力の強弱をつける必要に応じて厚みには変化をつけられるようにする。また、風が当たる側はやや高くして、熱がまんべんなく均等に行きわたるようにする。

焼き網は必ず前もって火床に据え、素材をのせる時には充分に熱くなっておくようにする。網を熱くしておかないと素材が貼り付き、裏返す際に素材を壊してしまうことになる。

\hypertarget{ux30b0ux30eaux30ebux306eux5206ux985e}{%
\subsection{グリルの分類}\label{ux30b0ux30eaux30ebux306eux5206ux985e}}

グリルは4種に分類され、それぞれ注意すべき点が異なる。

\begin{itemize}
\tightlist
\item
  赤身肉のグリル(牛、羊、ジビエ)
\item
  白身肉のグリル(仔牛、仔羊、家禽)
\item
  魚のグリル
\item
  パン粉衣をつけたグリル。パン粉のみをまぶす場合と、イギリス風パン粉衣を用いる場合がある。
\end{itemize}

\hypertarget{ux8d64ux8eabux8089ux306eux30b0ux30eaux30eb}{%
\subsection{赤身肉のグリル}\label{ux8d64ux8eabux8089ux306eux30b0ux30eaux30eb}}

グリルの作業は何よりもまず、各素材に適切な加熱温度を見きわめることから始まる。

素材が大きければ大きい程、肉汁が多ければ多い程、強火で表面をしっかり焼き固める必要がある。

この「リソレ」の役割と利点については「ブレゼ」のページで既に述べたが、グリルに関してもう一度おさらいしておこう。

牛や羊のような赤身肉を大きく切ったものをグリルする場合、上質の素材で肉汁の豊富なものであればなおのこと、しっかりした層が出来るよう表面を焼き固める必要がある。

内部の肉汁が多ければそれだけ、表面の焼き固めた層に向かう圧力は強くなる。肉汁が熱されるにつれてこの圧力は強くなる。

素材の内部にゆっくり熱が伝わるように火力を上手く調節していれば、次のような調理メカニズムとなる。

肉の火に接している側は、焼き網を通った熱によって線維が収縮し、肉の内部に熱が伝わっていく。熱は層をなすように肉の内部に広がり、肉汁を逆流させる。しまいには肉汁が肉の反対側に逹し、火にあたっていない生のままの面に滲み出てくる。このタイミングで肉を裏返し、焼いていない面について同じプロセスを行なう。肉汁が上に向かって逆流して最初に焼いた面でいったん止まり、そこから血の雫が浮かび上がってきたら、程良く火が通っている。

素材が大きい場合、表面をしっかり焼き固めたらすぐに火を弱め、熱がゆっくりと内部に伝わるようにする。同じ火の強さのまま焼き続けたら、焼き固めた表面がすぐに焦げてしまい、熱が肉の内部に伝わるのを邪魔する結果となる。外側は黒焦げなのに内部はまるで生という状態になってしまう。

あまり厚みのない肉をグリルする場合は、強火で表面を焼き固め、数分間そのまま焼けば程良く火が通る。火を弱める必要はない。例)
ランプやシャトブリヤンに程良く火を通すには、まず表面を強火で焼き固めて肉汁が流れ出ないようにしてから、弱めの火加減にしてやり、熱がゆっくり伝わって中まで火が通るようにする。

トゥルヌドやフィレミニョン、ノワゼート、コトレートのような小さいカットの肉の場合は、必ず表面を強火で焼き固め、そのままの火の強さで焼き上げる。厚みがないために熱が中まですぐに伝わるからである。

\hypertarget{ux8d64ux8eabux8089ux306eux30b0ux30eaux30ebux306eux969bux306bux6c17ux3092ux3064ux3051ux308bux3079ux304dux3053ux3068}{%
\subsection{赤身肉のグリルの際に気をつけるべきこと}\label{ux8d64ux8eabux8089ux306eux30b0ux30eaux30ebux306eux969bux306bux6c17ux3092ux3064ux3051ux308bux3079ux304dux3053ux3068}}

肉を焼き網にのせる前に、澄ましバターを刷毛でまんべんなく塗っておく。焼いている間も同様に澄ましバターを小まめに塗り、火が当たっている部分が乾かないようにする。

肉を裏返すにはパレットナイフか、出来たらグリル用のトングを使う。肉用フォーク等で刺して裏返すのは避けるべきだ。肉を刺せば肉汁の流れ出る通り道が出来るわけだから、それまで気を配ってきたこと全てをわざわざ台無しにしてしまうことになる。

\hypertarget{ux7a0bux826fux3044ux706bux306eux901aux308aux5177ux5408}{%
\subsection{程良い火の通り具合}\label{ux7a0bux826fux3044ux706bux306eux901aux308aux5177ux5408}}

赤身肉の場合、火の通り具合は指で表面に触れてみて押すと抵抗を感じる程度が、内部まで熱が伝わって程良く火が通っている目安となる。反対に、指で押しても全然抵抗を感じないようなら、熱がまだ内部まで伝わっていない。肉の焼き固めた表面にピンク色の肉汁がわずかに浮び上ってくる状態がいちばん確かな目印だ。

\hypertarget{ux767dux8eabux8089ux306eux30b0ux30eaux30eb}{%
\subsection{白身肉のグリル}\label{ux767dux8eabux8089ux306eux30b0ux30eaux30eb}}

赤身肉の場合は必ず表面を強火で焼き固めるが、白身肉の場合その必要はまったくない。仔牛や仔羊のような白身肉は肉汁がアルブミンの状態でしか存在しておらず、焼く際に凝縮させるよう気を配る必要がないからだ。

白身肉のグリルは弱めの火加減で、肉に火を通しながら同時に表面にきれいな焼き色がつくようにする。焼いている間、小まめにバターをかけてやり、表面が乾かないようにする。

浸み出てくる肉汁がすっかり透明になったら、程良く火が通っている。

\hypertarget{ux9b5aux306eux30b0ux30eaux30eb}{%
\subsection{魚のグリル}\label{ux9b5aux306eux30b0ux30eaux30eb}}

魚は大小にかかわらず、バターか植物油をたっぷりかけてから、やや弱めの火加減でグリルする。火入れの最中も小まめにバターまたは油をかけてやること。

魚の場合は、骨から簡単に身が離れるようになったら程良く火が通っている。

鯖{[}さば{]}、ルージェ、鰊{[}にしん{]}のような脂ののった魚以外は、焼く前に小麦粉をまぶしつけ、溶かしバターをかける。こうして魚を黄金色の殻で覆うようにすると、身が乾くのを防ぐと同時に、見た目も良くなる。

\hypertarget{ux30a4ux30aeux30eaux30b9ux98a8ux307eux305fux306fux30d0ux30bfux30fcux3067ux30d1ux30f3ux7c89ux8863ux3092ux3064ux3051ux305fux30b0ux30eaux30eb}{%
\subsection{イギリス風またはバターでパン粉衣をつけたグリル}\label{ux30a4ux30aeux30eaux30b9ux98a8ux307eux305fux306fux30d0ux30bfux30fcux3067ux30d1ux30f3ux7c89ux8863ux3092ux3064ux3051ux305fux30b0ux30eaux30eb}}

一般的に小さな素材でしかこの方法は用いないが、ごく弱火でグリルし、主素材の火入れとパン粉衣の色づけが同時に行なわれるようにする。

加熱中は小まめに澄ましバターをかけてやる。衣は素材の肉汁を閉じ込めるためのものなので、裏返す時は衣を壊さないように注意すること。

\hypertarget{ux63daux3052ux3082ux306e}{%
\section{揚げもの}\label{ux63daux3052ux3082ux306e}}





%%% Chapitre XIII Legumes
% エスコフィエ『料理の手引き』全注解
% 五島 学

\href{未、原文対照チェック}{} \href{未、日本語表現校正}{}
\href{未、その他修正}{} \href{未、原稿最終校正}{}

\hypertarget{legumes-farineux-et-pates-alimentaires}{%
\chapter{XIII. 野菜料理・パスタなど}\label{legumes-farineux-et-pates-alimentaires}}

\frchap{Légumes --- Farineux et Pâtes alimentaires}

\hypertarget{serie-des-legumes}{%
\section{野菜料理}\label{serie-des-legumes}}

\begin{center}
\headfont\large 下拵えと注意事項\label{observations-sur-les-operations-preliminaires}

\normalfont\textit{Observation sur les Opérations préliminaires.}
\end{center}

\normalfont\normalsize

\hypertarget{blanchissage}{%
\subsection[湯がく・下茹で]{\texorpdfstring{湯がく・下茹で\footnote{blanchir(ブロンシール)。もともとの意味は「白くする」.食材を冷蔵保存出来なかった中世には肉類はいったん下茹でしてから調理するのが一般的であり,肉を下茹ですると表面が白くなることからこの用語が用いられるようになった.素材の種類によっては白く茹であげるために単なる湯ではなく「ブラン」を用いる場合もある.これは、水1ℓにスプーン1杯強の小麦粉,塩6gとスプーン2杯の溶いて沸かす.クローヴを刺した玉ねぎ1ヶとブーケガルニ,下茹でする素材,空気に素材が触れて変色するのを防ぐための脂を入れる.脂は,牛あるいは仔牛の生のケンネ脂を細かく刻んだもの.必要なら脂を事前に冷水にさらして血等の夾雑物がないようにしておくこと.
  (原注)
  野菜の下茹で用のブランには,ヴィネガーではなくレモン汁を用いたほうがいい(原書
  p.405).}}{湯がく・下茹で}}\label{blanchissage}}

この作業は2つの目的で行なう。第1に、例えばほうれんそう、プティポワ、さやいんげん等の一般的な葉物野菜では、完全に火を通すのが目的。たっぷりの湯で手早く茹で、クロロフィルすなわち葉緑素を失わないようにすること。第
2は、野菜に自然にあるえぐ味を消す目的。例えばキャベツ、セロリ、シコレ等。原則的に、新野菜は下茹でしない。下茹でで完全に火を通してしまう野菜については、1リットルあたり7gの塩を湯に加えること。

\hypertarget{rafraichissage}{%
\subsection[冷水にはなす]{\texorpdfstring{冷水にはなす\footnote{rafraîchir
  ラフレシール}}{冷水にはなす}}\label{rafraichissage}}

湯がいた後、冷水にとるのは、野菜をブレゼにする場合と、オペレーションの都合から事前に茹でておかなければならない場合のみ。ただし、後でバターやクリームで合える場合は、冷水にとると風味が失われる。

\hypertarget{cuisson-des-legumes-a-l-anglaise}{%
\subsection{アラングレーズ}\label{cuisson-des-legumes-a-l-anglaise}}

沸騰した湯で茹でるのみ。次によく水切りをして、さらに水気をとばす。深皿に盛りつけ、貝殻形のバターを添えて供する。味付けは食べ手がする。葉物野菜なら何でもこのイギリス風に調理、提供可能。

\hypertarget{cuisson-des-legumes-secs}{%
\subsection{乾燥豆}\label{cuisson-des-legumes-secs}}

乾燥豆を水でもどしておくのはよろしくない。その年に穫れた良質のものなら、水から弱火でゆっくり沸かして茹でればいい。あく取りをして香味野菜\footnote{ベシャメルソース1ℓに生クリーム2dlを加え,火にかけてへらで混ぜながら
  ¾ℓになるまで煮つめる.
  布で漉し,クレーム・ドゥーブル1½dlとレモン汁½個分を少しずつ加えていき、濃さを調節する(原書p.32)。}を加え、蓋をしてごく弱火で茹でる。

あまりにも古い豆や品質が劣るものは水でもどす。ただし、豆が膨れるのに必要な時間きっかり、すなわち1時間半から2時間とすること。

何時間も水につけておくと、発酵が始まってしまう。そうなると豆の組織が損なわれて使いものにならなくなってしまうことさえある。

\hypertarget{braisage-des-legumes}{%
\subsection{野菜のブレゼ}\label{braisage-des-legumes}}

野菜は事前に湯がいて冷水にとる。形を整える。鍋の底と周囲に豚背脂のシートを張り、野菜を入れる。上面を背脂で覆う。鍋に蓋をして弱火でかるく汗をかかせるように蒸し煮した後、ひたひたまで白いフォンを注ぐ。鍋に蓋をし、中温のオーヴンに入れる。火が通ったら、野菜の水気をきり、用途に合わせて形を整える。その後すぐに使う場合は、煮汁の浮き脂を取り除いて煮つめ、野菜とともにソテ鍋で保温する。事前に仕込んでおく場合には、鍋から皿あるいは専用の容器に移す。煮汁は浮き脂を取り除かずにそのまま加える。バターを塗った紙で覆ってストックする。

\hypertarget{sauce-des-legumes-braises}{%
\subsection{野菜のブレゼのソース}\label{sauce-des-legumes-braises}}

ブレゼの煮汁を煮詰め、浮き脂を丁寧に取り除いて使う。場合によってはグラスドヴィアンド、あるいは相応量のドゥミ・グラスを加える。どちらの場合にも、

ソースがまろやかになるようバターを加えて仕上げる。必要ならレモン汁数滴も加える。

\hypertarget{liaison-des-legumes-vert-au-beurre}{%
\subsection{葉野菜をバターであえる}\label{liaison-des-legumes-vert-au-beurre}}

茹でた野菜はしっかりと水をきっておく。味付けをしてバターを加え、鍋をあおるようにしてバターが野菜全体にまわるようにする。バターの風味を失なわないためには、「火から外した状態」でバターを加えること。

\hypertarget{liaison-des-legumes-a-la-creme}{%
\subsection{クリームであえる}\label{liaison-des-legumes-a-la-creme}}

この方法で調理する場合は、野菜に火を通す際、やや歯応えを残しておくこと。

しっかり水気をきり、野菜を鍋に入れる。沸かした生クリームを野菜が上に顔を出す程度に加える。

時々、よくかきまぜながら加熱する。

クリームがほぼすっかり煮詰まったら、バターとレモン汁少量を加える。必要なら、生クリームにソース・クレーム\footnote{blanchir(ブロンシール)。もともとの意味は「白くする」.食材を冷蔵保存出来なかった中世には肉類はいったん下茹でしてから調理するのが一般的であり,肉を下茹ですると表面が白くなることからこの用語が用いられるようになった.素材の種類によっては白く茹であげるために単なる湯ではなく「ブラン」を用いる場合もある.これは、水1ℓにスプーン1杯強の小麦粉,塩6gとスプーン2杯の溶いて沸かす.クローヴを刺した玉ねぎ1ヶとブーケガルニ,下茹でする素材,空気に素材が触れて変色するのを防ぐための脂を入れる.脂は,牛あるいは仔牛の生のケンネ脂を細かく刻んだもの.必要なら脂を事前に冷水にさらして血等の夾雑物がないようにしておくこと.
  (原注)
  野菜の下茹で用のブランには,ヴィネガーではなくレモン汁を用いたほうがいい(原書
  p.405).}を足してもいい。

\hypertarget{cremes-et-puree-de-legumes}{%
\subsection{野菜のクレームとピュレ}\label{cremes-et-puree-de-legumes}}

乾燥豆とでんぷん質の野菜は裏漉ししてピュレにする。次にバター1かけを加えて火にかけ、水気をとばす。牛乳か生クリームを加えて濃さを調節する。

さやいんげん、カリフラワー等のように水分の多い野菜をピュレにする場合は、濃さを出すため、味のバランスのとれたでんぷん質の野菜のピュレを加えること。

野菜の「クレーム」にする場合は、でんぷん質の野菜のピュレではなく、濃く仕上げたソース・ベシャメルを加える。
\newpage
\href{未、原文対照チェック}{} \href{未、日本語表現校正}{}
\href{未、その他修正}{} \href{未、原稿最終校正}{}

\begin{Main}

\hypertarget{artichauts}{%
\subsection[アーティチョーク]{\texorpdfstring{アーティチョーク\footnote{artichaut
  (アルティショー)キク科の多年草で、草丈は1
  m程にもなる。フランス語としては、16世紀初頭には carchoffle あるいは
  artichault
  の綴りで記録されている。しばしば、カトリーヌ・ド・メディシスがイタリアからフランスに「紹介した」とか「もたらした」といわれるが、これは俗説であり、それ以前からフランスでも知られていたし、南フランスでは栽培されていた。実際のところ、カトリーヌ・ド・メディシスはフランスの王宮においてアーティチョークの料理が流行するきっかけ程度には普及に貢献したのだろう。16世紀末オリヴィエ・ド・セール『農業経営論』において既に、南フランスの気候を活かした周年栽培の方法が記されており、その方法論の基礎はこんにちでも変化していない。食材としては、主に花蕾を利用する。固く厚みのある花弁のような花萼に覆われており、ある程度成熟したものは花萼をすべてナイフで切り落して取り除き(この作業はアーティチョーク本体を回すようにして剥くようにするのでトゥルネtournerという)、さらに内部の繊毛をスプーン等で取り除いて皿のような形状にしたものを加熱したのちに、タルトレットのようにアパレイユを詰めるなどする。この基底部をfond
  d'artichaut (フォンダルティショー)または cul
  d'ratichaut(キュダルティショー)と呼ぶ。やや若どりで小さめのものは花萼も下半分は加熱すれば柔らかいため、花蕾の上部は切り落してごく外側の花萼だけ剥いてから、半割りまたは四つ割りにし、繊毛を取り除いてから加熱調理する。これは
  coeur d'artichaut
  (クールダルティショー)と呼ばれることが多い(ただしこれらの呼び名はやや曖昧なところがあり、これをフォンダルティショーと呼んでいると解釈可能なケースもある)。さらにごく若どりのものは生食も可能であり、
  poivrade
  (ポワヴラード)と呼ばれる。いずれの場合も、空気に触れるとすぐに黒く変色するので、レモンを擦り付けながら作業し、作業後はレモン果汁を加えた水にすぐに入れるといい。フランスのアーティチョークはブルターニュ産のものがとりわけ有名で、大きな花蕾を付ける品種が中心であり、南フランス産のものは比較的小ぶりで、花萼が紫色がかった品種が代表的。日本には明治時代に伝わり、何度も生産が試みられているが、一般野菜としての需要を喚起することが出来ずにいるため、現在も輸入品が中心。}}{アーティチョーク}}\label{artichauts}}

\index{artichaut@artichaut|(}
\index{あーていちよーく@アーティチョーク|(}

\end{Main}

\begin{recette}

\hypertarget{artichauts-a-la-barigoule}{%
\subsubsection[アーティチョーク・バリグール]{\texorpdfstring{アーティチョーク・バリグール\footnote{プロヴァンス地方に自生する乳茸の一種。もとはこの茸を用いた料理だったといわれるが、どんな種類の茸を用いてもよいとされる。実際には、いわゆるマッシュルームを用いたデュクセル・セッシュを使うケースが多い。}}{アーティチョーク・バリグール}}\label{artichauts-a-la-barigoule}}

\frsub{Artichauts à la Barigoule}

\index{artichaut@artichautl!barigoule@--- à la Barigoule}
\index{barigoule@barigoule!artichauts@Artichauts à la ---}
\index{あーていちよーく@アーティチョーク!はりくーる@---・バリグール}
\index{はりくーる@バリグール!あーていちよーく@アーティチョーク・バリグール}

アーティチョークは新鮮で柔らかいものを選ぶこと。花蕾の上部\footnote{この場合、花萼をまったく残さずに小さな皿、あるいはタルトレットの形状に剥くことになるので、上から
  \(\frac{1}{3}\) 〜 \(\frac{3}{4}\)
  は切り落すことになる。また、本文では言及されていないが、もし多少でも茎が付いている場合は、底面が安定するように平らに切り落しておくこと。}を切り落し、周囲を花萼をナイフで削り取る\footnote{この作業も、アーティチョークを回すようにして行なうため、tourner
  (トゥルネ)の語が用いられる。また、この料理の場合は全体に茶色い仕上りになるため、アーティチョークが空気に触れることで黒ずんでしまうことが問題にならないためか、レモンを擦り付けるなどの指示はされていない。}。\protect\hyperlink{blanchissage}{下茹で}し、それから繊毛を取り除く。完全に取り除くよう心を配ること\footnote{この作業は下茹で前に行なうこともある。その場合はスプーン等を使って削るようにする。大型のアーティチョークの基底部の場合は、本文にあるように、下茹で後に手で繊毛を毟り取るほうがきれいに仕上がる。}。

内側に塩こしょうする。\protect\hyperlink{duxelles-seche}{デュクセル}に、\(\frac{1}{4}\)の重さの豚背脂を器具を用いておろして加え、さらに豚背脂と同量のバターを合わせ、アーティチョークの中央に詰める。こうして詰め物をしたアーティチョークをごく薄い豚背脂のシートで包み、紐で縛る。これを、\protect\hyperlink{braisage-des-legumes}{ブレゼ}用に準備した\footnote{\protect\hyperlink{braisage-des-legumes}{野菜のブレゼ}の項を必ず参照のこと。}鍋に並べ、\protect\hyperlink{fonds-brun}{茶色いフォン}をアーティチョークの高さまで注いで蓋をしてごく弱火で加熱し\footnote{この手法で火を通すことそれ自体が
  braiser (ブレゼ)と呼ばれる。}、しっかりと火を通す。

提供直前に、紐を外してアーティチョークを皿に盛り付ける。ブレゼした煮汁は漉して、浮いている脂を取り除く\footnote{dégraisser
  (デグレセ)。}。上等な\protect\hyperlink{sauce-demi-glace}{ソース・ドゥミグラス}適量を合わせてとろみを付け、アーティチョークに上からかけて余らない程度の量になるまで煮詰める。

\index{artichaut@artichaut|)}
\index{あーていちよーく@アーティチョーク|)}

\end{recette}



%%% Chapitre XV. Glaces
%% エスコフィエ『料理の手引き』全注解
% 五島 学


\hypertarget{xv.ux30a2ux30a4ux30b9ux30afux30eaux30fcux30e0glaces}{%
\chapter{XV. アイスクリーム Glaces}\label{xv.ux30a2ux30a4ux30b9ux30afux30eaux30fcux30e0glaces}}

\index{あいすくりーむ@アイスクリーム} \index{glace@glace}

アイスクリームは小菓子とともに供される。少なくとも料理の面ではディナーの総まとめだ。だから、上手に作られて美しくプレゼンテーションされるアイスクリームの品々は極上の美味しさという理想そのものなのだ。この食事の締めくくりという分野においては、料理芸術の天才\footnote{ここではおそらくカレームのことを指しているを思われるが、カレームの『王都パリのパティスリ』(1815年)第2巻にはわずかだがcrème
  glacée
  (クレームグラセ)すなわちアイスクリームのレシピが掲載されている(pp.144-145)。また、デュボワ、ベルナール『古典料理』第2巻(1856年)にもアイスクリーム、シャーベットについての記述がある
  (pp.665-675)。そもそも、広義のアイスクリームをフランスの宮廷に伝えたのは16世紀、フィレンツェから輿入れしたカトリーヌ・ド・メディシス(1519〜1589)だと言われているし、17世紀のルイ14世は冬の間に貯蔵した氷を利用してアイスクリームを作らせて夏に愉しんでいたという。また、
  18世紀末にパリのブルヴァール・モンマルトルにカフェ・フラスカティを開いたガルキなる人物はナポリ出身のアイスクリーム職人で、王政復古〜七月王政期にかけて店は盛況だったという。とりわけアイスクリームが評判を呼んだらしい。\protect\hyperlink{garniture-frascati}{ガルニチュール・フラスカーティ}訳注も参照。}もアイスクリーム程の自由な発想をすることは出来なかったし、これほどまでに美味しい知的発明ともいえる甘味を生みだすことは出来なかった。もっとも、イタリアがアイスクリーム技術発祥の地\footnote{アイスクリームをどのように定義するかによって変わってくるが、広義の氷菓を夏に愉しむということ自体は紀元前からいくつかの文明においておこなわれていた。ただ、マルコ・ポーロが13世紀末にに中国から、氷に硝石を加えることで効率的に冷却する技術を伝えたとも言われており、その意味ではイタリアがアイスクリーム発祥の地であるという表現は可能かも知れない。}で、ナポリの職人達がアイスクリームを作る技術において当然ともいうべき名声を得続けてきたとはいえ、我々フランスの職人達が、このアイスクリームという食品科学における重要分野のひとつで技術革新を頂点まで究めたのだ\footnote{まったく根拠なしの我田引水というわけではなく、19世紀中葉に冷凍庫を実用化したのはフランスの技術者、フェルディナン・カレ(1824〜1900)他。それまでは氷に塩と硝石を混ぜて-15℃程度に冷やしてアイスクリーム、シャーベットなどの氷菓が作られていた。}。

\href{原稿下準備なし}{} \href{訳と注釈\%2020180405進行中}{}
\href{未、原文対照チェック}{} \href{未、日本語表現校正}{}
\href{未、注釈チェク}{} \href{未、原稿最終校正}{}

\hypertarget{compositions-diverses-de-glaces-cremes}{%
\subsection{アイスクリームのアパレイユ}\label{compositions-diverses-de-glaces-cremes}}

\frsec{Compositions diverses de Glaces Crèmes}

\href{このindexの行は強制改行を入れないようにしてください}{}
\index{glaces@glaces!composition diverses cremes@compositions diverses de --- crèmes}
\index{あいすくりーむ@アイスクリーム!あぱれいゆ@---のアパレイユ}

\hypertarget{nota-compositions-diverses-de-glaces-cremes}{%
\subparagraph{【原注】}\label{nota-compositions-diverses-de-glaces-cremes}}

ここで示しているアイスクリームのアパレイユで砂糖と卵黄の分量、作業手順はどれもまったく同じだ。それぞれのアイスクリームの特徴となる香料や煎じ汁\footnote{infusion
  (アンフュジオン)ハーブなど煮出した液体。一般的にはハーブティのことも意味する。}によってのみ、変更を加えるものもある。
\begin{recette}
\hypertarget{aux-amandes}{%
\subsubsection{アーモンド}\label{aux-amandes}}

\index{glaces@glaces!compositions diverses cremes@compositions diverses de --- crèmes!amandes@--- aux amandes}
\index{あいすくりーむ@アイスクリーム!あぱれいゆ@---のアパレイユ!あーもんと@アーモンド}

\href{この節のインデックスのつけかたは要検討}{}

湯剥きしたばかりのスイートアーモンド100 gとビターアーモンド\footnote{アーモンドには主に2種あり、一般的なスイートアーモンドと、香り付けにごく少量のみ用いられるビターアーモンドがある。\protect\hyperlink{beurre-d-amande}{アーモンドバター}訳注も参照。}5〜6粒を、香りが溶けだすように少しずつ水を加えながら細かくすり潰す。

こうして出来たアーモンドペーストを、あらかじめ沸かしておいた牛乳で20分、香りを煮出す。その後は上述のとおりの砂糖と卵黄の分量でクリームを用意する。
\end{recette}
%\input{15-glaces/15-03-pp861-862}
%\input{15-glaces/15-04-pp862-863}
%\input{15-glaces/15-05-pp863-865}













\appendix

%%%索引ページ出力
\backmatter

\printindex


%%% Important! 文書終了%
\end{document}



%% Local Variables:
%% TeX-engine: luatex
%% End: