\hypertarget{xv.ux30a2ux30a4ux30b9ux30afux30eaux30fcux30e0glaces}{%
\chapter{XV. アイスクリーム Glaces}\label{xv.ux30a2ux30a4ux30b9ux30afux30eaux30fcux30e0glaces}}

\index{あいすくりーむ@アイスクリーム} \index{glace@glace}

アイスクリームは小菓子とともに供される。少なくとも料理の面ではディナーの総まとめだ。だから、上手に作られて美しくプレゼンテーションされるアイスクリームの品々は極上の美味しさという理想そのものなのだ。この食事の締めくくりという分野においては、料理芸術の天才\footnote{ここではおそらくカレームのことを指しているを思われるが、カレームの『王都パリのパティスリ』(1815年)第2巻にはわずかだがcrème
  glacée
  (クレームグラセ)すなわちアイスクリームのレシピが掲載されている(pp.144-145)。また、デュボワ、ベルナール『古典料理』第2巻(1856年)にもアイスクリーム、シャーベットについての記述がある
  (pp.665-675)。そもそも、広義のアイスクリームをフランスの宮廷に伝えたのは16世紀、フィレンツェから輿入れしたカトリーヌ・ド・メディシス(1519〜1589)だと言われているし、17世紀のルイ14世は冬の間に貯蔵した氷を利用してアイスクリームを作らせて夏に愉しんでいたという。また、
  18世紀末にパリのブルヴァール・モンマルトルにカフェ・フラスカティを開いたガルキなる人物はナポリ出身のアイスクリーム職人で、王政復古〜七月王政期にかけて店は盛況だったという。とりわけアイスクリームが評判を呼んだらしい。\protect\hyperlink{garniture-frascati}{ガルニチュール・フラスカーティ}訳注も参照。}もアイスクリーム程の自由な発想をすることは出来なかったし、これほどまでに美味しい知的発明ともいえる甘味を生みだすことは出来なかった。もっとも、イタリアがアイスクリーム技術発祥の地\footnote{アイスクリームをどのように定義するかによって変わってくるが、広義の氷菓を夏に愉しむということ自体は紀元前からいくつかの文明においておこなわれていた。ただ、マルコ・ポーロが13世紀末にに中国から、氷に硝石を加えることで効率的に冷却する技術を伝えたとも言われており、その意味ではイタリアがアイスクリーム発祥の地であるという表現は可能かも知れない。}で、ナポリの職人達がアイスクリームを作る技術において当然ともいうべき名声を得続けてきたとはいえ、我々フランスの職人達が、このアイスクリームという食品科学における重要分野のひとつで技術革新を頂点まで究めたのだ\footnote{まったく根拠なしの我田引水というわけではなく、19世紀中葉に冷凍庫を実用化したのはフランスの技術者、フェルディナン・カレ(1824〜1900)他。それまでは氷に塩と硝石を混ぜて-15℃程度に冷やしてアイスクリーム、シャーベットなどの氷菓が作られていた。}。
