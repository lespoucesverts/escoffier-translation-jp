\href{原稿下準備なし}{} \href{訳と注釈\%2020180405進行中}{}
\href{未、原文対照チェック}{} \href{未、日本語表現校正}{}
\href{未、注釈チェク}{} \href{未、原稿最終校正}{}

\hypertarget{compositions-diverses-de-glaces-cremes}{%
\subsection{アイスクリームのアパレイユ}\label{compositions-diverses-de-glaces-cremes}}

\frsec{Compositions diverses de Glaces Crèmes}

\href{このindexの行は強制改行を入れないようにしてください}{}
\index{glaces@glaces!composition diverses cremes@compositions diverses de --- crèmes}
\index{あいすくりーむ@アイスクリーム!あぱれいゆ@---のアパレイユ}

\hypertarget{nota-compositions-diverses-de-glaces-cremes}{%
\subparagraph{【原注】}\label{nota-compositions-diverses-de-glaces-cremes}}

ここで示しているアイスクリームのアパレイユで砂糖と卵黄の分量、作業手順はどれもまったく同じだ。それぞれのアイスクリームの特徴となる香料や煎じ汁\footnote{infusion
  (アンフュジオン)ハーブなど煮出した液体。一般的にはハーブティのことも意味する。}によってのみ、変更を加えるものもある。
\begin{recette}
\hypertarget{aux-amandes}{%
\subsubsection{アーモンド}\label{aux-amandes}}

\index{glaces@glaces!compositions diverses cremes@compositions diverses de --- crèmes!amandes@--- aux amandes}
\index{あいすくりーむ@アイスクリーム!あぱれいゆ@---のアパレイユ!あーもんと@アーモンド}

\href{この節のインデックスのつけかたは要検討}{}

湯剥きしたばかりのスイートアーモンド100 gとビターアーモンド\footnote{アーモンドには主に2種あり、一般的なスイートアーモンドと、香り付けにごく少量のみ用いられるビターアーモンドがある。\protect\hyperlink{beurre-d-amande}{アーモンドバター}訳注も参照。}5〜6粒を、香りが溶けだすように少しずつ水を加えながら細かくすり潰す。

こうして出来たアーモンドペーストを、あらかじめ沸かしておいた牛乳で20分、香りを煮出す。その後は上述のとおりの砂糖と卵黄の分量でクリームを用意する。
\end{recette}