\begin{main}

\hypertarget{asperges}{%
\subsection{アスパラガス}\label{asperges}}

\begin{frsecbenv}

Asperges

\end{frsecbenv}

アスパラガスは主に4種類\footnote{variété
  (ヴァリエテ)野菜についての場合通常は「品種」と訳すが、ここではバラエティ、種類くらいの意。なお、いわゆる「アスパラソバージュ」asperges
  sauvages
  (アスペルジュソヴァージュ)はまったくの別種であり、ここには含まれていない。}に分けられる。フランス産アスパラガスの典型的な品種、アルジャントゥイユ\footnote{Argenteuil
  (アルジョントゥイユ)。パリ近郊の地名。かつてここでアスパラガスの生産が盛んだったという。またこの地名を冠した品種が21
  世紀になった現在でも主流であることは事実。穂先がやや紫がかる傾向にあるが基本的には緑色の品種。}。グリーンアスパラガス\footnote{現在ではオランダの種苗会社が交配したF1品種が増えている。}。ジェノヴァ産紫アスパラガス\footnote{同上。ただしこの紫色は茹でると失なわれる。}、イタリア産アスパラガスの典型で繊細な風味だがややえぐ味がある。ベルギー産ホワイトアスパラガス\footnote{ホワイトアスパラガスは「品種」ではなく栽培方法が異なる(軟白の工程が入る)。理屈のうえではどんな品種であってもホワイトアスパラガスにすることは可能。秋のうちに地上部を切り落し、根株を中心にプラウ(鋤の一種)などを用いて土を盛り上げる。全体に大きなかまぼこのような格好になる。春になると、地中深くの根株から伸びてきたアスパラガスの茎は日光に当たっていないので軟白されている。それを地上に出る直前に収穫する。収穫は一定期間で終了させ、盛り上げた土を平らに戻し、緑の茎葉を茂らせて翌年のために根株を養成する。トラクタがプラウを曳けるようかなり条間を広くとる必要があり、単位面積あたりの収量は低い。かつては缶詰の材料として北海道で盛んに栽培されていたが、だんだん生産量が落ちている。近年日本ではハウス栽培で土盛りをせずトンネルに遮光率100%のシートで暗闇を作って栽培する方式が増えつつある。いずれもフランス料理においてあまり高い評価を得られていないのは、品種の選定と栽培方法に負うところが大きいだろう。}、これも繊細な風味だが、輸送による劣化がはげしい。

\end{main}
