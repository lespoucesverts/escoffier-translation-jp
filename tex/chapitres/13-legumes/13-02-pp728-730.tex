

\begin{main}

\hypertarget{artichauts}{%
\subsection[アーティチョーク]{\texorpdfstring{アーティチョーク\footnote{artichaut
  (アルティショー)キク科の多年草で、草丈は1
  m程にもなる。フランス語としては、16世紀初頭には carchoffle あるいは
  artichault
  の綴りで記録されている。しばしば、カトリーヌ・ド・メディシスがイタリアからフランスに「紹介した」とか「もたらした」といわれるが、これは俗説であり、それ以前からフランスでも知られていたし、南フランスでは栽培されていた。実際のところ、カトリーヌ・ド・メディシスはフランスの王宮においてアーティチョークの料理が流行するきっかけ程度には普及に貢献したのだろう。16世紀末オリヴィエ・ド・セール『農業経営論』において既に、南フランスの気候を活かした周年栽培の方法が記されており、その方法論の基礎はこんにちでも変化していない。食材としては、主に花蕾を利用する。固く厚みのある花弁のような花萼に覆われており、大型の品種の1番花、2番花など大ぶりのものは、花萼をすべてナイフで切り落して取り除き(この作業はアーティチョーク本体を回すようにして剥くようにするのでトゥルネtournerという)、さらに内部の繊毛をスプーン等で取り除いて皿のような形状にしたものを加熱したのちに、タルトレットのようにアパレイユを詰めるなどする。この基底部をfond
  d'artichaut (フォンダルティショー)または cul
  d'ratichaut(キュダルティショー)と呼ぶ。また、\protect\hyperlink{artichauts-avec-sauces-diverses}{アーティチョーク・いろいろなソースで}のように花萼の1枚ずつむしって、その下部を葉でしごくようにして食べることもあり、かつては専用の器もポピュラーだった。小型品種および大型品種の3番花以降のような小さめのものは、\textbf{若どりであれば}、内側の花萼の下半分も加熱すれば柔らかく美味のため(やや竹の子の姫皮に似たテクスチュア。もちろん風味はまったく違う)、花蕾の上部は切り落して外側の固い花萼だけナイフでむいてから、半割りまたは四つ割りにし、繊毛を取り除いてから加熱調理する。これは
  coeur d'artichaut
  (クールダルティショー)と呼ばれることが多い(ただしこれらの呼び名はやや曖昧なところがあり、これをフォンダルティショーと読みかえても解釈可能なケースもある)。さらにごく若どりのものは生食も可能であり、poivrade
  (ポワヴラード)と呼ばれる。よく誤解されることだが、Poivrade
  という品種はなく、収穫タイミングとサイズ(というよりも生食出来るくらい柔らかいものをポワヴラードと呼んでいるのであり、実際にはViolet
  de
  Provence(ヴィオレドプロヴォンス)という品種(およびそれを親としたF1品種)が中心だが、本来的にはポワヴラードに品種の決まりはない。。いずれの場合も、空気に触れるとすぐに黒く変色するので、レモンを擦り付けながら作業し、作業後はレモン果汁またはアスコルビン酸を加えた水にすぐに入れるといい。フランスのアーティチョークはブルターニュ産のものがとりわけ有名で、大きな花蕾を付ける品種が中心。南フランス産のものは比較的小ぶりで、花萼が紫色がかった品種(上述のViolet
  de
  Provence)が代表的。日本には明治時代に伝わり、何度も生産が試みられているが、一般野菜としての需要を喚起することが出来ずにいるため、営利栽培は事実上ほぼ不可能。このため現在も輸入品が中心。なお、アーティチョークは株分けで殖やすのが一般的であり、ブルターニュ地方の代表品種Camus(カミュ)は種子の流通すらしていない。近年は株分けせず種子から栽培して特性の揃う
  F1
  品種も普及しつつあるが、1回植えると3〜5年はそのまま株を更新しないので、フランスにおいてもF1品種の普及のペースは比較的遅い。なお、以下の訳本文の料理名は、原文に「アーティチョークの基底部」「アーティチョークの芯」「櫛切りにしたアーティチョーク」に相当する表現がある場合でも、「アーティチョーク」で統一した。これは、2018年現在の日本において、本書のレシピで指示されている大きさで充分なクオリティのアーティチョークがいつも容易に入手できるとはかぎらないのが理由。本書のレシピはどれもそうだが、指示されている形態にこだわらず、自由な発想の源として活用することが望ましいだろう。}}{アーティチョーク}}\label{artichauts}}

\begin{frsecbenv}

Artichauts

\end{frsecbenv}

\index{artichaut@artichaut|(}
\index{あーていちよーく@アーティチョーク|(}

\end{main}

\begin{recette}

\hypertarget{artichauts-a-la-barigoule}{%
\subsubsection[アーティチョーク・バリグール]{\texorpdfstring{アーティチョーク・バリグール\footnote{プロヴァンス地方に自生する乳茸の一種。もとはこの茸を用いた料理だったといわれるが、どんな種類の茸を用いてもよいとされる。実際には、いわゆるマッシュルームを用いたデュクセル・セッシュを使うケースが多い。}}{アーティチョーク・バリグール}}\label{artichauts-a-la-barigoule}}

\begin{frsubenv}

Artichauts à la Barigoule

\end{frsubenv}

\index{artichaut@artichautl!barigoule@--- à la Barigoule}
\index{barigoule@barigoule!artichauts@Artichauts à la ---}
\index{あーていちよーく@アーティチョーク!はりくーる@---・バリグール}
\index{はりくーる@バリグール!あーていちよーく@アーティチョーク・---}
\srcDemiGlace{artichaut barigoule}{Artichauts à la Barigoule}{あーていちよーくはりくーる}{アーティチョーク・バリグール}

アーティチョークは新鮮で柔らかいものを選ぶこと。花蕾の上部\footnote{この場合、花萼をまったく残さずに小さな皿、あるいはタルトレットの形状に剥くことになるので、上から
  \(\frac{1}{3}\) 〜 \(\frac{3}{4}\)
  は切り落すことになる。また、本文では言及されていないが、もし多少でも茎が付いている場合は、底面が安定するように平らに切り落しておくこと。}を切り落し、周囲の花萼をナイフで削り取る\footnote{この作業も、アーティチョークを回すようにして行なうため、tourner
  (トゥルネ)の語が用いられる。また、この料理の場合は全体に茶色い仕上りになるため、アーティチョークが空気に触れることで黒ずんでしまうことが問題にならないためか、レモンを擦り付けるなどの指示はされていない。}。\protect\hyperlink{blanchissage}{下茹で}し、それから繊毛を取り除く。完全に取り除くよう心を配ること\footnote{この作業は下茹で前に行なうこともある。その場合はスプーン等を使って削るようにする。大型のアーティチョークの基底部の場合は、本文にあるように、下茹で後に手で繊毛を毟り取るほうがきれいに仕上がる。}。

内側に塩こしょうする。\protect\hyperlink{duxelles-seche}{デュクセル}に、\(\frac{1}{4}\)の重さの豚背脂を器具を用いておろして加え、さらに豚背脂と同量のバターを合わせ、アーティチョークの中央に詰める。こうして詰め物をしたアーティチョークをごく薄い豚背脂のシートで包み、紐で縛る。これを、\protect\hyperlink{braisage-des-legumes}{ブレゼ}用に準備した\footnote{\protect\hyperlink{braisage-des-legumes}{野菜のブレゼ}の項を必ず参照のこと。}鍋に並べ、\protect\hyperlink{fonds-brun}{茶色いフォン}をアーティチョークの高さまで注いで蓋をしてごく弱火で加熱し\footnote{この手法で火を通すことそれ自体が
  braiser (ブレゼ)と呼ばれる。}、しっかりと火を通す。

提供直前に、紐を外してアーティチョークを皿に盛り付ける。ブレゼした煮汁は漉して、浮いている脂を取り除く\footnote{dégraisser
  (デグレセ)。}。上等な\protect\hyperlink{sauce-demi-glace}{ソース・ドゥミグラス}適量を合わせてとろみを付け、アーティチョークに上からかけて余らない程度の量になるまで煮詰める。

\atoaki{}

\hypertarget{artichauts-cavour}{%
\subsubsection[アーティチョーク・カヴール]{\texorpdfstring{アーティチョーク・カヴール\footnote{サルデーニャ王国の宰相として、国王ヴィットーリオ・エマヌエーレ2
  世ととにも、リソルジメント(イタリア統一運動)を率いた。巧みな外交戦略によりイタリア統一を実現させた英雄のひとりと見なされている。イタリア王国成立後も首相、外相として活躍した。イタリアのおもな都市には必ずといっていい程、彼の名を冠した「カヴール通り」がある。なお、歴史的には北イタリアおよび南フランスのニース周辺にいたる地域をめぐって、オーストリア、フランス、イタリアが領土の奪い合いが行なわれ、イタリア統一後、「未回収のイタリア」運動ではニース周辺をイタリアに編入しようとイタリアは強く主張していた。まさしくエスコフィエの少年時代、ニースでフランス料理の修業を始めた頃のことでもあり、彼の自身がフランス人たる意識は否が応にも強まったものと思われる。}}{アーティチョーク・カヴール}}\label{artichauts-cavour}}

\begin{frsubenv}

Artichauts Cavour

\end{frsubenv}

\index{artichaut@artichautl!cavour@--- Cavour}
\index{cavour@Cavour!artichauts@Artichauts ---}
\index{あーていちよーく@アーティチョーク!かうーる@---・カヴール}
\index{かうーる@カヴール!あーていちよーく@アーティチョーク・---}

プロヴァンス産\footnote{プロヴァンスでは Artichaut Violet de Provence
  という品種がもっとも有名。この品種は草勢がおとなしく、小振りの花蕾を若どりしてポワヴラードとして出荷されることも多い。}の小振りで柔らかいアーティチョークの外側の花萼を卵形に剥く。

これを\protect\hyperlink{consomme-blanc}{白いコンソメ}で茹で、火が通ったら取り出して湯きりし、さらに圧し絞って余計な水分を完全に出させる。溶かしバターに漬け込む。取り出して、おろしたグリュイエールチーズとパルメザンチーズの上を転がしてチーズをまぶし付ける。これをグラタン皿に環状に並べ、高温のオーブンで焼く。

アーティチョーク6個につき固茹で卵1個をみじん切りにして、バターでソテーする。バターが充分に泡立ってきたら、アンチョビエッセンス少々とパセリのみじん切りを加えてソースを仕上げ、アーティチョークに注ぎかける。

\atoaki{}

\hypertarget{coeurs-d-artichauts-clamart}{%
\subsubsection[アーティチョーク・クラマール]{\texorpdfstring{アーティチョーク\footnote{coeurs
  d'artichauts
  日本語にすると「アーティチョークの芯」としか表現しようがないが、「りんごの芯」「キャベツの芯」など「芯」は非可食部=廃棄部分のイメージが強いので注意したい。アーティチョークの場合は比較的小振り〜中位のサイズの花蕾の外側の固い花萼を剥いた状態のものを指す。レチュやセロリ、カルドン他でも用いられる表現であり「芯に近い柔らかい部分」と理解するといいだろう。}・クラマール\footnote{パリ郊外南西の地名で、かつては豆類(プチポワなど)の生産が盛んだった。}}{アーティチョーク・クラマール}}\label{coeurs-d-artichauts-clamart}}

\begin{frsubenv}

Coeurs d'artichauts Clamart

\end{frsubenv}

\index{artichaut@artichautl!coeur clamart@Coeurs d'---s Clamart}
\index{clamart@Clamart!coeur artichauts@Coeurs d'artichauts ---}
\index{あーていちよーく@アーティチョーク!くらまーる@アーティチョークの芯・クラマール}
\index{くらまーる@クラマール!あーていちよーく@アーティチョークの芯・---}

中位のサイズで柔らかいアーティチョークを選ぶこと。掃除をして\footnote{花蕾の上半分程を切り落し、花萼の固い部分と茎の皮を剥く。内部の繊毛はこの時点でスプーン等で取り除いてもいいし、櫛切りにした後に取り除いてもいい。}6つに櫛切りにする。

内側にバターを塗ったココット鍋に並べ、若どりのミニキャロットを四つ割りにして加える。アーティチョーク1つにつき大さじ3杯の莢から出したばかりのプチポワを加える。たっぷりのブーケガルニと水少々を加えて軽く塩をし、蓋をしてごく弱火で蒸し煮する\footnote{原文
  cuire très doucement à l'étuvée
  (キュイールトレドゥスモンアレチュヴェ)。}。

提供直前に、ブーケガルニを取り出し、\protect\hyperlink{beurre-manie}{ブールマニエ}少々で軽く煮汁にとろみを付ける。ココット鍋のまま供する。

\atoaki{}

\hypertarget{coeurs-d-artichauts-grand-duc}{%
\subsubsection[アーティチョーク・グランデュック]{\texorpdfstring{アーティチョーク・グランデュック\footnote{大公の意。ロシア皇太子もこの称号で呼ばれる。\href{garniture-grand-duc}{ガルニチュール・グランデュック}も参照。}}{アーティチョーク・グランデュック}}\label{coeurs-d-artichauts-grand-duc}}

\begin{frsubenv}

Coeurs d'artichauts Grand-Duc

\end{frsubenv}

\index{artichaut@artichaut!coeur grand-duc@Coeurs d'---s Grand-Duc}
\index{grand-duc@Grand-Duc!coeur artichauts@Coeur d'artichauts ---}
\index{あーていちよーく@アーティチョーク!くらんてゆつく@---・グランデュック}
\index{くらんてゆつく@グランデュック!あーていちよーく@アーティチョーク・---}
\srcCreme{coeur artichaut grand-duc}{Coeurs d'artichauts Grand-Duc}{あーていちよーくのしんくらんてゆつ}{アーティチョークの芯・グランデュック}
\srcGlaceDeViande{artichaut grand-duce}{Coeurs d'artichauts Grand-Duc}{あーていちよーくのしんくらんてゆつく}{アーティチョークの芯・グランデュック}

中位のサイズのよく揃ったアーティチョークを選ぶこと。外側の固い花萼を剥き\footnote{tourner
  (トゥルネ)。}、塩湯で茹でる。しっかりと湯切りをし、底に薄く\protect\hyperlink{sauce-creme}{ソース・クレーム}を塗った皿に花輪状に並べる。同じソースを上から覆いかけ、おろしたパルメザンチーズを軽く振りかけ、溶かしバターをかけてからサラマンダー\footnote{非常に強い上火だけのオーブン。もっぱらグラタンなどの焼き色を付ける目的で用いられる。}に入れて焼き色を付ける。アーティチョークを花輪状に並べて空いていた中心部分に、バターであえたアスパラガスの穂先の束を盛り込み、アーティチョークひとつひとつにトリュフのスライスをのせる。トリュフのスライスは溶かした\protect\hyperlink{glace-de-viande}{グラスドヴィアンド}を混ぜた溶かしバターで加熱しておくこと。

\atoaki{}

\hypertarget{artichauts-avec-sauce-diverses}{%
\subsubsection{アーティチョーク・いろいろなソースで}\label{artichauts-avec-sauce-diverses}}

\begin{frsubenv}

Artichauts avec sauce diverses

\end{frsubenv}

\index{artichaut@artichaut!sauces diverses@---s avec sauces diverses}
\index{あーていちよーく@アーティチョーク!いろいろなそーすて@---・いろいろなソースで}
\srcBeurre{artichaus sauces diverses}{Artichauts avec sauces diverses}{あーていちよーくいろいろなそーすて}{アーティチョーク・いろいろなソースで}
\srcHollandaise{artichaus sauces diverses}{Artichauts avec sauces diverses}{あーていちよーくいろいろなそーすて}{アーティチョーク・いろいろなソースで}
\srcMousseline{artichaus sauces diverses}{Artichauts avec sauces diverses}{あーていちよーくいろいろなそーすて}{アーティチョーク・いろいろなソースで}
\srcVinaigrette{artichaus sauces diverses}{Artichauts avec sauces diverses}{あーていちよーくいろいろなそーすて}{アーティチョーク・いろいろなソースで}

アーティチョークは高さ \(\frac{2}{3}\) くらいで揃うように上部を切り落す
\footnote{すなわち、上 \(\frac{1}{3}\) 程度を切り落す。}。周囲をきれいに掃除する\footnote{具体的には、花萼にトゲがあるようならハサミなどを用いてトゲのある部分をきれいに切り落す。}。紐で縛り\footnote{茹でる際に花萼がばらばらにならないようにするため、十文字に縛ることが多い。また切り落した底部にレモンのスライスをあてて縛るとする教本もあるが、底部を見せる料理ではないので、あまり意味はない。底部を切り落した際に、レモンを擦り付ければ充分だろう。}、軽く塩を加えた沸騰した湯に投入する。強火で火を通すこと。提供直前に湯から取り出して水気をよくきり、紐をほどく。

これをナフキンの上に盛り付け\footnote{かつてはこの方法でアーティチョークを盛るための専用の皿が一般的にあったが、近年はあまり見かけない。}、\protect\hyperlink{sauce-au-beurre}{ソース・オブール}や\protect\hyperlink{sauce-hollandaise}{オランデーズソース}、\protect\hyperlink{sauce-mousseline}{ソース・ムスリーヌ}などを別添で供する\footnote{いわゆるソース入れではなく、小さめの陶製のスフレ型や小さな深皿を人数分用意してソースを供するのがいいだろう。}。このように調理したアーティチョークを冷製として供する場合には、内部の繊毛を取り除き、ナフキンの上に盛り、\protect\hyperlink{vinaigrette}{ヴィネグレット}とともに供する\footnote{この茹でただけの調理は、食べ手がそれぞれに花萼を剥がして、その最下部にある身にソースを付けて、歯でしごくようにして食するのが一般的。この調理には
  Camus(カミュ)や Laon(ロン)に代表される大型品種が適している。}。

\atoaki{}

\hypertarget{artichauts-a-la-provencale}{%
\subsubsection{アーティチョーク・プロヴァンス風}\label{artichauts-a-la-provencale}}

\index{artichaut@artichaut!provencale@---s à la Provençale}
\index{provencal@provençal!artichauts@Artichauts à la ---}
\index{あーていちよーく@アーティチョーク!ふろうあんすふう@---・プロヴァンス風}
\index{ふろうあんすふう@プロヴァンス風!あーていちよーく@アーティチョーク・---}

\begin{frsubenv}

Artichauts à la Provençale

\end{frsubenv}

プロヴァンス産のごく小さなアーティチョーク\footnote{いわゆる poivrade
  (ポワヴラード)。外から基底部のあたりを親指と人差し指か中指で挟んでみて弾力があるものがいい。}を用意する。掃除をして\footnote{ポワヴラードのような小さなアーティチョークは茎を5〜10
  cm程度付けて収穫されることが多く、茎も食用となるとめ、表皮にある筋を取り除き、また花蕾の下に数枚ある小さな葉を取り除く。}、熱した油を入れた素焼きの鍋に投入する。塩こしょうで調味し、鍋に蓋をする。そのまま10分間加熱する。さらに、アーティチョーク
12 個につ莢から出したばかりの柔らかいプチポワ\(\frac{1}{2}\)
Lと粗く刻んだレチュの葉を1個分加える。

再度蓋をして、弱火で加熱する。水などの液体は加えない。鍋にしっかる蓋をして中火で加熱してやれば水気が蒸発することもないから、プチポワとレチュの水分だけで充分。

\atoaki{}

\hypertarget{artichauts-stanley}{%
\subsubsection[アーティチョーク・スタンリー]{\texorpdfstring{アーティチョーク・スタンリー\footnote{イギリスの探検家ヘンリー・モートン・スタンリー卿(1841〜1904)。アフリカを探検し、遭難した医師デイヴィッド・リヴィングストンを発見したことや、ベルギー国王レオポルド2世の要請によりコンゴ自由国の設立にも関与した。}}{アーティチョーク・スタンリー}}\label{artichauts-stanley}}

\begin{frsubenv}

Artichauts Stanley

\end{frsubenv}

\index{artichaut@artichaut!stanley@---s Stanley}
\index{stanley@Stanley!artichauts@Artichauts ---}
\index{あーていちよーく@アーティチョーク!すたんりー@---・スタンリー}
\index{すたんりー@スタンリー!あーていちよーく@アーティチョーク・---}
\srcBechamel{artichauts stanley}{Artichauts Stanley}{あーていちよーくすたんりー}{アーティチョーク・スタンリー}

柔らかい小ぶりのアーティチョーク20個を周囲の固い花萼をナイフでむくか、あるいは相当量のアーティチョークの基底部の周囲をナイフでむいて繊毛を取り除く。ソテー鍋にたっぷりとバターを塗る。玉ねぎ(大)2個のスライスをしっかり下茹でし、生ハムのスライス
150
gとともに鍋底に敷き詰める。鍋に蓋をして、弱火で軽く汗をかかせるイメージで加熱してから白ワインをグラス
1杯\footnote{グラス1杯 un verre de
  (アンヴェールド)の意味する具体的な量は時代や地域によって異なるが、『料理の手引き』では約1
  dL (≒ 100 ml)。つまり、1 dL = グラス1杯、という感覚がある。}注ぐ。これを煮詰めてからごく薄い\protect\hyperlink{sauce-bechamel}{ベシャメルソース}を材料が浸るまで注ぐ。

アーティチョークを茹でて、野菜料理用の深皿に盛る。

ソースを煮詰め、生クリーム 2 \(\frac{1}{2}\)
dLを加えてちょうどいい濃度にする。これを布で圧力を掛けながら漉し、バターをたっぷり加える。

アーティチョークの上からソースをかけ、加熱ハムの脂身のまったくないところを5
mm角程度のさいの目\footnote{原文
  \protect\hyperlink{salpicons-divers}{salpicon (サルピコン)}。}に刻んでアーティチョークの表面に振りかける。

\atoaki{}

\hypertarget{cromesquis-et-croquettes-d-artichauts}{%
\subsubsection{アーティチョークのクロメスキ、クロケット}\label{cromesquis-et-croquettes-d-artichauts}}

\begin{frsubenv}

Cromesquis et Croquettes d'artichauts

\end{frsubenv}

\index{artichaut@artichaut!cromesquis croquettes@Cromesquis et Croquettes d'---s}
\index{cromesqui@cromesqui!artichaut@ ---s et Croquettes d'artichauts}
\index{croquette@croquette!artichaut@Cromesquis et ---s d'artichauts}
\index{あーていちよーく@アーティチョーク!くろめすきくろけつと@---のクロメスキ、クロケット}
\index{くろめすき@クロメスキ!あーていちよーく@アーティチョークの---とクロケット}
\index{くろけつと@クロケット!あーてひちよーく@アーティチョークのクロメスキ、---}

アーティチョークの基底部だけを用いて、他の材料で作るクロメスキ、クロケットと同様に作る(\protect\hyperlink{croquettes}{前菜}参照)。

\atoaki{}

\hypertarget{croute-aux-fonds-d-artichauts}{%
\subsubsection[アーティチョークのクルート]{\texorpdfstring{アーティチョークのクルート\footnote{パンの中身をくり抜いてバターなどを内側に塗り、オーブンで乾燥させてケースとしたもの。さまざまな形状、大きさの仕立てがあるが、ここでは本文にあるマッシュルームのクルートを参考にすべきだろう。}}{アーティチョークのクルート}}\label{croute-aux-fonds-d-artichauts}}

\begin{frsubenv}

Croûte aus Fonds d'artichauts

\end{frsubenv}

\index{artichaut@artichaut!croute fonds@Croûte aux Fonds d'---}
\index{croute@croûte!fonds artichauts@--- aux Fonds d'artichauts}
\index{あーていちよーく@アーティチョーク!くるーと@---のクルート}
\index{くるーと@クルート!あーていちよーく@アーティチョークの---}

アーティチョークの基底部\footnote{周囲の花萼をすべてナイフで剥き、繊毛を取り除いた状態。\protect\hyperlink{artichaut-a-la-barigoule}{アーティチョーク・バリグール}の最初の段落を参照。}は、生のまま薄切りにし、\protect\hyperlink{blanc-pour-viandes-et-certains-legumes}{ブラン}で固めに下茹でする。取り出して湯をきる。ソテー鍋にバターをたっぷりと塗り、アーティチョークを入れて塩こしょう、ナツメグ少々でで調味し、生クリームを加えて弱火で蒸し煮\footnote{étuver
  (エチュヴェ)。}する。

アーティチョークによく火が通ったら、クリームを煮詰める。\protect\hyperlink{croute-aux-champignons}{マッシュルームのクルート}のレシピで示しているように、仕上げにバターを加えて滑らかで艶やかな仕上りにする\footnote{monter
  au beurre (モンテオブール)。}。

\atoaki{}

\hypertarget{fonds-d-artichauts-cussy}{%
\subsubsection[アーティチョーク・キュシー]{\texorpdfstring{アーティチョーク・キュシー\footnote{キュシー侯爵。\protect\hyperlink{osbservation-sur-la-sauce}{ソース概説}および訳注参照。}}{アーティチョーク・キュシー}}\label{fonds-d-artichauts-cussy}}

\begin{frsubenv}

Fonds d'artichauts Cussy

\end{frsubenv}

\index{artichaut@artichaut!fonds cussy@Fonds d'--- Cussy}
\index{cussy@cussy!fonds artichauts@Fonds d'artichauts ---}
\index{あーていちよーく@アーティチョーク!きゆしー@---・キュシー}
\index{きゆしー@キュシー!あーていちよーく@アーティチョーク・---}
\srcDemiGlace{fonds d'artichauts cussy}{Fonds d'artichauts Cussy}{あーていちよーくきゆしー}{アーティチョーク・キュシー}

小さめのアーティチョーク12個の花萼と皮をむいて基底部だけを取り出す
\footnote{レモンをこすり付けながら作業する。レモン果汁またはアスコルビン酸を加えた水に順次付けて変色を防ぐこと。}。フォワグラとトリュフ同量ずつで固めに作ったピュレをドーム状に盛り込む。これを\protect\hyperlink{sauce-villeroy}{ソース・ヴィルロワ}にくぐらせ、冷ます。冷めたら余分なソースをナイフで切り落して形を整える。イギリス式パン粉衣\footnote{小麦粉、溶き卵、パン粉の順で付けて衣にする方法。ただしパン粉は日本で一般的なものと違い、細かいものが使われるのが一般的。}、油で揚げる。ナフキンの上に盛り付け、素揚げしたパセリを添える。

別添でトマト入り\protect\hyperlink{sauce-demi-glace}{ソース・ドゥミグラス}を供する。

\atoaki{}

\hypertarget{fonds-d-artichauts-farcis}{%
\subsubsection{アーティチョーク・ファルシ}\label{fonds-d-artichauts-farcis}}

\begin{frsubenv}

Fonds d'artichauts farcis

\end{frsubenv}

\index{artichaut@artichaut!fonds farci@Fonds d'--- farcis}
\index{farci@farci(e)!fonds artichauts@Fonds d'artichauts ---s}
\index{あーていちよーく@アーティチョーク!ふあるし@---・ファルシ}
\index{きゆしー@キュシー!あーていちよーく@アーティチョーク・---}
\srcDemiGlace{fonds d'artichauts farcis}{Fonds d'artichauts farcis}{あーていちよーくふあるし}{アーティチョーク・ファルシ}

中くらいのサイズのアーティチョークを用意し、花萼と底面の皮をむいて基底部を取り出す。レモン果汁を加えた水に順次漬けていく。取り出して水気を取り、塩湯で固めに茹でる。取り出してよく湯きりをしたら、固めに作った\protect\hyperlink{duxelles-seche}{デュクセル}を詰める。バターを塗った天板に並べ、デュクセルの上に細かいパン粉を振る。溶かしバターを少量かけてやり、強火のオーブンでこんがり焼く。

別添でトマト入り\protect\hyperlink{sauce-demi-glace}{ソース・ドゥミグラス}を供する。

\atoaki{}

\hypertarget{fonds-d-artichauts-aux-pointes-d-asperges}{%
\subsubsection{アーティチョークとアスパラガスの穂先}\label{fonds-d-artichauts-aux-pointes-d-asperges}}

\begin{frsubenv}

Fonds d'artichauts aux Pointes d'asperges

\end{frsubenv}

\index{artichaut@artichaut!fonds pointe d'asperges@Fonds d'--- aux Pointes d'asperges}
\index{asperge@asperge!fonds artichauts@Fonds d'artichauts aux Pointes d'---s}
\index{あーていちよーく@アーティチョーク!あすはらかすのほさき@---とアスパラガスの穂先}
\index{あすはらかす@アスパラアス!あーていちよーく@アーティチョークと---の穂先}
\srcMornay{fonds d'artichauts aux pointes d'asperges}{Fonds d'artichauts aux pointes d'asperges}{あーていちよーくとあすはらかすのほさき}{アーティチョークとアスパラガスの穂先}

ファルシにする際と同様にアーティチョークの基底部を下処理する。これをバターで弱火にかけて蓋をして蒸し煮し\footnote{étuver
  (エチュヴェ)。}、そこに、生クリームであえて余分な水気をとばしたアスパラガスの穂先をピラミッド形に盛り込む。これをバターを塗った天板に並べ、\protect\hyperlink{sauce-mornay}{ソース・モルネー}をかかて強火のオーブンでこんがり焼く。

\atoaki{}

\hypertarget{fonds-d-artichauts-sautes}{%
\subsubsection{アーティチョーク基底部のソテー}\label{fonds-d-artichauts-sautes}}

\begin{frsubenv}

Fonds d'artichauts sautés

\end{frsubenv}

\index{artichaut@artichaut!Fonds sautes@Fonds d'--- sautés}
\index{saute@sauté!fonds artichauts@Fonds d'artichauts sauté}
\index{あーていちよーく@アーティチョーク!そてー@---基底部のソテー}
\index{そてー@ソテー!あーていちよーく@アーティチョーク基底部のソテー}

アーティチョークの花萼と繊毛をすべて取り除く。ナイフで皮をむき、生のまま薄切りにする\footnote{過熟のものでは固くて困難。比較的若どりで新鮮なものを選ぶこと。}。

塩こしょうで調味し、バターでソテーする。野菜料理用の深皿に盛り、上から香草を散らす。

\atoaki{}

\hypertarget{quartiers-d-artichauts-dietrich}{%
\subsubsection[アーティチョーク・ディートリッヒ]{\texorpdfstring{アーティチョーク・ディートリッヒ\footnote{ここではドイツ語風にカナ書きしたが、フランス式にディエトリックあるいはディエトリッシュ、あるいはイタリア語風にディエトリーチェとすべきかも知れない。いずれにしても
  Dietrich はフランスの Thierry
  (ティエリ)に相当する非常にありれた男性名であり、家名でもある。いうまでもなく、女優、歌手として有名なマレーネ・ディートリッヒ(1901〜
  1992)ではない。このレシピは初版から掲載されているから明らかに彼女の名を冠したものではない。このレシピのポイントは、アーティチョークの仕立てそのものも、合わせるリゾットもきわめてイタリア風であるところにヒントがあるのだろう。}}{アーティチョーク・ディートリッヒ}}\label{quartiers-d-artichauts-dietrich}}

\begin{frsubenv}

Quartiers d'artichauts Diétrich

\end{frsubenv}

\index{artichaut@artichaut!quarthiers dietrich@Quarthier d'--- Diétrich}
\index{dietrich@Diétrich!quartiers artichauts@Quartier d'artichauts ---}
\index{あーていちよーく@アーティチョーク!ていーとりつひ@---・ディートリッヒ}
\index{ていーとりつひ@ディートリッヒ!あーていちよーく@アーティチョーク・---}
\srcVeloute{quartier d'artichauts dietrich}{Quartiers d'artichauts Diétrich}{あーていちよーくていーとりつひ}{アーティチョーク・ディートリッヒ}

プロヴァンス産アーティチョーク12個は上 \(\frac{1}{3}\)
程を切り落し、周囲の固い花萼と茎があれば茎の皮をナイフでアーティチョークを回すようにしてむく。これを縦4つに切り、玉ねぎのみじん切りを加えてバターを熱した鍋に入れ、強火で色よく炒める。軽く仕上げた\protect\hyperlink{veloute}{ヴルテ}とマッシュルームの茹で汁を加えて弱火で煮込む。

ピエモンテ産トリュフ\footnote{いうまでもなく白トリュフ。}風味のリゾットを型に詰めて作った縁飾りの中央に、アーティチョークを盛り込み、ソースを煮詰めて仕上げに生クリーム少々で調えてソースを仕上げる。これをアーティチョークの上からかけてやる。

\atoaki{}

\hypertarget{quartiers-d-artichauts-a-l-italienne}{%
\subsubsection[アーティチョーク・イタリア風]{\texorpdfstring{アーティチョーク・イタリア風\footnote{イタリア風ソースを用いているからこの料理名だが、内容的にはアーティチョークのブレゼに他ならない。\protect\hyperlink{braisage-des-legumes}{野菜のブレゼ}は肉料理における\protect\hyperlink{es-braises-ordinaires}{ブレゼ}が基本になっているため、\protect\hyperlink{le-fonds-de-braise}{フォンドブレーズ}も肉料理のものに準じていることに注意。また、『料理の手引き』におけるブレゼは基本的に豚背脂のシートで素材を包むか上下から挟むようにしてフォンなどの液体を注いで鍋に蓋をしてオーブンで加熱調理することに注意。間違っても単なる蒸し煮ではない。}}{アーティチョーク・イタリア風}}\label{quartiers-d-artichauts-a-l-italienne}}

\begin{frsubenv}

Quartiers d'artichauts à l'Italienne

\end{frsubenv}

\index{artichaut@artichaut!quarthiers italienne@Quarthier d'--- à l'italienne}
\index{italien@italien(ne)!quartiers artichauts@Quartier d'artichauts ---ne}
\index{あーていちよーく@アーティチョーク!いたりあふう@---・イタリア風}
\index{いたりあふう@イタリア風!あーていちよーく@アーティチョーク・---}
\srcItalienne{quartiers artichauts itaienne}{Quartiers d'artichauts à l'Italienne}{あーていちよーくいたりあふう}{アーティチョーク・イタリア風}

中くらいのサイズのアーティチョークは花蕾の上から \(\frac{1}{3}\)
くらいを切り落し、外側の固い花蕾をナイフで剥く。茎も皮をむいて筋を取り除き、きれいな形に整える。これを縦4つに櫛切りにする。中心の繊毛をきれいに取り除く。レモンの切れ端をこすり付け、黒ずむのを防ぐ。下処理が済んだ順に冷水に入れていく。下茹でしたら取り出して湯をきる。これを鍋に敷き詰めた\protect\hyperlink{fonds-de-braise}{フォンドブレーズ}の上に並べる。蓋をして低温のオーブンに7〜8分間入れて野菜が汗をかくようなイメージで加熱する。白ワインを注いで煮詰めたら、\protect\hyperlink{fonds-brun}{茶色いフォン}をアーティチョークの高さまで注ぐ。低めの温度のオーブンに入れて、アーティチョークがすっかり柔らかくなるまでじっくり火を通す。

提供直前に、野菜料理用の深皿に盛る。煮汁は漉して浮き脂を取り除き、煮詰めてから\protect\hyperlink{sauce-italienne}{イタリア風ソース}に加え、アーティチョークにかける。

\atoaki{}

\hypertarget{quartiers-d-artichauts-a-la-lyonnaise}{%
\subsubsection{アーティチョーク・リヨン風}\label{quartiers-d-artichauts-a-la-lyonnaise}}

\begin{frsubenv}

Quartiers d'artichauts à la Lyonnaise

\end{frsubenv}

\index{artichaut@artichaut!quarthiers lyonnaise@Quarthier d'--- à la Lyonnaise}
\index{lyonnais@lyonnais(e)!quartiers artichauts@Quartier d'artichauts à la ---e}
\index{あーていちよーく@アーティチョーク!りよんあふう@---・リヨン風}
\index{りよんふう@リヨン風!あーていちよーく@アーティチョーク・---}

玉ねぎをみじん切りにしてバターで炒め、ソテー鍋の底に敷き詰める。その上に、櫛切りにして下茹でしたアーティチョークを並べる。その後は\protect\hyperlink{quartiers-d-artichauts-a-l-italienne}{アーティチョーク・イタリア風}と同様の手順で調理する。

野菜料理用の深皿に盛り、煮汁を煮詰めてソースにする。ソースは火から外してバターを加えて仕上げること。これをアーティチョークにかけ、パセリのみじん切りを散らす。

\atoaki{}

\hypertarget{puree-ou-creme-d-artichauts}{%
\subsubsection{アーティチョークのピュレ、クリーム}\label{puree-ou-creme-d-artichauts}}

\begin{frsubenv}

Purée ou Crèpe d'artichauts

\end{frsubenv}

\index{artichaut@artichaut!puree creme@Purée ou Crèpe d'---}
\index{puree@purée!artichauts@--- ou Crème d'artichauts}
\index{creme@crème!artichauts@Purée ou --- d'artichauts}
\index{あーていちよーく@アーティチョーク!ひゆれくりーむ@---のピュレ、クリーム}
\index{ひゆれ@ピュレ!あーていちよーく@アーティチョークの---、クリーム}
\index{くりーむ@クリーム!あーていちよーく@アーティチョークのピュレ、---}

アーティチョークは充分に柔らかいものを選ぶこと。花萼をすべて除去し、基底部だけを取り出す。これを白さが失なわれないよう注意して、軽く下茹でする。これを別鍋に移し、バターを加えて蓋をして弱火で蒸し煮するようにして完全に火を通す。火が通ったら、加熱に使ったバターとともに、目の細かい網で裏漉しする。こうして出来たアーティチョークのピュレを片手鍋に入れ、きめ細かく柔らかいじゃがいものピュレ同量を合わせる。

仕上げに、フレッシュなバターと焦がしバターそれぞれ若干量を加える。焦がしバターを加えるのはアーティチョークの風味を引き立てるのが目的。

\index{artichaut@artichaut|)}
\index{あーていちよーく@アーティチョーク|)}

\end{recette}
