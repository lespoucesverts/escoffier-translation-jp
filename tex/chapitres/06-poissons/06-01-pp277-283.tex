\href{✓原稿下準備なし}{} \href{訳と注釈\%2020180420進行中}{}
\href{未、原文対照チェック}{} \href{未、日本語表現校正}{}
\href{未、注釈チェク}{} \href{未、原稿最終校正}{}

\hypertarget{poissons}{%
\chapter{VI 魚料理 Poissons}\label{poissons}}

\hypertarget{serie-de-courts-bouillons-de-poisson}{%
\section{クールブイヨン}\label{serie-de-courts-bouillons-de-poisson}}

\begin{recette}
\hypertarget{ux30afux30fcux30ebux30d6ux30a4ux30e8ux30f3-a}{%
\subsubsection{クールブイヨン
A}\label{ux30afux30fcux30ebux30d6ux30a4ux30e8ux30f3-a}}

水5 L に対し、ヴィネガー2.5 dL、粗塩60 g、薄切りにしたにんじん600
gと玉ねぎ500 g、タイム1枝、ローリエの小さな葉2枚、パセリの茎100
g、粒こしょう20
g(こしょうを加えるのはクールブイヨンを漉す10分前)。材料を全て鍋に入れ、火にかけて1時間弱火で煮、漉す(原書
p.277)。

\hypertarget{ux30afux30fcux30ebux30d6ux30a4ux30e8ux30f3-b-4-ux9c52ux3046ux306aux304eux30d6ux30edux30b7ux30a7ux7b49-ux539fux66f8p.277}{%
\subsubsection[クールブイヨン B (鱒、うなぎ、ブロシェ等)
原書p.277]{\texorpdfstring{クールブイヨン B \footnote{クールブイヨンは用途に応じ、AからEまでの5種が挙げられている(原書pp.277-278)。}
(鱒、うなぎ、ブロシェ等)
原書p.277}{クールブイヨン B  (鱒、うなぎ、ブロシェ等) 原書p.277}}\label{ux30afux30fcux30ebux30d6ux30a4ux30e8ux30f3-b-4-ux9c52ux3046ux306aux304eux30d6ux30edux30b7ux30a7ux7b49-ux539fux66f8p.277}}

5 L 分の材料\ldots{}\ldots{}白ワイン2.5 L 。水2.5 L
。薄切りにした玉ねぎ600 g。パセリの茎80g
。タイムの小枝1本。ローリエの葉(小) \(\frac{1}{2}\) 枚。粗塩
60g。大粒のこしょう15 g(クールブイヨンを漉す10分前に加える)。

作業手順\ldots{}\ldots{}作業:液体、香味素材、調味料を鍋に入れ、沸かす。弱火で30分程煮て、漉す。

\href{欠落アリ}{}

原注:クールブイヨンBとC\footnote{クールブイヨンBの白ワインを赤ワインに代え、香味素材としてにんじん400gを加える。鱒、鯉、マトロート用(原書pp.277-278)。}で調理した魚はクールブイヨン添えとして供する。つまり、少量の煮汁とクールブイヨンに用いた野菜を添える。野菜はよく火が通っていること。煮汁はしっかり煮詰め、提供直前に新鮮なバター少量を加えて仕上げる。
\end{recette}
\hypertarget{ux30afux30fcux30ebux30d6ux30a4ux30e8ux30f3ux306eux4f7fux3044ux65b9-ux539fux66f8-p.278}{%
\subsection{クールブイヨンの使い方 原書
p.278}\label{ux30afux30fcux30ebux30d6ux30a4ux30e8ux30f3ux306eux4f7fux3044ux65b9-ux539fux66f8-p.278}}

\begin{enumerate}
\def\labelenumi{\arabic{enumi}.}
\item
  加熱時間が30分以内の場合は、クールブイヨンは必ず事前に用意しておくこと。
\item
  加熱時間が30分を越える場合は、クールブイヨンの材料は冷たい状態のままで合わせせておく。香味素材はポワソニエールの網の下に入れる。
\item
  ごく少量のクールブイヨンでポシェ\footnote{原文 pochage à court
    mouillement『ル・ギード・キュリネール』では、この表現はテュルボタン(小型のテュルボ)、バルビュ、舌びらめ等の平たい魚をポシェする際に用いられる。本連載「舌びらめのボヌ・ファム」
    2011年3月号pp.110-111 参照。}する場合、材料は(白または赤ワインを含む場合も)魚を火にかける際に合わせる。クールブイヨンの量は魚の
  \(\frac{1}{3}\)
  の高さとし、加熱中ひんぱんに煮汁を魚にかけてやること\footnote{arroser
    アロゼ。}。この調理法の場合は通常、クールブイヨンは上で記したように
  \footnote{「クールブイヨンB」原注。}、提供直前に軽くバターを加えて仕上げ、魚に添える。
\item
  冷製にする場合は、必ずクールブイヨンに魚が浸った状態で冷ますこと。当然ながら、火にかけている時間は短かくなる\footnote{余熱で火が通るため。}。
\end{enumerate}

\hypertarget{ux539fux6ce8}{%
\subparagraph{【原注】}\label{ux539fux6ce8}}

いくつかの魚種の加熱時間は該当する項で示してある。

\hypertarget{ux9b5aux306eux8abfux7406ux6cd5}{%
\section{魚の調理法}\label{ux9b5aux306eux8abfux7406ux6cd5}}

魚料理は全て、下記のいずれかの調理法による。

\begin{enumerate}
\def\labelenumi{\arabic{enumi}.}
\item
  塩水(湯)またはクールブイヨン\footnote{court-bouillon直訳は「量の少ない煮汁」。魚の他、甲殻類、鶏などの白身肉、野菜などをポシェするのに用いる。とりわけ魚や鶏を丸ごとポシェする場合には、その名称のとおり、できるだけ少量でポシェする必要がある。また、ポシェに用いたクールブイヨンをベースにソースを作る場合が多い。}Bを用いたポシェ\ldots{}\ldots{}大きな魚丸ごと、および切り身。
\item
  ごく少量のクールブイヨンを用いたポシェ\ldots{}\ldots{}魚のフィレ、またはやや小さい魚。
\item
  ブレゼ\ldots{}\ldots{}もっぱら大きな魚。
\item
  オ・ブルー\footnote{比較的小さめの淡水魚に主として用いられる調理法。生きたままの魚の表面のぬめりをとらないように洗い、内臓を取り除いたらすぐに、塩とヴィネガーを加えたクールブイヨンで茹でる。冷製、温製どちらでも供する。原書p.281参照。}\ldots{}\ldots{}とりわけ\ruby{鱒}{ます}、鯉、ブロシェ\footnote{川かますの一種。本連載「ブロシェのクネル」2011年10月号
    pp.124-125 参照。}に合う。
\item
  揚げもの\ldots{}\ldots{}もっぱら小さい魚、切り身。
\item
  ムニエール\ldots{}\ldots{}揚げものにするのと同じ小さい魚、切り身。
\item
  グリエ\ldots{}\ldots{}小さい魚、および切り身。
\item
  グラタン\ldots{}\ldots{}小さい魚、切り身。
\end{enumerate}

\hypertarget{ux5869ux6c34ux6e6fux304aux3088ux3073ux30afux30fcux30ebux30d6ux30a4ux30e8ux30f3bux3092ux7528ux3044ux305fux52a0ux71b1ux8abfux7406}{%
\subsection{塩水(湯)およびクールブイヨンBを用いた加熱調理}\label{ux5869ux6c34ux6e6fux304aux3088ux3073ux30afux30fcux30ebux30d6ux30a4ux30e8ux30f3bux3092ux7528ux3044ux305fux52a0ux71b1ux8abfux7406}}

魚を丸ごと調理する場合は、魚に合ったポワソニエール\footnote{大きな魚を丸ごと煮るための細長い鍋。魚の形を崩さずに取り出せるよう、中に専用の網を敷いて使う。似たものに、舌びらめ等の平たい魚にぴったり合う菱形をしたテュルボティエールがある。いずれも、できるだけ少量の煮汁で魚を加熱できるように工夫されたもの。(図参照)}を用いる。魚を掃除し(テュルボは水にさらして血抜きをし)、ひれ等を切り落して形を整え、ポワソニエールの網に乗せる。魚種に応じて塩水または冷たいクールブイヨンをかぶるまで注ぐ。強火にかけて沸騰したらすぐにレンジの火の弱いところに鍋を移動させ、ポシェする。

切り身(薄すぎは絶対にいけない)の場合、沸騰した液体(塩湯またはクールブイヨン)に投入したらすぐにレンジの火の弱いところに鍋を移動させ、沸騰しない程度の温度でゆっくりと火を通す。

こうするのは、魚の身のエキスを閉じこめるためである。冷水から火にかけた場合にはエキスの大部分が流れ出してしまう。大きな魚丸ごとの場合にはこのやり方はしない。沸騰した液体に魚を投入すると身が収縮するので、大きな魚の場合は身が割れたり形が崩れたりするからだ。

塩湯あるいはクールブイヨンでポシェした魚は、ナフキンまたは専用の網に盛る。周囲をパセリで飾り、塩茹でしたじゃがいもと1種類または数種のソースを添えて供する。ガルニテュールがパセリのみの場合、魚の周囲にパセリを飾るのは客に料理を見せる\footnote{当時の宴席で主流だったロシア式サーヴィスでは、大きな銀盆に盛った料理をまず食客に見せてから、とり分けて給仕する。}直前にすること。どんな場合でも、ガルニテュールを添えたらクロッシュ\footnote{銀または陶製の保温用皿カバー。}は被せないこと。

\hypertarget{ux3054ux304fux5c11ux91cfux306eux30afux30fcux30ebux30d6ux30a4ux30e8ux30f3ux3092ux7528ux3044ux305fux30ddux30b7ux30a7-ux539fux66f8-pp.279-280}{%
\subsection{ごく少量のクールブイヨンを用いたポシェ 原書
pp.279-280}\label{ux3054ux304fux5c11ux91cfux306eux30afux30fcux30ebux30d6ux30a4ux30e8ux30f3ux3092ux7528ux3044ux305fux30ddux30b7ux30a7-ux539fux66f8-pp.279-280}}

この火入れの方法は主としてテュルボタン、バルビュ、舌びらめ、丸ごとおよびそれぞれの魚のフィレで用いる。バターを塗った天板あるいはソテ鍋に魚丸ごとあるいはそのフィレを置き、軽く塩をして、所要量の魚のフュメかマッシュルームの煮汁を注ぐ。フュメとマッシュルームの煮汁を合わせたものを用いる場合もある。蓋をして、中温のオーヴンに入れる。魚丸ごとの場合は時折煮汁をかけてやる。

魚(丸ごとでもフィレでも)に火が通ったら、注意して汁気をきり、皿に盛る。ガルニテュールを含む料理の場合、ガルニテュールを魚の周囲に盛り、ソースをかける\footnote{原文通りの順で訳したが、実際にはソースをかけてからガルニテュールを盛ったほうが良い場合もあるだろう。}。多くの場合、魚の煮汁を煮詰めてソースに加える。

\href{欠落アリ}{}

\hypertarget{ux9b5aux306eux30d6ux30ecux30bc17-ux539fux66f8-p.280}{%
\subsection[魚のブレゼ 原書 p.280]{\texorpdfstring{魚のブレゼ\footnote{ここではブレゼの語が限定的な意味で用いられていることに注意。牛のアロワイヨのような大きな塊肉のブレゼと同様の調理法、ということである。それは、香味素材を色づくまで炒めてから用いることや、主素材に豚背脂やトリュフをピケ針で差したり、豚背脂のシートで覆って加熱するという点によく表れている。ただし、これらは必須というわけではないため、事実上は「ごく少量のクールブイヨンを用いたポシェ」と区別がつきにくい。実際、モンタニェ『ラルース・ガストロノミーク』初版では、魚のブレゼについて「本来的な意味でのブレゼというよりは、ごく少量のクールブイヨンを用いたポシェである」と述べられている。逆に言えば、こんにち魚の加熱方法についてしばしば「ブレゼ」と呼ばれているものが、エスコフィエやモンタニェにおいては「少量のクールブイヨンを用いたポシェ」と表現されていたということである。}
原書
p.280}{魚のブレゼ 原書 p.280}}\label{ux9b5aux306eux30d6ux30ecux30bc17-ux539fux66f8-p.280}}

この調理法を用いるのは通常、丸ごとまたは筒切りにした鮭、大ぶりの鱒、テュルボ、テュルボタンのうち大きなもの、である。

場合によっては、魚の片面に、小さく切った豚背脂、トリュフ、コルニション、にんじん等をピケ針で差し込む。

香味素材等\footnote{原文 fonds de braisage フォン・ド・ブレザージュ
  (fonds de
  braiseフォン・ド・ブレーズ、とも)。通常は、厚い輪切りにしたにんじんと玉ねぎをバターか獣脂で色づくまで炒め、ブーケガルニ、下茹でした豚皮を合わせる(原書pp.394-395)。また、これを用いた煮汁のことも指す。}は肉料理のブレゼの場合と同じように用意するが、豚皮は用いない。提供方法に応じて、白または赤ワインと軽い魚のフュメ同量ずつを、魚の厚みの
\(\frac{3}{4}\) またはひたひたの高さまで注ぐ。厳密に肉断ち \footnote{カトリックの生活習慣として、四旬節(復活祭までの46日間)および週1
  回程度、肉類を食べないということが行なわれた(本連載2012年5月号「ソース・エスパニョル(4)」p.110、訳注1参照)。}のための仕立てにする場合を除いて、薄くスライスした豚背脂のシートを魚にかぶせる。加熱中\footnote{鍋を火にかけ、沸騰したら蓋をして中火のオーヴンに入れ、加熱する。}こまめに煮汁を魚にかけてやる\footnote{arroser
  アロゼ。}。また、完全には蓋をせず、加熱中に煮汁が煮詰まるようにしてやる。

ほぼ火が通ったら、鍋の蓋をとり、魚にかけた煮汁の水分をオーヴンの熱で蒸発させて表面につやを出す\footnote{glacer
  グラセ。}。魚を鍋から出して汁気をきり、皿に盛り保温しておく。

煮汁\footnote{原文 fonds de braisage (訳注2参照)。}を漉し、しばらく休ませたら浮き脂を取り除き、必要なら煮詰める。これを加えてソースを仕上げる。

魚のブレゼには通常、各ルセットに示してあるガルニテュールを添える。

\hypertarget{ux30aaux30d6ux30ebux30fc36-ux539fux66f8-pp.280-281}{%
\subsection[オ・ブルー 原書
pp.280-281]{\texorpdfstring{オ・ブルー\footnote{au bleu
  ヴィネガーを加えることで魚の表面のぬめりが青みがかることから。} 原書
pp.280-281}{オ・ブルー 原書 pp.280-281}}\label{ux30aaux30d6ux30ebux30fc36-ux539fux66f8-pp.280-281}}

オ・ブルーは鱒、鯉、ブロシェ\footnote{川かますの一種。}のみに用いられる特殊な調理法で、基本的なポイントは以下のとおり。

\begin{enumerate}
\def\labelenumi{\arabic{enumi}.}
\item
  必ず、生きた魚を使う。
\item
  魚の表面のぬめりをとらないように、なるべく手で触れずに、わたを抜く。鱗も引かない。
\item
  魚が大きい場合は、専用の網を敷いたポワソニーエルに入れ、「沸騰したヴィネガーをかける」。ヴィネガーは、通常のクールブイヨンに加える分量\footnote{以下の「クールブイヨンA」の分量比率を参照。}。次に、ヴィネガーを入れずに用意した温かい\footnote{原文
    tiède ぬるい、温かい。}クールブイヨンを注ぎ入れる。これは、なるべく身が割れないようにするためである。その後は通常どおり加熱する\footnote{レンジで沸騰させたらオーヴンに入れる。}。
\item
  小さい鱒の場合は、生きたままのものを手早く中抜きし、塩、ヴィネガーを加えただけの沸騰したクールブイヨンで煮る。
\item
  オ・ブルーは冷製、温製どちらの仕立てにしてもいい。実際の作り方の項で示してあるソースを添えて供する。
\end{enumerate}

\href{欠落アリ}{}

\hypertarget{ux30e0ux30cbux30a8ux30fcux30eb43-ux539fux66f8-p.282}{%
\subsection[ムニエール 原書 p.282]{\texorpdfstring{ムニエール\footnote{à
  la meunière 「粉挽き職人風」の意。} 原書
p.282}{ムニエール 原書 p.282}}\label{ux30e0ux30cbux30a8ux30fcux30eb43-ux539fux66f8-p.282}}

ムニエールは素晴らしい調理法だが、小型の魚と、大きな魚の場合は切り身にしか用いない。とはいえ、丁寧にやれば1.5
kg以下のテュルボタンはムニエールで調理できる。

魚丸ごと、あるいは切り身、フィレに味つけをして小麦粉をまぶし、バターを熱したフライパンで焼く。

魚が小さい場合は普通のバターでいいが、大きい場合は澄ましバターを使った方がいい。

魚の両面を焼き、程良く火が通ったら、予め熱しておいた皿に盛る。

飾り切りにした半割りのレモンを添えて、そのまま供することも可能である。ただし、このような提供方法の場合は本来の「ムニエール」と区別するために「黄金色に焼いた\footnote{doré
  (ドレ)
  一般的な色の表現として「黄金色」の意だが、ムニエールの場合、通常は大きい魚についてのみこの表現を用いる。}」と表現する。

「ムニエール」の場合には、焼き上がった魚に少量のレモン汁をふり、塩、こしょう少々で味を整える。粗みじん切りにして湯通ししたパセリを魚の表面に散らし、焦がしバターをかけてすぐに供する。湯通ししたパセリの水分に熱いバターが触れて泡がたつので、それが消えないうちに客に料理を見せるようにする。
