\begin{main}

\hypertarget{asperges}{%
\subsection{アスパラガス}\label{asperges}}

\begin{frsecbenv}

Asperges

\end{frsecbenv}

アスパラガスは主に4種類\footnote{variété
  (ヴァリエテ)野菜についての場合通常は「品種」と訳すが、ここではバラエティ、種類くらいの意。なお、いわゆる「アスパラソバージュ」asperges
  sauvages
  (アスペルジュソヴァージュ)はまったくの別種であり、ここには含まれていない。}に分けられる。フランス産アスパラガスの典型的な品種、アルジャントゥイユ\footnote{Argenteuil
  (アルジョントゥイユ)。パリ近郊の地名。かつてここでアスパラガスの生産が盛んだったという。またこの地名を冠した品種が21
  世紀になった現在でも主流であることは事実。穂先がやや紫がかる傾向にあるが基本的には緑色の品種。}。グリーンアスパラガス\footnote{現在ではオランダの種苗会社が交配したF1品種が増えている。}。ジェノヴァ産紫アスパラガス\footnote{同上。ただしこの紫色は茹でると失なわれる。}、イタリア産アスパラガスの典型で繊細な風味だがややえぐ味がある。ベルギー産ホワイトアスパラガス\footnote{ホワイトアスパラガスは「品種」ではなく栽培方法が異なる(軟白の工程が入る)。理屈のうえではどんな品種であってもホワイトアスパラガスにすることは可能。秋のうちに地上部を切り落し、根株を中心にプラウ(鋤の一種)などを用いて土を盛り上げる。全体に大きなかまぼこのような格好になる。春になると、地中深くの根株から伸びてきたアスパラガスの茎は日光に当たっていないので軟白されている。それを地上に出る直前に収穫する。収穫は一定期間で終了させ、盛り上げた土を平らに戻し、緑の茎葉を茂らせて翌年のために根株を養成する。トラクタがプラウを曳けるようかなり条間を広くとる必要があり、単位面積あたりの収量は低い。かつては缶詰の材料として北海道で盛んに栽培されていたが、だんだん生産量が落ちている。近年日本ではハウス栽培で土盛りをせずトンネルに遮光率100%のシートで暗闇を作って栽培する方式が増えつつある。いずれもフランス料理においてあまり高い評価を得られていないのは、品種の選定と栽培方法に負うところが大きいだろう。}、これも繊細な風味だが、輸送による劣化がはげしい。

アスパラガスは出来るだけ新鮮なものを用いること。丁寧に皮をむき、手早く洗う。紐で束ねてたっぷりの塩湯で茹でる。

種類によっては多少エグ味があるので、茹であがったらすぐに水を替えてエグ味を取り除く。少なくともエグ味を弱めることは可能だ。茹でたアスパラガスは専用の銀の網の上に盛るか、ナフキンの上に盛り付ける。

\end{main}

\begin{recette}

\hypertarget{asperges-a-la-flamande}{%
\subsubsection[アスパラガス・フランドル風]{\texorpdfstring{アスパラガス・フランドル風\footnote{現在のベルギー西部からフランス北部にかけて
  の北海に面する地域。フランダース。\protect\hyperlink{garniture-a-la-flamande}{ガルニチュール・フランドル風}訳注参照。本文ではとくに指定されていないが、概説部分に「ベルギー産ホワイトアスパラガス」の説明があること、初版の料理名が「ホワイトアスパラガス・フランドル風」であることを考慮する必要がある。}}{アスパラガス・フランドル風}}\label{asperges-a-la-flamande}}

\begin{frsubenv}

Asperges à la Flamande

\end{frsubenv}

フランドル地方の流儀では1人あたり熱々の半熟卵1個とバター 30 gを添える。

食べ手自身が、半熟の黄身をすり潰してから、塩こしょうで調味し、バターを加えて食する。

もちろん、この半熟卵とバターのソースをあらかじめ作っておき、ソース入れで添えて供してもいい。

\end{recette}
