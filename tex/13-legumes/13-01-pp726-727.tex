\begin{main}

\hypertarget{legumes-farineux-et-pates-alimentaires}{%
\chapter{XIII. 野菜料理・パスタ}\label{legumes-farineux-et-pates-alimentaires}}

\begin{frchapenv}

Légumes --- Farineux et Pâtes alimentaires

\end{frchapenv}

\hypertarget{serie-des-legumes}{%
\section{野菜料理}\label{serie-des-legumes}}

\begin{frsecenv}

Série des Légumes

\end{frsecenv}

\begin{center}
\headfont\large 野菜の仕込みにおける注意事項\label{observations-sur-les-operations-preliminaires}

\normalfont\textit{Observation sur les Opérations préliminaires.}
\end{center}

\normalfont\normalsize

\hypertarget{blanchissage}{%
\subsection[下茹で]{\texorpdfstring{下茹で\footnote{blanchir(ブロンシール)。もともとの意味は「白くする」。古代ローマ時代から17世紀頃まで、肉類は最初に下茹でしてから調理するのが一般的だった。赤身肉を茹でると表面が白くなることからこの用語が用いられるようになった。野菜にかぎらず、素材によっては白く茹であげるために単なる湯ではなく「ブラン」を用いる場合もある。以下は本書、肉料理の概説部分にあるブランの要旨。水
  1
  Lに大さじ1杯強の小麦粉、塩6gとスプーン2杯の溶いて沸かす。クローヴを刺した玉ねぎ1個とブーケガルニ、下茹でする素材、さらに空気に素材が触れて変色するのを防ぐために獣脂を加える。脂は牛あるいは仔牛の生のケンネ脂を細かく刻んだものを使う。脂を事前に冷水にさらして血等の不純物が残っていないようにしておくこと。また、原注において、野菜の下茹で用のブランには,ヴィネガーではなくレモン汁を用いたほうがいいと述べてある(\protect\hyperlink{blanc-pour-viandes-et-certains-legumes}{肉およびある種の野菜に用いるブラン}
  参照。)}}{下茹で}}\label{blanchissage}}

\begin{frsecbenv}

Blanchissage

\end{frsecbenv}

この作業を行なう場合、2つのケースに分けられる。第1に、例えばほうれんそう、プチポワ、アリコヴェール等の一般的な青物野菜の場合、完全に火を通すのが目的。たっぷりの湯で手早く茹で、クロロフィルすなわち葉緑素を失わないよう注意すること。第2は、野菜に本来あるえぐ味を消す目的\footnote{いわゆる「アク」だが、例えばサヴォイキャベツの場合など、しっかり下茹でをしないと、後の調理の段階で変色することがある。厳冬期に霜に何度もあたったものなら下茹で時間は数分〜10分程度で済むが、それ以外の時期のものは2時間〜3時間の下茹でが必要(丸ごと下茹でする場合)。冬季の霜にあたった状態を人為的に作りだす、つまり冷凍庫に入れて水分が凍る際に膨張する力を利用して細胞壁を破壊すれば、下茹で時間は少なくて済む。シコレの場合、軟白されているものはそのまま生食可能なくらいアクも苦味も少ないので、ここでは軟白されていない緑のものを指している。セロリは品種によって風味やアク、苦味の強さが違うので注意。とりわけ日本ではコーネルという生食に適した品種の栽培が多いため、下茹ではあまりしっかり行なう必要もないだろう。逆に、サルシフィなどは変色を防ぐためにレモン果汁を用いるが、この変色はアクではなく、同じキク科の野菜であるレタス類の切り口が空気に触れて変色するのとまったく同じ現象。ただし、サルシフィは長時間加熱することで線維が柔らかく美味しくなるので、調理によっては充分に下茹でをしたほうがいい。また、ほうれんそうやブレットなどアカザ科の野菜にはシュウ酸が含まれていて結石の原因となるが、これも下茹ですることで容易にほとんどを除去出来るので下茹でが必須。いっぽう、プチポワのごく若どりのものについては新鮮なものに限られるが少量なら生食可能。冷凍品や大きく育った豆は下茹でが必須。アリコヴェールも同様に、わずかだがレクチンが含まれているので下茹で必須。これは他の豆類も同様なので注意。}。例えばキャベツ、セロリ、シコレ等。原則的に、若どりの野菜は下茹でしない。下茹でで完全に火を通してしまう野菜については、1リットルあたり7gの塩を湯に加えること。

\hypertarget{rafraichissage}{%
\subsection[冷水にはなす]{\texorpdfstring{冷水にはなす\footnote{rafraîchir
  (ラフレシール)。原義は、リフレッシュさせる。}}{冷水にはなす}}\label{rafraichissage}}

\begin{frsecbenv}

Rafraîchissage

\end{frsecbenv}

下茹で後に冷水にとるのは、野菜をブレゼにする場合と、オペレーションの都合から事前に茹でておかなければならない場合だけにすること。また、後でバターやクリームであえる場合、冷水にとると風味が負けてしまうので注意。

\hypertarget{cuisson-des-legumes-a-l-anglaise}{%
\subsection[アラングレーズ]{\texorpdfstring{アラングレーズ\footnote{à
  l'anglaise
  (アロングレーズ)。イギリス風、の意だが、常識的に考えて、イギリスにおいてのみ野菜をただ茹でるという調理法が一般的というわけではない。1907年に刊行された本書の英語版
  \emph{A guide to modern cookery}
  において、イギリス式パン粉衣については pané à l'anglaise, treated à
  l'anglaise として説明がなされているが
  (pp.70-71)、野菜類を茹ですることについての言及は見られない。つまり、この表現が通用したのはフランスにおいてのみ、ということになる。}}{アラングレーズ}}\label{cuisson-des-legumes-a-l-anglaise}}

\begin{frsecbenv}

Cuisson des légumes à l'anglaise

\end{frsecbenv}

沸騰した湯で茹でるだけでいい。次によく湯切りをして、さらに水気をとばす。野菜料理用の深皿に盛りつけ、貝殻形のバターを添えて供する。味付けは食べ手自身に行なっていただく。葉物野菜なら何でもこのアラングレーズで調理、提供可能。

\hypertarget{cuisson-des-legumes-secs}{%
\subsection[乾燥豆]{\texorpdfstring{乾燥豆\footnote{légume
  (レギューム)という語は古い時代には「豆類」を意味していた。その他の野菜は
  feuilles (フイユ、葉)とか racine
  (ラシーヌ、根=根菜)などと呼ばれることが多かった。légume
  が野菜の総称となったのは比較的新しい時代のことであり、現代でもイタリア語ではlégumes
  に対応する legumi
  (レグーミ)という語は「豆類」を意味する。なお、イタリア語での野菜の総称は
  ortaggi (オルタッジ、複数形)。}}{乾燥豆}}\label{cuisson-des-legumes-secs}}

\begin{frsecbenv}

Cuisson des légumes secs

\end{frsecbenv}

乾燥豆を水に漬けてもどすのはよろしくない。その年に穫れた良質のものなら、水から弱火でゆっくり沸かして茹でるだけでいい。あく取りをして香味野菜
\footnote{一般的にはクローヴを刺した玉ねぎと縦に四つ割りにしたにんじん、ブーケガルニ。}を加え、蓋をしてごく弱火で茹でる。

あまりにも古い豆や品質が劣るものはあらかじめ水でもどしてもいい。ただし、豆が膨れるのに必要な時間きっかり、つまり1時間半から2時間程度に留めること。

何時間も水につけておくと、発酵が始まってしまう。そうなる美味さもほとんどなくなってしまうし、豆の組織が損なわれて使いものにならなくなってしまうことさえある。

\hypertarget{braisage-des-legumes}{%
\subsection[野菜のブレゼ]{\texorpdfstring{野菜のブレゼ\footnote{よく誤解されがちなものなので注意したい。}}{野菜のブレゼ}}\label{braisage-des-legumes}}

\begin{frsecbenv}

Braisage des légumes

\end{frsecbenv}

野菜は事前に湯がいて冷水にとり、その後に余分な葉などを切り落して成形する。\protect\hyperlink{braisage}{肉料理のブレゼ}と同様に、鍋の底と周囲に豚背脂のシートを張り、野菜を入れる。上面を背脂のシートで覆う。鍋に蓋をして弱火でかるく汗をかかせるように蒸し煮\footnote{suer
  (スュエ)日本ではむしろシュエと呼ぶ現場が多い。}した後、材料がかぶる高さまで\protect\hyperlink{fonds-blanc-ordinaire}{白いフォン}を注ぐ。鍋に蓋をし、中温のオーヴンに入れて火を通す。

火が通ったら、野菜を取り出して水気をきり、料理での用途に合わせて成形する。その後すぐに使う場合は、煮汁の浮き脂を取り除いて\footnote{dégraisser
  (デグレセ)。}から煮詰め、野菜とともにソテー鍋で保温する。事前に仕込んでおく場合には、鍋から皿あるいは専用の陶製の器に広げる。煮汁は浮き脂を取り除かずにそのまま加える。バターを塗った紙で覆ってストックしておく。

\hypertarget{sauce-des-legumes-braises}{%
\subsection{野菜のブレゼのソース}\label{sauce-des-legumes-braises}}

\begin{frsecbenv}

Sauce des légumes braisés

\end{frsecbenv}

ブレゼの煮汁を煮詰め、浮き脂を丁寧に取り除いて使う。場合によっては\protect\hyperlink{glace-de-viande}{グラスドヴィアンド}、あるいは相応量の\protect\hyperlink{sauce-demi-glace}{ドゥミグラス}を加える。どちらの場合も、バターを加えてソースをまろやかに仕上げる。必要ならレモン果汁をほんの少量加える。

\hypertarget{liaison-des-legumes-vert-au-beurre}{%
\subsection{青物野菜をバターであえる}\label{liaison-des-legumes-vert-au-beurre}}

\begin{frsecbenv}

Liaison des légumes verts au beurre

\end{frsecbenv}

茹でた野菜はしっかりと水をきっておく。味付けをしてバターを加え、鍋をあおるようにしてバターが野菜全体にまわるようにする。バターを加えるのは\textbf{
火から外した状態}で行なうこと。そうすればバターの風味が失なわれずに仕上がる。

\hypertarget{liaison-des-legumes-a-la-creme}{%
\subsection{生クリームであえる}\label{liaison-des-legumes-a-la-creme}}

\begin{frsecbenv}

Liaison des légumes à la crème

\end{frsecbenv}

この方法で調理する場合は、野菜をやや固めの状態になるよう下茹でしておくこと。よく水気をきってから野菜を鍋に入れる。沸かした生クリームを野菜が上に顔を出す程度に加える。時々、ヘラでゆっくりかきまぜ\footnote{vanner
  (ヴァネ)。}ながら火入れを仕上げる。

クリームがほぼすっかり煮詰まったら、バターとレモン果汁少々を加える。必要なら、生クリームに\protect\hyperlink{sauce-creme}{ソース・クレーム}を少量加えてもいい。

\hypertarget{cremes-et-puree-de-legumes}{%
\subsection{野菜のクレームとピュレ}\label{cremes-et-puree-de-legumes}}

\begin{frsecbenv}

Crèmes et Purées de légumes

\end{frsecbenv}

乾燥豆とでんぷん質の野菜は裏漉ししてピュレにする。次にバター1片を加えて火にかけ、水気をとばす。牛乳か生クリームを加えて濃さを調節して仕上げる。

アリコヴェール、カリフラワー等のように水分の多い野菜をピュレにする場合は、濃度を出すため、その野菜との相性のいいでんぷん質の野菜のピュレを加えること。

野菜の\textbf{クレーム}にする場合は、でんぷん質の野菜のピュレではなく、しっかりした味で濃く仕上げた\protect\hyperlink{sauce-bechamel}{ベシャメルソース}を加える。

\end{main}
