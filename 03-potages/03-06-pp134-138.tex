\href{原稿下準備20180414五島、連載からコピー}{} \href{訳と注釈}{}
\href{未、原文対照チェック}{} \href{未、日本語表現校正}{}
\href{未、その他修正}{} \href{未、原稿最終校正}{}

\hypertarget{ux3068ux308dux307fux3092ux4ed8ux3051ux305fux30ddux30bfux30fcux30b8ux30e5}{%
\section{とろみを付けたポタージュ}\label{ux3068ux308dux307fux3092ux4ed8ux3051ux305fux30ddux30bfux30fcux30b8ux30e5}}

\frsec{Potages Liés}

\hypertarget{ux30ddux30bfux30fcux30b8ux30e5ux30d4ux30e5ux30ec}{%
\subsection{ポタージュ・ピュレ}\label{ux30ddux30bfux30fcux30b8ux30e5ux30d4ux30e5ux30ec}}

\frsecb{les Purées}

主素材とつなぎ:ポタージュ・ピュレの主素材として用いるのは次のとおり。
1種類または数種を組み合わせた野菜、鶏、ジビエ、甲殻類。

ほぼ全てのポタージュ・ピュレにはつなぎを加える。すなわち、

米\ldots{}\ldots{}鶏、甲殻類のポタージュ・ピュレおよび野菜のポタージュ・ピュレ
のいくつか。

じゃがいも\ldots{}\ldots{}香草や、かぼちゃのように水分の多い野菜のポタージュ・
ピュレ。

レンズ豆\ldots{}\ldots{}ジビエのポタージュ・ピュレ。

バターで揚げたクルトン\ldots{}\ldots{}クラシックなポタージュ・ピュレ。

昔の料理では、他にもつなぎに用いるものはあったが、とりわけクーリとビス
ク\footnote{「昔の料理」におけるビスクは甲殻類のポタージュ・ピュレのことで
  はなく、鳩などの煮込み料理のこと(本連載「ポタージュ(1)」2012年6月
  号p.115 参照)。}には、クルトンが主に用いられていた\footnote{中世〜18世紀には、とろみをつけるために、硬くなったパンを加えて
  弱火で煮込む(mitonnerミトネ)ことが一般的だった。}。とてもまろやかな仕上り
になるので、現代でもこの手法を用いる価値はある。

いんげん豆やレンズ豆、じゃがいものようなでんぷん質の素材のポタージュ・
ピュレにはつなぎを加える必要はない。主素材である野菜それ自体がつなぎと
なるからだ。

加える液体とつなぎの分量:ポタージュ・ピュレに加える液体は、主素材の種
類に応じて、白いコンソメ、ジビエのコンソメ、魚のコンソメを用いる。野菜
のポタージュ・ピュレでは牛乳を用いる場合もある。

加える液体\ldots{}\ldots{}ベースとなるピュレ1 Lに対して2 L。

つなぎ\ldots{}\ldots{}

\begin{enumerate}
\def\labelenumi{\arabic{enumi}.}
\item
  米\ldots{}\ldots{}野菜500 gあたり85〜120 g。鶏、ジビエ、甲殻類の身500
  gあ たり75〜100 g。
\item
  レンズ豆\ldots{}\ldots{}ジビエの肉500 gあたり190 g。
\item
  じゃがいも\ldots{}\ldots{}香草と野菜500 gあたり250 g。
\item
  バターで揚げたクルトン\ldots{}\ldots{}野菜または甲殻類の身500
  gあたり270 g。
\end{enumerate}

作業と仕上げ:野菜は次のいずれかの方法で処理する。(a)薄切りにした野菜
600〜700 gあたり80〜100 gのバターでエテュヴェする。(b)薄切りにした野菜を
湯通し\footnote{原文 blanchir
  (ブランシール)。下茹でする、湯がくこと。野菜類の
  ブランシールは塩を加えた湯で行なうが、素材の性質により2種に分けら
  れる。ひとつは大量の湯で素材に完全に火が通るまで茹でること。もうひ
  とつは刳味(アク)を除くための下茹で、湯通し(原書p.726)。ここでは後 者。}してからバターでエテュヴェする。どちらの方法を用いるかは、
本書では個々のルセットに記してある。

ジビエはサルミを調理する際と同様にロティール\footnote{鶏、猟鳥の胸肉の部分を豚背脂のシートで包んでセニャンにロティー
  ルする。なお、「サルミ」は古くは「猟鳥肉の煮込み」の意であった。
  『ル・ギード・キュリネール』では、ロティールした猟鳥の肉を切り分け
  て保温し、摺り潰したガラと端肉を煮込んで作ったソースと合わせる(本
  連載「雉のサルミ」2011年11月号pp.128-129 参照)。}してから、レンズ豆と
ともに煮る。火が通ったら骨を外す。肉とレンズ豆を摺り潰し、布漉しした後、
濃さを調節する。

鶏は白いコンソメでポシェする。つなぎに用いる米も一緒に煮る。火が通った
ら骨を外し、その後はジビエのピュレと同様にする。鶏およびジビエにちょう
ど火が通ったところで、浮き実にする分の胸肉は別にとっておくこと。

野菜のポタージュ・ピュレは、濃さを調節したらデプイエ、つまり微沸騰の状
態で25〜30分間かけて不純物を取り除く。

このデプイエの作業の際、時折、冷たいコンソメを若干量加えるとよい。ピュ
レの中に紛れている不純物が表面に浮かび上がって、取り除きやすくなる。

鶏、ジビエ、甲殻類のピュレは沸騰したら湯煎にかける。デプイエする必要は
ない。

どのポタージュ・ピュレも、仕上げにバターを加える直前に、目の細かいシノ
ワで漉すこと。

仕上げは提供直前に行なう。火から外し、ポタージュ1ℓあたり80〜100
gのバター を加える。

つなぎに白いんげん豆、じゃがいも、米などのような白いでんぷん質やクルト
ンを用いるポタージュは、さらにつなぎとして卵黄を補ってもよい。

バターを加えたら、再沸騰させてはいけないと肝に銘じること。沸騰するとバ
ターの風味が失なわれてしまう。ポタージュにおいて、バターの風味は明瞭で
フレッシュでなくてはいけない。

(略)

ピュレの展開\footnote{本連載「ポタージュ(1)」2012年6月号 p.115 参照。}:以下に記す方法で、ピュレの多くはポタージュ・ヴルテ、
ポタージュ・クレームにすることが出来る。ポタージュ・ピュレに用いるつな
ぎの代わりに、鶏または魚のヴルテや薄いソース・ベシャメルを主素材に加え
るのだ。

ただし、素材によっては、ポタージュ・ピュレ以外の仕立てに出来ないものも
ある。

\hypertarget{ux30ddux30bfux30fcux30b8ux30e5ux30f4ux30ebux30c6}{%
\subsection{ポタージュ・ヴルテ}\label{ux30ddux30bfux30fcux30b8ux30e5ux30f4ux30ebux30c6}}

\frsecb{les Veloutés}

ベースとなるヴルテ\footnote{基本ソースとしてのヴルテ(原書
  p.15)がベースとなる。}:

\begin{enumerate}
\def\labelenumi{\arabic{enumi}.}
\item
  野菜のポタージュ・ヴルテの場合は、やや薄い通常のヴルテ。
\item
  鶏や魚のポタージュ・ヴルテの場合は、それぞれ対応するヴルテ。
\end{enumerate}

ポタージュのベースにするヴルテは、主素材となる野菜、鶏、ジビエおよび魚
に応じて、通常の白いコンソメ、鶏のコンソメ、ジビエのコンソメ、魚のコン
ソメ1ℓあたり白いルゥ100 gを加えて作る。

材料比率:この方法で作るポタージュは全て、次の分量配分となる。

・ベースとなるヴルテはポタージュ全体の半量。

・ポタージュの性格を決めるピュレは全体の\textbf{1/4}。

・濃さを整えるのに加えるコンソメも\textbf{1/4}。ただし、つなぎとして加える
生クリームの分量もこれに含める。

例えば、仕上がり2ℓの「ポタージュ・ヴルテ王妃風\footnote{à la reine
  (ア・ラ・レーヌ)優美で繊細な料理に用いる表現。この名
  称の料理には鶏を素材としたものが多い。「ポタージュ・ピュレ王妃風」
  のルセットは原書p.146。}」の場合には分量は 次のようになる。

\begin{quote}
鶏のヴルテ\textbf{1}ℓ。鶏のピュレ\textbf{5dl}。仕上げに加える白いコンソメ
\textbf{3dl}。つなぎ\textbf{(}生クリーム\textbf{)2dl}。計\textbf{2}ℓ。
\end{quote}

作業:

\begin{enumerate}
\def\labelenumi{(\arabic{enumi})}
\item
  主素材が鶏や魚の場合は、予め骨を外してからベースとなるヴルテで素材
  を煮る。次に、肉を取り出して摺り潰し、肉を煮たヴルテでのばしてから布漉
  しする。このピュレにコンソメを加えて濃さを整える。
\item
  野菜の場合は、素材の性質に応じて、湯通ししたものをバターでエテュヴェ
  するか、生の野菜をバターでエテュヴェしてから、ベースとなるヴルテに加え
  る。野菜に火が通った後は上記と同様にする。
\item
  甲殻類の場合は、通常どおりミルポワを用いて火を通し、細かく摺り潰し
  てからベースとなるヴルテに加えて煮、布漉しする。
\end{enumerate}

つなぎと仕上げ:ポタージュ・ヴルテのつなぎには、仕上り1ℓあたり卵黄3ヶ
と生クリーム1dlを加える。

提供直前に、鍋を火から外して、1ℓあたりバター80〜100 gを加えて仕上げる。
(略)

\hypertarget{ux30ddux30bfux30fcux30b8ux30e5ux30afux30ecux30fcux30e0}{%
\subsection{ポタージュ・クレーム}\label{ux30ddux30bfux30fcux30b8ux30e5ux30afux30ecux30fcux30e0}}

\frsecb{les Crèmes}

ポタージュ・クレームの作り方はポタージュ・ヴルテと同じだが、以下の点が
違う。

\begin{enumerate}
\def\labelenumi{(\arabic{enumi})}
\item
  ヴルテではなく薄いソース・ベシャメルをベースとして用いる。牛乳1ℓあ
  たり白いルゥ100 gで作る。
\item
  多くの場合、仕上げに濃さを調節する際、コンソメではなく牛乳を加える。
\end{enumerate}

材料比率:ポタージュ・ヴルテと同様。つまり、ベシャメルはポタージュ全体
の半量、ポタージュの性格を決めるピュレが1/4、濃さを整えるための白いコ
ンソメまたは牛乳が1/4(仕上げに加える生クリームもこれに含める)。

作業:主素材が鶏、ジビエ、野菜、甲殻類いずれの場合も、作業はポタージュ・
ヴルテの項で示したのと同じ。(略)

仕上げ:提供直前に、ポタージュ1ℓあたり2dlの生クリームを加える。

原注:ポタージュ・ヴルテもポタージュ・クレームもデプイエは行なわない。
ポタージュの濃さを整えたら、沸騰寸前まで温め、湯煎にかけて保温しておく。
表面が乾かないようバターのかけら数片を載せる。ポタージュ・ヴルテは、供
する前に卵黄、生クリーム、バターを加えて仕上げる。ポタージュ・クレーム
の仕上げは供する前に生クリームだけを加える。
