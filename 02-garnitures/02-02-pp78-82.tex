\hypertarget{serie-des-appareiles-et-preparations-diverses-pour-garnitures-chaudes}{%
\section[温製ガルニチュール用アパレイユなど]{\texorpdfstring{温製ガルニチュール用アパレイユ\footnote{料理用語としての
  appareil アパレイユとは、具体的な何かを指す言葉
  ではなく、\textbf{ある料理を作る過程において用いられる、複数の材料を組み
  合わせたもの}、という一種の概念。現実には、キッシュのアパレイユ
  (生クリームと卵、塩漬け豚バラ肉など)、クレーム・ブリュレのアパレ
  イユ(卵黄、砂糖、生クリーム、牛乳)というように用いられるが、概ね、
  加熱して凝固する液体または半液状のもの、およびそれらを「つなぎ」と
  して固形物をあえたものを指す、と考えていい。アパレイユの概念として
  は、まったくの固形物である、3〜4 mmのさいの目に切った香味野菜(場
  合によってはハムも入る)である\protect\hyperlink{matignon}{マティニョン}も
  appareil à matignon
  と表現されることはフランスの料理書においては珍しくない
  し、本節の\protect\hyperlink{duxelles-seche}{デュクセル・セッシュ}もまたアパレイユ
  の一種に含められる。実際のところ、アパレイユという語はそれぞれの調
  理現場および料理人によって使い方がさまざまであり、概念としての理解
  も必ずしも共通しているとは限らない。本書では基本的に、上述のように
  加熱凝固する液体の場合と、半固形状あるいはクリーム状のものを指す場
  合がほとんど。この節のタイトルを直訳すると「温製ガルニチュール用の
  アパレイユおよびその他の仕込み」となるが、概念のレベルでいえば、こ
  の節に収められているレシピはほぼ全て「アパレイユ」と呼び得るものに
  他ならない。ただ、そう言い切ってしまうと現実問題として理解出来ない
  であろうことを想定したのか、やや曖昧な表現になっているのだと思われ
  る。また、既出の\protect\hyperlink{sauce-villeroy}{ソース・ヴィルロワ}などもまた、
  ソースというよりはむしろアパレイユと呼んでおかしくないものと言える。}など}{温製ガルニチュール用アパレイユなど}}\label{serie-des-appareiles-et-preparations-diverses-pour-garnitures-chaudes}}

\frsec{Série des Appareils et Préparations diverses pour Garnitures chaudes}

\index{garniture@garniture!appareils garnitures chaudes@appareils et préparations diverses pour garnitures chaudes}
\index{appareil@appareil!garnitures chaudes@--- et préparations diverses pour garnitures chaudes}
\index{かるにちゆーる@ガルニチュール!あはれいゆおんせい@温製ガルニチュールのためのアパレイユなど}
\index{あはれいゆ@アパレイユ!おんせいかるにちゆーる@温製ガルニチュールのための---など}
\begin{recette}
\hypertarget{appareils-a-cromesquis-et-a-croquettes}{%
\subsubsection{クロメスキとクロケットのアパレイユ}\label{appareils-a-cromesquis-et-a-croquettes}}

\frsub{Appareils à Cromesquis et à Croquettes}

\index{garniture@garniture!appareil@appareil!cromesquis croquettes@appareils à cromesquis et à croqeuttes}
\index{appareil@appareil!cromesquis croquettes@---s à cromesquis et à croquettes}
\index{cromesqui@cromesqui!appareil@appareils à --- et à croquettes}
\index{croquette@croquette!appareil@appareils à cromesquis et à ---}
\index{かるにちゆーる@ガルニチュール!あはれいゆ@アパレイユ!くろめすきとくろけつと@クロケットとクロメスキのアパレイユ}
\index{あはれいゆ@アパレイユ!くろめすきとくろけつと@クロケットとクロメスキの---}
\index{くろめすき@クロメスキ!あはれいゆ@---とクロケットのアパレイユ}
\index{くろけつと@クロケット!あはれいゆ@クロメスキと---のアパレイユ}

⇒ \protect\hyperlink{hors-d-oeuvres-chauds}{温製オードブル}の章を参照。

\hypertarget{appareils-a-pomme-dauphine-duchesse-marquise}{%
\subsubsection{じゃがいものドフィーヌ、デュシェス、マルキーズのアパレイユ}\label{appareils-a-pomme-dauphine-duchesse-marquise}}

\frsub{Appareils à pomme Dauphine, Duchesse et Marquise}

\index{garniture@garniture!appareil@appareil!pomme dauphine@appareils à pomme Dauphine, Duchesse et Marquise}
\index{appareil@appareil!pomme dauphine@---s à pomme Dauphine, Duchesse et Marquise}
\index{dauphin@dauphin(e)!appareil@appareils à pomme ---e}
\index{duc@duc / duchesse!appareil@appareils à pomme duchesse}
\index{marquis@marquis(s)!appareil@appareils à pomme ---e}
\index{かるにちゆーる@ガルニチュール!あはれいゆ@アパレイユ!しやかいものとふいーぬ@じゃがいものドフィーヌ、デュシェス、マルキーズのアパレイユ}
\index{あはれいゆ@アパレイユ!しやかいものとふいーぬと@じゃがいものドフィーヌ、デュシェス、マルキーズの---}
\index{とふいーぬ@ドフィーヌ!あはれいゆ@アパレイユ!しやかいものとふいーぬ@じゃがいもの---、デュシェス、マルキーズのアパレイユ}
\index{とゆしえす@デュシェス!あはれいゆ@アパレイユ!しやかいものてゆしえす@じゃがいものドフィーヌ、---、マルキーズのアパレイユ}
\index{まるきーす@マルキーズ!あはれいゆ@アパレイユ!しやかいものまるきーす@じゃがいものドフィーヌ、デュシェス、---のアパレイユ}

⇒ \protect\hyperlink{legumes}{野菜料理}の章、\protect\hyperlink{pommes-de-terre}{じゃがいも}の項を参照。

\hypertarget{appareils-maintenon}{%
\subsubsection{アパレイユ・マントノン}\label{appareils-maintenon}}

\frsub{Appareils Maintenon}

\index{garniture@garniture!appareil@appareil!maintenon@appareils Maintenon}
\index{appareil@appareil!maintenon@--- Maintenon}
\index{maintenon@Maintenon!appareil@appareils ---}
\index{かるにちゆーる@ガルニチュール!あはれいゆ@アパレイユ!まんとのん@アパレイユ・マントノン}
\index{あはれいゆ@アパレイユ!まんとのん@---・マントノン}
\index{まんとのん@マントノン!あはれいゆ@アパレイユ・---}

(\protect\hyperlink{cotelettes-maintenon}{羊のコトレット マントノン}用)

\protect\hyperlink{sauce-bechamel}{ベシャメルソース}4
dlと\protect\hyperlink{sauce-soubise}{スビーズ}1
dlを半量になるまで煮詰める。

卵黄3個を加えてとろみを付ける。あらかじめマッシュルーム100
gを薄切りにしてバターでごく弱火で鍋に蓋をして蒸し煮\footnote{étuver
  (エチュヴェ)。}したものを加える。
\end{recette}