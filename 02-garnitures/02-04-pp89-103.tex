\hypertarget{garnitures-recettes}{%
\subsection{ガルニチュールのレシピ}\label{garnitures-recettes}}

\frsecb{Garnitures}

\begin{center}
\medlarge(ここで示す分量はすべて仕上がり10人分)
\end{center}
\normalsize
\begin{recette}
\hypertarget{garniture-algerienne}{%
\subsubsection{アルジェリア風ガルニチュール}\label{garniture-algerienne}}

\frsub{Garniture à l'Algérienne}

\index{garniture@garniture!algerienne@--- à l'Algérienne}
\index{algerien@algérien(nne)!garuniture à l'---ne}
\index{かるにちゆーる@ガルニチュール!あるしえりあふう@アルジェリア風---}
\index{あるしえりあふう@アルジェリア風!かるにちゆーる@ガルニチュール---}

(牛、羊の塊肉用)

\begin{itemize}
\item
  ワインの栓の形にしたさつまいものクロケット10個。
\item
  小さなトマト10個は中をくり抜いて味付けをし、植物油少々で弱火で蒸し煮する。
\end{itemize}

\ul{ソース}\ldots{}\ldots{}薄く仕上げた\protect\hyperlink{sauce-tomate}{トマトソース}に、グリルして皮を剥き、細かい千切りにしたポワヴロン\footnote{いわゆる青果としてのパプリカ。}を加えたもの。

\hypertarget{garniture-alsacienne}{%
\subsubsection{アルザス風ガルニチュール}\label{garniture-alsacienne}}

\frsub{Garniture à l'Alsacienne}

\index{garniture@garniture!alsacienne@--- à l'Alsacienne}
\index{alsacien@alsacien(ne)!garuniture à l'---ne}
\index{かるにちゆーる@ガルニチュール!あるさすふう@アルザス風---}
\index{あるさすふう@アルザス風!かるにちゆーる@ガルニチュール---}

(牛、羊の塊肉、牛フィレ、トゥルヌド用)

\begin{itemize}
\tightlist
\item
  ブレゼ\footnote{\protect\hyperlink{chou-braise}{キャベツのブレゼ}を参考にすること。}したシュークルート\footnote{専用品種の生食出来ないくらい固くて大きなキャベツを千切りにして香
    辛料とともに塩蔵、醗酵さたもの。ドイツのザワークラウトが原型だが、
    フランスとドイツで領土の取り合いとなったアルザス地方で独自に発展し
    た。温めたシュークルートにソーセージなどの豚肉加工品を添えた
    choucoûte barnie(シュークルートガルニ)はアルザスの名物料理のひと
    つ。なおシュークルート用の品種はQuintal d'Alsace(カンタルダルザ
    ス)が最良とされている。また、日本でも北海道で栽培され、鰊の漬物な
    どに使われる札幌大球甘藍という品種はこの系統が先祖らしい。。}を詰めてハムの脂身のないところを円く切ってのせたタルトレット10個。
\end{itemize}

\ul{ソース}\ldots{}\ldots{}\protect\hyperlink{jus-de-veau-lie}{とろみを付けた仔牛のジュ}。

\hypertarget{garniture-americaine}{%
\subsubsection{ガルニチュール・アメリケーヌ}\label{garniture-americaine}}

\frsub{Garniture à l'Américaine}

\index{garniture@garniture!americaine@--- à l'Américaine}
\index{americain@américain(e)!garuniture à l'---e}
\index{かるにちゆーる@ガルニチュール!あめりけーぬ@---・アメリケーヌ}
\index{あめりかん@アメリカン/アメリケーヌ!かるにちゆーる@ガルニチュール・アメリケーヌ}

(魚料理用)

\begin{itemize}
\tightlist
\item
  このガルニチュールは必ず、\protect\hyperlink{homard-americaine}{オマール・アメリケー
  ヌ}の方法で調理した尾の身をやや斜めに1 cm程度の 薄切り\footnote{escalope
    (エスカロップ)肉などを筋線維と直角に、丸くスライスしたもの。}にして供する。
\end{itemize}

\ul{ソース}\ldots{}\ldots{}オマール・アメリケーヌのソース。
\end{recette}