\href{✓原稿下準備なし}{} \href{訳と注釈\%2020180420進行中}{}
\href{未、原文対照チェック}{} \href{未、日本語表現校正}{}
\href{未、注釈チェク}{} \href{未、原稿最終校正}{}

\begin{Main}

\hypertarget{garnitures-recettes}{%
\subsection{ガルニチュールのレシピ}\label{garnitures-recettes}}

\frsecb{Garnitures}

\begin{center}
\medlarge(ここで示す分量はすべて仕上がり10人分)
\end{center}
\normalsize

\end{Main}

\begin{recette}

\hypertarget{garniture-algerienne}{%
\subsubsection{ガルニチュール・アルジェリア風}\label{garniture-algerienne}}

\frsub{Garniture à l'Algérienne}

\index{garniture@garniture!algerienne@--- à l'Algérienne}
\index{algerien@algérien(nne)!garniture à l'---ne}
\index{かるにちゆーる@ガルニチュール!あるしえりあふう@---・アルジェリア風}
\index{あるしえりあふう@アルジェリア風!かるにちゆーる@ガルニチュール・---}

(牛、羊の塊肉\footnote{原文 Pour les pièce de boucherie
  より正確に訳すなら、「\ul{肉屋
  (boucherie)が伝統的に扱かってきた、白身肉を除く畜産精肉}、具体的には牛、羊(馬も含まれる)の塊肉」であり、牛の場合は基本的にランプ、イチボに相当する部位、羊の場合は鞍下肉から腿上部にかけての部位を塊のまま調理したものを意味することがほとんど。}の料理に添える)

\begin{itemize}
\item
  ワインの栓の形にしたさつまいもの\protect\hyperlink{croquettes}{クロケット}
  10個。
\item
  小さなトマト10個は中をくり抜いて味付けをし、植物油少々で弱火で蒸し煮する。
\item
  【別添】薄く仕上げた\protect\hyperlink{sauce-tomate}{トマトソース}に、グリルして皮を剥き、細かい千切りにしたポワヴロン\footnote{いわゆる青果としてのパプリカ。}を加える。
\end{itemize}

\atoaki{}

\hypertarget{garniture-alsacienne}{%
\subsubsection{ガルニチュール・アルザス風}\label{garniture-alsacienne}}

\frsub{Garniture à l'Alsacienne}

\index{garniture@garniture!alsacienne@--- à l'Alsacienne}
\index{alsacien@alsacien(ne)!garniture à l'---ne}
\index{かるにちゆーる@ガルニチュール!あるさすふう@---・アルザス風}
\index{あるさすふう@アルザス風!かるにちゆーる@ガルニチュール・---}

(牛、羊の塊肉、牛フィレ、トゥルヌドに添える)

\begin{itemize}
\item
  ブレゼ\footnote{\protect\hyperlink{chou-braise}{キャベツのブレゼ}を参考にすること。}したシュークルート\footnote{生食出来ないくらい固くて大きな専用品種であるキャベツを千切りにして香辛料などとともに塩蔵、醗酵さたもの。ドイツのザワークラウトが原型だが、歴史的にフランスとドイツで領土の取り合いとなったアルザス地方で独自に発展した。温めたシュークルートにソーセージなどの豚肉加工
    choucroutes
    garnies(シュークルートガルニ)はアルザスの名物料理のひとつ。}を詰めてハムの脂身のないところを円く切ってのせたタルトレット10個。
\item
  【別添】\protect\hyperlink{jus-de-veau-lie}{とろみを付けた仔牛のジュ}。
\end{itemize}

\atoaki{}

\hypertarget{garniture-americaine}{%
\subsubsection[ガルニチュール・アメリケーヌ]{\texorpdfstring{ガルニチュール・アメリケーヌ\footnote{\protect\hyperlink{sauce-americaine}{ソース・アメリケーヌ}も参照されたい。}}{ガルニチュール・アメリケーヌ}}\label{garniture-americaine}}

\frsub{Garniture à l'Américaine}

\index{garniture@garniture!americaine@--- à l'Américaine}
\index{americain@américain(e)!garniture à l'---e}
\index{かるにちゆーる@ガルニチュール!あめりけーぬ@---・アメリケーヌ}
\index{あめりかん@アメリカン/アメリケーヌ!かるにちゆーる@ガルニチュール・アメリケーヌ}

(魚料理に添える)

\begin{itemize}
\item
  このガルニチュールは必ず、\protect\hyperlink{homard-americaine}{オマール・アメリケーヌ}の方法で調理した尾の身をやや斜めに1
  cm程度の薄切り\footnote{escalope
    (エスカロップ)肉などを筋線維と直角に、丸くスライスしたもの。}にして供する。
\item
  【別添】オマール・アメリケーヌのソース。
\end{itemize}

\atoaki{}

\hypertarget{garniture-andalouse}{%
\subsubsection[ガルニチュール・アンダルシア風]{\texorpdfstring{ガルニチュール・アンダルシア風\footnote{アンダルシア風、つまりスペイン風といいながら、ギリシャ風ライスを使うという点からも、料理名に付けられた地名がしばしば不確かで大雑把な理由さえないことが多いことが理解されよう。}}{ガルニチュール・アンダルシア風}}\label{garniture-andalouse}}

\frsub{Garniture à l'Andalouse}

\index{garniture@garniture!andalouse@--- à l'Andalouse}
\index{andalou@andalou(se)!garniture à l'---se}
\index{かるにちゆーる@ガルニチュール!あんたるしあふう@---・アンダルシア風}
\index{あんたるしあふう@アンダルシア風!かるにちゆーる@ガルニチュール・---}

(牛、羊の塊肉料理や鶏料理に添える)

\begin{itemize}
\item
  中位の大きさのポワヴロン10個をグリル焼きして中をくり抜き、\protect\hyperlink{riz-grecque}{ギリシャ風ライス}を詰める。
\item
  なす\footnote{フランスで伝統的なタイプのなすはヘタが緑色で、風味や調理特性はいわゆる米なすに近いが、形状は比較的細長い。直径4〜6
    cm、長さ25 cmくらいのものが多い。}を4
  cmの厚さの輪切りにして面取りをし、中に窪みをつくって油で揚げ、提供直前に油で炒めたトマトをのせる。
\item
  【別添】\protect\hyperlink{jus-de-veau-lie}{とろみを付けたジュ}。
\end{itemize}

\atoaki{}

\hypertarget{garniture-arlesienne}{%
\subsubsection[ガルニチュール・アルル風]{\texorpdfstring{ガルニチュール・アルル風\footnote{南フランスの都市
  Arles
  (アルル)の形容詞および名詞形。名詞の場合は「アルルの人」の意味になる。アルルはオランダ出身の印象派〜ポスト印象派の画家フィンセント・ファン・ゴッホ
  Vincent van Gogh
  (フランス語では昔からヴァンソンヴァンゴーグと呼ぶ習慣が付いてしまっており、現代フランス語の原語発音尊重の風潮にもかかわらず、そのように発音されることは多いようだ)が1888年から1889年までアトリエを構え、「ひまわり」など多くの傑作を描いた。有名な、自分の耳を切り落すという「事件」を起こしたのもアルルでのことだ。この時期の作品のひとつに、「アルルの女(ジヌー夫人)」と呼ばれる一連のものがある。モデルはアルルのカフェの経営者だといわれている。もっとも、フランスにおいて画家としてのゴッホおよび彼の作品は生前はほとんど評価されることがなく、生前に売れた絵は1枚だけだったとさえいわれている。このレシピは初版つまり1903年から収められているため、ゴッホの絵との関連はほぼないと
  ~
  考えていいだろう。むしろ、小説家アルフォンス・ドーデ原作を戯曲化してジョルジュ・ビゼーが劇音楽を付けた『アルルの女』(1872年初演、
  1878年再演)との関連があると見るのがいいだろう。この作品は初演時点ではあまり好評ではなかったが、再演で大ヒットとなった。\protect\hyperlink{sauce-bohemienne}{ソース・ボヘミアの娘}のように、人気のある劇やオペラのタイトルを料理名につけて、その人気にあやかろうという風潮が19世紀後半には比較的多かった。そのため、トマトとなすという南フランスを思わせる食材を使ってはいてもアルルという土地に何の関係もないと思われる、内容的にも凡庸なこのガルニチュールに、当時の人気作品の名をつけて、いかにも流行のものであるかのように供したのが定着した、と考えることも可能だろう。その場合は「\ul{ガルニチュール・アルルの女}」と訳すべきかも知れない。なお、ビゼーが最初に作曲したのは27曲からなる舞台音楽であって、独立した音楽作品でもなければ、オペラでもなかったが、そのなかから数曲を選んで編曲し(あるいは作曲しなおし)、『アルルの女 組曲』としてこんにち広く知られている。第1組曲と第2組曲があり、前者はビゼー自身によるオーケストラ用編曲。後者はビゼーの死後1879年に友人エルネスト・ギローが完成させた。第1組曲の「メヌエット」や第2
  組曲の「ファランドール」など、曲名は知らずとも、メロディーを聴いたことのある読者も少なくないと思われる。}}{ガルニチュール・アルル風}}\label{garniture-arlesienne}}

\frsub{Garniture à l'Arlésienne}

\index{garniture@garniture!arlesienne@--- à l'Arlésienne}
\index{arlesien@arlésien(ne)!garniture à l'---ne}
\index{かるにちゆーる@ガルニチュール!あるるふう@---・アルル風}
\index{あるるふう@アルル風!かるにちゆーる@ガルニチュール・---}

(トゥルヌドやノワゼットの料理に添える)

\begin{itemize}
\item
  なす\footnote{なす、トマト、玉ねぎの分量は記されていないので適宜判断すること。}は1
  cm程の厚さにスライスして塩こしょうをし、小麦粉をまぶして油で揚げる。
\item
  トマトは皮を剥いてスライスし、バターでソテーする。
\item
  玉ねぎは輪切りにして指輪のようにばらばらにし、小麦粉をまぶして油で揚げ、こんもりと盛る。
\item
  【別添】トマト風味の\protect\hyperlink{sauce-demi-glace}{ソース・ドゥミグラス}。
\end{itemize}

\atoaki{}

\hypertarget{garniture-banquiere}{%
\subsubsection[ガルニチュール・銀行家夫人風]{\texorpdfstring{ガルニチュール・銀行家夫人風\footnote{原文の
  à la Banquière をここでは文字通り訳した。料理名において {[}à la +
  形容詞の女性形{]}は通常、à la manière/façon
  〜のmanièreあるいはfaçonが省略されたものと考える。これらmanière,
  façonいずれも女性名詞であるために、この後に付ける形容詞も女性形となる。ところが「〜風」」「〜を記念して/〜を称揚して」の意味で{[}à
  la + (固有)名詞 {]}という用法もある。これは à la manière de +
  名詞、のmanière
  deが省略されたものと考える。Banquier(ボンキエ)は「銀行家」を意味する名詞であり、対応する女性名詞がbanquièreとなり、女性銀行家あるいは銀行家夫人の意となる。そのため、従来は「銀行家風」と訳されていたが、あえて文法の原則に忠実に「銀行家夫人風」を訳した。さて、この料理名だが、日仏料理協会編『フランス 食の事典』(白水社、2000年)には「産業革命に伴う産業の隆盛を支えた銀行は、現代にいたるまで資本主義社会の根幹をなすもので、その経営者は19世紀において金持ちの代名詞ともなった。当時、「銀行家風」は王風、王妃風にかわる新しい表現だった
  (pp.162-163)」と説明されている。ところが、料理書においてこのà la
  Banquièreという表現は1856年のデュボワ、ベルナール共著『古典料理』以前には見つからない。しかも、「冷製料理用ガルニチュール・銀行家夫人風」Garniture
  à la banquière, pour froid (t.1,
  p.259)および「若鶏のガランティーヌ・銀行家夫人風」Galantine de poulet
  à la banquière (t.2,
  p.40)の2つでのみ料理名に使われているのみ。ガルニチュールの概要は、オマール2尾の身をやや斜めの円形(エスカロップ)にスライスする。これをひとつずつ別々の陶製の器に入れ、小さなアーティチョークの基底部を茹でたもの、大きな黒または白トリュフのスライス、マッシュルームのスライス、コルニションのスライスを盛り込み、塩、こしょう、植物油、パセリとエストラゴンのみじん切りで味付けし、銘々に供する、というもの。本書のガルニチュールと温製、冷製の違いはあっても、同じ名称とは思い難いくらい異なった内容。その前後および以前については、毎年のように版を重ねながら増補されたために料理の流行、変遷を見るのに非常に便利なヴィアールにもオドにも収録されておらず、グフェ『料理の本』(1867年)にも見あたらない。本書よりやや時代が下って、
  1838年の『ラルース・ガストロノミック』初版の「ガルニチュール・銀行家夫人風」は「鶏、仔牛胸腺肉(リドヴォー)の料理、ヴォロヴァン用。クネル、マッシュルーム、トリュフのスライス、ソース・バンキエール
  (p.136)」と定義されている。ソース・バンキエールsauce
  banquièreについては「卵料理、鶏料理、牛や羊の副生物(リドヴォーなど)、ヴォロヴァン用。ソース・シュプレーム2
  dLにマデイラ酒 \(\frac{1}{2}\)
  dLを加え、布で漉す。トリュフのみじん切り大さじ2杯を加えて仕上げる(p.959)」とある。2007年版の『ラルース・ガストロノミック』でもほぼ同様の内容だが、ソース・バンキエールのレシピはこの版では欠落している。また、20
  世紀についても、1950年に刊行されたレシピ集『フランス料理技法』
  (Flammarion)にソース・バンキエールのレシピは見られるが(p.147)、これはモンタニェの『料理大全』(1929年)からの引用であり、ガルニチュール・バンキエールについては何も出ていない。1952年のペラプラ『近代料理技術』にも、1953年のキュルノンスキー編『フランスの料理とワイン』にもこれらへの言及はない。}}{ガルニチュール・銀行家夫人風}}\label{garniture-banquiere}}

\frsub{Garniture à la Banquière}

\index{garniture@garniture!banauiere@--- à la Banquière}
\index{banquier@banquier(ère)!garniture à la Banquière}
\index{かるにちゆーる@ガルニチュール!きんこうかふしんふう@---・銀行家夫人風}
\index{きんこうかふしんふう@銀行家夫人風!かるにちゆーる@ガルニチュール・---}

(肥鶏の料理に添える)

\begin{itemize}
\item
  ひばり\footnote{mauviette
    (モヴィエット)、ひばりの食材としての名称。生物としてはalouette(アルエット)と呼ぶ。なお、オルレアネ地方の郷土料理に、
    pithiviers de mauviettes
    という、脳と鶏のファルスを詰めたひばりを折込みパイ生地で包んで焼いた料理があるが、pithiviers(ピティヴィエ)とだけ言う場合は、バターと砂糖、アーモンドパウダーなどを折込みパイ生地で包んで上部を渦巻模様に装飾したオルレアネ地方発祥の菓子を指すので注意。}10羽を背側から開いて骨をすべて取り除き\footnote{désosser
    (デゾセ)。日本の調理現場でも比較的よく使われる用語。この語に含まれるosは「骨」のこと、déは「反対、除去」などを意味する接頭辞、erは動詞であることを示す語尾。したがって、文字どおり「骨を取り除く」の意になる。}、\protect\hyperlink{farce-gratin-c}{ファルス・グラタン}を詰めて、表面を色よく焼き、カスロールで火を通す\footnote{en
    casserole
    (オンカスロール)カスロール仕立てと解釈も可能。\protect\hyperlink{sauce-smitane}{ソース・スミターヌ}訳注参照。}。
\item
  \protect\hyperlink{farce-b}{鶏のファルス}で小さなクネル10個。
\item
  トリュフのスライス10枚。
\item
  【別添】トリュフエッセンス入り\protect\hyperlink{sauce-demi-glace}{ソース・ドゥミグラス}。
\end{itemize}

\atoaki{}

\hypertarget{garniture-berrichonne}{%
\subsubsection[ガルニチュール・ベリー風]{\texorpdfstring{ガルニチュール・ベリー風\footnote{berrichon(ne)(ベリション/ベリショーヌ)
  はフランス中央部にある地方名 Berry の形容詞。ここでは女性形
  berrichonneとなる。山羊乳のチーズで有名。なおフランス史関連の書物ににおいてよく見かける、ベリー公
  duc de Berry
  (デュックドベリー)という公爵位はフランスの王族(つまりその時の王の近縁者)に与えられた爵位で、その後フランス王となった者も多い。このため、いわゆる「世襲」はされてこなかった。また、中世フランスでもっとも豪華で美しい写本とされる数部の『\href{http://gallica.bnf.fr/ark:/12148/btv1b520004510}{ベリー公のいとも豪華なる時祷書}』(14世紀)は当時のベリー公ジャン1世が作成させたもの。}}{ガルニチュール・ベリー風}}\label{garniture-berrichonne}}

\frsub{Garniture à la Berrreichonne}

\index{garniture@garniture!berrichonne@--- à la Berrichonne}
\index{berrichon@berrichon(ne)!garniture à la Berrichonne}
\index{かるにちゆーる@ガルニチュール!へりーふう@---・ベリー風}
\index{へりーふう@ベリー風!かるにちゆーる@ガルニチュール・---}

(牛、羊肉の大がかりな料理\footnote{ルルヴェ relevé
  のこと。\protect\hyperlink{releve}{第二版序文訳注}参照。}に添える)

\begin{itemize}
\item
  卵の大きさにした\protect\hyperlink{chou-braise}{サヴォイキャベツのブレゼ}
  20個。
\item
  キャベツとともに火を通した塩漬け豚バラ肉の小さなスライス10枚。
\item
  小玉ねぎ20個と大粒のマロン20個はこのガルニチュールを添える肉の煮汁で火を通す。
\item
  【別添】アロールート\footnote{Allow-root
    南米産クズウコンを原料とした良質のでんぷん。現代の日本ではコーンスターチで代用することがほとんど。}でとろみを付けた、ブレゼの煮汁。
\end{itemize}

\atoaki{}

\hypertarget{garniture-berny}{%
\subsubsection[ガルニチュール・ベルニ]{\texorpdfstring{ガルニチュール・ベルニ\footnote{ピエール・ド・ベルニPierre
  de Bernis
  (1715〜1794)のこと。なぜか料理名としてはBernyの綴りが一般的だが、個人名なのでもちろん誤り。
  29才でアカデミーフランセーズに入った俊才。ポンパドゥール夫人の庇護のもとルイ15世からも重用された。駐ヴェネツィア大使として食卓外交を展開したが、フランス革命後、ローマで客死した。}}{ガルニチュール・ベルニ}}\label{garniture-berny}}

\frsub{Garniture à la Berny}

\index{garniture@garniture!berny@--- à la Berny}
\index{Berny@Berny (Bernis)!garniture à la Berny}
\index{かるにちゆーる@ガルニチュール!へるに@---・ベルニ}
\index{へるに@ベルニ!かるにちゆーる@ガルニチュール・---}

(ジビエおよびマリネした牛、羊肉料理\footnote{シュヴルイユ仕立てのこと。\protect\hyperlink{sauce-poivrade}{ソース・ポワヴラード}および\protect\hyperlink{marinade-crue-pour-viandes-de-boucherie-ou-venaison}{マリナード}参照。}用)

\begin{itemize}
\item
  小さな俵形にしたじゃがいものクロケット・ベルニ\footnote{本書の温製オードブルの節に「クロケット・ベルニ」は掲載されていない。野菜料理の章にある\protect\hyperlink{pommes-de-terre-berny}{じゃがいも・ベルニ}をアパレイユとしてクロケットを作ることになる。}10個
\item
  空焼きしたタルトレット10個にバターを加えたマロンのピュレをドーム状に詰め、バターで軽くソテーして艶を出させたトリュフのスライスをタルトレットに1枚ずつのせる
\item
  【別添】軽く仕上げた\protect\hyperlink{sauce-poivrade}{ソース・ポワヴラード}。
\end{itemize}

\atoaki{}

\hypertarget{garniture-bizontine}{%
\subsubsection[ガルニチュール・ブザンソン風]{\texorpdfstring{ガルニチュール・ブザンソン風\footnote{Besonçon
  (ブゾンソン)フランス東部、ブルゴーニュ=フランシュ=コンテ圏の都市。形容詞は通常bisontin(e)(ビゾンタン/ビゾンティーヌ)だが、本書のようにbizontin(e)と綴ることもある。なお、1980年代に画期的といわれたフランス語教材\emph{C'est
  le
  printemps}の第1課においてはじめて出てくる地名がブザンソンだった。この教材は会話例のリアリティや題材としてdocuments
  authentiques(ドキュモンオトンティック=現実にあるドキュメントすなわち言語を用いたさまざまな書類、看板、広告など)を積極的に採用したこととともに、アプレ68(フランスの学生運動および現代思想における転換期のひとつとなった1968年の「五月革命」以後に多方面において展開された時代特有の雰囲気)が強く表われているのが特徴だった。同時期のフランス語教材の傑作とされる(やや保守的な傾向の)通称「カペル」\emph{Le
  français en direct}
  と並び、フランス語教育・教授法において現在のEUおよびフランスで定められ運用されている「外国語としての言語コミュニケーション能力」の概念形成の先駆けとなった。アプレ68的なものは食文化、料理の世界においても、ゴ\&ミヨの批評と店の格付けにおける、既存のミシュランのガイドブックのオルタナティヴとしての方向性、ヌーヴェルキュイジーヌ宣言などによく表われている。}}{ガルニチュール・ブザンソン風}}\label{garniture-bizontine}}

\frsub{Garniture à la Bizontine}

\index{garniture@garniture!bizontinne@--- à la Bizontine}
\index{bizontin@bizontin(e) ⇒ bisontin(e)!garniture à la ---e}
\index{かるにちゆーる@ガルニチュール!ふさんそんふう@---・ブザンソン風}
\index{ふさんそん@ブザンソン!かるにちゆーる@ガルニチュール・---風}

(牛、羊の塊肉料理およびトゥルヌドに添える)

\begin{itemize}
\item
  \protect\hyperlink{pommes-de-terre-duchesse}{じゃがいも・デュシェス}を成形して卵液を塗りオーブンで焼いた小さなクルスタード10個に、生クリーム入りのカリフラワーのピュレを絞り袋を使って詰める。
\item
  半割りにして\protect\hyperlink{laitues-farcies-pour-garniture}{詰め物をしたレチュのブレゼ}
  10個。
\item
  バターを加えて仕上げた\protect\hyperlink{jus-de-veau-lie}{とろみを付けたジュ}。
\end{itemize}

\atoaki{}

\hypertarget{garniture-boulangere}{%
\subsubsection[ガルニチュール・ブランジェール]{\texorpdfstring{ガルニチュール・ブランジェール\footnote{boulanger/boulangère
  は「パン屋、パン職人」の意。}}{ガルニチュール・ブランジェール}}\label{garniture-boulangere}}

\frsub{Garniture à la Boulangère}

\index{garniture@garniture!boulangere@--- à la Boulangère}
\index{boulanger@boulanger/boulangère!garniture à la ---ère}
\index{かるにちゆーる@ガルニチュール!ふらんしえーる@---・ブランジェール}
\index{ふらんしえ@ブランジェ/ブランジェール ⇒ パン屋!かるにちゆーる@ガルニチュール・ブランジェール}
\index{はんや@パン屋 ⇒ ブランジェ/ブランジェール!かるにちゆーる@ガルニチュール・ブランジェール}

(羊、乳呑み仔羊、鶏料理に添える)

\begin{enumerate}
\def\labelenumi{\arabic{enumi}.}
\item
  玉ねぎ250 gは薄切りにし\footnote{émincer (エマンセ)。}て、バターで色よく炒める。
\item
  じゃがいも750 gは櫛切りか薄切りにする。
\item
  塩15 gとこしょう5 g。
\end{enumerate}

\begin{itemize}
\item
  1〜3を混ぜ合わせて、このガルニチュールを添える肉を油を熱したフライパンで表面を焼き固め\footnote{rissoler
    (リソレ)。}とともにオーヴンに入れて、一緒に火を通す。
\item
  鶏の場合は、じゃがいもはオリーブ形に成形し\footnote{tourner
    (トゥルネ)}、小玉ねぎをあらかじめバターでこんがり焼き色を付けておく。
\item
  【別添】美味しい肉汁(ジュ)少々。
\end{itemize}

\atoaki{}

\hypertarget{garniture-bouquetiere}{%
\subsubsection[ガルニチュール・ブクティエール]{\texorpdfstring{ガルニチュール・ブクティエール\footnote{花売り娘、の意。}}{ガルニチュール・ブクティエール}}\label{garniture-bouquetiere}}

\frsub{Garniture à la Bouquetière}

\index{garniture@garniture!bouquetiere@--- à la Bouquetière}
\index{bouquetiere@bouquetière!garniture à la ---}
\index{かるにちゆーる@ガルニチュール!ふくていえーる@---・ブクティエール}
\index{ふくていえーる@ブクティエール ⇒ 花売り娘!かるにちゆーる@ガルニチュール・ブクティエール}
\index{はなうりむすめ@花売り娘 ⇒ ブクティエール!かるにちゆーる@ガルニチュール・ブクティエール}

(牛、羊の大掛かりな仕立ての料理\footnote{ルルヴェ relevé
  のこと。\protect\hyperlink{releve}{第二版序文訳注}参照。}に添える)

\begin{itemize}
\item
  にんじん250 gと蕪250
  gはスプーンで中をくり抜いて下茹でし、バターで色艶よく炒める\footnote{glacer
    (グラセ)。}。
\item
  小さなじゃがいも250 gはシャトー\footnote{長さ6
    cm程度の細長い樽の形状にすること。両端は切り落すので、ラグビーボール形ではない。}に成形する\footnote{いずれも適切に加熱調理するが、この節では細かく説明されていないので、対応する野菜のページを参照すること。}。
\item
  プチポワ\footnote{petits pois
    (プティポワ)いわゆるグリンピースのことだが、日本でよく知られているものよりも若どりで小さく、風味も軽やかで甘みがある。}250
  gと、さいの目に切ったアリコヴェール\footnote{haricots verts
    さやいんげんのことだが、これも日本のものより若どりに適した品種が好まれる。}250
  g。
\item
  カリフラワー250 gは花束の形状にバラしておく。
\end{itemize}

以上の材料をそれぞれ加熱調理した後に、塊肉の周囲に、ブーケ状に、それぞれを離してニュアンスが明確になるように盛り付ける。カリフラワーのブーケには\protect\hyperlink{sauce-hollandaise}{オランデーズソース}を薄く塗ること。

\begin{itemize}
\tightlist
\item
  【別添】塊肉を調理した際の肉汁の浮き脂を取り除き\footnote{dégraisser
    (デグレセ)。}、澄ませたもの。
\end{itemize}

\atoaki{}

\hypertarget{garniture-bourgeoise}{%
\subsubsection[ガルニチュール・ブルジョワーズ]{\texorpdfstring{ガルニチュール・ブルジョワーズ\footnote{bourgeois(e)
  (ブルジョワ/ブルジョワーズ)。ブルジョワ風の意。中世においては都市に住む貴族ではないある種の特権階級を意味したが、
  19世紀以降は、肉体労働をせずに快適できわめて豊かな生活をおくれる社会階層、の意に変化した。社会が物質的に、経済的に豊かになるにともない
  petit bourgeois
  (プティブルジョワ)なる階層も出現したが、ブルジョワの本義はあくまでも「大金持ち」であり、現代日本語でいうところの「セレブ」に相当すると思っていい。}}{ガルニチュール・ブルジョワーズ}}\label{garniture-bourgeoise}}

\frsub{Garniture à la Bourgeoise}

\index{garniture@garniture!bourgeoise@--- à la Bourgeoise}
\index{bourgeois@bourgeois(e)!garniture à la ---}
\index{かるにちゆーる@ガルニチュール!ふるしよわーす@---・ブルジョワーズ}
\index{ふるしよわーす@ブルジョワーズ!かるにちゆーる@ガルニチュール・ブルジョワーズ}
\index{ふるしよわふう@フルジョワ風 ⇒ ブルジョワーズ!かるにちゆーる@ガルニチュール・ブルジョワーズ}

(牛、羊の塊肉料理に添える)

\begin{itemize}
\item
  にんじん500 gは、にんにくのような形に成形して\footnote{tourner
    (トゥルネ)。}下茹でし、バターで色艶よく炒める\footnote{glacer
    (グラセ)。もともとは「鏡のようにする」ところから「艶を出す」の意となり、野菜の場合はもっぱら下茹でした後にバターで軽く炒めて艶を出すことをいうが、場合によっては茹でる段階で砂糖を煮含めたりもする。}。
\item
  小玉ねぎ\footnote{日本のいわゆる「ペコロス」は黄色系品種が多いが、フランスの小さな玉ねぎはもっぱら白系品種であり、甘さや風味がまったく異なるので注意。}500
  gは下茹でした後にバターで色艶よく炒める。
\item
  塩漬け豚バラ肉\footnote{原文 lard de poitrine
    (ラールドポワトリーヌ)は豚バラ肉のことだが、通常は塩蔵、熟成させたもの、およびそれを冷燻にかけたものを指す。しばしば「ベーコン」と誤訳されているが、日本語のいわゆるベーコンとは違うので注意。}125
  gはさいの目に切ってバターでこんがり炒める。
\item
  このガルニチュールは、塊肉にほぼ火が通った段階で、鍋の中の肉の周囲に入れてやり、ブレゼの煮汁で火入れを完全にすること。
\end{itemize}

\atoaki{}

\hypertarget{garniture-a-la-bourguignonne}{%
\subsubsection[ガルニチュール・ブルゴーニュ風]{\texorpdfstring{ガルニチュール・ブルゴーニュ風\footnote{いわゆるboeuf
  bourguignonne
  (ブフブルギニョンヌ)とほぼ同じと考えていいが、現代のそれが肉を比較的細かく切って調理するのに対して、ここでは19世紀的な大皿料理としてのブレゼを前提にしていることに注意。}}{ガルニチュール・ブルゴーニュ風}}\label{garniture-a-la-bourguignonne}}

\frsub{Garniture à la Bourgeoise}

\index{garniture@garniture!bourguignonne@--- à la Bourguignonne}
\index{bourguignon@bourguignon(ne)!garniture à la ---ne}
\index{かるにちゆーる@ガルニチュール!ふるこーにゆふう@---・ブルゴーニュ風}
\index{ふるこーにゆふう@ブルゴーニュ風!かるにちゆーる@ガルニチュール・---}

(牛の塊肉\footnote{原文 pièce de boeuf
  そのまま訳せば牛の塊肉となるが、いわゆるランプ、イチボの部分から2〜3
  kgの塊を切り出してブレゼにすることが多い。。}料理に添える)

\begin{itemize}
\item
  小玉ねぎ500 gはバターで色艶よく炒める。
\item
  マッシュルーム250 gは四つ割りにしてバターで炒める。
\item
  塩漬豚ばら肉125 gはさいの目に刻み、強火でこんがり炒める。
\end{itemize}

これらを適当なタイミングで、鍋で加熱中の肉の周囲に投入する。

肉を煮るのには、必ず上等の赤ワインを用いること。それがブルゴーニュ風の仕立てを特徴づける絶対条件。

\begin{itemize}
\tightlist
\item
  【別添】ブレゼの煮汁。
\end{itemize}

\atoaki{}

\hypertarget{garniture-brabanconne}{%
\subsubsection[ガルニチュール・ブラバント風]{\texorpdfstring{ガルニチュール・ブラバント風\footnote{現在はベルギー中部の州ブラバントBrabantの、の意。なお、この名称のガルニチュールは『ラルース・ガストロノミック』初版にも掲載されているが、内容がまったく異なる。アンディーヴとじゃがいものピュレ、ホップの若芽を茹でてバターか生クリームであえたもので構成するという
  (p.239)。なおブラバントは中世においてブラバント公国として独立した国家であった。ベルギー王国成立後は、儀礼称号としてベルギー王家の法定推定相続人にブラバント公の称号が授けられるようになった。なお、エスコフィエによる\protect\hyperlink{peches-melba}{ピーチメルバ}創案のきっかけとなったといわれるワーグナーの楽劇『ローエングリン』においてネリー・メルバNellie
  Melba(1861〜1931)が演じていたエルザ・フォン・ブラバントはブラバント公国の公女という設定。}}{ガルニチュール・ブラバント風}}\label{garniture-brabanconne}}

\frsub{Garniture à la Brabançonne}

\index{garniture@garniture!brabanconne@--- à la Brabançonne}
\index{brabanconne@brabançon(ne)!garniture à la ---ne}
\index{かるにちゆーる@ガルニチュール!ふらはんとふう@---・ブラバント風}
\index{ふらはんとふう@ブラバント風!かるにちゆーる@ガルニチュール・---}

(牛、羊の塊肉の料理に添える)

\begin{itemize}
\item
  空焼きしたタルトレット10個に、下茹でしてバターで蒸し煮した\footnote{étuver
    (エチュヴェ)。}芽キャベツ\footnote{芽キャベツはchoux de Bruxelles
    (シュドブリュクセル、ブリュッセルのキャベツの意)と呼ぶ。}をピュレにして詰め、\protect\hyperlink{sauce-mornay}{ソース・モルネー}を塗る。
\item
  \protect\hyperlink{pommes-de-terre-duchesse}{じゃがいも・デュシェス}で作った小さな円盤形のクロケット10個。
\item
  【別添】\protect\hyperlink{jus-de-veau-lie}{とろみを付けたジュ}。
\end{itemize}

\atoaki{}

\hypertarget{garniture-brehan}{%
\subsubsection[ガルニチュール・ブレオン]{\texorpdfstring{ガルニチュール・ブレオン\footnote{このガルニチュールについては、初版から掲載されているにもかかわらず、Bréhanがブルターニュ地方の町の名であることしかわかっていない。ファーヴルにもデュボワ、ベルナール『古典料理』にも言及は見られない。いささか疑問なのは、Bréhanの住人はbréhannaisという語で表わすことから、形容詞も同様であり、garniture
  à la bréhannaise
  (ガルニチュールアラブレアネーズ)の名称でもおかしくないのだが、第二版および第三版ではGarniture
  à la
  Bréhanとなっており、まるで人名のように扱われていることだろう。なお、ブルターニュ地方はアーティチョークの生産で有名だが旬は晩春から初夏にかけてであり、このガルニチュールの構成要素に初版はトリュフのスライスをそら豆のピュレを詰めたアーティチョークの上にのせる指示がある。カリフラワーも基本的には冬の野菜である。それに対してそら豆は乾物であれば1年中、フレッシュのものはやはり晩春から初夏が旬である。レシピには乾物を使うかフレッシュを使うかの指示がないが、「季節感」を演出するためには、フレッシュのそら豆を用いたいところだろう。}}{ガルニチュール・ブレオン}}\label{garniture-brehan}}

\frsub{Garniture Bréhan}

\index{garniture@garniture!brehan@--- Bréhan}
\index{brehan@Bréhan!garniture ---}
\index{かるにちゆーる@ガルニチュール!ふれおん@---・ブレオン}
\index{ふれおん@ブレオン!かるにちゆーる@ガルニチュール・---}

(牛、仔牛の塊肉の料理に添える)

\begin{itemize}
\item
  小さなアーティチョークの基底部に、そら豆のピュレをドーム状に詰める。
\item
  カリフラワーの小房10個は\protect\hyperlink{sauce-hollandaise}{ソース・オランデーズ}を軽く塗っておく\footnote{茹でてよく水気をきっておくこと。}。
\item
  小さなじゃがいも10個はバターで火を通し、パセリのみじん切りを振る。
\item
  【別添】塊肉をブレゼした煮汁。
\end{itemize}

\atoaki{}

\hypertarget{garniture-bretonne}{%
\subsubsection{ガルニチュール・ブルターニュ風}\label{garniture-bretonne}}

\frsub{Garniture à la Bretonne}

\index{garniture@garniture!bretonne@--- à la Bretonne}
\index{breton@breton(ne)!garniture à la ---ne}
\index{かるにちゆーる@ガルニチュール!ふるたーにゆふう@---・ブルターニュ風}
\index{ふるたーにゆふう@ブルターニュ風!かるにちゆーる@ガルニチュール・---}

(羊料理に添える)

\begin{itemize}
\item
  茹でた白いんげん豆またはフラジョレ\footnote{flageolet
    白いんげん豆の一種で、通常のものより小粒。} 1
  Lを\protect\hyperlink{sauce-bretonne}{ブルターニュ風ソース}(ブラウン系の派生ソース参照)であえる、パセリのみじん切りを振りかける。
\item
  【別添】塊肉の肉汁(ジュ)
\end{itemize}

\atoaki{}

\hypertarget{garniture-brillat-savarin}{%
\subsubsection[ガルニチュール・ブリヤサヴァラン]{\texorpdfstring{ガルニチュール・ブリヤサヴァラン\footnote{ジャン・アンテルム・ブリア=サヴァラン(Jean
  Anthelme
  Brillat-Savarin)(1755〜1826)。法律家であり、弁護士、一時はアメリカに亡命し、のちに裁判官として活躍したが、とりわけ、はじめ匿名で出版した『美味礼讃』\emph{Physiologie
  du
  Goût}(1825年、タイトルを直訳すれば「味覚の生理学」)で知られる。この著作は食をめぐる考察からなる随筆集だが、必ずしも生真面目な哲学的記述ばかりではない。むしろ「食をめぐる知的な面白読み物」ともいうべき内容であり、のちに「生理学もの」というジャンルが流行する嚆矢となった。これにインスパイアされたバルザックが『結婚の生理学』(1829年)を出版し文筆家バルザックとして最初のヒット作となった。その後に続けとばかりに「○○の生理学」と題した書物が19世紀中頃まで数多く出版された。その多くはほとんど文学的にも省みられることのないもので、「丸わかり○○」あるいは「○○
  のすべて」的なものばかりだった。このため、「生理学もの」のうちで文学史において一般的に価値を認められている作品は『美味礼讃』および『結婚の生理学』くらいしかない。}}{ガルニチュール・ブリヤサヴァラン}}\label{garniture-brillat-savarin}}

\frsub{Garniture Brillat-Savarin}

\index{garniture@garniture!brillat-savarin@--- Brillat-Savarin}
\index{brillat-savarin@Brillat-Savarin!garniture ---}
\index{かるにちゆーる@ガルニチュール!ふりやさうあらん@---・ブリヤサヴァラン}
\index{ふりやさうあらん@ブリヤサヴァラン!かるにちゆーる@ガルニチュール・---}

(鳥類のジビエ料理に添える)

\begin{itemize}
\item
  空焼きしたごく小さなタルトレットに、トリュフを加えた\protect\hyperlink{souffle-de-becasse}{ベカスのスフレ}\footnote{現行版の原書でベカスのスフレの項を見ると、\protect\hyperlink{becasse-favart}{ベカス・ファヴァール}と同じ、とある。なお、ファヴァールFavartというのは劇場の名称で、オペラコミック座が19世紀以来本拠地にしていたが、
    2度の火災に遭い、その度に再建された。19世紀にはイタリアオペラを主な演目とする「イタリア座」(テアトル・イタリアン)が間借りのようなかたちでファヴァール劇場を本拠にしていた時期もある。現在のファヴァール劇場は1898年に再建され、2005年以降国立となったオペラコミック座の本拠地となっている。}
  のアパレイユをピラミッド形に盛り、提供直前にやや低温のオーブンで焦がさないように火を通す。
\item
  大きなトリュフのスライス。
\item
  【別添】これを添える\protect\hyperlink{fonds-de-gibier}{ジビエのフュメ}で作った美味しい\protect\hyperlink{sauce-demi-glace}{ソース・ドゥミグラス}。
\end{itemize}

\atoaki{}

\hypertarget{garniture-bristol}{%
\subsubsection[ガルニチュール・ブリストル]{\texorpdfstring{ガルニチュール・ブリストル\footnote{Bristol
  はイギリス西部の港湾都市。このガルニチュールの名称となった由来などは不明。}}{ガルニチュール・ブリストル}}\label{garniture-bristol}}

\frsub{Garniture Bristol}

\index{garniture@garniture!bristol@--- Bristol}
\index{bristol@Bristol!garniture@garniture ---}
\index{かるにちゆーる@ガルニチュール!ふりすとる@---・ブリストル}
\index{ふりすとる@ブリストル!かるにちゆーる@ガルニチュール・---}

(牛、羊の塊肉料理に添える)

\begin{itemize}
\item
  アプリコットの形状、大きさの\protect\hyperlink{croquettes-de-riz}{米のクロケット}\footnote{本書の「米のクロケット」はアントルメすなわちデザートとして砂糖を加えて甘くつくるレシピであり、そのとおりにすべきかどうかは一考の余地がある。}
  10個。
\item
  茹でたフラジョレ\footnote{\protect\hyperlink{garniture-bretonne}{ガルニチュール・ブルターニュ風}訳注参照。}
  \(\frac{1}{2}\) Lを\protect\hyperlink{veloute}{ヴルテ}であえる。
\item
  くるみ大の丸い小さなじゃがいも20個はバターで火を通し、溶かした\protect\hyperlink{glace-de-viande}{グラスドヴィアンド}を塗る。
\item
  【別添】塊肉をブレゼした煮汁。
\end{itemize}

\atoaki{}

\hypertarget{garniture-bluxelloise}{%
\subsubsection[ガルニチュール・ブリュッセル風]{\texorpdfstring{ガルニチュール・ブリュッセル風\footnote{芽キャベツchoux
  de Bruxelles
  とアンディーヴendiveはいずれもベルギーで品種改良、開発された野菜であり、これらを組み合わせてブリュッセル風とするのはいささか安易なようにも思われる。}}{ガルニチュール・ブリュッセル風}}\label{garniture-bluxelloise}}

\frsub{Garniture à la Bruxelloise}

\index{garniture@garniture!bruxelloise@--- à la Bruxelloise}
\index{bruxellois@bruxellois(e)!garniture@garniture à la ---e}
\index{かるにちゆーる@ガルニチュール!ふりゆつせるふう@---・フリュッセル風}
\index{ふりゆつせるふう@ブリュッセル風!かるにちゆーる@ガルニチュール・---}

(牛、羊の塊肉料理に添える)

\begin{itemize}
\item
  アンディーヴ10個は白さを保つようにしてブレゼする。
\item
  シャトー\footnote{\protect\hyperlink{garniture-bouquetiere}{ガルニチュール・ブクティエール}訳注参照。}に成形したじゃがいも10個。
\item
  芽キャベツ500gは下茹でした後バターで蒸し煮する\footnote{étuver
    (エチュヴェ)。下茹での段階で \(\frac{2}{3}\)〜 \(\frac{3}{4}\)
    くらいまで火を通しておくこと。サヴォイキャベツもそうだが、下茹でにはアクを除去する意味もあり、エチュヴェの段階で変色してしまうことがあるため、アクを充分に取り除いてから比較的短時間でエチュヴェするのが望ましい。}。
\item
  【別添】マデイラ酒風味のやや軽く仕上げた\protect\hyperlink{sauce-demi-glace}{ソース・ドゥミグラス}。
\end{itemize}

\atoaki{}

\hypertarget{garniture-cancalaise}{%
\subsubsection[ガルニチュール・カンカル風]{\texorpdfstring{ガルニチュール・カンカル風\footnote{ブルターニュ地方の地名Cancale(カンカール)の形容詞
  cancalais(e) (カンカレ/カンカレーズ)。牡蠣の産地として知られ、
  cancaleという牡蠣の品種もある。17世紀、ルイ14世は、ヴェルサイユ宮殿へカンカル産カキを取り寄せていたといわれている。なお、ブルターニュ地方とはいえノルマンディ地方に非常に近い位置にあるため、牡蠣を中心にしたこのガルニチュールにブルターニュの地名を冠し、ノルマンディ風ソースを合わせるのは、一種の洒落とも考えられなくもないが、ブルターニュが言語文化的にフランスにおいてやや異質な歴史を持っていることを考慮すると、無神経な命名ともとられかねない。}}{ガルニチュール・カンカル風}}\label{garniture-cancalaise}}

\frsub{Garniture à la Cancalaise}

\index{garniture@garniture!cancalaise@--- à la Cancalaise}
\index{cancalais@cancalais(e)!garniture@garniture à la ---e}
\index{かるにちゆーる@ガルニチュール!かんかるふう@---・カンカル風}
\index{かんかるふう@カンカル風!かるにちゆーる@ガルニチュール・---}

(魚料理に添える)

\begin{itemize}
\item
  牡蠣20個の剥き身は、沸騰しない程度の温度の湯で火を通し、周囲をきれいに掃除する。殻を剥いたクルヴェットの尾125g
\item
  ノルマンディ風ソース
\end{itemize}

\atoaki{}

\hypertarget{garniture-cardinal}{%
\subsubsection[ガルニチュール・カルディナル]{\texorpdfstring{ガルニチュール・カルディナル\footnote{カトリック教会における枢機卿のこと。枢機卿の衣が真紅であることからオマールを用いた料理に付けられた名称とも、オマールが「海の枢機卿」と呼ばれるから、ともいわれている。なお、\ul{à la + 男性名詞}
  の形態は、固有名詞の場合および、対応する女性名詞がない場合にも成立する。これは
  \ul{à la manière de + 名詞} のmanière de
  が省略されたものと解釈される。さらに、料理名において à la
  も省略される傾向にあるため、garniture Cardinal あるいは garniture
  cardinal という表現も\ul{料理名においては}正しいとされている。}}{ガルニチュール・カルディナル}}\label{garniture-cardinal}}

\frsub{Garniture à la Cardinal}

\index{garniture@garniture!cardinal@--- à la Cardinal}
\index{cardinal@cardinal!garniture@garniture à la ---}
\index{かるにちゆーる@ガルニチュール!かるていなる@---・カルディナル}
\index{かるていなる@カルディナル!かるにちゆーる@ガルニチュール・---}
\index{すうききよう@枢機卿 ⇒ カルディナル!かるにちゆーる@ガルニチュール・カルディナル}

(魚料理に添える)

\begin{itemize}
\item
  立派なオマールの尾の身をやや斜めに厚さ1cm程度にスライスしたもの10枚。
\item
  真黒なトリュフのスライス10枚。
\item
  さいの目に切ったオマールの身60 gとトリュフ50 g。
\item
  \protect\hyperlink{sauce-cardinal}{ソース・カルディナル}。
\end{itemize}

\atoaki{}

\hypertarget{garniture-castillane}{%
\subsubsection[ガルニチュール・カスティリア風]{\texorpdfstring{ガルニチュール・カスティリア風\footnote{Castilla
  (カスティーリャ、カスティージャ)はスペイン中部の地域で、中世はカスティリア王国だった。「カステラ」の語源ともいわれる。}}{ガルニチュール・カスティリア風}}\label{garniture-castillane}}

\frsub{Garniture à la Castillane}

\index{garniture@garniture!castillane@--- à la Castillane}
\index{castillan@castillan(e)!garniture@garniture à la ---e}
\index{かるにちゆーる@ガルニチュール!かすていりあふう@---・カスティリア風}
\index{かすていりあふう@カスティリア風!かるにちゆーる@ガルニチュール・---}

(トゥルヌド、ノワゼットに添える)

\begin{itemize}
\item
  \protect\hyperlink{pommes-de-terre-duchesse}{ポム・デュシェス}で作ったた小さなケースにドリュールを塗ってオーブンで焼き色を付ける。そこに、軽くにんにく風味を効かせた\protect\hyperlink{portugaise}{トマトのフォンデュ}を詰める。
\item
  皿の周囲に、輪切りにして塩こしょうし、小麦粉をまぶして油で揚げた玉ねぎを飾る。
\item
  【別添】デグラセした肉汁(ジュ)\footnote{トゥルヌド、ノワゼットをフライパンでソテーし、デグラセしてトマトピュレまたは本文にあるトマトのフォンデュを加えてソースにするということ。}をトマト風味に仕上げる。
\end{itemize}

\atoaki{}

\hypertarget{garniture-chambord}{%
\subsubsection[ガルニチュール・シャンボール]{\texorpdfstring{ガルニチュール・シャンボール\footnote{シャンボールとは16世紀、ロワール河の近くに建てられた瀟洒な城の名。このガルニチュールを添えた場合、料理名にシャンボールが冠される。鯉、サーモンが代表的だが、とりわけ19世紀は鯉が好まれ、カレーム『19
  世紀フランス料理』第2巻では鯉のシャンボールだけで近代風、ロヤイヤル、レジャンスの3種の仕立てについて詳述されている(pp.181-189)。なお、このガルニチュールの構成も時代や料理人によって多少の変化があり、『ラルース・ガストロノミック』初版では、魚でつくった大小のクネル、マッシュルーム、舌びらめのフィレ、バターでソテーした白子、オリーヴ形に成形したトリュフ、クールブイヨンで火を通したエクルヴィス、揚げたクルトン、となっている(p.516)。}}{ガルニチュール・シャンボール}}\label{garniture-chambord}}

\frsub{Garniture Chambord}

\index{garniture@garniture!chambord@--- Chambord}
\index{chambord@Chambord!garniture@garniture ---}
\index{かるにちゆーる@ガルニチュール!しやんほーる@---・シャンボール}
\index{しやんほーる@シャンボール!かるにちゆーる@ガルニチュール・---}

(魚のブレゼの大掛かりな仕立てに添える\footnote{ルルヴェのこと。\protect\hyperlink{releve}{第二版序文訳注}参照。19世紀前半くらいまではカトリックの習慣としての「小斉」が比較的厳格に守られており、料理人たちは四旬節やその他の小斉の日の献立としていかに豪華で美味な魚料理を提供するかに腐心していたのが、17〜18世紀の料理書を読むとよくわかる。カレームの著書にも魚の大掛かりな仕立てのレシピが数多く収められている。})

\begin{itemize}
\item
  トリュフを加えてスプーンで成形した魚のファルスで作ったクネル10個。
\item
  長卵形の大きな、表面に装飾を施したクネル4個。
\item
  渦巻模様を付けた\footnote{原文 canneler (カヌレ)。この場合はtourner
    (トゥルネ)とほぼ同義だが、凹凸の刻み模様を付けた、の意。}小さなマッシュルーム200
  g。
\item
  鯉の白子を1
  cm程度の厚さにスライスして塩こしょうし、小麦粉をまぶしてソテーしたもの10枚。
\item
  オリーブ形に成形した\footnote{tourner
    (トゥルネ)。原義は「回す」。野菜などを包丁ではなく材料を回すようにして皮を剥いたり成形するところから。}トリュフ200
  g。
\item
  エクルヴィス\footnote{ecrevisse ヨーロッパザリガニ。}6尾は\protect\hyperlink{courtbouillon-a}{クールブイヨン}\footnote{court-bouillon
    (クールブイヨン)。court
    は少量の意。つまり、原則としてはできるだけ少量の液体を煮汁として魚介類その他を加熱調理するのに用いる。また、とりわけ魚介類の場合は沸騰しない程度の温度で火を通す(pocher
    ポシェする)のが原則。たんなる水、塩水だけでなく、ワインや香味野菜、香辛料などを加えて風味付け(および場合によっては臭みのマスキング効果)も兼ねて事前に用意しておくこともある。ただしこれらはあくまでも原則論にすぎない。詳細は\protect\hyperlink{poissons}{魚料理}の\protect\hyperlink{serie-de-courts-bouillons-de-poisson}{クールブイヨン}および\protect\hyperlink{ecrevisse-a-la-nage}{エクルヴィス・ナージュ}参照のこと。なお、エクルヴィスの場合は上記の「少量」にあまりこだわらず、後ではさみを背に回しやすくなるように鍋に入れて加熱すればいいだろう。エクルヴィスはジストマ(寄生虫)のリスクがあるためしっかり加熱すること。またエクルヴィスは腕が取れやすいが、その場合でも可食部である尾の身には問題がないので装飾以外の利用はもちろん可能であり、装飾用としてはロス分を見込んで用意しておくのがいいだろう。}で火を通し、はさみを背に回すように成形する\footnote{trousser
    (トゥルセ)。}(しなくてもよい)。
\item
  食パンを鶏のとさかの形に切りバターで揚げたクルトン6枚。
\item
  【別添】魚をブレゼした際の煮汁をベースにしたソース。
\end{itemize}

\atoaki{}

\hypertarget{garniture-chatelaine}{%
\subsubsection[ガルニチュール・シャトレーヌ]{\texorpdfstring{ガルニチュール・シャトレーヌ\footnote{châtelain(e)
  (シャトラン/シャトレーヌ)。城館の主の意。城館に住む者を思わせる豪華な、の意で料理名として使われるようになったようだ。}}{ガルニチュール・シャトレーヌ}}\label{garniture-chatelaine}}

\frsub{Garniture Châtelaine}

\index{garniture@garniture!chatelaine@--- Châtelaine}
\index{chatelaine@Châtelaine!garniture@garniture ---}
\index{かるにちゆーる@ガルニチュール!しやとれーぬ@---・シャトレーヌ}
\index{しやとれーぬ@シャトレーヌ!かるにちゆーる@ガルニチュール・---}

(牛、羊の塊肉や鶏料理に添える)

\begin{itemize}
\item
  アーティチョークの基底部10個に、固く作った\protect\hyperlink{sauce-soubise}{スビーズ}を詰める。
\item
  殻を剥いて塊肉をブレゼした煮汁で蒸し煮したマロン30個。
\item
  \protect\hyperlink{pommes-de-terre-noisette}{じゃがいも・ノワゼット}
  300 g。
\item
  【別添】\protect\hyperlink{sauce-madere}{ソース・マデール}を加えたブレゼの煮汁。
\end{itemize}

\atoaki{}

\hypertarget{garniture-chipolata}{%
\subsubsection[ガルニチュール・シポラタ]{\texorpdfstring{ガルニチュール・シポラタ\footnote{もとはイタリアで玉ねぎとソーセージを煮込んだ料理(cipollata
  チポッラータ \textless{} cipolla
  チポッラ=玉ねぎ)を意味していたが、フランスに伝わった際に、語本来の意味に含まれていた玉ねぎが脱落して、羊腸に豚挽肉を詰めた小さなソーセージをこう呼ぶようになったといわれている。}}{ガルニチュール・シポラタ}}\label{garniture-chipolata}}

\frsub{Garniture à la Chipolata}

\index{garniture@garniture!chipolata@--- à la Chipolata}
\index{chipolata@chipolata!garniture@garniture à la ---}
\index{かるにちゆーる@ガルニチュール!しほらた@---・シポラタ}
\index{しほらた@シポラタ!かるにちゆーる@ガルニチュール・---}

(牛、羊の塊肉および鶏料理に添える)

\begin{itemize}
\item
  小玉ねぎ20個は下茹でしてバターで色艶よく炒める\footnote{glacer
    (グラセ)。本文下のにんじんも同様の指示。}。
\item
  シポラタソーセージ10本。
\item
  コンソメで煮たマロン10個。
\item
  塩漬け豚バラ肉125 gはさいの目に切って、強火でこんがり炒める。
\item
  オリーブ形に成形して下茹でし、バターで色艶よく炒めたにんじん20個(なくてもよい)。
\item
  【別添】このガルニチュールを添える料理の煮汁を加えた\protect\hyperlink{sauce-demi-glace}{ソース・ドゥミグラス}。
\end{itemize}

\atoaki{}

\hypertarget{garniture-choisy}{%
\subsubsection[ガルニチュール・ショワジー]{\texorpdfstring{ガルニチュール・ショワジー\footnote{パリのセーヌ川上流(=東側)約12
  kmのところにある Choisy-le-Roi
  の地名に由来。17世紀にショワジー城が建てられ、18世紀にこれを相続したルイ15世が狩りの際に使う邸宅として利用し、現在の名称ショワジールロワになった。その後、ポンパドゥール夫人がここに移り住み、豪華な夕食会がしばしば開かれたという。ショワジーの名称はレチュを用いた料理に付けられることが多い。}}{ガルニチュール・ショワジー}}\label{garniture-choisy}}

\frsub{Garniture Choisy}

\index{garniture@garniture!choisy@--- Choisy}
\index{choisy@Choisy!garniture@garniture ---}
\index{かるにちゆーる@ガルニチュール!しよわしー@---・ショワジー}
\index{しよわしー@ショワジー!かるにちゆーる@ガルニチュール・---}

(トゥルヌドおよびノワゼットに添える)

\begin{itemize}
\item
  半割りにした\protect\hyperlink{laitue-braise}{レチュのブレゼ} 10個。
\item
  シャトーに成形した小さなじゃがいも20個。
\item
  【別添】バターを加えた\protect\hyperlink{glace-de-viande}{グラスドビアンド}
\end{itemize}

\atoaki{}

\hypertarget{garniture-choron}{%
\subsubsection[ガルニチュール・ショロン]{\texorpdfstring{ガルニチュール・ショロン\footnote{19世紀にあったパリの有名レストラン、ヴォワザンの料理長の名。\protect\hyperlink{sauce-bearnaise-tomatee}{ソース・ショロン}も参照。}}{ガルニチュール・ショロン}}\label{garniture-choron}}

\frsub{Garniture Choron}

\index{garniture@garniture!choron@--- Choron}
\index{choron@Choron!garniture@garniture ---}
\index{かるにちゆーる@ガルニチュール!しよろん@---・ショロン}
\index{しよろん@ショロン!かるにちゆーる@ガルニチュール・---}

(トゥルヌドおよびノワゼットに添える)

\begin{itemize}
\item
  中位か小さいアーティチョークの基底部をにバターであえたアスパラガスの穂先を詰める。アスパラガスがなければ、バターであえた小粒のプチポワでもいい。
\item
  \protect\hyperlink{pommes-de-terre-noisette}{じゃがいものノワゼット}
  30個。
\item
  【別添】\protect\hyperlink{sauce-bearnaise-tomatee}{トマト入りソース・ベアルネーズ}。
\end{itemize}

\atoaki{}

\hypertarget{garniture-clamart}{%
\subsubsection[ガルニチュール・クラマール]{\texorpdfstring{ガルニチュール・クラマール\footnote{パリ郊外の町の名。プチポワを使った料理にこの名が冠されるものがいくつかある。}}{ガルニチュール・クラマール}}\label{garniture-clamart}}

\frsub{Garniture à la Clamart}

\index{garniture@garniture!clamart@--- à la Clamart}
\index{clamart@Clamart!garniture@garniture à la ---}
\index{かるにちゆーる@ガルニチュール!くらまーる@---・クラマール}
\index{くらまーる@クラマール!かるにちゆーる@ガルニチュール・---}

(牛、羊の塊肉の料理に添える)

\begin{itemize}
\item
  \protect\hyperlink{petits-pois-francaise}{プチポワ・アラフランセーズ}に細かく刻んだレチュの葉を加えてバターであえ、空焼きしたタルトレット10個に詰める。
\item
  \protect\hyperlink{pommes-de-terre-macaire}{じゃがいものマケール}で作った円形の小さな台の上に、タルトレットをひとつずつのせる。
\item
  【別添】\protect\hyperlink{jus-de-veau-lie}{とろみを付けたジュ}\footnote{このガルニチュールを添える料理がポワレ(\protect\hyperlink{sauce-bigarade}{ソース・ビガラード}訳注および\protect\hyperlink{releves-et-entrees}{肉料理}参照)の場合には、鍋に残った香味野菜(マティニョン)にフォン少量を注いで風味を引き出し、それにコーンスターチでとろみを付けることになるだろう。}。
\end{itemize}

\atoaki{}

\hypertarget{garniture-de-compote}{%
\subsubsection[ガルニチュール・コンポート]{\texorpdfstring{ガルニチュール・コンポート\footnote{compote
  (コンポート)。果物のシロップ煮のイメージが強いが、肉や野菜をばらばらになるまで煮込んだ料理のことも指す。}}{ガルニチュール・コンポート}}\label{garniture-de-compote}}

\frsub{Garniture de Compote}

\index{garniture@garniture!compote@--- de Compote}
\index{compote@compote!garniture@garniture de ---}
\index{かるにちゆーる@ガルニチュール!こんほーと@---・コンポート}
\index{こんほーと@コンポート!かるにちゆーる@ガルニチュール・---}

(鳩およびプレ・ド・グラン\footnote{poulet de grain
  鶏の大きさや飼育方法による区別については\protect\hyperlink{sauce-chaud-froid-vert-pre}{ソース・ショフロワ・ヴェールプレ}参照。}に添える)

\begin{itemize}
\item
  塩漬け豚バラ肉\footnote{lard de poitrine
    (ラールドポワトリーヌ)、lard maigre
    (ラールメーグル)あるいは原文のように合わせて lard de poitrine
    maigre
    (ラールドポワトリーヌメーグル)とも呼ぶが、塩漬けにして熟成させた豚バラ肉のこと。通常、lard
    だけの場合は lard gras
    (ラールグラ)すなわち豚背脂のことを意味するので注意。}250は拍子木\footnote{lardon
    (ラルドン)。たんに lardon
    というだけで、この豚バラ肉の塩漬けを拍子木状に切ったものを意味することはごく一般的で、塩漬け後に冷燻にかけた
    lard de poitrine fumé を拍子木に切ったものは lardon fumé
    (ラルドンフュメ)と呼ばれる。}に切り、下茹でしてからバターでこんがり炒める\footnote{rissoler
    (リソレ)油脂を熱して、素材の表面をこんがり焼くこと。語源は中世からある
    rissole
    (リソール)という円形または半円形、塩味または甘い焼き菓子(揚げ菓子)---
    つまり時代や地域とともに非常にバリエーションに富むものだが、こんがりとした色合いに仕上げるのは共通している。}。
\item
  小玉ねぎ300 gは下茹でしてバターで色艶よく炒める\footnote{glacer
    (グラセ)。}。
\item
  小さなマッシュルーム300 gは生のまま2つに切り、バターで炒める。
\item
  これらは鳩とともに火入れを仕上げ、供する際には鳩を覆うようにガルニチュールを盛る。
\end{itemize}

\atoaki{}

\hypertarget{garniture-conti}{%
\subsubsection[ガルニチュール・コンティ]{\texorpdfstring{ガルニチュール・コンティ\footnote{ブルボン王家のひとつCondé(コンデ)家の傍流(いわゆる分家筋)で、代々のコンティ大公
  le prince de Conti
  (ルプランスドコンティ)がいる。もとはピカルディ地方アミアンの近くにあるContyというところを領地にしていたのが家名の起源。コンティの名は本書でも、レンズ豆のポタージュ、\protect\hyperlink{puree-conti}{ピュレ・コンティ}が収められている。このガルニチュールは18世紀のコンティ大公ルイ・フランソワ・ド・ブルボン(1727〜
  1776)の料理人が考案したと伝えられているが、たとえ事実であったとしても、あまりにシンプルなものなので、ガルニチュールとして供することを考えた、というのがせいぜいのところか。}}{ガルニチュール・コンティ}}\label{garniture-conti}}

\frsub{Garniture Conti}

\index{garniture@garniture!conti@--- Conti}
\index{conti@Conti!garniture@garniture de ---}
\index{かるにちゆーる@ガルニチュール!こんてい@---・コンティ}
\index{こんてい@コンティ!かるにちゆーる@ガルニチュール・---}

(牛、羊の塊肉のブレゼに添える)

\begin{itemize}
\item
  レンズ豆\footnote{lentille
    (ロンティーユ)。西アジア原産。レンズ豆はおそらく農耕がはじまったごく初期からの作物で、エジプトや地中海沿岸で多く栽培されていた。温暖な気候に向いた作物であり、その意味ではフランス北部に縁があるコンティ大公の名はふさわしくないかも知れない。旧約聖書の「創世記」にも出てくる。アブラハムの息子イサクの双子のうちのひとりエサウはすぐれた狩人に、もうひとりのヤコブは「穏かな人で天幕の周りで働くのを常として」いた。(中略)ある日のこと、ヤコブが煮物をしていると、エサウが疲れきって野原から帰ってきた。エサウはヤコブに言った。『お願いだ、その赤いもの(アドム)、そこの赤いものを食べさせてほしい。わたしは疲れきっているんだ。』(中略)エサウは誓い、長子の権利をヤコブに譲ってしまった。ヤコブはエサウにパンとレンズ豆の煮物を与えた。(中略)こうしてエサウは長子の権利を軽んじた。(「創世記」
    25-27〜34、新共同訳『聖書』)」。これを踏まえると、Condé
    すなわちコンデ大公の名を冠した料理、とりわけポタージュ\protect\hyperlink{puree-conde}{ピュレ・コンデ}が赤いんげん豆のポタージュであることと、レンズ豆を主素材とした「ガルニチュール・コンティ」およびポタージュ「ピュレ・コンティ」の関係には考えさせられるところがある。}のピュレ750
  g。
\item
  脂身のほとんどない豚バラ肉の塩漬け250
  gは長方形に切って、レンズ豆を煮る際に一緒に煮ておく。
\item
  【別添】ブレゼの煮汁。
\end{itemize}

\atoaki{}

\hypertarget{garniture-a-la-commodore}{%
\subsubsection[ガルニチュール・コモドール]{\texorpdfstring{ガルニチュール・コモドール\footnote{もとは英語
  commodore であり、イギリスでは艦隊司令官、アメリカでは准将の意。}}{ガルニチュール・コモドール}}\label{garniture-a-la-commodore}}

\frsub{Garniture à la Commodore}

\index{garniture@garniture!commodore@--- à la Commodore}
\index{commodore@commodore!garniture@garniture de ---}
\index{かるにちゆーる@ガルニチュール!こもとーる@---・コモドール}
\index{こもとーる@コモドール!かるにちゆーる@ガルニチュール・---}

(魚の大掛かりな仕立てに添える)

\begin{itemize}
\item
  エクルヴィスの尾の身を入れた小さなグラタン皿10個。
\item
  メルラン\footnote{merlan (メルロン)タラ科の海水魚。}の\protect\hyperlink{farce-a}{ファルス}に\protect\hyperlink{beurre-d-ecrevissse}{エクルヴィスバター}を加え、スプーンで成形したクネル10個。
\item
  大きな\protect\hyperlink{moules-a-la-villeroy}{ムール貝のヴィルロワ}
  10個。
\item
  仕上げにエクルヴィスバターを加えた\protect\hyperlink{sauce-normande}{ノルマンディ風ソース}。
\end{itemize}

\atoaki{}

\hypertarget{garniture-cussy}{%
\subsubsection[ガルニチュール・キュシー]{\texorpdfstring{ガルニチュール・キュシー\footnote{キュシー侯爵(1767〜1841)。\protect\hyperlink{osbservation-sur-la-sauce}{基本ソース 概説}訳注参照。}}{ガルニチュール・キュシー}}\label{garniture-cussy}}

\frsub{Garniture Cussy}

\index{garniture@garniture!cussy@--- Cussuy}
\index{cussy@Cussy (marquis de)!garniture@garniture ---}
\index{かるにちゆーる@ガルニチュール!きゆしー@---・キュシー}
\index{きゆしー@キュシー!かるにちゆーる@ガルニチュール・---}

(トゥルヌド、ノワゼット、鶏料理に添える)

\begin{itemize}
\item
  マロンのピュレを詰めてグリル焼きした大きなマッシュルーム10個。
\item
  完全に球形に成形し、マデイラ酒風味で火を通した小さなトリュフ10個。
\item
  大きな雄鶏のロニョン\footnote{rognon
    仔牛などでは腎臓のこと。鶏の場合はrognon
    blanc(ロニョンブロン)とも呼び、精巣のこと。この場合は後者。もちろんきちんと加熱調理したものをガルニチュールの構成要素とする。}20個。
\item
  【別添】\protect\hyperlink{sauce-madere}{ソース・マデール}
\end{itemize}

\atoaki{}

\hypertarget{garniture-Daumont}{%
\subsubsection[ガルニチュール・ドモン]{\texorpdfstring{ガルニチュール・ドモン\footnote{ドモン公爵家duché
  d'Aumon(デュシェドモン)にちなんだ名称をいわれている。}}{ガルニチュール・ドモン}}\label{garniture-Daumont}}

\frsub{Garniture Daumont}

\index{garniture@garniture!daumont@--- Daumont}
\index{daumont@Daumont!garniture@garniture ---}
\index{かるにちゆーる@ガルニチュール!ともん@---・ドモン}
\index{ともん@ドモン!かるにちゆーる@ガルニチュール・---}

(魚料理に添える)

\begin{itemize}
\item
  バターで鍋に蓋をして弱火で火を通した\footnote{étuver au beurre
    (エチュヴェオブール)}マッシュルーム10個に、それぞれエクルヴィスの尾の身を半分に切ったもの6枚ずつ添える。
\item
  \protect\hyperlink{farce-c}{生クリーム入り魚のファルス}を小さな球形にし、トリュフで装飾を施したクネル10個。
\item
  厚さ1cm程にスライスした\footnote{escalope (エスカロップ)肉や魚を1〜2
    cmの厚さで、筋腺維と直角にスライスした円形または楕円形にスライスしたもの。}白子10枚はイギリス式パン粉衣\footnote{paner
    à l'anglaise
    (パネアロングレーズ)。素材に小麦粉をまぶしてから、卵液に浸し、細かいパン粉で衣を付けること。日本でフライなどをつくる際に一般的な方法とよく似ているが、日本では粗いパン粉が好まれるのに対して、フランスやイギリスでは細かいパン粉を使うのが一般的。}を付けて油で揚げる。
\item
  \protect\hyperlink{sauce-nantua}{ソース・ナンチュア}
\end{itemize}

\atoaki{}

\hypertarget{garniture-a-la-dauphine}{%
\subsubsection[ガルニチュール・ドフィーヌ]{\texorpdfstring{ガルニチュール・ドフィーヌ\footnote{à
  la Dauphine
  (アラドフィーヌ)王太子妃風、の意。この料理名には由来や理由がないことがほとんど。あえていえば「豪華」であるという程度だが、存外、簡素な仕立ての料理にも付けられることがある。}}{ガルニチュール・ドフィーヌ}}\label{garniture-a-la-dauphine}}

\frsub{Garniture à la Dauphine}

\index{garniture@garniture!dauphine@--- à la Dauphine}
\index{dauphin@dauphin(e)!garniture@garniture ---e}
\index{かるにちゆーる@ガルニチュール!とふいぬ@---・ドフィーヌ}
\index{とふあん@ドファン/ドフィーヌ!かるにちゆーる@ガルニチュール・ドフィーヌ}

(牛、羊の塊肉の料理に添える)

\begin{itemize}
\item
  \protect\hyperlink{pomme-de-terres-dauphine}{じゃがいものドフィーヌ}をアパレイユにした\protect\hyperlink{croquettes}{クロケット}
  20個。大きな塊肉料理に添える場合はコルクの栓の形状に、トゥルヌドやノワゼットに添えるときは平たい円盤の形にする。
\item
  【別添】マデイラ酒風味の\protect\hyperlink{sauce-demi-glace}{ソース・ドゥミグラス}。
\end{itemize}

\atoaki{}

\hypertarget{garniture-a-la-dieppoise}{%
\subsubsection[ガルニチュール・ディエープ風]{\texorpdfstring{ガルニチュール・ディエープ風\footnote{dieppois(e)
  (ディエポワ/ディエポワーズ) \textless{} Dieppe
  (ディエープ)ノルマンディ地方の港町の名。}}{ガルニチュール・ディエープ風}}\label{garniture-a-la-dieppoise}}

\frsub{Garniture à la Dieppoise}

\index{garniture@garniture!dieppoise@--- à la Dieppoise}
\index{dieppois@dieppois(e)!garniture@garniture ---e}
\index{かるにちゆーる@ガルニチュール!ていえーふふう@---・ディエープ風}
\index{ていえーふふう@ディエープ風!かるにちゆーる@ガルニチュール・---}

(魚料理に添える)

\begin{itemize}
\item
  殻を剥いたクルヴェット\footnote{小海老。小さめのcrevette
    grise(クルヴェットグリーズ)とやや大きめのcrevette
    rose(クルヴェットローズ)が代表的。}の尾の身100g。
\item
  \(\frac{3}{4}\)
  L(約30個)のムール貝は白ワインを加えた湯で沸騰させない程度の温度で火を通し\footnote{pocher
    (ポシェ)}、周囲をきれいに掃除する\footnote{ébarber
    (エバルベ)貝類の身の周囲をきれいにすること。帆立貝の場合は「ひもを取る」ともいう。}。
\item
  このガルニチュールを添える魚の煮汁を煮詰めて加えた\protect\hyperlink{sauce-vin-blanc}{白ワインソース}。
\end{itemize}

\atoaki{}

\hypertarget{garniture-doria}{%
\subsubsection[ガルニチュール・ドリア]{\texorpdfstring{ガルニチュール・ドリア\footnote{原書現行版ではDorlaとなっているが初版〜第三版はDoria
  であり、現行版の明らかな誤植。19世紀パリのカフェ・アングレの顧客として知られていた名家ドリアの名を冠したといわれている。このドリア家は12世紀ジェノヴァの
  de Auria (ラテン語の filiis
  Auriaeすなわちアウリアの子孫の意)から発する由緒ある家系として有名。なお、日本の洋食のドリアは
  1930年頃横浜ホテルニューグランド総料理長サリー・ワイルが発案したものといわれており、上記のドリア家とはまったく関係がない。また古代ギリシア時代の民族ドーリア人とも関係がない。ちなみに、バルザックの小説『幻滅』におなじ発音の名のDauriatという登場人物がいる。}}{ガルニチュール・ドリア}}\label{garniture-doria}}

\frsub{Garniture Doria}

\index{garniture@garniture!doria@--- Doria}
\index{doria@Doria!garniture@garniture ---e}
\index{かるにちゆーる@ガルニチュール!とりあ@---・ドリア}
\index{とりあ@ドリア!かるにちゆーる@ガルニチュール・---}

(魚料理に添える)

\begin{itemize}
\item
  オリーブ形に剥いたきゅうり\footnote{concombre
    (コンコンブル)日本で一般的なきゅうりと品種系統も異なるものが多く、サイズも太さ4〜5
    cm、長さ30〜45
    cmで収穫する(品種によって異なる)。青臭さがなく、加熱調理することが多い。}30個をバターで蒸し煮する\footnote{étuver
    (エチュヴェ)。}。
\item
  表皮を剥いて種を取り除いたレモンのスライスを魚の上に並べる。魚は\protect\hyperlink{meuniere}{ムニエル}にしたもの。
\end{itemize}

\atoaki{}

\hypertarget{garniture-dubarry}{%
\subsubsection[ガルニチュール・デュバリー]{\texorpdfstring{ガルニチュール・デュバリー\footnote{Madame
  du Barry (マダムデュバリー)デュバリー夫人
  (1743〜1793)のこと。ルイ15世の公妾であり、フランス革命により断頭台に送られ命を落したことで知られる。もとはシャンパーニュ地方の貧しい家庭の生まれ。パリに出てのち「お針子」などの仕事や娼婦をしていたが、デュ・バリー子爵に囲われ、いわゆるdemi-mondaine(ドゥミモンデーヌ)、
  courtisane
  (クルティザーヌ)すなわち高等娼婦として知られるようになる。その後、ポンパドゥール夫人を亡くしたルイ15世が彼女を妾にすることにし、形式上、デュ・バリー子爵の弟と結婚したことにして、正式な社交界デビューを果たした。フランス史において「女性」であることを最大限利用して社会的にのしあがった典型例のひとつ。フランス革命のさなか、捕えられて断頭台へ連れていかれる際に、ほとんどの貴族の女性が取り乱さず凛として死に臨んだのに対し、デュバリー夫人ただひとりだけが狂乱し泣き叫んで命乞いした、という逸話が残っている。ただし、それはロベスピエールによる恐怖政治への警鐘になり得たという見解も少なくない。}}{ガルニチュール・デュバリー}}\label{garniture-dubarry}}

\frsub{Garniture Dubarry}

\index{garniture@garniture!dubarry@--- Dubarry}
\index{dubarry@Dubarry!garniture@garniture ---}
\index{かるにちゆーる@ガルニチュール!とゆはりー@---・デュバリー}
\index{とゆはりー@デュバリー!かるにちゆーる@ガルニチュール・---}

(牛、羊の塊肉やノワゼット、トゥルヌドに添える)

\begin{itemize}
\item
  小さく分けたカリフラワーの花房を小さなボウルに詰め半球形にまとめて裏返し\protect\hyperlink{sauce-mornay}{ソース・モルネー}で覆ったもの10個。おろした
  \footnote{râper (ラペ)}チーズを振りかけて高温のオーブンでこんがり焼く\footnote{gratiner
    (グラティネ)。また、原文moulés en
    boulesを文字通りに読むと「完全な球形」にするようにも解釈出来なくはないが、そのためには強力な「つなぎ」が必要になる。ソース・モルネー以外に「つなぎ」の役割を果たすものの指定がないため、これでは球形を維持する「つなぎ」として熱に弱過ぎるだろう。ここは\protect\hyperlink{chou-fleur-au-gratin}{カリフラワーのグラタン}にあるように
    moulé dans un bol
    「ボウルに詰める」と同様と解釈していいと思われる。。}。
\item
  \protect\hyperlink{pommes-de-terre-fondantes}{じゃがいものフォンダント}
  10個。
\item
  【別添】塊肉をブレゼあるいはポワレした際のフォン、もしくはノワゼットやトゥルヌドをソテした後にデグラセしてソースに仕上げる。
\end{itemize}

\atoaki{}

\hypertarget{garniture-a-la-duchesse}{%
\subsubsection[ガルニチュール・デュシェス]{\texorpdfstring{ガルニチュール・デュシェス\footnote{duc
  (デュック=公爵)、duchesse(デュシェス=公爵夫人)。ここではたんに、\protect\hyperlink{pommes-de-terre-duchesse}{じゃがいものデュシェス}を用いるからこの名称になっているが、デュシェスそれ自体にも料理名としての由来や根拠はほとんどない。}}{ガルニチュール・デュシェス}}\label{garniture-a-la-duchesse}}

\frsub{Garniture à la Duchesse}

\index{garniture@garniture!duchesse@--- à la Duchesse}
\index{duc@duc / duchesse!garniture@garniture à la Duchesse}
\index{かるにちゆーる@ガルニチュール!てゆせす@---・デュシェス}
\index{てゆしえす@デュシェス!かるにちゆーる@ガルニチュール・---}

(牛、羊の塊肉の料理やノワゼット、トゥルヌドに添える)

\begin{itemize}
\item
  じゃがいも・デュシェスを舟形または円盤状かブリオシュ型に詰めて成形し、溶き卵\footnote{dorer
    (ドレ)\textless{} dorure
    (ドリュール)焼いた際に艶を出すために塗る溶き卵。水や牛乳などを混ぜることもある。}を塗って、提供直前にオーブンでこんがり焼いたもの20個。
\item
  【別添】\protect\hyperlink{sauce-madere}{ソース・マデール}
\end{itemize}

\atoaki{}

\hypertarget{garniture-a-la-favorite}{%
\subsubsection[ガルニチュール・ラファヴォリータ]{\texorpdfstring{ガルニチュール・ラファヴォリータ\footnote{『愛の妙薬』や『ランメルモールのルチア』で知られる作曲家ガエタノ・ドニゼッティ(1797〜1848)のグランドオペラ\emph{La
  Favorite}(1840
  年初演)にあやかって付けられた名称。グランドオペラ(grand opéra
  グロントペラ、複数形grands
  opérasグロンゾペラ)とは19世紀前半から中葉にかけて、パリのオペラ座において、豪華な舞台装置と派手な演出、大編成のオーケストラ、歴史的題材などをテーマとしたわかりやすい悲劇的筋書きなどを特徴としたオペラ作品の様式のこと。ジャコモ・マイアベーア『悪魔ロベール』(1831年)や『ユグノー教徒』(1836年)などが代表的。なお、ロッシーニはこの様式が流行る前のオペラ作曲家と位置付けられていることが多いが、『ウィリアム・テル』(1829年。フランス語原題
  \emph{Guillaume
  Tell}ギヨーム・テル)あるいはそれに先立つ1827年の『モーセとファラオン』をこのジャンルの嚆矢と見なす場合もある。その他の代表的なグランドオペラの作曲家にダニエル=フランソワ・オーベール(1782〜
  1871)やジャック=フロマンタル・アレヴィ(1799〜1862)がいる。ドニゼッティのこの作品もロッシーニやマイアベーアの諸作品同様、フランス語の台本、歌詞であり、原題もフランス語で
  \emph{La
  Favorite}(ラファヴォリット)だが、どういうわけか、こんにちの日本ではイタリア語式に直した『ラファヴォリータ』と呼ばれることが多いためにここではそれに合わせた。なおこのオペラのリブレット(台本、歌詞)はアルフォンス・ロワイエとギュスターヴ・ヴァエズによるものだが、18世紀バキュラール・ダルノー(1718〜1805)の戯曲『不幸な恋人たち』を原作としている。さらにいえば、バキュラール・ダルノーの戯曲もまた、クロディーヌ・ゲラン・ド・トンサン(1682〜1749)の小説『コマンジュ伯爵の手記』を翻案したもの。『ロメオとジュリエット』の物語のバリエーションのひとつともいえるこの小説は18世紀に大きな反響を呼び、多くの小説、戯曲に影響を与えた。ダルノーの戯曲はその代表例。}}{ガルニチュール・ラファヴォリータ}}\label{garniture-a-la-favorite}}

\frsub{Garniture à la Favorite}

\index{garniture@garniture!favorite@--- à la Favorite}
\index{favorite@Favorige (La)!garniture@garniture à la Favorite}
\index{かるにちゆーる@ガルニチュール!らふあうおりーた@---・ラファヴォリータ}
\index{らふあうおりーた@ラファヴォリータ!かるにちゆーる@ガルニチュール・---}

(ノワゼット、トゥルヌドに添える)

\begin{itemize}
\item
  小さめのフォワグラを厚さ1 cm程にスライス\footnote{éscalope
    (エスカロップ)。}し、塩こしょうしてから小麦粉をまぶしてバターでソテーしたもの10枚。
\item
  大きなトリュフのスライスをソテーしたフォラグラに1枚ずつのせる。
\item
  アスパラガスの穂先を束にしたもの。
\item
  【別添】\protect\hyperlink{jus-de-veau-lie}{とろみを付けたジュ}。
\end{itemize}

\atoaki{}

\hypertarget{garniture-a-la-fermiere}{%
\subsubsection[ガルニチュール・フェルミエール]{\texorpdfstring{ガルニチュール・フェルミエール\footnote{日本語にすれば「農場主風」。野菜を厚さ1
  mmくらい、長さ1 cm程度の四角形に切ることを détailler en paysanne
  (デタイエオンペイザーヌ)というが、そのpaysanneとはpaysan(ペイゾン=農民)の女性形であり、このガルニチュールでは野菜をすべてそのように切るところにかけての名称。}}{ガルニチュール・フェルミエール}}\label{garniture-a-la-fermiere}}

\frsub{Garniture à la Fernière}

\index{garniture@garniture!fermiere@--- à la Fermière}
\index{fermier@fermier/fermière!garniture@garniture à la Fermière}
\index{かるにちゆーる@ガルニチュール!ふえるみえーる@---・フェルミエール}
\index{ふえるみえ@フェルミエ/フェルミエール!かるにちゆーる@ガルニチュール・フェルミエール}

(鶏料理に添える)

\begin{itemize}
\item
  にんじん150 gと蕪150 gは厚さ1 mm程度、長さ1
  cm程度の四角形に切る\footnote{原文émincer en paysanne
    (エマンセオンペイザーヌ)。ペイザーヌに切る場合、動詞にはémincer
    薄くスライスする、も使われる。}。
\item
  玉ねぎ50 gとセロリ50 gも同様に切る。
\item
  これらを鍋に入れてバターと、塩3g、粉砂糖5
  gを加えて蓋をして弱火で軽く蒸し煮する\footnote{étuver (エチュヴェ)。}。
\item
  野菜を鶏の周囲に盛り、\ul{火入れを仕上げる}\footnote{このガルニチュールは\protect\hyperlink{poulet-saute-a-la-fermiere}{鶏のソテー・フェルミエール}に添えるという前提がある。表面に焼き色を付けた鶏を、あらかじめ軽く蒸し煮しておいたこのガルニチュール・フェルミエールとともに陶製の鍋に入れて、さいの目に切ったハムを加え、蓋をしてオーブンに入れて鶏と野菜の火入れを仕上げることになる。}。
\end{itemize}

\atoaki{}

\hypertarget{garniture-a-la-financiere}{%
\subsubsection[ガルニチュール・フィナンシエール]{\texorpdfstring{ガルニチュール・フィナンシエール\footnote{徴税官風の意。名称について詳しくは\protect\hyperlink{sauce-financiere}{ソース・フィナンシエール}訳注参照のこと。なお、カレームは『19世紀フランス料理』で、多少の違いはあるが、これらの具材とソースを合わせることで「ラグー・フィナンシエール」と呼んでいる(t.3,
  pp.146-148)。これはつまり、ソース・フィナンシエールの訳注でも述べたように、もともとはガルニチュールとソースが別々のものではなく、一体化したものとして調理されていたことを示唆している。実際、このフィナンシエールという料理名の初出と思われる1755年ムノン『宮廷の晩餐』第2巻「肥鶏・フィナンシエール」および第3巻「鯉・フィナンシエール」は19世紀のものと内容、素材は違えどラグーとして扱われている。前者は肥鶏を掃除して中抜きした後、背から開いて骨を取り除き、大きなトリュフ4個とフォワグラとマッシュルーム、おろした豚背脂、卵黄、粒こしょう、バジルの粉末を混ぜて詰める。これを豚背脂のシートで包んで鍋に入れ、液体は注がずに熱い灰の上に鍋を置く。加熱していると肉汁などが出てくる。そこにビガラード(南フランスのビターオレンジ)の搾り汁と塩、こしょうで調味する(p.280)。後者は大きな鯉を掃除し、舌を残すようにしてエラは取り除く。片側の包丁を入れていない面の皮を剥がし、細かく刻んだ豚背脂を表面に刺す。鯉の中に詰めるラグーを作る。仔牛胸腺肉、トリュフ、フォワグラ、マッシュルームを鍋にバター1片とともに入れ、パセリ、シブール、にんにく、クローブ、バジルのブーケガルニを加える。鍋を火にかけて、小麦粉を振りかけ、シャンパーニュをグラス2杯注ぐ。塩こしょうで調味し、具材に火を通す。浮き脂を取り除き、冷めたら鯉の腹に詰め、ラグーが出てこないようしっかり鯉の腹を縫う。鯉の大きさにぴったりの魚用鍋に、ハムのスライスとたっぷりの仔牛腿肉のスライスを敷き、その上に鯉をのせる。豚背脂のシートで多い、玉ねぎのスライス、根菜の皮、パセリ、シブール、にんにく、クローブ、タイム、ローリエの葉、バルジのブーケガルニを入れる。中火にかけて汗をかかせるイメージで少し火を通し、それから上等のブイヨンとシャンパーニュを同量ずつ、鯉が液体に浸るまで注ぐ。塩こしょう。弱火にかける。火が通ったら鍋から鯉を取り出して水気をきる。縫った糸を取り除き、仔牛のグラスを塗って艶を出す。周囲にはお好みでエクルヴィスやまるごとのトリュフ、大きな鶏のとさか、鶏の胸肉、ペルドロー、こんがり焼いた鳩などをセンスよく配する。ソース・エスパニョルを添えて供する(pp.43-44)。その後、19世紀になるとヴィアール『帝国料理の本』1806年において「鳩・フィナンシエール」のレシピが掲載される。概要は、鳩6羽をバター、塩こしょう、レモン果汁を入れた鍋でさっと表面を色付けないように焼き固めたら、豚背脂で包んで鍋に入れ、ポワル(ここではソースの一種と考えていい)を注ぎ、柔らかく火を通す。提供直前に水気をきり、皿の周囲に鳩を配置する。その中央に雄鶏のとさかとロニョン、フォワグラ、トリュフのラグーを流し入れる
  (p.332)。残念ながらヴィアールの1806、1807年版は不完全なもののため、このラグーのレシピそのものは1820の第10版でようやく掲載に至る。概要は、マッシュルーム大24個、ボール形にしたトリュフ24個は辛口のマデイラ酒\(\frac{1}{2}\)瓶とともに鍋に入れ、唐辛子2本、トマト少々、仔牛のグラス1オンスを加えて火にかけてほとんどシロップ状になるまで煮詰める。それからソース・エスパニョルをレードル4杯、仔牛のブロンド(ソース)スプーン2杯を注いでよく混ぜる。沸騰させたら火の弱い場所に移して浮き脂を取り除き、煮詰めていく。このソースを布で漉し、きれいな鍋にマッシュルームとトリュフを移し入れて漉したソースを注ぐ。ここに雄鶏のとさかとロニョン24個ずつ、スプーンで成形したクネル24個、仔羊または仔牛の胸腺肉のスライス24枚を入れる(p.67)。この時点つまり1820年頃には、カレームが記した「ラグー・フィナンシエール」とほぼ同じ内容になっていることが注目されよう。カレームの「ラグー・フィナンシエール」は、トリュフ500
  gを円く成形し、マデイラ酒を加えて10分間弱火で蒸し煮する。ここにソース・フィナンシエールを注ぐ。ひと煮立ちさせたら、マデイラ酒から出たアクを取り除き、マッシュルーム12個、鶏のとさか12個(マデイラ酒少々を煮立たせて火を通しておく)、雄鶏のロニョン
  12個を加える。ひと煮立ちさせたらバター少々、鶏のクネル、フォラグラの1〜2
  cmのスライス、仔羊胸腺肉を加える。このラグーの半量はこれを添える料理上に盛り、周囲に白い立派な鶏のとさかとロニョンを配する。ソースが皿の上の料理をのせる台からはみ出ないようにしないと優美さが失なわれ、皿の縁飾りが乱れてしまうことに注意。ラグーの残りはソース入れで別添で供する(pp.146-147)。カレームはもうひとつ、「フォワグラのラグー・フィナンシエール」というレシピも残している(pp.147-148)。デュボワ、ベルナール共著『古典料理』(1868年)では「ガルニチュール・フィナンシエール」の項目は見られず、「サルピコン・フィナンシエ」
  (p.65)\ldots{}\ldots{}
  これは黒トリュフ、鶏胸肉、赤く漬けた舌肉、マッシュルームに火を通して小さなさいの目に切り、茶色いソース・フィナンシエールであえたもの、となっている。この他、「仔牛の耳・フィナンシエール
  (p.167)」「ほろほろ鳥のフィレ・フィナンシエール(p.179)」「ベカシーヌのグラタン・フィナンシエール(p.187)」「雉のクネル・フィナンシエール(p.190)」「温製パイ包み焼き・フィナンシエール(p.210)」「うずら・フィナンシエール(p.228)」「鳩・フィナンシエール(p.223)」といったレシピが収録されている。これらのレシピを見ると「ガルニチュール・フィナンシエール」を用いる指示になっているものがほとんどのため「ガルニチュール・フィナンシエール」の項が抜けているのは執筆あるいは何らかのミスによるものに過ぎないだろうと思われる。}}{ガルニチュール・フィナンシエール}}\label{garniture-a-la-financiere}}

\frsub{Garniture à la Financière}

\index{garniture@garniture!financiere@--- à la Financière}
\index{financier@financier/financère!garniture@garniture à la Financière}
\index{かるにちゆーる@ガルニチュール!ふいなんしえーる@---・フィナンシエール}
\index{ふいなんしえ@フィナンシエ/フィナンシエール!かるにちゆーる@ガルニチュール・フィナンシエール}

(牛、羊の塊肉あるいは鶏料理に添える)

\begin{itemize}
\item
  仔牛か鶏のファルスでつくった標準的なクネル20個。ファルスに仔牛を使うか鶏を使うかは、このガルニチュールを添える料理に合わせて決めること。
\item
  渦巻状の刻み模様を入れた小さなマッシュルーム150 g。
\item
  雄鶏のとさかとロニョン\footnote{rognonは通常「腎臓」を指すが、雄鶏の場合はrognon
    blanc(ロニョンブロン)=
    testicule(テスティキュル)すなわち精巣のこと。}100 g。
\item
  トリュフのスライス50 g。
\item
  皮を剥いて下茹でしたオリーブ12個。
\item
  \protect\hyperlink{sauce-financiere}{ソース・フィナンシエール}
\end{itemize}

\atoaki{}

\hypertarget{garniture-a-la-flamande}{%
\subsubsection[ガルニチュール・フランドル風]{\texorpdfstring{ガルニチュール・フランドル\footnote{flamand(e)
  (フラモン/フラモンド) \textless{} Flandre
  (フロンドル)フランドル地方 =
  現在のベルギー西部からフランス北部にかけての北海に面する地域。フランダース。ただし『フランダースの犬』はイギリスの児童文学なので、フランスおよびベルギーではあまり知られていない。}風}{ガルニチュール・フランドル風}}\label{garniture-a-la-flamande}}

\frsub{Garniture à la Flamande}

\index{garniture@garniture!flamande@--- à la Flamande}
\index{flamand@flamand(e)!garniture@garniture à la ---e}
\index{かるにちゆーる@ガルニチュール!ふらんとるふう@---・フランドル風}
\index{ふらんとるふう@フランドル風!かるにちゆーる@ガルニチュール・---}

\begin{itemize}
\item
  球形に成形した小さな\protect\hyperlink{chou-braise}{サヴォイキャベツのブレゼ}
  10個。
\item
  オリーブ形に成形し、コンソメで煮たにんじんと蕪、各10個ずつ。
\item
  \protect\hyperlink{pommes-de-terres-a-l-anglaise}{じゃがいものアラングレーズ}\footnote{à
    l'anglaise
    イギリス風、の意だが、必ずしもイギリス料理に由来するとは限らない。野菜の場合、アラングレーズとはすなわち「塩を加えた湯で茹でる」ことを意味するが、本書の該当個所にも、イギリスでは塩を加えない、とある。}
  小10個。
\item
  塩漬け豚バラ肉250
  gは10枚の長方形の板状に切り、キャベツとともにブレゼする。
\item
  輪切りにしたソシソン\footnote{熟成、乾燥させてつくる太いソーセージ。多くの場合、調理せず薄切りにして食べる。}
  10枚(150 g)。
\item
  【別添】塊肉をブレゼした煮汁\footnote{このガルニチュールは\protect\hyperlink{piece-de-boeuf-a-la-flammande}{牛塊肉・フランドル風}に添えるのを前提に書かれているため、ブレゼと特定出来るが、本書における\protect\hyperlink{les-poeles}{ポワレ}の手法でももちろん可能だろう。}。
\end{itemize}

\atoaki{}

\hypertarget{garniture-a-la-florentine}{%
\subsubsection[ガルニチュール・フィレンツェ風]{\texorpdfstring{ガルニチュール・フィレンツェ風\footnote{florentin(e)
  (フロロンタン/フロロンティーヌ)\textless{} Florence
  (フロロンス)フィレンツェのこと。}}{ガルニチュール・フィレンツェ風}}\label{garniture-a-la-florentine}}

\frsub{Garniture à la Florentine}

\index{garniture@garniture!florentine@--- à la Florentine}
\index{florentin@florentin(e)!garniture@garniture à la ---e}
\index{かるにちゆーる@ガルニチュール!ふいれんつえふう@---・フィレンツェ風}
\index{ふいれんつえふう@フィレンツェ風!かるにちゆーる@ガルニチュール・---}

(魚料理に添える場合)

\begin{itemize}
\item
  ほうれんそうの葉250 gは下茹でしてから、バターで蒸し煮する\footnote{étuver
    au beurre (エチュヴェオブール)}。
\item
  このほうれんそうを皿の底に敷き、その上に煮立たせないように茹で\footnote{pocher
    (ポシェ)。}
  て火を通した魚をのせ、\protect\hyperlink{sauce-mornay}{ソース・モルネー}を覆いかける。高温のオーブンに入れて焼き色を付ける。
\end{itemize}

(牛、羊の塊肉の料理に添える場合)

\begin{itemize}
\item
  \protect\hyperlink{sucric-d-epinards}{ほうれんそうのシュブリック}
  10個。
\item
  セモリナ粉を獣脂で加熱し、卵とおろしたチーズを混ぜ込んだアパレイユで円盤状につくった小さな\protect\hyperlink{croquettes}{クロケット}
  10個。
\item
  【別添】トマト風味を効かせた、薄い仕上りの\protect\hyperlink{sauce-demi-glace}{ソース・ドゥミグラス}。
\end{itemize}

\atoaki{}

\hypertarget{garniture-Florian}{%
\subsubsection[ガルニチュール・フローリアン]{\texorpdfstring{ガルニチュール・フローリアン\footnote{ヴェネツィア、サンマルコ広場にある18世紀からあるカフェ。}}{ガルニチュール・フローリアン}}\label{garniture-Florian}}

\frsub{Garniture Florian}

\index{garniture@garniture!florian@--- Florian}
\index{florian@Florian!garniture@garniture ---}
\index{かるにちゆーる@ガルニチュール!ふろーりあん@---・フローリアン}
\index{ふろーりあん@フローリアン!かるにちゆーる@ガルニチュール・---}

(乳呑仔羊\footnote{本書で
  agneauという場合には、いわゆるプレサレ(agneau de pré-salé
  アニョドプレサレ)はmouton(ムトン=羊の成獣)に準ずる扱いであり、それ以外は基本的にagneau
  de lait (アニョドレ)乳呑仔羊を指すことに留意。}の料理に添える\footnote{\protect\hyperlink{epaule-d-agneau-florian}{乳呑仔羊肩肉・フローリアン}に添えるのを前提としたガルニチュールであることに留意。。})

\begin{itemize}
\item
  大きめのレチュ3個は四つ割りにして外葉を取り除き、ブレゼする\footnote{\protect\hyperlink{laitue-braisee}{レチュのブレゼ}参照。}。
\item
  オリーブの大きさと形にしたにんじん20個は下茹でしてバターで色艶よく火を通す\footnote{glacer
    (グラセ)。これらの野菜の場合は下茹でして半ば火を通しておくことと、必要に応じて砂糖を加える場合があることに留意。}。
\item
  小玉ねぎ20個は下茹でしてバターで色艶よく炒める。
\item
  小さな\protect\hyperlink{pommes-de-terre-fondantes}{じゃがいものフォンダント}
  10個。
\item
  【別添】仔羊の肉汁\footnote{原文ではfondsとなっているが、前提となっている仕立て「乳呑仔羊肩肉・フローリアン」の場合はバターをかけながらローストするので、いわゆる「ジュ」と考えていい。}。
\end{itemize}

\atoaki{}

\hypertarget{garniture-a-la-Forestiere}{%
\subsubsection[ガルニチュール・フォレスティエール]{\texorpdfstring{ガルニチュール・フォレスティエール\footnote{forestier
  (フォレスティエ)形容詞は森林の、の意。名詞の場合は森林管理人。一般には「森番風」などと訳されることが多いようだ。}}{ガルニチュール・フォレスティエール}}\label{garniture-a-la-Forestiere}}

\frsub{Garniture à la Forestière}

\index{garniture@garniture!forestiere@--- à la Forestière}
\index{forestier@forestier/forestière!garniture@garniture à la Forestière}
\index{かるにちゆーる@ガルニチュール!ふおれすていえーる@---・フォレスティエール}
\index{ふおれすていえーる@フォレスティエール!かるにちゆーる@ガルニチュール・---}

(牛、羊の塊肉や鶏の料理に添える)

\begin{itemize}
\item
  モリーユ\footnote{茸の一種。和名アミガサタケ。生食出来ないので注意。}300
  gはバターと植物油同量ずつでソテーする。
\item
  脂身の少ない豚バラ肉の塩漬け125
  gは拍子木に切って下茹でし、バターでこんがりと焼く\footnote{rissoler
    (リソレ)。油脂を鍋に熱し、高温で素材の表面に焼き色を付けること。}。
\item
  じゃがいも300 gは大きめのさいの目に切ってバターでソテーする。
\item
  【別添】ブレゼの煮汁あるいはデグラセした液体を加えた\protect\hyperlink{sauce-duxelles}{ソース・デュクセル}
\end{itemize}

\atoaki{}

\hypertarget{garniture-frascati}{%
\subsubsection[ガルニチュール・フラスカーティ]{\texorpdfstring{ガルニチュール・フラスカーティ\footnote{フラスカーティは古代ローマの避暑地として有名だったところ。18世紀末にナポリ出身のアイスクリーム職人ガルキがパリのブルヴァール・モンマルトルにカフェ・フラスカティという店名のカジノ兼レストラン、パティスリを開き盛況だったという。とりわけアイスクリームが評判を呼んだらしい。その後経営者が何度か代わり、1857年に建物は取り壊された。このガルニチュールおよび「牛フィレ肉・フラスカーティ」がどちらに由来しているかは不明。フランス語風に発音するなら「フラスカティ」となる。}}{ガルニチュール・フラスカーティ}}\label{garniture-frascati}}

\frsub{Garniture Frascati}

\index{garniture@garniture!forestiere@--- Frascati}
\index{frascati@Frascati!garniture@garniture ---}
\index{かるにちゆーる@ガルニチュール!ふらすかーてい@---・フラスカーティ}
\index{ふらすかーてい@フラスカーティ!かるにちゆーる@ガルニチュール・---}

(牛、羊の塊肉の豪華な仕立てに添える\footnote{直訳すると「牛、羊の塊肉のルルヴェ用」(ルルヴェについては「\protect\hyperlink{releve}{第二版序文訳注}」参照)だが、本書では「\protect\hyperlink{filet-de-boeur-frascati}{牛フィレ肉・フラスカーティ}」くらいしか目ぼしいレシピがない。})

\begin{itemize}
\item
  厚さ1〜2 cm程度スライスした\footnote{escalope (エスカロップ)。}フォワグラ(出来るだけ生のものがいい)10枚に小麦粉をまぶし、バターでソテーする。
\item
  アスパラガスの穂先300 gは茹でてからバターであえる。
\item
  小さめの真っ白なマッシュルーム10個は軸を落とし、渦巻状に飾り模様を入れる。
\item
  大きめのオリーブくらいのサイズに成形したトリュフ10個はバターでかるく炒めて艶を出す。
\item
  トリュフ風味にした\protect\hyperlink{pommes-de-terre-duchesse}{じゃがいものデュシェス}をアパレイユにして細長く作ったクロワッサン10個は提供直前に溶き卵を塗り、オーブンで焼いて艶を出す。このクロワッサンを並べてでガルニチュールの外枠にする。
\item
  【別添】軽くとろみを付けた肉汁(ジュ)\footnote{「牛フィレ肉・フラスカーティ」の場合は\protect\hyperlink{les-poeles}{ポワレ}するので、適量のフォンを肉を加熱する際鍋の底に敷いたマティニョンに注ぎ、肉汁の風味をひき出してから布で漉し、でんぷんでとろみを付けることになる。}。
\end{itemize}

\atoaki{}

\hypertarget{garniture-a-la-gastronome}{%
\subsubsection[ガルニチュール・ガストロノーム]{\texorpdfstring{ガルニチュール・ガストロノーム\footnote{美食家、食通、の意。}}{ガルニチュール・ガストロノーム}}\label{garniture-a-la-gastronome}}

\frsub{Garniture à la Gastronome}

\index{garniture@garniture!gastronome@--- à la Gastronome}
\index{gastronome@gastronome!garniture@garniture à la ---}
\index{かるにちゆーる@ガルニチュール!かすとろのーむ@---・ガストロノーム}
\index{かすとろのーむ@ガストロノーム!かるにちゆーる@ガルニチュール・---}
\index{ひしよくかふう@美食家風 ⇒ ガストロノーム!かるにちゆーる@ガルニチュール・---}

(牛、羊の塊肉および鶏の料理に添える)

\begin{itemize}
\item
  大きめのマロン20個は、皮を剥いてコンソメで煮、小玉ねぎのようにバターで色艶よく炒める\footnote{glacer
    (グラセ)。}。
\item
  中位のサイズのトリュフ10個はシャンパーニュ風味に茹でる。
\item
  立派な雄鶏のロニョン\footnote{rognons de coq
    (ロニョンドコック)ここでは鶏の睾丸のこと。}
  20個はブロンド色の\protect\hyperlink{glace-de-viande}{グラスドヴィアンド}でコーティングする。
\item
  大きなモリーユ\footnote{morille
    茸の一種。和名アミガサタケ。生食不可なのでよく加熱する必要がある。}は縦二つ割りにし、バターでソテーする。
\item
  【別添】トリュフエッセンス入り\protect\hyperlink{sauce-demi-glace}{ソース・ドゥミグラス}。
\end{itemize}

\atoaki{}

\hypertarget{garniture-godard}{%
\subsubsection[ガルニチュール・ゴダール]{\texorpdfstring{ガルニチュール・ゴダール\footnote{18世紀の徴税官(つまりフィナンシエ)であり作家としても活動したクロード・ゴダール・ドクール
  Claude Godard d'Aucour (1716〜
  1795)の名を冠したものと考えられる。\protect\hyperlink{sauce-godard}{ソース・ゴダール}も参照のこと。本書のレシピだけを見ていると\protect\hyperlink{garniture-a-la-financiere}{ガルニチュール・フィナンシエール}と非常によく似ているけれどもソースの違うパターン、くらいにしか見えないかも知れぬが、このガルニチュールのほうが圧倒的に大掛かりで豪華な仕立てにすることを前提としており、ガルニチュール・ゴダールあるいはゴダールという名称の仕立てはフィナンシエールの完成形というべきか、究極の到達点のひとつだったのではないか?
  19世紀前半をとおして版を重ね、そのたびに増補されたヴィアールの本を版ごとに見ていくと、初版から
  1817年の第9版まではフィナンシエールのみ。1820年の第10版以降から「牛アロワイヨ・ゴダール」というレシピが登場する。長いレシピなので要点だけ見ると、約7〜8
  kgの牛アロワイヨ(日本語では「腰肉」すなわちフィレを含むサーロインからランプ、イチボにかけての部分)を四角形に切り整えて骨は取り除き、紐で縛ってからマデイラ酒を加えて6時間ブレゼする。肉を取り出したら煮汁を漉して、卵白でクラリフィエし、さら布で漉して煮詰める。その煮汁の半分にコンソメを足して、肉を鍋に戻し入れてさらに2時間弱火で煮込む。肉を皿に盛り付け、周囲に若鳩4羽、拍子木に切った豚背脂やトリュフ、赤く漬けた舌肉を表面に刺して装飾した仔羊胸腺肉4枚、スプーンで成形した大きなクネル8個、大きなエクルヴィス8尾、鶏またはその他の揚げものを刺した飾り串8本をアロワイヨの上から刺す。ドゥミグラス半量と合わせたラグー・フィナンシエールをかける。強火のオーブンで照りを付け、熱々を供する(p.101)。これを見るかぎり、本質的にはフィナンシエールの変形もしくは豪華版と考えていいだろう。カレーム『19世紀フランス料理』では「牛アロワイヨのブレゼ・ゴダール」に2種のレシピが記述されており、ひとつは上記と似たアロワイヨ全体をブレゼしたもの。もうひとつはアロワイヨの一部をブレゼし、残りはローストにした仕立てになっている。いずれにしても、非常に豪華な仕立てであり、ものすごいコストがかかるため、きわめて格式の高い荘厳な宴席でしか出来ないだろうが「食卓外交に携わる料理人はこうした料理の知識を大切にして、これらの料理を供すべきだ」と述べている(t.3,
  p.327)。デュボワ、ベルナール共著『古典料理』に至るとむしろゴダールという仕立ては簡略される方向に向かい、本書と同様に「ルルヴェ用ガルニチュール・ゴダール」として記述される。「このガルニチュールはトリュフで装飾を施した大きなクネル、表面に装飾をしてソースをかけてオーブンで焼き色を付けた仔牛胸腺肉、トリュフ、マッシュルームで構成される。これらを各まとまりごとに料理の周囲に添える。クネルとマッシュルームには軽くソースをかけてやり、トリュフと仔牛胸腺肉には艶を出させてやる(グラセ)こと」(p.94)とある。こうしたことから、フィナンシエールおよびその発展形としての仕立てであるゴダールが19世紀後半にむかってだんだんと広まっていき、盛んに作られるようになったが、ゴダールについてはそのコストゆえに簡素化していく傾向にあった。いずれにしても両者ともにきわめて19世紀的なソースとガルニチュールの組み合わせ、あるいは料理の仕立てといえよう。}}{ガルニチュール・ゴダール}}\label{garniture-godard}}

\frsub{Garniture Godard}

\index{garniture@garniture!godard@--- Godard}
\index{godard@Godard!garniture@garniture ---}
\index{かるにちゆーる@ガルニチュール!こたーる@---・ゴダール}
\index{こたーる@ゴダール!かるにちゆーる@ガルニチュール・---}

(牛、羊、鶏の大掛かりで豪華な仕立てに添える)

\begin{itemize}
\item
  マッシュルームとトリュフのみじん切りを加えた、\protect\hyperlink{farce-a}{バター入りのファルス}をスプーンで成形したクネル10個。
\item
  トリュフと赤く漬けた舌肉で装飾を施した大きな楕円形のクネル4個。
\item
  小さめのマッシュルーム10個は軸を除き、螺旋状に切れ込み模様を付ける。
\item
  雄鶏のとさかとロニョン125 g。
\item
  上等の仔羊胸腺肉200
  gは高温のオーブンで焼き色を付ける。または仔牛胸腺肉の喉側を高温のオーブンで焼き色を付け、スライスする
\item
  オリーブ形に成形したトリュフ10個。
\item
  \protect\hyperlink{sauce-godard}{ソース・ゴダール}
\end{itemize}

\atoaki{}

\hypertarget{garniture-grand-duc}{%
\subsubsection[ガルニチュール・グランデュック]{\texorpdfstring{ガルニチュール・グランデュック\footnote{grand-duc
  (グロンデュック)大公およびロシアの皇太子の意。
  Prince(プランス)大公とほぼ同義だが使われる場面などで違いがある。料理においてはアスパラガスの穂先とトリュフを用いた料理に付されることが多い。}}{ガルニチュール・グランデュック}}\label{garniture-grand-duc}}

\frsub{Garniture Grand-Duc}

\index{garniture@garniture!grand-duc@--- Grand-Duc}
\index{grand-duc@grand-duc!garniture@garniture ---}
\index{かるにちゆーる@ガルニチュール!くらんていゆつく@---・グランデュック}
\index{くらんていゆすく@グランデュック!かるにちゆーる@ガルニチュール・---}
\index{たいこう@大公(風)⇒グランデュック!かるにちゆーる@ガルニチュール・---}

(魚料理に添える)

\begin{itemize}
\item
  アスパラガスの穂先200 gは下茹でしてバターであえる。
\item
  殻をむいたエクルヴィスの尾の身10個。
\item
  大きなトリュフのスライス10枚。
\item
  \protect\hyperlink{sauce-mornay}{ソース・モルネー}。
\end{itemize}

\atoaki{}

\hypertarget{garniture-a-la-grecque}{%
\subsubsection[ガルニチュール・ギリシア風]{\texorpdfstring{ガルニチュール・ギリシア風\footnote{grec
  / grecque
  は「ギリシアの」の意。ここではあえて「ギリシア風」訳したが、ギリシアに起源あるいは縁のないと思われる調理が少なくないので注意。}}{ガルニチュール・ギリシア風}}\label{garniture-a-la-grecque}}

\frsub{Garniture à la Grecque}

\index{garniture@garniture!grecque@--- à la  Grecque}
\index{grec@grec/grecque!garniture@garniture à la grecque}
\index{かるにちゆーる@ガルニチュール!くれつく@---・グレック}
\index{くれいつく@グレック!かるにちゆーる@ガルニチュール・---}
\index{きりしあふう@ギリシャ風⇒グレック!かるにちゆーる@ガルニチュール・---}

(乳呑仔羊および鶏料理に添える)

\begin{itemize}
\item
  \protect\hyperlink{riz-a-la-grecque}{ギリシア風ライス}\footnote{ピラフの一種だが、実際のところまったくギリシア風ではないことに注意。}
  250 g(\protect\hyperlink{riz}{野菜料理「米」の項}参照)。
\item
  【別添】\protect\hyperlink{sauce-tomate}{トマトソース}。
\end{itemize}

\atoaki{}

\hypertarget{garniture-henri-iv}{%
\subsubsection{ガルニチュール・アンリ4世亭風}\label{garniture-henri-iv}}

\frsub{Garniture Henri IV }

\index{garniture@garniture!henri iV@--- Henri IV}
\index{henri iv@Henri IV!garniture@garniture ---}
\index{かるにちゆーる@ガルニチュール!あんりよんせいていふう@---・アンリ4世亭風}
\index{あんりよんせいていふう@アンリ4世亭風!かるにちゆーる@ガルニチュール・---}

(ノワゼットやトゥルヌドに添える)

\begin{itemize}
\item
  肉に合わせて中くらいから小さめのアーティチョークの芯\footnote{比較的小ぶりであっても完熟のアーティチョーク(開花がやや近い状態のもの)は下茹で後に花萼をすべて取り除く。この状態を
    fond d'artichaut (フォンダルティショー)またはcul d'artichaut
    (キュダルティショー)と呼ぶ。とりわけ大きなアーティチョークは花萼部が完全に固いことが多いために、上半分よりやや下で切り離して、繊毛を取り除いてから下茹でする。花萼を全て剥いて皿のような形状の基底部のみを取り出す。丸い皿のような底面になるので、そこに詰め物をすることが多い。小さく比較的若どりのアーティチョークは花萼を全部は取り除かず、周囲の固いところを2周くらい剥いて使う。これを
    coeur d'artichaut
    (クールダルティショー)と呼ぶ。サイズによっては縦半分または四つ割りにして繊毛を取り除いてから下茹でする。四つ割りの場合はquartiers
    d'artichaut
    (カルティエダルティショー)と呼ぶ。若どりのアーティチョークの内側の花萼は柔らかく火が通り、とても美味。また、生食できるくらい若どりのアーティチョークはpoivrade(ポワヴラード)とも呼ばれる。ただし若どりであればそれだけ、アーティチョーク特有の風味は弱い。}に、溶かした\protect\hyperlink{glace-de-viande}{グラスドヴィアンド}の中に入れて転がしてグラスをコーティングさせた小さな\protect\hyperlink{pommes-de-terre-noisette}{じゃがいものノワゼット}を詰める。
\item
  【別添】\protect\hyperlink{sauce-bearnaise}{ソース・ベアルネーズ}。
\end{itemize}

\atoaki{}

\hypertarget{garniture-a-la-hongroise}{%
\subsubsection[ガルニチュール・ハンガリー風]{\texorpdfstring{ガルニチュール・ハンガリー風\footnote{この名称の根拠となっているのはパプリカを使用していることのみ。\protect\hyperlink{sauce-hongroise}{ハンガリー風ソース}訳注も参照。}}{ガルニチュール・ハンガリー風}}\label{garniture-a-la-hongroise}}

\frsub{Garniture à la Hongroise}

\index{garniture@garniture!hongroise@--- à la Hongroise}
\index{hongrois@hongrois(e)!garniture@garniture à la ---e}
\index{かるにちゆーる@ガルニチュール!はんかりーふう@---・ハンガリー風}
\index{はんかりーふう@ハンガリー風!かるにちゆーる@ガルニチュール・---}

(いろいろな料理に添えられる)

\begin{itemize}
\item
  小房に分けたカリフラワーをクリームであえて、いくつかの小さな容器に詰め、バターを塗ったグラタン皿に裏返して並べ、上からおろしたチーズを振りかけ、みじん切りにしたハムを加えたパプリカ風味の\protect\hyperlink{sauce-mornay}{ソース・モルネー}で覆い、高温のオーブンに入れてこんがりと焼く。
\item
  【別添】パプリカで風味付けした軽いソースを添える。
\end{itemize}

\atoaki{}

\hypertarget{garniture-a-l-italienne}{%
\subsubsection{ガルニチュール・イタリア風}\label{garniture-a-l-italienne}}

\frsub{Garniture à l'Italienne}

\index{garniture@garniture!italienne@--- à l'Italienne}
\index{italien@italien(ne)!garniture@garniture à l' ---ne}
\index{かるにちゆーる@ガルニチュール!いたりあふう@---・イタリア風}
\index{いたりあふう@イタリア風!かるにちゆーる@ガルニチュール・---}

(牛、羊の塊肉および鶏料理に添える)

\begin{itemize}
\item
  小さなアーティチョークを縦4つに切って、\protect\hyperlink{quartiers-d-artichauts-a-l-italienne}{イタリア風}に調理する(野菜料理、\protect\hyperlink{artichauts}{アーティチョーク}参照)20個。
\item
  茹でたマカロニにたっぷりチーズをあえて円盤型にしたクロケット10個。
\item
  【別添】\protect\hyperlink{sauce-italienne}{イタリア風ソース}。
\end{itemize}

\atoaki{}

\hypertarget{garniture-a-l-indienne}{%
\subsubsection{ガルニチュール・インド風}\label{garniture-a-l-indienne}}

\frsub{Garniture à l'Indienne}

\index{garniture@garniture!indienne@--- à l'Indienne}
\index{indien@indien(ne)!garniture@garniture à l' ---ne}
\index{かるにちゆーる@ガルニチュール!いんとふう@---・インド風}
\index{いんとふう@インド風!かるにちゆーる@ガルニチュール・---}

(魚、牛、羊の塊肉や鶏料理に添える)

\begin{itemize}
\item
  \protect\hyperlink{riz-a-l-indienne}{インド風に調理}したパトナ米\footnote{パトナはコメの品種名。いわゆる「長粒種」だがバスマティのような香り米ではない。}
  125 g(野菜料理「\protect\hyperlink{riz}{米}」参照)。
\item
  【別添】\protect\hyperlink{sauce-a-l-indienne}{インド風ソース}。
\end{itemize}

\atoaki{}

\hypertarget{garniture-a-la-japonaise}{%
\subsubsection[ガルニチュール・日本風]{\texorpdfstring{ガルニチュール・日本風\footnote{このガルニチュールが「日本風」であるのは、ちょろぎを用いているから。中国原産のシソ科の根菜で、現代日本では慶事などの際に用いられる程度だが、どういうわけか日本原産と誤解されたまま19世紀にフランスで栽培されるようになり、以来、日本風の名を付けた料理にはほとんど必ずといっていい程、ちょろぎが用いられる。}}{ガルニチュール・日本風}}\label{garniture-a-la-japonaise}}

\frsub{Garniture à la Japonaise}

\index{garniture@garniture!japonaise@--- à la Japonaise}
\index{japonais@japonais(e)!garniture@garniture à la ---e}
\index{かるにちゆーる@ガルニチュール!にほんふう@---・日本風}
\index{にほんふう@日本風!かるにちゆーる@ガルニチュール・---}

(牛、羊の塊肉の料理に添える)

\begin{itemize}
\item
  ちょろぎ 625
  gは\protect\hyperlink{veloute}{ヴルテ}であえ、ブリオシュ型でつくりオーブンでこんがり焼いた\protect\hyperlink{croustades}{クルスタード}に詰める。
\item
  米の\protect\hyperlink{croquettes}{クロケット} 10個\footnote{初版〜第三版は「\protect\hyperlink{croquettes-de-pommes-de-terre}{じゃがいものクロケット}
    10個」となっている。「じゃがいものクロケット」は初版からレシピが掲載されているが、初版〜第四版すなわち現行版において「\href{Ecroquettes-de-riz}{米のクロケット}」のレシピはアントルメすなわちデザートとして砂糖を加えて甘くつくるレシピしか掲載されておらず、そのとおりにすべきかは一考の余地がある。また、「\protect\hyperlink{croquettes-a-l-indienne}{インド風クロケット}」も米を使用している。}。
\item
  【別添】肉の煮汁を澄ませたソース。
\end{itemize}

\atoaki{}

\hypertarget{garniture-a-la-jardiniere}{%
\subsubsection[ガルニチュール・ジャルディニエール]{\texorpdfstring{ガルニチュール・ジャルディニエール\footnote{jardinier/jardinière
  (ジャルディニエ、ジャルディニエール)には名詞で「庭師」の意味もあるが、ここでは
  jardin potager (ジャルダンポタジェ)すなわち野菜畑、菜園、の意。}}{ガルニチュール・ジャルディニエール}}\label{garniture-a-la-jardiniere}}

\frsub{Garniture à la Jardinière}

\index{garniture@garniture!jardiniere@--- à la Jardinière}
\index{jardinier@jardinier/jardinière!garniture@garniture à la Jardinière}
\index{かるにちゆーる@ガルニチュール!しやるていにえーる@---・ジャルディニエール}
\index{しやるていにえーる@ジャルディニエール!かるにちゆーる@ガルニチュール・---}

(牛、羊の塊肉の料理に添える)

\begin{itemize}
\item
  にんじん125 gと蕪125
  gは、プレーンな、あるいは刻み模様の入ったスプーンでくり抜く。あるいは円柱形にしてもいい。これをコンソメで煮て、最後にバターで色艶よく炒める。
\item
  プチポワ125 g。小さなフラジョレ125 g。アリコヴェール125
  gは小さな菱形に切る。これらを別々にバターであえる\footnote{しっかり加熱調理してからバターであえること。}。
\item
  茹であげたばかりのカリフラワーの小房10個。
\item
  以上の構成要素を肉の周囲に、別々にはっきりとニュアンスが代わるように配置する。カリフラワーの小房は\protect\hyperlink{sauce-hollandaise}{オランデーズソース}小さじ1杯程度をそれぞれに塗ってやる。
\item
  【別添】澄んだジュ(肉汁)。
\end{itemize}

\atoaki{}

\hypertarget{garniture-joinville}{%
\subsubsection[ガルニチュール・ジョワンヴィル]{\texorpdfstring{ガルニチュール・ジョワンヴィル\footnote{フランソワ・ドルレアン・ジョワンヴィル海軍大将(1818〜
  1900)のこと。\protect\hyperlink{sauce-joinville}{ソース・ジョワンヴィル}も参照。}}{ガルニチュール・ジョワンヴィル}}\label{garniture-joinville}}

\frsub{Garniture Joinville}

\index{garniture@garniture!joinville@--- Joinville}
\index{joinville@Joinille!garniture@garniture ---}
\index{かるにちゆーる@ガルニチュール!しよわんういる@---・ジョワンヴィル}
\index{しよわんういる@ジョワンヴィル!かるにちゆーる@ガルニチュール・---}

(魚料理に添える)

\begin{itemize}
\item
  以下のものを5 mm角くらいの小さな角切り\footnote{salpicon
    (サルピコン)。}か短かい拍子木状\footnote{julienne courte
    (ジュリエーヌクルト)。}に刻む\ldots{}\ldots{}茹でたマッシュルーム125
  g、トリュフ50 g。これにクルヴェットの尾の身125
  gを加え、スプーン数杯の\protect\hyperlink{sauce-joinville}{ソース・ジョワンヴィル}であえる。
\item
  追加として\ldots{}\ldots{}トリュフのスライス10枚。白くて大きなマッシュルームに殻をむいたクルヴェット8尾を刺す。
\item
  ソース・ジョワンヴィル。
\end{itemize}

\atoaki{}

\hypertarget{garniture-judic}{%
\subsubsection[ガルニチュール・ジュディック]{\texorpdfstring{ガルニチュール・ジュディック\footnote{女優アンナ・ジュディック(1849〜1911)の名を冠したもの。}}{ガルニチュール・ジュディック}}\label{garniture-judic}}

\frsub{Garniture Judic}

\index{garniture@garniture!judic@--- Judic}
\index{judic@Judic!garniture@garniture ---}
\index{かるにちゆーる@ガルニチュール!しゆていつく@---・ジュディック}
\index{しゆていつく@ジュディック!かるにちゆーる@ガルニチュール・---}

(ノワゼットやトゥルヌド、鶏料理に添える)

\begin{itemize}
\item
  小さめのレチュを縦半割りにしてきれいに掃除し、ブレゼしたもの10個。
\item
  大きな雄鶏のロニョン10個。
\item
  トリュスのスライス10枚。
\item
  【別添】ごく上等の\protect\hyperlink{sauce-demi-glace}{ソース・ドゥミグラス}。
\end{itemize}

\atoaki{}

\hypertarget{garniture-languedocienne}{%
\subsubsection[ガルニチュール・ラングドック風]{\texorpdfstring{ガルニチュール・ラングドック風\footnote{languedocien(ne)
  (ラングドスィヤン、ラングドスィエーヌ)\textless{} Languedoc
  ラングドック地方。フランス南西部の地方名。もとは「オック語」langue
  d'oc から。中世プロヴァンス語と考えていい。オック oc
  とは古語で、現代フランス語の oui
  に相当する肯定の語。ラテン語の格変化の消失が比較的遅かった。バスク地方を除く(バスク語は別言語として扱われる)ロワール河以南で話された諸語の総称。これに対し、オイル語
  langue d'oil (ラングドイル)があり、oil
  が肯定の語であるというところが代表的な違い。主としてロワール河以北で話された諸語の総称。現代フランス語は後者の系統にあたるが、語彙の面などではラングドックの影響を大きく受けている。}}{ガルニチュール・ラングドック風}}\label{garniture-languedocienne}}

\frsub{Garniture Languedocienne}

\index{garniture@garniture!languedocienne@--- Languedocienne}
\index{languedocien@languedocien(ne)!garniture@garniture ---}
\index{かるにちゆーる@ガルニチュール!らんくとつくふう@---・ラングドック風}
\index{らんくとつくふう@ラングドック風!かるにちゆーる@ガルニチュール・---}

(牛、羊の塊肉、鶏料理に添える)

\begin{itemize}
\item
  なすは1 cm厚の輪切りを10枚用意し、小麦粉をまぶして油で揚げる。
\item
  セープ\footnote{cèpe
    (セープ)、茸の一種、和名ヤマドリタケ。イタリアのポルチーニと同種だが、フランス産、イタリア産で風味や調理特性が異なる。また日本に多いのはヤマドリタケモドキという種で、食用できるが風味などはまったく及ばないという。類似のものにウツロイグチ、ドクヤマドリという毒茸があるので注意が必要。}
  400 gはスライスして植物油でソテーする。
\item
  トマト400
  gは河を剥いて圧しつぶし、粗く刻んで、にんにく1片を加えて油でソテーする。
\item
  パセリのみじん切り。
\item
  【別添】\protect\hyperlink{jus-de-veau-lie}{とろみを付けたジュ}。
\end{itemize}

\atoaki{}

\hypertarget{garniture-lorette}{%
\subsubsection[ガルニチュール・ロレット]{\texorpdfstring{ガルニチュール・ロレット\footnote{7月王政期(1830〜1848)ロレットと呼ばれる娼婦たちがいた。ノートルダム・ド・ロレット(現在のパリ9区にある教会)に因んでいるという。かつて「洗濯娘」を意味する娼婦grisettes(グリゼット)と呼ばれた社会階層のやや下に位置する者が多かったため、高等娼婦を意味する
  courtisane(クルティザーヌ)と区別されることもある。グリゼットおよびロレットの例としては、バルザックの小説『幻滅』や『高等娼婦の燦めきと惨め(浮かれ女盛衰記)』にこの種の娼婦が主要人物として登場する。また、ロレットたちは、第二帝政期にはcocotte(ココット)と呼ばれる高等娼婦にとって代わられた(たとえばゾラの『ナナ』やデュマ・フィス原作ヴェルディ作曲のオペラ『椿姫』の主人公などがこれに相当する)。また、高等娼婦の中には、18世紀末のデュバリー夫人(\protect\hyperlink{garniture-dubarry}{ガルニチュール・デュバリー}
  訳注参照)や、女優サラ・ベルナールのように社会的に成功した例も少なくはない。貴族やブルジョワがこうした高等娼婦との「社交の場」として19世紀にはレストランや高級カフェを利用することが多かった。これをdemi-monde(ドゥミモンド、半社交界)と呼び、その華やかな主役たる高等娼婦たちはdemi-mondaine(ドゥミモンデーヌ)と呼ばれた。このため、19世紀〜20世紀初頭にかけて創案された料理のなかには高等娼婦の名を冠したものも少なくない。}}{ガルニチュール・ロレット}}\label{garniture-lorette}}

\frsub{Garniture Lorette}

\index{garniture@garniture!lorette@--- Lorette}
\index{lorette@lorette!garniture@garniture ---}
\index{かるにちゆーる@ガルニチュール!ろれつと@---・ロレット}
\index{ろれつと@ロレット!かるにちゆーる@ガルニチュール・---}

(ノワゼット、トゥルヌドに添える)

\begin{itemize}
\item
  小さくつくった\protect\hyperlink{croquettes-de-volaille}{鶏のクロケット}
  10個。
\item
  アスパラガスの穂先またはプチポワをバターであえる。
\item
  トリュフのスライス。
\item
  【別添】\protect\hyperlink{jus-de-veau-lie}{とろみを付けたジュ}
\end{itemize}

\atoaki{}

\hypertarget{garniture-louisiane}{%
\subsubsection[ガルニチュール・ルイジアナ風]{\texorpdfstring{ガルニチュール・ルイジアナ風\footnote{アメリカ合衆国のルイジアナ州のこと。とうもろこしは伝統的なフランス料理ではあまり好まれる食材ではないが、それを用いているところからの命名だろう。}}{ガルニチュール・ルイジアナ風}}\label{garniture-louisiane}}

\frsub{Garniture Louisiane}

\index{garniture@garniture!louisiane@--- Louisiane}
\index{louisiane@Louisiane!garniture@garniture ---}
\index{かるにちゆーる@ガルニチュール!るいしあなふう@---・ルイジアナ風}
\index{るいしあなふう@ルイジアナ風!かるにちゆーる@ガルニチュール・---}

(家禽\footnote{原文は volaille
  (ヴォライユ)。本来は家禽全般つまり鶏、鴨(あひる)、七面鳥、鳩なども含まれるが、本書ではほとんどの場合、鶏(大きさや肥育方法により名称が多数ある)を意味するため、そのように訳しているが、ここでは七面鳥もしくはがちょうを前提としていると考えるのが妥当であり、調理方法も、ソースの指示からブレゼであると解釈される。なお、七面鳥は16世紀にアメリカ大陸からフランスにもたらされ、17世紀には流行の食材となった。当初は
  poulet d'Inde
  (プレダンド、インドの鶏の意)などと呼ばれていたが、やがてdinde
  (ダンド、七面鳥のメス)、dindon(ダンドン、七面鳥のオス)、dindonneau
  (ダンドノー、七面鳥の雛、若いオスの七面鳥)のように用語が定着していった。}の料理に添える)

\begin{itemize}
\item
  とうもろこし500 gはクリームであえる。
\item
  \protect\hyperlink{riz-au-gras}{リオグラ}\footnote{riz au gras
    直訳すると脂気の多い米、だが、実際には下茹でした米を脂気がやや残ったままのブイヨンで煮込んだもの。}をダリオル型\footnote{小さな円筒形の型。}に詰めて形づくった小さなタンバル\footnote{timbale
    直径に対してやや背の低い円筒形にする仕立て。大きなものはタンバル型
    moule à timbale
    (ムーラタンバール)に料理を詰めるが、ここでは小さなものを10個つくるので、ダリオル型を用いている。なお、
    timbale
    は元来「小太鼓、スネアドラム」の意。ただし、料理においては上述のような仕立てとともに、野菜料理用のやや深い皿のこともタンバルと呼ぶ。}
  10個。
\item
  バナナの輪切り20個は油で揚げる。
\item
  【別添】家禽を調理した際のフォンを煮詰めて仕上げたソース。
\end{itemize}

\atoaki{}

\hypertarget{garniture-lucullus}{%
\subsubsection[ガルニチュール・ルクッルス]{\texorpdfstring{ガルニチュール・ルクッルス\footnote{ルキウス・ルキニウス・ルクッルス(前118〜前56)。共和政ローマの軍人、政治家で美食家として知られた。}}{ガルニチュール・ルクッルス}}\label{garniture-lucullus}}

\frsub{Garniture Lucullus}

\index{garniture@garniture!lucullus@--- Lucullus}
\index{lucculus@Lucullus!garniture@garniture ---}
\index{かるにちゆーる@ガルニチュール!るくつるす@---・ルクッルス}
\index{るくつるす@ルクッルス!かるにちゆーる@ガルニチュール・---}

(牛、羊の塊肉や鶏料理に添える)

\begin{enumerate}
\def\labelenumi{\arabic{enumi}.}
\item
  平均60
  gのトリュフ10個は\protect\hyperlink{mirepoix}{ミルポワ}にマデイラ酒風味で火を通す。\ul{中はくり抜き、蓋\\になる部分を取り置く}。雄鶏のロニョンをトリュフ1つにつき2つ、バターを加えた\protect\hyperlink{glace-de-viande}{グラスドヴィアンド}をまぶしてコーティングして詰める。取り置いておいた蓋をして、鶏のファルスでつくった小さなリボンで蓋とトリュフをつなぐ。低温のオーブンでファルスに火を通す。
\item
  トリュフをくり抜いた中身を混ぜ込んだ\protect\hyperlink{farce-c}{鶏の滑らかなファルス}をスプーンで成形して10個のクネルを用意する。混ぜ込むトリュフはあらかじめすり潰して裏漉しすること。
\end{enumerate}

\begin{itemize}
\item
  カールさせた大きな鶏のとさか10個。
\item
  【別添】トリュフエッセンス入り\protect\hyperlink{sauce-demi-glace}{ソース・ドゥミグラス}。
\end{itemize}

\atoaki{}

\hypertarget{garniture-macedoine}{%
\subsubsection[ガルニチュール・マセドワーヌ]{\texorpdfstring{ガルニチュール・マセドワーヌ\footnote{この語の初出は匿名で出版された\href{http://gallica.bnf.fr/ark:/12148/bpt6k1511832p}{『ガスコーニュ料理の本』}(1740年)であり、Macedoine
  à la Paysanne
  というレシピが掲載されている。ただしこの本はいわゆる「偽書」あるいは「奇書」に類するもので、ガスコーニュ地方の料理などひとつも掲載されていない。パリで印刷、出版されたにもかかわらず、アムステルダム(18世紀にアムステルダム版といえば「海賊版」の代名詞だった)出版として匿名で上梓された。匿名なので著者名はないのだが、献辞に「ドンブ大公閣下へ」とあり、実際の著者はまさにそのドンブ大公であるルイ・オーギュスト・ド・ブルボンそのひとであったと考えられている(17世紀にルイ14世の子で同名の者がいるが混同しないよう注意)。狩猟と料理が趣味であったという。一般的な言い回しとして「料理上手」の意味でcuisinier
  gascon(キュイジニエガスコン)「ガスコーニュの料理人」ということがあり、しかも内容は料理書として見た場合、どこまで真面目でどの程度冗談めいたものなのか判断に苦しむところがある。要は「殿様」の道楽本ともいえる。一例として「徴税官風鶏の袋詰め」\emph{Poulette
  en musette à la Financière}
  というほとんど冗談としか思えぬ、けれども歴史的に非常に興味深いレシピがある。茹でたプレ・レーヌ(若鶏と肥鶏の中間くらいの大きさ)を羊の膀胱にサルピコンとともに詰めて、息を吹き込んで膀胱を膨らませて口を縛る。1皿に3袋のせるべし。というもの(pp.128-129)。20世紀にフェルナン・ポワンが「鶏の膀胱包み」という歴史に残る名作料理をスペシャリテのひとつとしていたが、おそらくは膀胱に鶏を入れて膨らませるというプレゼンテーションについてはこれが最初の例であろう(この例では調理において膀胱を用いる必然性はまったくなく、たんに見せ方だけの問題だが)。また、「徴税官風」すなわちフィナンシエールという語が、後代のラグー・フィナンシエールあるいはフィナンシエール仕立てとはまったく関係ない文脈で使われている点も興味深い。要は税として徴収した財貨を詰め込んだ袋のイメージを演出するための小道具に過ぎないということ。マセドワーヌについては、えんどう豆とそら豆とアリコヴェール(これらはえんどう豆とおなじ大きさに刻む)、細切りにしたにんじんをバターを入れた鍋で弱火にかけ汗をかかせるようなイメージで蒸し煮し、時々混ぜながら、火が通ったら味付けしてソースを少量のソースとともに供するというもの(pp.139-140)。マセドワーヌの語は料理とは関係なく、同じ18世紀のラクロの小説『危険な関係』において「カードを混ぜる行為」という意味で用いられており、よく混ぜる、という意味において間違いはない。なお、一般にマセドワーヌというと\ul{小さなさいの目に切った}蕪やにんじん、アリコヴェール、プチポワなどを混ぜたものであり、日本のマセドアンサラダの原型にもなったが、料理用語の原義としては、必ずしも「さいの目」に刻む必要はない。このガルニチュール・マセドワーヌも『料理の手引き』における「ガルニチュール・ジャルディニエール」の指示どおりに野菜を下ごしらえして混ぜても成立するだろうし、切り方を揃えるという方法もあるだろう。}}{ガルニチュール・マセドワーヌ}}\label{garniture-macedoine}}

\frsub{Garniture Macédoine}

\index{garniture@garniture!macedoine@--- Macédoine}
\index{macedoine@macédoine!garniture@garniture ---}
\index{かるにちゆーる@ガルニチュール!ませとわーぬ@---・マセドワーヌ}
\index{ませとわーぬ@マセドワーヌ!かるにちゆーる@ガルニチュール・---}

(牛、羊の塊肉の料理に添える)

\begin{itemize}
\tightlist
\item
  このガルニチュールは\protect\hyperlink{garniture-jardiniere}{ジャルディニエール}とまったくおなじ構成要素だが、すべての材料を混ぜてあえてしまう点が異なる。これを野菜料理用の深皿に別途盛り付けるか、アーティチョークの基底部に詰めるか、もしくは皿の中心にドーム状に盛り、その周囲に肉料理を並べるようにして盛り付ける。
\end{itemize}

\atoaki{}

\hypertarget{garniture-madeleine}{%
\subsubsection[ガルニチュール・マドレーヌ]{\texorpdfstring{ガルニチュール・マドレーヌ\footnote{マドレーヌといえば誰もが焼き菓子を想い浮かべるだろうが、このガルニチュールにはそれと類似する、あるいは想起させる要素がまったくない。マドレーヌは聖マドレーヌに由来し、教会の名称として珍しくないばかりか、女性の名前としてもごく一般的なものだ。他の料理書に同名のガルニチュールが見あたらないこと、本書初版から掲載されているものであることを考えると、あえていうなら、オクターヴ・ミルボー作の戯曲『\href{http://gallica.bnf.fr/ark:/12148/bpt6k203007v}{酷い羊飼いども}』初演(1897年)の際にサラ・ベルナールが演じて話題となった主人公の名がマドレーヌであることくらいか。資本家に虐げられた労働者階級が反乱を起こして失敗するという悲劇で、テーマとしてはゾラの『ジェルミナル』に近い。もしこのガルニチュールがミルボーの戯曲の登場人物を示唆しているなら、その料理を食べる側すなわち富裕層、資本家と、その料理を作る側の労働者との対立の図式が透けて見える、いわば強烈な風刺とも考えられるだろう。もっとも、エスコフィエは、その真意まではわからぬが常に資本家、富裕層の側に寄り添った料理人であったのもまた事実だ。オクターヴ・ミルボーについては、自然主義文学の作家としてスタートしたとはいえ、1964年にルイス・ブニュエルがジャンヌ・モロー主演で映画化した小説『小間使いの日記』で知られるように、いわゆる自然主義文学の枠にとどまることなく、独自の文学活動を展開した。画家ファン・ゴッホをモデルとした小説『天空にて』(新聞連載1892〜1893年)や、発表後にバルザックの遺族の抗議により作品の撤回を余儀なくされた『バルザックの死』などにより、フランス文学史においては世紀末文学の作家として位置付けられている。}}{ガルニチュール・マドレーヌ}}\label{garniture-madeleine}}

\frsub{Garniture Madelaine}

\index{garniture@garniture!madeleine@--- Madeleine}
\index{madeleine@Madeleine!garniture@garniture ---}
\index{かるにちゆーる@ガルニチュール!まとれーぬ@---・マドレーヌ}
\index{まとれーぬ@マドレーヌ!かるにちゆーる@ガルニチュール・---}

(牛、羊の塊肉、鶏料理に添える)

\begin{itemize}
\item
  小さめのアーティチョークの基底部10個に固めにつくった\protect\hyperlink{sauce-soubise}{スビーズ}を詰める。
\item
  白いんげん豆のピュレ1
  Lあたり卵黄6個と全卵1個を加えてとろみを付け、仕上げにバター150
  gを加えたものを、ダリオル型に詰めて湯をはった天板にのせて低めの温度のオーブンで火を通したタンバル10個。
\item
  【別添】\protect\hyperlink{sauce-demi-glace}{ソース・ドゥミグラス}。
\end{itemize}

\atoaki{}

\hypertarget{garniture-a-la-maillot}{%
\subsubsection[ガルニチュール・マイヨ]{\texorpdfstring{ガルニチュール・マイヨ\footnote{maillot
  (マイヨ、男性名詞)には、産着、肌着、maillot de danseuse
  (マイヨドドンスーズ、踊り子のタイツ)、maillot
  jaune(マイヨジョーヌ、トゥールドフランスでトップの走者が着る黄色いウェア)などいろいろな意味があるが、ここではハムの料理に合わせること(ハムは本来、豚腿肉を加工したもの)からcancan(コンコン、いわゆるフレンチカンカン\ldots{}\ldots{}例えばロートレックの版画、絵画に描かれたような)で踊り子がタイツを履いた脚を高く上げて踊る姿を示唆していると解釈されよう。}}{ガルニチュール・マイヨ}}\label{garniture-a-la-maillot}}

\frsub{Garniture Madelaine}

\index{garniture@garniture!madeleine@--- Madeleine}
\index{madeleine@Madeleine!garniture@garniture ---}
\index{かるにちゆーる@ガルニチュール!まとれーぬ@---・マドレーヌ}
\index{まとれーぬ@マドレーヌ!かるにちゆーる@ガルニチュール・---}

(牛、羊の塊肉、とりわけハムの料理に添える)

\begin{itemize}
\item
  大きなオリーブ形に成形した\footnote{tourner (トゥルネ)。}にんじん10個と蕪10個は、コンソメで煮る。
\item
  小玉ねぎ20個は下茹でしてからバターで色艶よく炒める\footnote{glacer
    (グラセ)。}。
\item
  縦半割りにした\protect\hyperlink{laitues-braisees-au-jus}{レチュのブレゼ}10個。
\item
  プチポワ100 gとアリコヴェール100 gはバターであえる。
\item
  【別添】\protect\hyperlink{jus-de-veau-lie}{とろみを付けたジュ}。
\end{itemize}

\atoaki{}

\hypertarget{garniture-a-la-maraichere}{%
\subsubsection[ガルニチュール・マレシェール]{\texorpdfstring{ガルニチュール・マレシェール\footnote{maraîcher/maraîchère
  比較的小規模な野菜生産者のこと。そのため、既出のガルニチュール・ジャルディニエールと非常に似た意味、すなわちあえて日本語にするならどちらも「菜園風」くらいの訳になろう。}}{ガルニチュール・マレシェール}}\label{garniture-a-la-maraichere}}

\frsub{Garniture à la Maraîchère}

\index{garniture@garniture!maraichere@--- à la Maraîchère}
\index{maraicher@maraîcher/maraîchère!garniture@garniture à la Maraîchère}
\index{かるにちゆーる@ガルニチュール!まれしえーる@---・マレシェール}
\index{まれしえーる@マレシェール!かるにちゆーる@ガルニチュール・---}

(牛、羊の塊肉の料理に添える)

\begin{itemize}
\item
  サルシフィ\footnote{Salsifis
    こんにちフランス語で一般的にサルシフィと呼ばれているのは和名キバナバラモンジン。キク科の根菜。表皮が黒く、ごぼうに似ているが風味は異なる。また、じっくり時間をかけて加熱すれば筋っぽさがなくなり、とても柔らかくなる。別名scorzonère(スコルゾネール)。本来のサルシフィは表皮がやや白く、風味もやや異なるが、生産量はスコルゾネールと逆転するかたちで減少しつつある。}は長さ4
  cmの筒切りにして柔らかく茹で、固めにつくった\protect\hyperlink{veloute}{ヴルテ}であえる。
\item
  大きめのじゃがいも10個は\protect\hyperlink{pommes-de-terre-chateau}{シャトー}\footnote{大きめのオリーブのような形に剥き、塩こしょうして澄ましバターでゆっくり柔らかく火を通す。仕上げにパセリのみじん切りを散らす。}にする。
\item
  芽キャベツ(小)300
  gは下茹でしてからバターを入れた鍋で弱火で蒸し煮する\footnote{étuver
    (エテュヴェ)。}。
\item
  【別添】肉をブレゼまたはポワレした際のフォンをソースに仕上げて添える。
\end{itemize}

\atoaki{}

\hypertarget{garniture-marechal}{%
\subsubsection[ガルニチュール・マレシャル]{\texorpdfstring{ガルニチュール・マレシャル\footnote{元帥の意。}}{ガルニチュール・マレシャル}}\label{garniture-marechal}}

\frsub{Garniture Maréchal}

\index{garniture@garniture!marrechal@--- Maréchal}
\index{marechal@maréchal!garniture@garniture ---}
\index{かるにちゆーる@ガルニチュール!まれしやる@---・マレシャル}
\index{まれしやる@マレシャル!かるにちゆーる@ガルニチュール・---}

\hypertarget{a.}{%
\subparagraph{A.}\label{a.}}

仔牛胸腺肉、牛、羊の塊肉の料理に添える場合\ldots{}\ldots{}

\begin{itemize}
\item
  トリュフ入りの鶏のファルスをスプーンで成形したクネル10個。
\item
  50〜60gのトリュフをスライスし、\protect\hyperlink{sauce-italienne}{イタリア風ソース}であえる\footnote{原文
    lié à l'italienne
    訳文は英訳第5版の「\protect\hyperlink{sauce-italienne}{イタリア風ソース}であえる」としているのに倣ったが、この表現自体は細かいさいの目に刻んだマッシュルーム(茸)であえる、の意。そのため、可能性としては\protect\hyperlink{duxelles-seche}{デュクセル・セッシュ}であえるということもあり得るが、実際には「つなぎ」に相当するものが必要になるだろう。}。
\item
  【別添】マデイラ酒風味の\protect\hyperlink{sauce-demi-glace}{ソース・ドゥミグラス}
\end{itemize}

\hypertarget{b.}{%
\subparagraph{B.}\label{b.}}

鶏胸肉のフィレ、仔牛胸腺肉の薄切り、ノワゼット、乳呑仔羊の骨付き背肉に添える場合\ldots{}\ldots{}

\begin{itemize}
\item
  ガルニチュールはバターで色艶よく炒めたトリュフの大きなスライスのみをメインの素材の上にのせる。バターであえたアスパラガスの穂先、季節でない場合にはごく小さなプチポワを添える。

  このガルニチュールを合わせる料理は必ず、細かい生パン粉に
  \(\frac{1}{3}\)量のトリュフのみじん切りを混ぜたイギリス式パン粉衣\footnote{素材に小麦粉をまぶして卵液にくぐらせ、パン粉衣を付けて油で揚げる方法。ただし日本と異なりパン粉は粒子の細かいものが一般的。}を付けて揚げ焼きしたもの\footnote{初版「これらの素材はパン粉衣を付けるか、トリュフのみじん切りをまぶし付けるか、パン粉の\(\frac{1}{3}\)量のトリュフを混ぜた衣を付けて調理する」。第二版〜第三版「これらの素材は必ず、トリュフのみじん切りをまぶし付けるか、細かい生パン粉に\(\frac{1}{3}\)量のトリュフのみじん切りを混ぜた衣を付けて調理する」。なお、このように直接素材にパン粉衣がうまく付くとはかぎらないので、通常は溶かしバターを素材に塗ってからパン粉の衣を付ける。これをフランス式パン粉衣
    pané à la française (パネアラフロンセーズ)という。}。
\end{itemize}

\atoaki{}

\hypertarget{garniture-marie-louise}{%
\subsubsection[ガルニチュール・マリ=ルイーズ]{\texorpdfstring{ガルニチュール・マリ=ルイーズ\footnote{マリア・ルイーザ(1791〜1847)。神聖ローマ皇帝フランツ2世の娘で、フランス皇帝ナポレオン1世の皇后。ナポレオンを憎み恐れて育ったにもかかわらず、ナポレオンがジョゼフィーヌとの離婚後に名家との婚姻を望んだため、オーストリアの外務大臣メテルニヒの計略により婚姻させられる。ナポレオン失脚後はパルマ公国の女公となる(在位1814〜1847)。ドイツ語ではMaria
  Ludovica von Österreich (マリア ルドウィカ フォン
  エスターライヒ)、フランス語ではMarie-Louise d'Autriche (マリルイーズ
  ドートリッシュ)と呼ばれる。}}{ガルニチュール・マリ=ルイーズ}}\label{garniture-marie-louise}}

\frsub{Garniture Marie-Louise}

\index{garniture@garniture!marie-louise@--- Marie-Louise}
\index{marie-louise@Marie-Louise (d'Autriche)!garniture@garniture ---}
\index{かるにちゆーる@ガルニチュール!まりるいーす@---・マリ=ルイーズ}
\index{まりるいーす@マリ=ルイーズ!かるにちゆーる@ガルニチュール・---}

(ノワゼット、トゥルヌド、鶏料理に添える)

\begin{itemize}
\item
  アーティチョークの基底部\footnote{fond d'artichaut
    (フォンダルティショー)。}は添える肉に応じてサイズを選び、バターで蒸し煮して、\protect\hyperlink{sauce-soubise}{スビーズ}を
  \(\frac{1}{4}\)
  量加えたマッシュルームの固めのピュレを絞り袋でドーム状に詰める。
\item
  【別添】\protect\hyperlink{jus-de-veau-lie}{とろみを付けたジュ}。
\end{itemize}

\atoaki{}

\hypertarget{garniture-marniere}{%
\subsubsection[ガルニチュール・マリニエール]{\texorpdfstring{ガルニチュール・マリニエール\footnote{marinier/marinière
  \textless{} mare
  ラテン語「海」から派生した語。\protect\hyperlink{sauce-mariniere}{ソース・マリニエール}も参照。}}{ガルニチュール・マリニエール}}\label{garniture-marniere}}

\frsub{Garniture à la Marinière}

\index{garniture@garniture!mariniere@--- à la Marinière}
\index{mariniere@marinière!garniture@garniture à la ---}
\index{かるにちゆーる@ガルニチュール!まりにえーる@---・マリニエール}
\index{まりにえーる@マリニエール!かるにちゆーる@ガルニチュール・---}

(魚料理に添える)

\begin{itemize}
\item
  小さなムール貝\(\frac{3}{4}\)
  L(35個)は白ワインで蒸し煮して、身の周囲をきれいに掃除する\footnote{ébarber
    (エバルベ)。}。
\item
  殻をむいたクルヴェット\footnote{crevette
    小海老。小さめで生のときは灰色がかった crevette grise
    (クルヴェットグリーズ)とやや大きめで美味なcrevette
    rose(クルヴェットローズ)の2種が代表的。}の尾の身100 g。
\item
  \protect\hyperlink{sauce-mariniere}{ソース・マリニエール}
\end{itemize}

\atoaki{}

\hypertarget{garniture-marquise}{%
\subsubsection[ガルニチュール・マルキーズ]{\texorpdfstring{ガルニチュール・マルキーズ\footnote{marquis
  / marquise 侯爵、侯爵夫人の意。}}{ガルニチュール・マルキーズ}}\label{garniture-marquise}}

\frsub{Garniture Marquise}

\index{garniture@garniture!marquise@--- Marquise}
\index{marquise@marquise!garniture@garniture ---}
\index{かるにちゆーる@ガルニチュール!まるきーす@---・マルキーズ}
\index{まるきーす@マルキーズ!かるにちゆーる@ガルニチュール・---}

(ノワゼット、トゥルヌド、鶏料理に添える)

\begin{enumerate}
\def\labelenumi{\arabic{enumi}.}
\item
  縁に波形の飾り模様の付いた小さなタルトレット10個を空焼きする。これに以下を詰める。アムレット\footnote{牛、仔牛、仔羊の脊髄(=moelle
    モワル)。}250
  gをやや低温で火を通して小さな筒切りにする。アスパラガスの穂先125
  g、太さ1〜2 mmの千切り\footnote{julienne (ジュリエーヌ)。}にしたトリュフ
  50
  g、これらを、\protect\hyperlink{beurre-d-ecrevisse}{エクルヴィスバター}を仕上げに加えた\protect\hyperlink{sauce-allemande}{ソース・アルマンド}
  1 \(\frac{1}{2}\) dLであえる。
\item
  濃いトマトピュレを混ぜ込んだ\protect\hyperlink{pommes-de-terre-duchesse}{じゃがいものデュシェス}を絞り袋で天板に小さな卵形に絞り出し、縦中央にナイフなどで切れ込みを入れ、オーブンに入れて提供数分前にこんがりと焼きあげたもの\footnote{ここで説明されているのは、\protect\hyperlink{pommes-de-terre-marquise}{じゃがいものマルキーズ}のレシピそのものといっていい内容で、かなりの部分が野菜料理の節にあるレシピと重複している。じゃがいものマルキーズにおいて形状は2パターン提示されており、これはそのうちのひとつであり、こんにちではあまりつくられなくなったパターンの方。大雑把なイメージとしては大きなコーヒー豆のあるいはキドニービーンズのような形を想像すればいいだろう。ここではかなり意訳したが、直訳すると「じゃがいものデュセスでつくったPain
    de la Mecque
    (パンドラメック=メッカのパンの意)」とある。本来これはシュー生地を卵形に天板に絞り出して溶き卵を塗り、グラニュー糖を振ってから縦中央にナイフで切れ込みを入れて焼くという、19世紀には比較的ポピュラーだった焼き菓子。グフェ『パティスリの本』(1873年)にもレシピが2種掲載されている
    (pp.287〜288)。シュー生地とほぼ同じものを使うため、内部に空洞が出来るが、そこにクレーム・シャンティイなどを絞り袋を使って詰めるバリエーションもあった。この焼き菓子になぜ「メッカ」の地名が付けられているのか、また、トマトピュレを加えたじゃがいものデュシェスをその形状に似せて焼いたものをなぜ、じゃがいものマルキーズ(=侯爵夫人)と呼ぶのかといった理由、由来は不明。ちなみに貴族の格としては公爵夫人(デュシェス)のほうが侯爵夫人よりも一般的に上位とされた。}20個。
\end{enumerate}

\atoaki{}

\hypertarget{garniture-a-la-marseillaise}{%
\subsubsection[ガルニチュール・マルセイエーズ]{\texorpdfstring{ガルニチュール・マルセイエーズ\footnote{marseillais(e)
  (マルセイエ / マルセイエーズ)マルセイユ Marseille
  の、の意。現在のフランス国歌 \emph{La Marseillaise}
  はフランス革命期の1792年に作曲され、同年8月10日のチュイルリー宮襲撃事件の際にマルセイユの義勇兵たちが口ずさんでいたことをきっかけにパリ市民の間で流行した。ただしマルセイユの義勇兵たちが作曲、作詞したわけでもなければ、その内容がマルセイユと関連があるわけでもない。いずれにしても第一帝政から王政復古期にかけては歌詞の「暴君を倒せ」という部分に問題があるいう理由から禁止され、1830年の七月革命以降解禁、第三共和政(1870〜1940)において正式に国歌として制定された。日本ではあまり知られていないが、決して平和的な内容の歌詞ではなく、むしろ激しい戦意を鼓舞する内容。とりわけルフラン(繰り返し部分)の「武器を取れ 市民らよ 隊列を組め 進もう 進もう 穢れた血が 我らの畝を満たさんことを」にその激烈さがよく表われている。1979年にセルジュ・ゲンズブールがこの曲をレゲエ風に編曲して発表したところ、「愛国者」からの脅迫が相次いだというエピソードは有名。}}{ガルニチュール・マルセイエーズ}}\label{garniture-a-la-marseillaise}}

\frsub{Garniture à la Marseillaise}

\index{garniture@garniture!marseillaise@--- à la Marseillaise}
\index{marseillais@marseillais(e)!garniture@garniture ---e}
\index{かるにちゆーる@ガルニチュール!まるせいえーす@---・マルセイエーズ}
\index{まるせいえーす@マルセイエーズ!かるにちゆーる@ガルニチュール・---}

(牛、羊の塊肉の料理に添える)

\begin{itemize}
\item
  小さめのトマト5個を半割りにして中をくり抜き、にんにく1片と油をひとたらししてオーブンで焼く。立派なアンチョビのフィレを円環状になるようトマトに盛り込み、さらに成形した大きなオリーブを詰める。
\item
  それぞれのトマトの間に、細かく切って揚げたフライドポテトを配する。
\item
  【別添】\protect\hyperlink{sauce-provencale}{プロヴァンス風ソース}
\end{itemize}

\atoaki{}

\hypertarget{garniture-mascotte}{%
\subsubsection[ガルニチュール・マスコット]{\texorpdfstring{ガルニチュール・マスコット\footnote{エドモン・オドロン(1842〜1901)作曲のオペラコミック『ラ・マスコット』(1880年初演)にちなんだ名称。}}{ガルニチュール・マスコット}}\label{garniture-mascotte}}

\frsub{Garniture Mascotte}

\index{garniture@garniture!mascotte@--- Mascotte}
\index{mascotte@Mascotte!garniture@garniture ---}
\index{かるにちゆーる@ガルニチュール!ますこつと@---・マスコット}
\index{ますこつと@マスコット!かるにちゆーる@ガルニチュール・---}

(ノワゼット、トゥルヌド、鶏料理に添える)

\begin{itemize}
\item
  アーティチョークの芯\footnote{原文は fonds d'artichaut
    なので字義通りに解すれば「アーティチョークの基底部」だが、下茹でしないということは比較的若どりのアーティチョークを用いる必要がある。その場合は花萼部の上半分程を切り捨て、茎の皮を剥いて四つ割りにするのが一般的。}10個は生のまま四つ割りに切り、バターでソテーする。
\item
  小さなじゃがいも20個はオリーブ形に剥き、バターで火を通す。
\item
  小さな玉にくり抜いたトリュフ10個。
\item
  【別添】肉を焼いた鍋に白ワインを注いでデグラセ\footnote{déglacerr
    ソテーする際に流れ出た肉汁が煮詰まって鍋底にシロップ状に貼り付いたのを、何らかの液体を注いで溶かし出すこと。}し、\protect\hyperlink{fonds-de-veau-brun}{仔牛のフォン}を加えてソースに仕上げる。
\end{itemize}

\hypertarget{nota-garniture-mascotte}{%
\subparagraph{【原注】}\label{nota-garniture-mascotte}}

このガルニチュール・マスコットは、必ずココット仕立て\footnote{こんにちのココット仕立てを同じだが、本書においては「ポワレ」のバリエーションのひとつとして位置付けられている。\protect\hyperlink{poeles-speciaux-dits-en-casserole-ou-en-cocotte}{特殊なポワレ、カスロール仕立て、ココット仕立て}参照。}の肉の周囲を飾るようにして供する。

\atoaki{}

\hypertarget{garniture-massena}{%
\subsubsection[ガルニチュール・マセナ]{\texorpdfstring{ガルニチュール・マセナ\footnote{ナポレオン軍の元帥を務めたアンドレ・マセナ(1758〜1817)のこと。スイス戦役や半島戦争で著しい功績をあげた。また、1898年就役、1915年退役となったフランス海軍の戦艦マセナは彼のちなんで命名された。}}{ガルニチュール・マセナ}}\label{garniture-massena}}

\frsub{Garniture Masséna}

\index{garniture@garniture!massena@--- Masséna}
\index{massena@Masséna!garniture@garniture ---}
\index{かるにちゆーる@ガルニチュール!ませな@---・マセナ}
\index{ませな@マセナ!かるにちゆーる@ガルニチュール・---}

(ノワゼット、トゥルヌドに添える)

\begin{itemize}
\item
  中位か小さめのアーティチョークの基底部\footnote{あらかじめ適切に下処理をし、火を通しておくこと。}に固く仕上げた\protect\hyperlink{sauce-bearnaise}{ソース・ベアルネーズ}を詰める。
\item
  新鮮で大きな牛骨髄を輪切りにしてコンソメで沸騰しないよう火を通した\footnote{pocher
    (ポシェ)。}もの10枚。
\item
  【別添】\protect\hyperlink{sauce-tomate}{トマトソース}。
\end{itemize}

\atoaki{}

\hypertarget{garniture-matelote}{%
\subsubsection[ガルニチュール・マトロット]{\texorpdfstring{ガルニチュール・マトロット\footnote{水夫風の意。\protect\hyperlink{sauce-matelote}{ソース・マトロット}および魚料理「\protect\hyperlink{matelotes-types}{マトロット}」も参照。}}{ガルニチュール・マトロット}}\label{garniture-matelote}}

\frsub{Garniture Matelote}

\index{garniture@garniture!matelote@--- Matelote}
\index{matelote@matelote!garniture@garniture ---}
\index{かるにちゆーる@ガルニチュール!まとろつと@---・マトロット}
\index{まとろつと@マトロット!かるにちゆーる@ガルニチュール・---}

(魚料理その他に添える)

\begin{itemize}
\item
  小玉ねぎ300 gは下茹でしてバターで色艶よく炒める\footnote{glacer
    (グラセ)。}。
\item
  マッシュルーム(小)200 gは茹でる。
\item
  食パンをハート形に切ってバターで揚げたクルトン10枚。場合によっては\protect\hyperlink{court-bouillon-c}{クールブイヨン}で火を通したエクルヴィスも添える。
\end{itemize}

\atoaki{}

\hypertarget{garniture-medicis}{%
\subsubsection[ガルニチュール・メディシス]{\texorpdfstring{ガルニチュール・メディシス\footnote{ルネサンス期のイタリア、フィレンツェにおいて金融業などにより実質的な支配者として君臨した名家。フランス史においてもっとも有名なカトリーヌ・ド・メディシス(1519〜1589)は後のフランス国王アンリ2世のもとに嫁ぎ、アンリ2世の死後15才で即位した長男フランソワ2世の摂政として権力を掌握、フランソワ2世が病死してシャルル9世が即位した後も実権を握り続けた。カトリックとプロテスタントが対立したユグノー戦争の時代であり、サン・バルテルミの虐殺(1572年)に関与したさえ当時はまことしやかに語られたという。また、カトリーヌがフランスの宮廷にフィレンツェの、とりわけ食文化を紹介、導入したという逸話は、アーティチョークからフォークにいたるまで非常に多いが、逸話の域を出ないものがほとんどで、史料として残っているものは非常に少ない。逆に言えば、ルネサンス期に文化的先進国だったイタリアから王妃を娶るということそれ自体が象徴するように、イタリア文化がさまざまなルート、形態でフランス文化に吸収された時代と見るのが妥当だろう。また、メディチ家はカトリーヌの後もフランス王アンリ4世(1553〜1610)にマリー・ド・メディシス(1575〜1642)を嫁がせている。なお、イタリア語の家名Medici(メディチ)はフランス語で伝統的にMédicis(メディシス)と呼ばれているため、ここではその慣習に倣って表記した。}}{ガルニチュール・メディシス}}\label{garniture-medicis}}

\frsub{Garniture Médicis}

\index{garniture@garniture!medicis@--- Médicis}
\index{medicis@médicis!garniture@garniture ---}
\index{かるにちゆーる@ガルニチュール!めていしす@---・メディシス}
\index{めていしす@メディシス!かるにちゆーる@ガルニチュール・---}
\index{めていちけ@メディチ家 ⇒ メディシス!かるにちゆーる@ガルニチュール・---}

(牛、羊の塊肉、ノワゼット、トゥルヌドに添える)

\begin{itemize}
\item
  空焼きしたタルトレット10個に、マカロニとさいの目に切ったトリュフをフォワグラのピュレであえて詰める。
\item
  バターであえたプチポワ。
\item
  【別添】\protect\hyperlink{jus-de-veau-lie}{とろみを付けたジュ}。
\end{itemize}

\atoaki{}

\hypertarget{garniture-a-la-mexicaine}{%
\subsubsection{ガルニチュール・メキシコ風}\label{garniture-a-la-mexicaine}}

\frsub{Garniture à la Mexicaine}

\index{garniture@garniture!mexicaine@--- à la Mexicaine}
\index{mexicain@mexicain(e)!garniture@garniture à la ---}
\index{かるにちゆーる@ガルニチュール!めきしこふう@---・メキシコ風}
\index{めきしこふう@メキシコ風!かるにちゆーる@ガルニチュール・---}

(牛、羊の塊肉、鶏料理に添える)

\begin{itemize}
\item
  マッシュルーム10個はグリル焼きし、濃く煮詰めた\protect\hyperlink{portugaise}{トマトのフォンデュ}を添える。
\item
  ポワヴロン10個はグリル焼きする。
\item
  【別添】カイエンヌを強く効かせた\protect\hyperlink{jus-lie-tomate}{トマト風味のジュ}。
\end{itemize}

\atoaki{}

\hypertarget{garniture-mignon}{%
\subsubsection{ガルニチュール・ミニョン}\label{garniture-mignon}}

\frsub{Garniture Mignon}

\index{garniture@garniture!mignon@--- Mignon}
\index{mignon@mignon!garniture@garniture ---}
\index{かるにちゆーる@ガルニチュール!みによん@---・ミニョン}
\index{みによん@ミニョン!かるにちゆーる@ガルニチュール・---}

(ノワゼット、トゥルヌドに添える)

\begin{itemize}
\item
  小さなアーティチョークの基底部10個はバターで蒸し煮\footnote{étuver
    (エチュヴェ)。}し、バターであえたプチポワを詰める。
\item
  \protect\hyperlink{farce-b}{鶏の滑らかなファルス}で丸く作った小さなクネル10個に、それぞれトリュフのスライスをのせる。
\item
  【別添】デグラセ\footnote{déglacer
    (デグラセ)。肉を焼いた際に流れ出た肉汁が煮詰まって鍋底に濃いペースト状に貼り付いているのを酒類やフォンを加えて溶かし出すこと。焦げを取ることではないので注意。}したフォンにバターを加えて滑らかなソースに仕上げる。
\end{itemize}

\atoaki{}

\hypertarget{garniture-a-la-milanaise}{%
\subsubsection{ガルニチュール・ミラノ風}\label{garniture-a-la-milanaise}}

\frsub{Garniture à la Milanaise}

\index{garniture@garniture!milanaise@--- à la Milanaise}
\index{milanais@milanais(e)!garniture@garniture ---e}
\index{かるにちゆーる@ガルニチュール!みらのふう@---・ミラノ風}
\index{みらのふう@ミラノ風!かるにちゆーる@ガルニチュール・---}

(牛、羊の塊肉料理に添える)

\begin{itemize}
\item
  マカロニ400 gは茹でて4 cmの長さに切る。
\item
  \protect\hyperlink{saumure-liquide-pour-langues}{赤く漬けた舌肉} 50
  g。
\item
  ハムとマッシュルーム各50 g。
\item
  トリュフ40 g\ldots{}\ldots{}これらは千切り\footnote{julienne
    (ジュリエーヌ)。}にする。
\item
  おろしたグリュイエールチーズ 50 gとパルメザンチーズ50 g。
\item
  トマトピュレ1 dL。
\item
  バター100 g。
\item
  【別添】澄んだ\protect\hyperlink{sauce-tomate}{トマトソース}。
\end{itemize}

\atoaki{}

\hypertarget{garniture-mirabeau}{%
\subsubsection[ガルニチュール・ミラボー]{\texorpdfstring{ガルニチュール・ミラボー\footnote{フランス革命期、立憲君主主義を主張したオノレ・ガブリエル・ミラボー伯爵(1748〜1891)の名を冠した料理名だが、多くの日本人にとっては「
  ミラボー橋の下を セーヌ川が流れ われらの恋が流れる」という堀口大學の美しい訳で知られたアポリネールの詩のほうがなじみがあるだろうか。実際にパリにかかる橋のなかでは下流寄りに位置し、幅の広い橋ではあるが、それ自体や周囲の風景にとくに風情があるとは言い難い。実際に文学散歩としてミラボー橋を訪ずれた日本人観光客や留学生のほとんどが、がっかりしたという感想を持つというが、これは堀口訳が原詩以上といっていいくらい美しく、日本人の琴線に触れたがゆえに過度の情緒的な期待をさせてしまうことによるのだろう。}}{ガルニチュール・ミラボー}}\label{garniture-mirabeau}}

\frsub{Garniture Mirabeau}

\index{garniture@garniture!mirabeau@--- Mirabeau}
\index{mirabeau@Mirabeau!garniture@garniture ---}
\index{かるにちゆーる@ガルニチュール!みらほー@---・ミラボー}
\index{みらほー@ミラボー!かるにちゆーる@ガルニチュール・---}

(牛、羊の赤身肉のグリルに添える)

\begin{itemize}
\item
  アンチョビのフィレ20枚を網目状に肉の上に配置する。
\item
  大きなオリーブ10個は種を抜いておく。
\item
  茹がいたエストラゴンの葉で皿の周囲を飾る。
\item
  \protect\hyperlink{beurre-d-anchois}{アンチョビバター} 125 g。
\end{itemize}

\atoaki{}

\hypertarget{garniture-mirette}{%
\subsubsection[ガルニチュール・ミレット]{\texorpdfstring{ガルニチュール・ミレット\footnote{目、瞳、瞼、の意。}}{ガルニチュール・ミレット}}\label{garniture-mirette}}

\frsub{Garniture Mirette}

\index{garniture@garniture!mirette@--- Mirette}
\index{mirette@mirette!garniture@garniture ---}
\index{かるにちゆーる@ガルニチュール!みれつと@---・ミレット}
\index{みれつと@ミレット!かるにちゆーる@ガルニチュール・---}

\begin{itemize}
\item
  \protect\hyperlink{pommes-de-terre-mirette}{じゃがいものミレット}で作った小さなタンバル\footnote{円筒形にする仕立て。大きなものはmoule
    à
    timbale(ムーラタンバール)という専用の型を用いる。原義は「小太鼓」であり、直径に比べてやや高さのないものが本来の姿だが、19世紀にはかなり高さのある円筒形の仕立てにもこの名称が用いられた。}仕立て10個(\protect\hyperlink{legumes}{野菜料理}、\protect\hyperlink{pommes-de-terre-mirette}{じゃがいものミレット}参照)。
\item
  【別添】\protect\hyperlink{sauce-chateaubriand}{ソース・シャトーブリヤン}。
\end{itemize}

\atoaki{}

\hypertarget{garniture-a-la-moderne}{%
\subsubsection[ガルニチュール・アラモデルヌ]{\texorpdfstring{ガルニチュール・アラモデルヌ\footnote{近代風、の意。}}{ガルニチュール・アラモデルヌ}}\label{garniture-a-la-moderne}}

\frsub{Garniture à la Moderne}

\index{garniture@garniture!moderne@--- à la Moderne}
\index{moderne@moderne!garniture@garniture à la ---}
\index{かるにちゆーる@ガルニチュール!あらもてるぬ@---・アラモデルヌ}
\index{もてるぬ@モデルヌ(アラ)!かるにちゆーる@ガルニチュール・アラモデルヌ}
\index{きんたいふう@近代風 ⇒ アラモデルヌ!かるにちゆーる@ガルニチュール・アラモデルヌ}

(牛、羊の塊肉の料理に添える)

\begin{itemize}
\item
  六角形の小さな型10個の底にトリュフのスライスを敷き、シャルトルーズ仕立て\footnote{\protect\hyperlink{bordures-en-legumes}{野菜で作る縁飾り}および訳注参照。この本文は非常に簡潔に記述されているが、実際の工程としては、\protect\hyperlink{chou-braise}{サヴォイキャベツを下茹でした後にブレゼ}し、バターを塗った型の底にトリュフのスライスを敷いた後、それぞれに下茹でして小さな拍子木に切ったにんじん、蕪、アリコヴェールなどを配色よく周囲に貼り付け、さらに\protect\hyperlink{farce-a}{ファルス}を塗ってからブレゼしたサヴォイキャベツを詰め、湯煎にかけて加熱して固める、という手の込んだもの。}に周囲を装飾して、ブレゼしたサヴォイキャベツを詰める。加熱後に型を裏返して型から外す。
\item
  半割りにして\protect\hyperlink{laitues-farcies-pour-garniture}{詰め物をしたレチュのブレゼ}
  10個。
\item
  \protect\hyperlink{farce-b}{バター入り仔牛のファルス}をスプーンで楕円型に成形し、\protect\hyperlink{saumure-liquide-pour-langues}{赤く漬けた舌肉}で装飾を施した小さなクネル10個。
\item
  【別添】\protect\hyperlink{jus-de-veau-lie}{とろみを付けたジュ}。
\end{itemize}

\atoaki{}

\hypertarget{garniture-montbazon}{%
\subsubsection[ガルニチュール・モンバゾン]{\texorpdfstring{ガルニチュール・モンバゾン\footnote{トゥール近郊の町の名だが、このガルニチュールの由来は不明。}}{ガルニチュール・モンバゾン}}\label{garniture-montbazon}}

\frsub{Garniture Montbazon}

\index{garniture@garniture!montbazon@--- Montbazon}
\index{montbazon@montbazon!garniture@garniture ---}
\index{かるにちゆーる@ガルニチュール!もんはそん@---・モンバゾン}
\index{もんはそん@モンバゾン!かるにちゆーる@ガルニチュール・---}

(鶏料理に添える)

\begin{itemize}
\item
  大きさを揃えた子羊胸腺肉10個に拍子木あるいは楔形に切ったトリュフを刺し、\protect\hyperlink{les-poeles}{ポワレ}する。
\item
  \protect\hyperlink{farce-b}{バター入りの鶏のファルス}で楕円形に作りトリュフで装飾を施したクネル10個。
\item
  表面に渦巻状の刻み模様をつけた白いマッシュルーム10個。
\item
  トリュフのスライス10枚。
\item
  【別添】\protect\hyperlink{sauce-supreme}{ソース・シュプレーム}
\end{itemize}

\atoaki{}

\hypertarget{garniture-a-la-montmorency}{%
\subsubsection[ガルニチュール・モンモランシー]{\texorpdfstring{ガルニチュール・モンモランシー\footnote{10世紀頃から続く貴族の名家。歴史上重要な活躍をした人物を多く輩出した。イルドフランス県にある街の名称でもある。}}{ガルニチュール・モンモランシー}}\label{garniture-a-la-montmorency}}

\frsub{Garniture à la Montmorency}

\index{garniture@garniture!montmorency@--- à la Montmorency}
\index{montbazon@montbazon!garniture@garniture à la ---}
\index{かるにちゆーる@ガルニチュール!もんもらんしー@---・モンモランシー}
\index{もんもらんしー@モンモランシー!かるにちゆーる@ガルニチュール・---}

(牛、羊の塊肉料理、鶏料理に添える)

\begin{itemize}
\item
  アーティチョークの基底部10個はバターで蒸し煮\footnote{étuver
    (エチュヴェ)}して、オランデーズソースでであえた\protect\hyperlink{garniture-macedoine}{マセドワーヌ}を詰める。
\item
  アスパラガスの小さな穂先10束。
\item
  【別添】肉の煮汁を加えて仕上げた\protect\hyperlink{sauce-madere}{ソース・マデール}。
\end{itemize}

\atoaki{}

\hypertarget{garniture-a-la-moissonneuse}{%
\subsubsection[ガルニチュール・モワソヌーズ]{\texorpdfstring{ガルニチュール・モワソヌーズ\footnote{moissonneur/moissoneuse
  (モワソヌール/モワソヌーズ)麦などの刈り入れをする人の意。基本的にレシピはアルファベット順の配列だが、ここはその原則が崩れている。}}{ガルニチュール・モワソヌーズ}}\label{garniture-a-la-moissonneuse}}

\frsub{Garniture à la moissonneuse}

\index{garniture@garniture!moissoneuse@--- à la Moissonneuse}
\index{moissoneur@moissoneur/moissoneuse!garniture@garniture à la moissoneuse}
\index{かるにちゆーる@ガルニチュール!もわそぬーす@---・モワソヌーズ}
\index{もわそぬーる@モワソヌール/モワソヌーズ!かるにちゆーる@ガルニチュール・モワソヌーズ}

(牛、羊の塊肉料理に添える)

\begin{itemize}
\item
  千切りにしたレチュを加えた\protect\hyperlink{petits-pois-a-la-francaise}{プチポワ・アラフランセーズ}
  1 L。
\item
  じゃがいも2個はスライスする。
\item
  脂身の少ない豚ばら肉の塩漬125 gはさいの目に切って下茹でする。
\item
  上記全部をまとめて火入れを仕上げる。
\item
  \protect\hyperlink{beurre-manie}{ブールマニエ}を加えてごく軽くとろみ付けする。
\end{itemize}

\atoaki{}

\hypertarget{garniture-montreuil}{%
\subsubsection[ガルニチュール・モントルイユ]{\texorpdfstring{ガルニチュール・モントルイユ\footnote{フランス北端の海に近い町、Montreuil-sur-Mer(モントルイユシュルメール)のこと。内陸部のセーヌサンドゥニ県には
  Montreuil-sous-Bois(モントルイユスーボワ)という非常によく似た名の町があり、その他にもモントルイユの地名はいくつかあるので注意。}}{ガルニチュール・モントルイユ}}\label{garniture-montreuil}}

\frsub{Garniture Montreuil}

\index{garniture@garniture!montreuil@--- Montreuil}
\index{montreuil@Montreuil!garniture@garniture ---} \index{かるにちゆー
る@ガルニチュール!もんとるいゆ@---・モントルイユ} \index{もんとるいゆ@
モントルイユ!かるにちゆーる@ガルニチュール・---}

(魚料理に添える)

\begin{itemize}
\item
  じゃがいも20個を剥いて成形し、\protect\hyperlink{pommes-de-terres-a-l-anglaise}{アラングレーズ}\footnote{à
    l'anglaise
    イギリス風の意だが、フランス料理の用語としては、単に塩を加えた湯で火を通すことを指す。野菜の調理法のうちで基本のひとつ。なお、実際にイギリスで塩茹でがポピュラーな調理法かというと必ずしもそうではないらしく、1907年の英語版にはただし書きが付けられている。}にする。これを魚の周囲に縁飾りのように配する。
\item
  魚には\protect\hyperlink{sauce-vin-blanc}{白ワインソース}を、じゃがいもには\protect\hyperlink{sauce-aux-crevettes}{ソース・クルヴェット}を塗るように覆いかける。
\end{itemize}

\atoaki{}

\hypertarget{garniture-montpensier}{%
\subsubsection[ガルニチュール・モンパンシエ]{\texorpdfstring{ガルニチュール・モンパンシエ\footnote{15世紀まで遡ることの出来る名家で、とりわけ後の7月王政期で国王となったルイ・フィリップの弟で、フランス革命期には革命派であったアントワーヌ・フィリップ・ドルレアン・モンパンシエ公爵(1775〜1807)の名を冠した料理名。なおフランス語の発音を無理矢理カナで書くと
  Montpensierはモンポンシエのほうが近い音に聞こえるケースも多い。}}{ガルニチュール・モンパンシエ}}\label{garniture-montpensier}}

\frsub{Garniture Montpensier}

\index{garniture@garniture!montpensier@--- Montpensier}
\index{montpensier@Montpensier!garniture@garniture ---}
\index{かるにちゆーる@ガルニチュール!もんはんしえ@---・モンパンシエ}
\index{もんはんしえ@モンパンシエ!かるにちゆーる@ガルニチュール・---}

(ノワゼット、トゥルヌド、鶏料理に添える)

\begin{itemize}
\item
  バターであえたアスパラガスの穂先の束。
\item
  ノワゼットまたはトゥルヌドにトリュフのスライスをのせる。
\item
  【別添】ソテーした鍋をデグラセし、バターを加えて滑らかな口あたりに仕上げたソース。
\end{itemize}

\atoaki{}

\hypertarget{garniture-nantua}{%
\subsubsection[ガルニチュール・ナンチュア]{\texorpdfstring{ガルニチュール・ナンチュア\footnote{エクルヴィスが獲れることで知られるローヌ・アルプ地方のナンチュア湖に由来した名称。\protect\hyperlink{sauce-nantua}{ソース・ナンチュア}も参照}}{ガルニチュール・ナンチュア}}\label{garniture-nantua}}

\frsub{Garniture Nantua}

\index{garniture@garniture!nantua@--- Nantua}
\index{nantua@Nantua!garniture@garniture ---}
\index{かるにちゆーる@ガルニチュール!なんちゆあ@---・ナンチュア}
\index{なんちゆあ@ナンチュア!かるにちゆーる@ガルニチュール・---}

(魚料理に添える)

\begin{itemize}
\item
  エクルヴィスの尾の身30は\protect\hyperlink{sauce-nantua}{ソース・ナンチュア}であえる。
\item
  トリュフのスライス20枚。
\item
  ソース・ナンチュア。
\end{itemize}

\atoaki{}

\hypertarget{garniture-a-la-napolitaine}{%
\subsubsection{ガルニチュール・ナポリ風}\label{garniture-a-la-napolitaine}}

\frsub{Garniture à la Napolitaine}

\index{garniture@garniture!napolitaine@--- à la Napolitaine}
\index{napolitain@napolitain(e)!garniture@garniture à la  ---e}
\index{かるにちゆーる@ガルニチュール!なほりふう@---・ナポリ風}
\index{なほりふう@ナポリ風!かるにちゆーる@ガルニチュール・---}

(牛、羊の塊肉や鶏料理に添える)

\begin{itemize}
\item
  スパゲッティ500 gを茹でて\footnote{原文は Spaghetti pochés
    つまり「沸騰しない程度の温度で茹でる」とあり、塩を加えることなどは一切記されていない。もっとも、20世紀初頭までスパゲッティをナポリの庶民の多くは「手づかみ」で食べていたのであり、高級料理の食材としてはあまりなじみのないものだった。なお、日本の「ナポリタン」の発祥には諸説あるが、少なくともこのガルニチュールとは関係がないだろう。}、おろしたグリュイエールチーズ50
  gとパルメザンチーズ50 g、トマトピュレ1 dLであえる。バター100
  gを加えて仕上げる。
\item
  【別添】肉をブレゼあるいはポワレ、ポシェしたフォンをソースに仕上げる。
\end{itemize}

\atoaki{}

\hypertarget{garniture-aux-navets}{%
\subsubsection[蕪のガルニチュール]{\texorpdfstring{蕪\footnote{navet
  (ナヴェ)「蕪」と訳してはいるが、日本で主流の「金町小かぶ」系とはまったく系統が違い、いくつかの系統が存在する。いずれの系統も、調理特性、味わいが日本のものと異なる点に注意。ちなみに俗語で
  C'est un navet.
  「味気ない、つまらない」という言い廻わしがあるが、野菜としてはむしろ、よく火を入れることで味わいを引き出すものと考えるべきであり、その点では日本料理における大根とやや近いところもあるだろう。}のガルニチュール}{蕪のガルニチュール}}\label{garniture-aux-navets}}

\frsub{Garniture aux Navets}

\index{garniture@garniture!navets@--- aux Navets}
\index{navet@navet!garniture@garniture aux ---s}
\index{かるにちゆーる@ガルニチュール!かふ@蕪の---}
\index{かふ@蕪!かるにちゆーる@---のガルニチュール}

(羊や仔鴨の料理に添える)

\begin{itemize}
\item
  細長い大きめのオリーブのように成形した蕪30個は、バターと粉糖1つまみを加えてフライパンで色よく炒める。
\item
  バターで色よく炒めた小玉ねぎ20個
\item
  これらの野菜は、添える料理そのものに加えて火入れを仕上げるようにする。
\end{itemize}

\atoaki{}

\hypertarget{garniture-a-la-nicoise}{%
\subsubsection{ガルニチュール・ニース風}\label{garniture-a-la-nicoise}}

\frsub{Garniture à la Niçoise}

\index{garniture@garniture!nicoise@--- à la Niçoise}
\index{nicois@niçois(e)!garniture@garniture à la  ---e}
\index{かるにちゆーる@ガルニチュール!にーすふう@---・ニース風}
\index{にーすふう@ニース風!かるにちゆーる@ガルニチュール・---}

(魚料理に添える場合)

\begin{itemize}
\item
  トマト250
  gは皮を剥き、潰して、叩いたにんにく1編とともにバター\footnote{オリーブオイルではなくバターを使っている点に注目すべきだろう。}でソテーする。最後にエストラゴンのみじん切りを1つまみ加える。
\item
  アンチョビのフィレ10枚。
\item
  黒オリーブ10個。
\item
  ケイパー大さじ1杯。
\item
  \protect\hyperlink{beurre-d-anchois}{アンチョビバター} 30 g。
\item
  粗く皮を剥いて、種子を取り除いたレモンのスライス。
\end{itemize}

(牛、羊の塊肉や鶏料理に添える場合)

\begin{itemize}
\item
  トマト250 gは上述のとおりにする。
\item
  アリコ・ヴェール 300 gはバターであえる\footnote{豆類は未熟なものであっても消化阻害酵素を含むため、必ず下茹でして不活性化させる必要がある。アリコ・ヴェールすなわち「さやいんげん」やpois
    mange-tout (ポワモンジュトゥ)さやえんどう、も同様。}。
\item
  \protect\hyperlink{pommes-de-terre-chateau}{じゃがいものシャトー} 400
  g。
\end{itemize}

盛り付け\ldots{}\ldots{}トマトは肉の上にのせ、アリコ・ヴェールとじゃがいもは適量ずつ交互にまとめて配すること。

\begin{itemize}
\tightlist
\item
  【別添】\protect\hyperlink{jus-de-veau-lie}{とろみを付けたジュ}
\end{itemize}

\atoaki{}

\hypertarget{garniture-a-la-nivernaise}{%
\subsubsection[ガルニチュール・ニヴェルネ風]{\texorpdfstring{ガルニチュール・ニヴェルネ\footnote{ニヴェルネとはフランス革命以前の地方名であり、北がオルレアネ、南はブルボネ、東はブルゴーニュ、西がベリー領に椄していた。現在はニエーヴル県
  le département de la Nièvre (ルデパルトモン ドラ
  ニエーヴル)の一部。食文化的にはブルゴーニュ地方に近いとされている。}風}{ガルニチュール・ニヴェルネ風}}\label{garniture-a-la-nivernaise}}

\frsub{Garniture à la Nivernaise}

\index{garniture@garniture!nivernaise@--- à la Nivernaise}
\index{nivernais@nivernais(e)!garniture@garniture à la  ---e}
\index{かるにちゆーる@ガルニチュール!にうえるねふう@---・ニヴェルネ風}
\index{にうえるねふう@ニヴェルネ風!かるにちゆーる@ガルニチュール・---}

(牛、羊の塊肉料理に添える)

\begin{itemize}
\item
  オリーヴ形に成形したにんじん500
  gは\protect\hyperlink{consomme-blanc-simple}{コンソメ}で煮てから、バターで色艶よく炒める。
\item
  小玉ねぎ300 gもバターで色艶よく炒める\footnote{必要に応じて下茹でしてから炒めるといい。とりわけ日本の「ペコロス」を用いる場合はそのほとんどが黄色系の品種で辛味が強いため、下茹ですることが望ましいだろう。逆に、オニオンブランとも呼ばれる白系品種で比較的若どりの場合は下茹でしない方がいいだろう。}。
\item
  【別添】ブレゼの煮汁。
\end{itemize}

\atoaki{}

\hypertarget{garniture-a-la-normande}{%
\subsubsection{ガルニチュール・ノルマンディ風}\label{garniture-a-la-normande}}

\frsub{Garniture à la Normande}

\index{garniture@garniture!normande@--- à la Normande}
\index{normand@normand(e)!garniture@garniture à la  ---e}
\index{かるにちゆーる@ガルニチュール!のるまんていふう@---・ノルマンディ風}
\index{のるまんていふう@ノルマンディ風!かるにちゆーる@ガルニチュール・---}

(魚料理に添える)

\begin{itemize}
\item
  牡蠣10個とムール貝10個は沸騰しない温度で加熱して周囲を掃除する。
\item
  マッシュルーム(小)10個。
\item
  殻を剥いたエクルヴィスの尾の身100 g。
\item
  トリュフのスライス10枚。
\item
  クールブイヨンで火を通し、殻を剥かずに、はさみを後ろに回した\footnote{trousser
    (トゥルセ)。}中位のサイズのエクルヴィス10尾。
\item
  グジョンまたは小さなエペルランを尾の部分を残してパン粉衣を付けて揚げたもの10尾。
\item
  食パンを小さな菱形に切って提供直前に揚げたクルトン、または折り込みパイ生地を抜き型で三日月形や葉の形にして素焼きしたもの\footnote{fleurons
    de feuilletage cuit à blanc
    (フルロンドフイユタージュキュイタブロン)フルロンは折り込みパイ生地を三日月形や葉の形に抜いて、卵黄を塗って焼いたものだが、ここでは
    à blanc 「白く」とあるので説明的に訳した。}10個。
\item
  \protect\hyperlink{sauce-normande}{ノルマンディ風ソース}
\end{itemize}

\hypertarget{ux539fux6ce8}{%
\subparagraph{【原注】}\label{ux539fux6ce8}}

トリュフは省いてもいい。

\atoaki{}

\hypertarget{garniture-de-nouilles}{%
\subsubsection[ヌイユのガルニチュール]{\texorpdfstring{ヌイユ\footnote{卵入りの平麺。イタリアのフェットッチーネに近い。}のガルニチュール}{ヌイユのガルニチュール}}\label{garniture-de-nouilles}}

\frsub{Garniture de Nouilles}

\index{garniture@garniture!nouilles@--- de Nouilles}
\index{nouilles@nouilles!garniture@garniture de ---}
\index{かるにちゆーる@ガルニチュール!ぬいゆの@ヌイユの---}
\index{ぬいゆ@ヌイユ!かるにちゆーる@---のガルニチュール}

(牛、羊の塊肉、鶏料理に添える)

\begin{itemize}
\item
  生の\protect\hyperlink{nouilles}{ヌイユ}をやや固めに茹で、おろしたグリュイエールチーズとパルメザンチーズ各50
  gであえ、仕上げにバター50 gを混ぜ込む。
\item
  【別添】ブレゼの煮汁。
\end{itemize}

\atoaki{}

\hypertarget{garniture-opera}{%
\subsubsection[ガルニチュール・オペラ]{\texorpdfstring{ガルニチュール・オペラ\footnote{現在は「オペラ・ガルニエ」と呼ばれているシャルル・ガルニエ設計、
  1862年着工、1874年竣工の劇場のことを指す。第二帝政期に大々的に行なわれたパリ都市改造いわゆるオスマン計画においてエトワール広場の凱旋門とともに象徴的な建物。豪華な内装と、シャガールによる天井画で有名。現在も国立オペラ座として、1989年に完成したバスティーユのオペラ座とともに使用されている。なお、opéra
  (オペラ)というフランス語はもとはラテン語の opera
  (仕事、作品の意)。17〜18世紀にギリシア演劇の当時の解釈、ある意味では誤解から成立した音楽劇であり、イタリア語で「正式なオペラ」
  opera seria と呼ばれる4〜5幕の悲劇と、その幕間に上演された軽い内容の
  1幕または2幕の喜劇、「幕間劇」(イタリア語entratto フランス語
  entracteオントラクト)から発展したオペラ・ブッファopera buffa
  に大別される。オペラセリアとオペラ・ブッファの区別は18世紀前半のペルゴレージの『奥様女中』(1733年▽)、以後は、あまり明確な意味を持たなくなり、喜劇的内容の作品が規模的にも「幕合劇」の範疇に留まらず、独立した作品として扱われるようになった。チマローザ『秘密の結婚』(1792年)やモーツァルトの比較的後期のイタリアスタイルのオペラ『コジ・ファン・トゥッテ』(1790年)、ロッシーニの『ラ・チェネレントラ』(シンデレラ
  1817年)、ドニゼッティ『愛の妙薬』(1832年)などがそうである。また、モーツァルト『ドン・ジョヴァンニ』(1787年)などは喜劇的枠組みを利用しつつも、音楽的には悲劇的側面をも兼ね備えた傑作として名高い。その後、台本が喜劇的内容であれ悲劇的なものであれ、それまではひとつひとつが独立した曲としての体裁を保っていた歌手の独唱や重唱と、台詞に相当するレチタティーヴォの区別が19世紀後半、とりわけワーグナー以降あまり意味をなさなくなり、ヴェルディやプッチーニ以降のオペラではもはや判然としなくなり、リヒャルト・シュトラウスのように1作品全体でひとつの曲を見なさざるを得ない作品も増えてくる。ちなみにチョコレートケーキの一種であるオペラはオーストリア=ハンガリー帝国の首都ウィーンのオペラ座の外観に似ているという理由でつけられたウィーン菓子。}}{ガルニチュール・オペラ}}\label{garniture-opera}}

\frsub{Garniture Opéra}

\index{garniture@garniture!opera@--- Opéra}
\index{opera@Opéra!garniture@garniture ---}
\index{かるにちゆーる@ガルニチュール!おへら@---・オペラ}
\index{おへら@オペラ!かるにちゆーる@ガルニチュール・---}

(ノワゼット、トゥルヌドに添える)

\begin{itemize}
\item
  添える肉の大きさに合わせたサイズの空焼きしたタルトレットに、鶏レバーをソテーしてマデイラ風味に仕上げたものを詰める。
\item
  \protect\hyperlink{pommes-de-teres-duchesse}{じゃがいものデュシェス}をアパレイユとして成形し、パン粉衣を付けて揚げ、中をくり抜いてクルスタードを作る。そこに、バターであえたアスパラガスの穂先を詰める。
\item
  【別添】肉を焼いた後にデグラセした肉汁を、バターを加えて滑かな口あたりに仕上げてソースとする。
\end{itemize}

\atoaki{}

\hypertarget{garniture-a-l-orientale}{%
\subsubsection{ガルニチュール・オリエント風}\label{garniture-a-l-orientale}}

\frsub{Garniture à l'Orientale}

\index{garniture@garniture!oriental@--- à l'Orientale}
\index{oriental@oriental(e)!garniture@garniture à l'---e}
\index{かるにちゆーる@ガルニチュール!おりえんとふう@---・オリエント風}
\index{おりえんとふう@オリエント風!かるにちゆーる@ガルニチュール・---}

(鶏料理に添える)

\begin{itemize}
\item
  \protect\hyperlink{riz-a-la-grecque}{ギリシア風ライス}を小さな円筒形の型に詰めて10個のタンバルに仕立て、それぞれを半割りにして味付けし、ソテーしたトマトの上に盛る。
\item
  コルクの栓の形状にした、\protect\hyperlink{patates-douces}{さつまいも}のクロケット10
  個。
\item
  【別添】\protect\hyperlink{sauce-tomate}{トマトソース}。
\end{itemize}

\atoaki{}

\hypertarget{garniture-a-l-oreanaise}{%
\subsubsection[ガルニチュール・オルレアネ風]{\texorpdfstring{ガルニチュール・オルレアネ\footnote{Orléanais
  はロワール河中流域にある都市オルレアンを中心とした旧地方名。}風}{ガルニチュール・オルレアネ風}}\label{garniture-a-l-oreanaise}}

\frsub{Garniture à l'Orléanaise}

\index{garniture@garniture!orleanaise@--- à l'Orléanaise}
\index{orleanais@orléanais(e)!garniture@garniture à l'---e}
\index{かるにちゆーる@ガルニチュール!おるれあねふう@---・オルレアネ風}
\index{おるれあねふう@オルレアネ風!かるにちゆーる@ガルニチュール・---}

(家禽料理に添える)

\begin{itemize}
\item
  シコレをブレゼ\footnote{chicorée
    (シコレ)には、葉先が縮れるタイプのもの、すなわち chicorée
    frisée(シコレフリゼ) と縮れの少なく広葉の chicorée escarole
    (シコレエスカロール、単に scarole ともいう)、および chicorée de
    Bruxelles (シコレドブリュクセル)とも呼ばれる endive
    (オンディーヴ)の3タイプがある。最後のものは日本語で「アンディーブ」ともいうが、初夏に種を蒔いて、秋に根を掘り上げ、葉の根本を少し残して切り落してから暗室でピートモスか砂に根を植えて灌水し、温度管理しながら軟白栽培する結球タイプ。栽培方法としてはやや、イタリアのチコリア・ディ・トレヴィーゾ・タルディーヴォと似ているがイタリアのは水耕栽培により軟白する)。最初に挙げたシコレフリゼは日本では「エンダイブ」の名で流通しているが、ヨーロッパでは葉の細い品種が好まれる傾向にある。また、広葉のシコレ・エスカロールは日本であまり知られていないが、加熱調理にはこのタイプがもっとも適性がある。ブレゼの方法については、野菜料理、\protect\hyperlink{chicoree}{シコレ}の項参照。}して卵であえ、バターを混ぜ込む。
\item
  \protect\hyperlink{pommes-de-terre-a-la-maitre-d-hotel}{じゃがいも・メートルドテル}を野菜料理用深皿に盛り\footnote{この部分は、タンバル型に詰めて皿に裏返して型から外して供する、すなわち「タンバル仕立てにし」と解釈することも不可能ではないのだが、野菜料理において
    dresser en timbale (ドレセオンタンバール)は通常、
    \ul{野菜料理用深皿に盛り付ける}、の意であり、ここでは後者で訳した(『ラルース・ガストロノミック』初版
    timbale の項参照)。}、別添とする。
\item
  【別添】家禽をポシェまたはポワレ、ブレゼした際の煮汁をソースに仕上げる。
\end{itemize}

\atoaki{}

\hypertarget{garniture-de-haricots-panaches}{%
\subsubsection[ガルニチュール・いんげんのパナシェ]{\texorpdfstring{ガルニチュール・いんげんのパナシェ\footnote{panaché
  (パナシェ)盛り合わせ、相性のいい混ぜ合わせの意。 cf.~méli-mélo
  (メリメロ)混ぜ合わせ。なお、生ビールをリモナード(日本のサイダーに近い甘い炭酸飲料)で割ったものも
  panaché
  といい、比較的ポピュラーな飲み物。生ビールにグルナデンシロップを少量加えたものは
  monaco (モナコ)という。}}{ガルニチュール・いんげんのパナシェ}}\label{garniture-de-haricots-panaches}}

\frsub{Garniture de Haricots Panachés}

\index{garniture@garniture!haricots panaches@--- de Haricots Panachés}
\index{haricot@haricot!garniture@garniture de --- panachés}
\index{panache@panaché!garniture haricots@garniture de Haricots ---s}
\index{かるにちゆーる@ガルニチュール!いんけん@---・いんげんのパナシェ}
\index{いんけん@いんげん!かるにちゆーる@ガルニチュール・---のパナシェ}
\index{はなしえ@パナシェ!かるにちゆーるいんけん@ガルニチュール・いんげんの---}

アリコ・ヴェール 350 g とフラジョレ\footnote{小粒の白いんげん豆の品種。こちらは半熟(冷凍が多い)または完熟の乾燥品を水で戻してから煮る。また、アリコ・ヴェールとharicots
  beurre(アリコ・ブール、黄色さやいんげん、イエローワックスビーンズ、バターいんげん)を混ぜてバターであえるのも見た目がよく美味。その場合、アリコ・ヴェールとアリコ・ブールは太さを揃えること、最適な下茹で時間が異なることに留意する必要がある。}
350 gをバターであえる。みじん切りにしたパセリを振りかける。

\begin{itemize}
\tightlist
\item
  【別添】澄んだ\protect\hyperlink{jus-de-veau-lie}{ジュ}。
\end{itemize}

\atoaki{}

\hypertarget{garniture-a-la-parisienne}{%
\subsubsection{ガルニチュール・パリ風}\label{garniture-a-la-parisienne}}

\frsub{Garniture à la Parisienne}

\index{garniture@garniture!parisienne@--- à la Parisienne}
\index{parisien@parisien(ne)!garniture@garniture à la ---ne}
\index{かるにちゆーる@ガルニチュール!はりふう@---・パリ風}
\index{はりふう@パリ風!かるにちゆーる@ガルニチュール・---}

(牛、羊の塊肉、鶏料理に添える)

\begin{itemize}
\item
  \protect\hyperlink{pommes-de-terre-a-la-parisienne}{じゃがいも・パリ風}(\protect\hyperlink{legumes}{野菜料理}参照)600
  g。
\item
  アーティチョークの基底部10個はバターで蒸し煮し\footnote{étuver au
    beurre (エチュヴェオブール)}、\protect\hyperlink{saumure-liquide-pour-langues}{赤く漬けた舌肉}とマッシュルーム、トリュフを同量ずつ小さなさいの目に刻んで固めの\protect\hyperlink{veloute}{ヴルテ}であえた\protect\hyperlink{salpicons-divers}{サルピコン}をドーム状に盛る。強火のオーブンかサラマンダーに入れて焦げ目を付ける。
\item
  【別添】\protect\hyperlink{sauce-demi-glace}{ソース・ドゥミグラス}
\end{itemize}

\atoaki{}

\hypertarget{garniture-parmentier}{%
\subsubsection[ガルニチュール・パルマンティエ]{\texorpdfstring{ガルニチュール・パルマンティエ\footnote{アントワーヌ・オギュスタン・パルモンティエ(1737〜1813)。18世紀まであまりフランスでは省みられることのなかったじゃがいもの普及に尽力し、大きな功績をあげたことで知られる農学者。このため、じゃがいもを使用した料理にパルマンティエの名が冠せられたものが少なくない。ちなみに、じゃがいもが普及するにつれて、中世以来、でんぷん質野菜として好まれていたパースニップ(フランス語
  panais パネ)の生産量が19
  世紀には著しく低下することになる。これは、パースニップの方が単位面積あたりの収量が低く、栽培期間も長期にわたるという事情も関係したのだろう。『料理の手引き』においては香味野菜としての位置付けが主となっている。}}{ガルニチュール・パルマンティエ}}\label{garniture-parmentier}}

\frsub{Garniture Parmentier}

\index{garniture@garniture!parmentier@--- Parmentier}
\index{parmentiers@Parmentier!garniture@garniture ---}
\index{かるにちゆーる@ガルニチュール!はるまんていえ@---・パルマンティエ}
\index{はるまんていえ@パルマンティエ!かるにちゆーる@ガルニチュール・---}

(牛、羊の塊肉、鶏料理に添える)

\begin{itemize}
\item
  じゃがいも600
  gは均等なさいの目に刻むか、くり抜きスプーンを用いて楕円形に成形する。これを\protect\hyperlink{pommes-de-terre-chateau}{じゃがいも・シャトー}のようにバターで火を通す。パセリのみじん切りを振りかける。
\item
  【別添】澄んだ\protect\hyperlink{jus-de-veau-lie}{ジュ}。
\end{itemize}

\atoaki{}

\hypertarget{garniture-a-la-payasanne}{%
\subsubsection[ガルニチュール・ペイザンヌ]{\texorpdfstring{ガルニチュール・ペイザンヌ\footnote{paysan(ne)
  (ペイゾン/ペイザーヌ)農夫、農婦の意。}}{ガルニチュール・ペイザンヌ}}\label{garniture-a-la-payasanne}}

\frsub{Garniture à la Paysanne}

\index{garniture@garniture!paysanne@--- à la Paysanne}
\index{paysan@paysan(ne)!garniture@garniture à la ---ne}
\index{かるにちゆーる@ガルニチュール!へいさんぬ@---・ペイザンヌ}
\index{へいさんぬ@ペイザンヌ!かるにちゆーる@ガルニチュール・---}

(牛、羊の塊肉、鶏料理に添える)

\begin{itemize}
\tightlist
\item
  \protect\hyperlink{garniture-a-la-fermiere}{ガルニチュール・フェルミエール}に、にんにく形に剥いた小さなじゃがいもと、さいの目に切って下茹でした塩漬豚ばら肉を加える。
\end{itemize}

\atoaki{}

\hypertarget{garniture-a-la-peruvienne}{%
\subsubsection{ガルニチュール・ペルー風}\label{garniture-a-la-peruvienne}}

\frsub{Garniture à la Péruvienne}

\index{garniture@garniture!peruvienne@--- à la Péruvienne}
\index{peruvien@péruvien(ne)!garniture@garniture à la ---ne}
\index{かるにちゆーる@ガルニチュール!へるーふう@---・ペルー風}
\index{へるーふう@ペルー風!かるにちゆーる@ガルニチュール・---}

(トゥルヌドなどに添える)

\begin{itemize}
\item
  オカ芋\footnote{原文 oxalis
    (オグザリス)いわゆるオキザリス(カタバミソウ)の仲間で芋を食用とする品種。フランス語では
    oca du Péru
    (オカデュペリュ)ペルーのオカ、とも呼ばれる。生の状態では紫〜濃いピンク色だが、加熱するとこの色は失なわれる。「ペルー風」の名称はこれを用いていることによる。}は皮を剥き、安定がいいように底面を切り落とす。これの中をくり抜いてケースにする。鶏肉と生ハムを2対1の割合5
  mm角の細かいさいの目に刻む\footnote{原文 hachis (アシ)≒
    \protect\hyperlink{salpicons-divers}{salpicon (サルピコン)}。}。オカ芋のくり抜いた中身はみじん切りにする。合わせる料理により、\protect\hyperlink{sauce-demi-glace}{ドゥミグラス}か、よく煮詰めた\protect\hyperlink{sauce-allemande}{ソース・アルマンド}であえて、ケースに詰める。植物油少々を垂らしかけ、後は\protect\hyperlink{champignons-farcis}{詰め物をしたマッシュルーム}同様にオーブンで焼く。
\item
  【別添】薄い\protect\hyperlink{sauce-tomate}{トマトソース}。
\end{itemize}

\atoaki{}

\hypertarget{garniture-a-la-piemontaise}{%
\subsubsection{ガルニチュール・ピエモンテ風}\label{garniture-a-la-piemontaise}}

\frsub{Garniture à la Piémontaise}

\index{garniture@garniture!piemontaise@--- à la Piémontaise}
\index{piemontais@piémontais(e)!garniture@garniture à la ---e}
\index{かるにちゆーる@ガルニチュール!ひえもんてふう@---・ピエモンテ風}
\index{ひてもんてふう@ピエモンテ風!かるにちゆーる@ガルニチュール・---}

(牛、羊の塊肉、鶏料理に添える)

\begin{itemize}
\item
  \protect\hyperlink{rizotto-a-la-piemontaise}{リゾット} 1
  Lあたり、器具を使っておろした白トリュフ150
  gを加える。「ガトー・ド・リ\footnote{gâteau de riz
    (ガトドリ)。なお「餅」のことも gâteau de riz と呼ぶ。}」と呼ばれる楕円形の小さな型に詰めてタンバル仕立てにしたもの10個。
\item
  【別添】\protect\hyperlink{sauce-tomate}{トマトソース}。
\end{itemize}

\atoaki{}

\hypertarget{garniture-a-la-portugaise}{%
\subsubsection[ガルニチュール・ポルトガル風]{\texorpdfstring{ガルニチュール・ポルトガル風\footnote{トマトを使った料理にはこの名称が付けられることが多い。\protect\hyperlink{sauce-portugaise}{ポルトガル風そーす}および\protect\hyperlink{portugaise}{ポルチュゲーズ/トマトのフォンデュ}訳注参照。}}{ガルニチュール・ポルトガル風}}\label{garniture-a-la-portugaise}}

\frsub{Garniture à la Portugaise}

\index{garniture@garniture!portugaise@--- à la Portugaise}
\index{portugais@portugais(e)!garniture@garniture à la ---e}
\index{かるにちゆーる@ガルニチュール!ほるとかるふう@---・ポルトガル風}
\index{ほるとかるふう@ポルトガル風!かるにちゆーる@ガルニチュール・---}
\index{ほるちゆけーす@ポルチュゲーズ!かるにちゆーる@ガルニチュール・---}

(牛、羊の塊肉、鶏料理に添える)

\begin{itemize}
\item
  小さなトマト丸ごとに\protect\hyperlink{duxelles-seche}{デュクセル}を詰めたもの10個。
\item
  \protect\hyperlink{pommes-de-terre-chateau}{じゃがいも・シャトー}
  30個。
\item
  【別添】\protect\hyperlink{sauce-portugaise}{ポルトガル風ソース}
\end{itemize}

\atoaki{}

\hypertarget{garniture-a-la-printaniere}{%
\subsubsection[ガルニチュール・プランタニエール]{\texorpdfstring{ガルニチュール・プランタニエール\footnote{printanier/printanière
  (プランタニエ/プランタニエール)春の、の意。若どりのにんじん、蕪、アリコ・ヴェール、プチポワ、オニオンブランなどを用いた料理付けられる名称のひとつ。}}{ガルニチュール・プランタニエール}}\label{garniture-a-la-printaniere}}

\frsub{Garniture à la Printanière}

\index{garniture@garniture!printaniere@--- à la Printanière}
\index{printanier@printanier/printanière!garniture@garniture à la Printanière}
\index{かるにちゆーる@ガルニチュール!ふらんたにえーる@---・プランタニエール}
\index{ふらんたにえ@プランタニエ/プランタニエール!かるにちゆーる@ガルニチュール・プランタニエール}
\index{はるの@春の ⇒ プランタニエ/プランタニエール!かるにちゆーる@ガルニチュール・プランタニエール}

(鶏や牛、羊のソテーに添える)

\begin{itemize}
\item
  若どりのにんじん125 gと若どりの蕪125
  gは皮を剥いて成形し、コンソメで煮てからバターで色艶よく炒める。
\item
  オニオン・ヌーヴォー(小)20個はバターで色艶よく炒める\footnote{白系品種の新鮮なオニオン・ヌーヴォーならば下茹での必要はない。}。
\item
  プチポワ125 gとアスパラガスの穂先125 gは湯がいておく\footnote{とくにプチポワは若どりで新鮮な、豆の直径が8
    mm以下のものであれば、さっと湯通しして翡翠色に変わる程度でいいが、冷凍品の場合には、しっかり火を通すこと。}。
\item
  これらの野菜を添える肉とともに8〜10分間蒸し煮してすぐに供する。
\end{itemize}

\atoaki{}

\hypertarget{garniture-a-la-provencale}{%
\subsubsection{ガルニチュール・プロヴァンス風}\label{garniture-a-la-provencale}}

\frsub{Garniture à la Provençale}

\index{garniture@garniture!provencale@--- à la Provençale}
\index{provencal@provençal(e)!garniture@garniture à la ---e}
\index{かるにちゆーる@ガルニチュール!ふろうあんすふう@---・プロヴァンス風}
\index{ふろうあんすふう@プロヴァンス風!かるにちゆーる@ガルニチュール・---}

(牛、羊の塊肉の大掛かりな仕立てにに添える)

\begin{itemize}
\item
  丸ごとの小さなトマト10個。
\item
  にんにく1片を加えた\protect\hyperlink{duxelles-seche}{デュクセル}を詰めた大きなマッシュルーム10個。
\item
  【別添】\protect\hyperlink{sauce-provencale}{プロヴァンス風ソース}。
\end{itemize}

\atoaki{}

\hypertarget{garniture-de-purees}{%
\subsubsection{ピュレのガルニチュール}\label{garniture-de-purees}}

\frsub{Garniture de Purées}

\index{garniture@garniture!purees@--- de Purées}
\index{puree@purée!garniture@garniture de ---s}
\index{かるにちゆーる@ガルニチュール!ひゆれ@ピュレの---}
\index{ひゆれ@ピュレ!かるにちゆーる@---のガルニチュール}

\begin{itemize}
\tightlist
\item
  ガルニチュールとして用いることの出来る野菜のピュレはそれぞれの野菜料理の節に記してある(「\protect\hyperlink{legumes}{野菜料理}」の章を参照のこと。
\end{itemize}

\atoaki{}

\hypertarget{garniture-rachel}{%
\subsubsection[ガルニチュール・ラシェル]{\texorpdfstring{ガルニチュール・ラシェル\footnote{Rachel
  Félix (ラシェル・フェリクス
  1821〜1858)。一般には「マドモワゼル・ラシェル」と呼ばれた。主にテアトル・フランセ(現在のコメディ・フランセーズ)でラシーヌやコルネイユなどの悲劇を演じた、19
  世紀を代表する悲劇女優のひとり。}}{ガルニチュール・ラシェル}}\label{garniture-rachel}}

\frsub{Garniture Rachel}

\index{garniture@garniture!rachel@--- Rachel}
\index{rachel@Rachel!garniture@garniture ---}
\index{かるにちゆーる@ガルニチュール!らしえる@---・ラシェフ}
\index{らしえる@ラシェル!かるにちゆーる@ガルニチュール・---}

(ノワゼット、トゥルヌドに添える)

\begin{itemize}
\item
  中位の大きさのアーティチョーク20個\footnote{あらかじめ適切な処理をし、下茹でしておくこと。}の基底部に、沸騰しない程度の温度で火を通した\footnote{pocher
    (ポシェ)。}牛骨髄の大きなスライスを詰め、上からパセリのみじん切りを少々振りかける。
\item
  【別添】\protect\hyperlink{sauce-bordelaise}{ボルドー風ソース}。
\end{itemize}

\atoaki{}

\hypertarget{garniture-de-ravioles}{%
\subsubsection[ラビオリのガルニチュール]{\texorpdfstring{ラビオリ\footnote{ここでは
  ravioles
  (ラヴィオール、最後のsは明らかにフランス語の複数形)と綴られているが、参照先である原書p.779ではraviolisの綴りとなっている。パスタ類についてこうした表記の揺れは古くからのものだが、イタリア語のravioliはravioloの複数形であるから、フランス語の複数を示すsは本来的には余計とも言える。またraviolesの綴りは15世紀には既にフランス語の文献で確認されているので、どちらが正しいか、という議論はいささかナンセンスなものになってしまう。同様の例に、フランス語lasagnes(ラザーニュ)がある。これも最後のsが複数形であることを示しているが、もとになったイタリア語lasagne(ラザーニェ)それ自体がlasagna(ラザーニャ)の複数形である。ラテン語から派生したヨーロッパの言語(ロマンス語)は、名詞の複数形を
  s
  を付けることで表わすグループ(フランス語、スペイン語、ポルトガル語)と、語末の母音を変化させることによって単数、複数形を示すグループ(イタリア語、ルーマニア語)に大きく分けられる。前者は西ロマンス語、後者は南および東ロマンス語として分類されるのが一般的。英語およびドイツ語は俗ラテン語の影響を受けつつも、同じインド・ヨーロッパ語族の中ではゲルマン語派として分類されており、名詞の複数形の表わし方も本来的にはかなり違う(ex.
  ox \textgreater{} oxen)。}のガルニチュール}{ラビオリのガルニチュール}}\label{garniture-de-ravioles}}

\frsub{Garniture de Ravioles}

\index{garniture@garniture!ravioles@--- de Ravioles}
\index{ravioles@ravioles!garniture@garniture de ---}
\index{かるにちゆーる@ガルニチュール!らひおり@ラビオリの---}
\index{らひおり@ラビオリ!かるにちゆーる@---のガルニチュール}

(牛、羊の塊肉、鶏料理に添える)

\begin{itemize}
\tightlist
\item
  料理に合わせた内容の\protect\hyperlink{raviolis}{ラビオリ}
  30個を用意する(野菜料理、「\protect\hyperlink{farineux-et-pates-alimentaires}{パスタなど}」参照)。
\end{itemize}

\atoaki{}

\hypertarget{garniture-regence}{%
\subsubsection[ガルニチュール・レジャンス]{\texorpdfstring{ガルニチュール・レジャンス\footnote{摂政時代、すなわち18世紀初頭にオルレアン公フィリップが摂政政治を行なった時代のこと。\protect\hyperlink{sauce-regence}{ソース・レジャンス}訳注も参照。}}{ガルニチュール・レジャンス}}\label{garniture-regence}}

\frsub{Garniture Régence}

\index{garniture@garniture!regence@--- Régence}
\index{regence@Régence!garniture@garniture ---}
\index{かるにちゆーる@ガルニチュール!れしやんす@---・レジャンス}
\index{れしやんす@レジャンス!かるにちゆーる@ガルニチュール・---}

\textbf{A}. (魚料理に添える)

\begin{itemize}
\item
  メルラン\footnote{鱈の近縁種。}のファルスに\protect\hyperlink{beurre-d-ecrevisse}{エクルヴィスバター}を加え、スプーンで成形したクネル20個。
\item
  牡蠣10個は沸騰しない温度で火を通し\footnote{pocher (ポシェ)。}、周囲をきれいに掃除する\footnote{ébarber
    (エバルベ)。}。
\item
  オリーブ形に成形したトリュフ10個。
\item
  白子\footnote{フランス料理、とりわけオートキュイジーヌでは通常、鯉の白子を用いる。}を1
  cm程度に輪切りにして沸騰しない温度で火を通したもの10枚。
\item
  \protect\hyperlink{sauce-normande}{ノルマンディ風ソース}。
\end{itemize}

\textbf{B}. (鶏、仔牛胸腺肉料理に添える)

\begin{itemize}
\item
  トリュフを加えた鶏の\protect\hyperlink{farce-b}{滑らかなファルス}をスプーンで成形したクネル10個。
\item
  大きな丸いクネル2個にはトリュフで装飾を施す。
\item
  大きな、よく縮れた鶏のとさか10個。
\item
  厚さ 1cm程度の円形に切ったフォワグラ。
\item
  刻み模様を付けた小さなマッシュルーム10個。
\item
  オリーブ形にしたトリュフ10個。
\item
  トリュフエッセンスを加えた\protect\hyperlink{sauce-allemande}{ソース・アルマンド}。
\end{itemize}

\textbf{C}. (野鳥料理に添える)

\begin{itemize}
\item
  構成要素は上記Bと同じだが、クネルはジビエの滑らかなファルスを用いて、より小さく作ること。
\item
  トリュフエッセンス入り\protect\hyperlink{sauce-salmis}{ソース・サルミ}。
\end{itemize}

\atoaki{}

\hypertarget{garniture-renaissance}{%
\subsubsection[ガルニチュール・ルネサンス]{\texorpdfstring{ガルニチュール・ルネサンス\footnote{一般的には「文芸復興期」あるいは「人文主義の時代」の意味でルネサンスという言葉は用いられているが、re(再び、を表わす接頭辞)+
  naissance
  (生まれること)が語の成り立ちだから、本文にあるような解釈はある意味で当然とも言えよう。とりわけ冬から春にかけては「死と再生」の季節であり、四旬節が過ぎて復活祭の後、家畜が仔を産み、秋蒔きの麦が急激に成長し、晩冬=早春に蒔いた野菜類が若どりとはいえ穫れ始める時期としての春。その「はしり」(primeurs
  プリムール)を味わうことは非常に贅沢なことだったろう。ただし、ここで挙げられている食材の旬は、地域にもよるが概ね4月中旬以降にようやく「はしり」を迎えるものばかりであり、旬としての「盛り」は5月以降のものが多いことに注意。また、カリフラワーは本来、冬が旬なので、「名残り」にあたる。」}}{ガルニチュール・ルネサンス}}\label{garniture-renaissance}}

\frsub{Garniture Renaissance}

\index{garniture@garniture!renaissance@--- Renaissance}
\index{renaissance@Renaissance!garniture@garniture ---}
\index{かるにちゆーる@ガルニチュール!るねさんす@---・ルネサンス}
\index{るねさんす@ルネサンス!かるにちゆーる@ガルニチュール・---}

(牛、羊の塊肉の大掛かりな仕立てに添える)

このガルニチュールの構成要素は、この語本来の意味、つまり「再生」や「芽吹き」」が意味する通りのものだ。

だから、旬のはしりの食材をそれぞれに合った適切な方法で調理することで完成する。

満足のいく結果を得るには、大きな塊肉に添えるガルニチュール・ルネサンスは、それぞれの構成要素ごとにまとめて、はっきり目立つように盛り付けること。

\begin{itemize}
\item
  にんじんと蕪は模様付きの大きなくり抜きスプーンで成形し、コンソメで煮てからバターで色艶よく炒める。
\item
  若どりの細いアリコ・ヴェールは切らずに、あるいは半分に切るだけにしてまとめる。アリコ・ヴェールとプチポワはそれぞれバターであえて、まとめて盛り付ける。
\item
  アスパラガスの穂先の束。
\item
  カリフラワーの小房のまとまりには\protect\hyperlink{sauce-hollandaise}{オランデーズソース}を軽く塗り付けてやる。
\item
  バター風味の新じゃがいも。
\item
  【別添】塊肉を調理した煮汁。
\end{itemize}

\hypertarget{ux539fux6ce8-1}{%
\subparagraph{【原注】}\label{ux539fux6ce8-1}}

ガルニチュールの一部としてカリフラワーに軽くソースをかけてやるのが一般的な習慣だが、この方式は廃するべきとだと思う。

ソースをかけても、カリフラワーにはほとんど意味がなく、肉に添えられたジュ(肉汁)やブラウン系のソースとは不調和を起こすだけだ。カリフラワーはバターで蒸し煮\footnote{étuver
  (エチュヴェ)。}するだけで、白く自然な状態で供するべきだと考える。せいぜいが、オランデーズソースを別添で供するに留めるのがいいだろう。

\atoaki{}

\hypertarget{garniture-richelieu}{%
\subsubsection[ガルニチュール・リシュリュー]{\texorpdfstring{ガルニチュール・リシュリュー\footnote{フランス史においては、17世紀ルイ13世の時代に枢機卿だった
  cardinal de
  Richelieu(カルディナルドリシュリュ)が有名だが、料理の場合は、その甥であるアルマン・ド・ヴィニロー・ド・プレシ・ド・リシュリュー公爵(1696〜1788)およびその孫にあたるアルモン・エマニュエル・デュ・プレシ・ド・リシュリュー公爵(1766〜1822)の名が挙がる。とりわけ後者は王政復古期に首相となり、外務卿タレーランとともに食卓外交を展開したことで知られる。リシュリューの名を冠した料理はいくつもあり、上記3名のいずれを指しているか判然としないものも多い。このガルニチュールについては、19世紀に流行の食材だったトマトとじゃがいもを用いていることから、19世紀以降のものであり、そのためアルモン・エマニュエルの名を冠したと考えられるが、じゃがいもを「鳩の卵を模して」調理する点からは、18世紀料理に源がある可能性も否定出来ない。}}{ガルニチュール・リシュリュー}}\label{garniture-richelieu}}

\frsub{Garniture Richelieu}

\index{garniture@garniture!richelieu@--- Richelieu}
\index{richelieu@Richelieu!garniture@garniture ---}
\index{かるにちゆーる@ガルニチュール!りしゆりゆー@---・リシュリュー}
\index{りしうりゆー@リシュリュー!かるにちゆーる@ガルニチュール・---}

(牛、羊の塊肉の大掛かりな仕立てに添える)

\begin{itemize}
\item
  小さなトマト10個。
\item
  標準的な方法で作った中位の大きさの\protect\hyperlink{champignons-farcis}{マッシュルームの詰め物}
  10個。
\item
  小さなレチュ丸ごと、または半割りにした\protect\hyperlink{laitues-braisees-au-jus}{レチュのブレゼ}
  10個。
\item
  鳩の卵のサイズと形状に似せて成形したじゃがいも20個は提供直前にバターで火入れを仕上げる。
\item
  【別添】塊肉を焼いた鍋をデグラセし、軽くとろみを付けてソースに仕上げる。
\end{itemize}

\atoaki{}

\hypertarget{garniture-rohan}{%
\subsubsection[ガルニチュール・ロアン]{\texorpdfstring{ガルニチュール・ロアン\footnote{ブルターチュ地方の地名であると同時に、11世紀のその地に租を発する貴族の名家で高位聖職者、軍人、政治家など歴史に名を残す人物を多く輩出した。公爵、子爵などの分家が多く、モンバゾンもこの一族の領地のひとつ(\protect\hyperlink{garniture-montbazon}{ガルニチュール・モンバゾン}参照)。料理人マランが仕えたシャルル・ド・ロアン=スビーズ、通称スビーズ元帥(1715〜1787)も分家筋であるロアン=スビーズ家が出自(\protect\hyperlink{sauce-soubise}{ソース・スビーズ}参照)。}}{ガルニチュール・ロアン}}\label{garniture-rohan}}

\frsub{Garniture Rohan}

\index{garniture@garniture!rohan@--- Rohan}
\index{rohan@Rohan!garniture@garniture ---}
\index{かるにちゆーる@ガルニチュール!ろあん@---・ロオン}
\index{ろあん@ロアン!かるにちゆーる@ガルニチュール・---}

(鶏料理に添える)

\begin{itemize}
\item
  アーティチョークの基底部10個は底にグラスドヴィアンドを塗り、フォワグラの厚切りを詰め、上にトリュフのスライスをのせる。
\item
  空焼きしたタルトレット10個には\protect\hyperlink{sauce-allemande}{ソース・アルマンド}であえた雄鶏のロニョンを詰める。
\item
  よく縮れた雄鶏のとさか20はアーティチョークとタルトレットの間に配する。
\item
  【別添】マッシュルームのエッセンス入り\protect\hyperlink{sauce-allemande}{ソース・アルマンド}。
\end{itemize}

\atoaki{}

\hypertarget{garniture-a-la-romaine}{%
\subsubsection{ガルニチュール・ローマ風}\label{garniture-a-la-romaine}}

\frsub{Garniture à la Romaine}

\index{garniture@garniture!romaine@--- à la Romaine}
\index{romain@romain(e)!garniture@garniture à la  ---e}
\index{かるにちゆーる@ガルニチュール!ろーまふう@---・ローマ風}
\index{ろーまふう@ローマ風!かるにちゆーる@ガルニチュール・---}

(牛、羊の塊肉料理に添える)

\begin{itemize}
\item
  空焼きしたタルトレットに小さな\protect\hyperlink{gnokis-a-la-romaine}{ニョッキ・ローマ風}を詰めて強火のオーブンに入れて表面に焦げ目を付ける\footnote{gratiner
    (グラティネ)強力な上火だけのサラマンダーと呼ばれるオーブンや、ガスバーナーを用いてもいい。}。
\item
  ほうれんそうにさいの目に切ったアンチョビのフィレと卵黄を加え、模様付きのブリオシュ型に詰めて火を通した小さなパン\footnote{ここでのパンは「パンのような塊」の意であって、「ほうれんそう入りのパン」ではないことに注意。}、または、\protect\hyperlink{subric-d-epinards}{ほうれんそうのシュブリック}
  10個。
\item
  【別添】トマトピュレを \(\frac{1}{3}\)
  量加えた\protect\hyperlink{sauce-romaine}{ローマ風ソース}。
\end{itemize}

\atoaki{}

\hypertarget{garniture-rossini}{%
\subsubsection[ガルニチュール・ロッシーニ]{\texorpdfstring{ガルニチュール・ロッシーニ\footnote{ジョアキーノ・ロッシーニ(1792〜1868)。イタリアの作曲家。1824
  年にパリのイタリア座の音楽監督に就任し、29年には『ウィリアム・テル』と発表後、オペラ界から引退してイタリアに戻り隠居生活をする。晩年はパリに戻り、レストラン経営などもした。美食家としてとりわけ有名だが、作曲家としては速筆で知られている。オペラ『タンクレーディ』のアリア「ディ・タンティ・パルピティ」をリゾットを煮ているわずかの時間に作曲したという逸話も残っている。このガルニチュールについては\protect\hyperlink{tournedos-rossini}{トゥルヌド・ロッシーニ}も参照。}}{ガルニチュール・ロッシーニ}}\label{garniture-rossini}}

\frsub{Garniture Rossini}

\index{garniture@garniture!rossini@--- Rossini}
\index{rossini@Rossini!garniture@garniture ---}
\index{かるにちゆーる@ガルニチュール!ろつしーに@---・ロッシーニ}
\index{ろつしーに@ロッシーニ!かるにちゆーる@ガルニチュール・---}

(ノワゼット、トゥルヌドに添える)

\begin{itemize}
\item
  フォワグラを約1〜2
  cm厚にスライスして塩こしょうし、バターでソテーしたもの10枚。
\item
  トリュフの大きなスライス100 g。
\end{itemize}

【別添】トリュフエッセンス入り\protect\hyperlink{sauce-demi-glace}{ソース・ドゥミグラス}

\hypertarget{garniture-rossini}{%
\subsubsection[ガルニチュール・サンフロランタン]{\texorpdfstring{ガルニチュール・サンフロランタン\footnote{同名の地域名およびそこで作られているチーズの名称でもあり、また、同名のさくらんぼを用いた菓子もあるが、ここでは本文にあるように、\protect\hyperlink{pommes-de-terre-saint-florentin}{じゃがいも・サンフロランタン}を用いているのが名称の根拠。}}{ガルニチュール・サンフロランタン}}\label{garniture-rossini}}

\frsub{Garniture Saint-Florentin}

\index{garniture@garniture!saint-florentin@--- Saint-Florentin}
\index{saint-florentin@Saint-Florentin!garniture@garniture ---}
\index{かるにちゆーる@ガルニチュール!さんふろらんたん@---・サンフロランタン}
\index{さんふろらんたん@サンフロランタン!かるにちゆーる@ガルニチュール・---}

(牛、羊の塊肉料理に添える)

\begin{itemize}
\item
  \protect\hyperlink{pommes-de-terre-saint-florentin}{じゃがいも・サンフロランタン}(\protect\hyperlink{pommes-de-terre}{じゃがいも}参照)10個。
\item
  中位のセープ 300
  gの\protect\hyperlink{cepes-a-la-bordelaise}{ボルドー風ソテー}。
\end{itemize}

【別添】\protect\hyperlink{sauce-bonnefoy}{ボルドー風ソース・ボヌフォワ}。

\end{recette}

\index{garniture@garniture|)} \index{かるにちゆーる@ガルニチュール|)}
