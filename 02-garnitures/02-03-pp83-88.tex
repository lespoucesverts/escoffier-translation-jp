\href{未、原文対照チェック}{} \href{未、日本語表現校正}{}
\href{未、その他修正}{} \href{未、原稿最終校正}{}

\hypertarget{serie-des-appareiles-et-preparations-diverses-pour-garnitures-froides}{%
\section[冷製ガルニチュール用アパレイユなど]{\texorpdfstring{冷製ガルニチュール用アパレイユなど\footnote{この節は、初版で「冷製料理」の章の冒頭に概説としてまとめられていたものを、第二版の改訂時に、ほぼそのままの内容で現在の位置に移動させられている。もちろん順序および内容の加筆も行なわれており、異同は少なくない。}}{冷製ガルニチュール用アパレイユなど}}\label{serie-des-appareiles-et-preparations-diverses-pour-garnitures-froides}}

\frsec{Série des Appareils et Préparations diverses pour Garnitures froides}

\index{garniture@garniture!appareils garnitures froides@appareils et préparations diverses pour ---s froides}
\index{appareil@appareil!garnitures froides@--- et préparations diverses pour garnitures froides}
\index{かるにちゆーる@ガルニチュール!あはれいゆれいせい@冷製---のためのアパレイユなど}
\index{あはれいゆ@アパレイユ!れいせいかるにちゆーる@冷製ガルニチュールのための---など}

\hypertarget{mousses-mousselines-et-souffles-froids}{%
\subsection{冷製のムース、ムスリーヌ、スフレ}\label{mousses-mousselines-et-souffles-froids}}

\frsecb{Mousse, Moussseline, et Soufflé froids}

\index{mousse@mousse!froide@--- froide}
\index{mousseline@mousseline!froide@--- froide}
\index{souffle@soufflé!froid@--- froid}
\index{むーす@ムース!れいせい@冷製の---}
\index{むすりーぬ@ムスリーヌ!れいせい@冷製の---}
\index{すふれ@スフレ!れいせい@冷製の---}

温製の場合でも冷製の場合でも、\ul{ムースとムスリーヌはどちらも同じ材料から作られる}。

ムースとムスリーヌの違いは、温製でも冷製でも、通常は10人分が入る大きな型に詰めて作るのが\ul{ムース}と呼ばれ、いっぽう、\ul{ムスリーヌ}はスプーンで整形したり絞り袋を使ったり、あるいは大きなクネルの形をした専用の型に入れたりして作るが、基本的に\ul{1つ}で1人分と決まっている。スフレは小さなスフレ型に詰める。
\begin{recette}
\hypertarget{composition-de-l-appareil-pour-mousses-et-mousseline-froides}{%
\subsubsection{冷製のムースとムスリーヌのアパレイユ}\label{composition-de-l-appareil-pour-mousses-et-mousseline-froides}}

\frsub{Composition de l'Appareil pour Mousses et Mousseline froides}

\index{appareil@appareil!mousses mousselines froides@composition de l'--- pour mousses et mousselines froides}
\index{mousse@mousse!composition appareil froides@Composition de l'appareil pour --- et mousseline froides}
\index{mousseline@mousseline!composition appareil froides @Composition de l'appareil pour mousses et --- froides}
\index{あはれいゆ@アパレイユ!れいせいのむーすとむすりーぬ@冷製のムースとムスリーヌの---}
\index{むーす@ムース!れいせいむーすのあぱれいゆ@冷製の---とムスリーヌのアパレイユ}
\index{むすりーぬ@ムスリーヌ!れいせいむすりーぬのあぱれいゆ@冷製のムースと---のアパレイユ}

\begin{itemize}
\tightlist
\item
  材料\ldots{}\ldots{}主素材のピュレ\footnote{本書では加熱した肉や魚、甲殻類のピュレを作る方法への言及はないが、\textbf{本章冒頭にある\protect\hyperlink{farce-mousseline}{ファルス・ムスリーヌ}をそのまま使おうなどと考えてはいけない。ここで説明されている冷製のムース、ムスリーヌ、スフレの作り方に加熱の工程がまったく含まれていないのは、主素材のピュレが既に加熱済みであることを当然の前提としている}からだ。つまりここで材料として示されているピュレは\textbf{すべて加熱済みのものをピュレにしたものだ}と考えなければならない。『料理の手引き』の当時はローストするか茹でるなどの加熱後に、鉢に入れてすり潰し、裏漉ししてから何らかのソース(ここではヴルテ)を加えて漉さ(固さ)を調節するなどしていた。現代ではフードプロセッサーや冷凍粉砕調理機などを利用すればより容易に滑らかなピュレを作ることが可能だろう。また、第3章ポタージュに\protect\hyperlink{les-purees}{ポタージュ・ピュレ}についての概説があるが、そこではポタージュにすることを前提として「つなぎ」の使用が作業のプロセスに組込まれて説明されているために、あくまで参考程度に読むのがいいだろう。}1
  Lすなわち鶏のピュレ、ジビエ、フォワグラや魚、甲殻類のピュレ。溶かした\protect\hyperlink{gelees-ordinaires}{ジュレ}2
  \(\frac{1}{2}\) dL、\protect\hyperlink{veloute}{ヴルテ}4
  dL、生クリーム4 dLはちょうどいい固さに立てて6 dL相当にしておく。
\end{itemize}

素材の特性によって、これらの分量比率は多少変更してもいい。同様に、ある種のムースを作る際にはジュレまたはヴルテのどちらかしか用いなくてもいい。

\begin{itemize}
\tightlist
\item
  作業手順\ldots{}\ldots{}まずベースとなるピュレを入れたボウルを氷の上に置いて、軽く混ぜながら、ジュレとヴルテを加える(どちらかしか使わない場合は使うもののみ)。次に泡立てた生クリームを加える。
\end{itemize}

味付けを確認する。これは冷製料理ではとても重要なことだ。いつも気をつけて確認し、修正を加えるようにすること。

\hypertarget{nota-composition-de-l-appareil-pour-mousses-et-mousseline-froides}{%
\subparagraph{【原注】}\label{nota-composition-de-l-appareil-pour-mousses-et-mousseline-froides}}

生クリームは五分立てすること。完全に立ててしまうと、ムースに滑らかさが失なわれてパサついた仕上りになってしまう。

\hypertarget{moulage-des-mousses-froides}{%
\subsubsection{冷製ムースの型詰め}\label{moulage-des-mousses-froides}}

\frsub{Moulage des Mousses froides}

\index{moulage@moulage!mousses foirdes@--- des mousses froides}
\textbackslash{}index\{mousse@mousse!moulage froides@moulage des ---s
froides \index{かたつめ@型詰め!れいせいむーす@冷製ムースの---}
\index{むーす@ムース!れいせいかたつめ@冷製---の型詰め}

いまもそうしている料理人は少なくないようだが、かつては、プレーンな型あるいは浮き彫り模様の付いた型の中に透明なジュレを流して層をつくってやり\footnote{chemiser
  (シュミゼ)ジュレなどを型の内側に流して薄い層を作ること。}、ムースの主素材と関連あるものを装飾要素として貼り付けていた。

こんにちでは次の方法がむしろ好ましい。銀製のタンバル型\footnote{timbale
  (タンバル)円筒形の比較的浅い型。野菜料理用の深皿もこの語で呼ぶので注意。}の底面だけに透明なジュレの薄い層をつくる。型の側面の外側に紙の帯を冷たいバターで貼り付ける。型の\ruby{縁}{ふち}から2〜3
cmくらい高くなるようにすること。そうするとスフレのような見た目のムースになる。紙の帯は型の内側に貼り付けてもいい。この紙の帯は提供直前に、ぬるま湯で濡らしてナイフの刃を使ってムースからそっと引き剥してやる。

タンバル型の用意が整ったら、ムースを詰めて冷やす。アイスクリーム用の冷凍庫に入れるほうがいいだろう。この方法は、小さな銀製のスフレ型に詰めてやってもいいが、それは冷製のスフレにとっておいたほうがいいだろう。アパレイユの構成が同じであるにもかかわらず、冷製ムースと冷製スフレの違いをはっきりさせることが出来るからだ。

とりわけジビエのムースやフォワグラのムースについては、近代的な料理の提供方法に合わせて作られた銀製かガラス製の容器を用いてもいい。その場合は、型の底面だけジュレの層をつくってやり、アパレイユをそのまま流し込めばいい。表面はパレットナイフなどで丁寧に滑らかにならしてやってから、ムースを冷やす。その後、ムースに直接装飾を施し、ジュレをかけて艶を出させる。

ジビエのムースの場合には、そのジビエの胸肉を冷やして、ムースの周囲に飾るようにする。

\hypertarget{moulage-des-mousselines-froides}{%
\subsubsection{冷製ムスリーヌの型詰め}\label{moulage-des-mousselines-froides}}

\frsub{Moulage des Mousselines froides}

\index{moulage@moulage!mousselines froides@--- des mousselines froides}
\index{mousseline@mousseline!moulage froides@moulage des --- froides}

\index{かたつめ@型詰め!れいせいむすりーぬ@冷製ムスリーヌの---}
\index{むすりーぬ@ムスリーヌ!れいせいかたつめ@冷製---の型詰め}

冷製ムスリーヌの型詰めには2つの方法がある。たんに、型にジュレの層を作ってやるか、ソース・ショフロワの層を作ってやるかの違いでしかない。どちらの場合でも、卵形の型に詰めるか、大きなクネルの形状のものにするか、ということになる。

\hypertarget{procede-un-moulage-des-mousselines-froides}{%
\subparagraph{方法1\ldots{}\ldots{}}\label{procede-un-moulage-des-mousselines-froides}}

型の内側に透明なジュレを流して薄い層を作ってやる\footnote{chemiser
  (シュミゼ)。}。その上にアパレイユを張るように塗り、アパレイユのベースとなっている素材とおなじもの---
鶏、ジビエ、甲殻類の身など、とトリュフ---で構成された\protect\hyperlink{salpicons-divers}{サルピコン}を盛り込む。その上からアパレイユを塗って覆い、パレットナイフなどを使ってドーム形に滑らかにならす。冷蔵庫に入れて冷し固める。

\hypertarget{procede-deux-moulage-des-mousselines-froides}{%
\subparagraph{方法2\ldots{}\ldots{}}\label{procede-deux-moulage-des-mousselines-froides}}

型の内側にアパレイユを詰め、さらにサルピコンをその内側に射込む。アパレイユで覆って、冷し固める。

型から外す。ムスリーヌのアパレイユの素材と関連性のある\protect\hyperlink{sauce-chaud-froid-ordinaire}{ソース・ショフロワ}を表面を覆うように塗る\footnote{napper
  (ナペ)。覆いかける(ように塗る)こと。}。トリュフおよびその他の素材(これもムスリーヌと関連性があること)を装飾用に細工したものを飾り付ける。装飾が剥れないように、上からジュレを塗って艶を出させる。

銀製またはガラス製の深皿の底に透明なジュレの層を作り、その上にムスリーヌを並べる。再度ジュレを上からかけてやり、冷蔵庫に入れて提供するまで保管しておく。

\hypertarget{souffles-froids}{%
\subsubsection{冷製スフレ}\label{souffles-froids}}

\frsub{Soufflés froids}

\index{souffle@soufflé!froid@---s froids}
\index{すふれ@スフレ!れいせい@冷製---}

冷製スフレはムースそのものに他ならない。だから構成はまったく同じだ。ただ、先に見たようにスフレが10人分\footnote{1
  service
  (アンセルヴィス)、格式のある宴席料理などを作る際の単位。基本は10人分。}を確保できるだけの大きな型に詰めるのに対して、スフレはそもそも、小さなスフレ型に入れてひとり1つ宛で作るものだ。

アパレイユを型に詰める方法はムースの場合と同様、つまり、スフレ型の底にジュレの層を敷いてその上にアパレイユを盛り、型の縁より高くなるように周囲に巻いた紙の帯を利用して縁より高くアパレイユを盛る。そうすると、冷やし固めた後で紙の帯を取り除けば、まるで温製のスフレのように見えることになる。

\hypertarget{nota-souffles-froids}{%
\subparagraph{【原注】}\label{nota-souffles-froids}}

ここまで述べた3種の作り方の基礎はおなじだから、ポイントは次のようにまとめられる。

\begin{enumerate}
\def\labelenumi{\arabic{enumi}.}
\item
  ムースは「スフレ」の名称で供してもいいものだが、混同されるのを避けるために「ムース」の名称で約10人分をひとつの型に入れて作る。
\item
  ムスリーヌはサルピコンを射込んだものであってもそうでなくても、大きなクネルであって、ひとりあたり1つにする。
\item
  スフレは小さなムースであって、スフレ型あるいは似たような型に詰めて、これもひとりあたり1つとする。
\end{enumerate}
\end{recette}
\hypertarget{aspics}{%
\subsection[アスピック]{\texorpdfstring{アスピック\footnote{いわゆる「ゼリー寄せ」のやや大掛かりな仕立てだが、aspicという語は本来、フランスやイタリアに生息する蛇の名称。本文にあるように高さがあり中央に穴のあいたリング形にジュレとともに具材を詰め、装飾をおこなうが、その完成した姿が、アスピックという蛇がとぐろを巻いた姿を思わせるというところから付けられた名称といわれている。ジュレが柔らかいものであればそれだけ、大きな型に入れる場合、中空になっているリング型を用いないと自重で崩壊することになる。逆にいえば、リング型を使うのは自重で崩壊するのを防ぐための経験的な知恵なのだろう。}}{アスピック}}\label{aspics}}

\frsecb{Aspics}

\index{aspics@aspics!generalite@généralité}
\index{あすぴつく@アスピック!かいせつ@概説}

アスピックを作る際に、肝に銘じておくべき第一のポイントは、どんなアスピックでも、ジュレがジューシー\footnote{原文succulent(スュキュロン)はsuc(スュック=肉汁)から派生した形容詞で、もともとは「汁気の多い」の意味だったが、そこから転じて「美味な、滋味に富んだ」の意味で一般的に用いられている。ここでは、両方のニュアンスで表現されていると解釈できる。}で美味しく、完全に透き通ったもので、ちょうどいい加減に固まっていなければならないことである。

アスピックを作る際には、昔もそうだったが現代でも、中央に穴の空いたアスピック型\footnote{moule
  à douille (ムーラドゥイユ)サヴァラン型のような中央に穴が空いた型。
  現代では「アスピック型」というと楕円形で中央に穴のないものを指すことが多いが、それとは異なる。あるいはクグロフ型のようなものをイメージするとわかりやすいだろう。19世紀、アスピックには高さのある型が多く用いられたようだ。なお、現代では一般にサヴァラン型というと、型の高さや穴の大きさ等さまざまなタイプのものをまとめて指すことになるので注意。高さのない(低い)、中央の穴が大きな型について、エスコフィエはボルデュール型
  moule à bordure (ムーラボルデュール)と呼んで区別している。}でプレーンなもの、波模様等の装飾のあるものが用いられている。

ボルデュール型\footnote{moule à bordure
  (ムーラボルデュール)料理の縁り飾りを作るための、やや丈が低く中央の穴が大きいリング型。}も使われることがあるが、一般的に、アスピックの中心にガルニチュールを盛り込む場合のみである。

アスピックを型に入れる時には、まず、型の底と周囲に装飾をする。

そのために、型は砕いた氷の中に入れてよく冷やしておく。やや固まりかけたジュレ少量を流し入れ、型を氷の上で転がしながらジュレを周囲に貼り付かせる\footnote{chemiser
  (シュミゼ)。}。次に、装飾するパーツを、固まらない程度に冷たいジュレに浸してからすぐに貼り付ける。装飾については料理人のセンスとアイデア次第なので、ここで明確に述べておくべきことはほとんどない。ひとつだけ言えるのは、常に正確な作業をし、型からアスピックを出したときに装飾がはっきりと見えるようにすべき、ということのみ。

装飾に用いる素材はアスピックの主素材と関連性のあるものでなくてはならない。一般的には、トリュフ、ポシェした卵白、コルニション\footnote{cornichon
  主としてピクルスにする小型のきゅうり、およびそのピクルスのこと。日本では、ハンバーガーによく用いられているドイツ系のピクルス用品種であるガーキンス(英
  gherkins 独
  Einlegegurken)と混同されることがあるが、コルニションはより小さなサイズで収穫し、フレッシュな状態では「いぼ」が尖っているのが特徴。}、ケイパー、いろいろな香草の葉先、ラディッシュの薄い輪切り、オマールのコライユ\footnote{胴の背側にあるオレンジ色がかった「内子」。}、\protect\hyperlink{saumure-liquide-pour-langues}{赤く漬けた舌肉}、等。

アスピックの具材が種々のエスカロップ\footnote{escalope
  (エスカロップ)筋線維とは垂直方向に、厚さ1〜2
  cmに薄切りにした仔牛などの肉や魚の薄い切り身。}や長方形に切ったフォワグラ等で、型の大きさから何度も並べなければならない場合、ジュレの層と交互に重ねて型に入れていく。新しい層を並べる際には先に入れたジュレがある程度固まってからにする。

アスピックの型入れでは常に、最後のジュレの層を充分な厚みにする。できるだけ、型を氷に埋めるようにしながらジュレを流し込んでいくが、早く冷やすために氷に塩を加えてはいけない。塩を使うとジュレの透明さが損なわれるからである。

\noindent\textbf{型から外す方法}\ldots{}\ldots{}型を湯につけてただちに水気を拭い、折ったナフキンや彫刻した氷のブロック等に、アスピックを裏返して型から出す。

菱形や正方形に切ったジュレのクルトン\footnote{パンで作るクルトンとは別に、菱形やさいの目に切った冷製料理装飾用のジュレもクルトンと呼ぶ。}、またはアシェしたジュレで周囲を飾る。

\hypertarget{nota-aspics}{%
\subparagraph{【原注】}\label{nota-aspics}}

アスピックを型に入れて作るには、必然的に、ジュレが相当に固いものでなければならないが、これはまことによろしくない。そもそも固いジュレは口あたりがよくないのだ。だから現代の調理現場では、以下のような方法を採っている。タンバル型か、氷に嵌め込むようにした銀やガラスあるいは陶製の深皿の底に予めジュレの層を作って固めておき、その上にアスピックの素材を並べる。次に、固まりかけのジュレをたっぷり覆いかける。この方法では、装飾をする必要がある場合は、アスピックの調理をおこなう前に、主素材にじかに装飾することになる。

\hypertarget{chauds-froids}{%
\subsection[ショフロワ]{\texorpdfstring{ショフロワ\footnote{ショフロワという仕立てについては\protect\hyperlink{sauce-chaud-froid-brune}{茶色いソース・ショフロワ})訳注参照。なおこのchaud-froidという語の複数形は、それぞれにsを付けchauds-froidsとなる。合成語の複数形はいろいろなパターンがあるので、必要が出たらその都度覚えるようにしたほうがいい。}}{ショフロワ}}\label{chauds-froids}}

\frsecb{Chauds-froids}

\index{chaud-froid@chaud-froid!generalite@généralité}
\index{しよふろわ@ショフロワ!かいせつ@概説}

\protect\hyperlink{sauce-chaud-froid-ordinaire}{ソース・ショフロワ}には大抵の場合、切り分けた素材を浸す。が、時として大きな塊肉全体をソース・ショフロワで覆わなくてはならない場合もある。ただ、そういう仕立てにする場合には、別の料理名となっている。

ショフロワが複数のばらばらのパーツからなる場合には、それらをソース・ショフロワに漬けたら網の上に並べておく。ソースが冷えたら、それぞれのパーツに装飾をし、ジュレを覆いかけて艶を出してやる。さらに盛り付けの際にはみ出す余分なソースについてはきれいに取り除いておくこと。

大きな塊肉の場合は、よく冷えてはいるけれどまだ流動性のある状態のソース・ショフロワを一気に塗りつけて、その後に装飾をし、ジュレを塗って艶出しすること。

切り分けた素材からなるショフロワの盛り付けは、\protect\hyperlink{fonds-de-plats}{皿の上の台}の上に盛り付けてもいいし、縁飾りの内側に、パンまたは米、セモリナ粉で作った台を置いてその上に盛り付けてもいい。あるいは、銀製か陶製、ガラス製の深皿に盛り付けてもいい。

大きな塊肉のショフロワの場合、皿の上の台にのせてもいいし、あるいは、氷のブロックに料理が嵌まるようにブロックを削ってからそこに盛り付けるのもいい。

ショフロワ仕立ての鶏やジビエについては、正確に切り分けて\footnote{基本的に鶏および鳥類のジビエの可食部は胸肉のみとされていたことに留意。}皮は剥いでおくこと。手羽や下腿肉は使わないので、別の用途に取り置いておくといい。

細かく切った素材のショフロワ仕立ての場合、添えてやるマッシュルームや雄鶏のとさかとロニョン\footnote{rognon
  (ロニョン)牛、羊などの場合は腎臓だが、雄鶏の場合は精巣のこと。高級食材として珍重された。}にもソース・ショフロワを塗ってやること。トリュフはただジュレをかけて艶を出すだけでいい。

\hypertarget{pains-froids}{%
\subsection[パンフロワ]{\texorpdfstring{パンフロワ\footnote{pain froid
  直訳すると「冷たいパン」だが、いわゆるパンとはまったく違う。語の概念としては「パンに似た塊」のこと。}}{パンフロワ}}\label{pains-froids}}

\frsecb{Pains froids}

\index{pains froids@pains froids!generalite@généralité}
\index{はんふろわ@パンフロワ!かいせつ@概説}

古典料理におけるパンフロワとは、ファルスで出来たアパレイユを型に詰めて比較的低温で加熱調理し、冷ましてから型から出して装飾を施し、ジュレをかけて艶を出させたものでしかない。

近代の料理においてこの方法は用いられなくなっており、一般的にいって、パンフロワの代わりとしてムースが作られるようになったわけだ\footnote{この段落は第四版でかなり分量が減らされ、内容も書き換えられている。結果として大きく削られた後半部分の初版の文章は以下のとおり。「(近代の)パンフロワはいずれも、その中心となるアパレイユが次の構成になる。(1)そのパンフロワの主素材からひいた香りゆたかなフュメをほとんどグラス状に煮詰めたものと、卵黄とバターを\protect\hyperlink{sauce-hollandaise}{オランデーズソース}のように立てたもの。(2)このアパレイユが温いかどうかくらいまで冷めたら、溶かしたゼラチンを布で漉しながら流し入れ、さらに主素材のピュレと、それと同量の泡立てた生クリームを加える。
  (3)最後にこのアパレイユに、主素材から切り出した薄切り肉(エスカロップ)にトリュフのスライスを重ねていく。あるいは単純に、肉とトリュフをさいの目に切ったものでもいい。このようにして作ったアパレイユを、あらかじめジュレを内側に流して層を作っておいた型に流し入れ、冷やす、もしくは氷室に入れる。提供直前に、ぬるま湯にさっと型を浸していから米かセモリナ粉で作った台の上に裏返してのせてやる。あるいは皿の底にジュレを敷いただけでもいい。このパンフロワの周囲に、きちっと正確な形状に切ったジュレのクルトンを飾る。【原注】ジュレによるクルトンについては、冷製料理全般にあてはまる」(p.582)。}。

\hypertarget{garnitures-de-mets-froids}{%
\subsection{冷製料理のガルニチュール}\label{garnitures-de-mets-froids}}

\frsecb{Garnitures de Mets froids}

\index{garnitures mets froids@garnitures de mets froids (généralité)}
\index{れいせいりようりのかるにちゆーる@冷製料理のガルニチュール(概説)}

料理に合わせて、ガルニチュールは以下のようなもので構成すること。

\begin{itemize}
\tightlist
\item
  固茹で卵を半割りまたは四つ割りにして詰め物をし、装飾を施してジュレをかけて艶を出したもの
\item
  小さなトマトファルシが、いろいろな食材を添えたもの、または大きなトマトに何らかの詰め物をして正確に櫛切りにしたもの
\item
  小さな野菜皿または舟形の皿に盛った野菜サラダ
\item
  トマトピュレにジュレを混ぜて塗った小さなパンまたはタルトレット
\item
  真っ白なレチュ\footnote{いわゆる「サラダ菜」に属する系統の結球レタスのこと。}の中心部分
\item
  アンチョビのフィレ、オリーブなど\ldots{}\ldots{}
\end{itemize}

\newpage

\hypertarget{serie-des-garnitures}{%
\section{ガルニチュール}\label{serie-des-garnitures}}

\frsec{Série des Garnitures}

\hypertarget{consideration-sur-la-modification-de-forme-que-peuvent-subir-les-garniture}{%
\subsection{ガルニチュールの見た目を変えることについて}\label{consideration-sur-la-modification-de-forme-que-peuvent-subir-les-garniture}}

\vspace{-1\zw}
\begin{center}
\textit{Considérations sur le modifications de forme que peuvent subir les Garnitures.}
\end{center}
\vspace{1\zw}

\index{garniture@garniture!consideration modification forme@Considération sur les modificqtions de forme que peuvent subir les ---s}
\index{かるにちゆーる@ガルニチュール!みためをかえることについて@---の見た目を変えることについて}

他のどんなレシピでもそうだが、それぞれのガルニチュールの構成上の約束事を勝手に変えてはいけない。もし、どうしても何らかの変更が必要なら、料理本体に合わせて、配置を変えるとか、見た目の形状を変えるだけにすること。ガルニチュールを構成している素材を変えてはいけない。

そうすれば、「牛フィレ肉」のような大きな塊で供する料理か、「トゥルヌド
\footnote{牛フィレ肉を厚さ約2
  cmに切ったもの。周囲に豚背脂のシートを巻いて調理することが多いが、アメリカもしくはイギリス経由で周囲にベーコンを巻く調理法が日本に伝わったために、混同されやすいので注意。\ul{フランス料理としては、豚背脂\\のシートを巻く}。}」のような調理かにかかわらず、同じガルニチュールを合わせることが出来るが、その場合は必然的に、ガルニチュールの形状や盛り付けにおける配置などは変更せざるを得ないわけだ。そうしないと、主素材とガルニチュールの関係性が保てなくなる。

これは、薄切りにしたフィレ肉とシャトーブリヤン\footnote{牛フィレ肉の太い部分、およびそれを約3
  cmの厚さに切ったもの。}の場合も同様だ。理屈からいって当然だろう。

だから、この節において示しているガルニチュールの分量は10人分を基本としているが、大きな塊肉の料理に添えるか、1人分ずつに切って調理して供するかで、量を増やしたり減らしたりすることになる。

これはとても重要なことだ。というのも、本書はフランス料理の伝統的な作り方を集めた本なのだから、多種多様なガルニチュールを収録せざるを得なかったが、その中には近代的な料理にはもはやふさわしくないものだって含まれている。近代的な料理は何よりもまず複雑さを\ruby{厭}{いと}い、ガルニチュールをシンプルなものにする傾向にある。そうすれば皿出しが早くなるし、結果は完璧だ。料理というのは熱々の状態で供されてこそ、完璧な状態で味わっていただけるものだ。ガルニチュールがごくシンプルなものなら、素早い盛り付けにも対応出来る。

同様に、もし可能なら、ガルニチュールを料理の周囲に配置するよりは、別添で供したほうがいいだろう\footnote{大皿に約10人分をまとめて盛り付けるケースを想定して言っていることに留意。}。そうすればどんな料理であっても、本体は事前に切り分けて、ソースにまみれていない状態で盛り付けられた姿を、お客様方にご覧いただくことが可能だ。それからすぐにガルニチュールとソースを回していけばいい。この方式以外に、盛り付けを素早くおこない、清潔で熱々の状態で料理をご提供する手段はなかろう。

これはとりわけ、ルルヴェ\footnote{\protect\hyperlink{releve}{第二版序文訳注}および本章「\protect\hyperlink{farces}{ファルス}」訳注参照。}と呼ばれる大掛かりな仕立ての料理の場合にあてはまることだ。ノワゼット\footnote{約80
  gの牛フィレ肉の筒切り、および、円筒形に切った羊、仔羊の背肉の中心部分。}やトゥルヌドのようなさして大規模ではない仕立てのアントレ\footnote{Entrée
  現代フランス語では「前菜」のことを指すが、かつては約10人分を大きな皿にまとめて盛り付け、給仕の際に取り分ける肉料理(さらに古くは魚料理も)を意味していた。}と呼ばれる料理については、給仕の際に切り分けてガルニチュールを盛り付けてからお客様にお出しするよりは、おひとり様分ずつ盛り付けて供することにすれば、「アントレ」の存在理由はますます低いものとなる。

それでも、アントレについてはそうしたほうがいい。この問題に関しては、料理本体の盛り付けとガルニチュールを切り離したほうが、毎回確実により早く料理をご提供できるのだから、どんな盛り付けの料理だろうと、ぜひためらうことなくこの方式を採用していただきたい。

\hypertarget{remarque-importante-sur-les-sauces-applicables-aux-entrees-de-boucherie-garnie-de-legumes}{%
\subsection{牛、羊肉料理に野菜を添える場合にふさわしいソースについて}\label{remarque-importante-sur-les-sauces-applicables-aux-entrees-de-boucherie-garnie-de-legumes}}

\vspace{-1\zw}
\begin{center}
\textit{Remarque importante sur les sauces applicables aux Entrées de Boucherie garnies de Légumes.}
\end{center}

\index{garniture@garniture!remarque sauce entrees boucherie legumes@Remarque importante sur les sauces applicables aux Entrées de Boucherie garnies de Légumes}
\index{かるにちゆーる@ガルニチュール!うしひつしにくりようりにやさいをそえるはあいにふさわしいそーす@牛、羊肉料理に野菜を添える場合にふさわしいソースについて}

\vspace{1\zw}

\protect\hyperlink{sauce-espagnole}{エスパニョル}系の派生ソースは野菜を添えた牛、羊肉料理にはふさわしくない。\protect\hyperlink{jus-de-veau-lie}{とろみを付けたジュ}のほうが圧倒的にいい。

だが、いちばんいいのは、軽く仕上げた\protect\hyperlink{glace-de-viande}{グラスドヴィアンド}1
dLに125 gのバターを加えて\footnote{ソースを仕上げる際にバターを加えてより滑らかで艶やかな仕上がりにする。monter
  au beurre (モンテオブール)日本ではブールモンテとも呼ばれる。}、レモン果汁ほんの数滴で仕上げたものだ。とはいえ、このバターを加えたグラスドヴィアンドは野菜を包み込んでしまわない程度に充分に軽い仕上がりにすること。

アスパラガスの穂先とかプチポワ\footnote{petits pois
  いわゆるグリンピースのことだが、フランスではより若どりの、直径7〜8
  mm程度のものが好まれる。}、アリコヴェール\footnote{haricots verts
  いわゆる、さやいんげん。これもごく細い若どりのもの(太さ8〜9
  mm)程度のものが好まれる。}、マセドワーヌ \footnote{macédoine
  多くの場合、小さめのさいの目に切った蕪(navet
  ナヴェ)やアリコヴェール、プチポワ、にんじんなどを混ぜ合わせたもの。日本のマセドアンサラダの原型となった。ただし言葉の意味としては「各種の野菜を混ぜあわせたもの」であり、料理用語として切り方が決まっているわけではない。}などの野菜は、ソースをある意味、分解してしまう。それは野菜そのものが持つ水分によってだったり、野菜をあえているアパレイユのせいだったりする。

その結果、大皿から取り皿に分けてお客様のところに運ばれた時には、ほとんど食欲を失なわせるような見た目になってしまう。こういう事態は\protect\hyperlink{sauce-chateaubriand}{ソース・シャトーブリヤン}か、バターを加えたグラスドヴィアンドを料理に合わせれば解決する。これらのソースは分解しないどころか、野菜のガルニチュールととてもよく合う。同時に、野菜のガルニチュールにもこれらのソースはとても素晴らしいふんわりとした食感を与えてくれるからだ。

そんなわけで、以下の点にぜひとも留意していただきたい。出来るだけ、エスパニョル系の派生ソースやトマトソースは、\protect\hyperlink{garniture-a-la-financiere}{ガルニチュール・フィナンシエール}や\protect\hyperlink{garniture-godard}{ゴダール}のような、トリュフ、雄鶏のとさかとロニョン、クネル、マッシュルームなどを添える料理にとっておくべきだ。野菜のガルニチュールには、とろみを付けたジュ、もしくはバターを加えたグラスドヴィアンドのほうがずっと好ましい。
