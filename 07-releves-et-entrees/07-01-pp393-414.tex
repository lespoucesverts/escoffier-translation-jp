\href{✓原稿下準備なし}{} \href{訳と注釈\%2020180424古い原稿のコピペ}{}
\href{未、原文対照チェック}{} \href{未、日本語表現校正}{}
\href{未、注釈チェク}{} \href{未、原稿最終校正}{}

\hypertarget{vii.ux8089ux6599ux7406-relevuxe9s-et-entruxe9es}{%
\chapter{VII. 肉料理 Relevés et
Entrées}\label{vii.ux8089ux6599ux7406-relevuxe9s-et-entruxe9es}}

\begin{center}
\frsec{ブレゼ、ポワレ、ソテ、ポシェの基礎、下茹で\\グラタン、グリル、揚げものの調理理論}
\end{center}
\normalsize
\vspace{1\zw}
\frsec{Princie généraux de la Conduite de Braisée --- des Poêlés --- des Sautés --- et des Pochés --- Blanchissages\\\vspace{.5ex}Théorie des Gratins, des Grillades et des Fritures}
\normalsize

\hypertarget{les-braises}{%
\section{ブレゼ}\label{les-braises}}

いろいろな調理法の中でも、ブレゼはとりわけコストがかかり、しかもきわめて高度な技術が要求される。この調理法を習得するには時間をかけて、注意深く実践を重ねていくしかない。細心の注意を払って調理しなければならないのは勿論のこと、他の調理法もそうだが、主素材となる肉の品質は非常に重要だ。美味しいブレゼを作るにはさらに、加熱の際に上質のフォンを用い、香味素材
(フォン・ド・ブレーズ)もきちんと仕込みをしておく必要がある。

\hypertarget{ux30d6ux30ecux30bcux306bux7528ux3044ux308bux8089}{%
\subsection{ブレゼに用いる肉}\label{ux30d6ux30ecux30bcux306bux7528ux3044ux308bux8089}}

ブレゼの通常の調理法としてここで述べるのは、牛肉、羊肉を用いる。仔牛、乳飲み仔羊、家禽のブレゼについては後述。ブレゼの素材は、ロティの場合と違って、若い畜肉でなくていい。牛の場合は3〜6才、羊は1〜2才のものが最良だ。上記の年齢を過ぎたものはいい肉質を期待できない。そのような肉を使う場合は加熱時間をかなり長くとる必要が出てくるし、それでも大抵は、筋っぽくてぱさついた仕上りになってしまう。実際問題として、料理に老齢の畜肉を用いるのは、コンソメと各種フォンをとる場合のみと考えること。

\hypertarget{ux30e9ux30ebux30c7}{%
\subsection{ラルデ}\label{ux30e9ux30ebux30c7}}

アロワイヨや背肉の塊は、ペルシエつまり脂肪の筋が入ったものであれば良質で柔らかい。牛や羊のもも肉の場合は事情が異なる。牛、羊のもも肉はそれ自体が脂をあまり含んでいないため、長時間加熱するとぱさついてしまう。脂を補うために、約1cm角の棒状に切った豚背脂を、繊維方向に沿って肉の内部に刺し込む(ラルデ)。背脂は予{[}あらかじ{]}め、こしょう、ナツメグ、その他の香辛料で味つけし、パセリのみじん切りを振ってから、コニャック適量で2時間マリネしておくこと。

\hypertarget{ux30deux30eaux30cd}{%
\subsection{マリネ}\label{ux30deux30eaux30cd}}

背脂を刺す場合も、そうでない場合も、ワインと香味素材で素材を数時間マリネする。ワインは煮汁として、香味素材はフォン・ド・ブレーズとして用いることになる。マリネする前に、素材を塩、こしょう、その他の香辛料で味つけする。肉を転がし、調味料がよく浸み込むようにする。素材が丁度入る大きさの容器の底に香味素材を敷いて、その上に肉を置き、さらに香味素材で覆う。煮汁に用いるワインをひたひたの高さまで注ぐ。一般的には、白でも赤でも普通のワインを用いる。分量は肉1kgあたり3㎗。5〜6時間マリネする。途中、何度か肉を裏返す。

\hypertarget{ux9999ux5473ux7d20ux6750ux30d5ux30a9ux30f3ux30c9ux30d6ux30ecux30fcux30ba}{%
\subsection{香味素材(フォン・ド・ブレーズ)}\label{ux9999ux5473ux7d20ux6750ux30d5ux30a9ux30f3ux30c9ux30d6ux30ecux30fcux30ba}}

にんじんと玉ねぎを厚めの輪切りにし、バターまたはグレス・ド・マルミートを用いて強火でこんがり炒める(野菜の量は肉1kgあたり各60g)。ブーケガルニ
(にんにくを入れること)。肉1kgあたり50gの生豚皮(下茹でしておく)。

\hypertarget{ux30eaux30bdux30ecux30d6ux30ecux30bc}{%
\subsection{リソレ、ブレゼ}\label{ux30eaux30bdux30ecux30d6ux30ecux30bc}}

いい具合にマリネした状態になったら、肉を取り出して網にあげ、30分間水気をきる。さらに布などで水気を拭き取る。丁度いい大きさの、厚手の片手鍋またはブレゼ鍋に、不純物を取り除いたグレス・ド・マルミートを熱する。素材を鍋に入れ、強火で表面をまんべんなくこんがりと焼く(リソレ)。こうすることで、肉に一種の鎧をまとわせ、肉汁があまり早く外に流れ出ないようにする。そうでないと、ブレゼではなくブイイになってしまう。素材が大きければ大きい程、表面の焼き固めた層がそれだけ丈夫でなくてはいけないから、素材の表面を焼く時間は長くなることになる。

表面を焼き固めたら素材を鍋から取り出す。脂の少ない肉の場合は、豚背脂のシートでくるみ、紐で縛る。牛のアロワイヨや背肉の場合、この作業は必要ない。そのままの状態で、脂の層に覆われているからだ。

素材の大きさにぴったり合うサイズの鍋に、香味野菜(マリネに用いたもの)、豚皮、ブーケガルニを入れ、その上に素材を置く。マリネに用いたワインを注ぎ、強火にかけてシロップ状になるまで煮詰める。上質の茶色いフォンを素材がかぶるまで注ぎ、沸騰させる。鍋に蓋をして中火のオーヴンに入れ、安定した微沸騰の状態を保つようにする。

細いブリデ針を深く刺してみて、穴から血が上ってこなくなるまで火を通す。この時点で、ブレゼの第1段階は終了。以下が第2段階だが、その調理メカニズムについては後述する。

煮汁については、目的に応じて以下のどちらかの方法をとる。

煮汁を澄んだままにしておきたい場合は、肉を充分な大きさのきれいな別鍋に移し、布漉しした煮汁を注ぐ。この鍋をオーヴンに入れ、こまめに煮汁をかけてやりながら加熱を仕上げる。

通常の「とろみを付けたジュ」と同様に、コーンスターチでとろみを付ける。ブレゼのソースとして仕上げる場合は、煮汁を半量まで煮詰める。その⅔量のソース・エスパニョルと⅓量のトマトピュレまたは同等量の生のトマトを加え、もとの煮汁と同じ量にする。これを、上で述べたように別鍋に移しておいた肉にかけてオーヴンに入れ、こまめにソースをかけてやりながら加熱を仕上げる。

肉にナイフを刺してみて、抵抗なく入っていくようであれば、ほどよく火が通っているので、肉をソースから取り出す。ソースは布で漉し、脂がすっかり表面に浮いてくるまで10分程休ませる。浮いてきた脂は徹底的に取り除く。最後に、ソースが濃いようなら上質のフォン少量を加え、薄過ぎるようなら煮詰めて、ソースを仕上げる。

\hypertarget{ux30d6ux30ecux30bcux7b2cux6bb5ux968e-ux306eux8abfux7406ux30e1ux30abux30cbux30baux30e0-ux539fux66f8-p.396}{%
\subsection{ブレゼ第2段階 の調理メカニズム 原書
p.396}\label{ux30d6ux30ecux30bcux7b2cux6bb5ux968e-ux306eux8abfux7406ux30e1ux30abux30cbux30baux30e0-ux539fux66f8-p.396}}

作業の第1段階のところで述べたように、肉の塊が大きければ大きい程、表面をしっかり焼き固める必要がある。表面を焼き固めるのは、外に逃げ出そうとする肉汁を内側に押し返すことと、表面に一種の鎧を作ることが目的だ。この鎧は、加熱が進むにつれ周囲から中心に向かって厚いものになっていく。

鍋の液体が熱せられると、肉の筋繊維が締まり肉汁は中心に向かうが、やがて熱が中心まで届くと、そこに圧縮された肉汁は分解して余計な水分が分離される。水分は蒸気となり、筋繊維を膨らませてほぐすのだ。

つまり、第1段階では明らかに、肉の塊の中心に向かって肉汁が濃縮されていく。

第2段階ではこれとは逆向きの現象が起こる。

肉の塊の中心部に集まった肉汁の水分が蒸発するだけの温度に達すると、筋繊維がほぐれ始める。肉汁から分離した水分が、逃げ場がないために蒸気圧をかけるのだ。筋繊維はかなりの圧力の影響を受けるわけだが、この圧力は第1段階とは逆に、肉の中心から周囲に向かうことになる。

だから、加熱が進み、肉の内部の圧力が高まるとその結果、筋繊維はゆるんでいく。肉汁の水分は、外側の焼き固めた層に少しずつ達してその筋繊維をゆるめ、内部の肉汁が流れ出す通り道が出来る。肉汁はソースと混ざり、同時に、毛管現象によってソースも肉の内部に浸み込む。ここが、ブレゼの調理でもっとも注意を払わなければならないところだ。作業は最終段階で、煮汁はかなり煮詰まってきて、肉を覆ってはいない。
何も覆うものがないから肉はとてもすぐに乾燥してしまうから、こまめに煮汁をかけてやり、肉を裏返す。肉の組織が常にソースを吸い込んだ状態にしてやるのだ。こうすることで、他の調理法とは一線を画すブレゼの特徴、柔らかくてとろけるような仕上りとなる。

\hypertarget{ux7167ux308aux3092ux3064ux3051ux308bux30b0ux30e9ux30bb}{%
\subsection{照りをつける(グラセ)}\label{ux7167ux308aux3092ux3064ux3051ux308bux30b0ux30e9ux30bb}}

ブレゼを塊のまま客にプレゼンテーションする場合には絶対に照りをつける必要があるが、切り分けてから供する場合には必須ではないし、不要とも言える。

照りをつける場合、ほど良く火が通ったらすぐに鍋から肉を取り出し、平鍋に移してオーヴンの入口近くに入れる。煮汁かジュ、ソースを軽くかけると、オーヴンの熱で煮詰まって、表面に薄い膜が出来る。この作業を、肉が艶のある層ですっかり覆われるまで繰り返す。オーヴンから出したら皿に盛り、供するまでクロッシュをかぶせておく。

\hypertarget{ux6ce8ux610fux4e8bux9805}{%
\subsection{注意事項}\label{ux6ce8ux610fux4e8bux9805}}

例えばブフ・アラモードのように野菜を添える場合には、バターで色良く炒めてから、手順の第2段階で肉と一緒に煮るか、あるいは、煮汁の一部を取り分けて肉とは別に煮る。

一番いいのは最初のやり方だが、緻密な盛り付けをするには向かない。だから、臨機応変にどちらがいいか判断する必要がある。ブレゼの作り方について、一般的に行なわれてはいるけれども絶対に間違っていることが2つある。ひとつ目は、香味野菜(フォン・ド・ブレーズ)をパンセすることだ。

香味野菜をパンセするというのは、予{[}あらかじ{]}め色良く炒めておりた野菜の上に、焼き色を付けた肉を置くのではなく、ブレゼ鍋の底に生のままの野菜を並べ、その上に肉を置くが、多くの場合は肉にも焼き色を付けない。溶かしたグレス・ド・マルミートを少量かけてやり、野菜が鍋底に軽く焦げ付くまで加熱する。----
厳密に言えば、上手にやるならこの方法も許容されるだろう。けれど、片面しか焼き色を付けていない野菜では両面に焼き色を付けた場合ほどは風味は出ないし、その上、加熱時間が長過ぎるとほとんど黒焦げになりかねず、苦味が出てソースの風味を損ねてしまう。

実際のところ、このパンセという作業は、煮汁に用いるフォンを事前に仕込んでおくこともせず、そのフォンの材料をブレゼそれ自体と一緒に煮ていた大昔の料理のうわべだけ真似たものに過ぎない。

昔のブレゼの作り方は素晴しいけれど、とてもコストのかかるものだった。というのも、厚くスライスした生ハムと仔牛のもも肉にのせて主素材を同時にブレゼしていたのだ。コストの問題でこのやり方でブレゼを作らなくなって久しいのだが、本質的なところは無視して、形式的な手順だけが慣習として残ったのだ。しかも、昔はフォンの材料に肉を用いていたのを各種の獣骨で代用するようになってしまったのだから、誤りとしてますますひどいものになってしまった。

そこで、第2の誤りである。

よく知られているように、ブレゼに最もよく使われるのは仔牛の骨だが、それでさえ完全に煮出すのには10〜12時間かかる。まず5〜6時間煮てフォンをとった骨に、さらに液体を注いで6時間煮た方が、5〜6時間煮ただけのフォンよりも多くのグラス・ド・ヴィアンドが得られるのがその証拠だ。2番のフォンで作ったグラス・ド・ヴィアンドは風味は劣る一方で、ゼラチン質が多いのは事実だ。ブレゼに用いるフォンとしては、このゼラチン質は風味の要素に負けず劣らず意味がある。ゼラチン質によってブレゼはなめらかでとろけるような口当たりになる。これは他のものでは代用できないし、ソースの出来を決めるものなのだ。

ブレゼに生の骨を加えたとしても、ブレゼの加熱時間は最大でも4〜5時間以上には出来ないのだから、肉に火が通った時点で骨は表面しか煮えていないわけだ。骨に含まれているもののほとんどは煮出されないままになってしまう。----
つまり、ブレゼに骨を加えても全く意味がないのだ。

この間違った方法には、また別の欠点もある。素材を煮るのに大量の液体を用いなければならないということだ。ブレゼというのはソースがしっかり煮詰められてこくがないと完全ではない、というのは誰もが認めるところだろう。煮汁が多ければ多い程、ソースは薄味になるし、結果的に煮汁で肉を洗うようなことになってしまう。

だからこそ、既に述べたように、ブレゼでは素材の大きさにぴったり合った容量の鍋を用いるべきなのだ。肉は始めすっかり煮汁に浸っていなければならないわけだから、鍋の大きさが丁度良ければそれだけ煮汁の絶対量は少なくなり、素材から溶け出すものも加わってフォンに一層こくが出るのだ。

\hypertarget{ux767dux8eabux8089ux306eux30d6ux30ecux30bc}{%
\section{白身肉のブレゼ}\label{ux767dux8eabux8089ux306eux30d6ux30ecux30bc}}

当今の白身肉のブレゼは、本来的な意味ではブレゼとは呼べない。赤身肉のブレゼの作り方は2段階で作業を行なうのが特徴だが、その1段階で火入れを止めてしまうからである。

昔の料理ではブレゼの2段階の作業が行なわれていなかったのは事実だ。が、仔牛等の大きな塊肉はスプーンで切れるくらいまでよく火を通すことが多かったのだ。---
現代ではこうした調理は行なわれなくなってしまい、名称だけが残ったわけだ。

白身肉のブレゼは次のとおり。仔牛の背肉、鞍下肉、腰肉、もも肉。フリカンド、リ・ド・ヴォ、若い七面鳥、肥鶏。頻度は少ないが、仔羊のドゥーブルやバロン、鞍下肉でもよく作られる。

上記の肉すべて同じ作り方にするが、加熱時間は肉の大きさによって変わるので注意。

ブレゼに用いる香味素材は赤身肉のブレゼと同じだが、野菜は色づかないようバターで軽く炒めるだけにすること。煮汁には必ず白いフォンを用いる。

\hypertarget{ux767dux8eabux8089ux306eux30d6ux30ecux30bcux306eux4f5cux696dux624bux9806}{%
\subsection{白身肉のブレゼの作業手順}\label{ux767dux8eabux8089ux306eux30d6ux30ecux30bcux306eux4f5cux696dux624bux9806}}

リ・ド・ヴォは調理前に必ず下茹でするので別だが、ブレゼする白身肉や家禽は表面を全て、軽く色づく程度まで焼き固めてもいい。その方がパサつきにくくなる。とはいえ、この作業は省いても良い。

次に、底に香味素材を敷いた鍋に肉を入れる。鍋の大きさは肉がちょうど入るくらいで、蓋をした際に肉が蓋に当たらない程度の深さのものを用いる。

この鍋に仔牛のフォン少量を注ぎ、蓋をして弱火で煮詰める。再び同量のフォンを注いで同じように煮詰める。それから肉の半分の高さまで煮汁を注ぐ。沸騰したら弱火のオーヴンに入れる。弱火といっても、煮汁が微沸騰の状態を保つ程度の温度であること。

加熱中は肉の表面が乾かないように、こまめに煮汁を肉にかけてやること。フォンにはゼラチン質が多く含まれているので、表面に塗膜のようなものが出来て、熱による肉汁の蒸散を防いでくれる。肉の表面をごく軽く焼き固めただけでは保護層が充分に出来ていないからだ。

そのために、最終的に煮汁を注ぐ前に少量のフォンを煮詰めておいたわけだ。肉を鍋に入れてそのまま煮汁を注いだとして、上で述べたような膜が出来るほどフォンが濃くならないだろうから、肉はひどくパサついてしまうだろう。

白身肉のブレゼで火の通り具合をみるには、ブリデ針を深く刺す。穴から透明の肉汁が上がってくるようになったら程良く火が通っている。透明の肉汁が上がってくるというのは、肉の中心まで火が通って血が分解された証拠なのだ。

この火の通し加減という点で、赤身肉のブレゼと白身肉のブレゼは大きく異なるわけだ。実際のところ、白身肉のブレゼの火入れ加減はほとんどロティに近い。だから家禽とごく若い畜肉の脂がのっていて柔らかいものしか使わないのだ。というのも、この料理では、ロティと同じくらいの程良い火入れ加減を少しでも越えたらたちまちおいしくなくなってしまうからだ。

白身肉のブレゼは通常、照りをつけてやる。とりわけ、細く切った豚背脂をピケ針で刺している場合には照りをつけてやったほうがいい。豚背脂を刺すのは昔と比べて減ったが、まだこの方法を採る者も多い。

\hypertarget{les-poches}{%
\section{ポシェ}\label{les-poches}}

こんな表現が成り立つならの話だが、ポシェとは「沸騰させないで作るブイイ」とするのが最も正しい定義だろう。

「ポシェ」という用語は広い意味では、何らかの液体を用いて弱火でゆっくり火を通すことを指す。液体の量が多いか少ないかは問題とならない。だから、大きなテュルボや鮭を丸ごとクールブイヨンで煮るのはもとより、舌びらめの切り身を少量の魚のフュメで煮る場合や、温製のムースやムスリーヌ、クネル、クレーム、ロワイヤル等々の加熱についても、「ポシェ」と呼ばれる。

このようにポシェする対象は多岐にわたるので、それぞれの加熱時間は大きく異なる。けれども、全てに共通して絶対守るべき原則がある。ポシェする液体は決して沸騰させない、沸騰寸前の温度にするということだ。

もうひとつ大事なことだが、魚や鶏を丸ごとポシェする際は液体が冷たい状態で火にかけ、手早く所定の温度まで上げるようにする。ごく少量の液体で魚や鶏の切り身をポシェする場合も同様にしていい。これに対し、他のポシェの場合は事前に所定の温度にしておいた液体に投入する手順となる。

\hypertarget{preparation-des-volailles-a-pocher}{%
\subsection{鶏のポシェの下ごしらえ}\label{preparation-des-volailles-a-pocher}}

\frsecb{Préparation des Volailles à pocher}

鶏は下処理の後、指示があれば詰め物をし、ブリデ針を用い糸で手羽と脚を畳み込むように縛る。

細かく切ったトリュフ、ハム、ラング・エカルラートを鶏の胸や脚に刺す場合には、半割りにしたレモンで擦ってから、沸騰した白いフォンに数分間浸してやる。

こうすることで皮が締まり、トリュフ等を刺す作業がやり易くなる。

\hypertarget{pochage-de-la-volaille}{%
\subsection{鶏のポシェ}\label{pochage-de-la-volaille}}

\frsecb{Pochage de la Volaille}

詰め物をしたり、トリュフ等を刺すのは必要な時だけだが、どんな場合でも豚背脂のシートで包んでやる。丁度いい大きさの鍋に素材を入れ、事前に仕込んでおいた白いフォンをかぶるまで注ぐ。

火にかけて沸騰したらアクを引き、蓋をして所定の温度、つまり目で見てほとんどわからない程度の微沸騰の状態を保つようにする。熱がだんだん伝わって鶏に火を通すにはこれで充分な温度だ。

明らかな沸騰状態にしてしまうといろいろ不都合が起きる。とりわけ (1)
水分の蒸発が激しすぎて煮汁が煮詰まり、澄んだ状態を保てなくなる。(2)
詰め物をしている場合は特に、皮が弾けやすくなる。

鶏のポシェで火の通り加減を見るには、ドラムスティックに近い腿の裏側を刺してみる。完全に透明な汁が上がってくれば程良く火が通っている。

\hypertarget{nota-sur-les-pochages-de-volaille}{%
\subsection{注意事項}\label{nota-sur-les-pochages-de-volaille}}

\begin{itemize}
\item
  鶏をポシェするのに丁度いい大きさの鍋を使うべき理由は\ldots{}\ldots{}

  \begin{enumerate}
  \def\labelenumi{\arabic{enumi}.}
  \item
    加熱中、素材が常にフォンに浸っていなければならない。
  \item
    煮汁そのものをソースに用いるので、煮汁の全体量が少なければ少ない程、鶏から流れる肉汁が薄まりにくくなる。結果として、ソースの風味が良くなる。
  \end{enumerate}
\item
  {[}その他{]}\footnote{原書ではここに項目名はないが、箇条書きを見やすくするために訳者が補った。}  

  \begin{enumerate}
  \def\labelenumi{\arabic{enumi}.}
  \item
    ポシェに使う白いフォンは必ず事前に仕込んでおく。充分に澄んだフォンを用いること。
  \item
    もしもフォンをとる材料と鶏を一緒に火にかけたら、フォンの材料がどんなにたくさんでも、いい結果は得られないだろう。理由は、鶏の加熱時間は最大でも1時間〜1時間半であるのに対して、フォンの材料から香りと栄養素を充分に引き出すには最低6時間はかかる。その結果、単なるお湯に近い液体で鶏のポシェが完了し、その煮汁から作ったソースも味気ないものになってしまうだろう。
  \end{enumerate}
\end{itemize}

\hypertarget{les-poeles}{%
\section{ポワレ}\label{les-poeles}}

\frsec{Les Poêlés}

ポワレは事実上ロティの一種と言える。ロティもポワレも目指す火入れは同じだ。

ここで記すポワレは次のような古い調理法を単純化したものだ。古い調理法では、あらかじめ素材の表面に焼き色をつけ、たっぷりのマティニョンで覆ってから豚背脂のシートやバターを塗った紙で包み、オーヴンまたは串を刺して直火で、溶かしバターをかけながら焼いていた。

火が通ったらすぐに包みを外して脂をきる。マティニョンをブレゼ鍋または片手鍋に移し、マデイラと煮詰めたフォンを加える。

マティニョンの香味がフォンに移ったら、フォンを漉し、提供直前に浮き脂を取り除いて仕上げていた。

家禽を丸ごと調理する仕立てのいくつかについては、今なおこの古い方法で作る価値がある。

\hypertarget{ux30ddux30efux30ecux306eux4f5cux696dux624bux9806}{%
\subsection{ポワレの作業手順}\label{ux30ddux30efux30ecux306eux4f5cux696dux624bux9806}}

素材に対して余裕のある大きさの厚手の深鍋の底にマティニョンを敷きつめる
(マティニョンについては「ガルニテュールの仕込み」参照)。

畜肉あるいは家禽にしっかり味付けをし、野菜の上に置く。溶かしバターをたっぷりかけてやる。鍋に蓋をして、やや高温のオーヴンに入れる。

そうして、小まめにバターをかけながら、蓋をした状態でじっくり火を入れる。

火が通ったら鍋の蓋を取り、オーヴンの熱で素材に焼き色をつける。皿に移し、クロッシュをかぶせて保温しておく。

野菜(焦げていないこと)に充分な量の煮詰めた澄んだフォンを注ぐ。弱火で10
分間煮てから布で漉し、丁寧に浮き脂を取り除く。これをソース容器に入れ、主素材の周囲にガルニテュールを盛って供する。

\hypertarget{ux30ddux30efux30ecux306bux3064ux3044ux3066ux306eux6ce8ux610f}{%
\subsection{ポワレについての注意}\label{ux30ddux30efux30ecux306bux3064ux3044ux3066ux306eux6ce8ux610f}}

\begin{enumerate}
\def\labelenumi{\arabic{enumi}.}
\item
  ポワレは火入れに液体を用いないのが重要ポイント。液体を用いたら白身肉のブレゼと同じ風味になってしまう。ポワレは火入れにバターしか用いない。ただし、雉、ペルドロ、うずら等の猟鳥のポワレでは概ね火が通った時点で少量のコニャックを注いでフランベする。
\item
  フォンを注ぐ前に野菜から脂を取り除かないことも大事なポイントだ。
\end{enumerate}

実際、ポワレに用いたバターには主素材と野菜の風味が溶け込んでいるわけだ。この風味を取り出すためには、フォンを注いで最低10分以上バターと接しているようにする必要がある。その後であれば、バターを取り除いてもフォンの香味を損なうことはない。

\hypertarget{ux7279ux6b8aux306aux30ddux30efux30ec-ux30abux30b9ux30edux30fcux30ebux30b3ux30b3ux30c3ux30c8}{%
\subsection{特殊なポワレ ----
カスロール、ココット}\label{ux7279ux6b8aux306aux30ddux30efux30ec-ux30abux30b9ux30edux30fcux30ebux30b3ux30b3ux30c3ux30c8}}

カスロール、ココットという調理は畜肉、家禽、ジビエを専用の陶製の鍋で火入れし、鍋ごと供するが、これはまさにポワレそのものと言える。

一般的に、「カスロール」は野菜を加えずバターだけを用いて素材に火を通す。素材に程良く火が通ったら主素材を取り出し、鍋に仔牛のフォン少量を注ぐ。数分間沸かしてから、浮いている余分なバターを取り除く。主素材を鍋に戻し、保温しておくが、沸騰させないこと。

「ココット」も同様に調理するが、マッシュルーム、アーティチョークの萼の基底部、小玉ねぎ、にんじん、かぶ等の野菜を加えて調理する。野菜はそれぞれの性質に応じて形を整え、バターで炒めて半ば火を通しておくこと。

出来るだけ新野菜を使うようにすること。野菜は主素材の周りに入れるが、主素材と同時に火入れが終了するタイミングで加えること。「ココット」に用いる陶製の鍋はある程度使い込んだものの方がいい。重曹や洗剤は用いずに水できれいに洗って手入れすること。

新品の鍋を用いなければならない場合、軽く沸かした湯をいっぱいに注ぎ、12
時間以上は微沸騰の状態を保つようにしてやる。その後、水気を拭き取る。さらに、冷水をいっぱいに注ぎしばらく放置してから使用すること。

\hypertarget{ux30bdux30c6-les-sautes}{%
\section{ソテ \{les-sautes\}}\label{ux30bdux30c6-les-sautes}}

ソテと呼ばれる調理法は水を使わないで火入れをするのが特徴だ。バター、植物油や精製した獣脂だけを用いる。

ソテに用いるのは、捌いた家禽やジビエ、あるいはソテに適するようにカットした畜肉である。

ソテに用いる素材は全て、強く熱した油脂で表面を焼き固める。表面に層を作り、肉汁を外に流れ出させずに内部に留めておくのが目的だ。この作業はとりわけ牛や羊のような赤身肉の場合に行なう。

家禽とジビエのソテは、素材に焼き色をつけたらソテ鍋に蓋をしてレンジで、あるいは蓋をせずオーヴンに入れてロティールと同様に焼き脂をかけながら火入れを仕上げる。

次に、素材を鍋から取り出してデグラセする。主素材を鍋に戻してソースあるいは付合せとからめる場合はごく短時間、つまりソースの風味がなじむ程度にとどめる。

トゥルヌド、ノワゼート、コトレート、フィレ、アントルコートのような赤身肉のソテでは、少量の澄ましバターで表面を焼き固め、そのままレンジで加熱を行なう。

表面を焼き固める際には、素材が小さく、薄く切ったものであればそれだけ強火にする。

焼いていない生のままの面に血が滲み出てきたら裏返す。先に焼いた面にピンク色の肉汁が出てきたら程良く火が通っている。

ソテ鍋から肉を取り出し、脂を捨ててから、ソースの一部となるワイン等の液体を鍋に注いで沸騰させ、鍋底についた肉汁を溶かし出す。こうしてデグラセした液体にソースを加える
----
場合によってはその逆で、別途用意しておいたソースやガルニテュールに加えることもある。仕立てとしての「ソテ」ではデグラセを必ず行なうこと。

ソテ鍋は素材の大きさにぴったり合ったものを使用する。大き過ぎると、肉が接していない部分が高温になり、デグラセが上手く出来なくなってしまう。デグラセは、肉を焼いた際に流れ出て固形化した肉汁を液体で溶かし出し、それによってソースがおいしくなるわけだから、デグラセが上手く出来ないとソースがおいしくなくなってしまうのだ。

仔牛、仔羊のような白身肉のソテは、まず表面を焼き固めてから弱火で火を通す。

白身肉の他の調理法と同様、しっかりと火を通すこと。

「ソテ」という名称は、ソテとブレゼ両方の特徴を兼ね備えた料理にも使われる。これは実際のところ「ラグー」と呼ぶのがふさわしいものだ。

こうした料理には牛、仔牛、仔羊、ジビエ等が用いられる。本書ではエストゥファード、グーラーシュ、仔牛のソテ、仔羊のソテ、カルボナード、ナヴァラン、シヴェ等の名称でまとめてある。

調理の第1段階では通常のソテと同様に小さめに切った肉に焼き色を付ける。第2段階はソースやガルニテュールと合わせて時間をかけて火を通すという点でブレゼと良く似ている。

\hypertarget{ux4e0bux8339ux3067}{%
\section{下茹で}\label{ux4e0bux8339ux3067}}

\hypertarget{ux30b0ux30e9ux30bfux30f3}{%
\section{グラタン}\label{ux30b0ux30e9ux30bfux30f3}}

\hypertarget{ux30b0ux30eaux30eb}{%
\section{グリル}\label{ux30b0ux30eaux30eb}}

グリルにおける調理上の働きとして重要なのは「凝縮」である。

グリルで重要なのは肉汁であり、殆どの場合、肉汁を内部に凝縮させることをまず目指すべきだ。

グリルとは要するに直火で焼くロティと同じであり、人類が料理ということを始めた遥か遠い起源に遡れるだろう。原始人の堅い頭に最初に生じたこのグリエというアイデアは、「もっと美味しいものを食べたい」という本能的な欲求から生まれた進化であり、食べ物を加熱調理するのに用いられた初めての方法なのだ。

やがて、その当然の帰結としてグリルから串焼きのロティが発生した。その頃には既に人類は本能ではなく知性で考えるようになっており、理屈によって結果を推論し、実地から結論を導き出すようになっていた。こうして、料理は進歩への道を歩みはじめたのだ。

\hypertarget{ux30b0ux30eaux30ebux306eux71b1ux6e90}{%
\subsection{グリルの熱源}\label{ux30b0ux30eaux30ebux306eux71b1ux6e90}}

もっとも一般的に用いられており、明らかに最良のものと言える燃料は熾火、または小さめの木炭である。どんな種類の燃料でも、重要なのは煙を出さないということだ。火力を強めるために風を送る場合は、その風で煙が外に流れ去るようにしてやる。

ましてや、あまりないことだが、自然と火がくすぶってしまい人工的に風を送ってやらねばならない時でも、煙が出てはいけない。熱源以外の物が燃えたり、炭に脂が落ちて煙が出てしまったら、人工的に風を送ろうと、強い風が吹こうと、きちんと煙を追い出せない限りは、どうしようもなく不味いグリルになってしまうからだ。

とはいえ、他の種類の熱源をグリルに用いても構わない。本書はこの点で絶対を主張はしない。反対に、適切に使うならばどんな熱源でも良いと言える。

\hypertarget{ux706bux5e8a}{%
\subsection{火床}\label{ux706bux5e8a}}

火床あるいはグリル台の造りも重要だ。グリルする素材の性質や大きさはもとより、状況によって火力を強めたり弱めたり自在に出来なければならない。

だから火床は炉の中央に平らに配する。ただし火力の強弱をつける必要に応じて厚みには変化をつけられるようにする。また、風が当たる側はやや高くして、熱がまんべんなく均等に行きわたるようにする。

焼き網は必ず前もって火床に据え、素材をのせる時には充分に熱くなっておくようにする。網を熱くしておかないと素材が貼り付き、裏返す際に素材を壊してしまうことになる。

\hypertarget{ux30b0ux30eaux30ebux306eux5206ux985e}{%
\subsection{グリルの分類}\label{ux30b0ux30eaux30ebux306eux5206ux985e}}

グリルは4種に分類され、それぞれ注意すべき点が異なる。

\begin{itemize}
\tightlist
\item
  赤身肉のグリル(牛、羊、ジビエ)
\item
  白身肉のグリル(仔牛、仔羊、家禽)
\item
  魚のグリル
\item
  パン粉衣をつけたグリル。パン粉のみをまぶす場合と、イギリス風パン粉衣を用いる場合がある。
\end{itemize}

\hypertarget{ux8d64ux8eabux8089ux306eux30b0ux30eaux30eb}{%
\subsection{赤身肉のグリル}\label{ux8d64ux8eabux8089ux306eux30b0ux30eaux30eb}}

グリルの作業は何よりもまず、各素材に適切な加熱温度を見きわめることから始まる。

素材が大きければ大きい程、肉汁が多ければ多い程、強火で表面をしっかり焼き固める必要がある。

この「リソレ」の役割と利点については「ブレゼ」のページで既に述べたが、グリルに関してもう一度おさらいしておこう。

牛や羊のような赤身肉を大きく切ったものをグリルする場合、上質の素材で肉汁の豊富なものであればなおのこと、しっかりした層が出来るよう表面を焼き固める必要がある。

内部の肉汁が多ければそれだけ、表面の焼き固めた層に向かう圧力は強くなる。肉汁が熱されるにつれてこの圧力は強くなる。

素材の内部にゆっくり熱が伝わるように火力を上手く調節していれば、次のような調理メカニズムとなる。

肉の火に接している側は、焼き網を通った熱によって線維が収縮し、肉の内部に熱が伝わっていく。熱は層をなすように肉の内部に広がり、肉汁を逆流させる。しまいには肉汁が肉の反対側に逹し、火にあたっていない生のままの面に滲み出てくる。このタイミングで肉を裏返し、焼いていない面について同じプロセスを行なう。肉汁が上に向かって逆流して最初に焼いた面でいったん止まり、そこから血の雫が浮かび上がってきたら、程良く火が通っている。

素材が大きい場合、表面をしっかり焼き固めたらすぐに火を弱め、熱がゆっくりと内部に伝わるようにする。同じ火の強さのまま焼き続けたら、焼き固めた表面がすぐに焦げてしまい、熱が肉の内部に伝わるのを邪魔する結果となる。外側は黒焦げなのに内部はまるで生という状態になってしまう。

あまり厚みのない肉をグリルする場合は、強火で表面を焼き固め、数分間そのまま焼けば程良く火が通る。火を弱める必要はない。例)
ランプやシャトブリヤンに程良く火を通すには、まず表面を強火で焼き固めて肉汁が流れ出ないようにしてから、弱めの火加減にしてやり、熱がゆっくり伝わって中まで火が通るようにする。

トゥルヌドやフィレミニョン、ノワゼート、コトレートのような小さいカットの肉の場合は、必ず表面を強火で焼き固め、そのままの火の強さで焼き上げる。厚みがないために熱が中まですぐに伝わるからである。

\hypertarget{ux8d64ux8eabux8089ux306eux30b0ux30eaux30ebux306eux969bux306bux6c17ux3092ux3064ux3051ux308bux3079ux304dux3053ux3068}{%
\subsection{赤身肉のグリルの際に気をつけるべきこと}\label{ux8d64ux8eabux8089ux306eux30b0ux30eaux30ebux306eux969bux306bux6c17ux3092ux3064ux3051ux308bux3079ux304dux3053ux3068}}

肉を焼き網にのせる前に、澄ましバターを刷毛でまんべんなく塗っておく。焼いている間も同様に澄ましバターを小まめに塗り、火が当たっている部分が乾かないようにする。

肉を裏返すにはパレットナイフか、出来たらグリル用のトングを使う。肉用フォーク等で刺して裏返すのは避けるべきだ。肉を刺せば肉汁の流れ出る通り道が出来るわけだから、それまで気を配ってきたこと全てをわざわざ台無しにしてしまうことになる。

\hypertarget{ux7a0bux826fux3044ux706bux306eux901aux308aux5177ux5408}{%
\subsection{程良い火の通り具合}\label{ux7a0bux826fux3044ux706bux306eux901aux308aux5177ux5408}}

赤身肉の場合、火の通り具合は指で表面に触れてみて押すと抵抗を感じる程度が、内部まで熱が伝わって程良く火が通っている目安となる。反対に、指で押しても全然抵抗を感じないようなら、熱がまだ内部まで伝わっていない。肉の焼き固めた表面にピンク色の肉汁がわずかに浮び上ってくる状態がいちばん確かな目印だ。

\hypertarget{ux767dux8eabux8089ux306eux30b0ux30eaux30eb}{%
\subsection{白身肉のグリル}\label{ux767dux8eabux8089ux306eux30b0ux30eaux30eb}}

赤身肉の場合は必ず表面を強火で焼き固めるが、白身肉の場合その必要はまったくない。仔牛や仔羊のような白身肉は肉汁がアルブミンの状態でしか存在しておらず、焼く際に凝縮させるよう気を配る必要がないからだ。

白身肉のグリルは弱めの火加減で、肉に火を通しながら同時に表面にきれいな焼き色がつくようにする。焼いている間、小まめにバターをかけてやり、表面が乾かないようにする。

浸み出てくる肉汁がすっかり透明になったら、程良く火が通っている。

\hypertarget{ux9b5aux306eux30b0ux30eaux30eb}{%
\subsection{魚のグリル}\label{ux9b5aux306eux30b0ux30eaux30eb}}

魚は大小にかかわらず、バターか植物油をたっぷりかけてから、やや弱めの火加減でグリルする。火入れの最中も小まめにバターまたは油をかけてやること。

魚の場合は、骨から簡単に身が離れるようになったら程良く火が通っている。

鯖{[}さば{]}、ルージェ、鰊{[}にしん{]}のような脂ののった魚以外は、焼く前に小麦粉をまぶしつけ、溶かしバターをかける。こうして魚を黄金色の殻で覆うようにすると、身が乾くのを防ぐと同時に、見た目も良くなる。

\hypertarget{ux30a4ux30aeux30eaux30b9ux98a8ux307eux305fux306fux30d0ux30bfux30fcux3067ux30d1ux30f3ux7c89ux8863ux3092ux3064ux3051ux305fux30b0ux30eaux30eb}{%
\subsection{イギリス風またはバターでパン粉衣をつけたグリル}\label{ux30a4ux30aeux30eaux30b9ux98a8ux307eux305fux306fux30d0ux30bfux30fcux3067ux30d1ux30f3ux7c89ux8863ux3092ux3064ux3051ux305fux30b0ux30eaux30eb}}

一般的に小さな素材でしかこの方法は用いないが、ごく弱火でグリルし、主素材の火入れとパン粉衣の色づけが同時に行なわれるようにする。

加熱中は小まめに澄ましバターをかけてやる。衣は素材の肉汁を閉じ込めるためのものなので、裏返す時は衣を壊さないように注意すること。

\hypertarget{ux63daux3052ux3082ux306e}{%
\section{揚げもの}\label{ux63daux3052ux3082ux306e}}
