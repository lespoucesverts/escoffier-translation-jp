\input{../preamble/preamble-a4-ltjsarticle.tex}
\begin{document}
\hypertarget{hachis}{%
\subsection{アシ}\label{hachis}}

料理において「アシ」を字句どおりに、つまり「ミンチ状に細かく刻んだ肉」という意味で理解してはいけない。「アシ」とは\protect\hyperlink{salpicons-divers}{サルピコン}の一種だ。

アシは\protect\hyperlink{eminces}{エマンセ}\footnote{牛のアロワイヨー(腰肉の塊)、サーロイン、フィレなどのローストあるいはブレゼで供した残りを薄くスライスして、別途、料理として仕立て直すもの。}の扱いに準じる。その理由もエマンセと同じである。アシに用いる肉がローストしたものであるときは、決してソースの中で煮立てないこと。

アシは通常、熱い\protect\hyperlink{sauce-demi-glace}{ソース・ドゥミグラス}であえる。その割合は、アシにした肉
1 ㎏に対して2 \(\sfrac{1}{2}\) dLとすること。 (p.459)
\begin{recette}
\hypertarget{hachis-parmentier}{%
\subsubsection{アシ・パルマンティエ{[}\^{}3{]}}\label{hachis-parmentier}}

\begin{spacing}{0.2}\bfseries Hachis Parmentier\end{spacing}\normalfont\normalsize\setlength{\leftskip}{0pt}\par\vspace{1.1\zw}

\index{hachis@hachisl!parmentier@--- Parmentier}
\index{parmentier@Parmentierl!hachisl@Hachis ---}
\index{pommes de terre@pommes de terrel!Hachis Parmentier}
\index{あし@アシ!はるまんていえ@---・パルマンティエ}
\index{はるまんていえ@パルマンティエ!あし@アシ・パルマンティエ}
\index{しやかいも@じゃがいも!はるまんていえ@アシ・パルマンティエ}

大ぶりのじゃがいも(オランド種\footnote{じゃがいもの品種名。かつてはグラタン等に好んで用いられたが、古い品種のため現代では営利栽培されることは少ない。2007年版『ラルース・ガストロノミック』では、じゃがいもの主要品種としてリストアップされていない。なお、日本では植物検疫により生のじゃがいもの輸入が禁じられているため、この品種に限らず、ヨーロッパ産のじゃがいもは冷凍および加熱製品しか国内では利用できない。})はオーヴンで火を通す。上部を切りとって別に取り置く。中身をくり抜く。くり抜いたじゃがいもの中身をフォークでつぶし、これを、\protect\hyperlink{pommes-de-terre-macaire}{じゃがいものマケール}と同様に、バターでソテーする。

ソテーしたじゃがいものを以下と混ぜあわせる\ldots{}\ldots{}じゃがいもと同量の、細かいさいの目に切った牛肉。バターで炒めた玉ねぎのみじん切り大さじ3杯。パセリのみじん切り1つまみ。ヴィネガー少々。

くり抜いたじゃがいもの外側を器にして、上で混ぜたものを詰める。布で漉した\protect\hyperlink{sauce-lyonnaise}{リヨン風ソース}を何度も上からかけてやり、詰め物にたっぷり浸み込ませる。

取り置いてあったじゃがいもの上部で蓋をする。天板に並べ、オーブンで10分焼く。

オーヴンから取り出したらすぐに、じゃがいもをナフキンの上にに盛って供する。

\hypertarget{pommes-de-terre-macaire}{%
\subsubsection{じゃがいものマケール}\label{pommes-de-terre-macaire}}

\begin{spacing}{0.2}\bfseries Pommes de terre Macaire\end{spacing}\normalfont\normalsize\setlength{\leftskip}{0pt}\par\vspace{1.1\zw}

\index{pommes de terre@pommes de terre!macaire@--- Macaire}
\index{macairel@Macaire!pommes de terre@Pommes de terre ---}
\index{しやかいも@じゃがいも!まけーる@---のマケール}
\index{まけーる@マケール!しやかいも@じゃがいもの---}

オランダ種のじゃがいもはオーヴンで火を通す。火が通ったらすぐに中身をくり抜いて平鍋に入れる。塩、こしょうで調味し、フォークで潰す。じゃがいもの中身1kgあたりバター200gを加えてしっかり混ぜる。

これをガレットのように平たい円形に伸ばして、澄ましバターを熱したフライパンで、両面をこんがり焼く。(p.764)
\end{recette}
\begin{center}\rule{0.5\linewidth}{\linethickness}\end{center}

\hypertarget{ballottines-et-jambonneaux}{%
\subsection{バロティーヌとジャンボノー}\label{ballottines-et-jambonneaux}}

この種の調理には、胸肉を何か別の料理に用いた鶏などの家禽の腿肉が用いられる。

腿は骨を取り除き\footnote{désosser (デゾセ)。}、バロティーヌ\footnote{ballottine
  (バロティーヌ)円筒形の包み、の意。つまり形状を意味した名称の仕立て。しばしば\protect\hyperlink{galantine-type}{ガランティーヌ}と混同されるが、ガランティーヌは形状について特に定義がなく、ゼラチン質で固めたもの、が原義。}あるいはジャンボノー\footnote{jambonneau
  \textless{} jambon (ジョンボン)に縮小辞 eau
  が付いたもの。つまり小さいハム、の意。本来ハムとは骨付きあるいは骨を取り除いた豚腿肉で作られるが、ここでイメージされているジャンボノーの形とは、骨付きハムのそれ。}の形になるように詰め物をする。腿の皮はあらかじめこの用途に用いることを見越して長くとっておき、バロティーヌの形かジャンボノーの形になるように縫って閉じる。

バロティーヌもジャンボノーも成形した後に\protect\hyperlink{les-braises-de-viandes-blanches}{ブレゼ}する。家禽に合うガルニチュールを適宜添えるとよい。

\hypertarget{ux539fux6ce8}{%
\paragraph{【原注】}\label{ux539fux6ce8}}

冷製として仕立てる場合には、表面にジュレを塗るか、\protect\hyperlink{sauce-chaud-froid-blanche-ordinaire}{ホワイト系}か\protect\hyperlink{sauce-chaud-froid-brune}{ブラウン系のソース・ショフロワ}を覆いかけてやり、ガルニチュールは適宜添えること。
\end{document}