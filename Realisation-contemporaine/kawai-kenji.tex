\documentclass[twoside,14Q,a4paper,openany]{ltjsbook}

\usepackage{amsmath}
  \let\equation\gather
  \let\endequation\endgather
\usepackage{amssymb}
\usepackage[no-math]{fontspec}
\usepackage{geometry}
\usepackage{luaotfload}
\usepackage{graphicx}

\usepackage{setspace}


%%%%%%%%% hyperref %%%%%%%%%%%%%
\usepackage{refcount}
\usepackage[unicode=true,hyperfootnotes=false,pageanchor]{hyperref}
\hypersetup{hyperindex=false,%
             breaklinks=true,%
             bookmarks=true,%
             pdfauthor={五島 学},%
             pdftitle={エスコフィエ『料理の手引き』全注解},%
             colorlinks=false%true,%
             %colorlinks=true,%
             citecolor=blue,%
             urlcolor=cyan,%
             linkcolor=magenta,%
             bookmarksdepth=subsubsection,%
             pdfborder={0 0 0},%
             hyperfootnotes=false,%
             plainpages=false,
             }
\urlstyle{same}




%% 欧文フォント設定
% Libertine/Biolinum
\setmainfont[Ligatures=Historic,Scale=1.0]{Linux Libertine O}
\setsansfont[Ligatures=TeX, Scale=MatchLowercase]{Linux Biolinum O} 
%\usepackage{libertine}
\usepackage{unicode-math}
\setmathfont[Scale=1.2]{libertinusmath-regular.otf}
%\unimathsetup{math-style=ISO,bold-style=ISO}
%\setmathfont{xits-math.otf}
%\setmathfont{xits-math.otf}[range={cal,bfcal},StylisticSet=1]




%% Garamond
%\usepackage{ebgaramond-maths}
%\setmainfont[Ligatures=Historic,Scale=1.1]{EB Garamond}%fontspecによるフォント設定

%\usepackage{qpalatin}%palatino

%\setmainfont[Ligatures=Historic,Scale=MatchLowercase]{Tex Gyre Schola}
%\setmainfont[Ligatures=Historic,Scale=MatchLowercase]{Tex Gyre Pagella}
%\setsansfont[Scale=MatchLowercase]{TeX Gyre Heros}  % \sffamily のフォント
%\setsansfont[Scale=MatchLowercase]{TeX Gyre Adventor}  % \sffamily のフォント

%\setmonofont[Scale=MatchLowercase]{Inconsolata}       % \ttfamily のフォント

%\usepackage[cmintegrals,cmbraces]{newtxmath}%数式フォント

\usepackage{luatexja}
\usepackage{luatexja-fontspec}
%\ltjdefcharrange{8}{"2000-"2013, "2015-"2025, "2027-"203A, "203C-"206F}
%\ltjsetparameter{jacharrange={-2, +8}}
\usepackage{luatexja-ruby}

%%%%和文フォント設定
%\usepackage[sourcehan,bold,jis2004,expert,deluxe]{luatexja-preset}%Adobe源ノ明朝、ゴチ
%\usepackage[hiragino-pron,bold,jis2004,expert,deluxe]{luatexja-preset}
%\usepackage[yu-osx,expert,jis2004,bold]{luatexja-preset}
%\usepackage[moga-mogo-ex,bold]{luatexja-preset}

\newopentypefeature{PKana}{On}{pkna} % "PKana" and "On" can be arbitrary string
%%%%明朝にIPAexMincho、ゴチ(太字)にMoboGoBを使う設定。和文カナプロプーショナル使用可能だが読みづらくなる。
\setmainjfont[%
     %YokoFeatures={JFM=prop,PKana=On},%
     %CharacterWidth=AlternateProportional,%
%    CharacterWidth=Proportional,%Mogo, IPAExMinchoには不可
     %Kerning=On,%
     BoldFont={ MoboGoB },%
     ItalicFont={ MoboGoB },%
     BoldItalicFont={ MoboGoExB }%
     % ]{ MogaHMin }
     ]{ IPAExMincho }
     % ]{ IPAmjMincho }
\setsansjfont[%
     %YokoFeatures={JFM=prop,PKana=On},%
     %CharacterWidth=AlternateProportional,%
     % CharacterWidth=Proportional,%Mobo, IPAExGOthicには不可
     %Kerning=On,
     BoldFont={ MoboGoB },%
     ItalicFont={ MoboGoB },%
     BoldItalicFont={ MoboGoExB }%
     % ]{ MoboGo}
     ]{ IPAExGothic }
% %  %%%% 和文仮名プロプーショナルここまで
% %\ltjsetparameter{jacharrange={-2}}%キリル文字%引数に-3を付けるとギリシア文字も可能になるが、%三点リーダーも欧文化されてしまうので注意%


\renewcommand{\bfdefault}{bx}%和文ボールドを有効にする
\renewcommand{\headfont}{\gtfamily\sffamily\bfseries}%和文ボールドを有効にする
%\addfontfeature{Fractions=On}


\defaultfontfeatures[\rmfamily]{Scale=1.2}%効いていない様子
\defaultjfontfeatures{Scale=0.92487}%和文フォントのサイズ調整。デフォルトは 0.962212 倍%ltjsclassesでは不要?
%\defaultjfontfeatures{Scale=0.962212}
%\usepackage{libertineotf}%linux libertine font %ギリシア語含む
%\usepackage[T1]{fontenc}
%\usepackage[full]{textcomp}
%\usepackage[osfI,scaled=1.0]{garamondx}
%\usepackage{tgheros,tgcursor}
%\usepackage[garamondx]{newtxmath}

\usepackage{layout}

%% レイアウト調整(A4Paper,14Q,twoside,ltjsbook.cls) 
%%
\setlength{\hoffset}{0\zw}
\setlength{\oddsidemargin}{0\zw}%タブレット前提の中央配置
\setlength{\evensidemargin}{\oddsidemargin}
% \setlength{\oddsidemargin}{1\zw}%製本時に右ページのみをオフセット
%\setlength{\evensidemargin}{0pt}%
\setlength{\fullwidth}{45\zw}
\setlength{\textwidth}{45\zw}%%ltjsclassesのみ有効
%\setlength{\fullwidth}{159mm}
%\setlength{\textwidth}{159mm}
\setlength{\marginparsep}{0pt}
\setlength{\marginparwidth}{0pt}
\setlength{\footskip}{0pt}
\setlength{\voffset}{-17mm}
\setlength{\textheight}{265mm}
\setlength{\parskip}{0pt}
\setlength{\parindent}{0pt}

%%%% b5j%%%%%%
% \setlength{\hoffset}{-10mm}
% \setlength{\oddsidemargin}{0mm}
% \setlength{\evensidemargin}{0mm}
% %\setlength{\textwidth}{\fullwidth}%%ltjsclassesのみ有効
% \setlength{\fullwidth}{145mm}
% \setlength{\textwidth}{145mm}
% \setlength{\marginparsep}{0pt}
% \setlength{\marginparwidth}{0pt}
% \setlength{\footskip}{0pt}
% \setlength{\voffset}{-10mm}
% \setlength{\textheight}{225mm}
% \setlength{\parskip}{0pt}
%%%ベースライン調整
%\ltjsetparameter{yjabaselineshift=0pt,yalbaselineshift=-.75pt}


%\usepackage{fancyhdr}






%文字サイズ、見出しなどの再定義
\makeatletter
%\renewcommand{\large}{\jsc@setfontsize\large\@xipt{14}}
%\renewcommand{\Large}{\jsc@setfontsize\Large{13}{15}}

\newcommand{\medlarge}{\fontsize{11}{13}\selectfont}
\newcommand{\medsmall}{\fontsize{9.23}{9.5}\selectfont}
\newcommand{\twelveq}{\jsc@setfontsize\twelveq{9.230769}{9.75}\selectfont}
\newcommand{\fourteenq}{\jsc@setfontsize\fourteenq{10.7692}{13}\selectfont}
\newcommand{\fifteenq}{\jsc@setfontsize\fifteenq{11.53846}{14}\selectfont}

\renewcommand{\chapter}{%
  \if@openleft\cleardoublepage\else
  \if@openright\cleardoublepage\else\clearpage\fi\fi
  \plainifnotempty % 元: \thispagestyle{plain}
  \global\@topnum\z@
  \if@english \@afterindentfalse \else \@afterindenttrue \fi
  \secdef
    {\@omit@numberfalse\@chapter}%
    {\@omit@numbertrue\@schapter}}
\def\@chapter[#1]#2{%
  \ifnum \c@secnumdepth >\m@ne
    \if@mainmatter
      \refstepcounter{chapter}%
      \typeout{\@chapapp\thechapter\@chappos}%
      \addcontentsline{toc}{chapter}%
        {\protect\numberline
        % {\if@english\thechapter\else\@chapapp\thechapter\@chappos\fi}%
        {\@chapapp\thechapter\@chappos}%
        #1}%
    \else\addcontentsline{toc}{chapter}{#1}\fi
  \else
    \addcontentsline{toc}{chapter}{#1}%
  \fi
  \chaptermark{#1}%
  \addtocontents{lof}{\protect\addvspace{10\jsc@mpt}}%
  \addtocontents{lot}{\protect\addvspace{10\jsc@mpt}}%
  \if@twocolumn
    \@topnewpage[\@makechapterhead{#2}]%
  \else
    \@makechapterhead{#2}%
    \@afterheading
  \fi}
\def\@makechapterhead#1{%
  \vspace*{0\Cvs}% 欧文は50pt
  {\parindent \z@ \centering \normalfont
    \ifnum \c@secnumdepth >\m@ne
      \if@mainmatter
        \huge\headfont \@chapapp\thechapter\@chappos%変更
        \par\nobreak
        \vskip \Cvs % 欧文は20pt
      \fi
    \fi
    \interlinepenalty\@M
    \huge \headfont #1\par\nobreak
    \vskip 1\Cvs}} % 欧文は40pt%変更

\renewcommand{\section}{%
    \if@slide\clearpage\fi
    \@startsection{section}{1}{\z@}%
    {\Cvs \@plus.5\Cdp \@minus.2\Cdp}% 前アキ
    % {.5\Cvs \@plus.3\Cdp}% 後アキ
    {.5\Cvs}
    {\normalfont\Large\headfont\bfseries\centering}}%変更

\renewcommand{\subsection}{\@startsection{subsection}{2}{\z@}%
    {\Cvs \@plus.5\Cdp \@minus.2\Cdp}% 前アキ
    % {.5\Cvs \@plus.3\Cdp}% 後アキ
    {.5\Cvs}
  %  {\normalfont\large\headfont\bfseries\centering}} %変更
    {\normalfont\large\headfont\centering}} %変更

\renewcommand{\subsubsection}{\@startsection{subsubsection}{3}{\z@}%
  % {0\Cvs \@plus.8\Cdp \@minus.6\Cdp}%変更
    {1sp \@plus.5\Cdp \@minus.5\Cdp}%変更
    {\if@slide .5\Cvs \@plus.3\Cdp \else \z@ \fi}%
    % {\normalfont\medlarge\headfont\leftskip -1\zw}}
    {\normalfont\medlarge\headfont\leftskip -1\zw}}

\renewcommand{\paragraph}{\@startsection{paragraph}{4}{\z@}%
    {0.5\Cvs \@plus.5\Cdp \@minus.2\Cdp}%
    % {\if@slide .5\Cvs \@plus.3\Cdp \else -1\zw\fi}% 改行せず 1\zw のアキ
    {1sp}%後アキ
    {\normalfont\normalsize\headfont}}
\renewcommand{\subparagraph}{\@startsection{subparagraph}{5}{\z@}%
    {\z@}{\if@slide .5\Cvs \@plus.3\Cdp \else -.5\zw\fi}%
    {\normalfont\normalsize\headfont\hskip-.5\zw\noindent}}  

\newcommand{\frchap}[1]{\vspace*{-2ex}%
 \begin{center}\normalfont\headfont\LARGE\setstretch{0.8}
 \scshape#1\normalfont\normalsize
\end{center}\vspace{0.5\zw}\setstretch{1.0}}

\newcommand{\frsec}[1]{\vspace*{-2ex}%
 \begin{center}\normalfont\headfont\large\setstretch{0.8}
 \scshape#1\normalfont\normalsize
\end{center}\vspace{0.5\zw}\setstretch{1.0}}
  
\newcommand{\frsecb}[1]{\vspace*{-2ex}%
\begin{center}\normalfont\headfont\medlarge\setstretch{0.8}%
  \hspace{1em}\scshape#1\normalfont\normalsize%
\end{center}\vspace{0.5\zw}\setstretch{1.0}}

%\newcounter{frsub}[subsubsection]
%\newcommand{\frsub}{\@startsection{frsub}{6}{\z@}%
%  {1sp}{1sp}%
%  {\normalfont\normalsize\bfseries\baselineskip-.8ex\leftskip-1\zw}}
%\let\frsubmark\@gobble
%\newcommand*{\l@frsub}{%
%          \@tempdima\jsc@tocl@width \advance\@tempdima 16.183\zw
%          \@dottedtocline{7}{\@tempdima}{6.5\zw}}
%\renewcommand{\thefrsub}{6}
%\let\frsub\paragraph

\newenvironment{frsubenv}{\begin{spacing}{0.2}\setlength{\leftskip}{-1\zw}\bfseries}{\end{spacing}\normalfont\normalsize\setlength{\leftskip}{0pt}}
\newcommand{\frsub}[1]{\begin{frsubenv}#1\end{frsubenv}\par\vspace{1.1\zw}}

%\newcommand{\frsub}[1]{\vskip -.8ex\hskip -1\zw\textbf{#1}\leftskip0pt}
%\newcommand{\frsub}{\@startsection{frsub}{6}{\z@}%
%   {-1\zw}% 改行せず 1\zw のアキ
%   {-1\zw}%後アキ     
%   {\normalfont\normalsize\bfseries\leftskip -1\zw\baselineskip -.5ex}}%normalsizeから変更
%\newcommand*{\l@frsub}{%
%          \@tempdima\jsc@tocl@width \advance\@tempdima 16.183\zw
%          \@dottedtocline{5}{\@tempdima}{6.5\zw}}


%%%%%%%%%レシピと本文%%%%%%%%%%%%
\usepackage{multicol}
\setlength{\columnsep}{3\zw}

%%% 本文
\newenvironment{Main}{}{}
%%% レシピ
% \setlength{\columnwidth}{24\zw}
%本文ヨリ小%\small
%\newenvironment{recette}{\setlength{\parindent}{0pt}\begin{small}\begin{spaceing}{0.8}\begin{multicols}{2}}{\end{multicols}\end{spacing}\end{small}}
%本文やや小%\medsmall
%\newenvironment{recette}{\setlength{\parindent}{0pt}\begin{medsmall}\begin{spacing}{0.75}\begin{multicols}{2}}{\end{multicols}\end{spacing}\end{medsmall}}
%本文ナミ(無指定)
\newenvironment{recette}{\setlength{\parindent}{0pt}\begin{spacing}{0.8}\begin{multicols}{2}\setlength\topskip{.8\baselineskip}}{\end{multicols}\end{spacing}}


\makeatother


%%% 脚注番号のページ毎のリセットと脚注位置の調整
%\renewcommand{\footnotesize}{\small}

\makeatletter

\usepackage[bottom,perpage,stable]{footmisc}%
%\setlength{\skip\footins}{4mm plus 4mm}
%\usepackage{footnpag}
\renewcommand\@makefntext[1]{%
  \advance\leftskip 0\zw
  \parindent 1\zw
  \noindent
  \llap{\@thefnmark\hskip0.5\zw}#1}


\let\footnotes@ve=\footnote
\def\footnote{\inhibitglue\footnotes@ve}
\let\footnotemarks@ve=\footnotemark
%\def\footnotemark{\inhibitglue\footnotemarks@ve}
\renewcommand{\footnotemark}{\footnotemarks@ve}%変更
% %\def\thefootnote{\ifnum\c@footnote>\z@\leavevmode\lower.5ex\hbox{(}\@arabic\c@footnote\hbox{)}\fi}
\renewcommand{\thefootnote}{\ifnum\c@footnote>\z@\leavevmode\hbox{}\@arabic\c@footnote\hbox{)}\fi}
%\makeatletter
% \@addtoreset{footnote}{page}
% \makeatother
%\usepackage{dblfnote}
%\usepackage[bottom,perpage]{footmisc}

\makeatother

%subsubsectionに連番をつける
%\usepackage{remreset}

\renewcommand{\thechapter}{}
\renewcommand{\thesection}{\hskip-1\zw}
\renewcommand{\thesubsection}{}
\renewcommand{\thesubsubsection}{}
\renewcommand{\theparagraph}{}


% \makeatletter
% \@removefromreset{subsubsection}{subsection}
% \def\thesubsubsection{\arabic{subsubsection}.}
% \newcounter{rnumber}
% \renewcommand{\thernumber}{\refstepcounter{rnumber} }
% \makeatother
\renewcommand{\prepartname}{\if@english Part~\else {}\fi}
\renewcommand{\postpartname}{\if@english\else {}\fi}
\renewcommand{\prechaptername}{\if@english Chapter~\else {}\fi}
\renewcommand{\postchaptername}{\if@english\else {}\fi}
\renewcommand{\presectionname}{}%  第
\renewcommand{\postsectionname}{}% 節

%リスト環境
\def\tightlist{\itemsep1pt\parskip0pt\parsep0pt}%pandoc対策

\makeatletter
  \parsep   = 0pt
  \labelsep = .5\zw
  \def\@listi{%
     \leftmargin = 0pt \rightmargin = 0pt
     \labelwidth\leftmargin \advance\labelwidth-\labelsep
     \topsep     = 0pt%\baselineskip
     %\topsep -0.1\baselineskip \@plus 0\baselineskip \@minus 0.1 \baselineskip
     \partopsep  = 0pt \itemsep       = 0pt
     \itemindent = -.5\zw \listparindent = 0\zw}
  \let\@listI\@listi
  \@listi
  \def\@listii{%
     \leftmargin = 1.8\zw \rightmargin = 0pt
     \labelwidth\leftmargin \advance\labelwidth-\labelsep
     \topsep     = 0pt \partopsep     = 0pt \itemsep   = 0pt
     \itemindent = 0pt \listparindent = 1\zw}
  \let\@listiii\@listii
  \let\@listiv\@listii
  \let\@listv\@listii
  \let\@listvi\@listii
\makeatother








%% %%%%%%行取りマクロ
% \makeatletter
% \ifx\Cht\undefined
%  \newdimen\Cht\newdimen\Cdp
%  \setbox0\hbox{\char\jis"2121}\Cht=\ht0\Cdp=\dp0\fi
% \catcode`@=11
% \long\def\linespace#1#2{\par\noindent
%   \dimen@=\baselineskip
%   \multiply\dimen@ #1\advance\dimen@-\baselineskip
%   \advance\dimen@-\Cht\advance\dimen@\Cdp
%   \setbox0\vbox{\noindent #2}%
%   \advance\dimen@\ht0\advance\dimen@-\dp0%
%   \vtop to\z@{\hbox{\vrule width\z@ height\Cht depth\z@
%    \raise-.5\dimen@\hbox{\box0}}\vss}%
%   \dimen@=\baselineskip
%   \multiply\dimen@ #1\advance\dimen@-2\baselineskip
%   \par\nobreak\vskip\dimen@
%   \hbox{\vrule width\z@ height\Cht depth\z@}\vskip\z@}
% \catcode`@=12
% \setlength{\parskip}{0pt}
% \setlength{\topskip}{\Cht}
% \setlength{\textheight}{43\baselineskip}
% \addtolength{\textheight}{1\zh}
% \makeatother
 
%%%%%%%%%%%%失敗%%%%%%%%%%%%
%\let\formule\subsubsection
%\renewcommand{\subsubsection}[1]{\linespace{1}{\formule#1}}
%%%%%%%%%%%%失敗%%%%%%%%%%%%






% PDF/X-1a
% \usepackage[x-1a]{pdfx}
% \Keywords{pdfTeX\sep PDF/X-1a\sep PDF/A-b}
% \Title{Sample LaTeX input file}
% \Author{LaTeX project team}
% \Org{TeX Users Group}
% \pdfcompresslevel=0
%\usepackage[cmyk]{xcolor}

%biblatex
%\usepackage[notes,strict,backend=biber,autolang=other,%
%                   bibencoding=inputenc,autocite=footnote]{biblatex-chicago}
%\addbibresource{hist-agri.bib}
\let\cite=\autocite

% % % % 
\date{}



%%%%インデックス準備

%\usepackage{makeidx}
\usepackage{index}
%\usepackage[useindex]{splitidx}
\newindex{src}{adx}{and}{ソース名から料理を探す}
\makeindex
 
 \makeatletter
\renewenvironment{theindex}{% 索引を3段組で出力する環境
    \if@twocolumn
      \onecolumn\@restonecolfalse
    \else
      \clearpage\@restonecoltrue
    \fi
    \columnseprule.4pt \columnsep 2\zw
    \ifx\multicols\@undefined
      \twocolumn[\@makeschapterhead{\indexname}%
      \addcontentsline{toc}{chapter}{\indexname}]%
    \else
      \ifdim\textwidth=\fullwidth
        \setlength{\evensidemargin}{\oddsidemargin}
        \setlength{\textwidth}{\fullwidth}
        \setlength{\linewidth}{\fullwidth}
        \begin{multicols}{3}[\chapter*{\indexname}%
        \addcontentsline{toc}{chapter}{\indexname}]%
      \else
        \begin{multicols}{2}[\chapter*{\indexname}%
        \addcontentsline{toc}{chapter}{\indexname}]%
      \fi
    \fi
    \@mkboth{\indexname}{}%
    \plainifnotempty % \thispagestyle{plain}
    \parindent\z@
    \parskip\z@ \@plus .3\jsc@mpt\relax
    \let\item\@idxitem
    \raggedright
    \footnotesize\narrowbaselines
  }{
    \ifx\multicols\@undefined
      \if@restonecol\onecolumn\fi
    \else
      \end{multicols}
    \fi
    \clearpage
  }
 \makeatother


%%%% 本文中の参照ページ番号表示 %%%%%%%

\makeatletter

%\AtBeginDocument{%
%  \DeclareRobustCommand\ref{\@ifstar\@refstar\@refstar}%
%  \DeclareRobustCommand\pageref{\@ifstar\@pagerefstar\@pagerefstar}}
\let\orig@Hy@EveryPageAnchor\Hy@EveryPageAnchor
\def\Hy@EveryPageAnchor{%
    \begingroup
    \hypersetup{pdfview=Fit}%
    \orig@Hy@EveryPageAnchor
    \endgroup
  }
  \usepackage{etoolbox}
\if@mainmatter{\let\myhyperlink\hyperlink%
\renewcommand{\hyperlink}[2]{\myhyperlink{#1}{#2} [p.\getpagerefnumber{#1}{}] }}
  \AtBeginEnvironment{recette}{%
\let\myhyperlink\hyperlink%
\renewcommand{\hyperlink}[2]{\myhyperlink{#1}{#2} [p.\getpagerefnumber{#1}{}] }}
  \AtBeginEnvironment{Main}{%
\let\myhyperlink\hyperlink%
\renewcommand{\hyperlink}[2]{\myhyperlink{#1}{#2} [p.\getpagerefnumber{#1}{}] }}
% \if@mainmatter{\let\myhyperlink\hyperlink%
% \renewcommand{\hyperlink}[2]{\myhyperlink{#1}{#2} {\ltjsetparameter{yjabaselineshift=0pt,yalbaselineshift=-.75pt}\footnotesize [p.\getpagerefnumber{#1}{}]}}}
%   \AtBeginEnvironment{recette}{%
% \let\myhyperlink\hyperlink%
% \renewcommand{\hyperlink}[2]{\myhyperlink{#1}{#2} {\ltjsetparameter{yjabaselineshift=0pt,yalbaselineshift=-.75pt}\footnotesize [p.\getpagerefnumber{#1}{}]}}}
%   \AtBeginEnvironment{Main}{%
% \let\myhyperlink\hyperlink%
% \renewcommand{\hyperlink}[2]{\myhyperlink{#1}{#2} {\ltjsetparameter{yjabaselineshift=0pt,yalbaselineshift=-.75pt}\footnotesize [p.\getpagerefnumber{#1}{}]}}}


% \def\@@wrindex#1|#2|#3\\{%
%  \if@filesw
%  \ifx\\#2\\%
%   \protected@write\@indexfile{}{% 
%     %\string\indexentry{#1}{\thepage}%%%改変部分。もとは{#1|hyperpage}{\thepage}%
%     \string\indexentry{#1|hyperpage}{\thepage}%%オリジナル%
%         }%
%         \else
%           \HyInd@@@wrindex{#1}#2\\%
%         \fi
%       \fi
%       \endgroup
%       \@esphack
%     }

\makeatother



%%%%% Obsolete Reference Page Numbers 

%\newcommand{\pref}[1]{[p.\pageref{#1}]}
%\newcommand{\pref}[1]{}







%%%% pandoc が三点リーダーを勝手に変える対策
\renewcommand{\ldots}{\noindent…}
%%%%%下線
\usepackage{umoline}
\setlength{\UnderlineDepth}{2pt}
\let\ul\Underline

\newcommand{\maeaki}{}%使用しないので無効化
\newcommand{\atoaki}{\vspace{1.25mm}}
%%分数の表記Obsolete
\usepackage{xfrac}
\let\frac\sfrac
\newcommand{\undemi}{\hspace{.25\zw}$\sfrac{1}{2}$}
\newcommand{\untiers}{\hspace{.25\zw}$\sfrac{1}{3}$}
\newcommand{\deuxtiers}{\hspace{.25\zw}$\sfrac{2}{3}$}
\newcommand{\unquart}{\hspace{.25\zw}$\sfrac{1}{4}$}
\newcommand{\troisquarts}{\hspace{.25\zw}$\sfrac{3}{4}$}
\newcommand{\quatrequatrieme}{\hspace{.25\zw}$\sfrac{4}{4$}}
\newcommand{\uncinquieme}{\hspace{.25\zw}$\sfrac{1}{5}$}
\newcommand{\deuxcinquiemes}{\hspace{.25\zw}$\sfrac{2}{5}$}
\newcommand{\troiscinquiemes}{\hspace{.25\zw}$\sfrac{3}{5}$}
\newcommand{\quatrecinquiemes}{\hspace{.25\zw}$\sfrac{4}{5}$}
\newcommand{\unsixieme}{\hspace{.25\zw}$\sfrac{1}{6}$}
\newcommand{\cinqsixiemes}{\hspace{.25\zw}$\sfrac{5}{6}$}
\newcommand{\quatrequart}{\hspace{.25\zw}$\sfrac{4}{4}$}

\makeatletter
\def\ps@headings{%
  \let\@oddfoot\@empty
  \let\@evenfoot\@empty
  \def\@evenhead{%
    \if@mparswitch \hss \fi
    \underline{\hbox to \fullwidth{\ltjsetparameter{autoxspacing={true}}
%      \textbf{\thepage}\hfil\leftmark}}%
       \normalfont\thepage\hfill\scshape\small\leftmark\normalfont}}%
    \if@mparswitch\else \hss \fi}%
  \def\@oddhead{\underline{\hbox to \fullwidth{\ltjsetparameter{autoxspacing={true}}
        {\if@twoside\scshape\small\rightmark\else\scshape\small\leftmark\fi}\hfil\thepage\normalfont}}\hss}%
  \let\@mkboth\markboth
  \def\chaptermark##1{\markboth{%
    \ifnum \c@secnumdepth >\m@ne
      \if@mainmatter
        \if@omit@number\else
          \@chapapp\thechapter\@chappos\hskip1\zw
        \fi
      \fi
    \fi
    ##1}{}}%
  \def\sectionmark##1{\markright{%
%    \ifnum \c@secnumdepth >\z@ \thesection \hskip1\zw\fi
    \ifnum \c@secnumdepth >\z@ \thesection \hskip-1\zw\fi
    ##1}}}%
\makeatother

\makeatletter
%%%%%%%% Lua GC
\patchcmd\@outputpage{\stepcounter{page}}{%
  \directlua{%
	if jit then
      local k = collectgarbage("count")
      if k>900000 then 
        collectgarbage("collect")
        texio.write_nl("term and log", "GC: ", math.floor(k), math.floor(collectgarbage("count")))
      end
	end
  }%
  \stepcounter{page}%
}{}{}
\makeatother
%\usepackage{vgrid}% here only to help visualize the problem

%\input{preamble/sources-index}%ソース名からの逆引きインデックス用コマンド

\makeatlette

%%%%%%基本ソース

\def\srcEspagnole#1#2#3#4{%
  \index[src]{espagnole@Espagnole!{#1}@{#2}}
  \index[src]{えすはによる@エスパニョル!{#3}@{#4}}}

\def\srcEspagnoleMaigre#1#2#3#4{%
  \index[src]{espagnole maigre@Espagnole maigre!{#1}@{#2}}%
  \index[src]{えすはによるさかな@エスパニョル(魚料理用)!{#3}@{#4}}}

\def\srcDemiGlace#1#2#3#4{%
  \index[src]{demi-glace@Demi-glace!{#1}@{#2}}%
  \index[src]{とうみくらす@ドゥミグラス!{#3}@{#4}}}

\def\srcJusDeVeauLie#1#2#3#4{%
  \index[src]{jus veau lie@Jus de veau lié!{#1}@{#2}}%
  \index[src]{とろみをつけたこうしのしゆ@とろみを付けた仔牛のジュ!{#3}@{#4}}}

def\srcVeoute#1#2#3#4{%
  \index[src]{veloute@Velouté!{#1}@{#2}}%
  \index[src]{うるて@ヴルテ!{#3}@{#4}}}

def\srcVeouteDeVolaille#1#2#3#4{%
  \index[src]{veloute de volaille@Velouté de volaille!{#1}@{#2}}%
  \index[src]{とりのうるて@鶏のヴルテ!{#3}@{#4}}}

def\srcVeouteDePoisson#1#2#3#4{%
  \index[src]{veloute de poisson@Velouté de poisson!{#1}@{#2}}%
  \index[src]{さかなのうるて@魚のヴルテ!{#3}@{#4}}}

def\srcAllemande#1#2#3#4{%
  \index[src]{allemande@Allemande!{#1}@{#2}}%
  \index[src]{あるまんと@アルマンド!{#3}@{#4}}}

def\srcSupreme#1#2#3#4{%
  \index[src]{supreme@Suprême!{#1}@{#2}}%
  \index[src]{しゆふれーむ@シュプレーム!{#3}@{#4}}}

def\srcTomate#1#2#3#4{%
  \index[src]{tomate@Tomate!{#1}@{#2}}%
  \index[src]{とまと@トマト!{#3}@{#4}}}

%%%%%%%%ブラウン系の派生ソース

def\srcBigarade#1#2#3#4{%
  \index[src]{bigarade@Bigarade!{#1}@{#2}}%
  \index[src]{ひからーと@ビガラード!{#3}@{#4}}}

def\srcBordelaise#1#2#3#4{%
  \index[src]{bordelaise@Bordelaise!{#1}@{#2}}%
  \index[src]{ほるとーふう@ボルドー風!{#1}@{#2}}}

def\srcBourguignonne#1#2#3#4{%
  \index[src]{bourguignonne@Bourguignonne!{#1}@{#2}}%
  \index[src]{ふるこーにゆふう@ブルゴーニュ風!{#3}@{#4}}}

def\srcBretonne#1#2#3#4{%
  \index[src]{bretonne@Bretonne!{#1}@{#2}}%
  \index[src]{ふるたーにゆふう@ブルターニュ風!{#3}@{#4}}}

def\srcCerises#1#2#3#4{%
  \index[src]{cerises@Cerises (aux)!{#1}@{#2}}%
  \index[src]{すりーす@スリーズ!{#3}@{#4}}}



\makeatother
\begin{document}
\hypertarget{preparations-chaudes-du-foie-gras}{%
\subsection{フォワグラの温製料理}\label{preparations-chaudes-du-foie-gras}}

フォワグラを丸ごとで温製料理として提供する場合には、丁寧に成形して筋とりをし、拍子木に切ったトリュフを刺す。このトリュフはあらかじめ皮を剥いて四つ割りにし、塩こしょうしてからローリエの葉を加えてグラス1杯のコニャックで表面を焦がさないよう焼き固める。陶製の容器に入れてきっちり密閉させて冷ましておくこと。

こうして準備したトリュフを拍子木に切ってフォワグラに刺したら、ごく薄い豚背脂のシートまたは豚の網脂で包んで、陶製の容器に入れてしっかり蓋をして加熱するまで数時間置くこと。

フォワグラを丸ごと温製料理として供するのにいちばんいいのは、生地で包んで焼くことだ。そうすれば生地が余分な脂が溶け出してくるのを吸い取ってくれる。具体的には次のように調理する\ldots{}\ldots{}

フォワグラよりひと回り大きなサイズの楕円型に伸したパテ用生地を2枚用意する。

生地の1枚の上に、豚背脂のシートで包んだフォワグラをのせる。可能なら周囲に皮を剥いた中位のサイズのトリュフを配する。ローリエの葉
\(\frac{1}{2}\) 枚をフォワグラの上にのせる。生地の縁を水で湿らせる。もう
1枚の生地をかぶせ、しっかりつなぎ合わせて2枚の生地のたるみを規則的に折っていく。

溶き卵を塗り、ナイフなどで筋模様を付ける。中央に、加熱中に発生する蒸気を抜くための穴を空ける。750〜800
gのフォワグラの場合、適温のオーブンで 40〜45分間焼く。

そのまま供する。フォワグラにふさわしいガルニチュールを別添で供する。

\hypertarget{ux7d66ux4ed5}{%
\paragraph{給仕}\label{ux7d66ux4ed5}}

ホールでメートルドテルがフォワグラを覆っている焼けた生地の下の縁をぐるりと切り離し、上を覆っている焼けた生地も取り去る。

スプーンを用いてフォワグラを切り分けて取り皿に盛り、メニューに記したガルニチュールを添えて供する。

\hypertarget{ux539fux6ce8}{%
\subparagraph{【原注】}\label{ux539fux6ce8}}

フォワグラを温製で供する場合にテリーヌ型で火を通すことには賛成しかねる。上記の方法こそ、どんな場合でも、どんなガルニチュールを添えるにしても、ずっと好ましいものだと考えている。

温製のフォワグラにガルニチュールとしてヌイユやラザーニャ、マカロニ、米をぜひとも添えるといい。これらのパスタは標準的なもの同様に白いものをクリームであえて仕上げること。フォワグラの味がいっそう引き立ち、しかも消化を助けてくれるものだ。

上記のパスタ類を別にすれば、温製のフォワグラにもっともいいガルニチュールはトリュフを丸ごとあるいはスライスして添えるか、\protect\hyperlink{garniture-a-la-financiere}{フィナンシエール}だ。茶色いソースとしては、\protect\hyperlink{sauce-madere}{ソース・マデール}がよく合う。ただしマデイラ酒はごく上質で酒精強化し過ぎていないものを用いること。

軽く仕上げた\protect\hyperlink{glaces-diverses}{仔牛か鶏のグラス}にバターを加え、シェリーの古酒もしくはポルトの古酒を少々加えたソースもまたよく合う。\protect\hyperlink{sauce-hongroise}{パプリカ風味のハンガリー風ソース}や、最上の仕上がりの\protect\hyperlink{sauce-supreme}{ソース・シュプレーム}もガルニチュールとも合うのなら、フォワグラのソースとしていいだろう。

一般的に、温製料理としては鵞鳥のフォワグラが好まれ、鴨のフォワグラは保存用や冷製に用いられる。

\hypertarget{foie-gras-cuit-dans-une-brioche}{%
\subsubsection{フォワグラのブリオシュ包み(ストラスブール風)}\label{foie-gras-cuit-dans-une-brioche}}

\frsub{Foie gras cuit dans une brioche}

フォワグラに拍子木に切ったトリュフを刺す。「\protect\hyperlink{preparations-chaudes-du-foie-gras}{フォワグラの温製料理}」で述べたようにして、陶製の器に入れて蓋をして1時間程置いておく。次に、フォワグラを豚背脂のシートで包む。これを中温のオーブンに入れて20分間、完全に火が通らない程度に加熱する。そのまま冷ましておく。

フォワグラの大きさに合ったタンバル型\footnote{円筒形で比較的高さの低い型。}にバターを塗り、砂糖を加えずに作った\protect\hyperlink{pate-a-brioche-commune}{標準的なブリオシュ生地}をかなり厚く
\ruby{伸}{の}して敷き詰める。

フォワグラを型に縦にして詰める。きっちり隙間がないか、ほとんどない位にすること。同じブリオシュ生地で蓋をし、中央に蒸気を抜くための穴を空ける。丈夫な紙を型の円周の長さの長方形に切ってにバターを塗り、型の上の方に貼り付ける。これはブリオシュ生地が溢れ出てしまわないようにするため。そのまま充分に暖かい場所に置いて生地を醗酵させる。中温のオーブンに入れて焼く。大きな針\footnote{金串や鶏を成形する際に用いるブリデ針を使う。}を中心まで刺して、きれいに何も付かずに抜けるようになったら焼成を終了する。

フォワグラの料理に一般的なガルニチュールを添えてそのまま供する。が、このように調理した場合は冷ましてから提供するのがほとんど。

\hypertarget{ux30d0ux30eaux30a8ux30fcux30b7ux30e7ux30f3}{%
\subparagraph{バリエーション}\label{ux30d0ux30eaux30a8ux30fcux30b7ux30e7ux30f3}}

\ldots{}\ldots{}上記のように下拵えしたフォワグラをボール形に成形して、それを砂糖抜きのブリオシュ生地で包み、フォワグラを成形した際の切り落としを周囲に配する。この生地を財布\footnote{札入れや小銭入れではなく、金貨などを入れる大きな財布のイメージ。}のような形にして閉じ、これを周囲に波模様の付いたブリオシュ型に入れる。上部に、真ん丸にしたブリオシュ生地をしっかり埋め込む。いわゆる「こぶのあるブリオシュ」の形にするわけだ。15分間醗酵させたら、溶き卵など\footnote{dorer
  (ドレ)\textgreater{} dorure
  (ドリュール)を塗ること。溶き卵のみの場合もあれば、水や牛乳などを加える場合もある。}を塗って高温のオーブンで焼く。(pp.644-645)

\hypertarget{potage-queue-de-boeuf-a-la-francaise}{%
\subsubsection{牛テールの澄んだポタージュ・フランス風}\label{potage-queue-de-boeuf-a-la-francaise}}

\frsub{Potage Queue de boeur à la française}

p.128

\hypertarget{moscovite-a-la-creme}{%
\subsubsection{モスコヴィット・アラクレーム}\label{moscovite-a-la-creme}}

\frsub{Moscovite à la crème}

\href{ex.Bavarois\%20を料理名としてどう扱うか要検討}{}

p.839

\hypertarget{souffle-de-mais-au-paprika}{%
\subsubsection{とうもろこしのスフレ・パプリカ風味}\label{souffle-de-mais-au-paprika}}

\frsub{Soufflé de maïs au paprika}

p.755

\hypertarget{cotelettes-de-saumon-pojarski}{%
\subsubsection{サーモンのコトレット・ポジャルスキ}\label{cotelettes-de-saumon-pojarski}}

\frsub{Côtelettes de Saumon Pojarski}

\href{ポジャールスキー?}{}

p.301

\hypertarget{godiveau}{%
\subsubsection{ゴディヴォ}\label{godiveau}}

\hypertarget{godiveau}{%
\subsection[ゴディヴォ/仔牛肉とケンネ脂のファルス]{\texorpdfstring{ゴディヴォ\footnote{ゴディヴォgodiveau
  はフランソワ・ラブレーの小説『ガルガンチュアとパンタグリュエル』の「第三の書」(1546年)が初出。原書の綴りは
  guodiveaulx。これは「アンドゥイエット(のようなもの)」と一般に解釈されている。ラブレーはこれに先立つ1534年「ガルガンチュア」(=第一の書)において
  gaudebillaux
  という表現を用いている。これについては「ゴドビヨとは、たっぷり肥育した牛のトリップ(胃と腸)のこと」と本文で説明している。これらを敷衍すると、ゴディヴォはもともと牛などの胃や腸を刻んで詰めた腸詰すなわちアンドゥイエットのことだった、と考えたくなっても不思議はない。しかし、たとえ16世紀のラブレーにおけるゴディヴォが当時アンドゥイエットと呼ばれるものとほぼ同じだったとしても、アンドゥイエット
  andouilette がアンドゥイユ andouille
  に縮小辞を付したものであることから、中世のアンドゥイユを確認する必要が出てくる。14世紀末に書かれた『ル・メナジエ・ド・パリ』においてアンドゥイユは確かに「細かく刻んだ胃や腸を、腸詰にする」という説明がまず出てくるが、その他に、牛の第1胃だけを詰めるもの、豚のコトレットを切り出した端肉を材料にするもの、胸腺肉やレバーを掃除した残りの肉を材料にするもの、が挙げられている(t.2,p.127)。これに従うなら、中世におけるアンドゥイユとは素材の定義があまりはっきりしていなかったもの、言える。ところが17世紀、ピエール・ド・リュヌ『新料理の本』(1660年)に「スペイン風アンドゥイエット」というレシピがある。概要を記すと、仔牛肉を細かく刻む。豚背脂少々、香草、卵黄、塩、こしょう、ナツメグ、粉にしたシナモンを加える。豚背脂のシートで巻いてアンドゥイエットの形状にする。串を刺してローストする。ローストする際に滴り落ちてくる肉汁は受け皿で受ける。火が通ったらその肉汁をかける。茹で卵の黄身8〜10個分と細かくおろしたパン粉を順につけて、しっかりした衣を作る。提供時にレモン汁と羊のジュをかけ、揚げたパセリを添える、というものだ。1693年刊マシアロ『宮廷および大ブルジョワ料理の本』では豚のアンドゥイユ、仔牛のアンドゥイユとともに、仔牛のアンドゥイエットというレシピが掲載されている。最後のものには材料として「細かく刻んだ仔牛肉、豚背脂、香草、卵黄、塩、こしょう、ナツメグ、シナモンを加えて作る」とある(pp.108-109)。また、1750年に出版された『食品、ワイン、リキュール事典』でも、アンドゥイエットは「細かく刻んだ仔牛肉を楕円形に巻いたもの」と定義されている。実際、17、18世紀の料理書に出てくるアンドゥイエットは腸詰であるかどうかは別にしても、仔牛肉を主材料にしたものが多い。18世紀ヴァンサン・ラ・シャペル『近代料理』第1巻のアンドゥイエットも細かく刻んだ仔牛肉を豚の腸に詰めて作る。さて、ゴディヴォに戻ると、17世紀、1653年刊の『フランスのパティスリの本』(ラ・ヴァレーヌが著者だと言われている)にはFaire
  un pasté de gaudiueau
  「ゴディヴォのパテの作り方」という節があり、仔牛腿肉あるいは他の肉と脂身を細かく刻んだもの、をパテ(≒パイ包み焼き)に入れる。つまりここでも「仔牛腿肉」の使用が前提となっている。したがって、これら勘案すれば、ラブレーのゴディヴォもまた仔牛肉を材料にしていたものだった可能性は充分に考えられるだろう。
  もちろんゴドビヨという別の巻で出てくる名詞との関連性は無視出来ないものだが、中世〜ルネサンス期において、食にかかわる名詞、概念がしばしば曖昧だったことを考えると、多少のわかりにくさは許容せざるを得ない。したがって、本書において仔羊腿肉とケンネ脂を使うゴディヴォを「古典的」なファルスとして扱っているのはまことに正鵠を射ていると言えよう。}/仔牛肉とケンネ脂のファルス}{ゴディヴォ/仔牛肉とケンネ脂のファルス}}\label{godiveau}}

\frsecb{Farce de Veau à la Graisse de boeuf, ou Godiveau}

\index{farce!veau graisse de boeuf@--- de veau à la graisse de boeuf}
\index{garniture!farce!veau graisse de boeuf@Farce de veau à la graisse de boeuf}
\index{farce!veau glodiveau@Godiveau}
\index{garniture!farce!godiveau@Godiveau}
\index{かるにちゆーる@ガルニチュール!ふあるす@ファルス!こうしにくとけんねあふらのふあるす@仔牛肉とケンネ脂のファルス/ゴディヴォ}
\index{ふあるす@ファルス!こうしにくとけんねあふらのふあるす@牛仔牛肉とケンネ脂の---/ゴディヴォ}
\index{かるにちゆーる@ガルニチュール!ふあるす@ファルス!こていうお@ゴディヴォ}
\index{ふあるす@ファルス!こていうお@ゴディヴォ} \index{godiveau}
\index{こていうお@ゴディヴォ}

\hypertarget{godiveau-mouille-a-la-glace}{%
\subsubsection[A. 氷を入れて作るゴディヴォ]{\texorpdfstring{A.
氷を入れて作るゴディヴォ\footnote{氷を入れて作る方法についてはカレームが1815年刊『パリ風パティスリの本』の「シブレット入りゴディヴォ」原注において詳しく論じている。「不思議なことだが、氷を入れることでゴディヴォが滑らかなテクスチュアになり、素晴しくふんわりとしてとてもいい柔らかさに仕上がる。ゴディヴォが変質してしまうと、部分的とはいえそのクオリティはまったく失なわれてしまう。これは夏によく起こる事で、あまりに暑いとその熱で牛脂が仔牛肉としっかりつながらなくなってしまうからだ。一方(仔牛肉)は水分を含んでいて、もう一方(牛脂)は脂質そのものだからだ。だから、夏の暑い時期には必ず氷を加えて作るべきであり、逆に冬場はそこまでする必要はない(p.142)」。ほぼ同時期のヴィアール『王国料理の本』
  1817年版においてゴディヴォのレシピの末尾に、「夏に、水の代わりに少量でも氷を使えるならそのほうがずっといい仕上がりになる(p.145)」と書かれている。これは、製氷機、冷凍庫が実用化されるのが19世紀中頃なので、それよりやや早い時代ということになり、カレームの主たる活躍の舞台であった食卓外交というものが、いかに贅沢だったかを示しているとも言えよう。言うまでもなく、17〜18世紀の料理書、パティスリの本においてゴディヴォのレシピは多く見られるが、氷の使用について言及したものはいまのところ見つかっていない。}}{A. 氷を入れて作るゴディヴォ}}\label{godiveau-mouille-a-la-glace}}

\frsub{Godiveau mouillé à la glace}

\index{farce@farce!godiveau a@Godiveau A. Godeiveau mouillé à la glace}
\index{ふあるす@ファルス!こていうお@ゴディヴォ!a@A. 氷を入れて作るゴディヴォ}
\index{godiveau@godiveau!a@A. --- mouillé à la glace}
\index{こていうお@ゴディヴォ!a@A. 氷を入れて作る---}

\begin{itemize}
\item
  材料\ldots{}\ldots{}筋をきれいに取り除いた仔牛腿肉1
  kg、\ul{水気を含んでいない}牛ケンネ脂\footnote{腎臓の周囲を厚く覆っている脂肪。融解温度が低く、精製して牛脂(ヘット)の原料となる。}1.5
  kg、全卵8個、塩25 g、白こしょう5 g、ナツメグ1 g、透明な氷7〜800
  gまたは氷水7〜8 dL。
\item
  作業手順\ldots{}\ldots{}はじめに、仔牛肉とケンネ脂を別々に、細かく刻む。仔牛肉はさいの目に切り、調味料と合わせておく。牛脂は細かくして、薄皮は筋はきれいに取り除いておく。
\end{itemize}

仔牛肉と牛脂を別々の鉢に入れて、それぞれすり潰す。次にこれらを合わせてから、完全に混ざり合って一体化するまでよくすり潰し、卵を一個ずつ、すり潰す作業を止めずに加えていく。

裏漉しして、平皿に\footnote{大きなバット。}広げ、氷の上に置いて翌日まで休ませる。

翌日になったら、再度ファルスをすり潰す。この時、小さく割った氷を少しずつ加えていき、よく混ぜ合わせる。

ゴディヴォに氷を加え終わったら、必ずテスト\footnote{少量を、沸騰しない程度の温度で火を通し(ポシェ)て様子を見ること。}を行ない、必要に応じて修正する。固すぎるようなら水を少々加え、柔らかすぎるようなら卵白を少し加えること。

\hypertarget{nota-godiveau-a}{%
\subparagraph{【原注】}\label{nota-godiveau-a}}

ゴディヴォで作ったクネルはもっぱら、\protect\hyperlink{vol-au-vent}{ヴォロヴァン}の詰め物\footnote{原文
  garniture ガルニチュールの意味が広いことに注意。}にしたり、牛、羊の塊肉の料理に添える\protect\hyperlink{garniture-a-la-financiere}{ガルニチュール・フィナンシエール}に用いられる。

他のクネルがどれもそうであるように、沸騰しない程度の温度で茹でて\footnote{pocher
  (ポシェ)。}
火を通せばいいが、一般的には手で整形して塩を加えた沸騰しない程度の温度の湯で茹でる。

だが、「ポシャジャセック\footnote{pochage à sec
  直訳すると「乾燥した状態でポシェすること」。つまり水(湯)を用いずに、pocher
  と同様に低めの温度で加熱することを指している。}」と呼ばれる技法、すなわち弱火のオーブンで焼くのがいちばんいい。

以下に示す方法はとても短時間で出来るので特にお勧めだ。

ゴディヴォは充分に氷を加えて水気を含んだ状態にしておく。オーブンの天板に敷いたバターを塗った紙の上に、丸口金を付けた絞り袋から絞り出す。オーブンの天板にもバターを塗っておくこと。絞り出したクネルは触れ合うようにしていい。

これを低温のオーブンに入れて加熱する。

7〜8分すると、クネルの表面に脂が水滴状に浸み出してくる。これが、ちょうどいい具合に火が通った合図だ。オーブンから出して、クネルを別の銀製の盆か大理石の板の上に裏返しに広げる。クネルが\ruby{微温}{ぬる}くなるまで冷めたら、敷いてあった紙を端のほうから引き剥して取り除く。

クネルは完全に冷めるまで放置し、その後に皿に移すか、可能なら柳編みのすのこに載せてやるのがいい。

\hypertarget{godiveau-a-la-creme}{%
\subsubsection{B. 生クリーム入りゴディヴォ}\label{godiveau-a-la-creme}}

\frsub{Godiveau à la crème}

\index{farce@farce!godiveau b@Godiveau B. Godeiveau  à la crème}
\index{ふあるす@ファルス!こていうお@ゴディヴォ!b@B. 生クリーム入りゴディヴォ}
\index{godiveau@godiveau!b@B. --- à la crème}
\index{こていうお@ゴディヴォ!b@B. 生クリーム入り---}

\begin{itemize}
\item
  材料\ldots{}\ldots{}筋をきれいに取り除いた極上の白さの仔牛腿肉1
  kg、水気を含んでいない牛ケンネ脂1 kg、全卵4個、卵黄3個、生クリーム7
  dL、塩25 g、白こしょう5 g、ナツメグ1 g。
\item
  作業手順\ldots{}\ldots{}仔牛肉とケンネ脂は別々に、細かく刻む。これらを鉢に入れて合わせ、調味料、全卵、卵黄をひとつずつ加えながら、力強く全体をすり潰し、完全に一体化させる。
\end{itemize}

裏漉しして、天板に広げる。氷の上にのせて翌日まで休ませる。

翌日になったら、あらかじめ中に氷を入れて冷やしておいた鉢で再度すり潰す。この際に生クリームを少量ずつ加えていく。

クネルを整形する前にテストをして、必要があれば固さなどを修正してやること。
\end{document}