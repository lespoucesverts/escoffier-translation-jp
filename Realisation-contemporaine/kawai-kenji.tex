\input{../preamble/preamble-a4-ltjsarticle.tex}
\begin{document}
\hypertarget{preparations-chaudes-du-foie-gras}{%
\subsection{フォワグラの温製料理}\label{preparations-chaudes-du-foie-gras}}

\index{foie gras@foie gras!preparations chaudes!Préparations chaudes du ---}
\index{ふおわくら@フォワグラ!おんせい@---の温製料理}

フォワグラを丸ごとで温製料理として提供する場合には、丁寧に掃除をして筋とりをし、拍子木に切ったトリュフを刺す。このトリュフはあらかじめ皮を剥いて櫛切りにし、塩こしょうしてからローリエの葉を加えてグラス1杯のコニャックで表面を焦がさないよう焼き固めておく。陶製の容器に入れてきっちり密閉させて冷ましておくこと。

こうして準備したトリュフをフォワグラに刺したら、ごく薄い豚背脂のシートまたは豚の網脂で包んで、陶製の容器に入れてしっかり蓋をして加熱するまで数時間置くこと。

フォワグラを丸ごと温製料理として供するのにいちばんいいのは、生地で包んで焼くことだ。そうすれば生地が余分な脂が溶け出してくるのを吸い取ってくれる。具体的には次のように調理する\ldots{}\ldots{}

フォワグラよりひと回り大きなサイズの楕円型に伸したパテ用生地を2枚用意する。

生地の1枚の上に、豚背脂のシートで包んだフォワグラをのせる。可能なら周囲に皮を剥いた中位のサイズのトリュフを配する。ローリエの葉
\(\frac{1}{2}\) 枚をフォワグラの上にのせる。生地の縁を水で湿らせる。もう
1枚の生地をかぶせ、しっかりつなぎ合わせて2枚の生地のたるみを規則的に折っていく。

溶き卵を塗り、ナイフなどで筋模様を付ける。中央に、加熱中に発生する蒸気を抜くための穴を空ける。750〜800
gのフォワグラの場合、適温のオーブンで 40〜45分間焼く。

そのまま供する。フォワグラにふさわしいガルニチュールを別添で供する。

\hypertarget{service-des-preparations-chaudes-du-foie-gras}{%
\subparagraph{給仕}\label{service-des-preparations-chaudes-du-foie-gras}}

ホールでメートルドテルがフォワグラを覆っている焼けた生地の下の縁をぐるりと切り離し、上を覆っている焼けた生地も取り去る。

スプーンを用いてフォワグラを切り分けて取り皿に盛り、メニューに記したガルニチュールを添えて供する。

\hypertarget{nota-preparations-chaudes-du-foie-gras}{%
\subparagraph{【原注】}\label{nota-preparations-chaudes-du-foie-gras}}

フォワグラを温製で供する場合にテリーヌ型で火を通すことには賛成しかねる。上記の方法こそ、どんな場合でも、どんなガルニチュールを添えるにしても、ずっと好ましいものだと考えている。

温製のフォワグラにガルニチュールとしてヌイユやラザーニャ、マカロニ、米をぜひとも添えるといい。これらのパスタは標準的なもの同様に白いものをクリームであえて仕上げること。フォワグラの味がいっそう引き立ち、しかも消化を助けてくれるものだ。

上記のパスタ類を別にすれば、温製のフォワグラにもっともいいガルニチュールはトリュフを丸ごとあるいはスライスして添えるか、\protect\hyperlink{garniture-a-la-financiere}{フィナンシエール}だ。茶色いソースとしては、\protect\hyperlink{sauce-madere}{ソース・マデール}がよく合う。ただしマデイラ酒はごく上質で酒精強化し過ぎていないものを用いること。

軽く仕上げた\protect\hyperlink{glaces-diverses}{仔牛か鶏のグラス}にバターを加え、シェリーの古酒もしくはポルトの古酒を少々加えたソースもまたよく合う。\protect\hyperlink{sauce-hongroise}{パプリカ風味のハンガリー風ソース}や、最上の仕上がりの\protect\hyperlink{sauce-supreme}{ソース・シュプレーム}もガルニチュールとも合うのなら、フォワグラのソースとしていいだろう。

一般的に、温製料理としては鵞鳥のフォワグラが好まれ、鴨のフォワグラは保存用や冷製に用いられる。

\hypertarget{foie-gras-cuit-dans-une-brioche}{%
\subsubsection{フォワグラのブリオシュ包み(ストラスブール風)}\label{foie-gras-cuit-dans-une-brioche}}

\frsub{Foie gras cuit dans une brioche}

\index{foie gras@foie gras!cuit brioche@--- cuit dans une brioche}
\index{foie gras@foie gras!strasbourgeoise@--- à la Strasbourgeoise}
\index{ふおわくら@フォワグラ!ふりおしゆつつみ@---のブリオシュ包み}
\index{ふおわくら@フォワグラ!すとらすふーるふう@---のブリオシュ包み(ストラスブール風)}
\index{ふりおしゆ@ブリオシュ!ふおわくら@フォワグラの---包み}
\index{すとらすふーるふう@ストラスブール風!ふおわくら@フォワグラのブリオシュ包み(ストラスブール風)}

フォワグラに拍子木に切ったトリュフを刺す\footnote{原文 clouter
  (クルテ)楔状のものを刺す、が原義だが、フォワグラにトリュフを丸ごとあるいは櫛切りなどの形状にして射込むという方法もあるだろう。}。「\protect\hyperlink{preparations-chaudes-du-foie-gras}{フォワグラの温製料理}」で述べたようにして、陶製の器に入れて蓋をして1時間程置いておく。次に、フォワグラを豚背脂のシートで包む\footnote{『ラルース・ガストロノミック』初版では「豚背脂の薄いシートか網脂」(p.477)。}。これを中温のオーブンに入れて20分間、完全に火が通らない程度に加熱する。そのまま冷ましておく。

フォワグラの大きさに合ったタンバル型\footnote{円筒形で比較的高さの低い型。}にバターを塗り、砂糖を加えずに作った\protect\hyperlink{pate-a-brioche-commune}{標準的なブリオシュ生地}をかなり厚く
\ruby{伸}{の}して敷き詰める。

フォワグラを型に縦にして詰める。きっちり隙間がないか、ほとんどない位にすること。同じブリオシュ生地で蓋をし、中央に蒸気を抜くための穴を空ける。丈夫な紙を型の円周の長さの長方形に切ってにバターを塗り、型の上の方に貼り付ける。これはブリオシュ生地が溢れ出てしまわないようにするため。そのまま充分に暖かい場所に置いて生地を醗酵させる。中温のオーブンに入れて焼く。大きな針\footnote{金串や、鶏を成形する際に用いるブリデ針を使う。}を中心まで刺して、きれいに何も付かずに抜けるようになったら焼成を終了する。

フォワグラの料理に一般的なガルニチュールを添えてそのまま供する。が、このように調理した場合は冷ましてから提供するのがほとんど。

\hypertarget{ux30d0ux30eaux30a8ux30fcux30b7ux30e7ux30f3}{%
\subparagraph{バリエーション}\label{ux30d0ux30eaux30a8ux30fcux30b7ux30e7ux30f3}}

\ldots{}\ldots{}上記のように下拵えしたフォワグラをボール形に成形して、それを砂糖抜きのブリオシュ生地で包み、フォワグラを成形した際の切り落としを周囲に配する。この生地を財布\footnote{札入れや小銭入れではなく、金貨などを入れる大きな財布のイメージ。}のような形にして閉じ、これを周囲に波模様の付いたブリオシュ型に入れる。上部に、真ん丸にしたブリオシュ生地をしっかり埋め込む。いわゆる「こぶのあるブリオシュ」の形にするわけだ。15分間醗酵させたら、溶き卵など\footnote{dorer
  (ドレ)\textgreater{} dorure
  (ドリュール)を塗ること。溶き卵のみの場合もあれば、水や牛乳などを加える場合もある。}を塗って高温のオーブンで焼く。(pp.644-645)

\hypertarget{potage-queue-de-boeuf-a-la-francaise}{%
\subsubsection{牛テールの澄んだポタージュ・フランス風}\label{potage-queue-de-boeuf-a-la-francaise}}

\frsub{Potage Queue de boeur à la française}

\index{potage@potage!queue boeuf francaisel!--- Queue de boeuf à la française}
\index{queue boeuf@queue de boeuf!potage --- à la française}
\index{すんたほたーしゆ@澄んだポタージュ!うしてーる@牛テールの---}
\index{うしてーる@牛テール!すんたほたーしゆ@---の澄んだポタージュ}

(大鍋\footnote{marmite (マルミット)大きな寸胴鍋。}で作る)

関節のところでぶつ切りにした牛テール1 kg、仔牛のすね肉500
g、\protect\hyperlink{consomme-blanc-simple}{白いコンソメ}2
\(\frac{1}{2}\) L。これらを火にかけてアクを取り\footnote{écumer
  (エキュメ)本来は浮いてくる細かい泡を取り除く、の意。}、標準的な香味野菜\footnote{にんじん、蕪、玉ねぎ、セロリ、サヴォイキャベツなど。}を加える。弱火で5時間煮込む。

脂身を含まない牛肉、仔牛肉をミンチにしてバターで炒め、アロールート大さじ1
\(\frac{1}{2}\)
杯をふりかけたものをコンソメに加えてクラリフィエする\footnote{原文ではクラリフィエに卵白の使用が言及されていないので、このレシピのまま作るのであれば、このクラリフィエは「澄ませる」というよりは、風味と色合いを強化させるのが主目的と考えられよう。}。

ガルニチュール\ldots{}\ldots{}ぶつ切りにして掃除した牛の尾、にんにくの形状に成形し、コンソメで煮たにんじんと蕪。

\hypertarget{ux539fux6ce8}{%
\subparagraph{【原注】}\label{ux539fux6ce8}}

このポタージュは古典料理では「グランオシュポ\footnote{原文 Grand
  Hochepot (グランオシュポ)。hochepot
  は有音のhで始まるのでリエゾンしない。}」すなわち牛テールをコンソメで煮たポタージュと呼ばれ、ガルニチュールにはコンソメに用いたにんじんと蕪を薄切りにして肉のブイヨンで煮たものが添えられた。(p.128)

\hypertarget{moscovite}{%
\subsection{モスコヴィット(旧名バヴァロワ)}\label{moscovite}}

\href{ex.Bavarois\%20を料理名としてどう扱うか要検討}{}

モスコヴィットの作り方は2種、(1)モスコヴィット・アラクレーム、(2)果物のモスコヴィット。

\hypertarget{moscovite-a-la-creme}{%
\subsubsection{モスコヴィット・アラクレーム}\label{moscovite-a-la-creme}}

\frsub{Moscovite à la crème}

\index{moscovite@moscovite!creme@--- à la Crème}
\index{creme@crème (à la)!moscovite@mostovite à la ---}
\index{もすこういつと@モスコヴィット!あらくれーむ@---・アラクレーム}
\index{あらくれーむ@アラクレーム!もつこういつと@モスコヴィット・---}

\textbf{アラクレームのアパレイユ}\ldots{}\ldots{}片手鍋に粉砂糖500
gと卵黄16個を入れてしっかりとよく混ぜる。沸かした牛乳1
Lでのばす。牛乳はあらかじめヴァニラ1本で香りを煮出しておくこと。冷水に浸しておいた板ゼラチン25
gを加える。弱火にかけて、スプーンがコーティングされる状態になるまで加熱するが、沸騰させないこと。

釉のかかった陶製の容器に、シノワで漉し入れる。時々混ぜながら冷まし、粘り気が出てきたら泡立てた生クリーム1
Lと粉砂糖100 g、ヴァニラシュガー25 gを混ぜ込む。

\hypertarget{moscovite-aux-fruits}{%
\subsubsection{果物のモスコヴィット}\label{moscovite-aux-fruits}}

\textbf{果物のモスコヴィットのアパレイユ}\ldots{}\ldots{}果物のピュレ5
dLを30°Béのシロップでのばす。レモン果汁3個分を加える。30gの粉ゼラチンを溶かして布で漉し入れる。泡立てた生クリーム
\(\frac{1}{2}\) L を加える。

果物のモスコヴィットのアパレイユにはピュレに用いたのと同じ果物を加えてもいい。苺、フランボワーズ、グロゼイユ\footnote{すぐり。}などの場合は\ruby{生}{な
ま}のまま加える。洋梨、桃、アプリコットなど繊維質の果肉のものはシロップで沸騰させないよう加熱してから加えること。

\hypertarget{moulage-et-dressage-des-moscovites}{%
\subsubsection{モスコヴィットの型詰めと盛り付け}\label{moulage-et-dressage-des-moscovites}}

モスコヴィットは通常、中央に穴の空くタイプの型に、スイートアーモンド油を軽く塗ってからアパレイユを流し込む。円形に切った白紙でアパレイユに蓋をしてから、型を砕いた氷に埋め、冷し固める。

提供直前に、型をさっとぬるま湯に浸して、皿に逆さにして型から外す。皿は折り畳んだナフキンを敷いおくといいが、そうでなくてもいい。

型に油を塗るのではなく、砂糖をブロンド色のカラメル状にして塗ってもいい。そうするとデザートとしての見た目もよく、味わいも素晴しいものになる。

もうひとつ、お勧めの方法がある。それは、野菜料理用などの銀製の深皿にアパレイユを流し入れて周囲を氷で覆うというものだ。この場合は型から外さないので、アパレイユはかなり柔らかく作ることが可能となり、繊細な口あたりに仕上げることが出来る。

この最後の方法を採る場合には、果物のコンポートか生の果物のマセドワーヌ
\footnote{macédoine
  いろいろなものを混ぜた結果、が原義。通常はさいの目に切った何種類かの野菜をマヨネーズなどのソースであえたものや、フレッシュな果物数種をさいの目に切ってシロップをかけたものを指すことが多い。ただし、さいの目に切ることは「用語」の概念には含まれていないので、さいの目にするのはあくまでも慣習によるもので、切り方は自由と考えてもいい。}を添えることがある。だが、こうした果物の添え物はむしろ冷製のプディングにふさわしい。とはいえ、冷製プディングとモスコヴィットはよく似たものとは言えるが。

最後に、型に詰める場合でもそうでない場合でも、提供直前に、絞り袋で、真っ白またはロゼ色の\protect\hyperlink{creme-chantilly}{クレームシャンティイ}で装飾してやること。(p.839)

\hypertarget{mais}{%
\subsection{とうもろこし}\label{mais}}

とうもろこしはフレッシュでジューシーなものを選ぶこと。皮を付けたまま蒸すか塩茹でする。

火が通ったら花びらを外すように皮を剥き、丸ごと供するならば皮を取り除く。その場合はとうもろこしをナフキンの上にのせて、オードブル皿にフレッシュなバターを添えて供する。グリルする場合は、網にのせてオーブンに入れ、粒が膨らんでこんがり焼けたら、粒を外してナフキンの上に盛る。

とうもろこしは穂を丸ごと供する場合もある。

とうもろこしを付け合わせとして供する場合には、粒を穂から外して、プチポワと同様に、バターか生クリームであえること。

生のとうもろこしが手に入らない場合は、良質の瓶詰め、缶詰が市販されている。

\hypertarget{souffle-de-mais-a-la-creme}{%
\subsubsection{とうもろこしのスフレ・アラクレーム}\label{souffle-de-mais-a-la-creme}}

\frsub{Soufflé de maïs à la crème}

\index{mais@maïs!souffle creme!Soufflé de --- à la crème}
\index{souffle@soufflé!mais creme@--- de maïs à la crème}
\index{creme@creme (à la)!souffle mais@Soufflé de maïs à la ---}
\index{とうもろこし@とうもろこし!すふれくりーむ@---のスフレ・アラクレーム}
\index{すふれ@スフレ!とうもろこしくれーむ@とうもろこしの---・アラクレーム}
\index{なまくりーむ@生クリーム ⇒ アラクレーム!とうもろこしすふれ@とうもろこしのスフレ・アラ---}

とうもろこしは茹でるか蒸して火を通す。手早く裏漉しして出来たピュレを鍋に入れ、バター1かけらを加える。強火にかけて余計な水分をとばす。生クリームを加えて柔らかな生地状にする。このピュレ500
gあたり卵黄3個を加え、固く泡立てた卵白4個分を混ぜ込む。型に流し込み、標準的なスフレと同様に焼く。

\hypertarget{souffle-de-mais-au-paprika}{%
\subsubsection{とうもろこしのスフレ・パプリカ風味}\label{souffle-de-mais-au-paprika}}

\frsub{Soufflé de maïs au paprika}

\index{mais@maïs!souffle paprika!Soufflé de --- au paprika}
\index{souffle@soufflé!mais paprika@--- de maïs au paprika}
\index{paprika@paprika!souffle mais@Soufflé de maïs au ---}
\index{とうもろこし@とうもろこし!すふれはふりか@---のスフレ・パプリカ風味}
\index{すふれ@スフレ!とうもろこしはふりか@とうもろこしの---・パプリカ風味}
\index{はふりか@パプリカ!とうもろこしすふれ@とうもろこしのスフレ・---風味}

とうもろこしを裏漉しする前に、バターで色よく炒めた玉ねぎのみじん切り大さじ2杯と、とうもろこし500gあたりパプリカ1つまみ強を加える。

その後の手順は標準的なスフレと同様にする。

\hypertarget{ux539fux6ce8-1}{%
\subparagraph{【原注】}\label{ux539fux6ce8-1}}

この2種のスフレはガルニチュールとして供してもいいし、好みに合わせてタンバル型や小さなスフレ型で焼いて供してもいい。茹でた鶏の大皿仕立ての料理にいい付け合わせとなる。(p.755)

\hypertarget{cotelettes-de-saumon-pojarski}{%
\subsubsection[サーモンのコトレット・ポジャルスキ]{\texorpdfstring{サーモンのコトレット・ポジャルスキ\footnote{19世紀初頭にパリにあった宿屋兼食堂の名。ロシア皇帝1世がフランス訪問時におしのびで気まぐれにこの店に立ち寄り、仔牛肉をミンチにして再度コトレットの形状にした\protect\hyperlink{cote-de-veau-pojarski}{仔牛のコトレット・ポジャルスキ}を皇帝に出したところ大いに気に入られ、名物料理となったと言われている。}}{サーモンのコトレット・ポジャルスキ}}\label{cotelettes-de-saumon-pojarski}}

\frsub{Côtelettes de Saumon Pojarski}

きれいに掃除したサーモンの尾の身500 gを包丁で粗く刻む。冷たいバター125
gと、牛乳に浸して絞ったパンの身125
gを加える。全体をさらに刻み、滑らかで均質なアパレイユにする。塩、こしょう、ナツメグで味を調える。

これを10等分し、打ち粉をした台の上でコトレット\footnote{Côtelette
  (コトレット)仔牛、羊の骨付き背肉を肋骨1本ごとにカットしたもの。骨を除去した形状でも同じ名称。日本語の「カツレツ」の語源となった。}の形状にする
\footnote{コトレット型 moule à côtelette
  というものが市販されており、近年ではシリコンゴム製のものもあるが、薄い銅板やステンレス板から自作することも可能。この型の出来如何で、このレシピの最後に指示されている「チャップ花」を付けられるかどうか、そういう装飾が意味を持ち得るかどうかが決まるといってもいいだろう。}。これらを提供直前に、澄ましバターで両面こんがりと焼く。皿に環状に盛り付け\footnote{こういった盛り付け方法を採る場合には、中央にガルニチュールを盛り込むことが多い。}、チャップ花を飾る。

\textbf{ガルニチュール}\ldots{}\ldots{}小エビの尾の身、牡蠣、ムール貝、マッシュルーム、バターで蒸し煮したきゅうり\footnote{concombre
  (コンコンブル)日本では中国経由で導入された華南系(もっとも一般的な地這系であまりイボが尖っていないタイプ)、華北系(イボの尖った四葉(スーヨー)が代表的)の2系統が代表的だが、ヨーロッパではまた別の系統がいくつかあり、通常は直径4〜5
  cm程度、品種によるが長さ30〜50cm程度にまで大きくしてから収穫し、種子の部分を取り除いて加熱調理することが多い。もちろん輪切りにして生野菜として食べることも多い。}、フレッシュなプチポワなど。

\textbf{ソース}\ldots{}\ldots{}魚用ソースならどれでも合うが、とりわけ\protect\hyperlink{sauce-vin-blanc}{白ワインソース}、\protect\hyperlink{sauce-a-la-new-berg-cru}{ニューバーグ}、\protect\hyperlink{sauce-americaine}{アメリケーヌ}がいい。(p.301)

\hypertarget{godiveau}{%
\subsection[ゴディヴォ/仔牛肉とケンネ脂のファルス]{\texorpdfstring{ゴディヴォ\footnote{ゴディヴォgodiveau
  はフランソワ・ラブレーの小説『ガルガンチュアとパンタグリュエル』の「第三の書」(1546年)が初出。原書の綴りは
  guodiveaulx。これは「アンドゥイエット(のようなもの)」と一般に解釈されている。ラブレーはこれに先立つ1534年「ガルガンチュア」(=第一の書)において
  gaudebillaux
  という表現を用いている。これについては「ゴドビヨとは、たっぷり肥育した牛のトリップ(胃と腸)のこと」と本文で説明している。これらを敷衍すると、ゴディヴォはもともと牛などの胃や腸を刻んで詰めた腸詰すなわちアンドゥイエットのことだった、と考えたくなっても不思議はない。しかし、たとえ16世紀のラブレーにおけるゴディヴォが当時アンドゥイエットと呼ばれるものとほぼ同じだったとしても、アンドゥイエット
  andouilette がアンドゥイユ andouille
  に縮小辞を付したものであることから、中世のアンドゥイユを確認する必要が出てくる。14世紀末に書かれた『ル・メナジエ・ド・パリ』においてアンドゥイユは確かに「細かく刻んだ胃や腸を、腸詰にする」という説明がまず出てくるが、その他に、牛の第1胃だけを詰めるもの、豚のコトレットを切り出した端肉を材料にするもの、胸腺肉やレバーを掃除した残りの肉を材料にするもの、が挙げられている(t.2,p.127)。これに従うなら、中世におけるアンドゥイユとは素材の定義があまりはっきりしていなかったもの、言える。ところが17世紀、ピエール・ド・リュヌ『新料理の本』(1660年)に「スペイン風アンドゥイエット」というレシピがある。概要を記すと、仔牛肉を細かく刻む。豚背脂少々、香草、卵黄、塩、こしょう、ナツメグ、粉にしたシナモンを加える。豚背脂のシートで巻いてアンドゥイエットの形状にする。串を刺してローストする。ローストする際に滴り落ちてくる肉汁は受け皿で受ける。火が通ったらその肉汁をかける。茹で卵の黄身8〜10個分と細かくおろしたパン粉を順につけて、しっかりした衣を作る。提供時にレモン汁と羊のジュをかけ、揚げたパセリを添える、というものだ。1693年刊マシアロ『宮廷および大ブルジョワ料理の本』では豚のアンドゥイユ、仔牛のアンドゥイユとともに、仔牛のアンドゥイエットというレシピが掲載されている。最後のものには材料として「細かく刻んだ仔牛肉、豚背脂、香草、卵黄、塩、こしょう、ナツメグ、シナモンを加えて作る」とある(pp.108-109)。また、1750年に出版された『食品、ワイン、リキュール事典』でも、アンドゥイエットは「細かく刻んだ仔牛肉を楕円形に巻いたもの」と定義されている。実際、17、18世紀の料理書に出てくるアンドゥイエットは腸詰であるかどうかは別にしても、仔牛肉を主材料にしたものが多い。18世紀ヴァンサン・ラ・シャペル『近代料理』第1巻のアンドゥイエットも細かく刻んだ仔牛肉を豚の腸に詰めて作る。さて、ゴディヴォに戻ると、17世紀、1653年刊の『フランスのパティスリの本』(ラ・ヴァレーヌが著者だと言われている)にはFaire
  un pasté de gaudiueau
  「ゴディヴォのパテの作り方」という節があり、仔牛腿肉あるいは他の肉と脂身を細かく刻んだもの、をパテ(≒パイ包み焼き)に入れる。つまりここでも「仔牛腿肉」の使用が前提となっている。したがって、これら勘案すれば、ラブレーのゴディヴォもまた仔牛肉を材料にしていたものだった可能性は充分に考えられるだろう。
  もちろんゴドビヨという別の巻で出てくる名詞との関連性は無視出来ないものだが、中世〜ルネサンス期において、食にかかわる名詞、概念がしばしば曖昧だったことを考えると、多少のわかりにくさは許容せざるを得ない。したがって、本書において仔羊腿肉とケンネ脂を使うゴディヴォを「古典的」なファルスとして扱っているのはまことに正鵠を射ていると言えよう。}/仔牛肉とケンネ脂のファルス}{ゴディヴォ/仔牛肉とケンネ脂のファルス}}\label{godiveau}}

\frsecb{Farce de Veau à la Graisse de boeuf, ou Godiveau}

\index{farce!veau graisse de boeuf@--- de veau à la graisse de boeuf}
\index{garniture!farce!veau graisse de boeuf@Farce de veau à la graisse de boeuf}
\index{farce!veau glodiveau@Godiveau}
\index{garniture!farce!godiveau@Godiveau}
\index{かるにちゆーる@ガルニチュール!ふあるす@ファルス!こうしにくとけんねあふらのふあるす@仔牛肉とケンネ脂のファルス/ゴディヴォ}
\index{ふあるす@ファルス!こうしにくとけんねあふらのふあるす@牛仔牛肉とケンネ脂の---/ゴディヴォ}
\index{かるにちゆーる@ガルニチュール!ふあるす@ファルス!こていうお@ゴディヴォ}
\index{ふあるす@ファルス!こていうお@ゴディヴォ} \index{godiveau}
\index{こていうお@ゴディヴォ}

\hypertarget{godiveau-mouille-a-la-glace}{%
\subsubsection[A. 氷を入れて作るゴディヴォ]{\texorpdfstring{A.
氷を入れて作るゴディヴォ\footnote{氷を入れて作る方法についてはカレームが1815年刊『パリ風パティスリの本』の「シブレット入りゴディヴォ」原注において詳しく論じている。「不思議なことだが、氷を入れることでゴディヴォが滑らかなテクスチュアになり、素晴しくふんわりとしてとてもいい柔らかさに仕上がる。ゴディヴォが変質してしまうと、部分的とはいえそのクオリティはまったく失なわれてしまう。これは夏によく起こる事で、あまりに暑いとその熱で牛脂が仔牛肉としっかりつながらなくなってしまうからだ。一方(仔牛肉)は水分を含んでいて、もう一方(牛脂)は脂質そのものだからだ。だから、夏の暑い時期には必ず氷を加えて作るべきであり、逆に冬場はそこまでする必要はない(p.142)」。ほぼ同時期のヴィアール『王国料理の本』
  1817年版においてゴディヴォのレシピの末尾に、「夏に、水の代わりに少量でも氷を使えるならそのほうがずっといい仕上がりになる(p.145)」と書かれている。これは、製氷機、冷凍庫が実用化されるのが19世紀中頃なので、それよりやや早い時代ということになり、カレームの主たる活躍の舞台であった食卓外交というものが、いかに贅沢だったかを示しているとも言えよう。言うまでもなく、17〜18世紀の料理書、パティスリの本においてゴディヴォのレシピは多く見られるが、氷の使用について言及したものはいまのところ見つかっていない。}}{A. 氷を入れて作るゴディヴォ}}\label{godiveau-mouille-a-la-glace}}

\frsub{Godiveau mouillé à la glace}

\index{farce@farce!godiveau a@Godiveau A. Godeiveau mouillé à la glace}
\index{ふあるす@ファルス!こていうお@ゴディヴォ!a@A. 氷を入れて作るゴディヴォ}
\index{godiveau@godiveau!a@A. --- mouillé à la glace}
\index{こていうお@ゴディヴォ!a@A. 氷を入れて作る---}

\begin{itemize}
\item
  材料\ldots{}\ldots{}筋をきれいに取り除いた仔牛腿肉1
  kg、\ul{水気を含んでいない}牛ケンネ脂\footnote{腎臓の周囲を厚く覆っている脂肪。融解温度が低く、精製して牛脂(ヘット)の原料となる。}1.5
  kg、全卵8個、塩25 g、白こしょう5 g、ナツメグ1 g、透明な氷7〜800
  gまたは氷水7〜8 dL。
\item
  作業手順\ldots{}\ldots{}はじめに、仔牛肉とケンネ脂を別々に、細かく刻む。仔牛肉はさいの目に切り、調味料と合わせておく。牛脂は細かくして、薄皮は筋はきれいに取り除いておく。
\end{itemize}

仔牛肉と牛脂を別々の鉢に入れて、それぞれすり潰す。次にこれらを合わせてから、完全に混ざり合って一体化するまでよくすり潰し、卵を一個ずつ、すり潰す作業を止めずに加えていく。

裏漉しして、平皿に\footnote{大きなバット。}広げ、氷の上に置いて翌日まで休ませる。

翌日になったら、再度ファルスをすり潰す。この時、小さく割った氷を少しずつ加えていき、よく混ぜ合わせる。

ゴディヴォに氷を加え終わったら、必ずテスト\footnote{少量を、沸騰しない程度の温度で火を通し(ポシェ)て様子を見ること。}を行ない、必要に応じて修正する。固すぎるようなら水を少々加え、柔らかすぎるようなら卵白を少し加えること。

\hypertarget{nota-godiveau-a}{%
\subparagraph{【原注】}\label{nota-godiveau-a}}

ゴディヴォで作ったクネルはもっぱら、\protect\hyperlink{vol-au-vent}{ヴォロヴァン}の詰め物\footnote{原文
  garniture ガルニチュールの意味が広いことに注意。}にしたり、牛、羊の塊肉の料理に添える\protect\hyperlink{garniture-a-la-financiere}{ガルニチュール・フィナンシエール}に用いられる。

他のクネルがどれもそうであるように、沸騰しない程度の温度で茹でて\footnote{pocher
  (ポシェ)。}
火を通せばいいが、一般的には手で整形して塩を加えた沸騰しない程度の温度の湯で茹でる。

だが、「ポシャジャセック\footnote{pochage à sec
  直訳すると「乾燥した状態でポシェすること」。つまり水(湯)を用いずに、pocher
  と同様に低めの温度で加熱することを指している。}」と呼ばれる技法、すなわち弱火のオーブンで焼くのがいちばんいい。

以下に示す方法はとても短時間で出来るので特にお勧めだ。

ゴディヴォは充分に氷を加えて水気を含んだ状態にしておく。オーブンの天板に敷いたバターを塗った紙の上に、丸口金を付けた絞り袋から絞り出す。オーブンの天板にもバターを塗っておくこと。絞り出したクネルは触れ合うようにしていい。

これを低温のオーブンに入れて加熱する。

7〜8分すると、クネルの表面に脂が水滴状に浸み出してくる。これが、ちょうどいい具合に火が通った合図だ。オーブンから出して、クネルを別の銀製の盆か大理石の板の上に裏返しに広げる。クネルが\ruby{微温}{ぬる}くなるまで冷めたら、敷いてあった紙を端のほうから引き剥して取り除く。

クネルは完全に冷めるまで放置し、その後に皿に移すか、可能なら柳編みのすのこに載せてやるのがいい。

\hypertarget{godiveau-a-la-creme}{%
\subsubsection{B. 生クリーム入りゴディヴォ}\label{godiveau-a-la-creme}}

\frsub{Godiveau à la crème}

\index{farce@farce!godiveau b@Godiveau B. Godeiveau  à la crème}
\index{ふあるす@ファルス!こていうお@ゴディヴォ!b@B. 生クリーム入りゴディヴォ}
\index{godiveau@godiveau!b@B. --- à la crème}
\index{こていうお@ゴディヴォ!b@B. 生クリーム入り---}

\begin{itemize}
\item
  材料\ldots{}\ldots{}筋をきれいに取り除いた極上の白さの仔牛腿肉1
  kg、水気を含んでいない牛ケンネ脂1 kg、全卵4個、卵黄3個、生クリーム7
  dL、塩25 g、白こしょう5 g、ナツメグ1 g。
\item
  作業手順\ldots{}\ldots{}仔牛肉とケンネ脂は別々に、細かく刻む。これらを鉢に入れて合わせ、調味料、全卵、卵黄をひとつずつ加えながら、力強く全体をすり潰し、完全に一体化させる。
\end{itemize}

裏漉しして、天板に広げる。氷の上にのせて翌日まで休ませる。

翌日になったら、あらかじめ中に氷を入れて冷やしておいた鉢で再度すり潰す。この際に生クリームを少量ずつ加えていく。

クネルを整形する前にテストをして、必要があれば固さなどを修正してやること。
\end{document}