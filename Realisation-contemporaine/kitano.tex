\begin{Main}
\hypertarget{morue}{%
\subsection[モリュ]{\texorpdfstring{モリュ\footnote{鱈の近縁種。真鱈によく似ている。通常、morue
  (モリュ)の名で売られているのは干し鱈、塩鱈で、フレッシュのものは
  cabillaud
  (カビヨー)と呼ばれることが多いが、魚類の名称としてはモリュ。}}{モリュ}}\label{morue}}

\frsecb{Morue}

\index{morue@morue|(} \index{もりゆ@モリュ|(}

(約10人分\ldots{}\ldots{}1.25 kg)

\hypertarget{morue-au-beurre-noir-ou-au-beurre-noisette}{%
\subsubsection{モリュ・焦がしバター}\label{morue-au-beurre-noir-ou-au-beurre-noisette}}

モリュを沸騰しない程度の温度で茹でる\footnote{pocher (ポシェ)。}。取り出して湯をきり、皮を剥ぐ\footnote{魚を少ない煮汁(クールブイヨン)で茹でるための細長い魚用鍋
  poissonière
  (ポワソニエール)を使い、切り身ではなく丸ごと調理する前提になっていることに注意。}。そのまま皿にのせて、少しの時間、乾かす。上から粗みじん切りのパセリを振りかける。レモン果汁を身にかけてやり、10人分あたり200
gの黒バター\footnote{黒バター beurre noir
  という名称は19世紀、とりわけカレームが好んで使ったもので、要するに焦がしバターに他ならない。焦がす程度は料理人の判断や食べ手の嗜好にも左右されていいが、文字通り「黒」にする必要はまったくないので注意。こんにちでは、焦がしバターは
  beurre noisette
  (ブールノワゼット)と表現することの方が覆いだろう。なお、合わせバターに
  \protect\hyperlink{beurre-de-noisette}{beurre de noisette}
  (ブールドノワゼット)というのがあり、これは\protect\hyperlink{beurre-d-aveline}{ヘーゼルナッツバター}
  beurre d'aveline のことであり、混同しないよう注意。}すなわち焦がしバターを覆いかけてやる。

\hypertarget{raie}{%
\subsection{エイ}\label{raie}}

\frsecb{Raie}

\index{raie@raie|(} \index{えい@エイ|(}

(約10人分は掃除をしていない状態で2 kg)

エイにはいろんな種類があるが、レ・ブクレ\footnote{raie bouclée
  和名ループえい。体長70〜100
  cmで褐色の背にたくさんの輪のような模様がある。}がいちばんいい。

イギリス、ベルギー、オランダでは、エイの皮を剥ぎ掃除した状態で売られており、すぐに調理出来る。

\textbf{基本的な下拵え}\ldots{}\ldots{}掃除をしていない状態のエイは、ブラシで洗い、切り分ける。

これを水1 Lあたり12 gの塩とヴィネガー2
dLを加えた鍋に入れて沸騰させないよう加熱する。火が通ったらすぐに取り出して水気を切り、皮を剥く。すぐに使わない場合は、茹で汁を布で漉してそこに戻し入れておくこと。

\hypertarget{raie-au-beurre-noir-ou-au-beurre-noisette}{%
\subsubsection{エイ・焦がしバター}\label{raie-au-beurre-noir-ou-au-beurre-noisette}}

\protect\hyperlink{morue-au-beurre-noir-ou-au-beurre-noisette}{モリュ・焦がしバター}とまったく同様に調理する。バターの分量も同じ。

\begin{center}\rule{0.5\linewidth}{\linethickness}\end{center}

\hypertarget{boeuf}{%
\section{赤身肉(1)\ldots{}\ldots{}牛}\label{boeuf}}

\frsec{Boeuf}

\index{boeuf@boeuf|(} \index{うし@牛|(}

\hypertarget{cervelles}{%
\subsection{脳}\label{cervelles}}

\frsecb{Cervelles}

\hypertarget{cervelle-a-la-bourguignonne}{%
\subsubsection{牛の脳・ブルゴーニュ風}\label{cervelle-a-la-bourguignonne}}

\frsub{Cervelle à la Bourguignonne}

\index{cervelle@cervelle!bourguignonne@--- à la Bourguignonne}
\index{bourguignon@bourguignon(ne)!cervelle@Cervelle à la ---ne}
\index{うしののう@牛の脳!ふるこーにゆふう@---・ブルゴーニュ風}
\index{ふるこーにゆふう@ブルゴーニュ風!うしののう@牛の脳・---}
\index{のううし@脳(牛)!ふるこーにゆふう@---・ブルゴーニュ風}

牛の脳は厚さ1〜2 cm程度にスライスする\footnote{明記されていないが、下拵えとして、冷水にさらしてよく血抜きをしておくこと(dégorger
  デゴルジェ)。}。ガルニチュールにするマッシュルームをバターで色艶よく炒めた小玉ねぎ\footnote{petits
  oignons glacés (プチゾニョングラセ)\textless{} glacer
  (グラセ)。白系の小玉ねぎで若どりのものであれば下茹では不要。日本に多い、黄色系品種であれば下茹でしておいたほうが望ましいことが多い。}を加える。赤ワインソースいわゆる「\protect\hyperlink{sauce-bourguignonne}{ブルゴーニュ風}\footnote{\protect\hyperlink{sauce-au-vin-rouge}{赤ワインソース}も参照。}」を覆いかけ、ごく弱火で7〜
8分煮る\footnote{ごく弱火で煮込むことを mijoter (ミジョテ)と言う。}。

やや深い皿に盛り、周囲を、ハート形に切ったパンを澄ましバターで揚げたクルトンで飾る。

\hypertarget{matelote-de-cervelle}{%
\subsubsection{牛の脳のマトロット}\label{matelote-de-cervelle}}

\frsub{Matelote de Cervelle}

\index{matelote@matelote!cervelle@--- de Cervelle}
\index{cervelle@cervelle!matelote@Matelote de ---}
\index{まとろつと@マトロット!うしののう@牛の脳の---}
\index{うしののう@牛の脳!まとろつと@---のマトロット}
\index{のううし@脳(牛)!まとろつと@---のマトロット}

水にさらしてよく血抜きをした脳を、あらかじめたっぷり香味を効かせて用意しておいた\protect\hyperlink{cour-bouillon-c}{赤ワイン入りのクールブイヨン}で茹でる。その後、厚さ1〜2
cmにスライスし\footnote{escaloper (エスカロペ)。}、あらかじめ茹でておいた小さなマッシュルームとバターで色艶よく炒めた小玉ねぎを加える。

茹で汁\footnote{すなわちクールブイヨン。}を布で漉し、\protect\hyperlink{beurre-manie}{ブールマニエ}を加えてとろみを付ける。これを脳の上にかけ、やや深さのある皿に盛り付ける。ハート形に切って澄ましバターで揚げた小さなクルトンを添える。

\hypertarget{ux539fux6ce8}{%
\subparagraph{【原注】}\label{ux539fux6ce8}}

火入れのプロセス以外、このマトロットは上記のレシピ「\protect\hyperlink{cervelle-a-la-bourguignonne}{牛の脳・ブルゴーニュ風}」とまったく同じものだ。

\begin{center}\rule{0.5\linewidth}{\linethickness}\end{center}

\hypertarget{pigeonneaux-aux-petits-pois}{%
\paragraph{仔鳩とプチポワ}\label{pigeonneaux-aux-petits-pois}}

\frsub{Pigeonneaux aux petits pois}

\index{pigeonneau@pigeonneau(x)!petits pois@---x aux petits pois}
\index{petits pois@petits pois!pigeonneaux@Pigeonneaux aux ---}
\index{こはと@仔鳩!ふちほわ@---とプチポワ}
\index{ふちほわ@プチポワ!こはと@仔鳩と---}

塩漬豚ばら肉60
gはさいの目に切って下茹でしてから、仔鳩1羽あたり6個の小玉ねぎとともにバターでこんがり炒める。これらを鍋から取り出して油をきり、その鍋で仔鳩の表面に焼き色を付ける。鍋の脂は取り除き、少量の\protect\hyperlink{fonds-brun}{フォン}を注いで鍋の底に貼り付いた肉汁を溶かし出す\footnote{déglacer
  デグラセ。}。\protect\hyperlink{sauce-demi-glace}{ソース・ドゥミグラス}適量と鳩1羽あたり
1 \(\frac{1}{2}\)
dLのプチポワを加える。鍋に塩漬豚ばら肉と小玉ねぎを戻し入れ、ブーケガルニを加える。そのままごく弱火で火を通す。

\begin{center}\rule{0.5\linewidth}{\linethickness}\end{center}

\hypertarget{pates-pour-pates-moules}{%
\subsubsection{型に詰めて焼くパテ用の生地}\label{pates-pour-pates-moules}}

これは2種ある。(1)標準的な生地、(2)ラードを用いた生地。

\textbf{標準的な生地の材料}\ldots{}\ldots{}ふるった小麦粉1 kg、バター250
g、塩30 g、全卵2個、水およそ4
dL。水の量は使う粉の性質に合わせて加減すること。良質の小麦粉であればそれだけ水をよく吸い込む。

小麦粉を台の上に山にして中央に窪みを作る。そこに塩、水、卵、バターを加えてデトランプ\footnote{détrempe
  (デトロンプ)小麦粉が水分を吸って軽くまとまった状態のこと。}にする。

これを手の平の手首に近いあたりを使って伸ばすようにして捏ねる\footnote{fraiser
  (フレゼ)。}。これを2回行ない、滑らかで均質にまとまった生地にする。延し棒で延してから、布で包み、使うまで冷所に置いておく。

\textbf{ラードを使う生地の材料}\ldots{}\ldots{}ふるった小麦粉1
kg、微温い温度にもどして柔らかくしたラード250 g、全卵2個、塩30
g、ぬるま湯4 dL。

標準的な生地と同様に、デトランプを作り、手の平で捏ねる。

\hypertarget{ux539fux6ce8-1}{%
\subparagraph{【原注】}\label{ux539fux6ce8-1}}

生地は出来るだけ24時間前に仕込んでおくこと。そうすれば粘りは出にくい。休ませた生地の方が、捏ねたばかりの生地より圧倒的に扱いやすいし、よりきれいな焼き色に仕上げられる。

\begin{center}\rule{0.5\linewidth}{\linethickness}\end{center}

\hypertarget{croute-aux-rognons}{%
\subsubsection[仔牛腎臓のクルート]{\texorpdfstring{仔牛腎臓のクルート\footnote{パンやパイの固く焼けた皮、殻のこと。}}{仔牛腎臓のクルート}}\label{croute-aux-rognons}}

\frsub{Croûte aux Rognons}

\index{croute@croûte!rognons@--- aux Rognons}
\index{rognon@rognon!croute@Croûte aux ---s}
\index{くるーと@クルート!こうししんぞう@仔牛腎臓の---}
\index{こうししんそう@仔牛腎臓!くるーと@---のクルート}

パン・ジョコ\footnote{pain Joko
  パンの形状の名称のひとつで、バタールより太く短かく、ブール(boule
  ボール形)ほど丸くない形状のもの。もっとも、あまり厳密な定義はないため、バタールとほぼ同等と考えていいだろう。}あるいは見た目に面白い形状のパンを2
\(\frac{1}{2}\) cm
厚にスライスする。形状を整え、底がごく薄くなるようにして、内部をくり抜く。内側にバターを塗り、オーブンに入れてパリっとさせる。

仔牛の腎臓をマッシュルームとともにソテーし、マデイラ酒やシャブリなどで風味付けし、上記のクルートに盛り込む。

\hypertarget{ux539fux6ce8-2}{%
\subparagraph{【原注】}\label{ux539fux6ce8-2}}

このクルートは食パンを正方形や長方形にして作ってもいい。その場合は澄ましバターで揚げること。

\begin{center}\rule{0.5\linewidth}{\linethickness}\end{center}

\hypertarget{peches-imperatrice}{%
\subsubsection[ペッシュ・アンペラトリス
]{\texorpdfstring{ペッシュ・アンペラトリス \footnote{pêche(s)
  (ペッシュ)桃。impératrice
  (アンペラトリス)皇后の、の意。フランスが「帝国」を名乗ったのはナポレオンによる第一帝政(1804〜1814)およびナポレオン・ボナパルトの甥ナポレオン3世ルイ・ナポレオンによる第二帝政(1852〜1870)の期間だけであり、皇后
  impératrice
  はナポレオン1世の后ジョゼフィーヌおよびマリ=ルイーズ、ナポレオン3世の妻ユジェニーの3のみ。料理名としては、à
  la reine
  (アラレーヌ)王妃風、と同様に「豪華な」程度の意味しか持たないが、その料理が創案された時代の反映として見ることは出来るだろう。なお、桃
  pêche には同音同綴で「釣り」の意があり、よく煮た綴りで péché
  (ペシェ)宗教的な意味での「罪」の意。「ペシェアンペラトリス」と読みまちがえないよう注意。「皇后の背教的罪」という意味になってしまうので、洒落の通じる相手にしか許されないことを覚えておきたい。よくあるフランス語の日本語化で宗教的におかしな意味を持つものとしては他に、
  bûche de Noël
  (ビュッシュドゥノエール)がある。これを「ブッシュドノエル」と日本語的に発音すると、bouche
  de Noël となってしまうが、 bouche (ブッシュ)は amuse-bouche
  のブッシュ、すなわち「口」の意であるから「降誕祭の口」という意味になってしまう。現代日本人は宗教にやや無頓着な傾向が多いが、異文化における宗教はしばしば戦争の直接的原因となるなど、デリケートな部分が食のいたるところに潜んでいるので注意したい。}}{ペッシュ・アンペラトリス }}\label{peches-imperatrice}}

\frsub{Pêches Impératrice}

縁の高さのないグラタン皿\footnote{timbale(タンバル)円筒形の比較的浅い型および野菜料理用の深皿のこと。}の底に、キルシュかマラスキーノで香り付けした\protect\hyperlink{riz-pour-entremets}{製菓用ライス}を敷き詰める。

その上に、バニラ風味のシロップで煮た半割りの桃を配する。その上を薄く覆うようにライスの層を重ねる。さらにその上に、\protect\hyperlink{}{アプリコットソース}の層を作る。砕いたマカロン\footnote{ここではマカロン・クラクレに代表される固いタイプのマカロンのこと。詳しくは\protect\hyperlink{panade-frangipane}{パナード・フランジパーヌ}訳注参照。}を散りばめ、オーブンの入口の方に入れて
\footnote{すなわち低めの温度で。}10〜12分加熱する。

表面を焦がさないように注意すること。

\hypertarget{riz-pour-entremets}{%
\subsubsection{製菓用ライス}\label{riz-pour-entremets}}

\begin{itemize}
\item
  材料\ldots{}\ldots{}カロライナ米\footnote{いわゆる短粒種。}500
  g、砂糖300 g、塩1つまみ、牛乳2 L、卵黄
  12個、バニラ1本、レモンかオレンジの硬外皮を削ったもの\footnote{zeste
    (ゼスト)。}適量、バター 100 g。
\item
  作業手順\ldots{}\ldots{}米を洗う。下茹でして湯ぎりをし、さらに微温湯で洗う。再度水気をきり、沸かした牛乳で煮込む。牛乳にはあらかじめバニラの香りを煮出しておき、砂糖、塩、バターを加えておくこと。
\end{itemize}

沸騰し始めたら鍋に蓋をし、弱火のオーブンに入れて25〜30分加熱する。この間、液体の対流現象によって米が鍋底に絶対に触れないよう注意すること。そうでないと、鍋底に米が貼り付いてしまう。

オーブンから出したら、卵黄を加え、泡立て器で丁寧に混ぜる。米粒が完全に形を保っているよう、混ぜる際に米粒を崩さないこと。

\hypertarget{sauce-a-l-abricot}{%
\subsubsection{アプリコットソース}\label{sauce-a-l-abricot}}

よく熟したアプリコットを目の細かい網で裏漉しする。またはアプリコットジャムを用いてもいい。これを28°Béのシロップでゆるめる。火にかけて沸騰させ、丁寧にアクを引く。ソースがスプーンの表面をコーティング出来るくらいに煮詰まったら火から外し、アーモンドミルクかマデイラ酒、キルシュまたはマラスキーノなど好みで香り付けする。

\hypertarget{ux539fux6ce8-3}{%
\subparagraph{【原注】}\label{ux539fux6ce8-3}}

果物のクルート用にこのソースを作る場合は、上等なバター少々を加えてもいい。
\end{Main}