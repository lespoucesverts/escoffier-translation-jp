\hypertarget{poularde-polignac}{%
\subsubsection[肥鶏・ポリニャック]{\texorpdfstring{肥鶏・ポリニャック\footnote{ポリニャック家はフランス有数の貴族の名家のひとつ。マリ=アントワネットの取り巻きのひとりとして知られたポリニャック公爵夫人ヨランド・ド・ポラストロン(通称ポリニャック伯爵夫人、1749〜1793)およびその次男で王政復古期に極端な反動政治家として知られるジュール・ド・ポリニャック(1780〜1847)のいずれかに捧げられた料理名と考えられている。}}{肥鶏・ポリニャック}}\label{poularde-polignac}}

\frsub{Poularde Polignac}

肥鶏を\protect\hyperlink{les-poeles}{ポワレ}する。

胸肉を切り出し、腹の部分の骨を取り除く。\protect\hyperlink{farce-c}{ファルス・ムスリーヌ}
400 gにマッシュルーム100 gとトリュフ100 gを幅1〜2 mm の千切り\footnote{julienne
  (ジュリエーヌ。}にして混ぜ込み、鶏の腹の部分に詰める。

胸肉は線維に垂直に、厚さ1〜2 cmにスライスする\footnote{escaloper
  (エスカロペ)。}。これにファルスを塗り、トリュフのスライスを挟むのを繰り返して元の形状にする。こうして元の形にした肥鶏をオーブンの入口付近に入れて\footnote{すなわちやや低温のオーブンで。}ファルスに火を通す。胸肉に火が通り過ぎないよう注意すること。

皿に盛り付ける。マッシュルームのピュレ
\(\frac{1}{4}\)、ごく細い千切りにしたトリュフとマッシュルーム大さじ2杯を加えた\protect\hyperlink{sauce-supreme}{ソース・シュプレーム}
を全体を覆うように塗る。

\hypertarget{poularde-talleyrand}{%
\subsubsection{肥鶏・タレーラン}\label{poularde-talleyrand}}

\frsub{Poularde Talleyrand}

肥鶏をポワレする。次に胸肉を切り出し、大きめのさいの目に切る。

マカロニ\footnote{本書の定義によれば、\protect\hyperlink{macaroni}{マカロニ}とは円柱あるいは円筒状のパスタ全般を意味し、スパゲッティのような細いものからカンネローニのように太いものまで含まれる。ここではどの程度の太さのものを使うか指示はないが、必ずしも円筒形に穴が空いたものである必然性もあまり考えられず、また、当時は円筒形のものもブカティーニのように長いまま流通することが多かったようで、マカロニを用いるレシピのほとんどで、茹でてから短かく切り揃える、という指示が頻繁に見られる。なお、マカロニのイタリア語である
  maccheroni
  が最初にフランス語に訳されたのはボッカッチョ『デカメロン』のフランス語訳であり、その写本の多くはそのまま
  maccheroni
  と訳さずに転記するか、ややフランス風の綴りに直す程度であるなかで、macaron
  と訳した写本がある。これが macaron
  というフランス語の文献上最古の記録となっている(\protect\hyperlink{panade-frangipane}{パナード・フランジパーヌ}訳注参照)。なお、このボッカッチョの時代の
  maccheroni
  は、こんにちのニョッキのようなものだったというのが定説。さて、本書においてタレーランの名を冠するレシピのほとんどにマカロニが用いられているが、カレーム『19世紀フランス料理』の未完部分をプリュムレがまとめた第4巻、第5巻にタレーランの名を冠したレシピが見られる。鶉のフィレ・タレーラン(t.4,
  p216)とトリュフのタンバル・タレーラン (t.5;
  p.522)の2つだが、マカロニはまったく用いられていない。やや時代が下って、デュボワ\&ベルナール『古典料理』では鶏のピュレ・タレーラン
  (t.1, p.~199)および小さなタンバル・タレーラン (\emph{Ibid}., p.~205),
  七面鳥のフィレ・タレーラン (\emph{Ibid}.,
  p.~178)の3つのレシピも同様で、マカロニはおろかパスタの類はまったく使わない。グフェ『料理の本』(1867年)にある、トリュフのタンバル・タレーラン
  (pp.656-657)は、生地を敷きつめたタンバル型に、厚さ0.5cmにスライスしてバターで炒めたのをひたすら詰めて焼くというもので、ルイ・エブラという料理人の創案と記されている(pp.656-657)。『料理の手引き』とほぼ同時代の著書であるファーヴルの『事典』もこれをそのまま踏襲している(pp.1861-1862)。ファーヴルは他に、ガトー・タレーラン、肥鶏胸肉のエスカロップ・タレーランのレシピを記しており、後者は「古典料理」と注釈がある(pp.1844-1855)。つまり、『料理の手引き』以前の主要な料理書においてタレーランとマカロニを結びつける記述は見つかっていないわけだ。いっぽう、カレーム『パリ風パティスリの本』(1815)にはマカロニ(乾燥品)を茹でて他のラグー(フィナンシエールなど)とともにタンバルに詰めるレシピがpp.89-98まで4種あるが、これらのページにおいてタレーランには触れられていない。『ラルース・ガストロノミック』初版(1938)にはタレーラン自身の項は立てられているが、その項にもガルニチュールにも、プレ、プーラルド、タンバルにもタレーランの名を冠した料理は見あたらない。そのいっぽうでソース・タレーランというレシピが掲載されており、「鶏のヴルテ2
  dLと白いフォン2
  dLを半量に煮詰める。最後にひと煮立ちさせたら生クリーム大さじ4杯とマデイラ酒½
  dLを加える。火から外してバター50 gを加えてよく混ぜ、布で漉す。

  \par

  野菜のミルポワ大さじ1と細かいさいの目に切ったトリュフ大さじ1、赤く漬けた舌肉大さじ1を加える」(p.965)というもの。タレーランは正式にはタレーラン・ペリゴールが家名であるから、結び付けるとしたらむしろトリュフであり、その点では他のタレーランの名を冠したレシピ群の多くは一致している。また、フォワグラも関連性を認められると考えていいだろう。この『料理の手引き』初版および第二版には、鶏のタンバル・タレーランが図版付きで収録されており、これは、ドーム状の型にマカロニの端をファルスを糊でつなぐようにして型の周囲に敷き詰め、ファルスで塗り固めてから、\protect\hyperlink{garniture-talleyrand}{ガルニチュール・タレーラン}を詰める。おそらくはこの第三版以降消えてしまった「タンバル」仕立てとともにガルニチュール・タレーランが考案され、他の料理にも影響しているのだと考えられる。なお、初版から現行版までオードブルの章にある\protect\hyperlink{timbales-talleyrand-petites}{タンバル・タレーラン(小)}は一貫してマカロニを用いておらず、トリュフのピュレが主素材となっていることから、これがタレーランの名を料理名に冠する根拠あるいは由縁としてもっとも重要なものと考えていいだろう。}を茹でて短かく切り、おろしたパルメザンチーズと\protect\hyperlink{sauce-a-la-creme}{ソース・クレーム}であえたものを鶏胸肉と同量ずつ混ぜあわせる。

腹の骨を取り除き、ガラの内部に上で作ったガルニチュールを詰める。肥鶏を元の形にし、最後に\protect\hyperlink{farce-c}{ファルス・ムスリーヌ}で全体を覆う。

表面にトリュフのスライスを王冠状に貼り付けて飾り、バターを塗った紙で包み、中温のオーブンに入れ、(1)
ファルスに適温で火を通し、(2)
中に詰めたガルニチュールが充分に温まるようにする。

皿に盛り付ける。トリュフエッセンスとトリュフの千切りを加えた\protect\hyperlink{sauce-demi-glace}{ソース・ドゥミグラス}を皿の底に大さじ数杯流し入れる。

ソース入れで同じソースを添えて供する。

\hypertarget{timbale-de-pigeonneaux-la-fayette}{%
\subsubsection[仔鳩のタンバル・ラファイエット]{\texorpdfstring{仔鳩のタンバル・ラファイエット\footnote{ラファイエット侯爵(1757〜1834)。アメリカ独立革命とフランス革命の両方で活躍し、フランス人権宣言を起草したことで知られる。「両大陸の英雄」とも称される。原注にあるように、エスコフィエは、フランスとアメリカ合衆国との親善の意味を込めてこの名を冠したのだろう。}}{仔鳩のタンバル・ラファイエット}}\label{timbale-de-pigeonneaux-la-fayette}}

\frsub{Timbale de pigeonneaux La Fayette}

\begin{enumerate}
\def\labelenumi{\arabic{enumi}.}
\item
  高さが低めの\footnote{原文 une moule plus large que haut
    直訳すると「高さより幅の広い型」。}型でタンバルの殻を焼く\footnote{本書にはpâte
    brisée「ブリゼ生地」は出ておらず、本書により忠実に作るのであれば\protect\hyperlink{pate-a-foncer}{フォンセ生地}を用いることになるだろう。}。
\item
  \protect\hyperlink{mirepoix-fine}{ボルドー風ミルポワ}をバターで蒸し煮し\footnote{étuver
    (エチュヴェ)。}、その鍋でガルニチュール用のエクルヴィス60尾を、白ワイン
  3 dLとコニャック 1
  dL、塩、白こしょう、ピンクペッパーを加えて火を通す。火が通ったらすぐに尾の殻を剥き、煮汁は目の細かいシノワで漉してトリュフのスライス
  100
  gを加え、そこにエクルヴィスの尾の身を浸して温めておく。エクルヴィスの殻などはバター50
  gを加えて細かくすり潰す。これを生クリーム入りの\protect\hyperlink{sauce-bechamel}{ベシャメルソース}
  \(\frac{1}{2}\)
  Lに混ぜ込み、軽くひと煮立ちさせてから布で漉して、保温しておく。
\item
  上の作業と平行して、小さな仔鳩10羽をごく薄い豚背脂のシートで包んで焼く。火が通ったら、背脂のシートを外し、胸肉の部分の皮を剥いて、胸肉を切り出す。仔鳩を焼いた鍋に白ワインを注いでデグラセし、溶かした\protect\hyperlink{glace-de-viande}{グラスドヴィアンド}大さじ数杯とトリュフのスライス100
  gを加え、そこに胸肉を浸して保温しておく。
\item
  中位の太さのマカロニ400
  gを塩湯でやや固めに茹で、よく湯ぎりをしてから、バター100
  g、おろしたてのパルメザンチーズ150
  g、挽きたてのこしょう1つまみ、上で作ったエクルヴィスバター入りのベシャメルソースの
  \(\frac{1}{3}\) を加えてあえる。また、ベシャメルソースの別の
  \(\frac{1}{3}\) はエクルヴィスの尾をあえる。残りは仔鳩の胸肉とあえる。
\end{enumerate}

\textbf{盛り付け}\ldots{}\ldots{}タンバルの殻に、まずマカロニの
\(\frac{2}{3}\)
量を詰め、その上にエクルヴィスとトリュフの半量を詰める。再度マカロニを詰め、その上に仔鳩の胸肉を環状に並べる。その中央にエクルヴィスの尾の身の残りを詰める。最後に残ったマカロニで上面を覆い、大きなトリュフのスライスを何枚か飾る。

\hypertarget{ux539fux6ce8}{%
\subparagraph{【原注】}\label{ux539fux6ce8}}

これは、筆者がはじめてニューヨークを訪れた際に友人たちが催してくれた昼食会の際に、筆者自らが作った料理。

\hypertarget{homard-clarence}{%
\subsubsection[オマール・クラレンス]{\texorpdfstring{オマール・クラレンス\footnote{イギリス王族の公爵位として知られる。料理としては、本書以前のレシピの例は見あたらず、本書、\protect\hyperlink{filets-de-sole-clarence}{舌びらめのフィレ・クラレンス}ス{]}(\#filets-de-sole-clarence)の原注において、必ずこうすべきという共通理解があるものではない、と述べられていることからも、明確な定義は得難い。この名称を冠するのはもっぱら魚および甲殻類の料理で、そのほとんどが\protect\hyperlink{sauce-mornay}{ソース・モルネー}または\protect\hyperlink{sauce-new-burg}{ソース・ニューバーグ}にカレー風味を足したもの。本書では舌びらめのフィレ・クラレンスがレシピ本文ではソース・モルネーだけで作る指示だが、原注において、カレー粉を加えた\protect\hyperlink{sauce-americaine}{ソース・アメリケーヌ}に代えるケースを示唆されている。}}{オマール・クラレンス}}\label{homard-clarence}}

\frsub{Homard Clarence}

オマールは\protect\hyperlink{cout-bouillon-e}{クールブイヨン}で茹で、火が通ったらすぐに取り出して湯をきる。

微温くなるまで冷めたら、縦2つに切る。尾の身を取り出し、やや斜めに厚さ1〜
2
cmの輪切りにして、\protect\hyperlink{fumet-de-poisson}{魚のフュメ}またはマッシュルームの茹で汁少々を加えて保温しておく。

胴の身とクリーム状の部分を取り出し、これを鉢に入れて、生クリーム大さじ
2杯を加えてすり潰し、目の細かい網で漉す。これを、カレー風味の\protect\hyperlink{sauce-bechamel}{ベシャメルソース}
2 \(\frac{1}{2}\) dLに加える。

2つに割った胴のそれぞれに\protect\hyperlink{riz-a-l-indienne}{インド風ライス}を
\(\frac{2}{3}\)
程詰め、その上に輪切りにて保温しておいたオマールの尾の身を、トリュフのスライスを挟んで交互になるように盛り付ける。

用意しておいたカレー風味のベシャメルソースの一部をオマールに軽く塗る。温めておいた長い皿に盛り付ける。

\ldots{}\ldots{}ソースの残りを別添で供する。

\hypertarget{sole-grillee-aux-huuxeetres-a-l-americaine}{%
\subsubsection{舌びらめのグリル焼き、牡蠣添え・アメリカ風}\label{sole-grillee-aux-huuxeetres-a-l-americaine}}

\frsub{Sole grillée aux huîtres à l'Américaine}

舌びらめはグリル焼きにしてもいいし、バターをレモン果汁を加えてほとんど水気のない状態でやや低めの温度で火入れをしてもいい。同様の方法はフィレにおろした舌びらめにも合う。が、いちばん多いのはグリル焼き。火入れの方法がいずれであっても、舌びらめは温めた皿に盛り、提供直前に、少量のダービーソース\footnote{\protect\hyperlink{devilled-sauce}{デビルソース}
  訳注参照。}を沸かして軽く火を入れた牡蠣6個を周囲に飾る。

すぐに、揚げたてのパン粉にパセリのみじん切り1つまみを加え、舌びらめに覆いかける。

\hypertarget{laitues-a-la-moelle}{%
\subsubsection{レチュ・骨髄添え}\label{laitues-a-la-moelle}}

\frsub{Laitues à la Moelle}

レチュを\protect\hyperlink{braisage-des-legumes}{ブレゼ}し、皿に盛り付ける。その上に、やや低温で加熱した牛骨髄の大きなスライスを環状に飾る。軽くバターを加えた\protect\hyperlink{jus-de-veau-lie}{とろみを付けたジュ}をかけて供する。

\hypertarget{peches-imperatrice}{%
\subsubsection[ペッシュ・アンペラトリス
]{\texorpdfstring{ペッシュ・アンペラトリス \footnote{pêche(s)
  (ペッシュ)桃。impératrice
  (アンペラトリス)皇后の、の意。フランスが「帝国」を名乗ったのはナポレオンによる第一帝政(1804〜1814)およびナポレオン・ボナパルトの甥ナポレオン3世ルイ・ナポレオンによる第二帝政(1852〜1870)の期間だけであり、皇后
  impératrice
  はナポレオン1世の后ジョゼフィーヌおよびマリ=ルイーズ、ナポレオン3世の妻ユジェニーの3のみ。料理名としては、à
  la reine
  (アラレーヌ)王妃風、と同様に「豪華な」程度の意味しか持たないが、その料理が創案された時代の反映として見ることは出来るだろう。なお、桃
  pêche には同音同綴で「釣り」の意があり、よく煮た綴りで péché
  (ペシェ)宗教的な意味での「罪」の意。「ペシェアンペラトリス」と読みまちがえないよう注意。「皇后の背教的罪」という意味になってしまうので、洒落の通じる相手にしか許されないことを覚えておきたい。よくあるフランス語の日本語化で宗教的におかしな意味を持つものとしては他に、
  bûche de Noël
  (ビュッシュドゥノエール)がある。これを「ブッシュドノエル」と日本語的に発音すると、bouche
  de Noël となってしまうが、 bouche (ブッシュ)は amuse-bouche
  のブッシュ、すなわち「口」の意であるから「降誕祭の口」という意味になってしまう。現代日本人は宗教にやや無頓着な傾向が多いが、異文化における宗教はしばしば戦争の直接的原因となるなど、デリケートな部分が食のいたるところに潜んでいるので注意したい。}}{ペッシュ・アンペラトリス }}\label{peches-imperatrice}}

\frsub{Pêches Impératrice}

縁の高さのないグラタン皿\footnote{timbale(タンバル)円筒形の比較的浅い型および野菜料理用の深皿のこと。}の底に、キルシュかマラスキーノで香り付けした\protect\hyperlink{riz-pour-entremets}{製菓用ライス}を敷き詰める。

その上に、バニラ風味のシロップで煮た半割りの桃を配する。その上を薄く覆うようにライスの層を重ねる。さらにその上に、\protect\hyperlink{sauce-a-l-abricot}{アプリコットソース}の層を作る。砕いたマカロン\footnote{ここではマカロン・クラクレに代表される固いタイプのマカロンのこと。詳しくは\protect\hyperlink{panade-frangipane}{パナード・フランジパーヌ}訳注参照。}を散りばめ、オーブンの入口の方に入れて\footnote{すなわち低めの温度で。}10〜12分加熱する。

表面を焦がさないように注意すること。

\hypertarget{riz-pour-entremets}{%
\subsubsection{製菓用ライス}\label{riz-pour-entremets}}

\begin{itemize}
\item
  材料\ldots{}\ldots{}カロライナ米\footnote{いわゆる短粒種。}500
  g、砂糖300 g、塩1つまみ、牛乳2 L、卵黄
  12個、バニラ1本、レモンかオレンジの硬外皮を削ったもの\footnote{zeste
    (ゼスト)。}適量、バター 100 g。
\item
  作業手順\ldots{}\ldots{}米を洗う。下茹でして湯ぎりをし、さらに微温湯で洗う。再度水気をきり、沸かした牛乳で煮込む。牛乳にはあらかじめバニラの香りを煮出しておき、砂糖、塩、バターを加えておくこと。
\end{itemize}

沸騰し始めたら鍋に蓋をし、弱火のオーブンに入れて25〜30分加熱する。この間、液体の対流現象によって米が鍋底に絶対に触れないよう注意すること。そうでないと、鍋底に米が貼り付いてしまう。

オーブンから出したら、卵黄を加え、泡立て器で丁寧に混ぜる。米粒が完全に形を保っているよう、混ぜる際に米粒を崩さないこと。

\hypertarget{sauce-a-l-abricot}{%
\subsubsection{アプリコットソース}\label{sauce-a-l-abricot}}

よく熟したアプリコットを目の細かい網で裏漉しする。またはアプリコットジャムを用いてもいい。これを28°Béのシロップでゆるめる。火にかけて沸騰させ、丁寧にアクを引く。ソースがスプーンの表面をコーティング出来るくらいに煮詰まったら火から外し、アーモンドミルクかマデイラ酒、キルシュまたはマラスキーノなど好みで香り付けする。

\hypertarget{ux539fux6ce8-1}{%
\subparagraph{【原注】}\label{ux539fux6ce8-1}}

果物のクルート用にこのソースを作る場合は、上等なバター少々を加えてもいい。

\hypertarget{peches-eugenie}{%
\subsubsection[ペッシュ・ユジェニー]{\texorpdfstring{ペッシュ・ユジェニー\footnote{このレシピは第四版から。索引には
  Pêches Impératrice Eugénie
  とあるため、ナポレオン3世妃ユジェニーを指すと解釈可能。ただし、索引のミスの可能性もある。その場合には具体的に誰を指しているのかは不明ということになる。}}{ペッシュ・ユジェニー}}\label{peches-eugenie}}

いい具合に熟した桃を選ぶこと。丁寧に種を取り除き、皮を剥く。これをやや深い皿に、フレーズ・デ・ボワと交互になるように盛り込む。キルシュとマラスキーノをスプーン数杯ずつ上からかけて、蓋をして氷の上に1時間置いて冷やす。

提供直前に、よく冷えたシャンパーニュ風味のサバイヨンソースを桃に覆いかける。

\hypertarget{peches-imperatrice-froides}{%
\subsubsection{ペッシュ・アンペラトリス}\label{peches-imperatrice-froides}}

桃は半割りにし、バニラ風味のシロップで煮て、そのまま冷ます。桃をシロップから取り出して、よく水気を取り除く。半割りにした桃の種を抜いた穴にたっぷりとバニラアイスを詰め込んでいき、半割りにする前の桃の形状と大きさになるようにする。桃の側面にはよく煮詰めたアプリコットソースを塗り、シロップで炒りつけたアーモンドを細かくして、その上を桃を転がしてまぶす。

フランボワーズのジャムを塗って乾かしたタルトの台にジェノワーズを敷き、キルシュとマラスキーノを浸み込ませた上に桃を盛り付ける。

糸状にした飴を上から覆いかぶせる。
