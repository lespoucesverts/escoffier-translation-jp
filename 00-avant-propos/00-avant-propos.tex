\documentclass[twoside,14Q,a4paper,openany]{ltjsbook}
\documentclass[twoside,14Q,a4paper,openany]{ltjsbook}

\usepackage{amsmath}
  \let\equation\gather
  \let\endequation\endgather
\usepackage{amssymb}
\usepackage[no-math]{fontspec}
\usepackage{geometry}
\usepackage{luaotfload}
\usepackage{graphicx}

\usepackage{setspace}


%%%%%%%%% hyperref %%%%%%%%%%%%%
\usepackage{refcount}
\usepackage[unicode=true,hyperfootnotes=false,pageanchor]{hyperref}
\hypersetup{hyperindex=false,%
             breaklinks=true,%
             bookmarks=true,%
             pdfauthor={五島 学},%
             pdftitle={エスコフィエ『料理の手引き』全注解},%
             colorlinks=false%true,%
             %colorlinks=true,%
             citecolor=blue,%
             urlcolor=cyan,%
             linkcolor=magenta,%
             bookmarksdepth=subsubsection,%
             pdfborder={0 0 0},%
             hyperfootnotes=false,%
             plainpages=false,
             }
\urlstyle{same}




%% 欧文フォント設定
% Libertine/Biolinum
\setmainfont[Ligatures=Historic,Scale=1.0]{Linux Libertine O}
\setsansfont[Ligatures=TeX, Scale=MatchLowercase]{Linux Biolinum O} 
%\usepackage{libertine}
\usepackage{unicode-math}
\setmathfont[Scale=1.2]{libertinusmath-regular.otf}
%\unimathsetup{math-style=ISO,bold-style=ISO}
%\setmathfont{xits-math.otf}
%\setmathfont{xits-math.otf}[range={cal,bfcal},StylisticSet=1]




%% Garamond
%\usepackage{ebgaramond-maths}
%\setmainfont[Ligatures=Historic,Scale=1.1]{EB Garamond}%fontspecによるフォント設定

%\usepackage{qpalatin}%palatino

%\setmainfont[Ligatures=Historic,Scale=MatchLowercase]{Tex Gyre Schola}
%\setmainfont[Ligatures=Historic,Scale=MatchLowercase]{Tex Gyre Pagella}
%\setsansfont[Scale=MatchLowercase]{TeX Gyre Heros}  % \sffamily のフォント
%\setsansfont[Scale=MatchLowercase]{TeX Gyre Adventor}  % \sffamily のフォント

%\setmonofont[Scale=MatchLowercase]{Inconsolata}       % \ttfamily のフォント

%\usepackage[cmintegrals,cmbraces]{newtxmath}%数式フォント

\usepackage{luatexja}
\usepackage{luatexja-fontspec}
%\ltjdefcharrange{8}{"2000-"2013, "2015-"2025, "2027-"203A, "203C-"206F}
%\ltjsetparameter{jacharrange={-2, +8}}
\usepackage{luatexja-ruby}

%%%%和文フォント設定
%\usepackage[sourcehan,bold,jis2004,expert,deluxe]{luatexja-preset}%Adobe源ノ明朝、ゴチ
%\usepackage[hiragino-pron,bold,jis2004,expert,deluxe]{luatexja-preset}
%\usepackage[yu-osx,expert,jis2004,bold]{luatexja-preset}
%\usepackage[moga-mogo-ex,bold]{luatexja-preset}

\newopentypefeature{PKana}{On}{pkna} % "PKana" and "On" can be arbitrary string
%%%%明朝にIPAexMincho、ゴチ(太字)にMoboGoBを使う設定。和文カナプロプーショナル使用可能だが読みづらくなる。
\setmainjfont[%
     %YokoFeatures={JFM=prop,PKana=On},%
     %CharacterWidth=AlternateProportional,%
%    CharacterWidth=Proportional,%Mogo, IPAExMinchoには不可
     %Kerning=On,%
     BoldFont={ MoboGoB },%
     ItalicFont={ MoboGoB },%
     BoldItalicFont={ MoboGoExB }%
     % ]{ MogaHMin }
     ]{ IPAExMincho }
     % ]{ IPAmjMincho }
\setsansjfont[%
     %YokoFeatures={JFM=prop,PKana=On},%
     %CharacterWidth=AlternateProportional,%
     % CharacterWidth=Proportional,%Mobo, IPAExGOthicには不可
     %Kerning=On,
     BoldFont={ MoboGoB },%
     ItalicFont={ MoboGoB },%
     BoldItalicFont={ MoboGoExB }%
     % ]{ MoboGo}
     ]{ IPAExGothic }
% %  %%%% 和文仮名プロプーショナルここまで
% %\ltjsetparameter{jacharrange={-2}}%キリル文字%引数に-3を付けるとギリシア文字も可能になるが、%三点リーダーも欧文化されてしまうので注意%


\renewcommand{\bfdefault}{bx}%和文ボールドを有効にする
\renewcommand{\headfont}{\gtfamily\sffamily\bfseries}%和文ボールドを有効にする
%\addfontfeature{Fractions=On}


\defaultfontfeatures[\rmfamily]{Scale=1.2}%効いていない様子
\defaultjfontfeatures{Scale=0.92487}%和文フォントのサイズ調整。デフォルトは 0.962212 倍%ltjsclassesでは不要?
%\defaultjfontfeatures{Scale=0.962212}
%\usepackage{libertineotf}%linux libertine font %ギリシア語含む
%\usepackage[T1]{fontenc}
%\usepackage[full]{textcomp}
%\usepackage[osfI,scaled=1.0]{garamondx}
%\usepackage{tgheros,tgcursor}
%\usepackage[garamondx]{newtxmath}

\usepackage{layout}

%% レイアウト調整(A4Paper,14Q,twoside,ltjsbook.cls) 
%%
\setlength{\hoffset}{0\zw}
\setlength{\oddsidemargin}{0\zw}%タブレット前提の中央配置
\setlength{\evensidemargin}{\oddsidemargin}
% \setlength{\oddsidemargin}{1\zw}%製本時に右ページのみをオフセット
%\setlength{\evensidemargin}{0pt}%
\setlength{\fullwidth}{45\zw}
\setlength{\textwidth}{45\zw}%%ltjsclassesのみ有効
%\setlength{\fullwidth}{159mm}
%\setlength{\textwidth}{159mm}
\setlength{\marginparsep}{0pt}
\setlength{\marginparwidth}{0pt}
\setlength{\footskip}{0pt}
\setlength{\voffset}{-17mm}
\setlength{\textheight}{265mm}
\setlength{\parskip}{0pt}
\setlength{\parindent}{0pt}

%%%% b5j%%%%%%
% \setlength{\hoffset}{-10mm}
% \setlength{\oddsidemargin}{0mm}
% \setlength{\evensidemargin}{0mm}
% %\setlength{\textwidth}{\fullwidth}%%ltjsclassesのみ有効
% \setlength{\fullwidth}{145mm}
% \setlength{\textwidth}{145mm}
% \setlength{\marginparsep}{0pt}
% \setlength{\marginparwidth}{0pt}
% \setlength{\footskip}{0pt}
% \setlength{\voffset}{-10mm}
% \setlength{\textheight}{225mm}
% \setlength{\parskip}{0pt}
%%%ベースライン調整
%\ltjsetparameter{yjabaselineshift=0pt,yalbaselineshift=-.75pt}


%\usepackage{fancyhdr}






%文字サイズ、見出しなどの再定義
\makeatletter
%\renewcommand{\large}{\jsc@setfontsize\large\@xipt{14}}
%\renewcommand{\Large}{\jsc@setfontsize\Large{13}{15}}

\newcommand{\medlarge}{\fontsize{11}{13}\selectfont}
\newcommand{\medsmall}{\fontsize{9.23}{9.5}\selectfont}
\newcommand{\twelveq}{\jsc@setfontsize\twelveq{9.230769}{9.75}\selectfont}
\newcommand{\fourteenq}{\jsc@setfontsize\fourteenq{10.7692}{13}\selectfont}
\newcommand{\fifteenq}{\jsc@setfontsize\fifteenq{11.53846}{14}\selectfont}

\renewcommand{\chapter}{%
  \if@openleft\cleardoublepage\else
  \if@openright\cleardoublepage\else\clearpage\fi\fi
  \plainifnotempty % 元: \thispagestyle{plain}
  \global\@topnum\z@
  \if@english \@afterindentfalse \else \@afterindenttrue \fi
  \secdef
    {\@omit@numberfalse\@chapter}%
    {\@omit@numbertrue\@schapter}}
\def\@chapter[#1]#2{%
  \ifnum \c@secnumdepth >\m@ne
    \if@mainmatter
      \refstepcounter{chapter}%
      \typeout{\@chapapp\thechapter\@chappos}%
      \addcontentsline{toc}{chapter}%
        {\protect\numberline
        % {\if@english\thechapter\else\@chapapp\thechapter\@chappos\fi}%
        {\@chapapp\thechapter\@chappos}%
        #1}%
    \else\addcontentsline{toc}{chapter}{#1}\fi
  \else
    \addcontentsline{toc}{chapter}{#1}%
  \fi
  \chaptermark{#1}%
  \addtocontents{lof}{\protect\addvspace{10\jsc@mpt}}%
  \addtocontents{lot}{\protect\addvspace{10\jsc@mpt}}%
  \if@twocolumn
    \@topnewpage[\@makechapterhead{#2}]%
  \else
    \@makechapterhead{#2}%
    \@afterheading
  \fi}
\def\@makechapterhead#1{%
  \vspace*{0\Cvs}% 欧文は50pt
  {\parindent \z@ \centering \normalfont
    \ifnum \c@secnumdepth >\m@ne
      \if@mainmatter
        \huge\headfont \@chapapp\thechapter\@chappos%変更
        \par\nobreak
        \vskip \Cvs % 欧文は20pt
      \fi
    \fi
    \interlinepenalty\@M
    \huge \headfont #1\par\nobreak
    \vskip 1\Cvs}} % 欧文は40pt%変更

\renewcommand{\section}{%
    \if@slide\clearpage\fi
    \@startsection{section}{1}{\z@}%
    {\Cvs \@plus.5\Cdp \@minus.2\Cdp}% 前アキ
    % {.5\Cvs \@plus.3\Cdp}% 後アキ
    {.5\Cvs}
    {\normalfont\Large\headfont\bfseries\centering}}%変更

\renewcommand{\subsection}{\@startsection{subsection}{2}{\z@}%
    {\Cvs \@plus.5\Cdp \@minus.2\Cdp}% 前アキ
    % {.5\Cvs \@plus.3\Cdp}% 後アキ
    {.5\Cvs}
  %  {\normalfont\large\headfont\bfseries\centering}} %変更
    {\normalfont\large\headfont\centering}} %変更

\renewcommand{\subsubsection}{\@startsection{subsubsection}{3}{\z@}%
  % {0\Cvs \@plus.8\Cdp \@minus.6\Cdp}%変更
    {1sp \@plus.5\Cdp \@minus.5\Cdp}%変更
    {\if@slide .5\Cvs \@plus.3\Cdp \else \z@ \fi}%
    % {\normalfont\medlarge\headfont\leftskip -1\zw}}
    {\normalfont\medlarge\headfont\leftskip -1\zw}}

\renewcommand{\paragraph}{\@startsection{paragraph}{4}{\z@}%
    {0.5\Cvs \@plus.5\Cdp \@minus.2\Cdp}%
    % {\if@slide .5\Cvs \@plus.3\Cdp \else -1\zw\fi}% 改行せず 1\zw のアキ
    {1sp}%後アキ
    {\normalfont\normalsize\headfont}}
\renewcommand{\subparagraph}{\@startsection{subparagraph}{5}{\z@}%
    {\z@}{\if@slide .5\Cvs \@plus.3\Cdp \else -.5\zw\fi}%
    {\normalfont\normalsize\headfont\hskip-.5\zw\noindent}}  

\newcommand{\frchap}[1]{\vspace*{-2ex}%
 \begin{center}\normalfont\headfont\LARGE\setstretch{0.8}
 \scshape#1\normalfont\normalsize
\end{center}\vspace{0.5\zw}\setstretch{1.0}}

\newcommand{\frsec}[1]{\vspace*{-2ex}%
 \begin{center}\normalfont\headfont\large\setstretch{0.8}
 \scshape#1\normalfont\normalsize
\end{center}\vspace{0.5\zw}\setstretch{1.0}}
  
\newcommand{\frsecb}[1]{\vspace*{-2ex}%
\begin{center}\normalfont\headfont\medlarge\setstretch{0.8}%
  \hspace{1em}\scshape#1\normalfont\normalsize%
\end{center}\vspace{0.5\zw}\setstretch{1.0}}

%\newcounter{frsub}[subsubsection]
%\newcommand{\frsub}{\@startsection{frsub}{6}{\z@}%
%  {1sp}{1sp}%
%  {\normalfont\normalsize\bfseries\baselineskip-.8ex\leftskip-1\zw}}
%\let\frsubmark\@gobble
%\newcommand*{\l@frsub}{%
%          \@tempdima\jsc@tocl@width \advance\@tempdima 16.183\zw
%          \@dottedtocline{7}{\@tempdima}{6.5\zw}}
%\renewcommand{\thefrsub}{6}
%\let\frsub\paragraph

\newenvironment{frsubenv}{\begin{spacing}{0.2}\setlength{\leftskip}{-1\zw}\bfseries}{\end{spacing}\normalfont\normalsize\setlength{\leftskip}{0pt}}
\newcommand{\frsub}[1]{\begin{frsubenv}#1\end{frsubenv}\par\vspace{1.1\zw}}

%\newcommand{\frsub}[1]{\vskip -.8ex\hskip -1\zw\textbf{#1}\leftskip0pt}
%\newcommand{\frsub}{\@startsection{frsub}{6}{\z@}%
%   {-1\zw}% 改行せず 1\zw のアキ
%   {-1\zw}%後アキ     
%   {\normalfont\normalsize\bfseries\leftskip -1\zw\baselineskip -.5ex}}%normalsizeから変更
%\newcommand*{\l@frsub}{%
%          \@tempdima\jsc@tocl@width \advance\@tempdima 16.183\zw
%          \@dottedtocline{5}{\@tempdima}{6.5\zw}}


%%%%%%%%%レシピと本文%%%%%%%%%%%%
\usepackage{multicol}
\setlength{\columnsep}{3\zw}

%%% 本文
\newenvironment{Main}{}{}
%%% レシピ
% \setlength{\columnwidth}{24\zw}
%本文ヨリ小%\small
%\newenvironment{recette}{\setlength{\parindent}{0pt}\begin{small}\begin{spaceing}{0.8}\begin{multicols}{2}}{\end{multicols}\end{spacing}\end{small}}
%本文やや小%\medsmall
%\newenvironment{recette}{\setlength{\parindent}{0pt}\begin{medsmall}\begin{spacing}{0.75}\begin{multicols}{2}}{\end{multicols}\end{spacing}\end{medsmall}}
%本文ナミ(無指定)
\newenvironment{recette}{\setlength{\parindent}{0pt}\begin{spacing}{0.8}\begin{multicols}{2}\setlength\topskip{.8\baselineskip}}{\end{multicols}\end{spacing}}


\makeatother


%%% 脚注番号のページ毎のリセットと脚注位置の調整
%\renewcommand{\footnotesize}{\small}

\makeatletter

\usepackage[bottom,perpage,stable]{footmisc}%
%\setlength{\skip\footins}{4mm plus 4mm}
%\usepackage{footnpag}
\renewcommand\@makefntext[1]{%
  \advance\leftskip 0\zw
  \parindent 1\zw
  \noindent
  \llap{\@thefnmark\hskip0.5\zw}#1}


\let\footnotes@ve=\footnote
\def\footnote{\inhibitglue\footnotes@ve}
\let\footnotemarks@ve=\footnotemark
%\def\footnotemark{\inhibitglue\footnotemarks@ve}
\renewcommand{\footnotemark}{\footnotemarks@ve}%変更
% %\def\thefootnote{\ifnum\c@footnote>\z@\leavevmode\lower.5ex\hbox{(}\@arabic\c@footnote\hbox{)}\fi}
\renewcommand{\thefootnote}{\ifnum\c@footnote>\z@\leavevmode\hbox{}\@arabic\c@footnote\hbox{)}\fi}
%\makeatletter
% \@addtoreset{footnote}{page}
% \makeatother
%\usepackage{dblfnote}
%\usepackage[bottom,perpage]{footmisc}

\makeatother

%subsubsectionに連番をつける
%\usepackage{remreset}

\renewcommand{\thechapter}{}
\renewcommand{\thesection}{\hskip-1\zw}
\renewcommand{\thesubsection}{}
\renewcommand{\thesubsubsection}{}
\renewcommand{\theparagraph}{}


%\makeatletter
%\@removefromreset{subsubsection}{subsection}
%\def\thesubsubsection{\arabic{subsubsection}.}
%\newcounter{rnumber}
%\renewcommand{\thernumber}{\refstepcounter{rnumber} }

\renewcommand{\prepartname}{\if@english Part~\else {}\fi}
\renewcommand{\postpartname}{\if@english\else {}\fi}
\renewcommand{\prechaptername}{\if@english Chapter~\else {}\fi}
\renewcommand{\postchaptername}{\if@english\else {}\fi}
\renewcommand{\presectionname}{}%  第
\renewcommand{\postsectionname}{}% 節

%リスト環境
\def\tightlist{\itemsep1pt\parskip0pt\parsep0pt}%pandoc対策

\makeatletter
  \parsep   = 0pt
  \labelsep = .5\zw
  \def\@listi{%
     \leftmargin = 0pt \rightmargin = 0pt
     \labelwidth\leftmargin \advance\labelwidth-\labelsep
     \topsep     = 0pt%\baselineskip
     %\topsep -0.1\baselineskip \@plus 0\baselineskip \@minus 0.1 \baselineskip
     \partopsep  = 0pt \itemsep       = 0pt
     \itemindent = -.5\zw \listparindent = 0\zw}
  \let\@listI\@listi
  \@listi
  \def\@listii{%
     \leftmargin = 1.8\zw \rightmargin = 0pt
     \labelwidth\leftmargin \advance\labelwidth-\labelsep
     \topsep     = 0pt \partopsep     = 0pt \itemsep   = 0pt
     \itemindent = 0pt \listparindent = 1\zw}
  \let\@listiii\@listii
  \let\@listiv\@listii
  \let\@listv\@listii
  \let\@listvi\@listii
\makeatother








%% %%%%%%行取りマクロ
% \makeatletter
% \ifx\Cht\undefined
%  \newdimen\Cht\newdimen\Cdp
%  \setbox0\hbox{\char\jis"2121}\Cht=\ht0\Cdp=\dp0\fi
% \catcode`@=11
% \long\def\linespace#1#2{\par\noindent
%   \dimen@=\baselineskip
%   \multiply\dimen@ #1\advance\dimen@-\baselineskip
%   \advance\dimen@-\Cht\advance\dimen@\Cdp
%   \setbox0\vbox{\noindent #2}%
%   \advance\dimen@\ht0\advance\dimen@-\dp0%
%   \vtop to\z@{\hbox{\vrule width\z@ height\Cht depth\z@
%    \raise-.5\dimen@\hbox{\box0}}\vss}%
%   \dimen@=\baselineskip
%   \multiply\dimen@ #1\advance\dimen@-2\baselineskip
%   \par\nobreak\vskip\dimen@
%   \hbox{\vrule width\z@ height\Cht depth\z@}\vskip\z@}
% \catcode`@=12
% \setlength{\parskip}{0pt}
% \setlength{\topskip}{\Cht}
% \setlength{\textheight}{43\baselineskip}
% \addtolength{\textheight}{1\zh}
% \makeatother
 
%%%%%%%%%%%%失敗%%%%%%%%%%%%
%\let\formule\subsubsection
%\renewcommand{\subsubsection}[1]{\linespace{1}{\formule#1}}
%%%%%%%%%%%%失敗%%%%%%%%%%%%






% PDF/X-1a
% \usepackage[x-1a]{pdfx}
% \Keywords{pdfTeX\sep PDF/X-1a\sep PDF/A-b}
% \Title{Sample LaTeX input file}
% \Author{LaTeX project team}
% \Org{TeX Users Group}
% \pdfcompresslevel=0
%\usepackage[cmyk]{xcolor}

%biblatex
%\usepackage[notes,strict,backend=biber,autolang=other,%
%                   bibencoding=inputenc,autocite=footnote]{biblatex-chicago}
%\addbibresource{hist-agri.bib}
\let\cite=\autocite

% % % % 
\date{}



%%%%インデックス準備

\usepackage{makeidx}
\usepackage{index}
\makeindex

\newindex{src}{idx1}{ind1}{ソース別料理索引} 


\makeatletter
\renewenvironment{theindex}{% 索引を3段組で出力する環境
  \if@twocolumn
      \onecolumn\@restonecolfalse
    \else
      \clearpage\@restonecoltrue
    \fi
    \columnseprule.4pt \columnsep 2\zw
    \ifx\multicols\@undefined
      \twocolumn[\@makeschapterhead{\indexname}%
      \addcontentsline{toc}{chapter}{\indexname}]%変更点
    \else
      \ifdim\textwidth<\fullwidth
        \setlength{\evensidemargin}{\oddsidemargin}
        \setlength{\textwidth}{\fullwidth}
        \setlength{\linewidth}{\fullwidth}
        \begin{multicols}{3}[\chapter*{\indexname}
	\addcontentsline{toc}{chapter}{\indexname}]%変更点%
      \else
        \begin{multicols}{3}[\chapter*{\indexname}
	\addcontentsline{toc}{chapter}{\indexname}]%変更点%
      \fi
    \fi
    \@mkboth{\indexname}{\indexname}%
    \plainifnotempty % \thispagestyle{plain}
    \parindent\z@
    \parskip\z@ \@plus .3\p@\relax
    \let\item\@idxitem
    \raggedright
    \footnotesize\narrowbaselines
  }{
    \ifx\multicols\@undefined
      \if@restonecol\onecolumn\fi
    \else
      \end{multicols}
      \fi
      \clearpage
    }
\makeatother


%%%% 本文中の参照ページ番号表示 %%%%%%%

\makeatletter

%\AtBeginDocument{%
%  \DeclareRobustCommand\ref{\@ifstar\@refstar\@refstar}%
%  \DeclareRobustCommand\pageref{\@ifstar\@pagerefstar\@pagerefstar}}
\let\orig@Hy@EveryPageAnchor\Hy@EveryPageAnchor
\def\Hy@EveryPageAnchor{%
    \begingroup
    \hypersetup{pdfview=Fit}%
    \orig@Hy@EveryPageAnchor
    \endgroup
  }
  \usepackage{etoolbox}
\if@mainmatter{\let\myhyperlink\hyperlink%
\renewcommand{\hyperlink}[2]{\myhyperlink{#1}{#2} [p.\getpagerefnumber{#1}{}] }}
  \AtBeginEnvironment{recette}{%
\let\myhyperlink\hyperlink%
\renewcommand{\hyperlink}[2]{\myhyperlink{#1}{#2} [p.\getpagerefnumber{#1}{}] }}
  \AtBeginEnvironment{Main}{%
\let\myhyperlink\hyperlink%
\renewcommand{\hyperlink}[2]{\myhyperlink{#1}{#2} [p.\getpagerefnumber{#1}{}] }}
% \if@mainmatter{\let\myhyperlink\hyperlink%
% \renewcommand{\hyperlink}[2]{\myhyperlink{#1}{#2} {\ltjsetparameter{yjabaselineshift=0pt,yalbaselineshift=-.75pt}\footnotesize [p.\getpagerefnumber{#1}{}]}}}
%   \AtBeginEnvironment{recette}{%
% \let\myhyperlink\hyperlink%
% \renewcommand{\hyperlink}[2]{\myhyperlink{#1}{#2} {\ltjsetparameter{yjabaselineshift=0pt,yalbaselineshift=-.75pt}\footnotesize [p.\getpagerefnumber{#1}{}]}}}
%   \AtBeginEnvironment{Main}{%
% \let\myhyperlink\hyperlink%
% \renewcommand{\hyperlink}[2]{\myhyperlink{#1}{#2} {\ltjsetparameter{yjabaselineshift=0pt,yalbaselineshift=-.75pt}\footnotesize [p.\getpagerefnumber{#1}{}]}}}


% \def\@@wrindex#1|#2|#3\\{%
%  \if@filesw
%  \ifx\\#2\\%
%   \protected@write\@indexfile{}{% 
%     %\string\indexentry{#1}{\thepage}%%%改変部分。もとは{#1|hyperpage}{\thepage}%
%     \string\indexentry{#1|hyperpage}{\thepage}%%オリジナル%
%         }%
%         \else
%           \HyInd@@@wrindex{#1}#2\\%
%         \fi
%       \fi
%       \endgroup
%       \@esphack
%     }

\makeatother



%%%%% Obsolete Reference Page Numbers 

%\newcommand{\pref}[1]{[p.\pageref{#1}]}
%\newcommand{\pref}[1]{}







%%%% pandoc が三点リーダーを勝手に変える対策
\renewcommand{\ldots}{\noindent…}
%%%%%下線
\usepackage{umoline}
\setlength{\UnderlineDepth}{2pt}
\let\ul\Underline

\newcommand{\maeaki}{}%使用しないので無効化
\newcommand{\atoaki}{\vspace{1.25mm}}
%%分数の表記Obsolete
\usepackage{xfrac}
\let\frac\sfrac
\newcommand{\undemi}{\hspace{.25\zw}$\sfrac{1}{2}$}
\newcommand{\untiers}{\hspace{.25\zw}$\sfrac{1}{3}$}
\newcommand{\deuxtiers}{\hspace{.25\zw}$\sfrac{2}{3}$}
\newcommand{\unquart}{\hspace{.25\zw}$\sfrac{1}{4}$}
\newcommand{\troisquarts}{\hspace{.25\zw}$\sfrac{3}{4}$}
\newcommand{\quatrequatrieme}{\hspace{.25\zw}$\sfrac{4}{4$}}
\newcommand{\uncinquieme}{\hspace{.25\zw}$\sfrac{1}{5}$}
\newcommand{\deuxcinquiemes}{\hspace{.25\zw}$\sfrac{2}{5}$}
\newcommand{\troiscinquiemes}{\hspace{.25\zw}$\sfrac{3}{5}$}
\newcommand{\quatrecinquiemes}{\hspace{.25\zw}$\sfrac{4}{5}$}
\newcommand{\unsixieme}{\hspace{.25\zw}$\sfrac{1}{6}$}
\newcommand{\cinqsixiemes}{\hspace{.25\zw}$\sfrac{5}{6}$}
\newcommand{\quatrequart}{\hspace{.25\zw}$\sfrac{4}{4}$}

\makeatletter
\def\ps@headings{%
  \let\@oddfoot\@empty
  \let\@evenfoot\@empty
  \def\@evenhead{%
    \if@mparswitch \hss \fi
    \underline{\hbox to \fullwidth{\ltjsetparameter{autoxspacing={true}}
%      \textbf{\thepage}\hfil\leftmark}}%
       \normalfont\thepage\hfill\scshape\small\leftmark\normalfont}}%
    \if@mparswitch\else \hss \fi}%
  \def\@oddhead{\underline{\hbox to \fullwidth{\ltjsetparameter{autoxspacing={true}}
        {\if@twoside\scshape\small\rightmark\else\scshape\small\leftmark\fi}\hfil\thepage\normalfont}}\hss}%
  \let\@mkboth\markboth
  \def\chaptermark##1{\markboth{%
    \ifnum \c@secnumdepth >\m@ne
      \if@mainmatter
        \if@omit@number\else
          \@chapapp\thechapter\@chappos\hskip1\zw
        \fi
      \fi
    \fi
    ##1}{}}%
  \def\sectionmark##1{\markright{%
%    \ifnum \c@secnumdepth >\z@ \thesection \hskip1\zw\fi
    \ifnum \c@secnumdepth >\z@ \thesection \hskip-1\zw\fi
    ##1}}}%
\makeatother

\makeatletter
%%%%%%%% Lua GC
\patchcmd\@outputpage{\stepcounter{page}}{%
  \directlua{%
	if jit then
      local k = collectgarbage("count")
      if k>900000 then 
        collectgarbage("collect")
        texio.write_nl("term and log", "GC: ", math.floor(k), math.floor(collectgarbage("count")))
      end
	end
  }%
  \stepcounter{page}%
}{}{}
\makeatother


%\usepackage{vgrid}% here only to help visualize the problem


\PassOptionsToPackage{unicode=true}{hyperref} % options for packages loaded elsewhere
\PassOptionsToPackage{hyphens}{url}
%
%\documentclass[14Q,a4paperpaper,]{ltjsbook}


\usepackage{lmodern}
\usepackage{amssymb,amsmath}
\usepackage{ifxetex,ifluatex}
\usepackage{fixltx2e} % provides \textsubscript
\ifnum 0\ifxetex 1\fi\ifluatex 1\fi=0 % if pdftex
  \usepackage[T1]{fontenc}
  \usepackage[utf8]{inputenc}
  \usepackage{textcomp} % provides euro and other symbols
\else % if luatex or xelatex
  \usepackage{unicode-math}
  \defaultfontfeatures{Ligatures=TeX,Scale=MatchLowercase}
\fi
% use upquote if available, for straight quotes in verbatim environments
\IfFileExists{upquote.sty}{\usepackage{upquote}}{}
% use microtype if available
\IfFileExists{microtype.sty}{%
\usepackage[]{microtype}
\UseMicrotypeSet[protrusion]{basicmath} % disable protrusion for tt fonts
}{}
\IfFileExists{parskip.sty}{%
\usepackage{parskip}
}{% else
\setlength{\parindent}{0pt}
\setlength{\parskip}{6pt plus 2pt minus 1pt}
}
\usepackage{hyperref}
\hypersetup{
            pdftitle={エスコフィエ『料理の手引き』全注解},
            pdfauthor={五 島 学},
            pdfborder={0 0 0},
            breaklinks=true}
\urlstyle{same}  % don't use monospace font for urls
\setlength{\emergencystretch}{3em}  % prevent overfull lines
\providecommand{\tightlist}{%
  \setlength{\itemsep}{0pt}\setlength{\parskip}{0pt}}
\setcounter{secnumdepth}{0}
% Redefines (sub)paragraphs to behave more like sections
\ifx\paragraph\undefined\else
\let\oldparagraph\paragraph
\renewcommand{\paragraph}[1]{\oldparagraph{#1}\mbox{}}
\fi
\ifx\subparagraph\undefined\else
\let\oldsubparagraph\subparagraph
\renewcommand{\subparagraph}[1]{\oldsubparagraph{#1}\mbox{}}
\fi

% set default figure placement to htbp
\makeatletter
\def\fps@figure{htbp}
\makeatother


\title{エスコフィエ『料理の手引き』全注解}
\author{五 島 学}
\date{}

\begin{document}
\maketitle

{
\setcounter{tocdepth}{2}
\tableofcontents
}
\setlength{\parindent}{1\zw}

\hypertarget{avant-propos}{%
\chapter{序}\label{avant-propos}}

\begin{Main}

\fifteenq
\setstretch{1.3}

もう20年も前のことだ。本書の着想を我が尊敬する師、今は亡きユルバン・デュボワ\footnote{Urbain
  Dubois (1818〜1901)。19世紀後半を代表する料理人。}先生に話したのは。先生は\ruby{是非}{ぜひ}とも実現させなさいと強く勧めてくださった。けれども忙しさにかまけてしまい、\ruby{ようや}{漸}く
1898年になって、フィレアス・ジルベール\footnote{Philéas Gilbert
  (1857〜1942)。19世紀末から20世紀初頭に活躍した料理人。料理雑誌「ポトフ」を主宰した。}君と話し合い協力をとりつけることが出来た。ところがまもなく、カールトンホテル開業のために私はロンドンに呼び戻され、その厨房の準備や運営に忙殺されることとなった\footnote{エスコフィエはセザール・リッツの経営するホテルグループにおいて料理に関わる重要な役割を一手に担っていた。1890年〜1897年にかけてロンドンのサヴォイホテルの総料理長を勤めた後、1898年にはパリのオテル・リッツの、1899年にはロンドンのカールトンホテルの開業に携わり、1920
  年までカールトンホテルで総料理長を務めた。}。本書の計画を実現させるために落ち着いた時間を取り戻さねばならなくなってしまった。

1898年から放置したままだった本書に再び着手出来たのは、多くの同僚たる料理人諸君の助力と、友人でもあるフィレアス・ジルベール君とエミール・フェチュ\footnote{Emile
  Fétu 生没年不詳。}君の献身的な協力を得られたからに他ならない。この一大事業を完成させることが出来たのは、ひとえに皆の励ましと、とりわけ辛抱強く、粘り強く仕事を手伝ってくれた二人の共著者\footnote{ジルベールとフェチュを指しているが、初版にはこの二人の他にも共著者として4人の名が挙げられている。第二版以降は共著者としてジルベールとフェチュの名しかクレジットされていない。第二版は初版から構成も含め大幅な改訂が行なわれた。その作業を実際に行なったのがジルベールとフェチュだったために、他の共著者のクレジットが抹消されたと考えられる。なお、現行の第四版にはエスコフィエの名しかクレジットされていない。}のおかげだ。

私が作りたいと思ったのは立派な書物というよりはむしろ実用的な本だ。だから、執筆協力者の皆には、作業手順を各自の考えにもとづいて自由にレシピを書いてもらい、私自身は、40年にわたる現場経験に即して、少なくとも原理原則、料理における伝統的基礎を明確に説明するのに専念した\footnote{このとおりであれば、具体的なレシピの執筆者は上記のように複数おり、エスコフィエ自身は各章、各節における「概説」に相当する部分と「原注」を担当したことになる。とはいえ、第二版以降については、口頭によるコメントの「聞き書き」的なものが含まれている可能性が。原書の文体における「ゆらぎ」から推測されよう。つまり、エスコフィエは本書の制作にあたっても、やはり「総料理長」であった、と考えていい。そしてそのことはエスコフィエの偉業である本書の価値をいっかな減ずるものではない。}。

本書は、かつて私が構想したとおりとは言い難い出来だが、いずれはそうなるべく努めねばなるまい。それでもなお、現状でも料理人諸君にとって大いに役立つものと信じている。だからこそ、本書を誰にでも、とりわけ若い料理人にも買える価格にした\footnote{1903年の初版の売価は、\href{http://gallica.bnf.fr/ark:/12148/bpt6k65768837}{フランス国立図書館蔵}のものの表紙には、フランス国内で12フランと記したシールが貼られている。また、\href{https://archive.org/details/b21525912}{リーズ大学図書館蔵の第二版}にも同様に国内売価12フランのシールが貼られている。1912年の第三版も同じく12フランだった(\href{http://gallica.bnf.fr/ark:/12148/bpt6k96923116}{フランス国立図書館蔵}のものに価格を示すシールはないが、訳者個人蔵のものには12フランと記されたシールが貼られている)。なお、辻静雄は「1903年の初版発売当時は、800ページでたった8フラン、全く破格の値段だった」(「エスコフィエ 偉大なる料理人の生涯」、『辻静雄著作集』、新潮社、1995年、729〜730頁)と記しているが、その数字の典拠は示されていない。現在と当時の通貨価値、物価の違いが分りにくいため、この「破格に安い」という言葉にはやや疑問が残るだろう。1900年当時の書籍広告において『料理の手引き』初版と同様の八折り版800ページの料理書が、フランス装10フラン、厚紙の表紙のものが11フランとあるため、初版の12フランという価格は、むしろ料理書としては一般的だったと考えられる。つまり、豪華本ではなく、普通に利用できる料理書だということを強調しているに過ぎないと解釈すべきところだろう。なお、八折り判というのは書籍の大きさを表す用語で、概ね縦20〜25
  cm、横12〜16
  cm程度。この序文でことさらに「実用性」や入手しやすい価格であることが強調されているのは、何度も言及されているデュボワとベルナールの名著『古典料理』が四折り判(概ね縦45
  cm、横30 cm)の豪華本であったことを意識していたためとも推測されよう。}。そもそも若い料理人諸君にこそこの本を読んで\ruby{貰}{もら}いたい。今はまだ初心者であったとしても、20年後には組織のトップに立つべき人材なのだから。

私はこの本を豪華な装丁の\footnote{かつてフランスでは、大判の紙の両面に印刷して折ったものを糸で綴じただけの状態(いわゆる「フランス装」)で販売された本を、書店で買い求めた者が別途、業者に製本、装丁の依頼をして自分の好みの想定にさせることが一般的に行なわれていた。}、書棚の飾りのごときにはして欲しくない。そうではなく、いつでも、どんな時でも手元に置いて、分からないことを常に明らかにしてくれる\ruby{盟友}{めいゆう}として欲しい。

本書には五千を越える\footnote{初版、第二版は「五千近い」。第三版になってようやくこの表現になった。}レシピが掲載されているが、それでも私は、この教本が完全だとは思っていない。たとえ今この瞬間に完璧であったとしても、明日にはそうではないかも知れぬ。料理は進化し、新しいレシピが日々創案されている。まことに困ったことだが、版を重ねる毎に新しい料理を採り入れ、古くなってしまったものは改善せねばなるまい。

ユルバン・デュボワ、エミール・ベルナール\footnote{Emile Bernard
  (1827〜1897)。クラシンスキ将軍の料理人を務めた。}両氏の著作\footnote{デュボワとベルナールの共著は他にもあるが、ここでは『古典料理』(1856年)を指している。}に昔から慣れ親しみ、その巨大な影がなおも料理の地平を覆い尽している現在、私としては本書がその後継になって欲しいと思っている。カレーム以後、最高の料理の高みに逹した二人に対し、ここであらためて心から敬意を表させていただきいと思う。

調理現場を取り巻く諸事情により、私は、デュボワ、ベルナール両氏がもたらしたサービス(給仕)面での革新\footnote{\protect\hypertarget{service-russe}{19世紀後半に一般的となった
  「ロシア式サービス」のこと。中世以来、格式の高い宴席では、卓上に大
  皿の料理が一度に何種も並べられ、食べる者がそれぞれ好きなように取り
  分けていた。そして卓上の料理がほぼなくなると、また何種類もの皿が卓
  上に並べられる、というのが数回繰り返された。19世紀中頃から、献立を
  食べる順に1種ずつ、大皿料理の場合は食べ手に見せて回ってから、給仕
  が取り分けて供する方式に変えたものがロシア式サービスである。これと
  対比するかたちで旧来の方式をフランス式サービスと呼ぶようになった。
  ロシア式サービスでは、食卓に大皿を並べない代わりに、花を飾りナフキ
  ンを美しく折るなどの工夫により卓上も洗練されたものとなっていった。
  19世紀パリに駐在していたロシア帝国の外交官クラーキンが提唱したと言
  われている。デュボワとベルナールの『古典料理』序文において詳述され
  ている。}}に対し、こんにちのようなとりわけスピードが重視される目まぐるしい生活リズムに合わせて、より大きな変更を加えざるを得なかった。そもそも物理的理由から、料理を載せる飾り台\footnote{\protect\hypertarget{socle}{socle ソークル。パンや米、ジュレな
  どで作った、料理を盛り付けるために銀の盆の上に据える飾り台。カレー
  ムの時代、つまり19世紀前半にはその装飾に凝ることが多かった。食べも
  ので作られてはいるが、料理の一部ではなく、あくまで装飾的要素でしか
  なかった。この飾り台はロシア式サービスの時代になっても豪華絢爛たる
  宴席においては重要なものとして扱われており、デュボワとベ}ルナール『古典料理』でも相応のページ数を割いて説明がなされている。}を廃止して、シンプルな盛り付けにする新たなメソッドと新たな道具を考案する必要があったのだ。デュボワ、ベルナール両氏が推奨した壮麗な盛り付けを私自身が行なっていた頃はもちろん、今なお二方の思想にはまったく共感している。冗談でこんなことを言っているのではない。しかし、カレームを信奉する者たちは、装飾の才があるが\ruby{故}{ゆえ}に、時代にもはや\ruby{似}{そぐ}わなくなってしまった作品に対して改良を加えようとはしなかった。時代に合わせて改良することこそまさに重要ななのに、だ。本書で奨励している盛り付けは、少なくともそれなりの期間、有用であり続けると思う。全ては変化する。姿を変える。それなのに、装飾芸術の役割が変化しないとなどと主張するとは\ruby{蒙昧}{もうまい}ではないか。芸術は流行によって栄えるものだし、流行のように移ろいやすいものだ。

だが、カレームの時代にはこんにちと同じく\ruby{既}{すで}にあり、料理が続く限りなくならないだろうものがある。それが\href{-料理の基礎}{フォンド
キュイジーヌ}\footnote{この語つまりfonds de
  cuisineについておそらくカレームは使ってはいないが、その概念の萌芽は既にその著書に認められる。この序文において何度も言及されているデュボワ、ベルナール共著『古典料理』においては、1856年の初版では、Des
  Fonds pour
  Potages(デフォンプルポタージュ)の見出しがあるのみだが、fonds de
  Mirepoixなどのレシピ名が、とろみを付けたポタージュ用の鶏のフォン、などとともに収録されており、また、ソースの章の上部(いわゆる柱、ヘッダ部分)には「ソースとフォン」Sauces
  et fonds
  と印刷されている。1868年版になると、明確に、ソースの章の扉の次のページに、Fonds
  de cuisine という見出しが付けられている。このように、fonds de
  cuisineという語は本来、料理のベースとしてあらかじめ仕込んでおき、場合によってはストックしておくものであり、それは液体の場合もあれば固形物の場合もあり、実態は多岐にわたると考えるべきである。まことに残念なことに、日本および他の国々においても、いわゆる「出汁」に相当する液体だけを「フォン」と解釈し、それが定着してしまった結果、この序文および本文におけるfonds
  de cuisine の意味がわかりにくくなってしまっている。}だ。そもそも、料理が見た目にシンプルになっても料理そのものの価値は失なわれないが、その逆はどうだろう?
人々の味覚は絶え間なく洗練され続け、それを満足させるために料理そのものも洗練されることになる。こんにちの余剰活動が精神におよぼす悪影響に打ち\ruby{克}{か}つためには、料理そのものがいっそう科学的な、正確なものとなるべきなのだ。

その意味で料理が進歩すればする程、我々料理人たちにとって、19世紀、料理の行く末に大きく影響を与えた三人の料理人の存在は大きなものとなるだろう。カレームとデュボワ、ベルナールはともすれば技術的側面ばかり評価されるが、料理芸術の基礎において何よりも優れているのだ。

既に物故した名だけ挙げるが、確かにグフェ\footnote{Jules Gouffé
  (1807〜1877)。著書多数。主著『料理の本(1867年)は前半が家庭料理、後半が高級料理(オート・キュイジーヌ)の2部構成になっており、レシピもまず材料表を掲げた後に調理手順を説明するという現代の書き方に近く、挿絵も多く分りやすい。この『料理の手引き』とともに19世紀後半のフランス食文化史における名著のひとつ。19世紀前半からのヴィアールやオドが版を内容を増補しながら版を重ねたのに対して、この本は再版の際もほとんど異同がない点もまた特徴のひとつ。}、ファーヴル\footnote{Joséph
  Favre
  (1849〜1903)。スイス生まれの料理人で、パリ、ドイツ、イギリス、ベルギー等において活躍した。著書『料理および食品衛生事典』
  (1884〜1895年)。この『事典』に収録されているレシピの数は5,531であり(番号が振られている)、エスコフィエがレシピ数5千という表現にややこだわりを示しているように思われるのも、ほぼ同時代の出版物であるファーヴルの『事典』を意識していた可能性はある。}、エルーイ\footnote{Edouard
  Hélouis(生没年不詳)。イギリスのアルバート王配(ヴィクトリア女王の夫)(1819〜1861)やイタリアのヴィットーリオ・エマヌエーレ二世(1820〜1878)に仕えたという。著書『王室の晩餐』(1878年)。}、ルキュレ\footnote{『実践的料理』(1859年)の著者C.
  Reculetのこと。}はとても素晴らしい著作を残した。だが、『古典料理』という\ruby{稀代}{きたい}の名著に\ruby{比肩}{ひけん}し得るものはひとつとしてない\footnote{デュボワ、ベルナールの『古典料理』を称揚するあまりにこのような表現になっていると思われるが、そもそもファーヴルの著書は『事典』であって教本ではない。また、グフェ『料理の本』については、掲載レシピ数は少ないものの、記述の明快さと、技術の習熟度合いに合わせて家庭料理と高級料理(haute
  cuisine
  オートキィジーヌ)の二部構成にするなどの配慮から、こんにちでも充分に料理の入門〜中級教本として高く評価出来るものだろう。}。

料理人諸君に、新たに本書を使っていただくにあたり、言うべきことがある。いろいろな料理書、雑誌を読み散らかすのもいいが、偉大な先達の不朽の名著はしっかり読み込むように、と。\ruby{諺}{ことわざ}にあるように「知り過ぎることなはい」。学べば学ぶ程、さらに学ぶべきことは増えていく。そうしていくうちに、思考が柔軟なものとなり、料理が上達するためにより効果的な方法を知ることも出来るだろう。

本書を\ruby{上梓}{じょうし}するにあたって\ruby{唯}{ただ}ひとつ望むこと、切に願う\ruby{唯}{ゆいいつ}のことは、上記の点において、本書の対象たる読者諸君が我が\ruby{言}{げん}に耳を傾け、実践するさまを見ることに尽きる。
\nopagebreak

\begin{flushright}
A. エスコフィエ \nopagebreak
\end{flushright}

1902年11月1日

\end{Main}

\newpage

\hypertarget{introduction-deuxieme-edition}{%
\section[第二版序文]{\texorpdfstring{第二版序文\footnote{この第二版序文は文体が初版序文と異なり、とりわけ前半部分については、いわゆる「悪文」と見なさざるを得ないものとなっている。また、前半と後半でも文体の「ゆらぎ」のようなものが認められる。内容から判断するかぎり、エスコフィエ自身の言葉であることは確かだが、末尾に署名がなく日付のみ記されていることも含めて考えると、ジルベールとフェチュによる「聞き書き」によって作成された可能性も完全には否定できないと思われる。}}{第二版序文}}\label{introduction-deuxieme-edition}}

\normalsize
\setstretch{1.1}
\vspace*{1\zw}

\begin{Main}

ここに第二版を上梓するに至ったわけだが、二人の共著者による熱意あふれる仕事のおかげで、私の強い期待をさらに越える本書の成功が約束されたも同然だろう。だからこそ、共著者両君および本書の読者諸君に心からの謝辞を申しあげる次第だ。また、ありがたいことに、称賛の言葉を寄せてくださった方々と、貴重な批判をくださった方々にも御礼申しあげる。批判については、それが正当なものと思われる場合については、本書に反映させるべく努めさせていただいた。

かくも多くの人々に本書を受け入れていただけたことへの謝意を表するには、本書における技術的な価値を高め、初版ではロジカルにレシピを分類しようとしたが故に生じた欠点を解消する他ないだろう。それは、調理理論とレシピを損なうことなしに、本書の計画段階において簡単に済まさざるを得ないと思われたテーマについて\ruby{能}{あた}う限り肉薄することでもある。私たちは本文の見直しをするとともに、多くのレシピを追加した。そのほとんどは調理法と盛り付けにおいて、こんにちの顧客のニーズを\ruby{鑑}{かんが}みて着想したものであり、そのニーズが正当かつ実現可能な範囲において、顧客への給仕のペースが日増しに加速していく傾向をも考慮に入れたものだ。こういった傾向は数年来まさしく際立ってきているが\ruby{故}{ゆえ}に、我々としても常に気を配っておかねばならぬ。

「料理芸術」というものは、その表現形態において、社会心理に左右されるものだ。社会から受ける衝撃に逆らわぬことも必要であり、\ruby{抗}{あらが}
えぬことでもある。快適で安楽な生活がいかなる心配事にも乱されることのないような社会であれば、未来が保証され、財をなす機会もいろいろあるような社会であれば、料理芸術はたゆまぬことなく驚異的な進歩を遂げるだろう。料理芸術とは、ひとが得られる悦びのうちでもっとも快適なもののひとつに寄与しているのだから。

反対に、安穏とした生活の出来ぬ、商工業からもたらされる\ruby{数多}{あま
た}の不安で頭がいっぱいになるような社会において、料理芸術は心配事でいっぱいの人々の心のごく限られた部分にしか美味しさを届けられない。ほとんどの場合、諸事という渦巻きに巻き込まれた人々にとって、食事をするという必要な行為はもはや悦びではなく、辛い義務でしかないのだ。

\ruby{斯}{か}くのごとき生活習慣は\ruby{嘆}{なげ}いていい、\ruby{否}{い
な}、嘆くべきことなのだ。食べ手の健康という観点からも、食べたものを胃が受け付けないという結果になるとしたら、それは絶対に生活習慣が悪いのだ。そういう結果を抑える力は私に出来る範囲を越えている。そういう場合に調理科学が出来ることといえば、軽率な人々に\ruby{能}{あた}うかぎり最良の食べものを与えるという対症療法だけなのだ。

顧客は料理を早く出せと言う。それに対して私たち料理人としては、ご満足いただけるようにするか、失望させてしまうことのどちらかしか出来ない。料理を早く出せという顧客の要求を拒む方法があるとするなら、それ以上の方法で顧客にご満足いただけるようにすることしかない。だから、私たちは顧客の気まぐれの前に折れざるを得ないのだ。これまで私たちが慣れ親しんできた仕事のやり方では、これまでの給仕のスタイルでは、顧客の気まぐれに応えることが出来ぬ。意を決して仕事の方法を改革すべきなのだ。だがひとつだけ、変えてはならぬ、手をつけてはならぬ領域がある。料理ひとつひとつのクオリティだ。それは、料理人にとって仕事のベースとなるフォンや事前に仕込んでおいたストック類がもたらすゆたかな風味に他ならぬ。私たちは既に、盛り付けの領域においては改革に着手した。足手まといにしかならぬ多くのものは既に姿を消したか、いままさに消え去らんとしている。料理の飾り台\footnote{socle
  (ソークル)、\protect\hyperlink{socle}{「序」p.ii、訳注6}参照。}、料理の周囲の装飾\footnote{bordure
  ボルデュール。本書においてもガルニチュールの扱いにおいてこの指示はあるが、19世紀のものと比較するとかなりシンプルな内容になっている。}、飾り串\footnote{hâtelet
  アトレ。一方の端に動物などの姿の装飾の施された銀製の串に、トリュフやクルヴェット(海老)などを事前に別の串(ブロシェット)で焼いてからこの飾り串に刺し直し、それを大きな塊肉や丸鶏、大型の魚
  1尾の料理に刺した。19世紀初頭、カレームの時代に全盛となり、その著書『パリ風料理』において詳述されている。19世紀末まではこの装飾がなされることが多かった。また、その飾り串そのものが美麗な装飾品であるためにコレクションの対象になっていた。}などのことだ。この方向性は推し進められると思う。これについては後述しよう。私たちはシンプルであるということを極限まで追究したい。それと同時に、料理の風味や栄養面での価値を増すことも目指している。料理はより軽い、弱った胃にも優しいものにしたいと考えている。私たちはこの点にのみ尽力したい。料理において役をなさない大部分はすっかり剥ぎ取ってしまいたいと考えているのだ。一言でまとめると、料理は芸術であり続けつつも、より科学的なものとなるだろうし、その作り方はいまだ経験則に基づいただけのものばかりであるが、ひとつのメソッド、偶然などに左右されない正確なものになっていくことだろう。

こんにちは料理の過渡期にある。古典料理メソッドの愛好者はいまなお多く、私たちもそれを理解し、その思想に心から共感するところもある。だが、食事というものがセレモニーであり、かつパーティであった時代を懐しんでどうするというのだ?
古典料理がこんにちの美食家に至福の時を与えるために力を発揮出来る場がどこにあるというのだ?
いったいどうすれば、美食と宴の神コモス\footnote{フランス語 Comus
  (コミュス)。ラテン語では同じ綴りでコムスと読む。ギリシア、ローマ神話における、悦びと美食の神。18世紀の料理本作家マランの主著は『コモス神の贈り物』がタイトル。}に捧げ物を供えるという幸せな機会を毎回得られるのだろうか?だから私たちは本書において、個人的な創作よりむしろ伝統的なフランス料理のレシピ集として、こんにちの料理のレパートリーから姿を消してしまったものも残すことに固執した。その名に値する料理人なら、機会さえ与えられたら王侯貴族も近代の大ブルジョワもひとしく満足させるためには、知っておくべきものなのだ。時間のことなんぞ気にもせぬ穏かな美食家の方々にも、時こそ全てと言わんばかりの金融家やビジネスマンたちにも満足していただくために。だから、本書が新しいメソッドに偏ったものだという非難にはあたらない。私はただ単に、料理芸術の進化の歩みをたどり、いまの時代に即しつつ、食べ手すなわち食事会の主催者と招待客の皆様の意向を絶対的なものとして、それに従いたいと願っているだけなのだ。食べ手の意向に対して私たち料理人は
\ruby{頭}{こうべ}を垂れて従うことしか出来ぬのだから。

私たちは、料理の美味しさを損なうことなくより早く料理を提供できるような方法を、料理人各人が自らの嗜好を犠牲にすることなしに探求すべく
\ruby{誘}{いざな}うことこそが、料理人諸君にとって有益と信じている。全体として、私たちのメソッドはまだまだ日々のルーチンワークに依存し過ぎているものだ。顧客の求めに応えるため、私たちは既に仕事のやり方をシンプルなものにせざるを得なかった。だが、残念ながらいまだ\ruby{途}{み
ち}\ruby{半}{なか}ばに過ぎぬと感じている。私たちは自己の信念をしっかり堅持しており、どうしようもない場合にのみ自説を曲げることもある。だから、装飾に満ちた飾り台を廃止した一方で、盛り付けに時間のかかる厄介で複雑なガルニチュールは残してある。こういったガルニチュールを濫用することはガストロノミーの観点から言って、常に間違っているのは事実だが、残しておくべきものと思われる。それを求める顧客あるいは食事会主催者に絶対に従う必要のある場合はとりわけそうだ。ごく稀にとはいえ、料理の美味しさを損なうことなくそれらを実現可能なこともあるからだ。時間と金銭、広くてスタッフの充実した会場、という3つの本質的要素を最大限活用可能な場合のことだが。

通常の厨房業務においては、ガルニチュールをかなりシンプルな、せいぜい3〜
4種の構成要素からなるものに減らさざるを得なくなっている。そのガルニチュールを添える料理がアントレであれルルヴェ\footnote{19世紀前半まで主流であった「フランス式サービス」つまり、一度に多くの料理の皿を食卓に並べるという給仕方式において、ポタージュを入れた大きな深皿が空くと、それを給仕が下げて、豪華な装飾を施した大きな塊肉の料理がポタージュを置いてあった場所に据えられた。これを
  \protect\hypertarget{releve}{relevé}ルルヴェ(交代したもの、の意)と呼んだ。エスコフィエの時代にはフランス式サービスではなくロシア式サービスに移っており、大きな塊肉の料理や大型の魚1尾まるごとを大皿で出し、給仕が切り分けて配膳するようになっていたが、名称はそのまま残った。Entréeアントレ(もとは「入口」の意)は現代において「前菜」の意味で用いられているが、食卓に大皿で並べられた肉料理(場合によっては魚料理も含む)の総称としてこの語が用いられていた。本書はそれを踏襲している。本書においてルルヴェおよびアントレに分類されている料理の多くは現代においてコース料理の「メイン」に相当するものが多く、実際、英語ではコース料理のメインのことを現在でもこの語で表わすことが多い(前菜はappetizerアペタイザーと呼ぶ)。}であれ、牛・羊肉料理であれ、家禽であれ魚料理であれ、そうせざるを得ない。そのようにして構成要素を減らしたガルニチュールは、素早い皿出しが要求される場合には必ず、ソースと同様に別添で供するのがいい。その場合、盛り付けは奇抜というくらいシンプルなものとなるが\ldots{}\ldots{}メインの料理はより冷めない状態で、より早く、よりきれいに供することが可能になる。給仕が料理を取り皿に分けてお客様に出すにせよ、お客様が大皿を自分たちで受け渡して取り分けるにせよ、サービス担当者は安心して仕事が出来るし、そのほうが容易だ。メインの大皿が山盛りになることはないし、その上に盛り付けられたいろいろな素材のガルニチュールも簡単に取ることが出来るからだ。

こんにちの他のシステムだと、料理を載せるための台や装飾のための飾り串を作り、さらに料理の周囲にガルニチュールを配置するのに、看過出来ぬ程の時間を要していた。こういう盛り付けというのは、料理そのものがさして大きくないものであっても、食べ手の人数が少ない場合であっても、大面積の皿を用いる必要があった。だから、お客様が料理を自分たちで受け渡して取り分ける必要がある場合などは、お客様にとっても、サービス担当者にとってもまことに窮屈なものであった。これは、複雑な構成のガルニチュールの持つ大きな欠点のひとつとして無視できないことだ。他の欠点というのは、あらかじめ盛り付けを行なうことによって美味しさが減じてしまうこと、食べ手が少人数の場合には必然的に、料理を見せて周る間に冷めてしまうこと、などがある。こういう愉快とは言えぬことの結果は何とも情けないことになる。つまり、お客様に大皿に盛り付けた料理をお見せするのはほんの一瞬だけ、お客様は多少なりとも豪華で精密に盛り付けられた料理をちらりと見る暇があるかないか、ということだ。昔日のごとき豪華壮麗な料理を供することの可能な場所もこんにちでは少なくなってきたが、それ以外のところでもこういった悪習が頑固なまでに続けられているというのは、それが昔からの習慣だということでしか説明がつかぬ。

給仕のスピードを容易に上げるために、大きな塊肉の料理でない場合には毎回、下の図のごとき四角形の深皿を出来るだけ用いるよう是非ともお勧めしたい。温かい料理でも、冷製の料理でも、この皿は非常に優れたものであるから、その目的において厨房に備えておくべきものとして他の追随を許さないと言える
\footnote{この段落は、初版の序文の後にある「盛り付け方法をシンプルにすることについて」という挿絵付きの節の内容を短かく縮めたために、ややわかりにくいものになっている。ただし、第二版および第三版においては序文の最後に皿の挿絵が添えられている。}。

繰り返しになるが、本書が新しい方法を勧めているからといって、偏見で古典的なものを悪いと断じているのでは決してない。私たちは、料理人諸君に、顧客たちの生活習慣や味の好みを研究し、自らの仕事をそれらに適合させるよう
\ruby{誘}{いざな}いたいと思っているだけなのだ。我々料理人にとって高名な師とも呼ぶべきカレームは、ある日、同業たる料理人のひとりとおしゃべりをしていた際に、その料理人が仕えている主人の洗練さに欠けた食事の習慣や下卑た味覚を苦々しげに語るのを聞かされたという。その食事の習慣と味覚に憤慨して、自分が人生をかけて追究してきた知的な料理の原則を曲げてまで仕え続けるくらいなら、いっそ辞めてしまいたいと思っている、と。カレームはこう答えた。「そんなことをするのは君のほうが間違っているよ。料理において原則なんていくつも存在しないんだ。あるのはひとつだけ、仕えているお方に満足していただけるか、ということだけなんだよ」と。

今度は我々がその答を考える番だ。自分たちの習慣やこだわりを、料理を出す相手に押しつけるなどと言い張るとしたら、まったくもって馬鹿げたことだ。我々料理人は食べ手の味覚に合わせて料理することこそが第一でありもっとも本質的なことなのだと、私たちは確信している。

私たちがかくも安易に顧客の気まぐれにおもねったり、過度なまでに盛り付けをシンプルにするせいで、料理芸術の価値を下げ、単なる仕事のひとつにしてしまっている、と非難する向きもあるだろう。\ldots{}\ldots{}だがそれは間違いだ。シンプルであることは美しさを排するものではない\footnote{この序文における名句のひとつ。ただし、エスコフィエの時代における「シンプル」とポストモダン以降の時代であるこんにちの「シンプル」はもはや具体的な意味がまったく違うことに留意。もちろん、理念として普遍的な価値を持つ名句であることは確かだろう。}。\{\#la-simplicite\}

ここで、本書の初版において盛り付けについて述べた部分を繰り返すことをお許しいただきたい。

「どんなにささやかな作品にも自らの最高の印をつけられる才というのは、その作品をエレガントで歪みのないものに見せられるわけで、技術というものに不可欠だと私は信じている。

だが、職人が美しい盛り付けを行なうことで自らに課すべき目的とは、食材を他に類のない方法で節度をもって用いつつ大胆に配置することによってのみ、実現されるのだ。未来の盛り付けにおいて絶対に守るべきこととして、食べられないものを使わないこと、シンプルな趣味のよささこそが未来の盛り付けに特徴的な原則となるだろうことを、認めるべきなのだ。

そのような仕事を成し遂げるために、能力ある職人にはいくつもの手段がある。トリュフ、マッシュルーム、固茹で卵の白身、野菜、舌肉などの食べられるものだけを用いて、素晴らしい装飾を組み合わせ、無限に展開できるのだ。

王政復古期\footnote{1814年ナポレオンが退位して国外へ亡命、ルイ18世を戴く王政へ回帰した時期。1830年まで続いたが7月革命でブルボン家は断絶し、その後オルレアン朝による七月王政が1848年まで続いた。}に料理人たちによって流行した複雑な盛り付けの時代は終わった。だが、特殊な例になるが、古い方法で盛り付けをしなければならない場合もあり、そういう時は何よりもまず、盛り付けにかかる時間と利用できる手段を見積らなくてはならない。土台の形状を犠牲にしなくても、装飾の繊細さを忘れなくても、風味ゆたかな素材を軽んじたり劣化させてしまっては、価値のないものにしかならないのだ」。

以上の見解はずっと変わっていない。料理は進歩する(社会がそうであるように)。だが常に芸術であり続けるのだ。

例えば、1850年から人々の生活習慣、習俗が変化したことを皆が認めるにやぶさかでないように、料理もまた変化するのだ。デュボワとベルナールの素晴しい業績は当時のニーズに応えたものだ。だが、たとえ二人がその著書と同じく永遠の存在であったとしても、彼らが称揚した形態は、料理の知識として、我々の時代の要求に応えうるものではない。

私たちは二人の名著を尊重し、敬愛し、研究しなくてはならない。それはカレームとともに、料理人の仕事の\ruby{礎}{いしずえ}たるものだ。だが、書いてあることを盲目的に真似るのではなく、私たち自身で新たな道を切り
\ruby{拓}{ひら}き、私たちもまたこの時代の習俗や慣習に合わせた教本を残すべきと考える次第だ。

\begin{flushright}
1907年2月1日
\end{flushright}

\end{Main}

\newpage

\hypertarget{introduction-troisieme-edition}{%
\section{第三版序文}\label{introduction-troisieme-edition}}

\begin{Main}

\vspace*{1\zw}

『料理の手引き』第三版を同業たる料理人諸賢に向けて上梓するにあたり、絶えず本書を好意的に支持してくださったことと、多くの方々から著者一同にお寄せくださった励ましのお言葉に対し、あらためて深く御礼申しあげる次第だ。

第二版序文の内容につけ加えるべきことは何もない。というのも、第二版序文で料理という仕事について申しあげたことは、1907年当時も今も変わっていない事実だからだし、今後も長くそうであり続けるだろう。とはいえ、この第三版は内容を精査し、かなりの部分を改訂してある。かつては予測でしかなかったことを実証し、この『料理の手引き』初版の序文においてエスコフィエ氏\footnote{この表現から、第三版序文がエスコフィエ自身ではなく、フィレアス・ジルベールかエミール・フェチュのいずれか、あるいは二人によって書かれたと判断される。}が以下のように書かれた約束も果せたと思う。「本書には五千近くもの\footnote{初版および第二版では「五千近い」となっており、第三版で「五千以上」と表現が変更された。}レシピが掲載されているが、それでも私は、この教本が完全だとは思っていない。たとえ今この瞬間に完璧であったとしても、明日にはそうではないかも知れぬ。料理は進化し、新しいレシピが日々創案されているのだ。まことにもって不都合なことだが、版を重ねる毎に新しい料理を採り入れ、古くなってしまったものは改良を加えねばなるまい。」

この言葉が、前回の第二版から300ページを増やしたことの説明となっているわけで、この新版でいくつかの変更を我々が必要と考えた理由でもある。

\begin{enumerate}
\def\labelenumi{\arabic{enumi}.}
\item
  判型の変更\ldots{}\ldots{}あえて判型を大きくすることで、より扱いやすいものとしたこと\footnote{初版および第二版はいわゆる「八折り版」約21.5
    cm×13.5 cmであったのに対し、第三版は約24 cm×16
    cm、つまり現代のB5版よりほんの少し小さめの判型。}
\item
  巻末の目次の組みなおし\ldots{}\ldots{}当初は料理の種類別であったが、本書全体の項目をアルファベット順にまとめたこと\footnote{原文ではTable
    des
    Matière「目次」とあるが、これは巻頭の章を示す目次のことではなく、巻末の「索引」のこと。}
\item
  時代遅れになったと思われるレシピを相当数削除し、その代わりとしてこの数年の間に創案され好評を博したレシピを追加したこと
\end{enumerate}

既に大著であって本書にこれらの変更を加えるために、我々は第二版の巻末に付されていた献立のページを削除せざるを得なかった。

献立についても内容を一新し、多くの献立例を追加して、『メニューの本』という独立した書籍として、この第三版と同時に刊行する予定となっている。この『メニューの本』において我々は献立とその説明文はもちろんのこと、大規模な厨房における日々の業務配分を示す表を入れておいた。

このように別冊とすることで、献立の作成という非常に重要な問題を適切に展開し、ゆとりを持って論じることが可能となったわけだ。

この新刊『メニューの本』は料理人諸賢だけではなくメートルドテル、食事施設の責任者に必携のものとなった。さらには必要なものを奇抜なまでに単純化してしまう家庭の主婦にとっても必携となろう。我々は上記の改良点が、これまで多くの好意的見解をお寄せくださった料理関わる皆様方に、好意的に受け容れていただけると信じている。また、料理芸術の栄光のもと未来に続くモニュメントを建てるべく努めた我々のささやかなる尽力が、料理芸術に利をもたらさんことを信じる次第だ。

\begin{flushright}
1912年5月1日
\end{flushright}

\end{Main}

\hypertarget{introduction-quatrieme-edition}{%
\section{第四版序文}\label{introduction-quatrieme-edition}}

\begin{Main}

\vspace*{1\zw}

『料理の手引き』第三版刊行当時(1912年5月)から後、他の職業、産業と同様に料理界もまた大いなる危機に見舞われた\footnote{第一次世界大戦(1914〜1918)による社会的影響を指している。フランスは戦中から戦後にかけて激しいインフレに見舞われた。なお、この第四版から出版社がそれまでのラール・キュリネールからフラマリオン社に変わった。}。こんにちもなお料理は厳しい試練にさらされている。しかしながら、料理界はその試練に耐えてきたし、戦後のこの辛い時期に終止符を打ち、料理界がさらに前進し始めるのもさして遠いことではないと信じている。だが、目下のところ、あらゆる食材の異常なまでの高騰により、料理長諸賢が責務を果すことがひどく難しくなっている。料理長がその責務を果すということの困難さを経験上よく知っているからこそ、今回の版において我々は、多くのレシピ、とりわけガルニチュールについて、その本質的なところを曲げることなしに、よりシンプルなものにすることにこだわった。

さらに、もはやあまり興味を持たれないであろうレシピは全て削除して、その代わりに近年創案されたレシピを収録することとした。

したがって、料理人諸賢および料理に関心を持つ皆様方に向けてこの『料理の手引き』第四版を上梓するにあたり、旧版同様、皆様に温かく受け容れていただけると信じる次第である\footnote{原書の文体から、この序文も第三版序文と同様に、ジルベールとフェチュによって書かれた可能性も考えられる。}。

\begin{flushright}
1921年1月
\end{flushright}

\end{Main}

\newpage
\small
\setstretch{1.0}

\hypertarget{remarque-sur-la-simplification-des-procedes-de-dressage}{%
\section{【参考】盛り付けをシンプルにするということ(初版のみ)}\label{remarque-sur-la-simplification-des-procedes-de-dressage}}

\begin{Main}

本書では、かつては料理の盛り付けによく用いられた飾り串\footnote{hâtelet
  アトレ。}、縁飾り \footnote{bordure ボルデュール。}、クルトン\footnote{菱形やハート形にしたパンを揚げたもの。}、チョップ花\footnote{papillote
  パピヨット。紙製で、骨付き肉の先端を飾るもの。}などを使う指示がほとんど出てこない。著者としては、盛り付け方法を近代化すると同時に、ほぼ完全に上記のものどもを削除しなくしてしまいたいとさえ考えたくらいだ。

我らが先達が考えていたような盛り付けには、長所がたったひとつしかない。皿を荘厳に、魅力的な姿にすることで、料理を味わう前に、食べ手の目を楽しませ、喜んでいただくということだ。

だが、そうした盛り付けの作業は複雑で難しいものであり、かなりの時間を必要とする。比較的少人数の宴席でないかぎりは、こうした盛り付けは事前に用意しておく必要がある。そのようにして作られた料理は、それを置いておく場所のことを考えに入れないとしても、必ずといっていい程、冷めてしまっている。また、料理を載せる台や縁飾り、飾り串に費す時間も考えなくてはならないし、そういった装飾にかかる費用も考えなくてはならない。忘れてはならないことだが、そのように装飾した皿の見た目の調和がとれている時間というのは、その皿をお客様にお見せする間だけなのだ。メートルドテルのスプーンが料理に触れるやいなや、かくも無惨な姿となりお客様の目には不快なものとなってしまう。こういう不都合はなんとしても改善しなければならなかったのだ。

ここで図に示すような四角形の皿を採用したことで、上記のような問題は解決したと考えている。この皿はパリのリッツホテルで初めて用いられ、ロンドンのカールトンホテルにおいて正式に採用されることとなったものだ。この皿を用いることの利点は絶大で、これを用いない盛り付けなどもはや考えられない程だ。この皿は場所をとらず、皿の内側に盛り付けられた料理は冷めることがない。蓋との距離が近いから保温されているわけだ。魚や肉の切り身は上に重ねて盛るのではなく、ガルニチュールとともに並べて盛り付けることが出来る。そうすることで、最初に給仕されるお客様から最後に給仕される方まで、料理は美味しそうな見た目を保つことが出来るのだ。その結果、クルトンやチョップ花、皿の上にしつらえる料理を載せる台や縁飾り、飾り串、昔の給仕で用いられた面倒なクロッシュ\footnote{cloche
  主に金属製で半球形の保温を目的としたディッシュカバー。}は不要なものとなる。

この皿は冷製料理にもまた便利に使うことが出来る。周囲に氷を積み重ねて囲うか、薄い氷のブロックの上に盛り付ければ、飾りには、ごく繊細なジュレだけていい。そのような繊細なジュレを使うのは昔の方法では不可能だった。かくして、邪魔にさえ思える飾り台も、皿の底の飾りも、アトレも必要なくなった。ショフロワは1切れずつ並べて、周囲を琥珀色のとろけるようなジュレで満たしてやればいい。ムースはもはや「つなぎ」をまったく、あるいはほとんど必要としない。こういうことが、冷製料理の芸術的な見た目を、豪華さや美しさという点でいっかな失なうことなく可能となるのだ。

この新式の什器とそれによって実現可能となる料理に習熟することについて料理人諸君にお報せすることは我々の義務であると考える。利点がとても大きいので、あえて申しあげるが、これを使うことが、給仕を素早く、きれいに、経済的に、そして文句ないまでに実践的なものにする唯一の方法である。

\end{Main}

\hypertarget{avertissement-premier-edition}{%
\section{【参考】初版はしがき}\label{avertissement-premier-edition}}

\begin{Main}

本書はある特定の階層の料理人を対象としているものではなく、全ての料理人が対象であるため、本書のレシピは、経済的観点や料理人が実際に利用可能な手段に応じて、改変できるものだということを述べておきたい。

本書に収められたレシピはすべて、グランドメゾンでの仕事における原則にもとづいて組み立てて調整してある。だから、より格下の店舗などでも、必然的に量を減らせば作れるだろうし、適価で提供出来るようにもなるだろう。

ひとつひとつの項目において、いろいろな飲食を提供する形態を網羅するようにレシピを書くことが不可能だったということは理解されよう。料理人自身が自主性をもって本書の内容を補えるし、そうすべきなのだ。ある者たちにとって非常に大切なことが、大多数の者にとってはそこそこの興味しか引かず、一般的に見たら無益で幼稚に思われることだってあるのだ。

だから、本書に収録したレシピは最大の分量でまとめられたものを考えるべきであり、必要に応じて、各人の判断および物理的に出来る範囲に合わせて、量を減らして作るといい。

\end{Main}

\normalsize
\setstretch{1.0}

\setlength{\parindent}{0pt}

\end{document}
