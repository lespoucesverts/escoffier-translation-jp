\documentclass[twoside,12Q,b5paper,tombo]{ltjsbook}
\usepackage{amsmath}%数式
\usepackage{amssymb}
\usepackage[no-math]{fontspec}
%\usepackage{xunicode}

\usepackage[unicode=true]{hyperref}
\hypersetup{breaklinks=true,
             bookmarks=true,
             pdfauthor={},
             pdftitle={},
             colorlinks=true,
             citecolor=blue,
             urlcolor=blue,
             linkcolor=magenta,
             pdfborder={0 0 0}}
\urlstyle{same}

\setmainfont[Ligatures=TeX,Scale=1.4]{EB Garamond}%fontspecによるフォント設定
%\setmainfont[Ligatures=TeX]{TeX Gyre Pagella}%ギリシャ語を用いる場合はこちら
%\setsansfont[Scale=MatchLowercase]{TeX Gyre Heros}  % \sffamily のフォント
\setsansfont[Ligatures=TeX, Scale=1]{Linux Biolinum O}     % Libertine/Biolinum
\setmonofont[Scale=MatchLowercase]{Inconsolata}       % \ttfamily のフォント

\usepackage[cmintegrals,cmbraces]{newtxmath}%数式フォント
\usepackage{ebgaramond-maths}

\usepackage{luatexja}
\usepackage{luatexja-fontspec}
%\ltjdefcharrange{8}{"2000-"2013, "2015-"2025, "2027-"203A, "203C-"206F}
%\ltjsetparameter{jacharrange={-2, +8}}
\usepackage{luatexja-ruby}

%%%%和文仮名プロポーショナル
\usepackage[hiragino-pron,expert,deluxe]{luatexja-preset}
%\newopentypefeature{PKana}{On}{pkna} % "PKana" and "On" can be arbitrary string
%\setmainjfont[
%    JFM=prop,PKana=On,Kerning=On,
%    BoldFont={YuMincho-DemiBold},
%    ItalicFont={YuMincho-Medium},
%    BoldItalicFont={YuMincho-DemiBold}
%]{YuMincho-Medium}
%\setsansjfont[
%    JFM=prop,PKana=On,Kerning=On,
%    BoldFont={YuGothic-Bold},
%    ItalicFont={YuGothic-Medium},
%    BoldItalicFont={YuGothic-Bold}
%]{YuGothic-Medium}
%%%%和文仮名プロプーショナルここまで
\renewcommand{\bfdefault}{bx}%和文ボールドを有効にする
\renewcommand{\headfont}{\gtfamily\sffamily\bfseries}%和文ボールドを有効にする

\defaultfontfeatures[\rmfamily]{Scale=1.2}%効いていない様子
\defaultjfontfeatures{Scale=0.92487}%和文フォントのサイズ調整。デフォルトは 0.962212 倍%ltjsclassesでは不要?
%\defaultjfontfeatures{Scale=0.962212}
%\usepackage{libertineotf}%linux libertine font %ギリシア語含む
%\usepackage[T1]{fontenc}
%\usepackage[full]{textcomp}
%\usepackage[osfI,scaled=1.0]{garamondx}
%\usepackage{tgheros,tgcursor}
%\usepackage[garamondx]{newtxmath}
\usepackage{xfrac}

%レイアウト調整
\usepackage{layout}
%\setlength{\hoffset}{-1truein}
%\setlength{\hoffset}{50pt}
%\setlength{\oddsidemargin}{0pt}
%\setlength{\evensidemargin}{-.5truein}
%\setlength{\textwidth}{\fullwidth}%%ltjsclassesのみ有効
%\setlength{\textwidth}{14cm}
%\setlength{\marginparsep}{0pt}
%\setlength{\marginparwidth}{0pt}
%\makeatletter
%\setlength{\skip\footins}{9\p@ \@plus 4\p@ \@minus 2\p@}
%\makeatother
  


% PDF/X-1a
% \usepackage[x-1a]{pdfx}
% \Keywords{pdfTeX\sep PDF/X-1a\sep PDF/A-b}
% \Title{Sample LaTeX input file}
% \Author{LaTeX project team}
% \Org{TeX Users Group}

% \pdfcompresslevel=0
%\usepackage[cmyk]{xcolor}

%biblatex
%\usepackage[notes,strict,backend=biber,autolang=other,%
%                   bibencoding=inputenc,autocite=footnote]{biblatex-chicago}
%\addbibresource{hist-agri.bib}
\let\cite=\autocite

% % % % 
\date{}

\begin{document}
\layout

\section*{序}\label{ux5e8f}
\addcontentsline{toc}{section}{序}

もう20年も前のことだ。本書の着想を我が尊敬する師、今は亡きユルバン・デュ
ボワ\footnote{Urbain Dubois (1818〜1901)。19世紀後半を代表する料理人。}先生に話したのは。先生は是非とも実現させなさいと強く勧めてくだ
さった。けれども忙しさにかまけてしまい、漸く1898年になって、フィレアス・
ジルベール\footnote{Philéas Gilbert
  (1857〜1942)。19世紀末から20世紀初頭に活躍した料
  理人。料理雑誌「ポトフ」を主宰した。}氏と話し合い協力をとりつけることが出来た。とことがまもな
く、カールトンホテル開業のため、私はロンドンに呼び戻され、その厨房の準
備や運営に忙殺された\footnote{エスコフィエはセザール・リッツの経営するホテルグループにおいて料
  理に関わる重要な役割を一手に担っていた。1890年〜1997年にかけてロン
  ドンのサヴォイホテルの総料理長を勤めた後、1898年にはパリのオテル・
  リッツの、1899年にはロンドンのカールトンホテルの開業に携わり、1920
  年までカールトンホテルで総料理長を務めた。}。本書の計画をを実現させるために落ち着いた時間
を取り戻さねばならなくなってしまった。

1898年から放ったらかしにしてしまっていた本書に再度着手出来たのは、多く
の同僚たる料理人諸氏の助力と、友人でもあるフィレアス・ジルベールとエミー
ル・フェチュ\footnote{初版には、この二人の他にも共著者として、A.
  Suzanne, B. Beboul,Ch. Dietrich, A. Caillat
  らの名が挙げられている。第二版
  では共著者としてジルベールとフェチュの名しかクレジットされていない。
  第二版は初版から構成などの異同が多い。レシピの入れ替えや書き換えの
  他、第二版以降が「フォン(ストック)」から始まるのに対して、初版は
  「ポタージュ」を第1章としている点など。おそらくは、第二版での大幅
  な改訂作業を実際に行なったのがジルベールとフェチュであったとことか
  ら、他の共著者のクレジットが抹消されたと考えられる。}両氏の献身的な協力を得られたからに他ならない。この一大
事業を完成させることが出来たのは、ひとえに皆の励ましと、とりわけ辛抱強
く、粘り強く仕事を手伝ってくれた二人の共著者\footnote{フィレアス・ジルベールとエミール・フェチュのこと。}のおかげだ。

私が作りたいと思ったのは立派な書物というよりはむしろ実用的な本だ。だか
ら、執筆協力者の皆には、作業手順を各自の考えにもとづいて自由にレシピを
書いてもらい、私自身は、40年にわたる現場経験に即して、少なくとも原理原
則、料理における伝統的基礎に関する部分を明確に説明することに専念した。

本書は、かつて私が構想したとおりとは言い難い出来だが、それはいずれ実現
せねばなるまい。それでもなお、現状で料理人諸氏にとって大いに役立つもの
だと信じている。だからこそ、本書を誰にでも、とりわけ若い料理人にも買え
る価格に設定した\footnote{1903年の初版の売価は、\href{http://gallica.bnf.fr/ark:/12148/bpt6k65768837}{フランス国立図書館蔵}のものの表紙には、フランス国内で12フランと記したシールが貼られている。また、\href{https://archive.org/details/b21525912}{リーズ大学図書館蔵の第二版}にも同様に国内売価12フランのシールが貼られている。なお、同じく\href{https://archive.org/details/b21525730}{リーズ大学図書館蔵の初版}および訳者個人蔵の第二版に売価を示すものは何も認められない。\\
  これに対し、辻静雄は「1903年の初版発売当時は、800ページでたった8フラン、全く破格の値段だった」(「エスコフィエ 偉大なる料理人の生涯」、『辻静雄著作集』、新潮社、1995年、729〜730ページ。)と記している。\\
  しかし、オドの『都市および地方における女性料理人のための本』第78版(1900年)において広告されているレオン・スーシェ著『挿絵付き料理書』が『ル・ギード・キュリネール』初版とほぼ同じ800ページ、八折り判がフランス装で10フラン、厚紙の表紙のものが11フランとなっている。当時、急激なインフレがあったとは考えられていないため、初版の12フランという価格が、むしろ料理書としては一般的か、もしくはやや高めのものであったと判断した方がいいだろう。\\
  なお、判型の八折り判というのは、元来は大きな紙の両面に16ページ分印刷し、3回折ることで8枚にしたもの。具体的なサイズは時代、地方によってまちまちだが、『ル・ギード・キュリネール』の時代のフランスにおいては、概ね、縦20〜25cm、横12〜16cm程度で、フランス装の場合、製本業者が端を切り落すため、それより若干小さくなる。具体的には、訳者個人蔵の第二版は表紙が縦22.5cm、横13.5cm。感覚的には、日本のハードカバーに多い「菊判」に近い。\\
  この序文でことさらに「実用性」や入手しやすい価格であることが強調されて
  いるのは、何度も言及しているデュボワとベルナールの名著『古典料理』が四
  折り判(概ね縦45cm、横30cm)の豪華本であったことを意識しているためとも
  推測される。}。そもそも若い料理人にこそこの本を読んで欲しいのだ。
彼らは、いまはまだ初心者であったとしても、20年後には組織のトップに立つ
べき人材なのだから。

私はこの本を豪華な装釘の\footnote{かつてフランスでは、大判の紙に印刷して折ったものを綴じただけの状
  態(いわゆる「フランス装」)で販売された本を、買い求めた者が別途、
  業者に製本、装釘させることが一般的に行なわれていた。}、書棚の飾りのようにはして欲しくない。そう
ではなく、いつでも、どんな時でも手元に置いて、わからないことは常に明ら
かにしてくれる伴侶のようなものにして欲しい。

本書には五千を越えるレシピが掲載されているが、それでも私は、この手引書
が完全だとは思っていない。たとえ今この瞬間に完璧であったとしても、明日
にはそうではないかも知れぬ。料理は進化し、新しいレシピが日々創案されて
いるのだ。まことにもって不都合なことだが、版を重ねる毎に新しい料理を採
り入れ、古くなってしまったものは改良を加えねばなるまい。

ユルバン・デュボワとエミール・ベルナール両氏の著作\footnote{『古典料理』\emph{La
  Cuisine Classique} (1856),『芸術的料理』 \emph{LaCuisine Artistique}
  (1870) etc.}に昔から慣れ親し
み、その巨大な影がなおも料理の地平を覆い尽している現在、私としては本書
がその後に続くものになって欲しい。カレーム以後、最高の料理の高みに逹し
た二人に対し、ここであらためて心から敬意を示させていただきたいと思う。

調理現場を取り巻く諸事情により、私は、デュボワとベルナール両氏がもたら
したサービス(給仕)面での革新\footnote{19世紀後半に少しずつ一般的となった「ロシア式サービス」のこと。そ
  れまで宴席の料理は卓上に大皿が一度に何種も並べられ、食べる者がそれ
  ぞれ好きなように取り分けていた。これを、料理を食べる順に1種ずつ、
  大皿の場合は食べ手に見せて回ってから、給仕が取り分けて供する方式に
  改めたものがロシア式サービスである。デュボワとベルナールの『古典料
  理』の序文において詳述されている。}に対し、こんにちのようなとりわけスピー
ドが重視される目まぐるしい生活リズムに合わせて、大きく変更を加えざるを
得なかった。そもそも物理的理由から、料理を載せる飾り台\footnote{socle
  ソークル。パンや米、ジュレなどで作った、料理を盛り付ける
  ために銀の盆の上に据える飾り台。カレームの時代、つまり19世紀前半に
  はその装飾に凝ることが多かった。食べもので作られてはいるが、料理の
  一部ではなく、あくまで装飾的要素でしかなかった。}をやめて、
シンプルな盛り付けにする新たなメソッドと新たな道具を考案せざるを得なかっ
た。もちろん冗談なんかではなく、デュボワとベルナールが推奨した壮麗な盛
り付けを私自身も行なっていた頃はもちろん、今だって彼らの考えにはまった
く共感している。しかし、カレームを信奉する者たちは、装飾に才があるが故
に、その代もはやに合わなくなってしまった作品に改良を加えようとはしなかっ
た。それこそがまさに重要なのに。本書で奨励している盛り付けは、少なくと
もそれなりの期間は必要とされ続けると思う。全ては変化する。姿を変える。
それなのに、装飾芸術の役割を変化しないと主張するなんて愚かなことだろう。
芸術は流行によって栄えるものだし、流行のように移ろいやすいものだ。

だが、カレームの時代にはこんにちと同様に既にあり、料理が続く限りなくな
らないであろうものがある。それが料理の基礎だ。そもそも、料理が見た目で
シンプルになっても、料理そのものの価値は失なわれないが、その逆はどうだ
ろう? 人々の味覚は絶え間なく洗練され続け、それを満足させるために料理
そのものも洗練されることになる。こんにちの余剰活動が精神におよぼす悪影
響に打ち克つためには、料理そのものがいっそう科学的な、正確なものとなる
べきなのだ。

その意味で料理が進歩すればする程、我々料理人たちにとって、19世紀に料理
の行く末に大きく影響を与えた三人の料理人の存在は大きなものとなるだろう。
カレームとデュボワ、ベルナールはともすれば技術的側面ばかり評価されるが、
料理芸術の基礎について何よりも優れているのだ。

物故した者の名だけ挙げるが、確かにグーフェ\footnote{Jules Gouffé
  (1807〜1877)。ナポレオン三世の料理人として知られる。
  著書も多く、代表作『料理の本』\emph{Le Livre de Cuisine} (1867年) は前
  半が家庭料理、後半が高級料理(オートキュイジーヌ)の二部構成となっ
  ており、レシピの書き方も、まず材料表を掲げた後に調理手順を説明する
  という現代のそれに近いものになっている。}、ファーヴル\footnote{Joséph
  Favre (1849〜1903年)。スイス生まれの料理人で、パリ、ドイ
  ツ、イギリス、ベルギー等において腕を活躍した。『料理および食品衛生
  事典』\emph{Dictionnaire universel de cuisine et d'hygiène
  alimentaire}(1884〜1895年)は、こんにちでも事典としての有用性が失な
  われていない。}、エルー イ\footnote{Edouard Hélouis(生没年不詳)。
  イギリスのアルバート王配(ヴィ
  クトリア女王の夫)およびサルデーニャ国王ヴィットーリオ・エマヌエー
  レ二世の料理人を務めた。著書『王室の晩餐』\emph{Les Royal-Diners ---
  Guide du Gourmet}(1878年)。}、ルキュレ\footnote{『実践的料理』\emph{Le
  Cuisinier praticien ou la Cuisine simple et pratique} (1859年) の著者
  C. Reculet を指していると思われる。 cf.~G. Vicaire,
  \emph{Bbibliographie gastronomique}, Paris, P. Rouquette et fils,
  1890, p.733.}はとても素晴らしい著作を残した。だが、『古典料理』
という稀代の名著と同列といえるものはひとつもない。

料理人諸氏に、新たに本書を使っていただくにあたり、私はこう言わねばなら
ない。いろいろな料理書、雑誌を読むのもいいが、偉大な先達の不朽の名作を
しっかり読むように、と。諺にあるように「知り過ぎることなはい」のだ。学
べば学ぶ程、さらに学ぶべきことは増えていく。ひいては、精神が柔軟になり、
料理を作るということにおいて上達するより効果的な方法を知ることが出来る
だろう。

本書を公刊するにあたって唯ひとつ望むこと、切に願う唯一のことは、上記の
点において、本書が想定する読者が我が言に耳を傾け、実践するさまを見るこ
とに尽きる。

A. エスコフィエ

1902年11月1日



\end{document}
