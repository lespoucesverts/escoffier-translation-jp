\href{未、原文対照チェック}{} \href{未、日本語表現校正}{}
\href{未、その他修正}{} \href{未、原稿最終校正}{}

\begin{Main}

\hypertarget{artichauts}{%
\subsection[アーティチョーク]{\texorpdfstring{アーティチョーク\footnote{artichaut
  (アルティショー)キク科の多年草で、草丈は1
  m程にもなる。フランス語としては、16世紀初頭には carchoffle あるいは
  artichault
  の綴りで記録されている。しばしば、カトリーヌ・ド・メディシスがイタリアからフランスに「紹介した」とか「もたらした」といわれるが、これは俗説であり、それ以前からフランスでも知られていたし、南フランスでは栽培されていた。実際のところ、カトリーヌ・ド・メディシスはフランスの王宮においてアーティチョークの料理が流行するきっかけ程度には普及に貢献したのだろう。16世紀末オリヴィエ・ド・セール『農業経営論』において既に、南フランスの気候を活かした周年栽培の方法が記されており、その方法論の基礎はこんにちでも変化していない。食材としては、主に花蕾を利用する。固く厚みのある花弁のような花萼に覆われており、ある程度成熟したものは花萼をすべてナイフで切り落して取り除き(この作業はアーティチョーク本体を回すようにして剥くようにするのでトゥルネtournerという)、さらに内部の繊毛をスプーン等で取り除いて皿のような形状にしたものを加熱したのちに、タルトレットのようにアパレイユを詰めるなどする。この基底部をfond
  d'artichaut (フォンダルティショー)または cul
  d'ratichaut(キュダルティショー)と呼ぶ。やや若どりで小さめのものは花萼も下半分は加熱すれば柔らかいため、花蕾の上部は切り落してごく外側の花萼だけ剥いてから、半割りまたは四つ割りにし、繊毛を取り除いてから加熱調理する。これは
  coeur d'artichaut
  (クールダルティショー)と呼ばれることが多い(ただしこれらの呼び名はやや曖昧なところがあり、これをフォンダルティショーと呼んでいると解釈可能なケースもある)。さらにごく若どりのものは生食も可能であり、poivrade
  (ポワヴラード)と呼ばれる。よく誤解されることだが、Poivrade
  という品種はなく、収穫タイミングとサイズ(というよりも生食出来るくらい柔らかいものをポワヴラードと呼んでいるのであり、実際にはViolet
  de
  Provence(ヴィオレドプロヴォンス)という品種(およびそれを親としたF1品種)が中心だが、本来的にはポワヴラードに品種の決まりはない。。いずれの場合も、空気に触れるとすぐに黒く変色するので、レモンを擦り付けながら作業し、作業後はレモン果汁を加えた水にすぐに入れるといい。フランスのアーティチョークはブルターニュ産のものがとりわけ有名で、大きな花蕾を付ける品種が中心であり、南フランス産のものは比較的小ぶりで、花萼が紫色がかった品種(上述の
  Violet de
  Provence)が代表的。日本には明治時代に伝わり、何度も生産が試みられているが、一般野菜としての需要を喚起することが出来ずにいるため、現在も輸入品が中心。なお、アーティチョークは株分けで殖やすのが一般的であり、ブルターニュ地方の代表品種Camus(カミュ)は種子が流通していない。種子の流通している品種は安定性が低く、形質が揃わないのがほとんど。このため、日本でヨーロッパと同じ水準のものを栽培するには、苗を輸入することが生産者レベルでは困難なため、種子から育ててかなりの年数をかけて株を選別し、殖やしていくしか方法はない。さらに収穫期間を延長するには、株分けのタイミングやその後の栽培管理においてかなりの技術が必要となる。}}{アーティチョーク}}\label{artichauts}}

\frsecb{Artichauts}

\index{artichaut@artichaut|(}
\index{あーていちよーく@アーティチョーク|(}

\end{Main}

\begin{recette}

\hypertarget{artichauts-a-la-barigoule}{%
\subsubsection[アーティチョーク・バリグール]{\texorpdfstring{アーティチョーク・バリグール\footnote{プロヴァンス地方に自生する乳茸の一種。もとはこの茸を用いた料理だったといわれるが、どんな種類の茸を用いてもよいとされる。実際には、いわゆるマッシュルームを用いたデュクセル・セッシュを使うケースが多い。}}{アーティチョーク・バリグール}}\label{artichauts-a-la-barigoule}}

\frsub{Artichauts à la Barigoule}

\index{artichaut@artichautl!barigoule@--- à la Barigoule}
\index{barigoule@barigoule!artichauts@Artichauts à la ---}
\index{あーていちよーく@アーティチョーク!はりくーる@---・バリグール}
\index{はりくーる@バリグール!あーていちよーく@アーティチョーク・---}

アーティチョークは新鮮で柔らかいものを選ぶこと。花蕾の上部\footnote{この場合、花萼をまったく残さずに小さな皿、あるいはタルトレットの形状に剥くことになるので、上から
  \(\frac{1}{3}\) 〜 \(\frac{3}{4}\)
  は切り落すことになる。また、本文では言及されていないが、もし多少でも茎が付いている場合は、底面が安定するように平らに切り落しておくこと。}を切り落し、周囲を花萼をナイフで削り取る\footnote{この作業も、アーティチョークを回すようにして行なうため、tourner
  (トゥルネ)の語が用いられる。また、この料理の場合は全体に茶色い仕上りになるため、アーティチョークが空気に触れることで黒ずんでしまうことが問題にならないためか、レモンを擦り付けるなどの指示はされていない。}。\protect\hyperlink{blanchissage}{下茹で}し、それから繊毛を取り除く。完全に取り除くよう心を配ること\footnote{この作業は下茹で前に行なうこともある。その場合はスプーン等を使って削るようにする。大型のアーティチョークの基底部の場合は、本文にあるように、下茹で後に手で繊毛を毟り取るほうがきれいに仕上がる。}。

内側に塩こしょうする。\protect\hyperlink{duxelles-seche}{デュクセル}に、\(\frac{1}{4}\)の重さの豚背脂を器具を用いておろして加え、さらに豚背脂と同量のバターを合わせ、アーティチョークの中央に詰める。こうして詰め物をしたアーティチョークをごく薄い豚背脂のシートで包み、紐で縛る。これを、\protect\hyperlink{braisage-des-legumes}{ブレゼ}用に準備した\footnote{\protect\hyperlink{braisage-des-legumes}{野菜のブレゼ}の項を必ず参照のこと。}鍋に並べ、\protect\hyperlink{fonds-brun}{茶色いフォン}をアーティチョークの高さまで注いで蓋をしてごく弱火で加熱し\footnote{この手法で火を通すことそれ自体が
  braiser (ブレゼ)と呼ばれる。}、しっかりと火を通す。

提供直前に、紐を外してアーティチョークを皿に盛り付ける。ブレゼした煮汁は漉して、浮いている脂を取り除く\footnote{dégraisser
  (デグレセ)。}。上等な\protect\hyperlink{sauce-demi-glace}{ソース・ドゥミグラス}適量を合わせてとろみを付け、アーティチョークに上からかけて余らない程度の量になるまで煮詰める。

\atoaki{}

\hypertarget{artichauts-cavour}{%
\subsubsection[アーティチョーク・カヴール]{\texorpdfstring{アーティチョーク・カヴール\footnote{サルデーニャ王国の宰相として、国王ヴィットーリオ・エマヌエーレ2
  世ととにも、リソルジメント(イタリア統一運動)を率いた。巧みな外交戦略によりイタリア統一を実現させた英雄のひとりと見なされている。イタリア王国成立後も首相、外相として活躍した。イタリアのおもな都市には必ずといっていい程、彼の名を冠した「カヴール通り」がある。なお、歴史的には北イタリアおよび南フランスのニース周辺にいたる地域をめぐって、オーストリア、フランス、イタリアが領土の奪い合いが行なわれ、イタリア統一後、「未回収のイタリア」運動ではニース周辺をイタリアに編入しようとイタリアは強く主張していた。まさしくエスコフィエの少年時代、ニースでフランス料理の修業を始めた頃のことでもあり、彼の自身がフランス人たる意識は否が応にも強まったものと思われる。}}{アーティチョーク・カヴール}}\label{artichauts-cavour}}

\frsub{Artichauts Cavour}

\index{artichaut@artichautl!cavour@--- Cavour}
\index{cavour@Cavour!artichauts@Artichauts ---}
\index{あーていちよーく@アーティチョーク!かうーる@---・カヴール}
\index{かうーる@カヴール!あーていちよーく@アーティチョーク・---}

プロヴァンス産\footnote{プロヴァンスでは Artichaut Violet de Provence
  という品種がもっとも有名。この品種は草勢がおとなしく、小振りの花蕾を若どりしてポワヴラードとして出荷されることも多い。}の小振りで柔らかいアーティチョークの外側の花萼を卵形に剥く。

これを\protect\hyperlink{consomme-blanc}{白いコンソメ}で茹で、火が通ったら取り出して湯きりし、さらに圧し絞って余計な水分を完全に出させる。溶かしバターに漬け込む。取り出して、おろしたグリュイエールチーズとパルメザンチーズの上を転がしてチーズをまぶし付ける。これをグラタン皿に環状に並べ、高温のオーブンで焼く。

アーティチョーク6個につき固茹で卵1個をみじん切りにして、バターでソテーする。バターが充分に泡立ってきたら、アンチョビエッセンス少々とパセリのみじん切りを加えてソースを仕上げ、アーティチョークに注ぎかける。

\atoaki{}

\hypertarget{coeurs-d-artichauts-clamart}{%
\subsubsection[アーティチョークの芯・クラマール]{\texorpdfstring{アーティチョークの芯・クラマール\footnote{パリ郊外南西の地名で、かつては豆類(プチポワなど)の生産が盛んだった。}}{アーティチョークの芯・クラマール}}\label{coeurs-d-artichauts-clamart}}

\frsub{Artichauts Clamart}

\index{artichaut@artichautl!clamart@--- Clamart}
\index{clamart@Clamart!artichauts@Artichauts ---}
\index{あーていちよーく@アーティチョーク!くらまーる@---・クラマール}
\index{くらまーる@クラマール!あーていちよーく@アーティチョーク・---}

中位のサイズで柔らかいアーティチョークを選ぶこと。掃除をして\footnote{花蕾の上半分程を切り落し、花萼の固い部分と茎の皮を剥く。内部の繊毛はこの時点でスプーン等で取り除いてもいいし、櫛切りにした後に取り除いてもいい。}6つに櫛切りにする。

内側にバターを塗ったココット鍋に並べ、若どりのミニキャロットを四つ割りにして加える。アーティチョーク1つにつき大さじ3杯の莢から出したばかりのプチポワを食うぇる。たっぷりのブーケガルニと水少々を加えて軽く塩をし、蓋をしてごく弱火で蒸し煮する\footnote{原文
  cuire très doucement à l'étuvée
  (キュイールトレドゥスモンアレチュヴェ)。}。

提供直前に、ブーケガルニを取り出し、\protect\hyperlink{beurre-manie}{ブールマニエ}少々で軽く煮汁にとろみを付ける。ココット鍋のまま供する。

\atoaki{}

\hypertarget{coeurs-d-artichauts-grand-duc}{%
\subsubsection{アーティチョークの芯・グランデュック}\label{coeurs-d-artichauts-grand-duc}}

\frsub{Artichauts Grand-Duc}

\index{artichaut@artichaut!grand-duc@--- Grand-Duc}
\index{grand-duc@Grand-Duc!artichauts@Artichauts ---}
\index{あーていちよーく@アーティチョーク!くらんてゆつく@---・グランデュック}
\index{くらんてゆつく@グランデュック!あーていちよーく@アーティチョーク・---}

\index{artichaut@artichaut|)}
\index{あーていちよーく@アーティチョーク|)}

\end{recette}
