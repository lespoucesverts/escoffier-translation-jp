\href{未、原文対照チェック}{} \href{未、日本語表現校正}{}
\href{未、その他修正}{} \href{未、原稿最終校正}{}

\begin{Main}

\hypertarget{artichauts}{%
\subsection[アーティチョーク]{\texorpdfstring{アーティチョーク\footnote{artichaut
  (アルティショー)キク科の多年草で、草丈は1
  m程にもなる。フランス語としては、16世紀初頭には carchoffle あるいは
  artichault
  の綴りで記録されている。しばしば、カトリーヌ・ド・メディシスがイタリアからフランスに「紹介した」とか「もたらした」といわれるが、これは俗説であり、それ以前からフランスでも知られていたし、南フランスでは栽培されていた。実際のところ、カトリーヌ・ド・メディシスはフランスの王宮においてアーティチョークの料理が流行するきっかけ程度には普及に貢献したのだろう。16世紀末オリヴィエ・ド・セール『農業経営論』において既に、南フランスの気候を活かした周年栽培の方法が記されており、その方法論の基礎はこんにちでも変化していない。食材としては、主に花蕾を利用する。固く厚みのある花弁のような花萼に覆われており、ある程度成熟したものは花萼をすべてナイフで切り落して取り除き、皿のような形状にしたものを加熱したのちに、タルトレットのようにアパレイユを詰めるなどする。この基底部をfond
  d'artichaut
  (フォンダルティショー)と呼ぶ。やや若どりのものは花萼も下半分は加熱すれば柔らかいため、花蕾の上部は切り落して、半割りまたは四つ割りにして加熱調理する。これは
  coeur d'artichaut
  (クールダルティショー)と呼ばれる。さらにごく若どりのものは生食も可能であり、poivrade
  (ポワヴラード)と呼ばれる。ブルターニュ産のものが有名で、大きな花蕾を付ける品種が中心であり、南フランス産のものは比較的小ぶりで、紫色の品種が代表的。日本には明治時代に伝わり、何度も生産が試みられているが、一般野菜としての需要を喚起することが出来ずにいるため、現在も輸入品が中心。}}{アーティチョーク}}\label{artichauts}}

\index{artichaut@artichaut|(}
\index{あーていちよーく@アーティチョーク|(}

\end{Main}

\begin{recette}

\hypertarget{artichauts-a-la-barigoule}{%
\subsubsection{アーティチョーク・バリグール}\label{artichauts-a-la-barigoule}}

\frsub{Artichauts à la Barigoule}

\index{artichaut@artichautl!barigoule@--- à la Barigoule}
\index{barigoule@barigoule!artichauts@Artichauts à la ---}
\index{あーていちよーく@アーティチョーク!はりくーる@---・バリグール}
\index{はりくーる@バリグール!あーていちよーく@アーティチョーク・バリグール}

\index{artichaut@artichaut|)}
\index{あーていちよーく@アーティチョーク|)}

\end{recette}
