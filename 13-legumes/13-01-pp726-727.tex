\hypertarget{legumes-farineux-et-pates-alimentaires}{%
\chapter{XIII. 野菜料理・パスタなど}\label{legumes-farineux-et-pates-alimentaires}}

\frchap{Légumes --- Farineux et Pâtes alimentaires}

\hypertarget{serie-des-legumes}{%
\section{野菜料理}\label{serie-des-legumes}}

\begin{center}
\headfont\large 下拵えと注意事項\label{observations-sur-les-operations-preliminaires}

\normalfont\textit{Observation sur les Opérations préliminaires.}
\end{center}

\normalfont\normalsize

\hypertarget{blanchissage}{%
\subsection[湯がく・下茹で]{\texorpdfstring{湯がく・下茹で\footnote{blanchir(ブロンシール)。もともとの意味は「白くする」.食材を冷蔵保存出来なかった中世には肉類はいったん下茹でしてから調理するのが一般的であり,肉を下茹ですると表面が白くなることからこの用語が用いられるようになった.素材の種類によっては白く茹であげるために単なる湯ではなく「ブラン」を用いる場合もある.これは、水1ℓにスプーン1杯強の小麦粉,塩6gとスプーン2杯の溶いて沸かす.クローヴを刺した玉ねぎ1ヶとブーケガルニ,下茹でする素材,空気に素材が触れて変色するのを防ぐための脂を入れる.脂は,牛あるいは仔牛の生のケンネ脂を細かく刻んだもの.必要なら脂を事前に冷水にさらして血等の夾雑物がないようにしておくこと.
  (原注)
  野菜の下茹で用のブランには,ヴィネガーではなくレモン汁を用いたほうがいい(原書
  p.405).}}{湯がく・下茹で}}\label{blanchissage}}

この作業は2つの目的で行なう。第1に、例えばほうれんそう、プティポワ、さやいんげん等の一般的な葉物野菜では、完全に火を通すのが目的。たっぷりの湯で手早く茹で、クロロフィルすなわち葉緑素を失わないようにすること。第
2は、野菜に自然にあるえぐ味を消す目的。例えばキャベツ、セロリ、シコレ等。原則的に、新野菜は下茹でしない。下茹でで完全に火を通してしまう野菜については、1リットルあたり7gの塩を湯に加えること。

\hypertarget{rafraichissage}{%
\subsection[冷水にはなす]{\texorpdfstring{冷水にはなす\footnote{rafraîchir
  ラフレシール}}{冷水にはなす}}\label{rafraichissage}}

湯がいた後、冷水にとるのは、野菜をブレゼにする場合と、オペレーションの都合から事前に茹でておかなければならない場合のみ。ただし、後でバターやクリームで合える場合は、冷水にとると風味が失われる。

\hypertarget{cuisson-des-legumes-a-l-anglaise}{%
\subsection{アラングレーズ}\label{cuisson-des-legumes-a-l-anglaise}}

沸騰した湯で茹でるのみ。次によく水切りをして、さらに水気をとばす。深皿に盛りつけ、貝殻形のバターを添えて供する。味付けは食べ手がする。葉物野菜なら何でもこのイギリス風に調理、提供可能。

\hypertarget{cuisson-des-legumes-secs}{%
\subsection{乾燥豆}\label{cuisson-des-legumes-secs}}

乾燥豆を水でもどしておくのはよろしくない。その年に穫れた良質のものなら、水から弱火でゆっくり沸かして茹でればいい。あく取りをして香味野菜\footnote{通常は,クローヴを刺した玉ねぎと縦に四つ割りにしたにんじん,およびブーケガルニ.}を加え、蓋をしてごく弱火で茹でる。

あまりにも古い豆や品質が劣るものは水でもどす。ただし、豆が膨れるのに必要な時間きっかり、すなわち1時間半から2時間とすること。

何時間も水につけておくと、発酵が始まってしまう。そうなると豆の組織が損なわれて使いものにならなくなってしまうことさえある。

\hypertarget{braisage-des-legumes}{%
\subsection{野菜のブレゼ}\label{braisage-des-legumes}}

野菜は事前に湯がいて冷水にとる。形を整える。鍋の底と周囲に豚背脂のシートを張り、野菜を入れる。上面を背脂で覆う。鍋に蓋をして弱火でかるく汗をかかせるように蒸し煮した後、ひたひたまで白いフォンを注ぐ。鍋に蓋をし、中温のオーヴンに入れる。火が通ったら、野菜の水気をきり、用途に合わせて形を整える。その後すぐに使う場合は、煮汁の浮き脂を取り除いて煮つめ、野菜とともにソテ鍋で保温する。事前に仕込んでおく場合には、鍋から皿あるいは専用の容器に移す。煮汁は浮き脂を取り除かずにそのまま加える。バターを塗った紙で覆ってストックする。

\hypertarget{sauce-des-legumes-braises}{%
\subsection{野菜のブレゼのソース}\label{sauce-des-legumes-braises}}

ブレゼの煮汁を煮詰め、浮き脂を丁寧に取り除いて使う。場合によってはグラスドヴィアンド、あるいは相応量のドゥミ・グラスを加える。どちらの場合にも、

ソースがまろやかになるようバターを加えて仕上げる。必要ならレモン汁数滴も加える。

\hypertarget{liaison-des-legumes-vert-au-beurre}{%
\subsection{葉野菜をバターであえる}\label{liaison-des-legumes-vert-au-beurre}}

茹でた野菜はしっかりと水をきっておく。味付けをしてバターを加え、鍋をあおるようにしてバターが野菜全体にまわるようにする。バターの風味を失なわないためには、「火から外した状態」でバターを加えること。

\hypertarget{liaison-des-legumes-a-la-creme}{%
\subsection{クリームであえる}\label{liaison-des-legumes-a-la-creme}}

この方法で調理する場合は、野菜に火を通す際、やや歯応えを残しておくこと。

しっかり水気をきり、野菜を鍋に入れる。沸かした生クリームを野菜が上に顔を出す程度に加える。

時々、よくかきまぜながら加熱する。

クリームがほぼすっかり煮詰まったら、バターとレモン汁少量を加える。必要なら、生クリームにソース・クレーム\footnote{blanchir(ブロンシール)。もともとの意味は「白くする」.食材を冷蔵保存出来なかった中世には肉類はいったん下茹でしてから調理するのが一般的であり,肉を下茹ですると表面が白くなることからこの用語が用いられるようになった.素材の種類によっては白く茹であげるために単なる湯ではなく「ブラン」を用いる場合もある.これは、水1ℓにスプーン1杯強の小麦粉,塩6gとスプーン2杯の溶いて沸かす.クローヴを刺した玉ねぎ1ヶとブーケガルニ,下茹でする素材,空気に素材が触れて変色するのを防ぐための脂を入れる.脂は,牛あるいは仔牛の生のケンネ脂を細かく刻んだもの.必要なら脂を事前に冷水にさらして血等の夾雑物がないようにしておくこと.
  (原注)
  野菜の下茹で用のブランには,ヴィネガーではなくレモン汁を用いたほうがいい(原書
  p.405).}を足してもいい。

\hypertarget{cremes-et-puree-de-legumes}{%
\subsection{野菜のクレームとピュレ}\label{cremes-et-puree-de-legumes}}

乾燥豆とでんぷん質の野菜は裏漉ししてピュレにする。次にバター1かけを加えて火にかけ、水気をとばす。牛乳か生クリームを加えて濃さを調節する。

さやいんげん、カリフラワー等のように水分の多い野菜をピュレにする場合は、濃さを出すため、味のバランスのとれたでんぷん質の野菜のピュレを加えること。

野菜の「クレーム」にする場合は、でんぷん質の野菜のピュレではなく、濃く仕上げたソース・ベシャメルを加える\footnote{ベシャメルソース1ℓに生クリーム2dlを加え,火にかけてへらで混ぜながら
  ¾ℓになるまで煮つめる.
  布で漉し,クレーム・ドゥーブル1½dlとレモン汁½個分を少しずつ加えていき、濃さを調節する(原書p.32)。}。
